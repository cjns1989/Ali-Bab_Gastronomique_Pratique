\section*{\centering Grenouilles.}
\addcontentsline{toc}{section}{ Grenouilles.}
\index{Batraciens, crustacés et mollusques}
\index{Grenouilles}

Les grenouilles, « rana », sont des reptiles batraciens de la famille des Ranidés.
Les pattes, seules parties comestibles, dont le goût rappelle celui du poussin,
constituent un aliment léger, sain et agréable,

Dans le commerce, les grenouilles sont présentées dépouillées et débarrassées
des parties non comestibles, enfilées par le bas des reins sur des baguettes plates,
les pattes croisées. Au moment de les apprêter, on rogne l'extrémité des pattes et
on enlève la partie des reins attenant aux cuisses.

Les grenouilles peuvent être accommodées de bien des manières :

1° en potages, \hyperlink{p0221}{p. \pageref{pg0221}} ;

2° sautées, panées ou non, dans du beurre, de la graisse ou de l'huile ;
\textit{au naturel}, avec du persil, des fines herbes, etc. ; — \textit{à la
provençale}, avec de l'ail ; — \textit{à la portugaise}, avec des tomates ;
— \textit{à l'indienne}, avec du curry ; \textit{à la hongroise}, avec du
paprika ; — \textit{à l'italienne}, avec des champignons ;

3° au blanc ; à la poulette :

4° à la meunière ;

5° frites, par exemple en beignets ;

6° au gratin ;

7° en chaud-froid ; etc.

Je vais maintenant donner en détail les indications nécessaires pour deux des
préparations ci-dessus : grenouilles sautées, panées, et grenouilles au blanc ;
les autres préparations se concevant facilement.

\section*{\centering Grenouilles sautées, panées.}
\addcontentsline{toc}{section}{ Grenouilles sautées, panées.}
\index{Grenouilles sautées, panées}

Pour quatre personnes prenez :

\medskip

\setlength\tabcolsep{.15em}
\footnotesize
\begin{longtable}{rrrp{16em}}
  150 & grammes   & de & beurre,                                                                          \\
  100 & grammes   & de & mie de pain rassis, tamisée,                                                     \\
      &           &  2 & douzaines de belles grenouilles,                                                 \\
      &           &  2 & citrons,                                                                         \\
      &           &  2 & échalotes,                                                                       \\
      &           &    & fine champagne,                                                                  \\
      &           &    & farine,                                                                          \\
      &           &    & persil,                                                                          \\
      &           &    & estragon,                                                                        \\
      &           &    & ciboule,                                                                         \\
      &           &    & sel et poivre.                                                                   \\
\end{longtable}
\normalsize

Mettez pendant une heure les grenouilles, avec du sel et du poivre, dans de la
fine champagne ; remuez-les de temps en temps ; égouttez-les ensuite, puis
roulez-les dans de la farine,

Mélangez la mie de pain avec du sel, du poivre, du persil, de l’estragon et de la
ciboule hachés, au goût.

Faites sauter en même temps : d'une part, les grenouilles et les échalotes dans
{\ppp75\mmm} grammes de beurre, pendant {\ppp6\mmm} à {\ppp7\mmm} minutes ;
d'autre part, le mélange mie de pain et aromates dans le reste du beurre, de
manière à le bien imbiber et à le dorer.

Enlevez les échalotes ; réunissez dans la même sauteuse grenouilles et mie de
pain aromatisée ; puis, continuez à faire sauter le tout ensemble pendant un
moment, en mélangeant bien.

Servez avec les citrons, coupés par la moitié.

\section*{\centering Grenouilles au blanc.}
\addcontentsline{toc}{section}{ Grenouilles au blanc.}
\index{Grenouilles au blanc}

Pour huit personnes prenez :

\medskip

\footnotesize
\begin{longtable}{rrrp{16em}}
  750 & grammes   & de & vin blanc sec,                                                                   \\
  500 & grammes   & de & champignons de couche,                                                           \\
  250 & grammes   & d' & oignons,                                                                         \\
  250 & grammes   & de & rème,                                                                            \\
  100 & grammes   & de & lait,                                                                            \\
   60 & grammes   & de & beurre,                                                                          \\
   10 & grammes   & de & farine,                                                                          \\
      &           &  4 & douzaines de belles grenouilles,                                                 \\
      &           &  2 & jaunes d'œufs frais,                                                             \\
      &           &  1 & bouquet garni,                                                                   \\
      &           &    & jus de citron,                                                                   \\
      &           &    & sel et poivre.                                                                   \\
\end{longtable}
\normalsize

Lavez les grenouilles dans le lait.

Mettez dans une casserole le beurre et la farine maniés, chauffez sans laisser
prendre couleur, mouillez avec le vin, ajoutez les oignons coupés en rondelles
et le bouquet garni ; assaisonnez avec sel et poivre. Faites cuire pendant une
demi-heure.

Passez la sauce, mettez dedans les champignons épluchés et passés dans du jus
de citron ; puis, après dix minutes de cuisson, ajoutez les grenouilles.
Laissez cuire le tout pendant cinq minutes, liez ensuite la sauce avec la crème
et les jaunes d'œufs, goûtez, complétez l’assaisonnement s'il y a lieu et
servez.

\index{Crustacés}

\section*{\centering Langouste\footnote{Palinurus locusta : famille des Palinuridés.} au naturel.}
\addcontentsline{toc}{section}{ Langouste au naturel.}
\index{Langouste au naturel}

Le meilleur procédé de cuisson de la langouste pour les véritables amateurs est
la cuisson au naturel.

Prenez une langouste en vie, lourde, bien pleine ; ligottez-la et plongez-la dans
une bassine contenant suffisamment d'eau de mer bouillante ou, à défaut, de l'eau
salée avec du sel marin. Maintenez l'ébullition pendant une vingtaine de minutes
pour une langouste pesant un kilogramme environ.

Laissez refroidir l'animal dans la cuisson ; égouttez-le.

Servez sur un plat couvert d'une serviette et décoré avec du persil. Envoyez en
même temps une saucière de sauce mayonnaise ordinaire,
\hyperlink{p0323}{p. \pageref{pg0323}}, de mayonnaise à la pulpe de citron ou
aux œufs de homard, de mayonnaise au raifort ou à la moutarde.

\sk

Pour préparer une mayonnaise aux œufs de homard, prenez des œufs de homard
crus, écrasez-les, mettez-les dans une casserole, ébouillantez-les avec du
vinaigre, dans la proportion de {\ppp20\mmm} grammes de vinaigre
pour {\ppp50\mmm} grammes d'œufs, chauffez au bain-marie en tournant jusqu'à ce
que le mélange se soit épaissi et qu'il ait pris une belle couleur de pourpre
cardinalice.

Passez au travers d'une double mousseline, puis incorporez la purée obtenue
à de la mayonnaise ordinaire.

\sk

On aura une mayonnaise au raifort en mélangeant du raifort râpé, au goût, à de
la mayonnaise ordinaire.

\sk

La langouste au naturel peut être servie en salade.

On décorera alors le plat avec des cœurs de laitue et des œufs durs coupés en
quatre. On servira en même temps une sauce simple à l'huile et au vinaigre ou
une sauce douce, \hyperlink{p0415}{p. \pageref{pg0415}}.

\sk

On peut préparer de même le homard au naturel.

\section*{\centering Homard\footnote{« Homarus vulgaris », famille des Astacidés.} à l'américaine.}
\addcontentsline{toc}{section}{ Homard à l'américaine.}
\index{Homard à l'américaine}
\index{Civet de homard}

Le plat connu aujourd'hui sous le nom de « Homard à l'américaine » portait,
lors de sa création, en 1853, par Constant Guillet, chef des cuisines du
restaurant Bonnefoy, la dénomination de « Homard à la Bonnefoy ». Le
qualificatif « à l'américaine » a été donné plus tard à ce plat par des
concurrents qui ont plus ou moins modifié la formule primitive. Toutes ces
préparations sont des ragoûts de homard à sauce liée au sang ; elles devraient
donc logiquement porter le nom de « civet de homard ». Ces réserves faites, je
conserverai cependant le vocable « Homard à l'américaine » consacré par
l'usage.

En voici une formule.

\medskip

Pour six personnes prenez :

\medskip

\footnotesize
\begin{longtable}{rrrp{16em}}
  750 & grammes    & de & poissons ou de débris de poissons,                                              \\
  750 & grammes    & de & légumes de pot-au-feu,                                                          \\
  600 & grammes    & de & tomates,                                                                        \\
  300 & grammes    & d' & eau,                                                                            \\
  150 & grammes    & de & vin blanc sec,                                                                  \\
  100 & grammes    & de & beurre,                                                                         \\
   75 & grammes    & d' & huile d'olive,                                                                  \\
   60 & grammes    & de & fine champagne,                                                                 \\
   25 & grammes    & d' & oignons ciselés,                                                                \\
   25 & grammes    & d' & échalotes hachées,                                                              \\
      &            &  2 & homards (femelles de préférence), pesant chacun 750 grammes au moins,           \\
      &            &  1 & gousse d'ail haché,                                                             \\
      &            &    & jus de citron,                                                                  \\
      &            &    & thym, laurier,                                                                  \\
      &            &    & persil, cerfeuil, estragon, hachés séparément,                                  \\
      &            &    & sel, poivre, cayenne.                                                           \\
\end{longtable}
\normalsize

\index{Fond de poisson}
Mettez dans une casserole l'eau, le vin, les poissons, les légumes de pot-au-feu
coupés en morceaux, du sel, du poivre ; faites cuire pendant trois heures au moins,
après avoir écumé en temps utile. Passez le liquide à la passoire, concentrez-le
de façon à obtenir {\ppp200\mmm} grammes environ de fond de poisson. Tenez-le au chaud.

Coupez les homards vivants, recueillez le liquide, c'est-à-dire le sang, qui
s'écoule pendant cette opération ; séparez les queues en autant de morceaux
qu'il y a d’anneaux ; brisez la carapace des pinces ; fendez les coffres en
deux dans le sens de la longueur ; réservez les intérieurs, le corail et les
œufs ; enlevez la poche pierreuse qui se trouve près des mandibules.

Mettez l'huile dans une sauteuse ; chauffez à feu vif ; jetez dedans les
morceaux de homard, assaisonnez avec sel, poivre et faites sauter jusqu'à ce
que les carapaces soient bien rouges, ce qui demande une dizaine de minutes
environ ; puis ajoutez : d'abord {\ppp50\mmm} grammes de beurre, les oignons et les
échalotes auxquels vous laisserez prendre couleur ; ensuite l'ail, le persil,
le cerfeuil, du thym, du laurier, et faites sauter encore pendant un instant.

Flambez à la fine champagne ; mettez les tomates pelées, épépinées et coupées
grossièrement ; mouillez avec le fond de poisson, relevez avec plus ou moins de
cayenne, au goût ; laissez cuire pendant une vingtaine de minutes.

Disposez en couronne, sur un plat, les tronçons de queues de homard, les
pinces et les coffres au milieu ; tenez au chaud.

Passez au tamis les intérieurs, le corail, les œufs et le sang des homards ;
liez le liquide de cuisson avec ce mélange, en chauffant sans laisser bouillir.
Finissez la sauce avec le reste du beurre, du jus de citron et de l’estragon ;
goûtez, complétez, s'il y a lieu, l'assaisonnement qui doit être relevé,
chauffez encore pendant un moment, puis masquez avec cette sauce les morceaux
de homard.

Servez, en envoyant en même temps, dans un légumier, du riz sauté au beurre,
\hyperlink{p0710}{p. \pageref{pg0710}}.

\sk

En modifiant la nature et la proportion du vin, des aromates et des condiments,
on conçoit facilement de nombreuses variantes.

\sk

Pour rendre le plat plus commode à manger, on pourra enlever tout ou partie de
la carapace des homards, mais on s'éloignera ainsi de l'aspect du plat
primitif. Dans le cas où l’on supprimerait complètement toute la carapace, on
préparera avec la sauce, la chair des pinces, des pattes et des coffres un
salpicon qu'on servira au milieu d'une couronne formée par les tronçons de
queues décortiquées dressés sur le pourtour d'un plat. La préparation, ainsi
présentée, n'aura plus l'aspect du homard à l'américaine, mais elle en aura
absolument le goût. Je la désignera, pour la différencier, sous le nom
générique de « civet de homard ».

\sk

\index{Civet de homard}
\index{Civet de langouste}
On pourra préparer absolument de même la langouste à l'américaine et le civet
de langouste.

\section*{\centering Salmis de homard à la crème.}
\addcontentsline{toc}{section}{ Salmis de homard à la crème.}
\index{Salmis de homard à la crème}
\label{pg0284} \hypertarget{p0284}{}

Pour quatre personnes prenez :

\medskip

\footnotesize
\begin{longtable}{rrrrp{16em}}   
  &   500 & grammes & de & vin blanc sec,                                                                 \\
  &   150 & grammes & de & crème,                                                                         \\
  &   100 & grammes & de & vin de Madère, de Xérès ou de Porto blanc, au goût,                            \\
  &   100 & grammes & de & carottes,                                                                      \\
  &   100 & grammes & d' & oignons,                                                                       \\
  &    60 & grammes & de & beurre,                                                                        \\
  &    40 & grammes & de & sel gris,                                                                      \\
  &     5 & grammes & de & sel blanc,                                                                     \\
  & \multicolumn{2}{r}{ 2 décigrammes } & de & poivre blanc,                                              \\
  & \multicolumn{2}{r}{ 2 décigrammes } & de & paprika,                                                   \\
  & \multicolumn{2}{r}{ 2 décigrammes } & de & noix de muscade,                                           \\
  & \multicolumn{2}{r}{ 2 centigrammes} & de & poivre de Cayenne,                                         \\
  &       &         & 40 & grains de poivre, concassés,                                                   \\
  &       &         &  2 & jaunes d'œufs frais,                                                           \\
  &       &         &  1 & homard œuvé en vie, pesant 1 kilogramme environ,                               \\
  &       &         &  1 & bouquet garni (thym, laurier, persil),                                         \\
  &       &         &    & truffes cuites dans du madère, à volonté,                                      \\
  &       &         &    & eau (quod sufficit pour couvrir le homard).                                    \\
\end{longtable}
\normalsize

\index{Court-bouillon pour homards}
\index{Court-bouillon pour grosses crevettes de la Méditerannée}
\index{Court-bouillon pour langoustes}
\index{Court-bouillon pour langoustines}

Faites cuire le homard pendant une demi-heure dans un court-bouillon préparé
avec le vin blanc sec, les carottes, les oignons, le sel gris, le poivre en
grains, le bouquet et plus ou moins d'eau. Laissez-le refroidir dans le
liquide, escalopez la queue et les pinces ; réservez les débris de chair,
l'intérieur et les œufs.

Mettez dans une casserole {\ppp50\mmm} grammes de beurre et faites revenir dedans les
escalopes de homard, mouillez avec le vin de Madère, de Xérès ou de Porto,
ajoutez le sel blanc, les épices, les truffes coupées en tranches et laissez
cuire ensemble à petit feu pendant dix minutes. Passez au tamis les débris de
chair, l'intérieur et les œufs de homard, incorporez-les à la sauce, liez avec
les jaunes d'œufs délayés dans la crème, ajoutez le reste du beurre coupé en
petits morceaux, chauffez sans laisser bouillir, puis servez.

Ce salmis est une très jolie entrée de repas fin.

\sk

\index{Crevettes de la Méditerranée en salmis}
\index{Langouste en salmis}
\index{Crevettes en salmis}

On peut, cela va sans dire, préparer de même un salmis de langouste, de
langoustines, de grosses crevettes de la Méditerranée. etc.

\sk

Tous ces salmis peuvent être présentés en timbale.


\section*{\centering Crevettes à l'américaine.}
\addcontentsline{toc}{section}{ Crevettes à l'américaine.}
\index{Crevettes à l'américaine}

Les grosses crevettes de la Méditerranée « Penoœus caramota », préparées de la
façon suivante, constituent un plat analogue au homard à l'américaine, mais
beaucoup plus fin.

Pour six personnes prenez :

\medskip

\index{Fumet de poisson}
\footnotesize
\begin{longtable}{rrrrp{16em}}   
  & 200 & grammes & de & tomates,                                                                         \\
  & 200 & grammes & de & beurre,                                                                          \\
  & 200 & grammes & de & vin de Sauternes.                                                                \\
  & 200 & grammes & de & bouillon de poisson,                                                             \\
  & 150 & grammes & de & fine champagne,                                                                  \\
  &  75 & grammes & de & fumet de poisson\footnote{\label{pg0285} \hypertarget{p0285}{}On prépare 
                                  le fumet de poisson d'une manière analogue à celle 
                                  qu'on emploie pour préparer les glaces de viandes.} 
                                  ou, à défaut, de glace de viande,                                       \\
  &  50 & grammes & d' & œufs de homard,                                                                  \\
  &  25 & grammes & d' & huile d'olive,                                                                   \\
  &  25 & grammes & d' & échalotes,                                                                       \\
  &   8 & grammes & de & sel,                                                                             \\
  &     & grammes & de & poivre,                                                                          \\
  &     & grammes & de & quatre épices,                                                                   \\
  & \multicolumn{2}{r}{1 décigramme}  & de & poivre de Cayenne,                                           \\
  &     &         & 24 & grosses crevettes vivantes pesant ensemble 1 kilogramme environ,                 \\
  &     &         &  3 & piments d'Espagne,                                                               \\
  &     &         &    & persil,                                                                          \\
  &     &         &    & cerfeul,                                                                         \\
  &     &         &    & estragon,                                                                        \\
  &     &         &    & jus de citron.                                                                   \\
\end{longtable}
\normalsize

\label{pg0287-2} \hypertarget{p0287-2}{}
Mettez dans une sauteuse {\ppp75\mmm} grammes de beurre et l'huile d'olive,
chauffez, puis jetez les crevettes vivantes dans le liquide fumant. Dès que les
crevettes auront élé saisies et se seront colorées, activez le feu, versez la
fine champagne, faites-la flamber, puis mouillez avec le sauternes et le
bouillon ; ajoutez le fumet de poisson ou la glace de viande, les tomates
pelées, épépinées et hachées, les échalotes, le sel, le poivre, les quatre
épices, le cayenne, les piments d'Espagne, couvrez et laissez cuire pendant dix
minutes.

Retirez alors les crevettes et maintenez-les au chaud au bain-marie. Concentrez
la cuisson ; passez-la.

Écrasez les œufs de homard, passez-les au tamis, mettez-les dans la cuisson,
goûtez, complétez l'assaisonnement s'il y a lieu, puis ajoutez persil, cerfeuil
et estragon hachés, le reste du beurre et du jus de citron.

Disposez les crevettes sur un plat, masquez-les avec la sauce et servez.

\sk

On peut aussi éplucher les crevettes, conserver les queues au chaud, puis faire
avec les parures et le beurre qui reste un beurre de crevettes,
\hyperlink{p0287-3}{p. \pageref{pg0287-3}}.

On disposera les queues sur un plat en les masquant avec la sauce. La
préparation est moins jolie à l'œil que celle obtenue avec les crevettes
entières, mais elle présente l'avantage de pouvoir être mangée sans qu'il soit
nécessaire d'y mettre les doigts.

\sk


On peut apprêter de même des coquilles Saint-Jacques, des
langoustines\footnote{ou homard de Norvège ; « Néphrops Norvegicus » ; famille
des Astacidés.}, etc.

C'est également très fin.

\section*{\centering Queues d’écrevisses\footnote{Astacus fluviatilis et Astacus leptodactylus ; famille des Astacidés.} en hors-d’œuvre.}
\addcontentsline{toc}{section}{ Queues d’écrevisses en hors-d’œuvre.}
\index{Queues d'écrevisses en hors-d'œuvre}

Pour six personnes prenez :

\medskip

\footnotesize
\begin{longtable}{rrrp{16em}}
  300 & grammes    & de & lait,                                                                           \\
  250 & grammes    & de & vin blanc sec,                                                                  \\
  250 & grammes    & d' & eau,                                                                            \\
  125 & grammes    & de & beurre,                                                                         \\
  100 & grammes    & de & fine champagne,                                                                 \\
   50 & grammes    & de & carottes coupées en rondelles,                                                  \\
   20 & grammes    & d' & oignon coupé en rondelles,                                                      \\
   20 & grammes    & de & sel gris,                                                                       \\
    5 & grammes    & d' & estragon,                                                                       \\
      &            & 24 & écrevisses moyennes,                                                            \\
      &            & 20 & grains de poivre concassés,                                                     \\
      &            & 1  & bouquet garni (persil, thym, laurier et fenouil).                               \\
\end{longtable}
\normalsize

\label{pg0287} \hypertarget{p0287}{}
Mettez à dégorger les écrevisses dans le lait pendant deux heures.

Préparez un court-bouillon avec le vin blanc, l'eau, la fine champagne, les
carottes, l'oignon, le sel, le poivre, l’estragon, le bouquet garni et 10
grammes de beurre, que vous ferez cuire ensemble pendant une vingtaine de
minutes.

Prenez les écrevisses, arrachez-leur l'intestin, jetez-les ensuite dans le
court-bouillon bouillant, faites-les cuire pendant un quart d'heure ;
laissez-les refroidir dans le liquide.

Épluchez les écrevisses, mettez les queues de côté dans un plat creux ; puis,
avec les parures et le reste du beurre, faites un beurre d'écrevisses qui vous
servira à masquer les queues.

\sk

\label{pg0287-3} \hypertarget{p0287-3}{}
On prépare le beurre d'écrevisses de la façon suivante :

Pilez dans un mortier les parures d'écrevisses ou, mieux encore, des écrevisses
entières cuites ; mettez-les dans une casserole avec du beurre, faites cuire
à petit feu, pendant un quart d'heure, sans laisser roussir, mouillez avec un
peu d’eau et faites bouillir pendant une demi-heure. Passez ensuite le tout au
tamis à l'aide d'un pilon en bois ; le beurre coloré et parfumé montera à la
surface du liquide passé ; recueillez-le et égouttez-le.

Le beurre sera naturellement d'autant plus parfumé que vous aurez employé
une plus grande proportion d'écrevisses.

\sk

On prépare d'une façon analogue le beurre de crevettes.

\section*{\centering Écrevisses à la bordelaise.}
\addcontentsline{toc}{section}{ Écrevisses à la bordelaise.}
\index{Écrevisses à la bordelaise}

Pour dix personnes prenez :

\medskip

\footnotesize
\begin{longtable}{rrrp{16em}}
  800 & grammes    & de & lait,                                                                           \\
  600 & grammes    & de & vin de Bordeaux blanc. sec,                                                     \\
  600 & grammes    & de & beurre,                                                                         \\
  200 & grammes    & de & fine champagne,                                                                 \\
  200 & grammes    & de & fumet de poisson, \hyperlink{p0285}{p. \pageref{pg0285}},                       \\
  200 & grammes    & de & velouté maigre, pp. \hyperlink{p0292}{\pageref{pg0292}}, \hyperlink{p0388}{\pageref{pg0388}},  \\
  200 & grammes    & de & sauce tomate concentrée,                                                        \\
  200 & grammes    & de & parties rouges de carottes,                                                     \\
  200 & grammes    & d' & oignons,                                                                        \\
   20 & grammes    & d' & échalotes,                                                                      \\
   20 & grammes    & de & persil,                                                                         \\
   10 & grammes    & d' & estragon,                                                                       \\
      &            & 60 & belles écrevisses vivantes,                                                     \\
      &            &    & thym, laurier,                                                                  \\
      &            &    & sel, poivre, cayenne.                                                           \\
\end{longtable}
\normalsize

Faites dégorger les écrevisses dans le lait pendant deux heures.

Hachez ensemble carottes, oignons, échalotes, estragon et {\ppp10\mmm} grammes de persil.

Mettez dans une casserole {\ppp100\mmm} grammes de beurre, le hachis ci-dessus, du thym,
du laurier, du sel et du poivre ; laissez cuire.

Arrachez les intestins des écrevisses ; jetez-les vivantes dans la casserole,
saupoudrez-les avec un peu de cayenne et faites-les sauter jusqu'à ce qu'elles
soient bien rouges ; versez la fine champagne ; faites flamber.

Enlevez les écrevisses ; tenez-les au chaud.

Mettez alors le vin ; réduisez la cuisson d'un tiers environ, puis ajoutez les
écrevisses, le fumet de poisson, le velouté maigre, la sauce tomate ; laissez
cuire le tout ensemble pendant une dizaine de minutes.

Dressez les écrevisses dans une timbale tenue au chaud.

Passez la sauce, concentrez-la, montez-la avec le reste du beurre, ajoutez le
reste du persil blanchi et haché ; masquez-en les écrevisses.

Servez avec du bordeaux blanc, sec, vieux.

\section*{\centering Écrevisses au gratin.}
\addcontentsline{toc}{section}{ Écrevisses au gratin.}
\index{Écrevisses au gratin}

Pour six personnes prenez :

\footnotesize
\begin{longtable}{rrrp{16em}}
  125 & grammes    & de & crème épaisse,                                                                  \\
   50 & grammes    & de & beurre,                                                                         \\
   50 & grammes    & de & gruyère,                                                                        \\
   50 & grammes    & de & parmesan,                                                                       \\
   30 & grammes    & de & jambon maigre haché,                                                            \\
   15 & grammes    & de & farine,                                                                         \\
   10 & grammes    & de & mie de pain rassis tamisée,                                                     \\
      &            & 50 & Écrevisses,                                                                     \\
      &            &  2 & jaunes d'œufs frais,                                                            \\
      &            &    & muscade,                                                                        \\
      &            &    & sel et poivre.                                                                  \\
\end{longtable}
\normalsize

Court-bouillonnez les écrevisses.

Décortiquez les queues et les pinces.

Préparez un beurre d'écrevisses avec les parures et {\ppp55\mmm} grammes de beurre.

Faites revenir la farine et le jambon dans le reste du beurre ; mouillez avec
la quantité nécessaire de court-bouillon des écrevisses, ajoutez la chair des
pinces et les queues des écrevisses, du sel, du poivre, de la muscade au goût ;
laissez mijoter pendant quelques minutes. Mettez ensuite la crème, {\ppp40\mmm} grammes
de gruyère, {\ppp40\mmm} grammes de parmesan râpé, mélangez, chauffez sans laisser
bouillir et liez avec les jaunes d'œufs.

Prenez un plat de service allant au feu, versez dedans l'appareil, saupoudrez
avec le reste des fromages mélangés avec la mie de pain et faites gratiner au
four.

\section*{\centering Écrevisses au pilaf ou pilaf d’écrevisses.}
\addcontentsline{toc}{section}{ Écrevisses au pilaf ou pilaf d’écrevisses.}
\index{Écrevisses au pilaf ou pilaf d'écrevisses}

Pour six personnes prenez :

\medskip

\footnotesize
\begin{longtable}{rrrp{16em}}
  500 & grammes    & de & bouillon de poisson un peu relevé,                                              \\
  250 & grammes    & de & riz d'Égypte,                                                                   \\
  175 & grammes    & de & beurre,                                                                         \\
      &            & 36 & écrevisses.                                                                     \\
\end{longtable}
\normalsize

Faites cuire les écrevisses comme il est dit
\hyperlink{p0287}{p. \pageref{pg0287}}, épluchez-les, réservez les queues ; mettez
de côté les parures et les pattes. Avec ces déchets et le beurre, faites un
beurre d'écrevisses.

Préparez ensuite un pilaf, conformément aux indications données
\hyperlink{p0712}{p. \pageref{pg0712}}, en remplaçant simplement dans la formule
le bouillon de viande par le bouillon de poisson et le beurre ordinaire par le
beurre d’écrevisses.

Dressez ce pilaf sur un plat et garnissez-le avec les queues d'écrevisses.

\section*{\centering Chaussons aux écrevisses.}
\addcontentsline{toc}{section}{ Chaussons aux écrevisses.}
\index{Chaussons aux écrevisses}

Préparez une pâte demi-feuilletée\footnote{La composition de la pâte demi-feuilletée est la suivante :
                                            \label{pg0289} \hypertarget{p0289}{}
                                            \begin{tabular}{r r r l}
                                                500 & grammes & de & farine de blé dur de Hongrie,        \\
                                                375 & grammes & de & beurre ferme et élastique,           \\
                                                 30 & grammes & d' & eau,                                 \\
                                                 10 & grammes & de & sel fin.                             \\
                                            \end{tabular} 
                                            \\
                                           L'opération se fait comme pour la pâte feuilletée
                                           \hyperlink{p0319}{p. \pageref{pg0319}}, mais au lieu de donner 
                                           six tours à la pâte, on n'en donne que cinq.},

ou, si vous êles pressé, procurez-vous en chez un pâtissier. Abaissez-la à une
épaisseur de {\ppp4\mmm} à {\ppp5\mmm} millimètres et découpez-la en cercles de
{\ppp12\mmm} à {\ppp15\mmm} centimètres de diamètre.

Faites cuire des écrevisses comme dans la formule précédente, mais en
remplaçant le vin blanc sec par du sauternes, notamment du vin de Château
Yquem, ou encore par du xérès, et en supprimant le persil ajouté à la fin.

Décortiquez les écrevisses, escalopez les queues ; passez les débris au tamis,
ajoutez le produit obtenu à la sauce, réduisez-la de façon à l’amener à bonne
consistance ; mettez dedans les escalopes d'écrevisses : vous aurez ainsi le
remplissage des chaussons.

Placez sur la moitié de chaque rond de pâte une certaine quantité de l'appareil
de remplissage, fermez en chaussons et faites cuire au four.

Servez avec du vin de Château Yquem ou du xérès suivant que l’un ou l'autre
de ces vins aura été employé dans la préparation des chaussons.

\index{Mollusques}

\section*{\centering Huîtres.}
\addcontentsline{toc}{section}{ Huîtres.}
\index{Huîtres}

L'huître « Ostrea edulis », de la famille des Ostréidés, est un mollusque
lamellibranche, bivalve, acéphale, hermaphrodite.

Parmi les variétés comestibles, les plus estimées sont les huîtres de Belon, de
Marennes, d'Ostende, les côtes-rouges. les natives d'Angleterre, les burnham,
les colchester et les natives de Zélande. Ces huîtres-là, qui ne doivent être
mangées que crues et vivantes, constituent un aliment de luxe délicat et
léger ; d’autres, moins fines, peuvent être employées dans les sauces ou dans
les garnitures.

Comment doit-on manger les huîtres crues ?

Certaines personnes les avalent sans les mâcher. Que ne les prennent-elles en
cachets ! D'autres consentent à les mâcher, mais après les avoir arrosées de
sauces incendiaires qui en masquent absolument le goût, et dont je ne
comprendrais l'emploi que si j'étais condamné à manger des pieds de cheval ou
des portugaises\footnote{La portugaise, \textit{Gryphæa angulata}, n'est pas
une huître proprement dite ; c'est une gryphée.} ; d'autres enfin les
additionnent simplement de jus de citron et les accompagnent de tartines de
pain noir beurré ou de sandwichs au caviar. Chacun croit employer le procédé le
meilleur. Voici celui des amateurs qui estiment que la bonne huître mérite
d'être aimée pour elle-même.

Faites ouvrir, seulement au moment de les manger, des huîtres appartenant aux
espèces préférées, grasses, charnues, et assurez-vous pour chacune qu'elle est
bien vivante, en explorant ses réflexes ; c'est là un signe objectif qui ne
trompe pas\footnote{La recherche des réflexes chez l'huître se fait très
commodément : on touche simplement le bord des lamelles qui se rétractent si
l'animal est vivant.}. Puis, enlevez-la délicatement de sa coquille, portez-la
immédiatement à la bouche, toute nue, sans aucun accompagnement et aussitôt,
d'un coup de dent, percez-lui le foie. Si le sujet répond à ce que vous êtes en
droit d'attendre de lui, vos gencives doivent baigner dedans tout entières et
votre bouche doit être inondée de jus\footnote{C'est là pour moi un véritable
criterium, « a posteriori », de l'huître à point.}.

Restez un instant dans cette situation. puis avalez lentement le jus et achevez
la mastication et la déglutition du mollusque. Tonifiez-vous alors avec une
gorgée de bon vin blanc sec, mangez une bouchée de pain blanc ou noir, beurré
ou non, pour neutraliser les papilles de la langue et être en état d'apprécier
intégralement l'huître suivante.

\sk

On a accusé les huîtres de beaucoup de méfaits : il paraît établi que lorsque
les parcs sont mal tenus, en communication avec des fosses à purin, elles
peuvent donner la fièvre typhoïde. Le seul enseignement à tirer de ces faits
est de n'user que d'huîtres de bonne provenance.

\section*{\centering Bouchées\footnote{\index{Bouchées au maigre}
Les bouchées sont de petits vol-au-vent. On peut encore donner le nom de
bouchées à des petites préparations culinaires farcies dont l'enveloppe est
constituée par une pâte renfermant de la chair animale : poisson, volaille,
gibier, etc.} aux huîtres.}

\addcontentsline{toc}{section}{ Bouchées aux huîtres.}
\index{Bouchées aux huîtres}

Faites un feuilleté comme il est dit \hyperlink{p0319}{p. \pageref{pg0319}} ou, si
vous avez un bon pâtissier dans le voisinage, commandez-lui des croûtes de
bouchées et préparez-en la garniture, comme il suit.

\medskip

Pour dix bouchées prenez :

\medskip

\footnotesize
\begin{longtable}{rrrp{16em}}
  500 & grammes    & de & lait,                                                                           \\
  500 & grammes    & de & champignons de couche,                                                          \\
  300 & grammes    & de & beurre,                                                                         \\
  250 & grammes    & de & crème,                                                                          \\
  125 & grammes    & de & crevettes grises vivantes,                                                      \\
   50 & grammes    & de & farine,                                                                         \\
      &            & 60 & huîtres portugaises ou 60 cancales, grasses,                                    \\
      &            &    & jus de citron.                                                                  \\
\end{longtable}
\normalsize

Ouvrez les huîtres, recueillez dans une casserole les mollusques et leur eau ;
portez le tout à ébullition ; après le premier bouillon enlevez les huîtres
blanchies ; tenez-les au chaud ; passez l'eau.

Préparez un court-bouillon comme il est dit \hyperlink{p0287}{p. \pageref{pg0287}}
en remplaçant l'eau par l'eau des huîtres ; faites cuire dedans les crevettes,
épluchez-les, réservez les queues, passez à la presse les parures, ajoutez le
jus obtenu à l'eau de cuisson des crevettes, filtrez le tout et concentrez-le
de façon à avoir 200 grammes de liquide environ.

Mettez dans une casserole les champignons épluchés, {\ppp25\mmm} grammes de beurre et
un filet de jus de citron ; laissez cuire.

Faites blondir la farine dans {\ppp250\mmm} grammes de beurre, mouillez avec le lait et
le jus concentré ; laissez bouillir ensemble pour réduire encore le tout.
Ajoutez ensuite la crème en travaillant bien, puis le reste du beurre, les
champignons, les huîtres, les queues de crevettes. Tenez au chaud au bain-marie
jusqu'au moment de servir.

Ce moment venu, emplissez les croûtes chaudes avec cette garniture.

Les bouchées ainsi préparées sont incomparablement plus délicates que les
produits similaires du commerce.

\section*{\centering Huîtres frites.}
\addcontentsline{toc}{section}{ Huîtres frites.}
\index{Huîtres frites}

Pour douze personnes prenez :

\medskip

\footnotesize
\begin{longtable}{rrrp{16em}}
  500 & grammes   & de & velouté maigre\footnote{\label{pg0292} \hypertarget{p0292}{}Pour préparer 
                                      un velouté maigre, prenez du bouillon de poisson, 
                                      \hyperlink{p0218}{p. \pageref{pg0218}} ; relevez-le avec du vin 
                                      blanc, des aromates, des condiments, des champignons 
                                      au goût ; Liez-le avec du beurre manié de farine ou 
                                      avec un roux ; concentrez-le ; dépouillez-le.},                     \\
  500 & grammes   & de & gelée maigre, \hyperlink{p0350}{p. \pageref{pg0350}},                            \\
    4 & douzaines & d’ & huîtres grasses\footnote{Il est inutile de prendre des
                                      huîtres d'une grande finesse ; il faut surtout
                                      qu'elles soient grasses à point.},                                  \\
      &           &  6 & citrons,                                                                         \\
      &           &  4 & œufs entiers,                                                                    \\
      &           &  4 & jaunes d'œufs,                                                                   \\
      &           &    & mie de pain rassis tamisée,                                                      \\
      &           &    & crème,                                                                           \\
      &           &    & farine,                                                                          \\
      &           &    & persil.                                                                          \\
\end{longtable}
\normalsize

Ouvrez les huîtres ; réservez leur eau.

Mettez dans une casserole le velouté, la gelée et l'eau des huîtres passée et
réduite : achevez la liaison de l'ensemble avec les jaunes d'œufs et de la
crème de manière à avoir une sauce chaud-froid\footnote{D'une façon générale,
les sauces chaud-froid sont constituées par un mélange en proportions égales de
sauces à base de consommé de substances animales, blondes ou brunes, avec de la
gelée blonde ou brune.

Pour les poissons, les mollusques et les crustacés, on se sert d'un mélange de
velouté, d'allemande ou de sauce suprême maigre et de gelée maigre.

Les sauces chaud-froid destinées aux viandes blanches sont préparées avec de la
sauce veloutée, de l'allemande ou de la sauce suprême et de la gelée blonde,

Pour les sauces chaud-froid destinées aux viandes noires, on emploie de
l'espagnole et de la gelée brune.

Pour le gibier, on prend de l'espagnole au gibier et de la gelée de gibier.}
très consistante.

Cassez les quatre œufs entiers ; battez-les.

Enrobez les huîtres dans la sauce chaud-froid, passez-les ensuite légèrement
dans de la farine, puis dans les œufs battus, enfin dans de la mie de pain et
plongez-les une à une dans de l'huile bouillante. Dès que l'enveloppe sera
dorée, retirez-les.

Au moment de servir, réchauffez-les pendant quelques secondes dans la friture,
dressez-les sur un plat que vous garnirez avec du persil frit et les citrons
coupés en deux.

Les huîtres frites, contrairement à l'opinion de certaines personnes, ne
relèvent pas du tout de la cuisine barbare. Elles sont infiniment meilleures
préparées comme je l'indique que simplement frites en beignets car, n'ayant pas
eu le temps de cuire, elles sont très moelleuses et conservent leur goût. Je
les préfère même aux huîtres à la Villeroi. Enfin, la sauce employée donne au
plat le caractère maigre qui lui convient.

\section*{\centering Huîtres en aspic.}
\addcontentsline{toc}{section}{ Huîtres en aspic.}
\index{Huîtres en aspic}

Préparez une gelée d'aspic maigre en faisant cuire des poissons ou des débris
de poissons (merlan ou brochet par exemple) dans de la gelée de veau ;
clarifiez-la.

Mettez dans des moules à dariole un peu de cette gelée, puis une ou plusieurs
huîtres simplement blanchies dans leur eau ; au-dessus, de la mayonnaise à la
moutarde, aromatisée ou non avec des œufs de homard ; couvrez de gelée ;
continuez ces alternances jusqu'à ce que les moules soient pleins. Faites
prendre sur glace.

Démoulez au moment de servir.

\sk

\index{Aspic d'écrevisses}
\index{Aspic de crevettes}
\index{Aspic de caviar}
\index{Aspic de corail d'oursins}
\index{Aspic de moules}

On peut préparer de même des aspics de queues d'écrevisses ou de crevettes, de
caviar, de corail d'oursins, de moules, etc.

\sk

Tous ces aspics constituent des hors-d'œuvre très appétissants.

On peut aussi les employer comme garniture autour d'un poisson froid en gelée,
ce qui augmentera beaucoup la richesse du plat.

\section*{\centering Coquilles Saint-Jacques.}
\addcontentsline{toc}{section}{ Coquilles Saint-Jacques.}
\index{Coquilles Saint-Jacques}

Les peignes ou coquilles Saint-Jacques, « Pecten maximus » et « Pecten
Jacobæus », sont des mollusques pélécypodes de la famille des Pectinidés. Leur
chair est très délicate.

Voici l’une des meilleures façons de les préparer.

\medskip

Pour quatre personnes prenez :

\medskip

\footnotesize
\begin{longtable}{rrrp{16em}}
  250 & grammes    & de & champignons de couche,                                                          \\
  250 & grammes    & de & vin blanc sec,                                                                  \\
  100 & grammes    & de & beurre,                                                                         \\
   30 & grammes    & de & fine champagne,                                                                 \\
   20 & grammes    & de & farine,                                                                         \\
      &            &  4 & belles coquilles Saint-Jacques,                                                 \\
      &            &  1 & oignon,                                                                         \\
      &            &  1 & échalote,                                                                       \\
      &            &  1 & bouquet garni (persil, thym, laurier),                                          \\
      &            &    & mie de pain rassis tamisée,                                                     \\
      &            &    & persil,                                                                         \\
      &            &    & jus de citron,                                                                  \\
      &            &    & sel et poivre.                                                                  \\
\end{longtable}
\normalsize

Ouvrez les coquilles et recueillez leur eau. Lavez les mollusques à plusieurs
reprises dans de l'eau fraîche ; enlevez les parties noires, mettez de côté
séparément le blanc, le corail et les barbes.

Prenez les neuf dixièmes du vin blanc et de la fine champagne, faites cuire
dedans le blanc et le corail avec le bouquet garni, l'oignon, du sel, du
poivre ; retirez-les et remplacez-les par les barbes ; laissez-les cuire.

Passez la cuisson.

Coupez en morceaux un peu gros le blanc et le corail.

Réservez les barbes.

Faites cuire les champignons avec du beurre, un peu de sel et du jus de citron.

Hachez fin les barbes, les champignons et le persil.

Mettez dans uné casserole du beurre, la farine et l'échalote hachée, laissez
dorer légèrement ; mouillez avec une quantité suffisante de jus de cuisson
additionné d’eau filtrée des mollusques, ajoutez le hachis de barbes,
champignons et persil : cela constituera la farce.

Prenez alors les valves creuses des quatre coquilles, foncez-les de beurre,
étendez sur le beurre un quart du blanc et du corail, au-dessus un quart de la
farce, saupoudrez de mie de pain tamisée et terminez par un petit morceau de
beurre ; arrosez avec le reste du vin et de la fine champagne mélangés.

Mettez au four pendant dix minutes pour gratiner.

Au gaz, il faudrait le double de temps environ.

\section*{\centering Coquilles Saint-Jacques.}
\addcontentsline{toc}{section}{ Coquilles Saint-Jacques.}
\index{Coquilles Saint-Jacques}

\begin{center}
\sc\small(Autre formule)
\end{center}

\bigskip

Pour quatre personnes prenez :

\footnotesize
\begin{longtable}{rrrp{16em}}
    200 & grammes & de & vin blanc,                                                                       \\
    150 & grammes & de & champignons de couche,                                                           \\
    100 & grammes & de & crème,                                                                           \\
     50 & grammes & de & beurre,                                                                          \\
     20 & grammes & de & farine,                                                                          \\
        &         &  8 & coquilles Saint-Jacques,                                                         \\
        &         &  3 & échalotes,                                                                       \\
        &         &    & chapelure,                                                                       \\
        &         &    & sel et poivre.                                                                   \\
\end{longtable}
\normalsize

Ouvrez les coquilles, détachez les mollusques, lavez-les à plusieurs reprises
dans de l’eau fraîche ; enlevez les parties noires et les barbes ; réservez
séparément le blanc et le corail. Coupez le blanc en morceaux, laissez le
corail entier ; faites- les revenir dans {\ppp30\mmm} grammes de beurre, ajoutez ensuite
les échalotes hachées, la farine, les champignons pelés et émincés ; remuez
pendant quelques instants, puis mouillez avec le vin blanc ; salez, poivrez au
goût. Laissez cuire pendant une dizaine de minutes ; réduisez la cuisson.

Mettez alors la crème ; concentrez la sauce sans faire bouillir.

Emplissez quatre valves creuses des coquilles avec l'appareil ci-dessus,
saupoudrez de chapelure, mettez le reste du beurre coupé en petits morceaux.
Faites gratiner au four,

\section*{\centering Coquilles Saint-Jacques à la portugaise.}
\addcontentsline{toc}{section}{ Coquilles Saint-Jacques à la portugaise.}
\index{Coquilles Saint-Jacques à la portugaise}
\index{Appareil portugais}

Pour quatre personnes prenez :

\medskip

\footnotesize
\begin{longtable}{rrrp{16em}}
    300 & grammes & de & tomates,                                                                         \\
    125 & grammes & de & champignons de couche,                                                           \\
    125 & grammes & de & beurre,                                                                          \\
        &         &  8 & coquilles Saint-Jacques,                                                         \\
        &         &  1 & merlan,                                                                          \\
        &         &  1 & carotte,                                                                         \\
        &         &  1 & oignon,                                                                          \\
        &         &    & vin blanc,                                                                       \\
        &         &    & jus de citron,                                                                   \\
        &         &    & persil haché,                                                                    \\
        &         &    & thym, laurier,                                                                   \\
        &         &    & sel et poivre.                                                                   \\
\end{longtable}
\normalsize

Ouvrez les coquilles ; recueillez l'eau ; lavez les mollusques à l'eau froide ;
mettez à part le blanc et le corail ; réservez les barbes.

Videz le merlan ; lavez-le ; coupez-le en tronçons.

Mettez dans une casserole du vin blanc, l'eau des coquilles, le merlan, les
barbes réservées, l'oignon et la carotte coupés en morceaux, du thym, du
laurier, du sel et du poivre ; laissez cuire jusqu'à obtention de quelques
cuillerées de liquide. Passez ce fumet ; ajoutez-y un peu de beurre et faites
pocher dedans le blanc et le corail réservés.

En même temps, préparez un appareil portugais en faisant cuire à la casserole,
dans un peu de beurre, les tomates pelées et épépinées, les champignons pelés,
passés au jus de citron et hachés ; assaisonnez avec sel et poivre, ajoutez du
persil ; amenez par concentration l'appareil à la consistance d'un velouté. Au
dernier moment, incorporez-y, en fouettant, le reste du beurre.

Dressez, dans un plat en porcelaine allant au feu, le blanc et le corail pochés
dans le fumet ; masquez-les avec l'appareil portugais ; chauffez au four pendant
quelques minutes ; puis servez.

C'est absolument exquis.

\section*{\centering Ragoût de coquilles Saint-Jacques.}
\addcontentsline{toc}{section}{ Ragoût de coquilles Saint-Jacques.}
\index{Ragoût de coquilles Saint-Jacques}
\index{Coquilles Saint-Jacques en ragoût}

Pour quatre personnes prenez :

\medskip

\footnotesize
\begin{longtable}{rrrp{16em}}
    250 & grammes & de & bonne sauce tomate,                                                              \\
     60 & grammes & de & beurre,                                                                          \\
     50 & grammes & de & fine champagne ou-de cognac,                                                     \\
      5 & grammes & de & persil,                                                                          \\
      1 & gramme  & d’ & ail,                                                                             \\
        &         &  8 & coquilles Saint-Jacques,                                                         \\
        &         &    & parmesan ou gruyère,                                                             \\
        &         &    & sel, poivre, paprika.                                                            \\
\end{longtable}
\normalsize

Enlevez les mollusques des coquilles ; lavez-les bien dans de l'eau froide.

Faites revenir le blanc, coupé en morceaux, et le corail dans {\ppp30\mmm} grammes de
beurre, flambez avec la fine champagne ou le cognac, salez, poivrez, ajoutez
l'ail haché et la sauce tomate ; laissez cuire pendant une dizaine de minutes ;
mettez ensuite le persil haché et continuez la cuisson pendant quelques minutes
encore. Goûtez, relevez avec un peu de paprika.

Emplissez les coquilles avec le ragoût, saupoudrez de parmesan ou de gruyère
râpé, mettez dessus le reste du beurre coupé en petits morceaux ; poussez au
four pour gratiner.

\section*{\centering Coquilles Saint-Jacques froides.}
\addcontentsline{toc}{section}{ Coquilles Saint-Jacques froides.}
\index{Coquilles Saint-Jacques froides}

Enlevez les mollusques des coquilles ; lavez-les à l'eau froide ; puis faites
cuire le blanc et le corail dans du vin blanc avec sel, poivre, échalotes et
aromates au goût. Retirez le blanc et le corail ; escalopez le blanc ;
concentrez fortement la cuisson ; passez-la. Laissez refroidir.

Garnissez les valves profondes des coquilles avec le blanc, le corail et des
rondelles d'œufs durs, masquez avec une mayonnaise simple ou composée dans
laquelle vous aurez fait entrer le jus de cuisson concentré des coquilles.

C'est un excellent hors-d'œuvre.

\section*{\centering Moules\footnote{Mytilus edulis ; famille des Mytilidés.} au naturel.}
\addcontentsline{toc}{section}{ Moules au naturel.}
\index{Moules au naturel}

Pour quatre personnes prenez :

\medskip

\footnotesize
\begin{longtable}{rrrp{16em}}
      3 & litres & de & moules,                                                                           \\
        &        &    & persil,                                                                           \\
        &        &    & sel et poivre.                                                                    \\
\end{longtable}
\normalsize

Grattez soigneusement les moules une par une ; jetez celles qui ne se referment
pas, elles sont mortes ; lavez-les jusqu'à ce que l'eau soit absolument
claire ; égouttez-les.

Mettez dans une casserole les moules, du persil, du sel et du poivre au goût,
Faites sauter le tout ensemble de façon que les moules cuisent rapidement, ce
qui demande quatre à cinq minutes. Enlevez les moules et tenez-les au chaud
dans un plat couvert ; passez l'eau au chinois ; laissez-la déposer ;
décantez-la.

Servez les moules dans le plat et le jus à part.

C'est la façon la plus simple de préparer les moules.

\sk

Les moules au naturel, refroidies, sont excellentes en salade, assaisonnées
avec une mayonnaise.

\section*{\centering Moules au citron.}
\addcontentsline{toc}{section}{ Moules au citron.}
\index{Moules au citron}

Pour quatre personnes prenez :

\medskip

\footnotesize
\begin{longtable}{rrrp{16em}}
    100 & grammes & de & carottes,                                                                        \\
     65 & grammes & de & beurre,                                                                          \\
     30 & grammes & d’ & échalotes,                                                                       \\
     20 & grammes & de & farine,                                                                          \\
        & 3 litres& de & moules,                                                                          \\
        &         &  4 & citrons,                                                                         \\
        &         &    & bouquet garni,                                                                   \\
        &         &    & sel, poivre, qualre épices.                                                      \\
\end{longtable}
\normalsize

Nettoyez les moules comme il est dit plus haut.

Faites cuire dans {\ppp30\mmm} grammes de beurre carottes et échalotes émincées, bouquet
garni, le tout assaisonné au goût avec sel, poivre et quatre épices. Déglacez
avec le jus des citrons, puis ajoutez les moules et faites-les cuire rapidement
en les sautant. Tenez-les au chaud.

Faites dorer légèrement la farine dans {\ppp15\mmm} grammes de beurre, mouillez avec
l'eau des moules passée et décantée ; laissez cuire pendant un moment ; enfin,
montez la sauce avec le reste du beurre.

Dressez les moules débarrassées d’une de leurs valves sur un plat et servez.

Envoyez la sauce dans une saucière.

La moule se marie très bien avec le citron,

\section*{\centering Moules au vin blanc.}
\addcontentsline{toc}{section}{ Moules au vin blanc.}
\index{Moules au vin blanc}

Pour quatre personnes prenez :

\medskip

\footnotesize
\begin{longtable}{rrrp{16em}}
    200 & grammes & de & vin blanc,                                                                       \\
    100 & grammes & de & carottes,                                                                        \\
     30 & grammes & d' & échalotes,                                                                       \\
     30 & grammes & de & beurre,                                                                          \\
      1 & gramme  & d' & ail,                                                                             \\
        & 3 litres& de & moules,                                                                          \\
        &         &    & bouquet garni,                                                                   \\
        &         &    & persil,                                                                          \\
        &         &    & sel et poivre.                                                                   \\
\end{longtable}
\normalsize

Nettoyez les moules comme précédemment.

Mettez dans une casserole les carottes coupées en rondelles, les échalotes,
l'ail, le bouquet, le vin et faites cuire le tout avec un peu de sel et de
poivre pendant une demi-heure environ. Passez, ajoutez ensuite les moules et
sautez le tout jusqu'à ce que les moules soient cuites. Filtrez et décantez la
cuisson ; montez-la au beurre et saupoudrez de persil haché.

Enlevez une valve des coquilles ; servez les moules sur un plat et la sauce
à part dans une saucière.

C'est, à peu de chose près, la formule classique des moules à la marinière.

\section*{\centering Moules à la crème.}
\addcontentsline{toc}{section}{ Moules à la crème.}
\index{Moules à la crème}

Pour quatre personnes prenez :

\medskip

\footnotesize
\begin{longtable}{rrrp{16em}}
    250 & grammes &  de & champignons,                                                                    \\
    200 & grammes &  de & crème,                                                                          \\
    100 & grammes &  de & beurre,                                                                         \\
    100 & grammes &  de & carottes,                                                                       \\
     30 & grammes &  de & farine,                                                                         \\
        & 3 litres&  de & moules,                                                                         \\
        &         &     & vinaigre (facultatif),                                                          \\
        &         &     & bouquet garni,                                                                  \\
        &         &     & sel et poivre.                                                                  \\
\end{longtable}
\normalsize

Nettoyez soigneusement les moules.

Pelez les champignons ; hachez-les.

Faites cuire à petit feu les carottes coupées en rondelles minces et le bouquet
garni dans {\ppp40\mmm} grammes de beurre, pendant une demi-heure environ ; salez et
poivrez ; ajoutez ensuite les moules et faites-les s'ouvrir en les sautant
pendant le temps nécessaire pour qu'elles soient cuites. Tenez-les au chaud.

Mettez dans une casserole {\ppp30\mmm} grammes de beurre et les champignons hachés ;
faites-les suer, puis ajoutez la farine, tournez sans laisser prendre couleur,
mouillez avec l’eau des moules passée au chinois et décantée. Laissez cuire
pendant quelques minutes ; enfin, montez la sauce avec le reste du beurre et la
crème ; goûtez et ajoutez du vinaigre, si vous l'aimez.

Après avoir enlevé une valve à chaque coquille, dressez les moules sur un plat
ou sur autant de plats qu'il y a de convives, masquez-les avec la sauce et
servez.

Les moules à la crème sont exquises et font très bonne figure dans un déjeuner
d'amateurs.

\section*{\centering Moules au gratin.}
\addcontentsline{toc}{section}{ Moules au gratin.}
\index{Moules au gratin}

Pour six personnes prenez :

\medskip

\footnotesize
\begin{longtable}{rrrp{16em}}
    100 & grammes  & de & crème,                                                                          \\
     50 & grammes  & de & beurre,                                                                         \\
     50 & grammes  & de & parmesan râpé,                                                                  \\
     50 & grammes  & de & gruyère râpé,                                                                   \\
     15 & grammes  & de & farine,                                                                         \\
      10& grammes  & de & mie de pain rassis tamisée,                                                     \\
        & 3 litres & de & moules,                                                                         \\
        &          &  1 & beau tourteau.                                                                  \\
\end{longtable}
\normalsize

Faites cuire le tourteau, au court-bouillon, comme une langouste, videz-le,
réservez séparément la chair\footnote{La chair du tourteau peut être servie
à part, comme de la chair de langouste, avec une sauce à l'huile et au vinaigre
ou avec une mayonnaise. On pourrait aussi, dans la préparation, la mélanger
avec les moules, mais j'estime qu'il vaut mieux s'en servir à part.} et
l'intérieur.

Passez l'intérieur en purée, réservez. Un beau tourteau fournit {\ppp200\mmm} grammes
de purée environ.

Mettez les moules bien nettoyées dans une casserole avec du sel et du poivre,
faites-les s'ouvrir à feu vif ; au bout de quatre à cinq minutes elles seront
cuites. Sortez-les des coquilles et tenez-les au chaud.

Passez l'eau des moules et concentrez-la.

Faites blondir la farine dans le beurre, mettez la purée de tourteau, donnez
quelques bouillons ; ajoutez la crème, {\ppp45\mmm} grammes de gruyère et 45 grammes de
parmesan, mélangez, goûtez, mouillez avec la quantité d'eau de moules
nécessaire et suffisante pour assaisonner convenablement, chauffez encore un
peu jusqu'à bonne consistance.

Prenez un plat allant au feu, foncez-le d'un peu de cet appareil, disposez
dessus les moules, couvrez-les avec le reste de l'appareil, saupoudrez avec le
reste des fromages mélangés à la mie de pain et faites gratiner, à four doux,
pendant un quart d'heure.

\sk

On peut, dans le même esprit, faire un tourteau gratiné, en remplaçant les
moules par la chair du tourteau ; dans ce cas, on sert la préparation dans la
carapace du tourteau.

\sk

Il va sans dire que les homards et les langoustes peuvent être apprêtés et
servis de même.

\section*{\centering Suçarelle de moules.}
\addcontentsline{toc}{section}{ Suçarelle de moules.}
\index{Suçarelle de moules}

Pour quatre personnes prenez :

\medskip

\footnotesize
\begin{longtable}{rrrp{16em}}
    100 & grammes  & de & crème,                                                                          \\
    100 & grammes  & de & beurre,                                                                         \\
    100 & grammes  & de & mie de pain rassis tamisée,                                                     \\
    100 & grammes  & de & purée de tomates,                                                               \\
      3 & grammes  & d' & ail haché,                                                                      \\
        & 3 litres & de & moules,                                                                         \\
        &          & 4  & anchois,                                                                        \\
        &          & 2  & jaunes d'œufs,                                                                  \\
        &          & 1  & oignon moyen,                                                                   \\
        &          &    & lait,                                                                           \\
        &          &    & sel, poivre, paprika.                                                           \\
\end{longtable}
\normalsize

Pilez les anchois avec {\ppp50\mmm} grammes de beurre ; passez le tout au tamis.

Nettoyez les moules ; mettez-les dans une casserole avec les trois quarts de
l'ail et faites-les s'ouvrir à feu vif. Enlevez-les des coquilles ; réservez-en
quelques-unes dans l’une de leurs valves ; tenez-les au chaud. Réduisez un peu
l'eau des moules ; passez-la, décantez-la, réservez-la.

Faites revenir dans du beurre l'oignon ciselé très fin et le reste de l'ail,
ajoutez le beurre d’anchois, la purée de tomates, la mie de pain trempée dans
du lait, égouttée et écrasée, du sel, du poivre et du paprika ; éclaircissez
avec l'eau des moules jusqu'à consistance convenable. Laissez cuire à petit feu
pendant {\ppp15\mmm} à {\ppp20\mmm} minutes.

Au dernier moment, liez avec la crème et les jaunes d'œufs sans faire
bouillir ; mettez les moules et le reste du beurre ; chauffez pendant un
instant et servez dans un plat dont vous aurez décoré le bord avec les moules
réservées.

\sk

On peut préparer une autre suçarelle en remplaçant dans la formule précédente
la crème et les jaunes d'œufs par de l'ailloli. Cela convient aux grands
mangeurs d'ail.

\sk

Comme variantes, on peut, dans les deux cas, faire gratiner le plat après
l'avoir saupoudré d'un mélange de mie de pain rassis tamisée et de fromage
râpé,

\section*{\centering Moules sautées panées.}
\addcontentsline{toc}{section}{ Moules sautées panées.}
\index{Moules sautées panées}

Pour quatre personnes prenez :

\footnotesize
\begin{longtable}{rrrp{16em}}
    150 & grammes & de & mie de pain rassis tamisée,                                                      \\
    150 & grammes & de & beurre,                                                                          \\
    100 & grammes & de & crème,                                                                           \\
     60 & grammes & d' & oignons,                                                                         \\
      5 & grammes & de & persil haché,                                                                    \\
        & 3 litres& de & moules,                                                                          \\
        &         &    & sel et poivre.                                                                   \\
\end{longtable}
\normalsize

Mettez les moules bien nettoyées dans une casserole et faites-les s'ouvrir
à feu vif. Jetez celles qui resteraient fermées, enlevez à chacune des autres
une valve de la coquille et tenez-les au chaud.

Ciselez fin les oignons avec un couteau frotté d'ail pour les parfumer
légèrement, faites-les blondir avec le beurre dans une sauteuse, ajoutez
ensuite la mie de pain et amenez-la à la même couleur que les oignons ; salez,
poivrez, saupoudrez de persil haché, puis mettez les moules et faites sauter
le tout pendant une douzaine de minutes ; mouillez ensuite avec la crème et
continuez à faire sauter jusqu'à ce que la crème soit absorbée.

Tout l'appareil doit entrer et s'attacher dans l'intérieur des coquilles.

Servez chaud, sans sauce.

Les moules sautées panées préparées ainsi doivent être moelleuses sous une
enveloppe légèrement résistante.

\section*{\centering Moules frites.}
\addcontentsline{toc}{section}{ Moules frites.}
\index{Moules frites}

Prenez de belles et grosses moules, nettoyez-les soigneusement, faites-les
s'ouvrir sur le feu, retirez-les des coquilles, roulez-les dans de la farine et
plongez- les, ainsi enrobées, dans un bain chaud d'huile d'olive.

Servez-les aussitôt soit telles quelles, soit accompagnées d'une sauce
hollandaise aromatisée au goût.

A Constantinople, où j'ai goûté pour la première fois des moules frites, on les
sert comme hors-d'œuvre et on les mange en buvant du mastic\footnote{Le mastic
est la résine du \textit{Pistachia lentiscus}. L'infusion alcoolique de cette
résine, très prisée en Orient, a un goût rappelant celui de l'anisette.}.

\sk

On peut préparer de même les huîtres.

\section*{\centering Ormeaux.}
\addcontentsline{toc}{section}{ Ormeaux.}
\index{Ormeaux}

Les ormeaux, ou haliotides, scientifiquement « Halotis tuberculata », sont des
mollusques gastéropodes qui vivent dans les fentes des rochers et ne peuvent
guère être capturés qu'au moment des grandes marées. En France, on les trouve
surtout sur les côtes de la Bretagne.

On peut faire sauter les ormeaux ou les préparer en ragoût.

Dans les deux cas, on commence par les enlever de leur coquille, on leur
arrache l'intestin, ce qui se fait aisément à l'aide d'un couteau, on les lave
à plusieurs reprises et on les brosse bien pour enlever toutes les parties
noires, enfin on les bat avec un battoir en bois pour les attendrir.

Lorsqu'on veut les faire sauter, on les fait cuire d'abord dans de l'eau salée
pendant une heure, on les essuie, on les laisse refroidir et enfin on les fait
sauter avec du beurre dans une poêle, pendant un quart d'heure, en les salant
légèrement.

Lorsqu'on veut les préparer en ragoût, on les fait cuire à petit feu pendant
une heure et demie à deux heures avec du beurre, des oignons, des carottes, du
sel, du poivre au goût, de l’eau ou mieux du bouillon de poisson.

Certaines personnes trouvent le ragoût d'ormeaux plus digestible que les
ormeaux sautés. Quoi qu'il en soit, préparés de l'une ou de l'autre façon, les
ormeaux constituent un plat qui, à Paris, n'est pas banal.

\section*{\centering Escargots en coquilles.}
\addcontentsline{toc}{section}{ Escargots en coquilles.}
\index{Escargots en coquilles}
\index{Beurre d'escargots}
\label{pg0304} \hypertarget{p0304}{}

L'escargot est un mollusque gastéropode de la famille des Hélicidés. L'espèce
la plus recherchée est représentée par les gros escargots blancs, « Helix
pomatia ».

Certaines personnes se refusent à considérer l'escargot comme comestible, parce
qu'elles trouvent sa chair dure, indigeste, et qu'elles n'éprouvent en le
mastiquant qu'une sensation analogue à celle que produirait un morceau de
caoutchouc. D'autres déclarent que c'est l’assaisonnement seul qui fait passer
le limaçon. Or, en réalité, l'escargot est délicat, et c'est précisément pour
cela qu'il convient d'apporter des soins à sa préparation ; autrement, il perd
facilement son goût. Cuit convenablement et aromatisé avec des condiments bien
choisis et bien dosés, il est savoureux, tendre et digestible. Un de mes amis,
à l'estomac délabré, ayant préparé des escargots en suivant à la lettre la
méthode que je vais exposer et les ayant trouvés exquis, se laissa aller
jusqu'à en manger quatre douzaines et sa surprise fut grande de ne ressentir
aucun inconvénient de son imprudence.

Pour six personnes prenez :

\medskip

\footnotesize
\begin{longtable}{rrrrp{16em}}   
  & 250 & grammes     & de & beurre fin,                                                                  \\
  & 100 & grammes     & de & vin blanc sec,                                                               \\
  &  70 & grammes     & de & gros sel,                                                                    \\
  &  20 & grammes     & de & persil haché,                                                                \\
  &  16 & grammes     & de & sel fin,                                                                     \\
  & \multicolumn{2}{r}{4 grammes 1/2} & d' & ail,                                                         \\
  &   2 & grammes     & de & poivre fraîchement moulu,                                                    \\
  &     &             & 72 & beaux escargots de Bourgogne et, de préférence, des escargots de vigne,      \\
  &     &             &    & court-bouillon très relevé, préparé avec eau, sel, poivre, carottes,
                             panais, ail, thym, laurier, persil.                                          \\
\end{longtable}
\normalsize

Sachez d'abord qu'il convient d'apprêter les escargots la veille du jour où
l'on veut les manger ; ils s'imprègnent ainsi plus intimement des condiments
qu'on leur adjoint ; ils sont plus et mieux parfumés que si on les apprête le
jour même.

Si vous opérez en hiver, les escargots ont jeûné et ils sont recouverts d'une
taie ; enlevez-la, faites ensuite dégorger les escargots dans le gros sel
pendant une heure, puis lavez-les à cinq ou six reprises dans de l'eau froide,
en changeant l'eau chaque fois. Si vous opérez en dehors de la saison
hivernale, commencez par les faire jeûner pendant deux jours, puis procédez
comme ci-dessus,

\index{Court-bouillon pour escargots}
Faites cuire le court-bouillon pendant une demi-heure, mettez dedans les
escargots, laissez-les cuire pendant une heure, puis refroidir un peu dans le
liquide. Sortez-les ensuite du court-bouillon, retirez les corps des coquilles,
lavez corps et coquilles dans de l'eau chaude salée et essuyez-les. Enlevez, si
cela est nécessaire, l'extrémité noire de l’animal qui représente l'intestin et
qui, en été, est souvent terreux et amer.

\label{pg0305} \hypertarget{p0305}{}
\index{Farce pour escargots}
Triturez dans un mortier le beurre, le persil, le sel fin, l'ail et le poivre
fraîchement moulu, de façon à avoir une farce parfaitement homogène.

Remettez les mollusques dans leurs coquilles et achevez le remplissage avec
cette farce.

Au moment du repas, placez les escargots, l'orifice de la coquille en l'air,
dans un plat allant au feu et au fond duquel vous aurez mis un peu d'eau pour
éviter que le vase et les coquilles brûlent, arrosez légèrement chaque escargot
avec un peu de vin blanc sec, au moyen d'une pipette, et chauffez sur un feu
doux jusqu'à ce que la farce commence à bouillir dans la coquille. À ce moment,
l'ail est cuit et tout est à point.

Servez aussitôt avec fourchettes spéciales et pinces \textit{ad hoc} pour
éviter à vos convives de se brûler et de se tacher les doigts.

Faites verser du chablis dans les verres et vous entendrez des louanges.

\section*{\centering Escargots en coquilles.}
\addcontentsline{toc}{section}{ Escargots en coquilles.}
\index{Escargots en coquilles (autre formule)}

\begin{center}
\sc\small(Autre formule)
\end{center}

\bigskip

Pour six personnes prenez :

\footnotesize
\begin{longtable}{rrrrp{16em}}   
  & 300 & grammes     & de & vin blanc sec,                                                               \\
  & 300 & grammes     & de & madère,                                                                      \\
  & 250 & grammes     & de & beurre,                                                                      \\
  & 100 & grammes     & de & fine champagne,                                                              \\
  &  70 & grammes     & de & sel gris,                                                                    \\
  &   70&  grammes    & de & persil haché,                                                                \\
  & \multicolumn{2}{r}{7 grammes 1/2} & du & mélange aromatique ci-après,                                 \\
  &     &             & 72 & beaux escargots de Bourgogne,                                                \\
  &     &             &  3 & belles échalotes,                                                            \\
  &     &             &  1 & gousse d'ail,                                                                \\
  &     &             &  1 & sachet contenant thym, laurier, ail, échalote,                               \\
  &     &             &  1 & pied de veau.                                                                \\
\end{longtable}
\normalsize

On prépare le mélange aromatique en pulvérisant très fin, tamisant et
mélangeant intimement ensemble :

\medskip

\footnotesize
\begin{longtable}{rrrrp{16em}}   
  & 100 & grammes    & de & sel blanc,                                                                    \\
  &   6 & grammes    & de & piment,                                                                       \\
  &   6 & grammes    & de & clous de girofle,                                                             \\
  &   3 & grammes    & de & feuilles de laurier,                                                          \\
  &   3 & grammes    & de & thym,                                                                         \\
  &   3 & grammes    & de & muscade,                                                                      \\
  &   3 & grammes    & de & basilic,                                                                      \\
  & \multicolumn{2}{r}{1 gramme 1/2} & de & poivre blanc,                                                 \\
  & \multicolumn{2}{r}{1 gramme 1/2} & de & coriandre,                                                    \\
  & \multicolumn{2}{r}{1 gramme 1/2} & de & cannelle de Ceylan.                                           \\
\end{longtable}
\normalsize

Mettez les escargots dans une marmite pleine d'eau, chauffez et, après le premier
bouillon, retirez les corps des coquilles. Laissez les mollusques en contact avec le
sel gris pendant une demi-heure, lavez-les ensuite à plusieurs reprises jusqu'à ce
que l’eau reste claire.

\index{Court-bouillon pour escargots}
Préparez un court-bouillon avec le vin blanc, le madère, la fine champagne, le
pied de veau et le sachet ; passez-le ; ajoutez les escargots blanchis et
faites mijoter au four. Au bout de quatre heures, retirez la marmite du feu et
laissez refroidir les escargols dans leur cuisson.

Lavez et séchez les coquilles.

\index{Farce pour escargots}
Mettez au fond des coquilles un peu du court-bouillon pris en gelée, puis les
mollusques et garnissez avec une farce obtenue en triturant dans un mortier le
beurre, le persil haché, le mélange aromatique, les échalotes et l'ail, de
façon à avoir une pâte homogène.

Enfin, au moment du repas, faites chauffer comme précédemment, mais sans
ajouter de vin blanc.

\section*{\centering Escargots piqués, sautés et panés.}
\addcontentsline{toc}{section}{ Escargots piqués, sautés et panés.}
\index{Escargots piqués, sautés et panés}

Voici enfin une formule employée dans le Midi.

Faites bouillir dans de l’eau des escargots, mis au point par le jeûne, d'abord
pendant une demi-heure à petit feu, puis pendant une autre demi-heure à gros
bouillons. Égouttez-les et jetez l'eau de cuisson.

Retirez les escargots de leurs coquilles, piquez chaque escargot avec un filet
de jambon gras, puis remettez-les dans les coquilles, enduites de quelques
gouttes d'huile d'olive.

Plongez les escargots ainsi apprêtés dans de l'eau bouillante assaisonnée,
aromatisée et relevée avec sel, poivre, fenouil, laurier, menthe, thym,
eau-de-vie et laissez-les cuire pendant trois heures.

Concentrez une partie de la cuisson, corsez-la avec de la glace de viande,
ajoutez-y des fines herbes, de la chair d'anchois hachée, des noix pilées, de
la mie de pain rassis tamisée, le tout lié avec des jaunes d'œufs.

Mettez cette sauce et les escargots dans une sauteuse et faites sauter de façon
à faire pénétrer toute la sauce dans les coquilles.

\sk

On peut également retirer les escargots des coquilles et les servir dans la
sauce.

\section*{\centering Escargots en brochettes.}
\addcontentsline{toc}{section}{ Escargots en brochettes.}
\index{Escargots en brochettes}
\index{Brochettes d'escargots}

Préparez et faites cuire les escargots comme il est dit
\hyperlink{p0304}{p. \pageref{pg0304}}. Laissez-les refroidir dans la cuisson ;
retirez les corps des coquilles et enlevez les intestins.

Enfilez les escargots cuits sur des brochettes, en intercalant entre eux de minces
lamelles de lard maigre.

Préparez un beurre d'escargots comme il est dit
\hyperlink{p0305}{p. \pageref{pg0305}} ; liquéfiez-le ; trempez dedans les
brochettes garnies de façon à les bien imprégner de beurre aromatisé ;
passez-les ensuite dans de la mie de pain rassis tamisée ; puis faites-les
griller, à feu dessus, en les arrosant avec le reste du beurre fondu.

Servez très chaud.
