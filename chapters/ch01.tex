L'homme ne s'est véritablement distingué des animaux que lorsqu'il a su se
servir du feu.

\index{Cuisines Antiques}

Les premiers représentants de notre espèce vivaient misérablement de fruits,
d'herbes et de racines : un peu plus tard, leurs descendants s'avisèrent de
goûter aux insectes, aux coquillages, puis à la viande, qu'ils mangeaient crue,
telle quelle, ou attendrie par des procédés primitifs, dont quelques-uns sont
parvenus jusqu'à nous\footnote{Les Huns, qui sont pour nous presque des
contemporains relativement au temps qui nous sépare des premiers hommes,
attendrissaient les viandes en s'asseyant dessus, par terre, où mieux encore
à cheval ; dans ce dernier cas ils les mettaient entre leur siège et le dos de
l'animal. Chez les Hongrois, qui descendent des Huns, le procédé ne s'est pas
perdu et pendant la révolution de 1848 les hussards de Kossuth attendrissaient
ainsi la viande avant de la faire cuire.}.

Lorsque, à la flamme du premier incendie de forêt allumé vraisemblablement par
la foudre, des animaux furent grillés, le premier homme qui mangea de la viande
cuite s'en régala sans doute, malgré le goût de brûlé qu'elle devait avoir. Dès
lors le feu fut divinisé et, tant que l'homme ne sut pas le produire
à volonté\footnote{On conçoit que les habitants de certaines contrées isolées aient pu
ignorer le feu pendant très longtemps. C'est ainsi que les indigènes des îles
Marianne, par exemple, le connurent seulement en 1521, lors de la découverte du
pays par Magellan.

Aujourd'hui, la plupart des sauvages savent faire du feu. J'ai vu les
Hottentots battre le briquet ; j'ai vu, dans l'intérieur de l'Amérique du Sud,
les Indiens allumer de la moelle d'arbre desséchée en la frottant entre leurs
mains contre une pièce de bois sec, procédé qui, par parenthèse, demande un
certain doigté pour être couronné de succès.

Néanmoins, il existe encore des peuplades très arriérées pour lesquelles la
production du feu constitue une véritable difficulté ; les indigènes de la
Terre de Feu sont dans ce cas. Il est vrai qu'au point de vue intellectuel il
n'y a guère d'être humain inférieur au Fuégien. J'en ai rencontré qui
tremblaient de froid par une température de -15° C., portant sur la tête des
peaux qu'ils allaient échanger à la côte contre des verroteries ou de
l'alcool, sans qu'il leur vint à l'esprit de se couvrir avec une partie de leur
charge.

Ces misérables représentants de notre espèce connaissent pourtant le feu, qui
leur est probablement tombé du ciel. Ils l'apprécient et s'en servent, mais ils
ne savent généralement pas le produire ; leur intelligence se borne
à l'entretenir, et c'est en cela qu'ils se montrent supérieurs aux singes dont
j'ai vu des bandes se chauffer à des feux abandonnés par des voyageurs et
qu'ils laissaient éteindre bien qu'ayant du bois coupé à côté d'eux. C'est
ainsi que dans tous leurs canots (et chaque famille de la côte a le sien qui
lui est indispensable pour la pêche) il existe un compartiment garni de terre
damée dans lequel on entretient du feu. Lorsque, par malheur, le feu s'éteint
faute de combustible, ou pour tout autre cause, les canotiers fuégiens en sont
réduits à attendre le passage d'un autre canot pour lui en emprunter. Enfin,
dans les villages de l'intérieur, il y a des feux entretenus en permanence.

Comme beaucoup de personnes, je croyais, avant d'être allé dans ce pays, que la
Terre de Feu devait son nom à des volcans visibles du large ; depuis, j'ai
changé d'avis et je suis persuadé qu'il le doit tout simplement aux feux des
canots qui, semblables à des feux follets jalonnant tout le pourtour de l'île,
ont dû frapper l'attention des premiers navigateurs qui ont doublé le cap
Horn.}, il resta le monopole des prêtres qui s'en constituèrent les gardiens.
C'est à eux que les fidèles apportaient les victimes offertes aux dieux en
holocauste : ils les faisaient cuire et… ils les mangeaient : ces prêtres
furent les premiers cuisiniers et les premiers gastronomes. Leurs descendants
spirituels ont de qui tenir.

\sk

Comme tous les arts, l'art culinaire, dont l'histoire se déroule parallèlement à
celle de l'humanité, a eu ses périodes d'éclat et ses périodes d’éclipse, et la
plupart des guerres et des grands événements politiques ont exercé une notable
influence sur son développement.

Primitif et simple chez les peuples pasteurs et chez les guerriers, luxueux et
fréquemment de mauvais goût chez les conquérants de toute sorte arrivés depuis
peu à la fortune, l'art culinaire ne devient délicat et raffiné que chez les
peuples de vieille civilisation.

Son développement général suit une courbe sinueuse. Il ne trouvera guère sa
formule définitive que lorsqu'on aura découvert toutes les lois qui le
régissent ; jusque-là, condamné à des hauts et à des bas, il sera, jouet du
hasard ou des circonstances, toujours menacé d'un retour en arrière.

\sk

\textit{Temps préhistoriques}. — La plupart des historiens se défendent de
remonter au déluge ; il me faudra cependant remonter encore plus haut. Le
déluge des Écritures ne date en effet que de {\ppp6\mmm} {\ppp000\mmm} ans
avant notre ère ; or, l'antiquité de l'homme est bien plus grande et, s'il est
difficile de la préciser, on peut du moins dire que vraisemblablement notre
espèce existe depuis un certain nombre de milliers de siècles. Quoi qu'il en
soit, l'homme quaternaire de la période paléolithique, le contemporain du
mammouth, vivait déjà de chasse et de pêche et connaissait l'usage du
feu\footnote{\label{pg0013} \hypertarget{p0013}{En ce temps-là}, les viandes étaient
généralement cuites sur des blocs de pierre chauffés au préalable. \protect Il
v a quelques années, j'ai vu employer encore ce procédé primitif de cuisson
dans les pampas de l'Amérique du Sud. Les premières choses que l'on offre dans
ces pays à un voyageur qui arrive dans une ferme sont une tasse de
\textit{maté} (infusion de feuilles d'un arbrisseau de la famille des
Illicinées) et un \textit{asado con cuerro}, c'est-à-dire du bœuf rôti dans sa
peau, qu'on prépare de la façon suivante : on commence par prendre un bœuf au
lasso, on lui coupe tête et pattes et l'on coud aussitôt les orifices pour
éviter une trop grande perte de sang ; puis on couche l'animal sur un lit de
blocs de pierre chauffés au rouge dans un feu de bois. Le rôtisseur retourne la
bête sur son lit de pierres de façon à faire cuire convenablement le filet ; le
reste est plus où moins sacrifié et beaucoup de parties sont carbonisées. Je ne
sais si c'est parce que j'ai mangé de ces asados après de longues chevauchées
et quand j'avais très faim, mais j'en ai conservé un souvenir délicieux. Je
n'oublierai jamais le flot de sang chaud qui jaillit lorsque, pour la première
fois, je plantai mon couteau dans un asado pour le découper, ni le goût exquis
du rôti ; et il me semble bien n'avoir jamais de ma vie mangé Chateaubriand
aussi juteux et aussi à point.}.
  
On en a acquis la certitude par des fragments d'ossements humains\footnote{Un
mot sur l’anthropophagie. L'homme, ayant pris goût à la viande, est devenu
cannibale le jour où il a manqué d'animaux.

« Les loups ne se mangent pas entre eux », dit-on ; si c'était vrai, ce serait
humiliant pour l'homme ; heureusement cela n'est pas ; lorsque les loups sont
affamés ils se dévorent parfaitement les uns les autres : j'ai pu m'en assurer
de visu lors d'un de mes voyages en Sibérie.

\textit{Homo hominis lupus.}

S'il est exact que « le cadavre d'un ennemi sent toujours bon », la première
victime du cannibalisme a dû être un ennemi du premier cannibale ; mais quelle
que soit la valeur de cette circonstance atténuante, c'est vraisemblablement
par gourmandise que l'homme est devenu anthropophage. Il l'est resté pour la
même raison et à d'autres points de vue : par superstition, dans la pensée
qu'il hériterait ainsi des qualités de la victime, par vengeance, et même
quelquefois par respect pour les morts, auxquels il croyait assurer de la sorte
la plus honorable des sépultures.

Le développement de la morale d'un côté et l'élevage du bétail de l'autre l'ont
corrigé de cette habitude ; mais il est prêt à y revenir dès que son existence
entre en jeu : l'histoire du radeau de la Méduse se renouvelle de temps en
temps.

Malgré le développement de la civilisation, il existe encore nombre de
peuplades anthropophages.

Des indigènes de la côte de Krou, en Afrique, qui avaient mangé des prisonniers
faits dans les combats que les tribus sauvages se livrent entre elles, le plus
souvent pour se procurer de la nourriture, m'ont dit en avoir conservé un
excellent souvenir, « Li semblé cochon » ; et il est effectivement assez
naturel que la chair des deux omnivores soit analogue. Un jour, l'un de ces
Kroumen, en veine de confidences, revenant sur le même sujet et précisant ses
impressions, me dit en riant d'un rire large qui découvrait toutes ses dents :
« Blanc pas bon ; mo pli content nég ». Il parait en effet que la chair du blanc
est plutôt fade. C'est aussi l'avis des tigres du Bengale qui ont une certaine
compétence en la matière, et qui, ayant à choisir entre un blanc et un hindou,
n'hésitent jamais à prendre le second. C'est l'une des raisons pour lesquelles
les Européens emmènent toujours des Hindous avec eux à la chasse au tigre.

Faut-il le dire ! J'ai moi-même compris l'anthropophagie dans un de mes voyages
où, après être resté vingt-quatre heures sans rien prendre, par suite de la
perte de toutes mes provisions au passage d'un saut de rivière, j'ai mangé pour
la première fois du singe. Cet animal, gros comme un enfant, qu'on faisait
rôtir à la broche, me donna l'impression que j'allais manger mon semblable.} et
d'os d'animaux calcinés trouvés dans les cavernes de cette époque et par les
\textit{kjoekkenmoeddings}\footnote{ Les kjoekkenmoeddings ont été trouvés pour
la première fois au bord de la mer, en Danemark, d'où leur nom danois qui
signifie résidus de cuisine, puis ailleurs, un peu partout, en Europe : en
France, en Belgique, en Allemagne, en Angleterre, au Portugal ; en Asie : au
Japon ; et dans les deux Amériques,

Les différents kjockkenmoeddings ne datent pas tous de la même époque. Mon
savant camarade et ami de Morgan, dans son magistral ouvrage : « les premières
civilisations  », rapporte les kjockkenmoeddings danois à l'état mésolithique,
intermédiaire entre l'état paléolithique et l'état néolithique ou celui de la
pierre polie ; tandis que les kjockkenmoeddings portugais par exemple seraient
franchement néolithiques. }, vestiges d'anciennes agglomérations humaines
renfermant des instruments de silex, des fragments de poterie grossière, du
charbon, des cendres, des os calcinés et brisés, des coquilles marines, des
arêtes de poissons, etc.

Les premiers animaux domestiques paraissent avoir été le chien, le renne, la
chèvre, le porc et la poule.

C'est également de la fin de la période de la pierre taillée (phase
néolithique) et de l'époque des cités lacustres que datent l'élevage du bétail,
la culture des céréales, le tissage, l'emploi du miel et l'usage du
sel\footnote{La découverte du sel est l’une des plus précieuses au point de vue
gastronomique. Les propriétés de ce condiment sont nombreuses : Il stimule
l'appétit, provoque la salivation, active la circulation du sang dans la
muqueuse stomacale, facilite la digestion en fournissant par sa décomposition
l'acide chlorhydrique nécessaire au suc gastrique ; et ainsi, d'une façon
générale, il contribue à l'équilibre de notre statique chimique ; de plus, il
favorise l'oxygénation du sang, ravive la couleur des globules sanguins, il est
antiseptique, etc., etc. Indispensable à la vie animale, il se trouve à l'état
naturel dans la plupart de nos aliments, mais en proportion insuffisante, au
moins pour notre organisation actuelle. Le manque de sel est l'une des
privations physiques les plus pénibles. Les personnes soumises au régime
déchloruré ne peuvent s'en faire qu'une faible idée, car les ressources
actuelles de la gastronomie permettent d'y suppléer, en partie au moins, grâce
à des artifices culinaires et à d'autres condiments. Pour s'en rendre compte
exactement, il faut, comme cela m'est arrivé, en avoir été privé pendant
longtemps sans rien avoir à lui substituer. C'est ainsi qu'à la suite de
l'accident du canot, dont j'ai dit un mot dans la note {\ppp2\mmm} de la page
{\ppp13\mmm}, j'ai vécu exclusivement pendant tout un mois, dans les forêts de
la Guyane, de gibier et de poisson rôtis au naturel, sans aucun assaisonnement.
Le manque de sel m'avait fait prendre la nourriture en tel dégoût que je ne
mangeais presque plus. Aussi, à ce régime, je m'étais anémié au point que je
fus littéralement étourdi et obligé de me coucher, au premier petit verre de
vin que je pris à la fin de mon jeûne forcé. Mon appétit revint dès le premier
repas assaisonné et, quelques jours après, j'avais repris mes forces.}.

\sk

\textit{Antiquité. — Égyptiens. —} Sous les Pharaons de la
IV\textsuperscript{e} dynastie, {\ppp3\mmm} {\ppp000\mmm} ans avant notre ère,
en Égypte, berceau de la civilisation, on cultivait déjà le froment, l'orge, le
millet, la vigne : on faisait du pain, du vin, de l'hydromel, de la bière.

Mille ans plus tard, sous la XVIII\textsuperscript{e} dynastie, celle des
grands Pharaons, époque où l'Égypte atteignit son apogée, la conquête de la
Syrie, de la Phénicie, du pays de Chanaan, de la Nubie et de l'Éthiopie
augmentèrent les ressources culinaires des Égyptiens. À cette époque, ils se
nourrissaient de viande de boucherie, de volaille, de gibier, de poissons,
d'huîtres, d'œufs\footnote{On savait déjà à cette époque faire éclore
artificiellement les œufs.}, de légumes farineux : lentilles, pois, fèves,
etc., de fruits variés : olives, figues, dattes, pommes, grenades, abricots,
amandes, etc. ; ils connaissaient l'ail, le persil ; ils adoraient l'oignon et
le mangeaient avec un respect religieux. Thèbes aux cent portes avait ses
éleveurs de volaille, ses gargotiers, ses confiseurs, ses pâtissiers, et l'on
voyait sur la table des Pharaons des truffes merveilleuses, de dimensions
inouïes, dont le poids atteignait {\ppp36\mmm} kilogrammes. Cette espèce semble
disparue de nos jours.

\sk

\textit{Hébreux}. — À l'origine, les Hébreux vivaient très frugalement. Plus
tard, ils menèrent la vie des patriarches : ils se nourrissaient de céréales :
blé, orge ; de quelques légumineuses : lentilles, fèves, et de la chair de
leurs troupeaux.

Lors de leur séjour en Égypte, leur alimentation s'augmenta de tous les
produits qu'ils trouvèrent dans le pays. Vingt siècles avant notre ère ils
connurent le beurre.

Puis, durant les quarante années pendant lesquelles ils errèrent dans le désert
({\ppp1\mmm} {\ppp420\mmm}-{\ppp1\mmm} {\ppp380\mmm} avant Jésus-Christ), ils eurent
beaucoup à souffrir, mais leur plus grande privation fut de n'avoir pu
emporter, en fuyant d'Égypte, ni graines d'oignon, ni levure. C'est de cette
époque que date l'usage du pain azyme que les Juifs pratiquants mangent encore
à Pâques, en souvenir à la fois de l'ancienne fête des azymes et du séjour des
Hébreux dans le désert.

Après leur entrée dans la terre de Chanaan, la nourriture des Hébreux se
modifia. Pour obéir aux lois de Moïse, ils rejetèrent de leur alimentation
beaucoup d'animaux dont ils faisaient jadis usage, et le beurre fit place à la
graisse et à l'huile.

Ce fut l'origine de la cuisine juive\footnote{Voir \hyperlink{p0045}{p. {\ppp\pageref{pg0045}\mmm}.}}

Cependant, sous le règne de Salomon au luxe proverbial, les Hébreux,
transgressant les préceptes de la religion, poussèrent la richesse et la
somptuosité de la table à un tel degré que le prophète Isaïe s'en émut et s'en
indigna.

\sk

\textit{Assyriens. Chaldéens}. — Les Assyriens s'occupaient peu de travaux
agricoles ; ils tiraient de leur sol quelques produits sauvages, mais ils
vivaient surtout aux dépens des peuples voisins.

Les Chaldéens, au contraire, s'occupaient beaucoup de culture, ainsi que le
prouve la flore si riche des jardins suspendus de Babylone.

En Assyrie, la classe pauvre se nourrissait de pain grossier, de légumes
sauvages, de tourteaux de poissons et de sauterelles. Dans la classe riche la
table était plus abondamment pourvue ; on y voyait des viandes, les mêmes que
celles d'Égypte, des poissons, dont les plus estimés étaient la carpe, le
barbeau, l'anguille et la murène ; des herbes potagères, des légumes variés :
haricots, fèves, lentilles, pois chiches, gombos, concombres, courges,
aubergines, etc., des fruits : raisin, dattes, ananas, mûres, amandes,
pistaches, noix, figues, grenades, citrons, oranges, poires, prunes, abricots,
etc.

Après chaque victoire, les Assyriens se vautraient dans l'orgie, mais ce fut
seulement au \textsc{vii}\textsuperscript{e} siècle avant notre ère, sous le
règne d'Assurbanipal, que Ninive et Babylone atteignirent l'apogée de leur
splendeur et que l'amour du luxe et des plaisirs matériels fut poussé au plus
haut point. Enfin, au \textsc{vi}\textsuperscript{e} siècle, sous le règne de
Balthasar, de crapuleuse mémoire, les Assyriens sombrèrent dans la plus
répugnante débauche.

\sk

\textit{Hindous}. — Dans l'Inde, d'où nous viennent le riz et des condiments de
haut goût, l'art culinaire était très développé dès la plus haute antiquité,
ainsi qu'en témoigne le « Rāmāyana  », ancienne épopée hindoue consacrée à la
glorification du héros Rāma.

\sk

\textit{Perses}. — Les Perses, d'origine aryenne comme les Hindous, étaient dès
les temps les plus reculés un peuple d'agriculteurs.

Au \textsc{ix}\textsuperscript{e} siècle avant Jésus-Christ, leur nourriture se
composait de céréales, de plantes légumineuses, d'herbes potagères et de fruits
tels que raisin, olives, citrons, cerises, prunes, pêches, noix ; les viandes
dont ils faisaient usage étaient celles du cerf, de l'âne sauvage, du bœuf, du
mouton, du porc, de l'autruche et de la tortue de mer.

A la suite des victoires qu'ils remportèrent sur Nabuchodonosor et sur
Balthasar, et après la conquête de l'empire des Mèdes, de l’Assyrie, de la
Chaldée, de la Lydie, de l'Inde, etc., les Perses, au contact des vaincus,
prirent des habitudes de luxe et d'intempérance. A la table des rois et des
riches on servait entiers de gros animaux rôtis, un chameau, un bœuf, un âne ;
chez les pauvres on se contentait de pièces plus petites.

Au \textsc{v}\textsuperscript{e} siècle, après les guerres médiques, pendant
lesquelles ils abandonnèrent souvent aux Grecs vainqueurs des richesses inouïes
en vaisselle et en mobilier d'or et d'argent, les Perses suivirent l'exemple
des Égyptiens, des Assyriens et des Hébreux ; ils se laissèrent aller à la
mollesse et à la volupté ; la débauche envahit tout : ce fut le commencement de
la décadence.

\sk

\textit{Grecs}. — En {\ppp1\mmm} {\ppp582\mmm} avant Jésus-Christ, Cécrops,
fondateur d'Athènes, y apporta d'Égypte les olives et l'art d'en faire de
l'huile. Vers la même époque, le phénicien Cadmus, cuisinier du roi de Sidon,
vint s'établir en Grèce et y introduisit les principes de l'art culinaire.

Au temps d'Homère, la nourriture du peuple était presque exclusivement composée
de bouillies de céréales, de poissons, de légumes et de fruits du pays.

La classe riche y adjoignait du bœuf, du mouton, de la chèvre et du porc : on
faisait rôtir ces animaux en les arrosant de graisse\footnote{ C'est encore
cette méthode de cuisson qu'on emploie de nos jours dans l'intérieur de la
péninsule balkanique. L'animal vidé, parfumé d'un bouquet de thym et embroché
sur une branche de coudrier, est rôti devant un grand feu de bois. Pendant la
cuisson, le Palikare rôtisseur trempe, de temps en temps, dans de la graisse
fondue assaisonnée et relevée par un peu de jus de citron, un petit drapeau de
toile et en caresse légèrement le rôti, d'autant plus fréquemment que la viande
se dore davantage. Ce procédé de graissage est très recommandable ; il a pour
effet d'éviter le ramollissement de la viande et il donne à la peau une
consistance croustillante très agréable.

Le méchoui arabe est préparé d'une façon analogue.} et on les mangeait en
buvant des vins aromatisés.

La gastronomie, marchant de pair avec la civilisation, commençait à se
développer, lorsque survint l'invasion dorienne qui arrêta tout progrès. Ce fut
alors que Lycurgue, qui voulait faire des Spartiates exclusivement un peuple de
soldats, essaya, en n'accordant à chacun d'eux qu'un minimum de jouissances
égal pour tous, de leur inculquer avec l'amour immodéré des armes le plus
profond mépris pour tous les arts : et, afin d'atrophier en eux le sens du
goût, il ne trouva rien de mieux que son fameux brouet\footnote{Le brouet noir
était un ragoût au vinaigre de viandes plus ou moins carbonisées, accompagnées
de plantes aromatiques et amères.}, qu'il eut la prétention de rendre sinon
gratuit, du moins obligatoire. C'en était trop ; les estomacs se révoltèrent, et
Lycurgue ne survécut pas à cet échec.

Après sa mort, les Lacédémoniens, trouvant que leur diète avait assez duré,
entreprirent les guerres de Messénie à seule fin de s'emparer des troupeaux et
des récoltes de leurs voisins. A la suite de ces conquêtes la cuisine grecque
s'améliora ; elle dut aussi ses progrès à l'influence d'Archestrate, l'immortel
auteur de la « Gastrologie », ouvrage qu'il écrivit après avoir parcouru le
monde connu, à la recherche de tout ce qu'il pouvait y avoir de meilleur
à boire et à manger. L'alimentation s'enrichit alors notablement par l'emploi
du seigle, du riz, de l’avoine : on vit apparaître sur les tables les
volatiles, les poissons, les légumes et les fruits les plus variés ; les
préparations culinaires furent relevées par de nombreux condiments ; on peut
citer, outre ceux déjà mentionnés : le poireau, la ciboule, la câpre, le
raifort, le thym, la sauge, l'origan, la coriandre, le fenouil, etc.

C'est en vain que le végétarien Pythagore prêcha, sous prétexte de
métempsychose, l'abstinence des viandes, la gastronomie, étouffée pendant
plusieurs siècles, triompha de la philosophie et reprit irrésistiblement son
essor.

Au \textsc{v}\textsuperscript{e} siècle avant notre ère survinrent les guerres
médiques et leur principal résultat fut l'introduction en Grèce des
connaissances culinaires déjà assez développées chez les Mèdes et chez les
Perses.

Au siècle de Périclès, les menus comprennent des potages, des poissons rôtis,
frits ou bouillis, ceux-ci accompagnés d'une sauce à l'huile (avec vinaigre,
jaunes d'œufs et fines herbes), origine de notre rémoulade : des viandes de
boucherie et de porc ; de la volaille, en particulier l'oie blanche engraissée
avec des figues fraîches, très recherchée pour sa chair, son foie, sa
graisse\footnote{Les Grecs ne connurent le beurre que très tard, par les
Scythes, qui en fabriquaient de toute antiquité.} et ses œufs ; du gibier rôti,
braisé ou en ragoût, le tout relevé par des sauces diverses, les unes douces
à base de miel, les autres piquantes à base de vinaigre : des légumes, des
desserts variés, des fruits de toute sorte, des fromages\footnote{Notamment
ceux de Sicile et ceux de la ville de Tromilée, en Arcadie.}, des pâtisseries,
parmi lesquelles certains gâteaux saupoudrés de sel, que l'on fabrique encore
de nos jours : enfin des entremets sucrés, dont quelques-uns étaient les
premières ébauches du pudding.

Comme boisson, les Grecs faisaient usage de lait, de vins indigènes ou
étrangers naturels, cuits ou fumés, d'hydromel, de cervoise\footnote{Boisson
analogue à la bière.}, de bière, de tisanes. Les vins indigènes les plus
appréciés étaient ceux de Corinthe, d'Acanthos et des Îles Ioniennes ; les vins
étrangers, ceux de Syracuse, de Falerne, de Smyrne, de Phénicie, d'Égypte et en
particulier ceux de la Thébaïde.

Le déjeuner du matin et celui de midi étaient sommaires ; en revanche le dîner
était copieux et se composait de plusieurs services. Dans ces repas du soir,
avant le dessert, des esclaves apportaient aux convives de l'eau, des parfums,
des couronnes de fleurs et de feuillage. C'est seulement alors que l’on
commençait à verser à boire. Puis, pendant le dessert, il y avait des
représentations mimiques, des lectures, des chants, des danses : des musiciens
se faisaient entendre, et les convives, excités par la boisson, déployaient
leur verve dans de joyeuses conversations et dans des saillies épicées de sel
attique.

Ces réunions étaient surtout un régal pour l'esprit ; mais la préparation des
mets laissait à désirer, et il faut bien dire que les Grecs, à cette époque qui
marque l'apogée de leur civilisation, n'avaient pas poussé la
gastronomie\footnote{On trouve dans Aristophane l'indication de nombreuses
substances alimentaires employées de son temps. Je crois intéressant d'en faire
l'énumération. Comme viandes de boucherie : l'agneau, le baudet, la brebis, le
cochon de lait et la truie ; comme issues et charcuterie : l'andouille, le
boudin, la saucisse et les tripes ; comme volaille : la poule, le canard, l'oie
et le pigeon ; comme gibiers : le lièvre, l'alouette, le bec-figue, la caille,
la grive, la perdrix, le faisan, la poule d'eau, la sarcelle et l'autruche ;
comme poissons : l'anguille, la loche, le maquereau, le mulet, la plie, la
raie, le rouget, la sardine, le thon et le turbot ; comme crustacés et
mollusques : le crabe, la crevette, l'écrevisse et l'huître ; comme insectes :
les sauterelles ; comme reptiles : la tortue ; comme grains : le blé, l'orge ;
comme légumes secs : la fève, les haricots, les lentilles, les pois et les pois
chiches ; comme légumes frais et condiments : l'ail, l'anis, la bette, la
betterave, le cardon, la ciboule, la citrouille, le concombre, la coriandre, le
cresson, la faîne, l'oignon, l'olive, le persil, le poireau, le raifort, la
rave, le sésame, le thym ; comme fruits : les figues, les grenades, l'orange,
la poire, la pomme et le raisin.

Ajoutons qu'Aristote mentionne dans son « Éthique » vingt variétés de coulis.}
aussi loin que les autres arts.

Cependant il se manifestait depuis quelque temps déjà des symptômes de
décadence. Les Grecs avaient pris chez les Libyens la déplorable habitude de
manger couchés, première erreur gastronomique ; ils tombèrent ensuite dans une
seconde plus grave, celle de vivre pour manger. En vain Hippocrate, au nom de
l'hygiène, et Socrate, au nom de la morale, s'efforçaient-ils de lutter contre
l'envahissement de la goinfrerie ; la Grèce déclinait de plus en plus et, un
siècle après Périclès, en {\ppp146\mmm} avant J.-C., elle fut asservie par
Rome.

\sk

\textit{Romains}. — Les Romains des premiers âges, comme tous les peuples
primitifs, menaient une existence misérable. Numa Pompilius emprunta aux Sabins
qui le tenaient eux-mêmes des Perses, par l'intermédiaire des Pélages, le culte
de Vesta déesse du feu ; de cette époque (\textsc{vii}\textsuperscript{e}
siècle av. J.-C.) datent les commencements de l'art culinaire à Rome.

Au début, les Romains ne mangeaient guère de viande que les jours de fête, et
leur nourriture ordinaire consistait surtout en végétaux, ail, oignons, légumes
à cosses, navets, panais, poireaux, etc. ; et la bouillie
d'épeautre\footnote{Variété de froment qui pousse dans les terrains arides.}
faisait office de pain.

A l’origine, ils castraient les taureaux uniquement pour les dompter avec plus
de facilité. Mais bientôt ils s'aperçurent que cette opération améliorait
singulièrement la qualité de la viande, et cela leur donna par la suite l'idée
de chaponner les coqs. Telle fut chez eux l'origine de l'élevage.

Ils connurent la vigne par les Phocéens, qui la leur apportèrent de Perse.

Les guerres du Samnium mirent les Romains en contact avec les Grecs, alliés des
Samnites ; c'est à leur école qu'ils apprirent les principes de l'art culinaire.
Toutefois ce ne fut qu'à l'époque des guerres puniques\footnote{ Flaubert, dans
le premier chapitre de « Salammbô  », rapporte des détails précis sur cette
époque. Je crois intéressant de citer \textit{in extenso} la description qu'il
y a faite du festin donné dans les jardins d'Hamilcar pour célébrer
l'anniversaire de la bataille d'Eryx.

« Les cuisines d'Hamilcar n'étant pas suffisantes, le conseil leur avait envoyé
des esclaves, de la vaisselle, des lits, et l'on voyait au milieu du jardin,
comme sur un champ de bataille quand on brûle les morts, des grande feux clairs
où rôtissaient les bœufs. Les pains saupoudrés d'anis alternaient avec les gros
fromages plus lourds que des disques et les cratères pleins de vin, et les
canthares pleines d'eau auprès des corbeilles en filigrane d'or qui contenaient
des fleurs. La joie de pouvoir enfin se gorger à l'aise dilatait tous les
yeux ; çà et là les chansons commençaient.

D'abord on leur servit des oiseaux à la sauce verte, dans des assiettes
d'argile rouge rehaussée de dessins noirs, puis toutes les espèces de
coquillages que l'on ramasse sur les côtes puniques, des bouillies de froment,
de fève et d'orge et des escargots au cumin, sur des plats d'ambre jaune.

Ensuite les tables furent couvertes de viandes : antilopes avec leurs cornes,
paons avec leurs plumes, moutons entiers cuits au vin doux, gigots de chamelles
et de buffles, hérissons au garum, cigales frites et loirs confits. Dans des
gamelles en bois de Tamrapanni flottaient, au milieu du safran, de grands
morceaux de graisse. Tout débordait de saumure, de truffes et d'asa fœtida. Les
pyramides de fruits s'éboulaient sur les gâteaux de miel et l'on n'avait pas
oublié quelques-uns de ces petits chiens à gros ventre et à soies roses que
l'on engraissait avec du marc d'olives, mets carthaginois, en abomination aux
autres peuples. »} que la gastronomie fit vraiment à Rome des progrès sérieux.

La première guerre leur donna la Sicile, dont les cuisiniers\footnote{Les
cuisiniers siciliens, dont les premiers étaient d'origine grecque, ont joué un
rôle considérable dans l'histoire de Rome. C'était eux qu'Annibal, imprudent,
emmenait avec lui en campagne. Pendant la deuxième guerre punique, ils le
plongèrent à Capoue dans de telles délices qu'il s'y attarda avec son armée et
finit par se faire battre, tandis qu'en opérant rapidement il aurait pu,
dit-on, s'emparer de Rome qui était alors à deux doigts de sa perte.} (quantum
mutati) étaient alors les premiers du monde.

La deuxième leur valut les îles Baléares, où ils trouvèrent le lapin qu'ils
s'empressèrent d'acclimater dans leur pays.

On sait comment la troisième, celle qui se termina par la destruction de
Carthage, fut provoquée par Caton l'Ancien. Pendant la délibération, quelques
sénateurs hésitaient. Caton tira de sa toge des figues d'Afrique et s'écria :
« C'est la conquête du pays producteur de ces fruits que je demande. »
L'argument fut irrésistible.

C'est de celle même époque que date l'introduction de la grenade.

Plus tard, l'annexion de la Grèce apporta aux Romains de nouvelles conquêtes
gastronomiques. Ils trouvèrent dans ce pays le faisan que les Argonautes
avaient rapporté de leur expédition sur les bords du Phase, la bécasse et
d'autres gibiers dont nous avons déjà parlé, des légumes variés parmi lesquels
nous citerons, outre ceux mentionnés plus haut, l'asperge, la carotte, le
cerfeuil, les champignons, la laitue, le pissenlit, la truffe, originaire de
Libye, et des fruits tels que la noix et la pêche qui venaient de Perse.

D'Asie, ils avaient rapporté la cerise\footnote{La cerise fut importée par
Lucullus, de Gérasonte, ville de Pont.}, l'abricot\footnote{L’abricot fut
importé d'Arménie.}, le concombre, le citron\footnote{Le citron ne paraît avoir
été introduit à Rome que postérieurement à Pline.}, le paon\footnote{Le paon
fut importé de Samos par Hortensius.} ; et les Parthes leur avaient enseigné la
fabrication du pain mollet. D'Afrique, leur vint le melon à grosses côtes
qu'ils se mirent à cultiver près du village de Cantalupe, d'où le nom de
cantaloup.

Ils pêchaient dans leurs rivières et dans leurs deux mers des ablettes, des
anchois, des bars, des barbillons, des daurades, des esturgeons, des harengs,
des mulets, des murènes, des turbots, des grenouilles, des moules, des huîtres,
des oursins, des palourdes. Le gibier ne leur manquait pas non plus. Ils élevaient
le gibier à poil dans des parcs, le gibier à plumes dans des volières, dont la
première fut construite par Marcus Lœnius Strato.

C'est Sergius Orata qui eut le premier l'idée de parquer les huîtres, et les
parcs du lac Luerin fournissaient des mollusques à point : Fulvius Lippinus
avait imaginé d'engraisser des escargots avec une bouillie faite de farine et
de moût de vin et il obtenait ainsi des produits hors ligne ; le consul Scipio
Metellus, promoteur du gavage des oies dans l'obscurité, avait créé le foie
gras. Les Romains, très versés dans l'art de la charcuterie, obtenaient des
truies à chair très parfumée en les nourrissant de figues\footnote{On conçoit
parfaitement que les animaux ainsi nourris devaient être exquis. En Espagne,
dans la province de Séville, on donne à manger aux porcs des glands et des
olives, et l'on obtient ainsi une chair excellente.} ; ils préparaient des
jambons, des saucisses, etc. Ils accommodaient en salade le cresson, la laitue,
l'oseille, la manne, la rue. Ils faisaient certains fromages\footnote{Dans ses
ouvrages. Martial signale parmi les fromages connus de son temps ceux de Luna,
en Étrurie, ceux de Vélabre, qui étaient durcis au feu, et ceux de Toulouse.}
dans la composition desquels entrait souvent du thym en poudre. Ils
confectionnaient couramment de la crème fouettée, des oublies, des tourtes, des
croûtes, des puddings, des gâteaux au fromage, des sorbets, etc. Ils avaient
aussi une très grande variété de fruits : pommes, poires, prunes, châtaignes,
coings\footnote{Importé de Syrie du temps de Galien.}, raisin,
pistache\footnote{Importée de Syrie par Vitellius.}, etc.

En fait de vins\footnote{ Martial cite les vins de Campanie, de Cécube, de
Céré, les vins miellés de Crète, les vins de Falerne, de Fondi, de Mammerte, de
Marseille, des Mares, de Nomentanum, de la Sabine, de Sitie, de Sigui, de
Sorrente, de Spolète, de Tarragone et de Toscane.

Les Romains employaient volontiers comme boisson, pure ou coupée de vin, de
l'eau glacée qu'ils avaient fait bouillir au préalable, par une sorte de
prescience très curieuse de l’asepsie, fondée peut-être sur le vieil adage :
« le feu purifie tout ».}, ils avaient les vins connus des Grecs, les leurs et
ceux du Rhin,

Leurs ressources gastronomiques considérables et leurs connaissances culinaires
relativement\footnote{ Comme les Grecs, les Romains ne connurent le beurre que très
tard ; ils le trouvèrent chez les Germains. Leur cuisine était préparée
à l'huile et à la graisse, comme cela se fait encore de nos jours dans le Midi,

Parmi les plats les plus renommés de l’époque, il faut citer les tétines de
truie à la saumure de thon, les pâtés de volaille, les grives aux asperges, le
porc à la Troyenne farci de bec-figues et d'huîtres, le tout arrosé de vin et
de jus aromatisé.} étendues leur avaient permis de créer une cuisine plus
raffinée que la cuisine grecque ; malheureusement ils finirent par devenir trop
carnivores et ils eurent le tort de méconnaître la valeur des
légumes\footnote{Les Romains de la classe aisée n'estimaient guère que les
légumes plus ou moins rares, les grosses asperges, les choux nains, les cœurs
de laitue, qu'ils mangeaient trempés dans de la crème ou dans du garum de
maquereau, et les champignons.}, qu'ils trouvaient fades, alors que les Grecs,
plus délicats, avaient su les apprécier. Les repas romains, très somptueux,
étaient égayés par des représentations théâtrales et musicales\footnote{Martial
écrivait à ce sujet : « Vous me demandez quel est le meilleur festin ? C'est
celui où il n'y à pas de joueur de flûte ». Ceux qui, de nos jours, inscrivent
sur leurs programmes comme \textit{great attraction} : « Pas de Tsiganes »,
appartiennent, sans s'en douter, à l'école de Martial.}, mais, contrairement
à ce qui se passait en Grèce, les convives restaient spectateurs passifs de ces
divertissements.

En résumé, les Romains jouissaient d’un très grand bien-être matériel, mais la
facilité avec laquelle ils s'étaient enrichis par le pillage des contrées
conquises, les conduisit et devait fatalement les conduire à des excès de toute
sorte : aussi ils marchèrent à grands pas vers la décadence et ils finirent par
ne plus vivre que pour satisfaire leur gloutonnerie. L'introduction de l'usage
immonde du vomitorium marque la fin de la gastronomie romaine. Un véritable
vent de folie souffla alors sur Rome. Ce fut à qui dépenserait le plus. On
rivalisait de plats rares et le plus souvent insensés\footnote{Dans son
« Satyricon », Pétrone parle de plusieurs plats de cette catégorie en décrivant
une orgie chez Trymalcion, qui paraît être une critique de la fameuse orgie que
Néron, ce Nabuchodonosor romain, donna sur l'étang d'Agrippa. Citons entre
autres une laie en surprise, invraisemblable pièce montée, composée d'une
énorme laie en peau, portant à chacune de ses défenses une corbeille tissue de
petites branches de palmier et contenant l'une des dattes de Syrie, l'autre des
dattes de la Thébaïde, et autour de laquelle étaient groupés, en nombre égal
à celui des convives, des marcassins en pâte cuite. Lorsque l'écuyer tranchant
ouvrit la laie, il s'en échappa un vol de grives que les esclaves attrapèrent
à la glu. Chacun des convives reçut un marcassin en pâte, des grives en vie et
des dattes. Un autre plat en surprise était une truie à ventre rebondi, farcie
de boudins et de saucisses, que le cuisinier apportait en s'excusant d'avoir
oublié de la vider. Signalons encore une faisane en plumes, couvant des œufs en
pâte farcis d'une purée de jaunes d'œufs durs masquant des bec-figues rôtis.
Les surprises modernes sont incontestablement de meilleur goût à tous les
points de vue, et il me suffira de citer à cet égard les ortolans en
sarcophages, qui se présentent sous la forme de truffes farcies chacune d'un
ortolan, garni lui-même de foie gras, et qui laissent loin derrière ceux les
œufs de faisane dont je viens de parler.}. On servait des talons de chameau,
des trompes d'éléphants, des têtes de perroquets, des ragoûts de foies de
rossignols et de cervelles de paons, et des pâtés de langues d'oiseaux savants
qui atteignaient des prix équivalant à {\ppp20\mmm} {\ppp000\mmm} francs de
notre monnaie. Lucullus\footnote{Lucullus, qui s'était scandaleusement enrichi
dans ses campagnes en Cappadoce contre Mithridate, jetait littéralement
l’argent par les fenêtres. Il payait {\ppp20\mmm} {\ppp000\mmm} francs par an
ses écuyers tranchants. Dans la seule baie de Naples, il possédait trois
châteaux autour desquels il y avait des parcs et des volières regorgeant de
gibier et de volaille, des viviers pour poissons d'eau douce et d'autres pour
poissons de mer. Pour alimenter ses viviers d'eau fraîche, il avait fait percer
toute une chaîne de collines, ce qui était pour l’époque un travail colossal.}
dépensait {\ppp50\mmm} {\ppp000\mmm} francs en un seul repas. Sous le règne de
Tibère, Marcus Apicius\footnote{Ce fut lui qui imagina de noyer les rougets
dans le garum (apéritif aphrodisiaque, d'origine grecque, à base de saumure de
poisson) et de les préparer ensuite avec une sauce dont le fond était leur
propre foie. Le principe de cette sauce a survécu. De nos jours on sert les
rougets grillés masqués d'une purée faite avec leurs foies et du beurre.},
possesseur d'une fortune représentant cinquante millions de francs, trouvait le
moyen d'en gaspiller les trois quarts en orgies et finissait par le suicide,
estimant qu'il n'avait plus de quoi vivre décemment. Vitellius, qui ne
s'interrompait de manger que pour vomir, dévorait quatre-vingts millions en
huit mois de règne. Héliogabale, à peine âgé de {\ppp18\mmm} ans, dépassait en
débauche et en prodigalités les Césars ses prédécesseurs. Désireux de passer
à la postérité sous son vrai jour, il avait un historiographe attaché à sa
personne pour décrire ses orgies. Il donnait des festins comprenant jusqu'à
vingt-deux services ; et, pour augmenter la dépense, il faisait mélanger aux
plats servis à ses invités des perles, de l'or et des pierres précieuses.
Passons !

\sk

\textit{Invasion des Barbares}. — L'odeur des saturnales romaines se répandit
au loin : elle finit par attirer les Barbares, dont les invasions durèrent près
de trois siècles et plongèrent dans une nuit profonde la civilisation du monde
antique qui se mourait d'indigestion.

Foulées sous les pieds des Barbares, les truffes s'enfoncèrent dans le sol et
disparurent ; elles ne reparaîtront qu'à l’époque de la Renaissance.

Les savants, les penseurs, les artistes, traqués par les envahisseurs, se réfugièrent
dans les couvents alors regardés comme inviolables.

\sk

\textit{Christianisme}. — Sous l'influence du Christianisme, l’art culinaire se
simplifia dans les couvents et, après une phase de rigoureux ascétisme, il
redevint, ce qu'il n'aurait jamais dû cesser d'être, l'art de rendre les
aliments appétissants et digestibles. Le développement du jardinage, favorisé
par la vie monastique, introduisit dans la nourriture des légumes et des
fruits. L'institution très hygiénique des jours maigres enrichit de nouveaux
plats le répertoire de la cuisine païenne. Les jeûnes et les abstinences
reposèrent les estomacs et leur firent apprécier la cuisine simple.

Dans tous les monastères, les religieux devaient, par principe égalitaire,
vaquer à tour de rôle à la confection des mets ; ce fut l'origine de nombreux
progrès gastronomiques, car il se trouvait parmi les moines des gens de goût
que l'émulation conduisit tout naturellement à perfectionner les formules
anciennes. La cuisine des couvents fut toujours bonne, grâce à l'excellence des
matières premières employées et aux soins apportés à leur préparation, et il
est indiscutable que c'est à eux que nous devons la conservation et le
développement des principes rationnels de la science gastronomique.

\sk

\index{Cuisine française depuis les origines jusqu'à nos jours}

\textit{Gaule}. — Avant d'aller plus loin, retournons un peu en arrière et
disons un mot des Gaulois.

Dans la période qui précéda la conquête romaine, les Gaulois vivaient de
pain\footnote{Ce sont les Phéniciens qui introduisirent le blé à Marseille.} de
laitage, d'œufs, de légumes. d'oignons cuits sous la cendre, de viandes de
boucherie et de porc, ainsi que de gibier, de poissons, assaisonnés de sel, de
safran, de miel, de vinaigre et de brou de noix. Ils buvaient de l'hydromel, de
la bière et du vin\footnote{La vigne fut importée à Marseille, avant la
conquête des Gaules, par un Toscan banni de sa patrie.}

Les différentes invasions modifièrent beaucoup leur cuisine. Après la conquête
romaine le nombre des services s'accrut, le luxe se manifesta par la variété
des mets et des assaisonnements relevés, mais l'invasion des Huns, qui
mangeaient la viande à moitié crue et aux trois quarts corrompue, arrêta cet
élan et il faut arriver à l'époque des Francs pour retrouver une nourriture
quelque peu soignée.

\sk

\textit{Gaule franque}. — A l’époque des Mérovingiens, de nouvelles créations
culinaires apparurent. Un certain nombre de mets comprenaient des sauces au
bouillon, au vin, aux plantes aromatiques : on préparait au jus de viande
beaucoup de légumes, tels que pois, fèves, lentilles, haricots, choux rouges et
choux verts. Les fromages étaient souvent mouchetés de graines de fenouil : la
pêche était le fruit le plus apprécié et, comme entremets, la vogue était aux
confitures de roses et de violettes. On trouve dans les œuvres de Grégoire de
Tours la mention d'un potage à base de volaille qui lui avait été servi à la
table de Chilpéric I\textsuperscript{er}.

Comme chez les Romains, les repas étaient égayés par des concerts.

Sous les Carlovingiens la cuisine fit de nouveaux progrès ; la laitue, le
cresson de fontaine, le cresson alénois, la chicorée, la carotte, le navet, le
cerfeuil augmentèrent ses ressources. Dans les couvents, les jours de fête, on
vit pour la première fois des pâtés aux œufs, des pâtés de poisson, des tourtes
de viande, avant-coureurs des flans et des vol-au-vent.

La vaisselle, déjà luxueuse sous les Mérovingiens, devint de plus en plus riche :
et l'on mit tant de recherche dans le luxe de la table que le Concile de Francfort
s'en émut et édicta des peines sévères contre les ecclésiastiques qui s'écartaient
des règles de la sobriété et de la simplicité.

C'est de cette époque que datent nos premières lois somptuaires.

\sk

\textit{France. Moyen âge}. — Après la mort de Charlemagne, les guerres
civiles, les invasions des Normands, le brigandage plongèrent la France dans la
misère. Pendant les \textsc{ix}\textsuperscript{e},
\textsc{x}\textsuperscript{e}, \textsc{xi}\textsuperscript{e} siècles
d'horribles famines et d'effroyables épidémies sévirent dans tout le pays : on
vit renaître l'anthropophagie : les enfants disparaissaient alors comme par
enchantement !

Puis le luxe reparut et régna en maître jusqu'au moment où la première
croisade, sonnant le boute-selle, arracha les chevaliers aux plaisirs de la
table.

Au \textsc{xii}\textsuperscript{e} et au \textsc{xiii}\textsuperscript{e}
siècles, durant lesquels se succédèrent huit croisades consécutives, la
gastronomie fut plus ou moins négligée. Cependant ces guerres ne restèrent pas
sans influence sur son développement ; en mettant l'Europe occidentale en
contact avec l'Orient, elles nous valurent l'importation du sarrasin, du sucre,
de l'anis, du cumin, de la cannelle, du poivre, du gingembre, de la noix
muscade, du safran, des échalotes d'Ascalon, des prunes de Syrie, etc.

Au \textsc{xiii}\textsuperscript{e} siècle, Saint-Louis avait déjà deux
sauciers, et le goût des plats relevés se propageait à tel point que, dès la
fin du \textsc{xiv}\textsuperscript{e} siècle, les marchands de vinaigre, de
moutarde et de sauces préparées aux divers condiments furent constitués en
corps de métier.

C'est vers cette époque que Gaston Phœbus, seigneur de Béarn, créa le lièvre au
chaudron, que le haricot\footnote{De l’ancien mot « harigoter », qui signifiait
couper en morceaux} ou ragoût de mouton, imité d'un plat arabe, fit son
apparition dans les menus et que l'artichaut fut importé de Venise.

Un peu plus tard ({\ppp1\mmm} {\ppp421\mmm}), le citron, originaire de Chine,
puis le riz, originaire de l'Inde, furent introduits en France.

On cultivait le pommier, le poirier, le prunier, le noyer, le noisetier, le
châtaigner, le néflier, le cerisier ; on fabriquait des oublies, des échaudés,
des darioles, des talmouses, pâtisseries au fromage dorées au jaune d'œuf et
saupoudrées de sucre, des beignets, des crêpes, des tartes, des flans : les
pains d'épice de Paris et de Reims étaient très appréciés.

Néanmoins, la cuisine au moyen âge était encore relativement très médiocre. On
donnait des festins dans lesquels on prodiguait les plats, mais, en réalité, on
s'attachait beaucoup plus au luxe déployé dans les accessoires qu'au choix des
mets et à leur préparation ; on abusait des épices. Notons pourtant les vins
français, déjà renommés.

\sk

\textit{Temps modernes}. — Il faut arriver au commencement de la Renaissance
pour constater des progrès réels dans l’art culinaire.

Sous Charles VII, le potage au riz était très en faveur. Taillevent, cuisinier
du roi, auteur du « Viandier », dont la première édition remonte
à {\ppp1\mmm} {\ppp490\mmm}, créait plusieurs soupes : à l'oignon, à la
moutarde, aux fèves, au poisson ; différentes manières de préparer le gibier,
des sauces variées et la galimafrée, cette aïeule du poulet Marengo. De son
côté, Agnès Sorel imaginait le salmis de bécasses.

Sous Charles VIII, le beurre de Vanves était recherché, les fromages de
Champagne, de Brie, de la Grande-Chartreuse passaient pour excellents, et l’on
commençait à importer les fromages italiens. C'est sous ce roi que le melon fut
introduit en France.

Du temps de Rabelais, il existait déjà une quinzaine de sauces françaises,
parmi lesquelles des sauces blanches, des sauces vertes, des sauces au beurre
noir, des sauces à la moutarde, la sauce Robert, etc. : on connaissait soixante
manières d'accommoder les œufs, et Gauthier d'Andernach, médecin de François
I\textsuperscript{er}, inventait en moins de dix ans sept coulis, neuf ragoûts,
trente et une sauces et vingt et un potages. La bisque, les potages aux pâtes
d'Italie, aux oignons farcis, au jus de citron apparaissent alors.

A la table du roi galant, les plats délicats ne manquaient pas : l’un des plus
renommés de l'époque se composait de foies de lottes étuvés au vin d'Espagne.

Le branle était donné.

Sous Henri II, on servait des murènes en tronçons, des perdrix à la tonnelette,
des soleils de blanc de chapon et des oriflammes de gelée, qui font pressentir
l'anguille à la tartare, la chartreuse de perdrix, le suprême de volaille et
les aspics. Les épinards, originaires d'Asie, et importés d'abord en Hollande,
furent à cette époque acclimatés en France : servis en ragoût, ils étaient très
en vogue pendant le carême,

Sous Charles IX, furent importés d'Amérique le maïs et le dindon, le premier
par les Portugais, le second par les Jésuites.

La cuisine française, qui dès ce moment avait incontestablement conquis le
premier rang, reçut alors un nouvel élan grâce à l'arrivée en France, à la
suite de Marie de Médicis, de cuisiniers italiens, possédant certaines
traditions culinaires romaines, qui répandirent le goût des desserts et des
entremets glacés. À la même époque, on commença à distiller le moût de raisin,
le cidre, le poiré et on fabriqua les premières liqueurs spiritueuses.

Sous le règne de Louis XIII, Richelieu, pénétré de l'importance des services
qu'un maitre-queux habile peut rendre à un diplomate avisé, encourageait les
artistes culinaires. Du reste, les grands de l'époque ne dédaignaient pas de
s'occuper de cuisine ; la marquise de Sablé préparait de ses doigts roses des
potages, des ragoûts, des entremets de sa composition, et le roi lui-même avait
la réputation d'être un cuisinier de valeur.

C'est sous Louis XIII que fut créée la croquante ; c'est à la même époque que
Claude Gelée, dit le Lorrain, qui, avant d'être le Raphaël du paysage, avait
été un pâtissier de génie, inventa le feuilletage, et que le topinambour fut
introduit en France.

Louis XIV était un gourmand beaucoup plus vorace que délicat ; son appétit
était invraisemblable ; les repas à la Cour comprenaient huit services, se
composant chacun de vingt à trente plats. Ce fut à l'occasion des fêtes de ses
fiançailles avec l'infante Marie-Thérèse qu'apparut pour la première fois la
sauce connue sous le nom d’espagnole. Sous son règne, les petits pois entrèrent
dans l'alimentation, le potage Saint-Germain fut créé ; le café, le chocolat,
le thé furent importés en France, le premier par les Vénitiens, le second par
les Espagnols, le troisième par les Siamois. Ce fut à ce moment que le
cilicien\footnote{Petite confusion sur l'origine du nom Café
Procope. Renseignement pris, c'est effectivement à un Arménien (donc un
cilicien, voir \textit{Royaumme de Cilicie} à ce sujet) du nom de Grégoire, que
revient l'honneur d'avoir fondé ce café. Le plus ancien café de Paris, fut
racheté en 1686 par le Sicilien Francesco Procopio dei Coltelli qui le renomma
« Café Procope » et le redécora à grand frais, ndlr.} Procope fonda à Paris le
premier café-glacier\footnote{Ce café était situé en face de la
Comédie-Française, qui se trouvait alors rue Mazarine ; Procope le transporta
ensuite rue des Fossés-Saint-Germain, aujourd'hui rue de l’Ancienne-Comédie, ou
une enseigne porte encore son nom.} et que le vin de Champagne commença à être
apprécié.

Parmi les mets en vogue de ce temps-là, et dans l’assaisonnement desquels on
abusait souvent de la muscade\footnote{« Aimez-vous la muscade, on en a mis
partout », Boileau, satire III.}, on peut citer la fricassée de poulet et de
pigeon, la galimafrée déjà mentionnée, les rissoles, les côtelettes en
papillotes, dues à la collaboration de Mme de Maintenon et de son frère le
baron d'Aubigné, enfin plusieurs entremets dédiés au cardinal Mazarin.

Amateur de jardins, le grand roi favorisa beaucoup la culture des vergers et
des potagers, et l’on peut dire que grâce à son influence nombre de fruits et
de légumes furent améliorés, notamment la pêche de Montreuil, sélectionnée par
Girardot.

Pourtant, la grande cuisine française ne date vraiment que du régent.

Petit-fils de Louis XIII et ayant hérité des goûts gastronomiques de son
grand-père, le régent fut en cette matière un véritable initiateur. Il institua
les petits soupers, dans lesquels il préparait lui-même, aidé de ses roués, des
plats raffinés ; sa batterie de cuisine était en argent et le contenu de ses
casseroles valait le contenant. Ses matelotes étaient renommées.

C'est à partir de ce moment qu'on se mit à extraire des viandes des sucs légers
et nourrissants pour en faire la base des cuissons et des sauces, qu'on
s'appliqua à manier adroitement les assaisonnements, à les marier
harmonieusement et à créer, par des combinaisons empiriques de coulis, de
nouvelles sensations gustatives.

Louis XV, comme le régent, était non seulement gourmet, mais encore cuisinier :
il excellait en particulier dans la fabrication des pâtés, et il n'aurait
laissé à personne le soin de faire son café.

Ce fut sous son règne que figurèrent pour la première fois, dans les menus de
la Cour, l'ananas qui venait de Surinam, la fraise originaire du Chili et
importée en {\ppp1\mmm} {\ppp716\mmm} par M. Frezier, le sagou originaire de
l'Inde, et que le poivre de Cochinchine fut acclimaté dans l'Ile-de-France.

Les fromages les plus appréciés étaient ceux de Brie, Roquefort, Cantal, Berry,
Livarot, Pont-l'Évèque, Maroilles, Vanves, Clamart, Gournay.

Dans la longue liste des créations de ce temps, on trouve les pains à la
d'Orléans, œuvre du régent ; les pâtés de gibier truffés ; les
animelles\footnote{Testicules déboursés du bélier.} sautées ; l’'omelette à la
royale, aux crêtes de coq, aux testicules de poulets, aux filets d'ortolans et
aux champignons ; les filets de lapereau à la Berry, imaginé par la fille du
régent ; les filets de volaille et les tendrons d'agneau à la Bellevue,
exécutés pour la première fois, en l'honneur du roi, au château de Bellevue,
sous l'inspiration de la marquise de Pompadour ; le vol-au-vent à la Nesle ; la
chartreuse à la Mauconseil ; les poulets à la Villeroi ; les cailles
à l'appareil Mirepoix : les ris de veau à la d'Artois : les coulis d'écrevisses,
les coulis de gibier et le potage bisque du président Hénault : la garbure aux
marrons, de Sénac de Meilhan ; le consommé, les bouchées et les poulets à la
Reine, trouvailles de Marie Leszczynska ; les boudins à la Richelieu ; la sauce
à la crème connue sous le nom de Béchamel, qui est due au financier Béchameil,
plus tard marquis ; la sauce Soubise ; la sauce mayonnaise, qui, d'après les
uns, aurait d’abord été appelée bayonnaise, du nom de Bayonne, où elle aurait
été inventée, et qui, d'après d'autres, se serait nommée d'abord mahonnaise et
devrait être attribuée au duc de Richelieu, qui en aurait eu l'idée pendant le
siège de Port-Mahon : l'appareil d'Uxel ; les fraises aux oranges du comte de
Laplace : les échaudés, créés par Favart : les madeleines de Commercy ; les
crêpes du cardinal de Bernis : le baba du roi Stanislas ; les crèmes, les
mousses, les fromages glacés, etc. : j'en passe et des meilleures.

Les vins de Bordeaux en général, les vins de Bourgogne, et en particulier le
chambertin et le clos Vougeot, jouissent dès ce moment de toute la réputation
qu'ils méritent.

Enfin, en {\ppp1\mmm} {\ppp765\mmm}, un homme de bien, dont le nom doit être
conservé, Boulanger, fonda à Paris, dans l’ancienne rue des Poulies, le premier
restaurant\footnote{La différence fondamentale entre les restaurants et les
cabarets qui, comme nous l'avons vu, existaient déjà dans l’antiquité, c'est
que dans les cabarets il n'y avait pas de carte. Le plus souvent, le cabaretier
se bornait à faire cuire les aliments que les clients apportaient, et, pour
trouver un repas au cabaret, il était indispensable de le commander d'avance.

Les pieds de mouton à la poulette étaient la renommée du restaurant
Boulanger.}, et ce fait divers, en apparence banal, inaugura pour la
gastronomie une ère nouvelle. Jusqu'alors, la cuisine fine était pour ainsi
dire monopolisée par la noblesse, le clergé, la magistrature et la finance ou,
d'une façon plus générale, par les classes riches ; l'institution des
restaurants, en dehors de son caractère purement utilitaire, eut pour effet de
permettre à quiconque avait quelques louis en poche de s'offrir et d'offrir
à ses amis, sans aucun embarras, un repas délicat : la gastronomie y gagna
beaucoup d'adeptes ; nombre de vocations se révélèrent et, phénomène absolument
imprévu, l'art culinaire s'affina en se démocratisant.

Louis XVI était un boulimique et c'est ce qui a causé sa mort. Lors de sa
fuite, il ne sut pas résister, malgré les objurgations de la reine, aux charmes
d'un copieux déjeuner qui lui était offert, à Étoges, chez M. de Chamilly, son
premier valet de chambre. Il s’y attarda longuement, ne pouvant se décider
à quitter la table, ce qui le fit arriver en retard à Varennes, d'où les
cavaliers qui devaient l'escorter jusqu'à la frontière étaient partis, après
l'avoir longtemps attendu, désespérant de le voir arriver. A la vérité, il
avait calmé sa fringale, mais sa famille et lui-même payèrent de leurs têtes
cette fugitive satisfaction.

Sous son règne, la pomme de terre originaire de l'Amérique du Sud, introduite
en Europe dès {\ppp1\mmm} {\ppp565\mmm} par Hawkins, entra dans l'alimentation
grâce à la persévérance de Parmentier, et cet événement peut être considéré
comme fondamental dans l'histoire de la gastronomie.

La cuisine des provinces ne le cédait en rien à celle de la capitale. On lui
doit la garbure béarnaise, les escargots en coquilles, la bouillabaisse et les
paquets de Marseille, la bourride de Cette\footnote{Ancienne
orthographe du nom de la ville de Sète (avant 1928), ndlr.}, la brandade de morue,
l'ailloli, la meurette comtoise, la sole normande, le civet de lamproie gascon,
les tripes à la mode de Caen, connues dès la fin du
\textsc{xv}\textsuperscript{e} siècle, le gras-double et les quenelles de
brochet à la lyonnaise, le cassoulet de Castelnaudary, le lièvre à la royale,
les gratins dauphinois, les quenelles à la Nantua, le canard rouennais au sang,
les pâtés de foie gras truffés de Strasbourg, de Nancy, de Cahors, les terrines
de Nérac, les pâtés de perdreaux de Chartres, les pâtés de canard d'Amiens, les
pâtés d'alouettes de Pithiviers, etc.

Ce fut sous Louis XVI que Dutfoy remplaça dans la décoration des tables un
certain nombre de pièces d’orfèvrerie par des pièces montées en pâtisserie, de
forme architecturale.

\sk

\textit{Époque contemporaine}. — La Révolution amena au pouvoir un monde
nouveau. Quelques révolutionnaires seulement, entre autres le conventionnel
Barrère et le général Barras, savaient manger, mais en réalité c'étaient plutôt
des bourgeois égarés dans la tourmente.

Les cuisiniers des émigrés et des victimes de la Terreur, sans place alors,
fondèrent des restaurants, initièrent les couches nouvelles à la bonne chère,
et préparèrent l'avènement de la bourgeoisie moderne. Pendant ce temps, plus
d'un émigré, pour vivre, utilisait à l'étranger ses talents
gastronomiques\footnote{M. d'Albignac refit sa fortune à Londres on donnant des
consultations sur l’art d'accommoder les salades.} et contribuait ainsi
à répandre dans le monde la renommée de la cuisine française.

Les créations principales de cette époque sont : le bifteck à la Chateaubriand,
les tourtres aux rognons, les godivaux et les pâtés de ris de veau de Toutain à la
Toulouse, les langues fourrées, les andouilles de fraise de veau au ris de veau, les
boudins blancs aux truffes, aux pistaches et aux écrevisses, dus à Mouniot, etc.

Enfin, dans les dernières années du \textsc{xviii}\textsuperscript{e} siècle,
un confiseur de Paris, Appert, imagina la préparation des conserves, dont le
rôle dans l'alimentation est aujourd'hui si considérable.

Napoléon I\textsuperscript{er} était un grand stratège, mais c'était un triste
mangeur, complètement indifférent aux charmes des combinaisons culinaires.
Manger semblait n'être pour lui qu'une corvée. La seule fois qu'il exprima un
vœu gastronomique, ce fut pour demander des saucisses plates, réminiscence de
ses repas de sous-lieutenant. Le premier maître d'hôtel de la Cour, jugeant le
plat indigne de Sa Majesté, lui fit servir en place un hachis de perdreaux en
crépinettes, que l'empereur avala sans s'apercevoir de la substitution.
Cependant la table impériale était très abondamment et très luxueusement
servie, car l'empereur sut toujours s'entourer des meilleurs spécialistes dans
toutes les branches, et les cuisines des Tuileries furent une véritable école
d'où sortit toute une pléiade d'artistes de valeur.

Ce ne fut qu'à la fin de sa carrière, à Sainte-Hélène, lorsqu'il resta pendant
un certain temps privé du service de ses cuisiniers qui avaient fait, sans du
reste qu'il s'en doutât le moins du monde, de véritables tours de force pour
rendre mangeables les aliments mis à leur disposition, que Napoléon reconnut
l'utilité de la gastronomie et ajouta à d'autres regrets le regret tardif de
l'avoir méconnue toute sa vie.

Louis XVIII était à la fois un gourmand et un gourmet. Il se connaissait
remarquablement en fruits et, les yeux fermés, il distinguait au simple goûter
les variétés les plus voisines. On lui doit quelques potages parmi lesquels je
citerai une purée de lentilles aux croûtons et un potage aux pâtes fluides,
imité d’un potage autrichien qu'il avait sans doute apprécié pendant
l'émigration : on lui attribue aussi la paternité de la côtelette dite « la
victime\footnote{On prépare la côtelette à la victime de la façon suivante : on
prend trois côtelettes, on les lie ensemble en plaçant la plus belle entre les
deux autres : on fait cuire le tout sur le gril en retournant fréquemment la
viande, de façon à concentrer le jus dans la côtelette du milieu, que l'on sert
seule.} ».

Son frère Charles X, qui avait déjà donné sa mesure comme gastronome alors
qu'il était duc d'Artois, était un fin connaisseur. Habituellement froid et
réservé, il devenait aimable et expansif quand il était à table, devant un menu
soigné.

Les débuts du \textsc{xix}\textsuperscript{e} siècle comptèrent beaucoup de
gastronomes de marque, entre autres Talleyrand qui, grâce à l'excellence de sa
table, obtint des alliés \textit{inter pocula} certains adoucissements dans les
clauses de la capitulation de {\ppp1\mmm} {\ppp814\mmm} ; le marquis de Cussy
à qui l’on doit les asperges au gratin ; le marquis d'Aigrefeuille : Grimaud de
la Reynière et Brillat-Savarin, le célèbre auteur de la « Physiologie du
goût ».

Le plus grand cuisinier de l'époque fut Carême qui s'illustra surtout dans les
plats froids, les plats maigres et les entremets. Technicien consommé, très
érudit dans toutes les branches de son art, Carême, qui a laissé de nombreux
ouvrages, connaissait la préparation de trois cents potages différents ; il est
le créateur du vol-au-vent moderne et ce titre seul suffirait pour perpétuer
son nom.

À partir du milieu du \textsc{xix}\textsuperscript{e} siècle, la cuisine est
sensiblement celle de nos jours.

Comme créations culinaires on peut citer : le potage Camerani, aux foies de
poulets, du Café Anglais ; le homard à l'américaine, de chez Bonnefoy ; la
sauce Mornay, du Grand Véfour ; les filets de caneton aux oranges ; le poulet
braisé financière, de la Maison Dorée, création de Casimir : le macaroni et les
tournedos Rossini ; le poulet sauté Archiduc : le canard à la presse : le
soufflé de homard ; la langouste farcie gratinée ; les pommes de terre
soufflées ; les pommes de terre Anna ; les pommes de terre au jus, du Maître
Blau ; la salade japonaise décrite par Alexandre Dumas fils dans
« Francillon » ; le pudding à la diplomate, chef-d'œuvre de Montmirel ; le
savarin de Julien : l’omelette soufflée en surprise, etc., etc. Mentionnons
aussi, comme cuisiniers remarquables ayant laissé des œuvres : Urbain Dubois,
Emile Bernard, Jules Gouffé, Joseph Vuillemot et Joseph Fabre ; comme
gastronomes célèbres : Alexandre Dumas, Rossini, Jules Janin, le Dr. Véron, le
baron Brisse et Monselet, ce dernier plus gourmand que gourmet. Comme
restaurants de Paris disparus aujourd'hui et qui firent les beaux jours du
second empire, rappelons Bignon et la Maison Dorée sur la rive droite, Magny
sur la rive gauche.

L'art culinaire français semble être alors à son apogée. Sa supériorité se
manifeste en tout, dans la perfection des mets, dans la composition des menus,
dans le dressage des tables, dans le service. Cette supériorité est due
à plusieurs causes : à la richesse du sol, dont les produits sont exquis : à la
compétence des agriculteurs, des jardiniers et des éleveurs qui ont créé, tant
dans le règne végétal que dans le règne animal, des variétés admirablement
sélectionnées ; à l'art des fabricants de fromages et de conserves, et aussi
à la préparation particulièrement soignée des jus et des coulis qui sont la
base fondamentale de la bonne cuisine ; enfin, à nos vins, uniques au monde,
qui en sont le complément.

La plupart des cuisiniers français ont pour ainsi dire sucé avec le lait les
bons principes culinaires et ce fait seul suffit à leur assurer une maîtrise
incontestable.

Paris, centre du monde qui attire de partout les amateurs à la recherche de
sensations nouvelles, est particulièrement favorable au développement de tous
les arts et il n'est pas surprenant que, grâce à ce concours de conditions
exceptionnelles, l'art culinaire français, qui depuis plus de deux siècles
était à la tête du mouvement gastronomique mondial, soit arrivé à une
délicatesse et à une finesse absolument incomparables.

\sk

Mais de ce que notre cuisine et notre pâtisserie sont les premières du monde,
il n'en faudrait pas conclure qu'il n'existe rien de bon dans les autres pays.

Nous avons malheureusement une tendance prononcée à l'exclusivisme. De même
qu'autrefois chez les Grecs et à Rome tous les étrangers étaient considérés
comme des Barbares, de même nous voyons volontiers en eux des sauvages, à moins
que, par une exagération tout aussi absurde, nous n'en fassions des héros ou
des demi-dieux. Or, il y a en tout un juste milieu ; il convient d'être
impartial et il faut savoir être éclectique.

Il est indiscutable qu'il existe à l'étranger des plats absolument différents
de ceux auxquels nous sommes habitués et qui méritent pourtant qu'on s'y
arrête. Aussi, je crois bon de passer ici rapidement en revue les principales
cuisines étrangères.

\index{Cuisines étrangères}

\section*{\centering Cuisines étrangères.}

\textit{Italie}. — La cuisine italienne est le triomphe des pâtes, et les
Italiens ont une infinité de manières de les préparer. Ils ont aussi d’autres
plats originaux, tels que la \textit{polenta}, bouillie de maïs très nutritive
qui, préparée au jus de viande ou de gibier, est excellente ; le
\textit{risotto}, dont les variantes sont nombreuses ; le \textit{minestrone},
soupe milanaise aux légumes, au riz, au macaroni, avec du jambon, des saucisses
et du fromage, le tout aromatisé d'herbes, parmi lesquelles je citerai le
basilic, dont l'usage est très répandu dans le nord de l'Italie ; les
\textit{grisini}, sorte de pain-biscuit en forme de baguette, qu'on fait
à Turin avec un mélange de farine de manioc et de gruau ; les
\textit{agnoloti}, beignets de hachis de viandes, qu'on préparait autrefois
exclusivement avec de l'agneau, d'où leur nom, et qu'on fait aussi maintenant
avec du poulet et avec d'autres viandes ; enfin les \textit{ravioli}, hachis de
viandes ou de légumes enrobés dans de la pâte.

La volaille et la viande de boucherie, sauf parfois le veau et l'agneau, sont
franchement médiocres : aussi sert-on beaucoup de viandes hachées. L'une des
préparations les plus courantes du veau et du poulet est celle dite « à la
viennoise », importée par les Autrichiens. Le poisson, notamment celui de
l'Adriatique, est excellent et les fritures italiennes, sans valoir les nôtres,
sont bonnes. Comme charcuterie, je ne trouve guère à citer que la mortadelle de
Bologne. Les légumes les plus répandus sont le \textit{brocoli}\footnote{Le
brocoli est un chou de la variété des choux-fleurs, poussant en rameaux
séparés, d'une couleur soit jaune, soit violette, qui passe plus ou moins
complètement à la cuisson.} et les \textit{finocchi}\footnote{Les finocchi sont
les bourgeons du fenouil, plante odoriférante de la famille des Ombellifères.}
peu connus en France, et la tomate, qui est parfaite. Comme fromages je
mentionnerai le \textit{parmesan} qui est un remarquable fromage
d’assaisonnement et le \textit{gorgonzola}, fromage honorable, qui a la
prétention de concurrencer le roquefort, avec lequel il n'a guère de commun que
les moisissures qui le marbrent. Le beurre italien laisse malheureusement
souvent a désirer.

Qui n'a pas été en Italie, ne se doute pas de tout ce que l'on peut faire avec
les pâtes, le maïs, le riz, la tomate et le parmesan.

La pâtisserie italienne n'est pas fameuse ; je noterai pourtant le millefeuille,
gâteau feuilleté que l'on sert, le plus souvent, garni de crème, de fromage
blanc ou de confitures et qui rappelle alors le \textit{strudel} bavarois ou
viennois, et la \textit{pasta frolla}, gâteau napolitain aux amandes. Un gros
défaut de la pâtisserie italienne est l'abus du sucre, poussé dans certaines
régions jusqu'à l'invraisemblance. C'est ainsi que J'ai vu, en Sicile, des
gâteaux dans la composition desquels le sucre entrait pour plus de moitié.

Les vins du Vésuve et ceux de Sicile sont assez bons, le \textit{chianti} est
un vin de table suffisant et l'\textit{asti} est une piquette sucrée qui n'est
pas désagréable pour faire passer la pâtisserie.

En résumé, la cuisine italienne ne manque pas de qualités et ses défauts
tiennent surtout à l'infériorité des viandes et du beurre, infériorité dont les
cuisiniers ne sauraient être rendus responsables.

\sk

\textit{Espagne}. — L'Espagne est un bien beau pays, mais la cuisine y est bien
médiocre, pour ne pas dire davantage. Sauf à Barcelone, où j'ai pu vivre à la
française, je ne me souviens d'avoir mangé à peu près convenablement en Espagne
qu'une fois à Madrid, chez des amis, et une autre fois
à Séville\footnote{Depuis que ces lignes ont été écrites, les conditions
matérielles de la vie se sont améliorées en Espagne. On y trouve aujourd'hui
dans plusieurs villes, notamment à Madrid, Grenade, Algésiras, des
installations confortables et une cuisine soignée.}. 

En réalité, il n'y a guère que le porc qui y soit bon : le beurre y est
détestable, l'huile rarement bien préparée, et le vin, transporté dans des
outres, sent fréquemment le bouc.

Les plats les plus connus de la cuisine espagnole, acceptables s'ils sont bien
faits, sont le \textit{puchero}, pot-au-feu à base de bœuf et de porc avec des
légumes, entre autres des \textit{garbanzos} ou pois chiches : la morue à la
biscayenne, aux piments et à la tomate ; les
\textit{almondigillas\footnote{albóndigas, ndlr.}}, boulettes de hachis
de filet de bœuf au lard qu'on fait mijoter dans du jus de tomates ; les
\textit{criadillas} frites, ou friture d'animelles ; l'\textit{olla podrida},
pot pourri de viandes de boucherie, de porc, de volaille et de gibier avec les
inévitables garbanzos et toute une macédoine de légumes variés ; le
\textit{chorizo}, saucisson de bœuf, de veau et de porc ; le poulet à la
valencienne, qui est un poulet au riz avec des saucisses, des tomates farcies,
des fonds d'artichauts, etc. ; le jambon doux des Asturies cuit dans du xérès ;
les \textit{escabeche}, sorte de salmis de poissons et de gibier, dont le plus
apprécié est celui de perdreau, et le \textit{gaspacho}.

Comme vins espagnols, mentionnons l'alicante, le malaga et le xérès.

\textit{Portugal}. — La cuisine portugaise ne vaut guère mieux que la cuisine
espagnole. On consomme en Portugal beaucoup de pois chiches et la plupart des
sauces sont aromatisées avec de la tomate : on met même de la tomate dans le
pot-au-feu portugais, \textit{cucido\footnote{\textit{cozido} ? orthographe
plus courante de nos jours, ndlr}}, qui est assurément ce qu'on mange de
meilleur dans le pays. Les tripes à la mode de Porto, qui rappellent les tripes
à la mode de Caen, sont servies avec de la farine de manioc. Les vins portugais
les plus renommés sont le porto et le madère\footnote{Le madère authentique
n'est plus aujourd'hui qu'un souvenir. Je me rappelle en avoir bu pour la
dernière fois à Madère chez le fils d'un des derniers producteurs, il y a une
quinzaine d'années ; à cette époque déja, il eût été difficile d'en réunir une
centaine de bouteilles dans toute l'île.}.

\sk 

\textit{Grande-Bretagne}. — La cuisine anglaise est restée longtemps arriérée.
Il y a trois siècles à peine, on ne cultivait pas de légumes en Angleterre et
la nourriture y était presque exclusivement carnée : aujourd'hui hui, en dehors des
pommes de terre, les légumes les plus appréciés sont le céleri, le topinambour,
le chou et l'oignon doux.

L'Angleterre est essentiellement le pays de la viande et du poisson. Le bœuf,
le mouton, le porc y sont hors ligne : les soles, les saumons, les turbots
y sont exquis, les white bait\footnote{J'ai eu l'occasion de manger
d'excellentes fritures de white bait en Sicile : je note le fait, parce que
beaucoup de personnes croient à tort que ce poisson ne se trouve qu'en
Angleterre.} ou \textit{coregonus albus}, qu'on pêche dans la région de
l'embouchure de la Tamise, donnent une friture inconnue chez nous et les
huîtres anglaises sont parfaites. Un très bon gibier, que nous n'avons pas en
France, est la grouse de bruyère qui abonde en Écosse et en Irlande et dont la
chair, aromatisée de serpolet, tient à la fois de celle du coq de bruyère et de
celle de la gélinotte.

Le potage à la tortue de mer, potage national anglais, est à la hauteur de sa
réputation ; le \textit{mulligatawny}, d'origine indienne, bouillon de porc
assaisonné de poudre de curry, lié à la crème et à la fécule, et servi avec du
riz, est également très agréable ; le potage à la queue de bœuf, \textit{oxtail
soup,} n'est pas sans mérite et le \textit{porridge}, potage écossais, bouillie
à la farine d'avoine qu'on sert généralement au premier repas du matin, est
populaire. Les hadocks\footnote{Le hadock ou églefin, qu'on appelle encore
quelquefois aigrefin ou merluche, est une espèce de morue qui se trouve plus
particulièrement dans la mer du Nord.}, les harengs fumés d'Angleterre inondent
les marchés du continent. Le \textit{roast beef} anglais est exquis, les
\textit{mutton chops}, les sandwichs imaginés au
\textsc{xviii}\textsuperscript{e} siècle par lord Sandwich, les \textit{fried
eggs and bacon}, œufs frits au lard anglais, les \textit{puddings} et les
biscuits anglais sont classiques.

Comme fromages, le \textit{stilton} est renommé et le \textit{chester} très
apprécié. Les \textit{welsch rabbit} ou \textit{welsch rare bit} sont des
rôties beurrées sur lesquelles on étend du fromage de Gloucester fondu, mélangé
avec de la crème et relevé par de la moutarde.

Comme boissons, les bières anglaises sont bien connues,

L'infériorité de la cuisine anglaise se manifeste dans les sauces et dans les
ragoûts. Les Anglais abusent des herbes aromatiques, des condiments, des
pickles. Sous prétexte de cuisine simple, ils emploient souvent, au lieu de nos
fonds de cuisson à base de jus et de nos sauces mijotées, des sauces violentes
toutes préparées telles que le \textit{ketchup}, le \textit{soy}, le
\textit{Worcestershire sauce} et le \textit{Harvey sauce}, et à des doses
telles qu'elles finissent par masquer complètement le goût des préparations
culinaires. Cependant, depuis quelque temps, grâce à l'influence du roi Édouard
VII, qui était un fin gourmet, et au concours des artistes français qui
gouvernent les cuisines de beaucoup de grandes maisons anglaises, la
gastronomie a fait des progrès considérables de l'autre côté de la Manche et la
cuisine anglaise d'aujourd'hui se ressent fortement des perfectionnements que
nos compatriotes y ont apportés.

\sk

\textit{Allemagne}. — La cuisine allemande, qui dérive du génie allemand,
manque de légèreté et de finesse : elle ne peut guère convenir qu'à des
estomacs de grands buveurs de bière. L'une de ses caractéristiques est
l'alliance du sucré avec le salé : mais ce trait ne suffit pas pour la
condamner \textit{à priori}. Ne met-on pas souvent chez nous un peu de sel dans
les plats sucrés ? Pourquoi, inversement, ne mettrait-on pas quelquefois un peu
de sucre dans les plats salés ? C'est une question de mesure et de tact. Il
convient de mentionner un certain nombre de plats du pays ; les artistes
trouveront dans la composition de ces plats germaniques des idées qui,
appliquées avec goût, pourront donner des résultats intéressants.

Citons la soupe à la bière, qui ne vaudrait assurément rien préparée avec nos
bières ou avec des bières allemandes exportées, mais qui, fabriquée sur place
avec de la petite bière du pays, blanche, mousseuse, aigrelette, est acceptable
et originale ; les huîtres roulées dans du parmesan râpé, panées et frites ; la
fricassée de brochet aux écrevisses et aux morilles ; le bœuf salé de
Hambourg ; le bœuf à la berlinoise, mariné dans de la bière blanche aigrelette
et cuit dans sa marinade avec du lard et des légumes (on le sert avec la
cuisson concentrée, passée, liée, aromatisée de zeste d'orange, de chair de
citron et édulcorée avec un peu de gelée de groseilles) ; les
\textit{pfannenkuchen gefüllt mit}... etc., omelettes à la farine farcies d'un
hachis cuit de veau, jambon et foies de volaille, le tout saupoudré de parmesan
et recuit ensemble dans une casserole avec du beurre ; le ragoût de mouton,
\textit{hammelragout}, avec saucisses et purée de pommes de terre ; la poule de
Hambourg farcie de mie de pain maniée avec du beurre ; l'oie à la
mecklembourgeoise, farcie de pommes douces et de raisins, braisée, et servie
avec des choux rouges : le lièvre à la bavaroise, au vin du Rhin ; les filets
de lièvre à l'allemande, avec une sauce rendue douce par l'addition d'un peu de
gelée de groseilles et de raisins de Corinthe ; le faisan à la silésienne, à la
choucroute ; le pâté de lièvre à la saxonne, avec interposition de lits de
choucroute ; la salade de harengs salés, avec pommes de terre, betteraves,
cornichons et concombres salés ; la salade d'asperges et de queues
d'écrevisses, à l'huile et au vinaigre, avec une purée de jaunes d'œufs durs ;
le \textit{nampfkuchen}, sorte de baba ; le \textit{schmarr} à la bavaroise,
omelette à la farine, cuite à la poêle, puis coupée en morceaux qui sont
recuits ; le \textit{dampfnudel}, entremets de nouilles ; enfin les charlottes
de pommes et les flans de cerises, qui sont ce qui se fait de mieux en
pâtisserie dans le pays.

Les bières allemandes sont célèbres.

Les vins du Rhin ont une saveur originale ; les crus les plus renommés sont le
\textit{johannisberg}, le \textit{rudesheimer berg}, le \textit{schloss
volrathser}, le \textit{rosengarden} et le \textit{liebfraumilch} (lait de la
femme aimée !).

\sk

\textit{Pays-Bas}. — Parmi les plats intéressants de la cuisine des Pays-Bas,
citons, en Belgique : le \textit{waterzooi}, sorte de bouillabaisse de poissons
de rivière : les paupiettes de sole à la flamande, garnies d'œufs de harengs
saurs ; le ragoût de bœuf à la flamande, aux oignons et à la bière ; le ragoût
de queue et rognon de bœuf, ris de veau et pieds de mouton, aux champignons et
à la bière ; en Hollande : le \textit{kalbspolet}, pot-au-feu de veau au riz,
aux laitues et aux petits pois ; l'excellent cabillaud de la mer du Nord,
simplement poché dans de l'eau salée et servi avec des pommes de terre cuites
à la vapeur, le tout accompagné de beurre fondu ; les quenelles de petit salé ;
enfin, le bœuf des pâturages hollandais, préparé de toutes façons, réellement
incomparable.

\sk

\textit{Pays scandinaves}. — La cuisine danoise se rapproche beaucoup de la
cuisine allemande ; la cuisine suédoise se rapproche plutôt de la cuisine
russe ; quant à la cuisine norvégienne, elle n'existe pas. On ne trouve guère
en Norvège, sauf à Christiania et à Bergen, autre chose que du poisson et
surtout du saumon. Bouilli, grillé ou fumé, il est excellent, mais on s’en
fatigue vite. Pour donner une idée de la cuisine norvégienne, je citerai comme
plat de résistance le saumon en aspic, que l'on prépare de la façon suivante :
on enrobe dans une gelée de colle de poisson sucrée des darnes de saumon
bouilli, qu'on sert avec des pommes de terre, incomplètement cuites à l'eau, et
avec du lait caillé sucré !

Comme entremets, il n'y a que de la confiture de baies d'airelles ; comme
fromage, du fromage de renne ; comme pain, une espèce de mauvais pain d'épice
spongieux. C'est effrayant !

\sk

\textit{Pologne}. — La cuisine polonaise mérite d'être étudiée. Le potage n'y
est pas, comme en France, une entrée en matière souvent sans grande importance.
Les potages polonais sont généralement très consistants : ce sont de véritables
plats, au même titre que la bouillabaisse, mais ils ne renferment jamais de
pain. Lorsqu'ils sont à base de bouillon animal, ils contiennent presque
toujours de la viande sous une forme quelconque : en morceaux, en hachis, en
quenelles, ou enrobée dans de la pâte, et ils sont rarement passés. Les
liaisons sont faites surtout à la crème. Il existe en Pologne trois autres
variétés de potages inconnus en France : la première, caractérisée par une
saveur aigrelette obtenue soit par de la crème aigre, soit par un jus
fermenté ; la deuxième, celle des potages froids à la glace ; la troisième,
comprenant des potages plus ou moins sucrés aux amandes et aux fruits, que nous
considérerions plutôt comme une espèce d'entremets.

Les sauces polonaises sont ordinairement préparées à base de jus de viande ou
à base de crème : elles sont aromatisées avec du raifort, du fenouil, des
concombres saumurés, de la civette, des champignons, du jus de citron, etc. ;
elles renferment aussi quelquefois des substances plus ou moins sucrées, telles
que des raisins secs.

La volaille et la viande de boucherie sont rarement tendres, aussi prépare-t-on
beaucoup de viandes braisées et de hachis. Le porc est généralement bon, comme
dans la plupart des pays pauvres et les cervelas de Varsovie
(\textit{sardelki}) sont renommés. Le gibier est excellent, mais il a toujours
un goût sauvage prononcé ; on ne connaît guère en Pologne le fumet distingué
des perdreaux et des faisans nourris dans les chasses gardées de France.

Citons comme productions culinaires : le \textit{barszcz}, type des potages
aigrelets, au jus de betteraves fermenté ; le \textit{krupnik}, potage à l'orge
perlé ; la \textit{czernina}, potage au sang de canard, d'oie, de lièvre ou de
porc ; le potage aux écrevisses à la crème ; les \textit{chlodniki}, potages
froids à la glace ; le brochet au vin blanc, avec une sauce à la crème et au
raifort ; la carpe à l'hydromel, avec une sauce au sang ; les tripes et les
boudins au gruau de sarrasin ; le chaud-froid de tête de veau farcie, sauce au
raifort ; les \textit{zrazy}, tranches de viande de boucherie braisées ; les
\textit{zrazy zawijane}, paupiettes braisées ou rôties ; le filet de bœuf
haché, à la crème et au raifort ; le \textit{bigos}, ragoût de gibier, de
viandes de boucherie et de porc, aux choux mélangés avec des pommes aigres ; le
poulet farci, bardé et cuit à la vapeur : le râble de lièvre à la crème ; les
\textit{pierogi}, pâtés à la viande, aux légumes ou au gruau ; les
\textit{kluski}, pâtes au fromage.

La pâtisserie polonaise, sans approcher comme finesse de la pâtisserie
française qui est unique au monde, est agréable. Je me contenterai de citer les
crêpes polonaises à la farine et au gruau de sarrasin, \textit{nalesniki} ; le
baba et certains autres gâteaux : \textit{placki} (platski), aux amandes ;
\textit{paczki} (pontchki), aux confitures ; le pain d'épice de
Torun\footnote{Patrie de Kopernik.} (Thorn, en allemand) et d'autres
pâtisseries composées de lits alternés de pâte croustillante et de remplissages
moelleux, qui méritent qu'on s'y arrête.

Au \textsc{xvi}\textsuperscript{e} siècle, la princesse Bona Sforza, ayant
épousé le roi Sigismond le Vieux, amena avec elle en Pologne toute une suite
d'artistes italiens ; les uns y construisirent des monuments du plus pur style
Renaissance, dont il reste encore de beaux spécimens, notamment à Cracovie ;
d'autres y acclimatèrent la tomate qui porte dans le pays une dénomination
fleurant son origine : \textit{pomidor}, et ils enseignèrent aux Polonais l'art
de travailler les pâtes ainsi que l'emploi du parmesan.

Par suite des différentes invasions que la Pologne a subies, des éléments
étrangers sont venus se mêler à sa cuisine nationale proprement dite.

C'est ainsi que les Tartares ont importé en Lithuanie les \textit{kolduny},
farcis de chair de bœuf crue, précurseurs des \textit{ravioli} italiens ; les
Turcs ont vulgarisé en Pologne l'usage du riz, du gruau et celui des semoules,
\textit{kasza} (kacha), si employées en couscous dans les pays musulmans ; les
Autrichiens ont apporté leurs viandes panées frites et leurs plats doux ; les
Allemands, certaines combinaisons plus ou moins sucrées de viandes et de
fruits, dont quelques-unes, par exemple l'oie rôtie à la purée de pommes
douces, sont beaucoup moins invraisemblables qu'elles n'en ont l'air ; les
invasions moscovites ont valu aux Polonais, entre autres choses, le
caviar\footnote{Au sens propre du terme, le caviar est un excellent
hors-d'œuvre constitué par des œufs de sterlet on d'esturgeon plus ou moins
saumurés. Au figuré, ce mot désigne les taches d'encre d'imprimerie avec
lesquelles les agents de la censure russe masquent, dans les publications
indigènes ainsi que dans les publications étrangères qui pénètrent dans le
pays, les passages qu'ils jugent subversifs. La sombre ignorance de ces
fonctionnaires leur fait commettre parfois des gaffes dans lesquelles le
comique se mêle au tragique. J'ai vu dans un musée, en Galicie, une table de
logarithmes qui, après avoir été largement caviarée, avait été définitivement
confisquée, les censeurs ayant cru voir dans les chiffres composant le texte
une correspondance secrète, cabalistique et révolutionnaire. Quant au
propriétaire du livre auquel on avait demandé en vain, et pour cause, la clef
du secret, il avait été pendu, tout simplement. Doux pays !}.
  
Le seul reproche que l’on puisse faire à la cuisine polonaise, qui est
savoureuse et succulente, c'est d'exiger de bons estomacs, ce qui à la vérité
est un défaut ; mais elle présente des combinaisons culinaires peu connues en
France et susceptibles de fournir, avec des matériaux supérieurs et de légères
modifications, des mets de premier ordre.

\textit{Russie}. — L'Histoire nous apprend qu'en {\ppp1\mmm} {\ppp815\mmm} les
cosaques campés aux Champs-Élysées mangeaient de la chandelle. Plus récemment,
dans un voyage que j'ai fait en Sibérie, j'ai surpris d'autres cosaques,
successeurs de ceux de {\ppp1\mmm} {\ppp815\mmm}, buvant le contenu de mes
lampes. Mais il ne faudrait pas conclure de ces faits incontestables que ce
soit là le régime alimentaire de tous les Moscovites.

La cuisine russe présente un certain nombre de particularités qui méritent
d'être notées. Il existe en Russie des animaux comestibles qu'on ne trouve pas
chez nous, tels : le sterlet du Volga, poisson gras et très délicat,
l’esturgeon, le soudac\footnote{ou sandre, grand poisson de rivière du genre
\textit{Lucioperca sandra}, dont le goût rappelle celui de la perche.}, le
sigui\footnote{ou lavaret, \textit{Coregonus lavarelus}, espèce de saumon du
genre corégone.}, le riapouschka\footnote{Petit poisson qu'on mange surtout
fumé} du lac Ladoga, le kilkis\footnote{Petit poisson qu'on prépare à l'huile,
comme la sardine.} de Revel, le navaga\footnote{Sorte de morue expédiée gelée
d'Arkhangel.}, une variété de gélinotte, différente de celle des Pyrénées, que
l’on importe couramment en France depuis quelques années, l'élan, qui se
rapproche du chevreuil tout en étant moins fin, l'ours, qui constitue encore
chez nous une nourriture exceptionnelle et dont nous ne connaissons guère le
goût que depuis le siège de Paris, en {\ppp1\mmm} {\ppp850\mmm}, pendant lequel
nous eûmes l'ingratitude de dévorer notre favori l'ours Martin du Jardin des
Plantes.

La viande de boucherie est de qualité inférieure, sauf le bœuf et le mouton de
Circassie qui sont passables, et le veau de Moscou, nourri au repos de lait et de
noix, qui a très bon goût. Aussi fait-on dans la cuisine russe de nombreux hachis
que l'on baptise biftecks ou côtelettes suivant la forme qu'on leur donne.

Le service à la russe est original ; il est caractérisé par le dressage des mets,
découpés d'avance, et par une multitude de hors-d'œuvre, parmi lesquels
figurent toute espèce de poissons salés, fumés, marinés, des canapés de caviar ou
des sandwichs garnis de beurres composés, de purées de poissons, de mollusques,
de queues d'écrevisses, de crevettes, des bouchées, des petites croustades, des
petits pâtés de viandes et de gibiers masqués de fumets adéquats, etc., disposés
sur une table dans une pièce voisine de la salle à manger, et qu'on absorbe
debout, en buvant de l'eau-de-vie ou des liqueurs, avant de commencer le repas
proprement dit : ce sont les \textit{zakouski}. Ils précèdent les repas soignés.

Les potages russes ressemblent dans leurs grandes lignes aux potages polonais.
Les plus curieux sont : le \textit{chtchi} au gras, renfermant des légumes, de
la choucroute et des viandes telles que canard, poulet, bœuf, saucisses ; le
\textit{chtchi} au maigre, dans lequel le bouillon de viande est remplacé par
un bouillon de cèpes ; la \textit{botwina}\footnote{La botwina est en somme une
variété de chlodnik.}, potage d'esturgeon, acidulé par des concombres saumurés,
renfermant des écrevisses, de l'oseille, des épinards, que l'on sert glacé ;
l'\textit{ouka} au sterlet, potage très apprécié par les gourmets russes.

Comme plats originaux, on peut citer les \textit{koulibiak}, qui sont des pâtés
préparés de différentes façons, dont l'un des plus réputés est le koulibiak de
saumon et de lavaret, à la \textit{vesiga}\footnote{La vesiga est une substance
gélatineuse provenant de l'épine dorsale de l'esturgeon et du sterlet.} et au
gruau de sarrasin ; les \textit{bliny}, sorte de crêpes plus ou moins épaisses,
servies généralement avec du caviar, de la crème aigre et du beurre fondu,
consommés surtout pendant le carnaval.

Notons encore des glaces de viande qu'on fabrique en Sibérie où la volaille et le
gibier abondent, et qui entrent dans la préparation de beaucoup de sauces.
Comme entremets, différents blancs-mangers.

En résumé, la cuisine russe, qui à un certain nombre de points communs
avec la cuisine polonaise à laquelle elle a fait de nombreux emprunts, a une note
originale curieuse. Certains restaurants de Paris servent depuis quelque temps
des plats russes ou soi-disant tels.

\sk

\textit{Péninsule balkanigue}. — Dans la péninsule balkanique on trouve
l'ancienne cuisine byzantine, à base de légumes, de mouton et d'huile.

Les Turcs y ont importé de Perse l'usage du riz.

En Turquie, le mouton est apprêté de différentes manières : rôti, après avoir
été coupé en petits morceaux qu'on enfile sur une baguette ; en pilaf avec du
riz ; en ragoût avec des légumes variés tels que des courges coupées en petits
dés et des épinards ; en rissoles ; enfin en hachis plus ou moins aromatisés
accompagnés ou non de riz, le tout enrobé parfois dans des feuilles de vigne,
de figuier, de chou, ou encore dans des aubergines, et braisé dans du bouillon.
Cette dernière préparation prend le nom de \textit{dolma}, que l'on donne aussi
du reste à des feuilles de vigne, de figuier, de chou, ou à des aubergines
farcies de riz et d'oignons hachés, le tout cuit dans de l'huile d'olive. Comme
autres plats, je citerai le maquereau farci d'un hachis de chair de maquereau,
d'oignons, de riz et de raisins de Corinthe ; les moules frites ; les moules
garnies dans leurs coquilles d'un hachis d'oignons, de riz et de raisins secs ;
les aubergines, les courges et les paprikas farcis ; les \textit{beurek}, pâtés
feuilletés aux fromages turcs \textit{cacher} et \textit{misitra}, fourrés de
viandes hachées ; le \textit{kalaïf}, sorte de vermicelle sucré ; le
\textit{kourabis}, gâteau sec à la vanille ; le \textit{kaïmak}, crème qu'on
peut servir accompagnée de différentes substances, par exemple de coings ; le
\textit{mahaleli}, entremets composé de farine de riz, de lait et de poudre de
cannelle, aromatisé à l'eau de roses ; enfin le
\textit{locoum}\footnote{Corruption du mot turc « rahat el halkoum », qui veut
dire le repos, le bien-être du gosier.}, pâte gommeuse parfumée, qui est la
renommée du fameux Hadji-Bekir de Stamboul.

Dans la cuisine grecque moderne mentionnons l'agneau en pilaf avec piments,
gombos\footnote{Fruit mucilagineux de l'\textit{Hibiscus eseulentus}, plante de
la famille des Malvacées, servi comme légume.} et raisins de Smyrne ; le ragoût
de mouton au riz et aux tomates et les ravioli au hachis de mouton.

\sk

\textit{Autriche-Hongrie}. — L'Autriche-Hongrie est un
agrégat\footnote{Aujourd'hui cet agrégat est dissous,} hétérogène de pays et de
lambeaux de pays d'origines très différentes, dont la plupart ont gardé
l'empreinte du passé. Les sujets de l'Autriche-Hongrie d'aujourd'hui sont des
Allemands, des Hongrois, des Slaves, des Italiens, des Turcs : il n'est donc
pas surprenant que la cuisine soit très variée sur le territoire
austro-hongrois.

La base de la cuisine autrichienne proprement dite, la cuisine viennoise, est
caractérisée par des viandes blanches panées, cuites avec du beurre ; de la
charcuterie variée et des entremets farineux, généralement bons, grâce aux
excellentes farines de blé dur de Hongrie qui entrent dans leur préparation.

La note dominante de la cuisine hongroise est donnée par le paprika, poivre
du pays, base de la plupart des assaisonnements.

Comme plats intéressants citons : le potage aux pâtes fluides, les
\textit{wiener schnitzel}, escalopes de veau à la viennoise ; les
\textit{wiener rostbraten}, biftecks viennois cuits à l'étuvée avec beurre et
oignons, servis accompagnés d'une sauce à la glace de viande et à la crème
aigre ; les quenelles à la hongroise, avec une sauce tomate au paprika ; le
\textit{gulyàs} ou goulach préparé avec du bœuf ou du veau : le
\textit{pörkel}, goulach de poulet et de lard servi sur du riz ; les pieds de
veau aux anchois et au paprika ; le poulet au lard, à la crème aigre et au
paprika ; le faisan à la bohémienne, farci de chair de bécasse et de lard ; la
choucroute farcie, faite de feuilles de chou, non émincées, saumurées comme la
choucroute, et garnies d'un hachis de viandes et de riz ; les croustades de
fruits ; les flans de pommes ; les pains de marrons.

Nous ne pouvons passer sous silence les petits pains viennois appréciés dans le
monde entier, le café à la viennoise à la crème fouettée, et nous devons une
mention spéciale au jambon de Prague. Signalons comme poissons dans les régions
baignées par Le Danube le fogoch\footnote{Le fogoch, ou sandre, ou
brochet-perche est un beau poisson de rivière, qui peut atteindre
{\ppp1\mmm},{\ppp20\mmm}\textsuperscript{m} de longueur. Il se trouve dans
certaines rivières de l'Europe centrale et de l'Europe orientale, notamment
dans le Danube. On l'a acclimaté dans quelques lacs, en particulier dans le lac
de Constance en Allemagne et dans le lac Balaton en Hongrie.}, l'esturgeon, le
sterlet et, par conséquent, le caviar.

Il y a en Autriche-Hongrie des vins très agréables. En Hongrie, le plus renommé
est le vin de Tokay, dont certains crus se gardent très longtemps et acquièrent
en vieillissant des qualités extraordinaires.

La bière de Pilsen (Bohême) est appréciée dans le monde entier.

En résumé, la cuisine austro-hongroise synthétise à peu près toutes les cuisines
européennes, sauf la cuisine française.

\sk

\textit{États-Unis}. — Les cuisines des États-Unis de l'Amérique du Nord sont
incontestablement les plus grandes et les plus confortables du monde (largest
of the world), mais que dire de la cuisine américaine ?

Certes, on ne peut pas reprocher aux habitants des États-Unis de vivre pour
manger ; la plupart d’entre eux prennent à peine le temps d'avaler rapidement
leur nourriture : dans ces conditions il est bien difficile à l'art culinaire
de se développer dans le pays.

Les nouveaux venus sont absorbés par le \textit{struggle for life}, les
parvenus sont préoccupés sans cesse de la défense de leurs positions, et bien
rares sont dans le nouveau monde les philosophes qui prennent un peu de bon
temps. J'en ai cependant connu qui étaient de charmants compagnons et de fins
gourmets, mais c'était l'exception.

J'ai vu, dans \textit{Wall Street}, à l'heure de la Bourse, des milliardaires
déjeuner debout d'une tranche de \textit{corned beef} aux pickles et d'un
sandwich. Ils m'ont fait pitié ! J'ai vu, dans des restaurants américains de
l'Ouest, le spectacle suivant : un client arrive et s'assied ; une servante se
présente aussitôt et lui lit une longue litanie comprenant la nomenclature des
plats, toujours plus ou moins les mêmes du reste, et qui commence
invariablement par : \textit{roast beef, boiled beef, corned beef}, pour finir
par : \textit{corn, iced cream, cheese}. Le convive, sachant d'avance à quoi
s'en tenir, n'écoute pas, lit son journal et ne répond rien. On lui apporte
alors à la fois un certain nombre de petites assiettes contenant les différents
plats du menu et on les dispose sur la table autour de lui. Il pique sa
fourchette au hasard dans le tas, le plus souvent sans regarder et sans cesser
de lire, puis il la porte machinalement à sa bouche et avale ce qui est au
bout, en l'arrosant de ce qu'il trouve à sa portée : bière du pays,
généralement assez bonne, vin de Californie dont certains crus, comme le
\textit{zinfandel}, sont très buvables, \textit{gin, whisky and soda}, thé, le
tout glacé ; l'opération complète dure cinq minutes. C'est lamentable !

Je n'ai rien à dire de la table des grands hôtels, qui est celle de tous les
établissements de ce genre. Il est entendu que les hôtels américains sont les
plus vastes du monde, que leurs salles à manger sont les plus spacieuses de
l'univers, mais tout cela n'améliore pas les menus.

Cependant, il existe aux États-Unis des plats indigènes ayant une certaine
originalité : le potage à la tortue de terre : le potage aux huîtres et à la
crème ; le potage à la Penobscot aux gombos, aux huîtres et aux crabes mous ;
le \textit{chowder}, sorte de bouillabaisse aux \textit{clams}\footnote{ou
\textit{Mya arenaria}, mollusque bivalve de la famille des Myidés.} et à la
morue, avec du lard, des oignons, des pommes de terre et du lait ; les huîtres
panées et grillées, arrosées de jus de citron ; les beignets d'huîtres ; les
crabes mous grillés ou frits ; le homard à la Newbourg, excellent plat
new-yorkais ; le ragoût de tortue de mer ; le bœuf salé aux choux ; le ragoût
de pigeons au riz, aux tomates et aux gombos.

L'usage exclusif des boissons glacées constitue un véritable abus que beaucoup
d'estomacs ne peuvent supporter longtemps ; aussi l’une des spécialités
médicales les plus fructueuses aux États-Unis est celle des maladies de
l'estomac. Les Américains qui ont absorbé trop de glace finissent par boire de
l'eau tiède en mangeant ; c'est peut-être hygiénique, mais ce n'est guère
appétissant.

La passion des Américains pour les boissons glacées a développé
considérablement l'art de la préparation de ces breuvages aux États-Unis et
pour la confection des cocktails l'Amérique dame incontestablement le pion à la
vieille Europe\footnote{Il y a là un bel exemple de l'influence du milieu.
Alors que les cuisiniers européens perdent généralement la main aux États-Unis,
les \textit{bar-men} y sont extraordinaires et sortent littéralement de terre,
comme les généraux en France, sous Napoléon I\textsuperscript{er}. J'ai connu
à Omaha, dans le Nebraska, le prince des « cocktailiers » du temps, un
Français, venu aux États-Unis sans métier, qui, inspiré et dynamogénisé par le
milieu, avait en moins d'un an créé une cinquantaine de cocktails plus
remarquables les uns que les autres, et jouissait d'une réputation colossale de
l'Atlantique au Pacifique.}.

A titre de curiosité, je mentionnerai l'existence dans le pays de très
nombreuses sociétés dites de tempérance ; certains États même ont édicté contre
l'usage de l'alcool des lois draconiennes qui, à la vérité, ne sont pas
toujours observées\footnote{Il y a quelque vingt-cinq ans, je me rendais dans
le Far-West. Un matin (j'étais depuis {\ppp48\mmm} heures dans le train) comme
je prenais mon déjeuner dans le dining-car, le garçon, auquel je demandais de
la bière, refusa de m'en apporter, alléguant que nous traversions un État
puritain où régnaient des lois de tempérance très sévères qui l'exposaient
à des peines sérieuses s'il servait la moindre boisson fermentée pendant tout
le temps que le train serait sur le territoire dudit État. Devant mes
protestations, il me dit à l'oreille en souriant : « Commandez-moi tout haut du
thé ». Ce « tout haut » m'intrigua : je fis ce qu'il me conseillait et quelques
instants après il m'apportait sur un plateau une tasse, un sucrier vide et une
théière pleine de bière. Je jetai alors un regard sur les tables voisines. Tous
mes compagnons de voyage buvaient sans rire, dans des tasses, de la bière qu'on
leur avait servie dans des théières. Le principe était sauf, \textit{All
right} !}

\sk

\textit{Cuisine juive}. — Disons un mot de la cuisine juive.
\index{Cuisine juive}
\label{pg0045} \hypertarget{p0045}{}

De toutes les cuisines antiques, elle est la seule qui se soit perpétuée, dans
ses grands traits, à travers les âges\footnote{Les Juifs, tribu sémitique issue
d'Arabie, que l'histoire suit pas à pas depuis plus de
{\ppp3\mmm} {\ppp000\mmm} ans, ont mieux qu'aucun peuple conservé la pureté de
leur race, leurs mœurs et leurs coutumes.}. Elle semble fondée sur des règles
d'hygiène et d'économie domestique qui auraient été érigées en principes
religieux pour en assurer l'observation. C'était l'opinion des philosophes du
\textsc{xviii}\textsuperscript{e} siècle et c'était aussi celle de Renan. M.
Salomon Reinach soutient dans son « Orpheus »\footnote{Orpheus,
\textit{Histoire générale des religions}, par Salomon Reinach, chez Alcide
Picard, rue Soufflot.} que l'explication d'une prohibition alimentaire par des
raisons d'hygiène doit être considérée aujourd'hui comme un signe d'ignorance,
que si les Juifs religieux s'abstiennent de manger du porc, c'est parce que
leurs lointains ancêtres avaient pour \textit{totem}, c'est-à-dire pour
protecteur de clan, le sanglier ; que s'ils ne mangent pas certains poissons,
c'est également par superstition ; que s'ils ne travaillent pas le samedi,
c'est parce que le samedi est considéré chez eux comme un mauvais jour au même
titre que certaines personnes considèrent comme un jour funeste le vendredi et,
en particulier, quand il tombe un {\ppp13\mmm}.

En tout état de cause, la défense de manger d'autres viandes que celles
d'animaux ruminants ayant les sabots fendus avait pour résultat d'interdire le
gibier, dont l'abus présente des inconvénients, et le porc, fréquemment
trichiné, qui passait pour donner la lèpre ; quant au principe de la viande
\textit{kocher}, consistant en ce que l'animal sacrifié pour l'alimentation
doit avoir été saigné et sa viande débarrassée des nerfs et des artères, il
assurait à cette viande une meilleure conservation\footnote{Les pratiquants
rigides salent encore la viande et la font dégorger dans de l'eau pour en
enlever le plus possible de sang.}. Le précepte des livres sacrés : « Tu ne
feras pas cuire l'agneau dans le lait de sa mère » avait comme conséquence de
laisser le lait de la brebis à l'agneau et il favorisait l'élevage. Ce
commandement, observé à la lettre par les Juifs Karaïtes\footnote{Secte qui
existe notamment en Russie et qui s’en tient exclusivement à la lettre des
livres sacrés.}, a été généralisé par les Juifs Talmudistes\footnote{Le Talmud
est le commentaire des livres sacrés. Les Juifs Talmudistes, qui constituent
l'immense majorité des juifs pratiquants, acceptent et appliquent les
commentaires du Talmud.} qui n'emploient le lait, la crème et le beurre que
pour la préparation des poissons dans les repas sans viande de boucherie et
pour la confection des pâtisseries ne devant être mangées qu'un certain nombre
d'heures avant ou après l'ingestion de substances carnées. Ils poussent même le
scrupule, afin d'éviter toute pollution, jusqu'à employer des casseroles
spéciales pour leur cuisson. Les crustacés et les poissons sans écailles
considérés comme favorisant le développement des maladies de la peau,
auxquelles la race juive est particulièrement sujette, sont également
interdits.

Il me parait difficile de ne pas reconnaître qu'il y a au moins dans toutes ces
interdictions une coïncidence remarquable entre les règles de l'hygiène et des
scrupules religieux.

Quoi qu'il en soit, il en résulte que la cuisine juive présente des caractères
spéciaux : pas de viandes saignantes, pas de canard à la rouennaise, pas de
civets, pas de boudin, pas ou peu de rôtis. Et en fait, la viande de boucherie
saignée à blanc ne peut guère être préparée que braisée ou en ragoût : ragoûts
à la graisse de bœuf ou à l'huile chez les pauvres, à la graisse d’oie chez les
gens aisés. Ces ragoûts sont toujours très relevés : le poivre, le piment, le
gingembre, l'ail et surtout l'oignon dominent dans tous les plats de la cuisine
juive.

De même que les viandes, les légumes sont accommodés à la graisse de bœuf,
à l'huile ou à la graisse d'oie.

Les plats de poisson sont souvent aromatisés par du gingembre et du safran.

Enfin, les entremets sucrés servis à la fin des repas, et parmi lesquels on
peut citer le \textit{matroch kouguel}, pudding au pain azyme, ne renferment de
laitage que dans les repas sans viande.

La cuisine juive primitive ne paraît s'être intégralement conservée que dans
certains plats du sabbat : ragoûts de viande et d'herbes, entre autres
pourpier, épinards, avec des pois chiches ; hachis de graisse d'oie et
d'oignons bourré dans la peau d'un cou d'oie et servi avec des carottes, appelé
\textit{zimmès} ; saucissons de bœuf et de riz ; hachis de viande avec des
œufs ; pieds de bœuf aux lentilles, aux pois, etc. Tous ces mets sont cuits
très lentement, à l'étouffée, dans des marmites garnies de substances
isolantes, de carpettes de laine : on les met sur le feu le vendredi soir pour
qu'ils soient prêts sans qu'on ait à y toucher le lendemain, jour du repos
hebdomadaire, où tout travail est formellement interdit\footnote{Certains juifs
ne mangent même le samedi que des aliments cuits le vendredi avant minuit, où
des aliments crus.}.

La plupart des juifs pratiquants, éparpillés sur la Terre, ont adapté leurs
principes rituels à la cuisine des pays où ils ont planté leur tente ; par là,
ils ont donné naissance aux diverses cuisines juives : russe, allemande,
alsacienne, algérienne, etc., car il y en a à peu près autant que de pays, et
toutes ont un cachet commun.

Dans la cuisine juive alsacienne, on peut citer la choucroute garnie ; dans la
cuisine juive espagnole, la \textit{quesada}, pâté au fromage
\textit{cachecaval} et aux aubergines rôties, ou au fromage blanc, et le
\textit{boreka}, pâté au fromage cachecaval et aux épinards ; parmi les plats
du sabbat de la cuisine juive algérienne mentionnons le \textit{tefina}, le
\textit{bobinel}, saucisson de bœuf cuit à l'étouffée, le \textit{méguina},
hachis de bœuf aux œufs cuit à l'étouffée, et les pieds de bœuf aux légumes
tels que riz, haricots, etc.

\sk

\index{Cuisines exotiques} \textit{Cuisines exotiques}. Je passerai rapidement
sur les cuisines par trop bizarres. Faute de grives on mange des merles, et
certains plats qui paraissent invraisemblables ne sont que le résultat d'une
pénurie de ressources. Pendant le siège de Paris, en {\ppp1\mmm} {\ppp870\mmm},
n'avons-nous pas mangé certaines substances fort peu appétissantes et bien
vaguement alimentaires ? Je ne parlerai pas des mangeurs de larves de vers et
d'insectes, pas plus que des géophages ; je me contenterai d'indiquer quelques
analogies de goût pouvant présenter un certain intérêt et quelques plats
exotiques plus ou moins curieux, souvenirs personnels ou souvenirs de camarades
qui, comme moi, ont parcouru le monde en dehors des sentiers battus.

Comme analogie de goût, on peut rapprocher la chair du chien de celle de
l'agneau. En Chine, les chiens d'une variété spéciale, sans poils ou à poils
rares, sont engraissés pour la boucherie et leurs gigots sont très recherchés.
De tous les singes mangeables, car certaines espèces sont trop fortement
musquées, le meilleur, à mon avis, est le macaque\footnote{Genre de singes de
la famille des Cercopithécidés.} dont la chair ressemble à celle de l'écureuil,
ce qui n'a rien de surprenant, ces deux animaux vivant exclusivement de graines
et de baies. Le kanguroo à un goût de lapin ; l'agouti\footnote{Mammifère
rongeur.}, l'acouchi\footnote{Mammifère rongeur.}, le pac\footnote{Mammifère
rongeur.}, un goût intermédiaire entre celui du lièvre et celui du sanglier ;
le patira\footnote{Cochon sauvage de l'Amérique du Sud.} ressemble au
sanglier ; le chameau a un goût de bœuf avec une odeur de bouc ; la viande de
l'alpaca tient à la fois de celle de l'âne et de celle du mouton ; le
hocco\footnote{Gros oiseau gallinacé de la famille des Cracidés. Dans sa
poitrine épaisse et charnue, on taille sans peine des escalopes de {\ppp2\mmm}
centimètres d'épaisseur.} fait songer au dindon ; la chair de
l'iguane\footnote{Genre de lézard de la famille des Iguanidés, atteignant
jusque près de 2 mètres de longueur, se nourrissant de végétaux et d'insectes.}
rappelle celle du poulet ; le bison est analogue au cerf et les muscles d'acier
des grands fauves fournissent une viande dure à odeur forte.

Les œufs d'iguane, qui se présentent en chapelets de {\ppp40\mmm}
à {\ppp50\mmm}, recouverts d'une enveloppe souple et paraissant ne pas avoir de
blanc, sont très fins ; les œufs de tortue de terre, à structure granuleuse,
gros comme de belles noisettes, font bonne figure dans les karis ; quant aux
œufs d'autruche, extrêmement volumineux\footnote{Au cours d'un voyage en
Patagonie, à huit, nous eûmes beaucoup de mal pour manger en entier une
omelette faite avec un seul de ces œufs qui pesait plus de
{\ppp1\mmm} {\ppp500\mmm} grammes.} et assurément mangeables, ils manquent
totalement de finesse.

L'igname\footnote{Racine alimentaire féculente de la famille des
Dioscoréacées,} et la patate\footnote{Racine alimentaire féculente de la
famille des Convolvulacées.} rappellent la pomme de terre, avec une saveur plus
ou moins sucrée ; le chou-palmiste\footnote{Bourgeon de l'Arec olcracea, arbre
de la famille des Palmiers.}, excellent en salade, a un goût de fond
d’artichaut mâtiné de noisette et les vers du chou-palmiste (rhynchophorus),
très recherchés des gourmets, concentrent en eux ce goût.

Les plats les plus intéressants de l'Amérique du Sud sont, comme plats
créoles : la pimentade ; le potage gras aux gombos frais ; le potage maigre aux
gombos secs ; le bouillon wara, qu'on prépare avec du poisson boucané, de la
morue salée, des crabes et des graines rouges du palmier wara :
l'agami\footnote{Oiseau de la classe des gallinacés, genre échassier, famille
des Psophiidés.} braisé au riz ; le kalalou, ragoût de gombos, de feuilles de
chou caraïbe, de pourpier et de petits concombres ; les gombos à la crème ; les
gombos à l’étouffée avec du petit salé ; la salade de chou-palmiste ;
l’omelette à l'avocat\footnote{Fruit de l'avocatier commun, arbre de la famille
des Cinnamomées, genre persca. Ce fruit est une grosse baie plus ou moins
ovoïde, dont la pulpe, comestible, est également connue sous le nom de beurre
végétal.} ; les avocats en salade ; les bananes frites ; comme autres plats :
la \textit{casuela}, potage à base de bouillon de poule ou de mouton avec des
légumes et des tranches d'épis de maïs ; le \textit{seviche}, excellent
hors-d'œuvre péruvien, préparé avec un poisson du Pacifique appelé
\textit{corbina}, qui n'a que très peu d'arêtes, et qu'on mange cru, coupé en
petits morceaux assaisonnés avec du piment et de l'oignon, après l'avoir fait
mariner pendant {\ppp24\mmm} heures dans du jus d'oranges aigres ; la
\textit{carna}, salade de pommes de terre du Pérou\footnote{Les pommes de terre
du Pérou, pays de leur origine, sont les meilleures qui soient au monde.},
bouillies et écrasées, garnie d'écrevisses, de feuilles de laitue, d'œufs durs
coupés en tranches et assaisonnée à l'huile ; la \textit{mazamorra}, entremets
péruvien très original ayant une consistance gélatineuse, à base de maïs rouge,
de lait et d'œufs, saupoudré de cannelle ; la \textit{humita}, croquette de
farine de maïs enrobée dans une feuille fraîche de la plante, qu'on fait frire
dans du beurre ; les \textit{empanadas}, sorte de rissoles de viande et d'œufs
durs très appréciées notamment dans la République Argentine et en Bolivie : la
\textit{feijoada}, espèce de cassoulet préparé avec des haricots rouges et de
la viande boucanée, plat national du Brésil.

L'Asie, pays du riz et des épices, nous donne le \textit{kari} indien et le riz
à la persane au beurre et au citron, servis couramment aujourd'hui sur nos
tables ; le ragoût de mouton à la persane avec des pruneaux, des amandes et du
citron ; le \textit{kelap} persan, rôti de mouton coupé en tranches, enfilées
ensuite sur une baguette, séparées les unes des autres par des feuilles
fraîches de menthe, de basilic et de laurier, le tout arrosé de graisse de
mouton ; le saucisson de mouton, qu'on prépare également en Perse ; le
\textit{koubbé} syrien, pâté au hachis de mouton avec du blé concassé ; les
carottes farcies de mouton haché, de riz ou de graines de pin frites, le tout
assaisonné avec oignon, ail, tomate, jus de citron, menthe, sel et poivre, qui
sont très estimées en Syrie ; le potage chinois aux nids d’hirondelles, qu'on
peut s'offrir aisément de nos jours à Paris ; les œufs de cent ans\footnote{Les
œufs dits « de cent ans » sont des œufs qui en réalité ont rarement plus de
quelques années. Pour les amener au point voulu, on les met dans de la chaux
éteinte avec des herbes aromatiques et on les laisse ainsi pendant un temps
plus ou moins long, jamais inférieur à six semaines. À la longue le jaune se
liquéfie et prend une couleur vert foncé, le blanc se coagule et se colore en
vert clair. Le produit, qui a une forte odeur d'hydrogène sulfuré à laquelle on
s'habitue, est servi comme hors-d'œuvre. Il a un goût rappelant celui du
homard.}, très appréciés par les fils du Céleste Empire : la soupe coréenne au
\textit{cantjang}\footnote{Le cantjang est le produit de la fermentation du
haricot au contact de l'eau.} ; les ailerons de requin frits, prisés surtout en
Indo-Chine et le \textit{kadjiuri}, ragoût indien de poisson au kari. 

L'Afrique nous fournit le \textit{couscous} et le \textit{méchoui}, mets
algériens. 

L'Australie nous donne le potage à la queue de kanguroo.

\index{État actuel de la Gastronomie}
\section*{\centering  État actuel de la Gastronomie.}

Disons en terminant un mot sur l'état actuel de la gastronomie et cherchons à
prévoir son avenir.

Tout en m'efforçant d'éviter de louanger les temps passés, travers dans lequel
on tombe facilement, il me semble franchement qu'au point de vue gastronomique,
comme à beaucoup d'autres, nous sommes en train de traverser une crise.

L'élevage, les procédés modernes de culture, la préparation des conserves ont
certainement augmenté la quantité de nourriture disponible : le développement
des moyens de transport, l'emploi du froid ont permis de répandre cette
nourriture partout, et la famine, cet horrible fléau, est désormais impossible
dans les pays civilisés, à moins d'un cataclysme. Mais si, au point de vue
général, ces conditions nouvelles de la vie ont incontestablement une influence
heureuse, en est-il de même au point de vue purement gastronomique ?

Aujourd'hui les éleveurs, en gavant les animaux, produisent couramment des
viandes trop grasses ; la culture intensive modifie le plus souvent dans un
sens défavorable la qualité des produits du sol. Il nous suffira de citer comme
exemple la pomme de terre que l'on ne peut plus avoir parfaite qu'en la
cultivant tout exprès et sans la forcer, dans des terrains sablonneux, comme on
le faisait autrefois. Les châssis et les serres fournissent en toute saison des
légumes et des fruits merveilleux d'aspect, mais dépourvus de saveur : on n'est
pas encore parvenu à remplacer le soleil. L'industrie des conserves provoque
l'accaparement des produits alimentaires naturels, frais, au moment où ils sont
le meilleur marché ; les chemins de fer drainent de partout ce qu'il y a de
meilleur, au profit de consommateurs souvent incapables de l'apprécier et ils
en privent les habitants des pays producteurs, parmi lesquels se recrutaient
autrefois les gourmets les plus raffinés. On cueille les fruits avant leur
maturité pour pouvoir les transporter au loin, de sorte que peu de personnes
sont actuellement à même de manger des fruits vraiment à à point ; on n'a plus
de lait à la campagne ; il devient difficile de se procurer du poisson au bord
de la mer ; il est presque impossible d'obtenir un bon bifteck dans un pays
d'élevage ; en un mot nous vivons un peu comme dans le manoir a l'envers.

La falsification des aliments, très ancienne à la vérité puisque les Romains
s'en plaignaient déjà, mais qui se pratiquait jadis sur une échelle
relativement petite, constitue aujourd'hui, par suite des progrès de la chimie,
une branche de l'industrie : les procédés à employer pour atteindre ce but sont
discutés dans des congrès officiels et leurs auteurs, au lieu d'être pendus,
sont décorés !

Il devient incontestablement difficile de bien manger : cependant la chose est
encore possible, mais plus que jamais il est indispensable de s'occuper
soi-même de sa nourriture. En province, dans certains milieux où l’on ne se
désintéresse pas de la question, on sait encore faire bonne chère. On pense
à la cuisine ; on discute d'avance les menus : on s'adresse pour chaque produit
à des fournisseurs que l’on connaît et qui savent eux-mêmes à qui ils ont
affaire ; enfin, la préparation de tous les plats est l'objet des soins les
plus minutieux.

Mais à Paris, où l'on vit trop vite, où l'on est toujours pressé, peu de gens
consentent à consacrer quelques moments à ces questions ; aussi l'art culinaire
y est manifestement en décadence. Pourtant il semble que bien manger devrait
intéresser tout le monde, car personne n'oserait soutenir qu'il soit
indifférent de consommer des aliments bien ou mal préparés. La gastronomie
s'adresse à toutes les classes de la société et il n'est nullement nécessaire
d'avoir de la fortune pour se bien nourrir. Le repas le plus simple, quelque
modeste qu'il soit, peut être meilleur qu'un repas très coûteux, et l’on aura
toujours bien mangé si ce qu'on a mangé était de bonne qualité et bien préparé.

Malheureusement, ce qu'on recherche avant tout aujourd'hui c'est paraître. Le
modeste bourgeois d'autrefois, recevant des amis à sa table, ne leur donnait
pas plus de trois plats, simples mais soignés, préparés sous la direction
effective et jalouse de la maîtresse de maison. Le bourgeois de nos jours se
croirait déshonoré s'il ne présentait pas à ses convives des menus somptueux,
au moins en apparence, qu'il est hors d'état de faire exécuter chez lui.

Aussi commande-t-il ses repas priés au dehors, chez des entrepreneurs qui les
lui envoient tout prêts, avec des domestiques d'occasion pour les servir.

Les aigrefins peuvent donner à dîner dans des appartements vides, loués
à l'heure pour la circonstance : des agences leur fournissent à forfait la
nourriture, la boisson, la vaisselle, le linge, la valetaille et, s'ils le
désirent, elles leur procurent même, moyennant un petit supplément, quelques
invités décoratifs et décorés destinés à impressionner le gogo naïf, auquel le
mirage d'un intérieur familial cossu inspire toute confiance. Paraître, tout
est là !

Quant aux parvenus, ils rivalisent de faux luxe. Pour avoir l'air de ne pas
y regarder, ils font bourrer tous les plats de truffes et de foie gras, de
sorte que tout finit par avoir le même goût, et bien des dîners, dans des
maisons où l'on devrait pouvoir manger convenablement, deviennent aussi odieux
que des repas de table d'hôte auxquels, d'autre part, ils ressemblent souvent
par l'assemblage hétéroclite des invités.

\index{Avenir de la Gastronomie}
L'une des industries les plus florissantes aujourd'hui est celle de la
confection de mets à emporter. Partout on vend des plats tout faits et nombre
de femmes ont une tendance fâcheuse à se désintéresser de leur intérieur. Les
unes ont l'excuse des nécessités de la vie, qui les obligent à travailler
dehors ; d'autres courent les magasins et les \textit{five o' clock} à la
recherche du bonheur. L'idéal pour beaucoup d'entre elles serait la maison
\textit{up to date}, avec eau, gaz et nourriture à tous les étages, ce qui
permettrait de supprimer les cuisines, en attendant la fameuse pilule
synthétique entrevue par certains savants.

En ce qui concerne les établissements publics, on voit se multiplier les gargotes
à prix fixe ; les bons restaurants se transforment ou ferment successivement leurs
portes, et je serais véritablement embarrassé pour citer à Paris plus de quatre ou
cinq maisons où l'on soit assuré d'être toujours bien traité à tous égards.

L'internationalisme mal compris se développe d'une façon inquiétante, et ses
progrès, déplorables à bien des points de vue, sont désastreux au point de vue
gastronomique ; si l'on n'y prend garde, ils auront bientôt amené à un même
niveau, niveau peu élevé, la cuisine de tous les pays.

Au commencement du siècle dernier, un grand maître de l'Université était tout
fier de pouvoir dire : « Aujourd'hui, à cette heure, tous les élèves de toutes
les classes de seconde de tous les lycées de France font le même thème grec ».
Les syndicats internationaux d'aubergistes qui nourrissent les voyageurs des
deux hémisphères soumis à leur régime, paraphrasant le mot du ministre, peuvent
dire : « Du Far-West à l'extrême-orient, du pôle nord au pôle sud, depuis le
{\ppp1\mmm}\textsuperscript{er} janvier jusqu'à la Saint-Sylvestre, tous nos
clients font les mêmes repas ».

Et, en effet, que ce soit en bateau, en chemin de fer ou dans les hôtels,
partout ces malheureux sont condamnés à la même invraisemblable barbue sauce
hollandaise, au même aloyau braisé jardinière (quel aloyau et quelle
jardinière !), à la même inévitable poularde (de Bresse, naturellement).

Quand on pense que des gens paraissant à peu près équilibrés, dont une partie
voyagent soi-disant par plaisir, consentent à absorber tous les jours de
pareilles atrocités, c'est à désespérer du genre humain.

Je veux croire cependant que ce n'est qu'une crise que nous traversons et
j'espère, sinon un réveil général du goût, ce qui serait trop beau, au moins un
soulèvement des estomacs, comme au temps de Lycurgue.

En attendant cette révolution pacifique, que les gastronomes ne se découragent
pas ; leurs efforts ne seront pas stériles. Orientés avec méthode, ces efforts
persévérants finiront par faire de l'art culinaire purement expérimental, tel
qu'il est aujourd'hui, une science exacte. En précisant dans des formules
rigoureuses les connaissances que l’on possède, on fait plus que perpétuer des
recettes, on accumule des matériaux d'où se dégageront un jour les lois de la
gastronomie, qui seront la base indestructible de la \textit{Science du Bon}.
