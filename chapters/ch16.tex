\section*{\centering Matelote au vin blanc.}
\addcontentsline{toc}{section}{ Matelote au vin blanc.}
\index{Matelote au vin banc}
\index{Anguilles en matelote}
\index{Matelote d'anguilles}
\index{Carpe en matelote}
\index{Matelote de carpe}

Pour six personnes prenez :

\medskip

\footnotesize
\begin{longtable}{rrrp{16em}}
  1 000 & grammes & de & poisson : anguille d'eau douce\footnote{Anguilla vulgaris ;
                                                        famille des Murénidés.},
                                                        carpe\footnote{Cyprinus carpus ;
                                                        famille des Cyprinidés.} ou
                                                        brème\footnote{Cyprinus brama ;
                                                        famille des Cyprinidés.}, par exemple,            \\
    500 & grammes & de & vin blanc sec\footnote{Lorsque le vin employé est âpre, il est bon
                                                        de l'adoucir avec un peu de sucre.},              \\
    250 & grammes & d' & eau,                                                                             \\
    250 & grammes & de & champignons de couche,                                                           \\
    250 & grammes & de & laitances et d'œufs de poissons,                                                 \\
    125 & grammes & de & beurre,                                                                          \\
    125 & grammes & d' & oignons épluchés,                                                                \\
     50 & grammes & de & fine champagne,                                                                  \\
     30 & grammes & de & farine,                                                                          \\
        &         & 12 & petites tranches de pain,                                                        \\
        &         &  6 & écrevisses,                                                                      \\
        &         &    & fumet de poisson ou, à défaut, glace de viande,                                  \\
        &         &    & bouquet garni,                                                                   \\
        &         &    & sel et poivre.                                                                   \\
\end{longtable}
\normalsize

Coupez le ou les poissons en tronçons.

Faites un roux avec du beurre et la farine, ajoutez le bouquet garni et les
oignons entiers ; mouillez avec l’eau, le vin et la fine champagne, salez et
poivrez au goût ; laissez cuire pendant une heure.

Mettez ensuite le poisson ; laissez-le cuire plus ou moins longtemps, suivant
sa dureté et la grosseur des morceaux. Un quart d'heure avant la fin de la
cuisson, ajoutez les champignons et corsez avec du fumet de poisson ou de la
glace de viande.

Au dernier moment, faites frire les petites tranches de pain dans le reste du
beurre.

Enlevez le bouquet garni, passez la sauce si vous n'aimez pas trouver les
oignons, dressez la matelote sur un plat, garnissez avec les croûtons, les
écrevisses, les laitances et les œufs de poissons cuits à part et servez.

\bigskip

\sk

\bigskip

\index{Brochet en matelote}
\index{Matelote de brochet}
On peut préparer de même la matelote de brochet : c'est exquis.

\section*{\centering Matelote au vin rouge.}
\addcontentsline{toc}{section}{ Matelote au vin rouge.}
\index{Matelote au vin rouge}

Pour six personnes prenez :

\medskip

\footnotesize
\begin{longtable}{rrrp{16em}}
  1 000 & grammes   & de & poisson : anguille de rivière, brême, brochet ou carpe,                        \\
    500 & grammes   & de & têtes, arêtes ou parures de poissons,                                          \\
    450 & grammes   & de & bon vin rouge de Bourgogne,                                                    \\
    200 & grammes   & de & champignons,                                                                   \\
    250 & grammes   & de & laitances et œufs de poissons,                                                 \\
    125 & grammes   & de & beurre,                                                                        \\
     75 & grammes   & d' & oignons ciselés,                                                               \\
     30 & grammes   & de & farine,                                                                        \\
        & 1/2 litre & d' & eau,                                                                           \\
        &           & 12 & petites tranches de pain,                                                      \\
        &           &  6 & écrevisses,                                                                    \\
        &           &    & bouquet garni composé de 10 grammes de persil, 1/2 feuille de                  \\
        &           &    & laurier et quelques brindilles de thym,                                        \\
        &           &    & ail, au goût,                                                                  \\
        &           &    & sel et poivre.                                                                 \\
\end{longtable}
\normalsize

Nettoyez et pelez les champignons ; réservez les pelures.

Coupez le ou les poissons en tronçons.

\index{Fond de poisson au vin rouge}
Préparez un fond de poisson en faisant cuire ensemble, à l'étuvée, déchets de
poissons, oignons, ail, pelures de champignons, bouquet garni, sel, poivre, vin
et eau pendant une heure environ. Passez-le au tamis.

Faites un roux avec du beurre et de la farine ; mouillez avec le fond, mettez
le poisson, laissez cuire. Un quart d'heure avant la fin, ajoutez les
champignons.

Dressez la matelote sur un plat ; garnissez avec les tranches de pain, dorées
dans le reste du beurre, les écrevisses, les laitances et les œufs de poissons
cuits à part, puis servez.

\section*{\centering Matelote à l'ail.}
\addcontentsline{toc}{section}{ Matelote à l'ail.}
\index{Matelote à l'ail}
\index{Anguilles en matelote}
\index{Matelote d'anguilles}

Pour six personnes prenez :

\footnotesize
\begin{longtable}{rrrp{16em}}
    750 & grammes & de  & vin blanc,                                                                      \\
    250 & grammes & de  & champignons de couche,                                                          \\
    250 & grammes & de  & laitances et d'œufs de poissons,                                                \\
    110 & grammes & de  & beurre,                                                                         \\
     20 & grammes & d'  & ail,                                                                            \\
     15 & grammes & de  & persil haché,                                                                   \\
     15 & grammes & de  & farine,                                                                         \\
      5 & grammes & de  & sel blanc,                                                                      \\
      3 & grammes & de  & poivre,                                                                         \\
        &         & 12  & petites tranches de pain,                                                       \\
        &         &  6  & écrevisses,                                                                     \\
        &         &  1  & anguille vivante pesant 1 kilogramme environ,                                   \\
        &         & 1/2 & feuille de laurier,                                                             \\
        &         &     & fumet de poisson ou, à défaut, glace de viande.                                 \\
        &         &     & thym.                                                                           \\
\end{longtable}
\normalsize

Tuez l'anguille, dépouillez-la, videz-la, coupez-lui la tête et débitez le
reste en tronçons.

\index{Court-bouillon pour anguilles}

Faites bouillir pendant une heure dans le vin la tête d'anguille, l'ail,
{\ppp10\mmm} grammes de persil, du thym, le laurier, le sel et le poivre.

Passez ce court-bouillon, mettez dedans les tronçons d'anguille, laissez-les
cuire pendant une demi-heure. En même temps, faites cuire à part les
champignons dans {\ppp50\mmm} grammes de beurre.

Liez la sauce avec la farine maniée avec le reste du beurre, ajoutez du fumet
de poisson ou de la glace de viande, laissez cuire encore un peu. Un instant
avant la fin, mettez les champignons.

Dressez les tronçons d'anguille sur un plat, masquez-les avec la sauce,
disposez les champignons autour, saupoudrez de persil, achevez de garnir le
plat comme précédemment avec les croûtons frits, les écrevisses, les laitances
et les œufs de poissons cuits à part et servez.

\section*{\centering Matelote blanche.}
\addcontentsline{toc}{section}{ Matelote blanche.}
\index{Matelote blanche}

Prenez comme poissons, seuls ou mélangés, une anguille de rivière, une carpe ou
une tanche\footnote{Tinca vulgaris, famille des Cyprinidés.}. Coupez le ou les
poissons en tronçons, passez-les dans du beurre, puis retirez-les.

Faites blondir de la farine dans du beurre, mouillez avec du vin blanc, de
préférence du bordeaux un peu sec, ajoutez une pointe d'échalote, de l'oignon,
du sel et du poivre au goût, laissez cuire ; mettez ensuite le poisson et
laissez-le pendant le temps nécessaire pour qu'il soit cuit à point.

Retirez-le et tenez-le au chaud.

Montez la cuisson à la crème et acidulez légèrement avec du jus de citron.

Servez le poisson masqué avec la sauce et entouré de croûtons frits dans du
beurre.

\section*{\centering Poissons d'eau douce à la juive.}
\addcontentsline{toc}{section}{ Poissons d'eau douce à la juive.}
\index{Poissons d'eau douce à la juive}
\index{Carpe à la juive}

Il existe plusieurs façons de préparer le poisson à la juive ; en voici trois
ayant chacune sa caractéristique.

\medskip

A. — Pour six à huit personnes prenez :

\footnotesize
\begin{longtable}{rrrp{16em}}
    125 & grammes & d' & oignons,                                                                         \\
    125 & grammes & d' & échalotes,                                                                       \\
     50 & grammes & de & sel gris,                                                                        \\
     45 & grammes & d' & huile blanche,                                                                   \\
     30 & grammes & de & persil,                                                                          \\
     30 & grammes & de & farine,                                                                          \\
      5 & grammes & d' & ail,                                                                             \\
        & 1 litre & d’ & eau filtrée,                                                                     \\
        &         & 20 & grains de poivre,                                                                \\
        &         &  1 & brochet vivant ou 1 carpe laitée vivante,
                         pesant 1 kilogramme 1/2 à 2 kilogrammes,                                         \\
        &         &    & gingembre,                                                                       \\
        &         &    & sel blanc,                                                                       \\
        &         &    & poivre fraîchement moulu.                                                        \\
\end{longtable}
\normalsize

Tuez le poisson, grattez-le, lavez-le bien avant de le vider, essuyez-le,
entaillez le ventre et sortez avec précaution tout l'intérieur. Mettez à part
le foie et la laitance, recueillez le sang. Coupez le poisson en tronçons de
{\ppp5\mmm} centimètres d'épaisseur environ, mettez-le, y compris la tête dont
vous aurez enlevé les branchies, dans un vase d'assez grandes dimensions ;
saupoudrez avec le sel gris et laissez en contact pendant six heures en été et
douze heures en hiver. Au bout de ce temps, sortez le poisson et secouez-le
pour enlever l'excès de sel.

Hachez très fin, séparément, les oignons et les échalotes puis, ensemble, le
persil et l'ail.

\index{Court-bouillon à la juive}
Mettez l'huile dans une casserole en cuivre étamé, chauffez jusqu'à ce qu'elle
fume : ajoutez d'abord l'oignon haché et tournez sans interruption jusqu'au
moment où il commencera à jaunir légèrement, puis l’échalote et continuez
à tourner jusqu'à ce que l'ensemble ait une belle couleur jaune clair, enfin la
farine, en tournant toujours jusqu'à ce que le tout devienne également jaune
clair. Mouillez avec l'eau filtrée, faites bouillir, mettez le persil, l'ail,
le poivre en grains, et un peu de gingembre, au goût.

Plongez le poisson dans ce court-bouillon qui devra le couvrir juste. S'il est
insuffisant, ajoutez de l'eau bouillante.

Faites cuire, en casserole découverte, à gros bouillons, pendant {\ppp30\mmm}
à {\ppp35\mmm} minutes, goûtez pendant la cuisson et ajoutez, s'il y a lieu, le
sel et le poivre qui pourraient manquer. Retirez la casserole du feu et
couvrez-la pendant un quart d'heure avec un plat contenant de l'eau froide,
afin de raffermir le poisson. Découvrez ensuite en évitant qu'il tombe de l'eau
sur votre préparation ; puis dressez les morceaux sur un plat long de manière
à reconstituer le poisson. Écrasez alors le foie et la laitance, ajoutez-les
avec le sang au jus de cuisson, chauffez ensemble sans laisser bouillir et
versez sur le poisson. Laissez reposer suffisamment pour que la sauce se prenne
en gelée.

Au moment de servir, décorez avec une garniture de persil.

Ce plat, très relevé et très original, n’est pas toujours apprécié à sa valeur
la première fois qu'on y goûte.

\sk

\index{Brochet à la juive}
On peut préparer de même une truite saumonée.
On peut également ajouter une darne de saumon à une carpe ou à un brochet.

\sk

\index{Carpe à la juive}
B. — Prenez les proportions de poissons indiquées précédemment et commencez par
les préparer et les saler comme il a été dit plus haut ; puis, suivant la
grosseur de l'animal, et en laissant intactes la peau ainsi que l'arête du
milieu, levez dans chaque tranche de poisson, près du dos et de chaque côté de
l’arête, des rondelles de chair de {\ppp2\mmm} à {\ppp4\mmm} centimètres de
diamètre.

Hachez très fin {\ppp125\mmm} grammes d'oignons, hachez aussi les rondelles de
chair extraite, réunissez le tout et continuez à hacher, ajoutez ensuite un peu
de mie de pain tamisée et légèrement mouillée de lait, un œuf entier, du sel,
du poivre, un peu de beurre fondu ; triturez le tout jusqu'à l'obtention d’une
farce fine.

Emplissez avec cette farce les vides existant dans les tranches de poisson par
le fait de l'enlèvement des rondelles.

\index{Court-bouillon à la juive}
Préparez dans une casserole, en cuivre étamé et munie d'un couvercle, un
court-bouillon composé d'eau, de {\ppp2\mmm} carottes émincées, {\ppp3\mmm}
blancs de poireaux coupés en morceaux, {\ppp1\mmm} racine de persil,
{\ppp2\mmm} ou {\ppp3\mmm} oignons coupés en rondelles, sel, poivre au goût et
laissez cuire pendant une demi-heure. Évitez en particulier l'excès de sel.
Mettez dedans le poisson morceau par morceau, les morceaux de tête les premiers
au fond. Faites cuire à feu vif, ajoutez toutes les {\ppp5\mmm} ou {\ppp10\mmm}
minutes une petite cuillerée d'eau froide. La cuisson totale doit durer une
heure. Une demi-heure avant de retirer le poisson de la casserole, aromatisez
avec un peu d'infusion de safran si vous l'aimez.

Dressez les morceaux du poisson sur un plat et servez avec une réduction de la
cuisson passée au tamis.

Le poisson ainsi préparé peut être mangé chaud ou froid, à volonté.

C'est un plat très convenable qui a incontestablement moins de caractère que
le précédent, mais qui plaît davantage. Il diffère de ce qui se mange couramment
et il est intéressant à ce point de vue.

\sk

\index{Carpe froide, à la juive}
\index{Carpe à la juive}

C. — Enfin, voici une formule de carpe froide à la juive, de la cuisine
alsacienne.

\medskip

Pour six personnes prenez :

\footnotesize
\begin{longtable}{rrrp{16em}}
     45 & grammes & d' & huile d'olive,                                                                   \\
     30 & grammes & d' & oignons,                                                                         \\
     15 & grammes & de & farine,                                                                          \\
     10 & grammes & d' & ail,                                                                             \\
      5 & grammes & de & persil haché,                                                                    \\
        &         &  1 & belle carpe laitée pesant 1 kilogramme 1/2 environ,                              \\
        &         &    & bouillon de poisson, préparé comme il est dit 
                         \hyperlink{p0218}{p. \pageref{pg0218}},                                          \\
        &         &    & sel et poivre.                                                                   \\
\end{longtable}
\normalsize

Écaillez, videz et lavez le poisson.

Faites revenir les oignons ciselés fin dans l'huile ; lorsqu'ils seront bien dorés,
ajoutez l'ail haché fin et la farine ; laissez prendre couleur.

Mettez le tout avec le poisson dans une casserole assez grande pour le contenir
entier ; mouillez suffisamment avec du bouillon de poisson pour que la carpe
baigne dedans aux trois quarts ; assaisonnez avec sel et poivre. Laissez cuire
pendant une demi-heure environ.

Dressez la carpe sur un plat ; concentrez la cuisson ; passez-la et masquez-en
la carpe. Saupoudrez avec le persil et laissez refroidir.

\section*{\centering Paupiettes\footnote{Les paupiettes sont des tranches de
\index{Définition des paupiettes}
\index{Paupiettes (Définition des)}
chair de poisson ou de viande enduites d'une farce ou recouvertes d'une
garniture et roulées.} de poissons braisées.}

\addcontentsline{toc}{section}{ Paupiettes de poissons braisées.}
\index{Paupiettes de poissons braisées}

Prenez un beau brochet ou une belle sole, par exemple ; nettoyez le poisson,
levez-en les filets, salez-les et, suivant leur dimension. employez-les entiers
ou coupez-les en deux.

\index{Farce pour poissons}
Faites dorer des oignons hachés dans du beurre ; préparez ensuite une farce avec
les oignons revenus, de la chapelure, des cèpes secs détrempés et coupés fin, du
sel, du poivre, liez-la avec un jaune d'œuf.

Étendez sur chaque filet une partie de cette farce, roulez en paupiettes,
ficelez, enduisez-les d'un mélange de chapelure et de jaune d'œuf, faites
revenir à la poêle dans du beurre, puis achevez la cuisson dans une casserole
en mouillant avec du bouillon de poisson convenablement aromatisé de légumes.
A la fin, ajoutez un peu de jus de citron, réduisez la sauce, colorez-la au
besoin avec un peu de caramel et masquez-en les paupiettes.

\section*{\centering Friture mélangée maigre.}
\addcontentsline{toc}{section}{ Friture mélangée maigre.}
\label{pg0315} \hypertarget{p0315}{}
\index{Friture mélangée maigre}
\index{Fritto misto}

Les fritures mélangées « fritto misto » sont des plats de la cuisine italienne.
On les prépare avec différents éléments : poissons, crustacés, mollusques ;
viandes blanches ; issues (rognons, cervelles, pieds, tripes, foie, ris,
moelle, etc.) ; légumes (choux-fleurs, fonds d'artichauts, aubergines,
courgettes, etc.).

La composition des fritto misto est variable avec les goûts et les saisons.

Voici un exemple de fritto misto de poissons, de crustacés et de mollusques.

Pour quatre personnes prenez :

\medskip

\footnotesize
\begin{longtable}{rrrp{16em}}
    500 & grammes & de & marinade au vin blanc, \hyperlink{p0385}{p. \pageref{pg0385}},                   \\
    125 & grammes & d' & éperlans\footnote{Osmerus eperlanus, famille des Salmonidés.},                   \\
    125 & grammes & de & goujons\footnote{Gobio fluviatilis, famille des Cyprinidés.},                    \\
        &         & 12 & belles huîtres,                                                                  \\
        &         & 12 & Écrevisses,                                                                      \\
        &         &  1 & sole,                                                                            \\
        &         &    & jaunes d'œufs,                                                                   \\
        &         &    & farine,                                                                          \\
        &         &    & mie de pain rassis tamisée,                                                      \\
        &         &    & persil,                                                                          \\
        &         &    & jus de citron,                                                                   \\
        &         &    & cayenne, paprika.                                                                \\
\end{longtable}
\normalsize

Faites cuire les écrevisses ; décortiquez-les.

Levez les filets de la sole ; coupez-les en languettes.

Mettez dans la marinade filets de sole, éperlans, goujons et queues
d'écrevisses ; laissez-les dedans pendant une demi-heure.

Passez ensuite ces différents éléments et les huîtres dans de la farine, puis
dans des jaunes d'œufs battus et enfin dans de la mie de pain rassis tamisée,
assaisonnée avec cayenne et paprika, au goût.

Faites frire successivement, dans de l'huile bien chaude, d’abord les filets de
sole, ensuite les éperlans et les goujons, et en dernier lieu les queues
d’écrevisses et les huîtres.

Dressez les éléments de la friture sur un plat, arrosez-les avec du jus de
citron, décorez avec du persil frit et servez.

\section*{\centering Friture de laitances, sauce moutarde.}
\addcontentsline{toc}{section}{ Friture de laitances, sauce moutarde.}
\index{Friture de laitances, sauce moutarde}

Pour douze personnes prenez :

\medskip

1° {\ppp36\mmm} laitances de harengs ou {\ppp12\mmm} laitances de carpes ;

2° pour la pâte à frire :

\medskip

\footnotesize
\begin{longtable}{rrrp{16em}}
    200 & grammes & de & petite bière légère ou d'eau tiède,                                              \\
    125 & grammes & de & farine,                                                                          \\
     30 & grammes & d' & huile d'olive ou de beurre fondu,                                                \\
      3 & grammes & de & sel,                                                                             \\
        &         &  2 & œufs.                                                                            \\
\end{longtable}
\normalsize

3° pour la sauce :

\footnotesize
\begin{longtable}{rrrrp{16em}}
  & 300 & grammes & d' & eau chaude,                                                                      \\
  &  90 & grammes & de & beurre,                                                                          \\
  &  50 & grammes & de & crème épaisse,                                                                   \\
  &  30 & grammes & de & farine de gruau,                                                                 \\
  &  12 & grammes & de & sel blanc,                                                                       \\
  &  12 & grammes & de & moutarde à l'estragon ou aux fines herbes,                                       \\
  & \multicolumn{2}{r}{5 à 8 grammes} & de & vinaigre ou de jus de citron, au goût,                       \\
  &   2 & grammes & de & moutarde en poudre,                                                              \\
  & \multicolumn{2}{r}{1 gramme 1/2 } & de & poivre fraîchement moulu.                                    \\
\end{longtable}
\normalsize

Préparez une pâte légère avec les éléments indiqués ci-dessus.

Cassez les œufs, séparez les blancs des jaunes, mettez-les de côté pour les
fouetter ultérieurement. Mélangez tout le reste de façon à obtenir une pâte
lisse sans grumeaux, ayant la consistance d'une crème anglaise. Laissez-la
reposer pendant une heure.

Battez les blancs de manière à les rendre très fermes, incorporez-les à la pâte
au moment de l'employer.

Enrobez dans cette pâte les laitances de poisson, faites-les frire dans de la
graisse très chaude, puis servez-les avec une sauce moutarde chaude, qui n'est
autre chose qu'une sauce blanche à la moutarde.

\sk

Commencez par faire une sauce blanche ordinaire. Mettez dans une casserole
{\ppp60\mmm} grammes de beurre, la farine, le sel et le poivre ; chauffez,
mélangez bien, puis mouillez avec l’eau et faites cuire en tournant pendant une
dizaine de minutes environ. Ajoutez alors le reste du beurre par petits
morceaux, puis la crème, le vinaigre ou le jus de citron, et chauffez un peu en
tournant toujours jusqu'à ce que le beurre soit complètement fondu.

\sk

Pour faire une sauce moutarde, mettez dans la sauce blanche que vous venez
de préparer, les moutardes délayées avec le vinaigre ou le jus de citron.

\sk

En remplaçant la sauce blanche ordinaire par de l'allemande, on obtiendra une
sauce moutarde bien supérieure. Voici la manière de la préparer : faites cuire
de la farine dans du beurre sans laisser prendre couleur, mouillez avec un fond
à base de veau et volaille, liez avec des jaunes d'œufs, puis achevez la sauce
avec de la moutarde, du poivre et du cayenne, au goût.

\section*{\centering Saucisson maigre.}
\addcontentsline{toc}{section}{ Saucisson maigre.}
\index{Saucisson maigre}

Pour douze personnes prenez :

\medskip

\footnotesize
\begin{longtable}{rrrp{16em}}
  1 000 & grammes & d' & anguille de rivière,                                                             \\
    750 & grammes & de & colin,                                                                           \\
    150 & grammes & de & mie de pain rassis, avec un peu de croûte,                                       \\
        &         &  1 & langouste œuvée, en vie, pesant {\ppp600\mmm} grammes environ,                   \\
        &         &    & sel blanc,                                                                       \\
        &         &    & poivre fraîchement moulu.                                                        \\
\end{longtable}
\normalsize

Hachez séparément les poissons crus (anguille et colin) débarrassés de leurs
arêtes, réunissez-les, hachez de nouveau, ajoutez le pain, hachez encore,
salez, poivrez au goût, mélangez et triturez le tout.

Faites cuire la langouste pendant {\ppp10\mmm} minutes dans le court-bouillon
indiqué \hyperlink{p0284}{p. \pageref{pg0284}}. Laissez-la refroidir dans le
liquide, puis retirez-en les œufs et incorporez-les, après les avoir passés
à la double mousseline, au hachis préparé précédemment. Découpez la chair de la
langouste en dés.

Prenez deux boyaux de bœuf, emplissez-les avec le hachis et les dés de
langouste par couches alternées, ficelez et faites cuire les saucissons pendant
{\ppp25\mmm} minutes dans le court-bouillon de la langouste.

Laissez-les refroidir dans la cuisson.

\sk

Comme variante, on peut remplacer la chair de langouste par des queues de
crevettes.

\sk

On peut également préparer dans le même esprit des petites saucisses maigres,
à manger chaudes, avec ou sans garniture.

\section*{\centering Vol-au-vent.}
\addcontentsline{toc}{section}{ Vol-au-vent.}
\index{Vol-au-vent}
\index{Définition des vol-au-vent}
\index{Vol-au-vent (Définition des)}

Le vol-au-vent, triomphe de la pâtisserie française, dont la création remonte
à un siècle environ, consiste en une croûte d'une extrême légèreté, qui lui
a valu son nom, faite d'une façon spéciale et garnie d'un salpicon.

On prépare des vol-au-vent maigres et des vol-au-vent gras.

Voici la formule d'un joli vol-au-vent maigre.

\section*{\centering Vol-au-vent de carême.}
\addcontentsline{toc}{section}{ Vol-au-vent de carême.}
\index{Vol-au-vent de carême}
\index{Garniture pour vol-au-vent au maigre}

\label{pg0319} \hypertarget{p0319}{}

Pour huit personnes prenez :

\medskip

1° pour la croûte :

\medskip

\footnotesize
\begin{longtable}{rrrp{16em}}
    500 & grammes & de & farine,                                                                          \\
    500 & grammes & de & beurre,                                                                          \\
    250 & grammes & d' & eau,                                                                             \\
     10 & grammes & de & sel ;                                                                            \\
\end{longtable}
\normalsize

2° pour la garniture :



\footnotesize
\begin{longtable}{rrrp{16em}}
     20 & grammes & d' & œufs de homard crus,                                                             \\
        &         & 24 & quenelles de brochet et d'écrevisses de dimensions moyennes,
                         préparées comme il est dit \hyperlink{p0328}{p. \pageref{pg0328}},               \\
        &         & 24 & belles crevettes roses\footnote{ou bouquet. Palœmon sarratus,
                         famille des Carididés.},                                                         \\
        &         & 24 & belles huîtres de Cancale ou de Marennes,                                        \\
        &         & 24 & champignons de couche moyens,                                                    \\
        &         &  1 & petite langouste,                                                                \\
        &         &    & truffes à volonté,                                                               \\
        &         &    & vin de Champagne,                                                                \\
        &         &    & beurre,                                                                          \\
        &         &    & jus de citron,                                                                   \\
        &         &    & sel et poivre.                                                                   \\
\end{longtable}
\normalsize

Préparez d'abord une pâte feuilletée.

Disposez sur une table la farine en cratère de volcan ou, comme l'on dit en
cuisine, « formez fontaine », mettez dans le cratère le sel, un peu d'eau,
malaxez et ajoutez le reste de l'eau en plusieurs fois.

Travaillez la pâte en la pressant et en l’allongeant avec la main ; pliez-la
sur elle-même et allongez-la encore. Quand elle sera lisse, roulez-la et
laissez-la reposer pendant quelques heures, ce qui lui permettra de lever un
peu. Étendez-la alors au rouleau\footnote{Les meilleurs rouleaux sont en buis ;
les tables les meilleures pour abaisser la pâte sont en marbre.}, étalez dessus
le beurre ramolli ou rafraîchi suivant la saison, aplatissez et pliez en
quatre, de façon à enfermer le beurre dans la pâte.

Faites au rouleau une abaisse aussi longue que possible sans déchirure,
pliez-la en trois sur la longueur, puis repliez-la en trois dans l'autre sens
et laissez-la reposer pendant {\ppp10\mmm} minutes ; elle a ainsi ce qu'on
appelle \textit{deux tours} de feuilletage. Recommencez l'opération deux autres
fois en laissant reposer la pâte, chaque fois, après deux tours de
feuilletage ; la pâte aura en tout six tours de feuilletage : elle sera
à point.

Faites avec cette pâte une abaisse de {\ppp35\mmm} millimètres d'épaisseur.

\label{pg0320} \hypertarget{p0320}{}
Mettez sur une plaque en tôle une abaisse protectrice de quelques millimètres
de pâte à foncer\footnote{La composition de la pâte à foncer est la suivante :
                 \medskip
                 \begin{tabular}{rrrl}
                 \hspace{10em}150 & grammes & de & farine,                                                \\
                 \hspace{10em}100 & grammes & de & beurre,                                                \\
                 \hspace{10em} 70 & grammes & d' & eau,                                                   \\
                 \hspace{10em}  3 & grammes & de & sel.                                                   \\
                 \end{tabular}
                 \protect

Préparez une pâte avec ces éléments, pétrissez-les bien, \textit{fraisez} trois
fois, c'est-à-dire pressez la pâte en la creusant avec le bord inférieur de la
paume de la main ; enfin, abaissez-la au rouleau.}, beurrez-la légèrement et
posez dessus la pâte feuilletée que vous découperez au diamètre de {\ppp20\mmm}
centimètres avec un couteau pointu, en suivant les bords plus ou moins ondulés
d'un moule rond et cannelé.

Cela fait, enlevez le moule, dorez le dessus de la pâte à l'œuf sans dorer le
bord, faites une incision de {\ppp15\mmm} millimètres de profondeur, sur tout le
pourtour, à {\ppp25\mmm} millimètres du bord, pour préparer le couvercle.

Mettez à cuire au four moyennement chaud ; levez le couvercle de la croûte,
évidez l'intérieur et consolidez les parois, s'il y a lieu, avec une partie de
la pâte extraite.

Préparez en même temps la garniture ; faites cuire, dans un bon court-bouillon
au vin, la langouste et les crevettes, épluchez-les, réservez les parures ;
coupez la queue de la langouste en petits cubes, conservez les queues des
crevettes entières. Mettez à cuire dans le même court-bouillon les œufs de
homard.

Faites blanchir les huîtres dans leur eau, passez l'eau, concentrez-la,
réservez-la.

Épluchez les champignons, passez-les au jus de citron et faites-les cuire dans
du beurre.

Faites cuire les truffes dans du vin de Champagne et coupez-les en petits
cubes.

Tenez le tout au chaud.

Préparez une béchamel maigre, \hyperlink{p0269}{p. \pageref{pg0269}}, en
remplaçant une partie du beurre par du beurre de crustacés\footnote{\index{Beurre de crustacés}
On prépare le beurre de crustacés avec les parures des écrevisses qui ont servi
à la confection des quenelles, celles de la langouste et celles des crevettes
qui ont été réservées.}, accentuez la coloration avec le jus des œufs de homard
passés au tamis à l’aide d'un pilon, ajoutez la cuisson des truffes, plus ou
moins de l'eau des huîtres ; goûtez et complétez l'assaisonnement, de manière
à avoir une sauce relevée.

Mélangez à la sauce les quenelles, les queues de crevettes, les huîtres, la
langouste, les champignons et les truffes.

Passez la croûte pendant cinq minutes à la bouche du four et garnissez-la
aussitôt. Servez chaud.

Ce vol-au-vent me parait digne d'un conclave.

\sk

Comme variantes, on peut préparer d'autres vol-au-vent plus simples, dans le
même esprit.

I1 suffit pour cela de supprimer quelques-uns des éléments de la formule
précédente ou de les remplacer par d'autres éléments moins coûteux.

\section*{\centering Pains de poissons.}
\addcontentsline{toc}{section}{ Pains de poissons.}
\index{Pains de poissons}

Les pains de poissons sont des préparations chaudes ou froides de farces de
poissons accompagnées d'une garniture et d’une sauce adéquates.

Lorsqu'ils doivent être servis chauds, la farce est faite essentiellement avec
de la chair de poisson crue passée au tamis et additionnée de coulis
d'écrevisses ou de crevettes, de crème fraîche, de crème de riz et d'œufs, le
tout dûment assaisonné et amené à bonne consistance. L'appareil est mis dans un
moule, tenu au bain-marie, et cuit au four. La cuisson achevée, le pain est
démoulé sur un plat qu'on garnit au choix avec des écrevisses, des crevettes,
des huîtres, des moules, des champignons, des truffes, etc., et servi avec
accompagnement de sauce au fumet de poisson, de sauce homard, de sauce
crevette, de sauce Nantua, de sauce normande, de sauce hollandaise, par
exemple.

Lorsque les pains doivent être servis froids, on peut incontestablement. après
les avoir préparés comme précédemment, les mettre à refroidir à la glacière,
Mais le plus souvent ils sont préparés différemment : on les apprête avec de la
chair de poisson cuite, passée au tamis et additionnée de foie gras, de sauce
chaud-froid, de gelée et de beurre, le tout bien assaisonné et mélangé
intimement de manière à fournir un bon appareil. Cet appareil est mis dans un
moule tenu à la glacière jusqu'au moment de servir. On démoule le pain sur un
plat, on le décore et on le garnit avec des beurres composés, des truffes, des
écrevisses, des crevettes, etc., au goût, et on le sert en envoyant en même
temps une sauce mayonnaise aux œufs de homard ou au corail d'oursins, par
exemple.

Les pains de poissons sont surtout appréciés par les personnes qui ont la phobie
des arêtes.

\sk

On peut préparer de façons analogues des pains de crustacés.

\section*{\centering Caviar\footnote{Le caviar, qui est importé de Russie, est
constitué par des œufs saumurés d'esturgeon « Acipenser sturio » ou de sterlet
« Acipenser ruthenus », poissons de la famille des Acipenséridés.}.}

\addcontentsline{toc}{section}{ Caviar.}
\index{Caviar}
\index{Canapés de caviar}
\index{Caviar sur canapés}

Le caviar est généralement servi comme hors-d'œuvre. seul ou avec du jus de
citron et du pain beurré. On le présente souvent dans un bloc de glace creusé
en hémisphère et décoré avec des tranches de citron. On peut aussi le servir
sur canapés. En voici une formule.

Pour huit à dix personnes prenez :

\footnotesize
\begin{longtable}{rrrp{16em}}
240 & grammes & de & caviar frais,                                                                       \\
240 & grammes & de & fromage de Hollande frais,                                                          \\
120 & grammes & de & madère,                                                                             \\
120 & grammes & de & beurre,                                                                             \\
 6o & grammes & de & curaçao Focking,                                                                    \\
    &         &  1 & pain anglais.                                                                       \\
\end{longtable}
\normalsize

Pilez ensemble le fromage et le beure, ajoutez le madère et le curaçao ;
mélangez bien.

Coupez la mie du pain anglais en tranches d'un centimètre d'épaisseur,
débitez-les en carrés et en losanges de six centimètres de côté. Faites-en
{\ppp24\mmm}. Étendez le mélange précédemment préparé sur l'une des faces des
{\ppp24\mmm} canapés et mettez le caviar par-dessus. Servez froid.

\sk

\index{Canapés d'huîtres}
\index{Huîtres sur canapés }
On peut préparer d’une façon analogue des huîtres sur canapés.

\section*{\centering Brochet\footnote{Esox lucius, famille des Esocidés.} à la gelée.}
\phantomsection
\addcontentsline{toc}{section}{ Brochet à la gelée.}
\index{Brochet à la gelée}

Préparez ce mets la veille du jour où vous voudrez le manger ; il n'en sera que
meilleur.

\label{pg0323} \hypertarget{p0323}{}
Écaillez le brochet, videz-le, essuyez-le soigneusement et, si vous voulez lui
conserver tout son arome, ne le lavez ni dehors. ni dedans.

\index{Court-bouillon pour brochet}
\index{Court-bouillon pour poissons à la gelée}
Composez, de façon à vous permettre de couvrir complètement le poisson, un
court-bouillon au vin ({\ppp80\mmm} pour {\ppp100\mmm} de bon vin blanc et
{\ppp20\mmm} pour {\ppp100\mmm} de vin rouge), relevé et parfumé avec bouquet
garni, oignon, carotte, échalote, ail, épices, aromates au goût, et assaisonné
avec sel et poivre en grains.

Faites cuire dedans le poisson, à petit feu, puis retirez-le, mettez-le sur un
plat, masquez-le avec une réduction de la cuisson passée, additionnée de
gélatine de pied de veau, obtenue en faisant cuire du pied de veau avec des
légumes et en clarifiant le tout, laissez prendre ; décorez avec des tranches
de citron et des fleurs de capucines.

La qualité de la gelée dépend essentiellement de la finesse des vins emplovés ;
aussi, sans préconiser l'usage des grands vins, qui serait trop onéreux pour le
résultat à atteindre, je recommande de prendre des vins de bonne qualité.

En principe, le brochet à la gelée est servi sans sauce ; cependant rien
n empêche de faire passer en même temps une saucière de rémoulade froide
ordinaire ou de rémoulade verte, par exemple.

\sk

La rémoulade froide ordinaire est une mayonnaise additionnée d'un mélange
d' échalotes ciselées, de cornichons hachés, de câpres et de moutarde.

\label{pg0323-2} \hypertarget{p0323-2}{}
Pour préparer une mayonnaise ordinaire, mettez deux jaunes d'œufs frais, dans
une terrine, travaillez avec une cuiller en bois, puis versez petit à petit, en
tournant constamment, de la bonne huile d'olive, en alternant avec quelques
gouttes de jus de citron et en salant au fur et à mesure. Lorsque vous aurez
assez de sauce, ajoutez un filet de vinaigre, mélangez intimement, goûtez et
complétez l'assaisonnement, au goût, avec sel, poivre et jus de citron.

On peut également remplacer le jus de citron par de la pulpe de citron coupée
en petits morceaux.

\sk

On obtient des variantes de rémoulade en incorporant à la rémoulade précédente
de l'ail, des œufs durs hachés, des filets d'anchois, etc.

\sk

La rémoulade froide verte est une rémoulade ordinaire colorée par du vert
d'épinards.

\label{pg0324} \hypertarget{p0324}{}
Pour obtenir le vert d'épinards, on fait blanchir des épinards avec un peu de
persil ; le tout, rafraîchi, pressé, pilé, est passé à la double mousseline.

\sk

\index{Court-bouillon au vin blanc pour poissons}
\index{Court-bouillon pour carpe}
\index{Court-bouillon pour saumon}
\index{Court-bouillon pour truites}
\index{Court-bouillon pour poissons à la gelée}

La carpe, comme le brochet, peut être préparée à la gelée : la seule différence
dans la préparation consiste dans la proportion des vins composant le
court-bouillon. Pour la carpe, il convient de prendre {\ppp80\mmm} pour
{\ppp100\mmm} de vin rouge et {\ppp20\mmm} pour {\ppp100\mmm} de vin blanc.

\sk

On peut aussi apprêter de même la truite, en particulier la truite saumonée,
l'omble-chevalier\footnote{Salmo umbla, famille des Salmonidés.}, le
lavaret\footnote{Coregonus lavarelus, famille des Salmonidés.}, les
bars\footnote{Labrax lupus et Labrax nigrescens, famille des Percidés.}, le
saumon ; mais alors il est préférable de faire le court-bouillon au vin blanc
seulement et de servir le poisson avec une sauce verte constituée par une
mayonnaise colorée par du vert d'épinards.

\section*{\centering Brochet demi-deuil.}
\addcontentsline{toc}{section}{ Brochet demi-deuil.}
\index{Brochet demi-deuil}
\index{Bar demi-deuil}
\index{Carpe demi-deuil}

Ce plat ne diffère du précédent que par la façon dont on le décore. Au lieu de
tranches de citron et de fleurs de capucines, on emploie des rondelles de blanc
d'œufs durs et de truffes, dont on fait une armure au poisson, en les disposant
par-dessus la gelée, côte à côte, comme les cases d'un damier, chaque rondelle
blanche encadrée de rondelles noires et réciproquement.

\sk

\index{Bar à la gelée (autre formule)}
On peut préparer de même d’autres poissons : carpes, truites, truites
saumonées, saumons, bars, etc. La sauce qui accompagnera le plat sera, de
préférence, soit une simple mayonnaise verte, soit une rémoulade verte, suivant
la finesse du poisson employé.

\section*{\centering Brochet farci rôti.}
\addcontentsline{toc}{section}{ Brochet farci rôti.}
\index{Brochet farci rôti}

Pour quatre personnes prenez un brochel pesant {\ppp750\mmm} grammes environ,
écaillez-le, videz-le et essuyez-le sans le laver. Piquez-le de lardons de
chair d'anguille et de chair d'anchois, dans la proportion de deux lardons
d'anguille pour un d'anchois.

\index{Farce pour brochet}
\index{Farce pour poisson}
Emplissez-le avec une farce composée de laitances de harengs frais, de mie de
pain trempée dans du lait, puis égouttée, et de beurre manié avec du persil, le
tout assaisonné au goût.

Bardez de lard, ficelez et faites cuire au four jusqu'à ce que la barde ait
pris une couleur dorée.

Préparez une sauce avec du beurre, la cuisson du brochet et du jus de citron.

\index{Garniture pour poissons}
Servez le poisson sur un plat avec une garniture de légumes, et la sauce à part,
dans une saucière.

\section*{\centering Brochet aux pommes de terre et à la crème.}
\addcontentsline{toc}{section}{ Brochet aux pommes de terre et à la crème.}
\index{Brochet aux pommes de terre et à la crème}

Écaillez, videz et essuyez le poisson, coupez-le en tronçons de {\ppp4\mmm} centimètres
de longueur.

Pelez des pommes de terre nouvelles, coupez-les en tranches de {\ppp3\mmm} millimètres
d'épaisseur.

Foncez de beurre un plat allant au feu, mettez sur le beurre une couche de
tranches de pommes de terre, assaisonnez avec sel et poivre, disposez dessus
une couche de tronçons de brochet, assaisonnez, remettez une couche de tranches
de pommes de terre, assaisonnez encore, puis noyez le tout dans de la crème.
Faites cuire au four, à feu doux, pendant une heure à une heure un quart, en
arrosant fréquemment.

Les pommes de terre seront moelleuses, parfumées, imbibées de crème, et le tout
délicieux.

Lorsque les pommes de terre dont on dispose sont vieilles, il est bon de les
faire blanchir un peu avant de les employer.

\sk

\index{Brochet jardinière}
On peut préparer, de même manière, un brochet jardinière, en employant, au lieu
de pommes de terre seules, une jardinière de légumes composée de pommes de
terre, carottes, navets, haricots verts, petits pois, petits champignons,
pointes d'asperges, etc. : c'est également très bon.

\section*{\centering Bouchées de brochet farcies, sauce Nantua au coulis\footnote{On définit
généralement les coulis comme des purées liquides. Cette définition est vague car,
en l'admettant, toute purée simplement étendue d'un liquide quelconque deviendrait
un coulis, ce qui n'est évidemment dans l'esprit de personne.
\index{Définition des coulis}
\index{Coulis (Définition des)}
\protect\endgraf
A mon avis, il est préférable, afin d'éviter toute ambiguïté, de réserver ce
terme pour désigner les purées liquides de substances animales dans lesquelles
il n'entre que leur jus propre.
\index{Coulis d'écrevisses}
\protect\endgraf
On obtient un coulis d'écrevisses en passant au tamis des écrevisses
court-bouillonnées et décortiquées.} d’écrevisses.}

\addcontentsline{toc}{section}{ Bouchées de brochet farcies, sauce Nantua au coulis d’écrevisses.}
\index{Bouchées de brochet farcies, sauce Nantua au coulis d'écrevisses}
\index{Brochet (Bouchées de)}
\index{Enveloppe pour bouchées de brochet}
\index{Enveloppe pour bouchées maigres}
\index{Bouchées au maigre}
\index{Farce pour bouchées maigres}
\index{Farce pour poisson}
\index{Bouchées (Enveloppes pour)}

Pour dix personnes prenez :

\medskip

l° pour l'enveloppe :

\footnotesize
\begin{longtable}{rrrp{16em}}
    500 & grammes & de & lait,                                                                            \\
    350 & grammes & de & chair de brochet,                                                                \\
    250 & grammes & de & graisse de rognon de veau,                                                       \\
    200 & grammes & de & farine,                                                                          \\
        &         &  4 & œufs frais,                                                                      \\
        &         &    & crème double,                                                                    \\
        &         &    & muscade,                                                                         \\
        &         &    & sel et poivre ;                                                                  \\
\end{longtable}
\normalsize

2° pour la farce :

\footnotesize
\begin{longtable}{rrrp{16em}}
    300 & grammes & d' & un mélange de coquilles Saint-Jacques, crevettes, champignons
                    hachés, lié au velouté maigre ;                                                       \\
\end{longtable}
\normalsize

3° pour la sauce :

\footnotesize
\begin{longtable}{rrrp{16em}}
    300 & grammes & de & lait,                                                                            \\
     75 & grammes & de & beurre,                                                                          \\
     60 & grammes & de & crème épaisse,                                                                   \\
     20 & grammes & de & farine,                                                                          \\
        &         & 12 & écrevisses,                                                                      \\
        &         &  1 & petite carotte,                                                                  \\
        &         &  1 & petit oignon,                                                                    \\
        &         &  1 & bouquet garni (persil, céleri, thym et laurier),                                 \\
        &         &    & sel et poivre.                                                                   \\
\end{longtable}
\normalsize

Faites bouillir le lait.

Triturez ensemble {\ppp200\mmm} grammes de farine, {\ppp2\mmm} œufs entiers et
{\ppp5\mmm} grammes de sel, ajoutez en tournant {\ppp500\mmm} grammes de lait
très chaud par petites quantités, puis faites prendre l'appareil sur feu doux
en le travaillant pendant une demi-heure environ, de façon à obtenir une crème
de consistance convenable.

Pilez séparément la chair de brochet et la graisse de veau, réunissez-les,
assaisonnez avec sel, poivre et muscade au goût, pilez encore ; mettez ensuite
l'appareil ci-dessus, mélangez bien ; enfin amalgamez au mélange deux blancs
d'œufs battus en neige.

Passez le tout au tamis, incorporez au produit passé de la crème double, à la
spatule, en travaillant sur glace, de manière à obtenir une pâte homogène et
lisse de bonne consistance, dont vous ferez une abaisse de {\ppp5\mmm}
à {\ppp6\mmm} millimètres d'épaisseur, que vous partagerez en vingt morceaux
carrés.

Disposez sur chaque morceau de pâte un vingtième de la farce : fermez les
bouchées.

\index{Coulis d'écrevisses}
Court-bouillonnez les écrevisses comme il est dit
\hyperlink{p0287}{p. \pageref{pg0287}} : décortiquez-les ; passez les queues au
tamis : vous aurez ainsi un coulis d'écrevisses ; préparez, avec les parures et
{\ppp40\mmm} grammes de beurre, un beurre d'écrevisses.

Préparez la sauce Nantua.

Faites revenir légèrement dans le reste du beurre l'oignon et la carotte, mettez
ensuite la farine, tournez pendant cinq minutes sans laisser roussir, mouillez avec
le lait, ajoutez le bouquet garni, du sel, du poivre, et laissez cuire doucement.

Passez la sauce, montez-la avec la crème, amenez le tout à la consistance voulue
pour masquer une cuiller, finissez-la avec le beurre d'écrevisses et parachevez-la
avec le coulis d'écrevisses.

Faites pocher les bouchées dans de l'eau salée bouillante, égouttez-les, dorez-les
au jaune d'œuf et passez-les au four.

Servez les bouchées au sortir du four et envoyez en même temps la sauce dans
une saucière.

Ces bouchées, très moelleuses, constituent un hors-d'œuvre chaud exquis.

\sk

Il est aisé de concevoir un grand nombre de variantes de bouchées analogues en
modifiant la composition des enveloppes, des farces et des sauces.

\section*{\centering Croquettes de brochet et d'écrevisses.}
\addcontentsline{toc}{section}{ Croquettes de brochet et d'écrevisses.}
\index{Croquettes de brochet et d'écrevisses}
\index{Brochet (Croquettes de) }

Pour six personnes prenez :

\medskip

\footnotesize
\begin{longtable}{rrrp{16em}}
    250 & grammes & de & mie de pain rassis tamisée,                                                      \\
    125 & grammes & de & crevettes grises vivantes,                                                       \\
        &         & 24 & écrevisses vivantes,                                                             \\
        &         &  2 & œufs frais.                                                                      \\
        &         &  1 & brochet pouvant fournir 750 grammes de chair environ,                            \\
        &         &    & beurre.                                                                          \\
\end{longtable}
\normalsize

Faites cuire le brochet conformément aux indications données
\hyperlink{p0323}{p. \pageref{pg0323}}.

Préparez un court-bouillon pour écrevisses,
\hyperlink{p0287}{p. \pageref{pg0287}}, en remplaçant le vin par une même quantité
de liquide de cuisson du brochet, et en tenant compte de la proportion
d'assaisonnement introduite par ce liquide.

Faites cuire dedans les crevettes et les écrevisses ; concentrez ensuite la
cuisson.

Décortiquez crevettes et écrevisses ; réservez les parures.

Pilez la chair du brochet, incorporez-y la mie de pain en pilant toujours,
ajoutez les queues d'écrevisses et pilez encore, puis mouillez avec la quantité
nécessaire de jus de cuisson des crevettes et des écrevisses, de manière
à obtenir une pâte façonnable, dont vous ferez des croquettes ayant la forme de
saucisses plates.

Passez successivement ces croquettes dans les œufs battus et dans la mie de
pain rassis tamisée ; laites-les dorer dans du beurre.

Servez avec une sauce hollandaise au beurre de crevettes et d'écrevisses, dans
laquelle vous aurez mis les queues de crevettes.

Le beurre sera préparé avec les parures réservées des crevettes et des écrevisses.

\section*{\centering Quenelles de brochet et d’écrevisses.}
\addcontentsline{toc}{section}{ Quenelles de brochet et d’écrevisses.}
\index{Quenelles de brochet et d'écrevisses}
\index{Brochet (Quenelles de)}
\label{pg0328} \hypertarget{p0328}{}

Préparez un mélange identique à celui de la formule précédente, incorporez-y
deux œufs entiers, vous obtiendrez une pâte suffisamment consistante pour vous
permettre de mouler des quenelles que vous ferez pocher dans de l’eau
bouillante ou dans du bouillon de poisson.

Ces quenelles trouveront leur place comme garniture dans la plupart des plats
de poisson, timbales et vol-au-vent maigres.

\sk

On peut préparer de même des quenelles de brochet et de crevettes.

\section*{\centering Matelote de carpe à la graisse d'oie.}
\addcontentsline{toc}{section}{ Matelote de carpe à la graisse d'oie.}
\index{Matelote de carpe à la graisse d'oie}
\index{Carpe en matelote, à la graisse d'oie}
\index{Carpe en matelote}
\index{Matelote de carpe}

Prenez une belle carpe grasse de {\ppp1\mmm} kilogramme au moins, de
{\ppp2\mmm} kilogrammes au plus ; écaillez-la, videz-la, coupez-la en morceaux,
tête comprise.

Faites dorer légèrement dans un peu de fine graisse d'oie des oignons ciselés,
des poireaux et des carottes coupés en morceaux, un peu d'ail au goût ;
mouillez avec un demi-litre de vin blanc sec, vieux ; ajoutez un bouquet de
persil, du sel, du poivre en grains. Laissez cuire à feu doux pendant deux
à trois heures, jusqu'à ce que le liquide se soit réduit d'un tiers ;
passez-le.

Faites revenir, à feu vif, dans de la graisse d'oie. les morceaux de carpe en
évitant de les briser : mettez-les dans la sauce passée ; achevez la cuisson de
l'ensemble à petit feu pendant une heure environ. Les morceaux de poisson
doivent rester entiers.

Dressez le poisson dans un plat ; décorez avec des croûtons de pain grillé ;
masquez le tout avec la sauce et servez.

\section*{\centering Laitances de carpes au chablis, en turban de saumon.}
\addcontentsline{toc}{section}{ Laitances de carpes au chablis, en turban de saumon.}
\index{Laitances de carpes au chablis, en turban de saumon}

Pour dix à douze personnes prenez :

\medskip

\footnotesize
\begin{longtable}{rrrp{16em}}
    750 & grammes & de & vin de Chablis,                                                                  \\
    600 & grammes & de & beurre,                                                                          \\
    500 & grammes & de & saumon,                                                                          \\
    500 & grammes & de & pommes de terre,                                                                 \\
    250 & grammes & de & crevettes grises,                                                                \\
    250 & grammes & de & petits champignons de couche,                                                    \\
    175 & grammes & d' & oignons,                                                                         \\
    100 & grammes & d' & eau,                                                                             \\
     30 & grammes & de & farine,                                                                          \\
        & 1 litre & de & moules,                                                                          \\
        &         & 36 & huîtres,                                                                         \\
        &         & 18 & écrevisses,                                                                      \\
        &         & 12 & laitances de carpes,                                                             \\
        &         &  2 & jaunes d'œufs frais,                                                             \\
        &         &  2 & bouquets garnis,                                                                 \\
        &         &    & truffes à volonté,                                                               \\
        &         &    & madère,                                                                          \\
        &         &    & jus de citron,                                                                   \\
        &         &    & échalotes,                                                                       \\
        &         &    & ail,                                                                             \\
        &         &    & quatre épices,                                                                   \\
        &         &    & sel et poivre.                                                                   \\
\end{longtable}
\normalsize

Préparez un court-bouillon avec l’eau, {\ppp250\mmm} grammes de chablis. {\ppp50\mmm} grammes
d'oignons, des échalotes, de l'ail, un bouquet garni, un peu de quatre épices, du
sel et du poivre. Durée de cuisson : une heure environ.

Faites cuire en même temps :

\textit{a}) le saumon dans le court-bouillon préparé ci-dessus ; puis, passez-en la chair
au tamis. Concentrez la cuisson, passez-la, réservez-la ;

\textit{b}) les écrevisses et les crevettes comme il est dit
\hyperlink{p0287}{p. \pageref{pg0287}} : décortiquez-les, mettez les queues de
côté, réservez les parures ;

\textit{c}) les pommes de terre à la vapeur ; pelez-les ; passez-les au tamis ;

\textit{d}) les truffes brossées, lavées et pelées, dans du madère. Réservez
les pelures ; hachez-les.

Préparez un beurre de crevettes et d'écrevisses avec {\ppp500\mmm} grammes de beurre et
les parures réservées.

Mélangez purée de saumon, purée de pommes de terre, pelures de truffes et
{\ppp400\mmm} grammes de beurre de crevettes et d'écrevisses ; liez avec les jaunes d'œufs,
assaisonnez avec sel et poivre, puis emplissez un moule à couronne avec ce mélange.
Faites cuire au bain-marie ; tenez au chaud,

Mettez dans une casserole {\ppp70\mmm} grammes de beurre et la farine ; faites prendre
couleur ; ajoutez le reste des oignons et le second bouquet garni ; mouillez
avec le reste du chablis ; salez, poivrez ; laissez cuire pendant une heure ;
vous aurez ainsi une sauce maltelote ; passez-la, puis faites cuire dedans les
laitances. Sortez-les, égouttez-les, tenez-les au chaud. Tenez également la
sauce au chaud,

Faites s'ouvrir, dans une casserole, les moules, à feu vif ; retirez-les des
coquilles ; tenez-les au chaud.

Ouvrez les huîtres ; faites-les pocher dans leur eau.

Faites cuire les champignons, pelés, dans le reste du beurre, avec un peu de
jus de citron.

Préparez un salpicon avec huîtres, moules, champignons, truffes, queues
d'écrevisses et de crevettes, sauce matelote, le reste du beurre d'écrevisses
et de crevettes, les cuissons du saumon, des huîtres, des champignons et des
truffes.

Démoulez le turban de saumon sur un plat ; disposez les laitances sur le bord
du turban ; versez le salpicon dans l'intérieur et servez.

C'est un plat qui a grande allure.

\sk

On peut remplacer le turban de saumon par un turban de brochet : c'est
également très fin.

\sk

On pourra préparer d'une façon analogue d’autres laitances à d'autres vins et
les servir sur d'autres turbans. C'est ainsi qu'on peut, par exemple, présenter
des laitances de harengs cuites au vin d'Anjou, en turban de morue. Le plat est
moins distingué, assurément, mais il est encore très honorable.

\section*{\centering Perches\footnote{Perea fluviatilis et Acerina cernus, famille des Percidés.}.}
\addcontentsline{toc}{section}{ Perches.}
\index{Perches}

On accommode les perches différemment suivant leur grosseur : les petites sont
généralement grillées ou frites ; les moyennes sont le plus souvent préparées
à la meunière ; les grosses sont de préférence farcies et braisées.

\section*{\centering Perches aux écrevisses.}
\addcontentsline{toc}{section}{ Perches aux écrevisses.}
\index{Perches aux écrevisses}

Pour six personnes prenez :

\medskip

\footnotesize
\begin{longtable}{rrrp{16em}}
    250 & grammes & de & beurre,                                                                          \\
    200 & grammes & de & champagne,                                                                       \\
     60 & grammes & de & champignons,                                                                     \\
     60 & grammes & de & crème épaisse,                                                                   \\
     30 & grammes & d' & eau,                                                                             \\
        &         & 30 & écrevisses vivantes,                                                             \\
        &         &  6 & perches moyennes,                                                                \\
        &         &  4 & jaunes d'œufs frais,                                                             \\
        &         &    & court-bouillon pour écrevisses,                                                  \\
        &         &    & quelques têtes et arêtes de poissons,                                            \\
        &         &    & jus de citron ou vinaigre,                                                       \\
        &         &    & sel et poivre.                                                                   \\
\end{longtable}
\normalsize

Écaillez, videz, lavez les perches ; essuyez-les.

Court-bouillonnez les écrevisses suivant les indications données
\hyperlink{p0287}{p. \pageref{pg0287}} : décortiquez-les ; tenez les queues au
chaud.

Faites un beurre d'écrevisses avec les parures et {\ppp50\mmm} grammes de
beurre comme il est dit \hyperlink{p0287-3}{p. \pageref{pg0287-3}}.

Dans un peu du court-bouillon des écrevisses, faites cuire les champignons
émincés, les têtes et les arêtes de poissons. Concentrez fortement ce fumet,
passez-le.

Foncez un plat avec le beurre d’écrevisses, disposez dedans les perches, mouillez
avec le fumet et le champagne et faites cuire au four.

Préparez une sauce hollandaise avec le reste du beurre, les jaunes d'œufs,
l'eau, du jus de citron ou du vinaigre, du sel et du poivre au goût, montez-la
avec la crème et, au dernier moment, incorporez-y le jus de cuisson des
perches.

Dressez les poissons sur un plat, garnissez avec les queues d’écrevisses, masquez
le tout avec la sauce et servez sans attendre.

\section*{\centering Perches farcies braisées.}
\addcontentsline{toc}{section}{ Perches farcies braisées.}
\index{Perches farcies braisées}

Prenez de belles perches, écaillez-les, videz-les, lavez-les, essuyez-les.

Préparez une farce avec du saumon frais, des queues de crevettes grises, des
laitances de carpes, un peu de mie de pain rassis trempée dans du lait et
pressée, des jaunes d'œufs, du madère ou du porto au goût, du sel et du
paprika. Farcissez-en les perches.

\index{Fond de poisson}
Faites un fond de poisson très concentré avec les déchets du saumon, les
parures des crevettes, des têtes et des arêtes de poissons, de la carotte, de
l'oignon, un petit bouquet garni, du sel et du poivre, le tout cuit dans du vin
blanc additionné d'un peu d'eau. Passez-le.

Foncez une braisière avec du beurre, mettez dedans les perches, laissez-les
dorer légèrement, puis mouillez avec le fond de poisson et faites braiser au
four.

Disposez les perches sur un plat, masquez-les avec leur cuisson très réduite,
garnissez avec des goujons frits et servez.

\section*{\centering Truites\footnote{Salmo forio, famille des Salmonidés.} au beurre.}
\addcontentsline{toc}{section}{ Truites au beurre.}
\index{Truites au beurre}
\label{pg0332} \hypertarget{p0332}{}

La truite de rivière et. en particulier, la truite d'eau froide coulant sur un
sol granitique, est un poisson extrêmement fin. Le procédé le meilleur pour lui
conserver son arome est de la faire cuire simplement au beurre, procédé dit « à
la meunière », de la façon suivante, qui convient surtout pour des truites
moyennes, pesant de {\ppp200\mmm} à {\ppp300\mmm} grammes.

Pour quatre personnes prenez :

\medskip

\footnotesize
\begin{longtable}{rrrp{16em}}
    200 & grammes & de & beurre,                                                                          \\
        &         &  2 & truites, pesant ensemble 500 grammes environ,                                    \\
        &         &  1 & citron,                                                                          \\
        &         &    & farine,                                                                          \\
        &         &    & persil,                                                                          \\
        &         &    & sel blanc.                                                                       \\
\end{longtable}
\normalsize

Nettoyez les truites, roulez-les dans de la farine. Faites fondre {\ppp150\mmm} grammes
de beurre sans le laisser roussir, mettez dedans les truites, salez, ajoutez le
jus du quart du citron et laissez mijoter pendant une vingtaine de minutes.

Au moment de servir, maniez le reste du beurre avec du persil haché, chauffez
et versez sur les truites dressées sur un plat chaud garni avec le reste du
citron coupé en tranches,

\sk

Comme variante, on peut mettre un peu de jus de tomates dans la cuisson. On
pourra alors donner à ce plat le nom de « truite meunière tomatée ».

\index{Bar au beurre}
\index{Bar à la meunière}
\index{Bar à la meunière tomatée}
\index{Brème au beurre}
\index{Brème à la meunière}
\index{Mulet au beurre}
\index{Maquereau au beurre}
\index{Perche au beurre}
On peut préparer de même d'autres poissons de dimensions semblables, tels que
les bars, la brème, les mulets\footnote{ou Muges, famille des Mugilidés.}, le
maquereau, les perches, etc.

\section*{\centering Truites au bleu.}
\addcontentsline{toc}{section}{ Truites au bleu.}
\index{Truites au bleu}
\index{Brochet au bleu}
\index{Carpe au bleu}

Le mode de préparation dit « au bleu » consiste à saisir d'abord le poisson
dans un liquide acide bouillant, ce qui lui donne une teinte azurée\footnote{A
quoi tient cette teinte bleue ? La question n'est pas résolue. Ce qui est
certain c'est que cette coloration ne se produit qu'avec des poissons vivants
recouverts encore de leur enduit organique.

C'est probablement un phénomene de plaques minces,}, puis à le
court-bouillonner. Ce procédé ne s'applique guère qu'à la truite, et de
préférence à la petite truite de ruisseau, au brochet et à la carpe.

Voici une formule concrète pour truites.

\medskip

Pour six personnes prenez :

\medskip

\footnotesize
\begin{longtable}{rrrp{16em}}
    700 & grammes & de & vin blanc,                                                                       \\
    300 & grammes & d' & eau,                                                                             \\
     50 & grammes & de & vinaigre,                                                                        \\
     50 & grammes & de & carottes émincées.                                                               \\
     50 & grammes & d' & oignons émincés,                                                                 \\
     50 & grammes & de & sel gris,                                                                        \\
     15 & grammes & de & persil en branches,                                                              \\
     15 & grammes & de & racine de persil,                                                                \\
     10 & grammes & de & poivre en grains,                                                                \\
        &         &  6 & petites truites de ruisseau vivantes, pesant ensemble 600
                         grammes environ.                                                                 \\
\end{longtable}
\normalsize

\index{Court-bouillon pour poissons au bleu}
Préparez un court-bouillon avec le vin, l’eau, les carottes, les oignons, le
persil (racine et branches), le poivre et {\ppp35\mmm} grammes de sel ; faites
bouillir pendant une heure, passez et réservez.

Mettez le reste du sel dans le vinaigre, faites-le dissoudre, chauffez
à ébullition.

Tirez les truites de l'eau en les prenant par les ouïes, sans toucher le reste
du corps, donnez-leur un coup sur la tête, videz-les sans les écailler ni les
essuyer, passez-leur une ficelle dans l'intérieur du corps de la tête à la
queue. au moyen d'une aiguille à brider et roulez-les en cercle, en nouant les
deux extrémités de la ficelle\footnote{La ficelle n’est pas indispensable pour
obtenir l’enroulement, car le poisson se recroqueville tout seul sur lui-même
lorsqu'on le plonge dans le court-bouillon ; mais elle permet de lui donner une
forme circulaire régulière.}.

Mettez les truites ainsi apprêtées dans une casserole, arrosez-les vivement et
partout avec le vinaigre bouillant ; leur peau prendra aussitôt une coloration
azurée ; ajoutez le court-bouillon et faites cuire à température modérée
pendant un quart d'heure.

Retirez les ficelles, dressez les truites sur un plat et servez immédiatement.

Envoyez en même temps, dans une saucière, du beurre fondu aromatisé avec
du jus de citron.

\sk

Lorsqu'on prépare au bleu un brochet ou une carpe, le poisson, étant plus gros,
exige un peu plus de temps de cuisson, et on ne le roule pas sur lui-même.

\sk

Les poissons cuits au bleu peuvent aussi être servis froids. Dans ce cas, on
prépare le court-bouillon exclusivement au vin pour lui donner plus de goût, on
laisse refroidir le poisson dans la cuisson, puis on le sert avec une sauce
vinaigrette, dont voici une formule qui comporte de nombreuses variantes.

\medskip

Prenez :

\footnotesize
\begin{longtable}{rrrp{16em}}
    250 & grammes & d' & huile d'olive,                                                                   \\
    100 & grammes & te & vinaigre de vin,                                                                 \\
     35 & grammes & d' & oignons hachés fin,                                                              \\
     25 & grammes & de & persil haché,                                                                    \\
     20 & grammes & de & cerfeuil, ciboule et estragon hachés fin,                                        \\
     20 & grammes & de & câpres,                                                                          \\
      2 & gramme  & de & sel blanc,                                                                       \\
    1/2 & gramme  & de & poivre fraîchement moulu.                                                        \\
\end{longtable}
\normalsize

Mélangez intimement le tout et servez dans une saucière.

\sk

\index{Court-bouillon pour poissons au bleu}
On peut remplacer dans le court-bouillon le vin blanc par du vin rouge, mais la
préparation est plus appétissante avec du vin blanc.

\section*{\centering Truite saumonée à la crème.}
\addcontentsline{toc}{section}{ Truite saumonée à la crème.}
\index{Truite saumonée à la crème}

Faites cuire doucement au beurre, sans qu'il se colore, des darnes de truite
saumonée ; déglacez avec de la crème et du jus de citron ; chauffez sans
laisser bouillir pendant quelques instants et servez.

C'est excellent.

\section*{\centering Truite saumonée\footnote{Truite commune, « Salmo fario », à chair rouge.} braisée.}
\addcontentsline{toc}{section}{ Truite saumonée braisée.}
\index{Truite saumonée braisée}

Écaillez, videz, nettoyez le poisson, enlevez le sang figé qui se trouve le
long de la grosse arête, assaisonnez-le ensuite avec du sel, bardez-le de lard
et cuisez-le de la façon suivante.

Faites revenir, dans une braisière foncée de beurre, quelques rondelles
d’'oignon et de carotte, ajoutez {\ppp100\mmm} grammes de vin rouge ou blanc,
{\ppp30\mmm} grammes de fine champagne et un bouquet garni ; puis mettez la
truite, couvrez d'un papier beurré et faites braiser en arrosant fréquemment,
après avoir enlevé et remis chaque fois le papier. Laissez cuire en tout
pendant {\ppp35\mmm} minutes environ. Dressez la truite sur un plat et tenez-la
au chaud.

Réduisez la cuisson, montez-la au beurre et masquez le poisson avec une partie
de cette sauce.

\index{Garniture pour poissons}
Envoyez en même temps le reste de la sauce dans une saucière et un légumier
de pommes de terre cuites à la vapeur.

\section*{\centering Truite saumonée farcie, en aspic.}
\addcontentsline{toc}{section}{ Truite saumonée farcie, en aspic.}
\index{Truite saumonée farcie, en aspic}
\index{Farce pour poisson}

Pour six personnes prenez :

\footnotesize
\begin{longtable}{rrrp{16em}}
    300 & grammes & de & beurre,                                                                          \\
    125 & grammes & de & crevettes,                                                                       \\
    125 & grammes & d' & œufs de homard,                                                                  \\
        & 2 litres& de & court-bouillon, \hyperlink{p0323}{p. \pageref{pg0323}},                          \\
        & 1 litre & de & gelée maigre, \hyperlink{p0350}{p. \pageref{pg0350}},                            \\
        &         & 12 & Écrevisses,                                                                      \\
        &         &  1 & truite saumonée pesant 1 500 grammes environ,                                    \\
        &         &    & truffes cuites au champagne, à volonté.                                          \\
\end{longtable}
\normalsize

Videz la truite, ébarbez-la, fendez-la sur le dos, retirez-en l'arête que vous
réserverez. Court-bouillonnez ensuite la truite ; laissez-la refroidir dans sa
cuisson.

Faites cuire, comme il est dit \hyperlink{p0287}{p. \pageref{pg0287}}, les
crevettes et les écrevisses, puis faites rougir dans le même court-bouillon les
œufs de homard.

Décortiquez les crevettes et les écrevisses ; réservez les queues.

Préparez un beurre coloré et aromatisé, comme il est dit
\hyperlink{p0287-3}{p. \pageref{pg0287}}, avec les parures de crevettes et
d'écrevisses, les œufs de homard et le beurre ; mettez-en de côté une partie
pour la décoration du plat, mélangez au reste les queues de crevettes, pilées
ou non. et farcissez-en la truite.

Passez la cuisson de la truite, ajoutez-y la gelée, mettez dedans l'arête,
concentrez le liquide au volume d'un litre, clarifiez-le au blanc d'œuf et
passez-le à la serviette : vous aurez ainsi une belle gelée d'aspic.

Dressez la truite sur un plat, masquez-la avec la gelée d'aspic fondue, laissez
prendre ; enfin, décorez avec les queues d'écrevisses, des rondelles de truffe
et le beurre réservé.

C'est une très jolie entrée de poisson.

\sk

\index{Alose farcie, en aspic}
\index{Aspic d'alose}
\index{Aspic de bar}
\index{Aspic de barbue}
\index{Aspic de merlan}
\index{Aspic de perche}
\index{Aspic de saumon}
\index{Aspic de sole}
\index{Aspic de truite}
\index{Aspic de turbot}
\index{Alose (Aspic d')}
\index{Bar farci (Aspic de)}
\index{Barbue farcie (Aspic de)}
\index{Merlan farci (Aspic de)}
\index{Perche farcie (Aspic de)}
\index{Saumon farci (Aspic de)}
\index{Sole farcie (Aspic de)}
\index{Truite farcie (Aspic de)}
\index{Turbot farcie (Aspic de)}
On peut préparer de même beaucoup d'autres poissons, tels que alose, bar,
barbue, merlan, perche, saumon, sole, truite, turbot, par exemple, en faisant
varier les proportions des éléments accessoires avec la grosseur et la nature
du poisson.

\section*{\centering Omble-chevalier\footnote{Salmo umbla, famille des Salmonidés.}
                  et ombre commun\footnote{Salmo thymalus, famille dés Salimonidés.}.}

\addcontentsline{toc}{section}{ Omble-chevalier et ombre commun.}
\index{Omble-chevalier et ombre commun}

Ces deux poissons. de la même famille, sont souvent confondus parce que dans
certains pays on désigne, par corruption, l'omble-chevalier sous le nom
d'ombre-chevalier. Cependant, leur taille, leur teinte et leur aspect sont loin
d'être identiques ; leur goût est différent ; enfin, ils ne vivent pas dans les
mêmes milieux, l’omble étant essentiellement un poisson de lac, tandis que
l'ombre habite les eaux courantes, rapides et limpides.

Tous deux sont des poissons très fins : l’omble est supérieur à la truite de lac
et l'ombre, dont la chair a une légère odeur de thym, peut rivaliser avec la truite
de rivière,

Ils peuvent être accommodés de mêmes façons.

On peut apprêter l'omble et l'ombre entiers, en darnes ou en filets.

Les grosses pièces sont généralement court-bouillonnées et présentées entières,
chaudes ou froides. Dans le premier cas, elles peuvent être servies avec une
sauce hollandaise aromatisée ou non ; dans le second, avec une sauce mayonnaise
simple ou composée.

Les darnes peuvent être grillées et accompagnées d'une sauce béarnaise
à l'huile, d'une sauce mousseline, d'une sauce diable, d'une sauce maitre
d'hôtel, etc.

Les filets sont cuits le plus souvent dans du beurre, du fumet de poisson et du
vin. On peut également les présenter sous la forme de paupiettes farcies avec de
la pâte à quenelles de brochet et d'écrevisses, par exemple, cuites dans un fumet
de poisson additionné de vin blanc et servies dressées sur un socle de risotto, le
tout masqué avec une sauce Nantua et décoré avec des tranches de truffes cuites
dans du madère. On peut aussi les couper en morceaux qu'on passera dans de la
farine ou dans de la pâte et qu'on fera frire à pleine friture ; on les servira alors,
par exemple, avec une sauce tomate, du citron et du persil frit.

Le procédé de préparation le meilleur pour les petites pièces, ne dépassant pas
trente centimètres, est le procédé dit « à la meunière »,
\hyperlink{p0332}{p. \pageref{pg0332}}, dont on corsera un peu la cuisson par
l'addition de certains condiments lorsqu'il s'agira de l'omble, dont la chair
est plus fade que celle de l'ombre. C'est ainsi qu'on se trouvera très bien
d'ajouter à la cuisson de l'omble à la meunière un peu de jus de tomate
(meunière fomatée), ou de garnir le poisson soit avec des tomates concassées
cuites au beurre avec un peu d'oignon, soit encore avec des câpres.

De toutes façons, l'omble et l'ombre seront à leur place dans des menus de
repas soignés.

\section*{\centering Saumon\footnote{Salmo salmo, famille des Salmonidés.} grillé, au beurre d'anchois.}
\addcontentsline{toc}{section}{ Saumon grillé, au beurre d'anchois.}
\index{Saumon grillé, au beurre d'anchois}

\medskip

Pour six personnes prenez :

\medskip

\footnotesize
\begin{longtable}{rrrp{16em}}
  1 000 & grammes & de & saumon frais,                                                                    \\
    150 & grammes & de & beurre,                                                                          \\
     75 & grammes & d' & anchois\footnote{Engraulis encrasicholus, famille des Clupéidés.} salés.         \\
\end{longtable}
\normalsize

\index{Beurre d'anchois}
\label{pg0337} \hypertarget{p0337}{}
Préparez un beurre d'anchois de la façon suivante : lavez les anchois,
laissez-les tremper un peu pour les dessaler, essuyez-les, enlevez les arêtes,
pilez au mortier la chair avec le beurre ; passez le tout au tamis de crin.

Coupez le saumon en darnes d’un centimètre et demi d'épaisseur, puis faites-les
cuire, à feu vif, sans aucun assaisonnement, pendant trois à quatre minutes de
chaque côté, sur un gril chauffé au préalable et graissé.

Servez sur un plat chaud et envoyez en même temps le beurre d'anchois fondu,
dans une saucière.

\section*{\centering Pâté de saumon.}
\addcontentsline{toc}{section}{ Pâté de saumon.}
\label{pg0338} \hypertarget{p0338}{}
\index{Pâté de saumon}

Pour quinze à dix-huit personnes prenez :

\medskip

\index{Croûte pour pâtés}
1° pour la croûte :

\medskip

\footnotesize
\begin{longtable}{rrrp{16em}}
    550 & grammes & de & farine,                                                                          \\ 
    150 & grammes & de & beurre,                                                                          \\ 
    150 & grammes & d' & eau, à la température ambiante,                                                  \\ 
     50 & grammes & d' & huile d'olive non fruitée,                                                       \\ 
     15 & grammes & de & sel,                                                                             \\ 
        &         & 3  &j aunes d'œufs frais ;                                                            \\ 
\end{longtable}
\normalsize

\medskip

\index{Garniture pour pâtés}
2° pour la garniture :

\medskip

\footnotesize
\begin{longtable}{rrrrp{16em}}
  & \multicolumn{2}{r}{2 kilogrammes} & de & saumon frais, en une darne prise dans le milieu du poisson,  \\
  & 500 & grammes & de & brochet, soit un brochet moyen,                                                  \\
  & 375 & grammes & de & crevettes grises, cuites suivant le rite,                                        \\
  & 100 & grammes & de & saumon fumé,                                                                     \\
  & 100 & grammes & d' & œufs de homard crus,                                                             \\
  &  50 & grammes & de & beurre,                                                                          \\
  &  50 & grammes & de & madère,                                                                          \\
  &  30 & grammes & de & sel,                                                                             \\
  & 1/2 & gramme  & de & poivre,                                                                          \\
  & 1/2 & gramme  & de & paprika,                                                                         \\
  &     &         &  2 & jaunes d'œufs frais,                                                             \\
  &     &         &  1 & laitance de carpe ;                                                              \\
\end{longtable}
\normalsize

\medskip

3° pour la gelée :

\medskip

\footnotesize
\begin{longtable}{rrrp{16em}}
    750 & grammes    & de & tranche et nourrice,                                                          \\
    500 & grammes    & de & crosse et gîte,                                                               \\
    500 & grammes    & de & jarret de veau,                                                               \\
    500 & grammes    & de & vin blanc,                                                                    \\
     30 & grammes    & de & sel gris,                                                                     \\
    1/2 & gramme     & de & poivre,                                                                       \\
      2 & litres 1/2 & d’ & eau,                                                                          \\
        &            &  4 & carottes moyennes,                                                            \\
        &            &  2 & abatis de poulets,                                                            \\
        &            &  2 & poireaux moyens (le blanc seulement),                                         \\
        &            &  1 & navet moyen,                                                                  \\
        &            &  1 & oignon,                                                                       \\
        &            &  1 & petit morceau de panais,                                                      \\
        &            &  1 & petit morceau de céleri,                                                      \\
        &            &  1 & bouquet garni,                                                                \\
        &            &1/2 & pied de veau,                                                                 \\
        &            &1/2 & petite gousse d'ail,                                                          \\
        &            &1/2 & feuille de laurier,                                                           \\
        &            &    & quelques arêtes de sole et quelques têtes de poissons.                        \\
\end{longtable}
\normalsize

Avec tous les éléments du 3\textsuperscript{e} paragraphe, moins le vin blanc,
l'oignon, le bouquet garni et une carotte, faites un bouillon à bouilli perdu,
comme il est dit \hyperlink{p0201}{p. \pageref{pg0201}}. Concentrez-le de façon
à obtenir un litre de gelée d'aspic grasse.

\medskip

\textit{Préparation de la pâte}\footnote{La pâte pour pâté est désignée aussi
sous le nom de pâte à dresser.}. — Mélangez intimement les éléments du
1\textsuperscript{er} paragraphe de façon à obtenir une pâte bien homogène ;
roulez-la en boule ; laissez-la reposer pendant cinq heures environ.

\medskip

\textit{Préparation de la garniture}. — Séparez la darne de saumon en deux dans
le sens de la longueur ; réservez la peau et l'arête.

Découpez dans la partie la plus épaisse de la chair deux beaux morceaux ayant
la longueur et la largeur intérieures du pâté ; mettez-les, pendant deux
heures, à mariner dans le madère avec un peu de sel et de poivre. Réservez les
déchets.

Levez les filets du brochet ; réservez les déchets.

Mettez dans un mortier les déchets de saumon, les filets de brochet, le saumon
fumé, la laitance de carpe, les œufs de homard ; pilez ; ajoutez ensuite le
madère de la marinade, le reste du sel et du poivre, le paprika et passez le
tout au tamis de crin.

Faites un beurre de crevettes avec le beurre et les crevettes ; incorporez-le
ainsi que les jaunes d'œufs à la farce ci-dessus.

\medskip

\textit{Préparation de la gelée}. — Préparez un fumet maigre en faisant cuire
dans le vin blanc légèrement assaisonné la carotte, l'oignon et le bouquet
garni, l'arête et la peau du saumon, les déchets du brochet, les arêtes de sole
et les têtes de poissons, de manière à obtenir {\ppp150\mmm} grammes de fumet.

Mélangez gelée et fumet ; chauffez ; clarifiez le tout avec des blancs d'œufs.

\medskip

\textit{Dressage du pâté}. — Abaissez la pâte ; réservez-en une partie avec
laquelle vous ferez deux couvercles, l'un plus mince que l’autre ; réservez un
peu de pâte pour les bouchons des cheminées,

Garnissez avec cette abaisse un moule de forme rectangulaire, car la pâte, très
fine, risquerait de se briser au démoulage si l’on employait un moule orné de
nombreuses cannelures. Mettez sur la pâte au fond du moule une couche de farce
et au-dessus l'un des morceaux de saumon. Couvrez avec une autre couche de
farce ; mettez le second morceau de saumon et, par-dessus, le reste de la
farce ; lissez la surface. Couvrez le pâté, c'est-à-dire mouillez le couvercle
le plus mince ; collez-le sur la surface lissée ; faites dans ce couvercle de
nombreuses entailles pour le dégagement des gaz à la cuisson ; puis, mettez le
deuxième couvercle que vous percerez au centre de deux ou trois trous destinés
à faire cheminée ; fixez-le à la pince ; décorez le dessus et dorez-le au jaune
d'œuf.

\medskip

\textit{Cuisson}. — Au four chaud, pendant une heure et demie environ.

\medskip

\textit{Finissage}. — Retirez le pâté du four ; laissez-le refroidir un peu ;
puis, versez dedans, par les cheminées, de la gelée suffisamment refroidie pour
avoir une consistance aussi peu fluide que possible. Obstruez les ouvertures
avec les bouchons de pâte dorés à l'œuf et cuits à part.

Laissez refroidir complètement le pâté ; mettez-le sur un plat dont vous
décorerez le pourtour avec le reste de la gelée hachée et servez.

Si l'on veut avoir un pâté rigoureusement maigre, on remplacera le beurre de
crevettes par de l'huile de crevettes, on supprimera la gelée d'aspic grasse et
on préparera directement une gelée maigre en augmentant la quantité de vin, de
légumes et d'éléments maigres. On fera cuire, par exemple, dans un litre de vin
blanc tous les déchets de poissons, une plus grande quantité de têtes et
d’arêtes que précédemment, quelques poissons entiers, des légumes et un bouquet
garni. Mais, à moins d'employer de très grandes quantités de poissons et de
déchets de poissons, on sera conduit à ajouter à la gelée obtenue, qui
manquerait de fermeté, un peu de gélatine et naturellement cette gelée sera
moins fine.

Le pâté de saumon, préparé comme je l'ai indiqué, n'a rien de commun avec les
pâtés de saumon ordinaires dans lesquels la farce est constituée le plus
souvent par de la mie de pain plus ou moins imbibée d’un vague bouillon
d’arêtes et colorée par du carmin. Ici, le goût du saumon domine franchement et
la farce, très fine, à base de saumon et de brochet, relevée par le saumon
fumé, est parfumée délicieusement par le beurre de crevettes et les œufs de
homard ; de plus, l'addition de la laitance de carpe la rend aussi moelleuse
qu'une farce au foie gras.

Ce pâté de saumon, à farce maigre, sans mélange hétéroclite, est véritablement
délicat.

\sk

On peut préparer ce pâté de saumon avec des truffes ; mais cela me paraît
absolument inutile ; sincèrement, je préfère le pâté de saumon non truffé.

\section*{\centering Lamproie\footnote{Petromyzon marinus et fluviatilis,
famille des Pétromyzonidés.} rôtie à la broche.}
\addcontentsline{toc}{section}{ Lamproie rôtie à la broche.}
\index{Lamproie rôtie à la broche}

Pour quatre personnes prenez :

\footnotesize
\begin{longtable}{rrrrp{16em}}
  & \multicolumn{2}{r}{120 grammes} & de  & beurre,                                                       \\
  & \multicolumn{2}{r}{100 grammes} & de  & fumet de poisson,                                             \\
  & \multicolumn{2}{r}{100 grammes} & de  & vin de Madère, de Marsala ou de Porto, au goût,               \\
  & \multicolumn{2}{r}{100 grammes} & de  & fine champagne,                                               \\
  & \multicolumn{2}{r}{ 50 grammes} & d'  & oignons,                                                      \\
  & \multicolumn{2}{r}{ 15 grammes} & de  & farine,                                                       \\
  & \multicolumn{2}{r}{  5 grammes} & de  & sel,                                                          \\
  & \multicolumn{2}{r}{  4 grammes} & de  & persil,                                                       \\
  & \multicolumn{2}{r}{  3 grammes} & d’  & ail,                                                          \\
  & \multicolumn{2}{r}{1/2 décigramme} & de  & thym,                                                      \\
  & \multicolumn{2}{r}{1/2 décigramme} & de  & basilic,                                                   \\
  &       &         &  6  & grains de poivre,                                                             \\
  &       &         &  1  & lamproie pesant 800 grammes environ,                                          \\
  &       &         & 1/2 & feuille de laurier,                                                           \\
  &       &         &     & barde de lard,                                                                \\
  &       &         &     & citron.                                                                       \\
\end{longtable}
\normalsize

Enlevez la tête et le bout de la queue, puis nettoyez la lamproie, échaudez-la,
dépouillez-la et coupez-la en tronçons.

Préparez une marinade avec le vin, la fine champagne, les oignons coupés en
rondelles, le sel, le poivre, le persil, l'ail, le thym, le basilic, le
laurier : mettez dedans les tronçons de lamproie et laissez en contact pendant
deux heures.

Égouttez-les ensuite, bardez-les, fixez-les sur une broche et faites-les rôtir
à petit feu pendant trois quarts d'heure, en arrosant avec la moitié du beurre
fondu.

Faites un roux avec la farine et le reste du beurre, mouillez avec le fond de
poisson et la marinade passée, laissez cuire, dépouillez la cuisson,
réduisez-la à la moitié de son volume primitif et dégraissez-la.

Dressez sur un plat les tronçons de lamproie débarrassés des résidus de barde,
disposez autour des quartiers de citron et envoyez à part la sauce, dans une
saucière.

\section*{\centering Civet de lamproie.}
\addcontentsline{toc}{section}{ Civet de lamproie.}
\index{Civet de lamproie}
\label{pg0341} \hypertarget{p0341}{}

Prenez une lamproie vivante, coupez-lui la tête et la queue, recueillez le
sang, puis videz-la, plongez-la pendant trois minutes dans de l'eau bouillante,
dépouillez-la ensuite et coupez-la en morceaux.

Faites revenir : d'une part, les tronçons de lamproie avec des blancs de
poireaux jeunes dans de l'huile d'olive, pendant quelques instants ; d'autre
part, un hachis de lard, échalotes, persil, hysope, laurier, saupoudrez de
farine, mouillez avec du bon vin rouge, assaisonnez avec du sel et du poivre,
ajoutez lamproie et poireaux revenus ; laissez cuire ensemble pendant deux
heures.

Au dernier moment, liez la sauce avec le sang de la lamproie et du sang de
poulet, de façon qu'elle soit bien noire, goûtez, complétez l'assaisonnement,
sil y a lieu, avec du sel, du poivre, un peu de muscade et de sucre en poudre.

Servez avec des croûtons frits.

\sk

\index{Civet de lamproie en conserve}
Ce civet est excellent en conserve.

Pour le préparer ainsi, il suffira, après avoir fait revenir les ingrédients
mentionnés plus haut, de les mettre dans des boîtes en fer-blanc que l’on
soudera, et d'achever la cuisson pendant trois heures au bain-marie.

Lorsqu'on voudra s'en servir, on ouvrira les boîtes, on les fera simplement
chauffer au bain-marie, on versera le contenu dans un plat chaud et on garnira
avec des croûtons frits, préparés au dernier moment.

\sk

\index{Civet de saumon}
\index{Saumon en civet}
\index{Civet d'anguille de mer}
\index{Anguille de mer en civet}
Cette formule est applicable à l’anguille de mer. Elle est également excellente
avec le saumon.

\section*{\centering Anguilles grillées, sauce tartare.}
\addcontentsline{toc}{section}{ Anguilles grillées, sauce tartare.}
\index{Anguilles grillées, sauce tartare}

Prenez de belles anguilles, dépouillez-les, enlevez-en les arêtes délicatement
sans abîmer les poissons ; remplacez-les par une farce composée de champignons,
oignons, échalotes, ou fines herbes hachées, dûment assaisonnée, cuite au
beurre, puis liée avec des jaunes d'œufs frais. Cousez-les en leur conservant
leur forme, laissez-les entières ou tronçonnez-les, et faites-les pocher dans
un bon court-bouillon au vin blanc, pour les raidir. Égouttez-les, passez-les
successivement dans de l'œuf battu et dans de la mie de pain rassis tamisée,
puis faites-les griller.

En même temps, préparez la sauce. Réduisez en pâte fine des jaunes d'œufs durs,
assaisonnez avec sel, poivre, délayez avec de l'huile d'olive et du vinaigre,
ajoutez de la ciboulette ou de l'oignon haché fin, ou des fines herbes et des
cornichons hachés, un peu de moutarde, relevez avec une pointe de cayenne et
finissez avec de la mayonnaise. Les Anglais ajoutent volontiers de la sauce
Worcestershire ou de l'Harvey sauce.

Dressez les anguilles sur un plat entouré d'un cordon de persil et envoyez en
même temps la sauce dans une saucière.

\sk

\index{Crustacés, sauce tartare}
La sauce tartare accompagne parfaitement les crustacés, les poissons, les
viandes froides, le poulet et le lapereau grillés, la volaille froide.

\section*{\centering Matelote d’anguille aux raisins de Smyrne.}
\addcontentsline{toc}{section}{ Matelote d’anguille aux raisins de Smyrne.}
\index{Matelote d'anguille aux raisins de Smyrne}
\index{Anguilles en matelote}
\index{Matelote d'anguilles}

Pour quatre personnes prenez :

\medskip

\footnotesize
\begin{longtable}{rrrp{16em}}
    750 & grammes & d' & anguille,                                                                        \\
    400 & grammes & de & vin rouge,                                                                       \\
    150 & grammes & d' & eau,                                                                             \\
    150 & grammes & de & champignons,                                                                     \\
    100 & grammes & de & beurre,                                                                          \\
    100 & grammes & d' & oignons,                                                                         \\
     50 & grammes & de & fumet de poisson ou de glace de viande,                                          \\
     50 & grammes & de & raisins de Smyrne,                                                               \\
     50 & grammes & de & carotte,                                                                         \\
     30 & grammes & de & cognac,                                                                          \\
     20 & grammes & de & farine,                                                                          \\
      4 & grammes & d' & ail,                                                                             \\
        &         & 12 & croûtons,                                                                        \\
        &         &  1 & clou de girofle,                                                                 \\
        &         &    & bouquet garni,                                                                   \\
        &         &    & jus de citron,                                                                   \\
        &         &    & sel et poivre.                                                                   \\
\end{longtable}
\normalsize

Tronçonnez l'anguille.

Faites revenir les morceaux d'anguille avec les oignons entiers, l'ail, la
carotte, coupée en rondelles, dans {\ppp30\mmm} grammes de beurre ; enlevez le poisson ;
tenez-le au chaud.

Faites blondir la farine dans le même beurre de cuisson, mouillez avec eau,
vin, cognac ; ajoutez bouquet garni, clou de girofle, du sel et du poivre au
goût ; laissez cuire pendant trois quarts d'heure. Mettez alors l'anguille et
continuez la cuisson pendant trois quarts d'heure encore.

Faites revenir, à part, les champignons avec {\ppp30\mmm} grammes de beurre et le jus
d'un demi-citron.

Ébouillantez les raisins.

Un quart d'heure avant la fin, enlevez doucement les morceaux d’anguille ;
passez la sauce ; remettez anguille et sauce dans la casserole ; ajoutez le
fumet de poisson ou la glace de viande, les champignons, les raisins ; achevez
la cuisson.

Dressez les tronçons d'anguille sur un plat ; masquez-les avec la sauce ;
décorez avec les champignons et les croûtons frits dans le reste du beurre ;
servez.

Ce plat est excellent.

\section*{\centering Pâté d'anguille.}
\addcontentsline{toc}{section}{ Pâté d'anguille.}
\index{Pâté d'anguille}
\index{Anguilles en pâté}

Pour quinze à dix-huit personnes prenez :

\medskip

\index{Croûte pour pâtés}
1° pour la croûte :

\medskip

\footnotesize
\begin{longtable}{rrrp{16em}}
    600 & grammes & de & farine,                                                                          \\
    150 & grammes & de & beurre,                                                                          \\
    150 & grammes & d' & eau à la température ambiante,                                                   \\
     50 & grammes & d' & huile d'olive fine,                                                              \\
     15 & grammes & de & sel,                                                                             \\
        &         &  4 & jaunes d’œufs ;                                                                  \\
\end{longtable}
\normalsize

\medskip

\index{Garniture pour pâtés}
2° pour la garniture :

\medskip

\footnotesize
\begin{longtable}{rrrp{16em}}
    375 & grammes & de & crevettes grises,                                                                \\
    200 & grammes & de & chablis,                                                                         \\
    150 & grammes & de & champignons,                                                                     \\
    100 & grammes & de & porto blanc,                                                                     \\
     50 & grammes & de & beurre,                                                                          \\
     30 & grammes & de & sel,                                                                             \\
     10 & grammes & de & farine,                                                                          \\
      1 & gramme  & de & poivre,                                                                          \\
        &         & 12 & coquilles Saint-Jacques,                                                         \\
        &         & 12 & anchois dessalés,                                                                \\
        &         &  2 & anguilles de rivière vivantes, pesant ensemble 3 kilogrammes environ,            \\
        &         &  2 & jaunes d'œufs frais,                                                             \\
        &         &  1 & brochet pesant 400 grammes environ,                                              \\
        &         &  1 & merlan moyen,                                                                    \\
        &         &  1 & laitance de carpe,                                                               \\
        &         &  1 & grosse échalote ou 2 moyennes,                                                   \\
        &         &  1 & oignon moyen,                                                                    \\
        &         &  1 & petit bouquet garni,                                                             \\
        &         &    & carotte,                                                                         \\
        &         &    & beurre fondu,                                                                    \\
        &         &    & jus de citron,                                                                   \\
        &         &    & quatre épices,                                                                   \\
        &         &    & cayenne ;                                                                        \\
\end{longtable}
\normalsize

\medskip

3° pour la sauce :

\footnotesize
\begin{longtable}{rrrrp{16em}}
  & \multicolumn{2}{r}{ 750 grammes} & de & gîte de bœuf,                                                 \\
  & \multicolumn{2}{r}{ 600 grammes} & de & déchets de poissons (têtes, arêtes, parures de sole, merlan,
                                            barbue, congre, etc.),                                        \\
  & \multicolumn{2}{r}{ 500 grammes} & de & jarret de veau,                                               \\
  & \multicolumn{2}{r}{ 100 grammes} & de & lard de poitrine ou de jambon salé, non fumé.                 \\
  & \multicolumn{2}{r}{ 100 grammes} & de & vin blanc sec,                                                \\
  & \multicolumn{2}{r}{  80 grammes} & de & carottes,                                                     \\
  & \multicolumn{2}{r}{  75 grammes} & de & couenne maigre,                                               \\
  & \multicolumn{2}{r}{  65 grammes} & de & porto blanc,                                                  \\
  & \multicolumn{2}{r}{  50 grammes} & de & champignons,                                                  \\
  & \multicolumn{2}{r}{  50 grammes} & de & farine,                                                       \\
  & \multicolumn{2}{r}{  45 grammes} & d' & oignons,                                                      \\
  & \multicolumn{2}{r}{  40 grammes} & de & beurre clarifié,                                              \\
  & \multicolumn{2}{r}{  20 grammes} & de & beurre,                                                       \\
  & \multicolumn{2}{r}{  10 grammes} & de & persil,                                                       \\
  & \multicolumn{2}{r}{   4 grammes} & de & sel,                                                          \\
  & \multicolumn{2}{r}{   1 gramme 1/2} & de & poivre en grains,                                          \\
  &     & 2 litres    & d' & eau,                                                                         \\
  &     &             &    & graisse de porc,                                                             \\
  &     &             &    & thym,                                                                        \\
  &     &             &    & laurier.                                                                     \\
\end{longtable}
\normalsize

\medskip

La veille du jour où vous voudrez servir le pâté, préparez le fond suivant qui
sera la base de la sauce.

Désossez les viandes de boucherie, coupez la chair en morceaux, cassez fin les
os que vous ferez brunir plus ou moins au four.

Faites dorer, dans un peu de graisse, {\ppp50\mmm} grammes de carotte et
{\ppp20\mmm} grammes d'oignon émincés ; laissez pincer ; ajoutez les os ;
mouillez avec {\ppp1\mmm} litre {\ppp1\mmm}/{\ppp4\mmm} d'eau ; laissez mijoter
pendant {\ppp3\mmm} heures ; passez la cuisson.

Faites revenir dans un peu de graisse les viandes et la couenne coupées en
morceaux ; égouttez la graisse ; mouillez ensuite avec un peu de la cuisson
ci-dessus ; laissez tomber à glace deux ou trois fois en déglaçant chaque fois
avec un peu de cuisson des os. Déglacez définitivement avec le reste de la
cuisson, amenez à ébullition, écumez, dégraissez et continuez à faire cuire
pendant {\ppp6\mmm} à {\ppp7\mmm} heures en maintenant toujours le même volume
de liquide par des additions successives d’eau bouillante. Passez ce fond
à l'étamine.

Faites revenir dans un peu de graisse le lard ou le jambon coupé en petits
morceaux, ajoutez le reste des carottes émincées, {\ppp10\mmm} grammes
d'’oignon, le persil, le thym et le laurier ; laissez prendre couleur. Égouttez
la graisse ; mouillez avec {\ppp60\mmm} grammes de vin blanc ; réduisez de
moitié. Réservez cet appareil.

Faites dorer lentement la farine dans le beurre clarifié de façon à avoir un
roux brillant et lisse ; mouillez avec les {\ppp2\mmm}/{\ppp3\mmm} du fond,
amenez à ébullition, ajoutez l'appareil réservé et laissez cuire à tout petit
feu pendant six heures. Dépouillez fréquemment pendant la cuisson, passez au
chinois, refroidissez en vannant.

\medskip

Le lendemain, faites les opérations suivantes.

\medskip

\textit{Préparation de la pâte}. — Mélangez intimement tous les éléments du
premier paragraphe et travaillez de façon à obtenir une pâte bien homogène ;
roulez-la en boule ; laissez-la reposer pendant {\ppp5\mmm} heures.

\medskip

\textit{Préparation de la garniture}. — Tuez les anguilles, dépouillez-les,
videz-les, coupez-leur la tête, enlevez les arêtes ; mettez de côté les
déchets.

Prélevez sur chaque anguille deux tronçons aussi réguliers que possible et de
la longueur que vous voudrez donner au pâté ; réservez le reste.

Levez les filets du merlan et du brochet ; mettez-les de côté séparément ;
réservez les déchets.

Pilez et passez au tamis anchois et chair de merlan, assaisonnez avec {\ppp8\mmm} grammes
de sel et un peu de cayenne ; liez avec un jaune d'œuf et remplacez par cette
farce l'arête que vous avez enlevée aux tronçons d'anguille.

Ouvrez les coquilles Saint-Jacques, recueillez-les avec leur eau dans une
casserole et faites-les blanchir rapidement. Escalopez le blanc et le corail ;
réservez l'eau et les barbes.

Faites cuire les crevettes comme à l'ordinaire ; décortiquez-les ; réservez les
parures.

Pelez les champignons, passez-les dans du jus de citron, hachez-les, réservez les
pelures.

Passez la chair du brochet au tamis ; mettez-la dans un mortier avec les
crevettes décortiquées, les champignons hachés, le reste de la chair
d'anguille, la laitance de carpe et les barbes des coquilles Saint-Jacques ;
assaisonnez avec {\ppp15\mmm} grammes de sel, 1/{\ppp2\mmm} gramme de poivre et un peu de quatre
épices, au goût ; pilez jusqu'à obtention d'une farce fine.

Mettez dans une casserole les déchets d'anguilles, de merlan et de brochet, les
parures de crevettes, les pelures lavées des champignons, l'échalote, l'oignon,
{\ppp20\mmm} grammes de beurre, {\ppp2\mmm} ou {\ppp3\mmm} rondelles de carotte, le bouquet garni, le reste
du sel et le reste du poivre, un peu d'épices, au goût ; mouillez avec le
chablis, le porto et un peu d'eau des coquilles Saint-Jacques ; faites cuire
pendant une heure ; dégraissez, passez ce fond en pressant. Remettez-le sur le
feu, réduisez-le de façon à obtenir {\ppp150\mmm} grammes de liquide environ ;
dépouillez-le pendant la cuisson.

Faites un roux avec le reste du beurre et la farine, mouillez avec le fond de
poisson, laissez cuire pendant quelques minutes. Incorporez cette sauce à la
farce.

\medskip

\textit{Dressage du pâté}. — Abaissez la pâte, réservez-en une partie avec
laquelle vous ferez deux couvercles, l'un plus mince que l'autre et des
bouchons pour les cheminées. Garnissez avec cette abaisse un moule
rectangulaire allongé, lisse (la pâte, très fine, risquant de se briser au
démoulage avec un moule à cannelures), tapissez le fond et les parois du pâté
avec de la farce ; placez au fond deux tronçons d'anguille, au-dessus, une
couche de farce ; disposez côte à côte, dans la farce, suivant le grand axe et
au centre du pâté, les escalopes de coquilles Saint-Jacques, mettez au-dessus
les deux autres tronçons d'anguille, couvrez avec le reste de la farce ; lissez
la surface. Collez sur cette surface le couvercle le plus mince dans lequel
vous ferez de nombreuses entailles pour le dégagement des gaz, puis appliquez
le deuxième couvercle dans lequel vous percerez deux ou trois trous formant
cheminées ; fixez-le à la pince, décorez le dessus et dorez-le au jaune d'œuf
ainsi que les bouchons.

\medskip

\textit{Finissage de la sauce}. — Mettez dans une casserole le beurre non
clarifié, le reste des oignons et les champignons émincés, les déchets de
poissons, le sel, mouillez avec {\ppp750\mmm} grammes d'eau et le reste du vin blanc ;
chauffez, écumez, couvrez hermétiquement la casserole et faites cuire à feu
doux et régulier pendant une demi-heure. Ajoutez alors le poivre en grains,
laissez cuire encore pendant une dizaine de minutes, puis passez au tamis ce
fumet de poisson.

Réchauffez le fond, base de la sauce, préparé la veille, joignez-y le fumet de
poisson ; dépouillez encore, dégraissez et concentrez de façon à obtenir 700
grammes de sauce. Goûtez, complétez l'assaisonnement avec sel et poivre s'il
est nécessaire, corsez avec le porto. Passez la sauce.

\medskip

\textit{Cuisson du pâté}. — Pendant la préparation de la sauce, faites cuire le
pâté au four chaud pendant une heure et demie environ.

Démoulez le pâté, obturez les cheminées avec les bouchons de pâte cuits à part,
dressez-le sur un plat et servez chaud. Envoyez en même temps la sauce dans une
saucière.

Chaque tranche de pâté présentera en quinconce les éléments solides noyés dans
la farce, les angles étant formés par de l'anguille et le centre par les
escalopes de coquilles Saint-Jacques.

\sk

Comme variantes, on pourra remplacer les coquilles Saint-Jacques par des
escalopes de homard ou de saumon fumé ou non ; employer des morilles à la place
de champignons de couche ; ou truffer les anguilles, la farce et la sauce.

\section*{\centering Alose\footnote{Clupea alosa, famille des Clupéidés.} grillée, à l'oseille.}
\addcontentsline{toc}{section}{ Alose grillée, à l'oseille.}
\index{Alose grillée, à l'oseille}

L'alose est un poisson de mer qui remonte les rivières au printemps. Sa chair
est excellente, lorsqu'elle est très fraîche.

On peut préparer l'alose de bien des manières ; le procédé classique le plus
recommandable consiste à la faire griller ou à la faire rôtir à la broche et
à la servir avec de l'oseille,

En voici une formule concrète :

\medskip

Pour six personnes prenez :

\medskip

\footnotesize
\begin{longtable}{rrrrp{16em}}
  & \multicolumn{2}{r}{2 kilogrammes} & d’ & oseille,                                                     \\
  & \multicolumn{2}{r}{300 grammes}   & de & beurre,                                                      \\
  & \multicolumn{2}{r}{ 50 grammes}   & de & crème,                                                       \\
  &     &             &  3 & jaunes d'œufs frais,                                                         \\
  &     &             &  1 & alose, laitée de préférence, pesant 1 200 grammes environ,                   \\
  &     &             &    & persil haché,                                                                \\
  &     &             &    & jus de citron,                                                               \\
  &     &             &    & moutarde,                                                                    \\
  &     &             &    & sel et poivre.                                                               \\
\end{longtable}
\normalsize

Épluchez et lavez soigneusement l’oseille ; faites-la blanchir pendant quelques
minutes dans de l'eau bouillante ; écouttez-la.

Écaillez, videz et essuyez l'alose ; ciselez-la et enduisez-la de {\ppp50\mmm}
grammes de beurre ramolli suffisamment à la chaleur ; salez, poivrez ; puis,
faites-la cuire, sur un gril chauffé au préalable, pendant {\ppp15\mmm}
à {\ppp18\mmm} minutes de chaque côté, en l'arrosant avec la cuisson.

Mettez dans une casserole l'oseille blanchie, {\ppp125\mmm} grammes de beurre,
du sel, du poivre, achevez la cuisson, puis liez avec les jaunes d'œufs et la
crème.

\index{Beurre maître-d'hôtel à la moutarde}
Préparez un beurre maître d'hôtel à la moutarde avec le reste du beurre, du
persil, du jus de citron, de la moutarde, du sel et du poivre au goût.

\index{Garniture pour poissons}
Dressez l'alose sur un plat et servez en envoyant en même temps l'oseille dans
un légumier et le beurre maître d'hôtel dans une saucière.

Comme variantes, on pourra remplacer le beurre maitre d'hôtel par l'un des
beurres composés ou par l'une des sauces accompagnant la sole grillée,
\hyperlink{p0355}{p. \pageref{pg0355}}.

\section*{\centering Foies de lottes\footnote{Lota vulgaris, famille des Gadidés.} au chambertin, en turban de homard.}
\addcontentsline{toc}{section}{ Foies de lottes au chambertin, en turban de homard.}
\index{Foies de lottes au chambertin, en turban de homard}

Pour dix à douze personnes prenez :

\medskip

\footnotesize
\begin{longtable}{rrrp{16em}}
    600 & grammes & de & beurre,                                                                          \\
    500 & grammes & de & pommes de terre,                                                                 \\
    500 & grammes & de & chambertin,                                                                      \\
    250 & grammes & de & crevettes grises,                                                                \\
    250 & grammes & de & petits champignons de couche ;                                                   \\
    125 & grammes & d' & oignons,                                                                         \\
    100 & grammes & d' & eau,                                                                             \\
     30 & grammes & de & farine,                                                                          \\
        & 1 litre & de & moules.                                                                          \\
        &         & 36 & huîtres,                                                                         \\
        &         & 18 & écrevisses,                                                                      \\
        &         & 12 & foies de lottes de grosseur moyenne,                                             \\
        &         &  2 & jaunes d'œufs frais,                                                             \\
        &         &  1 & homard œuvé,                                                                     \\
        &         &    & truffes à volonté,                                                               \\
        &         &    & madère,                                                                          \\
        &         &    & jus de citron,                                                                   \\
        &         &    & bouquet garni,                                                                   \\
        &         &    & sel et poivre.                                                                   \\
\end{longtable}
\normalsize

Brossez, lavez et pelez les truffes ; réservez les pelures ; hachez-les.

Faites cuire en même temps :

\textit{a}) le homard, comme d'ordinaire : passez la chair et les œufs au tamis ;
réservez les déchets ;

\textit{b}) les écrevisses et les crevettes suivant le rite : épluchez-les ;
mettez de côté les queues ; réservez les parures ;

\textit{c}) les pommes de terre à la vapeur : pelez-les ; passez-les au tamis ;

\textit{d}) les truffes dans du madère.

\index{Beurre de crustacés}
Préparez un beurre de crustacés avec {\ppp500\mmm} grammes de beurre, les
déchets de homard, les parures d'écrevisses et de crevettes.

Mélangez intimement purée de homard, purée de pommes de terre, {\ppp400\mmm}
grammes de beurre de crustacés et pelures de truffes ; liez avec les jaunes
d'œufs, assaisonnez avec sel et poivre. Emplissez un moule à couronne avec ce
mélange ; faites cuire au bain-marie ; tenez au chaud.

Préparez une sauce matelote : mettez dans une casserole {\ppp50\mmm} grammes de
beurre et la farine ; faites roussir légèrement ; ajoutez ensuite les oignons
et le bouquet garni ; mouillez avec le chambertin et l’eau ; salez, poivrez ;
laissez cuire pendant une heure.

Passez la sauce : puis, faites cuire dedans les foies de lottes. Sortez-les
ensuite de la cuisson ; égouttez-les ; tenez-les au chaud. Tenez également la
sauce au chaud.

Faites s'ouvrir les moules à feu vif, comme d'ordinaire ; retirez-les des coquilles :
tenez-les au chaud.

Ouvrez les huîtres ; faites-les pocher dans leur eau.

Faites cuire les champignons pelés dans le reste du beurre avec un peu de jus
de citron.

Préparez un salpicon avec huîtres, moules, champignons, queues d'écrevisses
et de crevettes, truffes, le reste du beurre de crustacés, la sauce matelote et les
cuissons des foies de lottes, des champignons, des huîtres et des truffes.

Escalopez les foies de lottes.

Démoulez le turban de homard sur un plat ; dressez les escalopes de foies de
lottes sur le bord du turban ; versez le salpicon dans l'intérieur et servez.

\begin{center}
{\small « Pour une lotte, une femme donnerait sa cotte », dit un proverbe.}
\end{center}

\begin{center}
Que ne donnerait-elle pas pour un semblable plat !
\end{center}

\smallskip

\sk

\bigskip

\index{Filets de poissons au chambertin}
On peut préparer de la même manière des filets de différents poissons, entre
autres des filets de soles,.

\section*{\centering Bar à la gelée.}
\addcontentsline{toc}{section}{ Bar à la gelée.}
\index{Bar à la gelée}
\label{pg0350} \hypertarget{p0350}{}

Pour six personnes prenez un bar de {\ppp1\mmm} {\ppp500\mmm} grammes environ ;
faites-le cuire, de préférence, dans de l'eau de mer ou, à défaut, dans de
l'eau salée avec du sel marin. La cuisson doit être faite à liquide
frissonnant ; elle doit durer trois quarts d'heure environ.

Laissez refroidir le poisson.

En même temps, préparez une belle gelée, avec eau, vin blanc, poissons et
débris de poissons, pied de veau, légumes, bouquet garni, oignons, échalotes,
ail, épices, aromates, sel et poivre, au goût ; passez-la, clarifiez-la.

Dressez le poisson sur un plat ; décorez-le avec des feuilles d’estragon ;
masquez-le avec la gelée ; laissez prendre.

On peut servir le poisson tel quel, en l'accompagnant de sauce mayonnaise
verte ou de sauce mayonnaise à la moutarde.

\sk

Pour faire la sauce mayonnaise à la moutarde pour six personnes, prenez :

\footnotesize
\begin{longtable}{rrrp{16em}}
    150 & grammes & d' & huile d'olive,                                                                   \\
      2 & grammes & de & poivre,                                                                          \\
        &         &  2 & jaunes d'œufs frais,                                                             \\
\setlength\tabcolsep{.15em}
        &         &    &  $\left.
                               \begin{tabular}{lll}
                                moutarde,             \\
                                jus de citron,        \\
                                sel.                  \\
                               \end{tabular}
                             \right\} $ au goût.
\end{longtable}
\normalsize

Préparez une mayonnaise ordinaire comme il est dit
\hyperlink{p0323-2}{p. \pageref{pg0323-2}} ; incorporez-y la moutarde et
homogénéisez le tout.

\sk

\index{Garniture pour poissons}
\index{Barquettes de mousse de crevettes}
Une façon élégante de présenter le bar à la gelée consiste à entourer le
poisson de barquettes de pâtisserie garnies de salade de légumes, ou encore de
mousse\footnote{
\index{Définition des mousses}
\index{Mousses (Définition des)}
Le mot mousse a des acceptions différentes en gastronomie. En
cuisine, les mousses sont des préparations obtenues en travaillant sur glace
des purées, principalement avec de la crème fouettée et accessoirement avec du
beurre, des gelées et des veloutés. On prépare surtout des mousses avec des
crustacés, du gibier et du foie gras.

Les mousses entremets proprement dites sont des gelées aromatisées que l'on
fait mousser en les travaillant sur glace. On désigne aussi quelquefois sous le
nom de mousses des crèmes fouettées aromatisées et sanglées dans un moule, il
est préférable de leur réserver le nom de mousses-crèmes.

En pâtisserie, le qualificatif de mousse accolé parfois au nom de certains
gâteaux n'a d'autre prétention que d'indiquer qu'ils sont légers ; il ne
préjuge en rien ni de leur composition, ni de leur mode de préparation.} de
crevettes grises, truffée ou non, qu'on masquera avec de la gelée maigre et
qu'on décorera avec des crevettes roses ou des huîtres, par exemple.

\sk

Pour faire douze croûtes de barquettes\footnote{
\index{Définition des barquettes}
\index{Barquettes (Définition des)}
\index{Barquettes de crustacés}
\index{Barquettes de poissons}
\index{Barquettes de confitures}
\index{Barquettes de fruits}
\index{Barquettes de légumes}
On appelle barquettes des sortes de petites croûtes ou de petites tartelettes
ovales qu'on garnit de purées ou de salpicons de crustacés, de poissons, de
volaille, de gibier, de légumes, de confitures, de fruits.

La croûte des barquettes est salée ou sucrée suivant que leur remplissage est
lui-même salé ou sucré.} prenez :

\bigskip

\footnotesize
\begin{longtable}{rrrp{16em}}
    150 & grammes & de & farine,                                                                          \\
    100 & grammes & de & beurre,                                                                          \\
     50 & grammes & d' & eau,                                                                             \\
        &         &    & sel.                                                                             \\
\end{longtable}
\normalsize

Préparez une pâte homogène ; faites-en une abaisse mince et chemisez avec cette
abaisse des moules à barquettes ; emplissez l'intérieur avec des haricots ou
des cailloux lavés ; faites cuire au four ; laissez refroidir.

\sk

Pour préparer une mousse de crevettes, faites cuire des crevettes grises ;
décortiquez les queues ; réservez les parures. Passez les queues en purée ;
faites avec les parures et du beurre un beurre de crevettes. Travaillez sur
glace la purée de crevettes avec le beurre de crevettes, de la gelée maigre et
de la crème fouettée, jusqu'à obtention d'une belle mousse.

\sk

On peut apprêter à la gelée d’autres poissons de mer.

\section*{\centering Rougets.}
\addcontentsline{toc}{section}{ Rougets.}
\index{Rougets}

Les rougets sont des poissons très fins de la famille des Mullidés. Mais ils
s'altèrent très rapidement ; aussi ne sont-ils réellement parfaits qu'au bord
de la mer. Les meilleurs paraissent être ceux de la Méditerranée, surtout ceux
de la côte algérienne.

Deux espèces sont dignes de retenir l'attention des gastronomes ; ce sont : le
surmulet ou rouget barbet, « Mullus surmuletus » et le mulle rouget ou rouge
d'Yport, « Mullus barbatus ».

Les Romains étaient très friands des rougets. et ils dépensèrent des sommes
folles pour la construction de viviers destinés à les recevoir.

\section*{\centering Rougets au beurre.}
\addcontentsline{toc}{section}{ Rougets au beurre.}
\index{Rougets au beurre}

Lorsqu'on à des rougets fraîchement pêchés, le mieux est de les faire cuire
simplement sur le gril, après les avoir vidés, nettoyés, ciselés, assaisonnés et
trempés dans du beurre clarifié ; puis de les servir masqués d’une sauce faite de
leurs foies maniés avec du beurre et du persil haché.

\sk

On peut encore faire cuire les rougets à la poêle, dans du beurre auquel on
donnera un peu de corps, à la fin, en écrasant dedans les foies des poissons.

\section*{\centering Rougets grillés, au fenouil.}
\addcontentsline{toc}{section}{ Rougets grillés, au fenouil.}
\index{Rougets grillés, au fenouil}

Disposez sur une grille de lèchefrite un peu de fenouil frais (une douzaine de
brins) ; couchez dessus les rougets vidés, nettoyés, ciselés, assaisonnés et
passés dans du beurre clarifié ; faites-les griller des deux côtés, à feu
dessus, en prenant soin que le fenouil ne brûle pas, et en les retournant
délicatement pour ne pas les briser.

Dressez-les sur un plat chaud et masquez-les avec un beurre maitre d'hôtel
fortement relevé par du jus de citron.

Les rougets ainsi préparés seront parfumés bien autrement que si on les traite
par le procédé classique qui consiste à les faire mariner d'abord dans de
l'huile et du jus de citron aromatisés avec du fenouil et à les faire griller
ensuite.

\sk

Le mode de cuisson que je viens d'indiquer est général : on peut l'appliquer
à d'autres substances alimentaires, en variant les herbes aromatiques. C'est
ainsi, notamment, qu'on pourra préparer de la sorte un lièvre ou un lapin au
thym et au serpolet, ou des grives au genièvre.

\section*{\centering Rougets aux tomates.}
\addcontentsline{toc}{section}{ Rougets aux tomates.}
\index{Rougets aux tomates}

Dès qu'on s'éloigne des bords de la mer, le poisson étant moins frais, les
procédés de cuisson au beurre perdent de leurs avantages ; il est alors
préférable de préparer les rougets de la façon suivante.

\medskip

Pour quatre personnes prenez :

\footnotesize
\begin{longtable}{rrrrp{16em}}
  & \multicolumn{2}{r}{500 grammes} & de & lard frais,                                                    \\
  & \multicolumn{2}{r}{500 grammes} & de & tomates,                                                       \\
  & \multicolumn{2}{r}{ 50 grammes} & de & beurre,                                                        \\
  & \multicolumn{2}{r}{  7 grammes} & de & sel blanc,                                                     \\
  & \multicolumn{2}{r}{  2 grammes} & d' & échalote,                                                      \\
  & \multicolumn{2}{r}{  1 gramme } & d' & ail,                                                           \\
  & \multicolumn{2}{r}{3 décigrammes} & de & poivre blanc,                                                \\
  & \multicolumn{2}{r}{1 décigramme}  & de & quatre épices,                                               \\
  & \multicolumn{2}{r}{1/2 décigramme} & de & muscade râpée,                                              \\
  &     &             &  4 & petits rougets pesant ensemble {\ppp500\mmm} grammes environ,                \\
  &     &             &  1 & oignon piqué d'un clou de girofle,                                           \\
  &     &             &  1 & petit bouquet garni,                                                         \\
  &     &             &    & farine.                                                                      \\
\end{longtable}
\normalsize

Mettez dans une casserole les tomates, l'oignon et le bouquet garni ; laissez
cuire, passez en purée.

Faites fondre le beurre ; lorsqu'il sera bien chaud, sans être coloré, jetez
dedans l'ail et l'échalote ciselés ; mettez ensuite les rougets passés au
préalable dans de la farine, assaisonnez avec sel, poivre, quatre épices et
muscade, puis laissez-les dorer des deux côtés. Cette opération demande un
quart d'heure environ. Ajoutez alors la purée de tomates et achevez la cuisson
de l'ensemble pendant dix minutes. Servez.

\sk

On peut préparer de même d'autres poissons et notamment des darnes de thon.

\section*{\centering Soles.}
\addcontentsline{toc}{section}{ Soles.}
\index{Soles}

Les soles, dont le nom dérive du mot latin « solea » (sandale) par lequel elles
étaient désignées à Rome à cause de leur forme, sont des poissons de la famalle
des Pleuronectidés.

Elles sont représentées en Europe par cinq ou six espèces, dont deux sont
vendues couramment à Paris : la sole commune « Pleuronectes solea », à peau
dorsale brunâtre. et la sole lascaris « Pleuronectes lascaris », à peau dorsale
grisâtre, cette dernière la plus fine.

La sole est incontestablement la reine des poissons de mer. Sa chair blanche,
déliçate, légère et succulente plaît à tout le monde : petits et grands, jeunes
et vieux, bien portants et malades l’adorent et s'en délectent. Aussi, les
cuisiniers se sont-ils ingéniés à créer de nombreuses façons de la préparer, et
ils ont déployé toutes les ressources de leur art pour en faire apprécier la
saveur. Dans l'impossibilité matérielle de les détailler toutes, ce qui du
reste serait sans intérêt, je vais montrer que le nombre des combinaisons
possibles est pour ainsi dire infini, que chaque cuisinier, chaque amateur peut
facilement créer un plat personnel, soit qu'il traite le poisson entier, soit
qu'il en apprête seulement les filets.

Tous les procédés de cuisson de la sole sont compris dans les types suivants :

1° Grillade ;

2° Pochage à l'eau ou au court-bouillon ;

3° Cuisson au vin, suivie ou non de gratinage ;

4° Cuisson au beurre, à la crème, à l'huile ou à la graisse, suivie ou non de
glaçage ou de gratinage ;

5° Friture.

Voici quelques généralités sur chacun de ces modes de cuisson.

\begin{center}
1° \textit{Sole grillée.}
\end{center}
\label{pg0355} \hypertarget{p0355}{}

Pour préparer la sole grillée, on commence par la parer, on en détache plus ou
moins les filets qu'on cisèle, on arrose le poisson avec du beurre fondu ou de
l'huile et du jus de citron, on le passe ou non dans de la farine ou dans de la
mie de pain rassis tamisée et, dans le premier cas, on l'arrose encore avec du
beurre fondu ou de l'huile ; on le fait griller ensuite doucement, puis on le
sert tout simplement avec des tranches de citron ou avec une sauce :
maitre-d'hôtel, hollandaise, béarnaise, sauce diable, etc. ; 
\index{Beurres composés} 
ou avec un beurre composé : beurre de cresson, d'échalotes, d'ail, de
ravigote, de raifort, de paprika, de moutarde, d’anchois, de saumon fumé, de
crevettes, d'écrevisses, de langouste, de homard, de corail d'oursins, de
corail ou d'œufs de langouste ou de homard, de caviar, de
poutargue\footnote{Œufs de muges (mulets) séchés, salés et agglomérés.}, de
laitances, de truffes, etc.
\index{Garniture pour poissons} 
Enfin, on garnit le
plat, à volonté, avec des huîtres, des moules, cuites dans leur eau, des
laitances pochées dans un court-bouillon relevé ; des pommes de terre noisette,
duchesse, soufflées, chip ; des truffes, etc.

\begin{center}
2° \textit{Sole pochée.}
\end{center}

La sole, pochée simplement dans de l'eau de mer ou dans de l'eau salée
bouillante, est servie avec une sauce hollandaise ou du beurre fondu et des
pommes de terre cuites à l'eau ou à la vapeur. Lorsqu'elle est pochée dans un
court-bouillon aromatisé au goût, elle est servie ordinairement avec une sauce
constituée par le court-bouillon réduit et monté au beurre.

\begin{center}
3° \textit{Sole au vin\footnote{ On peut ranger dans cette classe la sole au cidre.}.}
\end{center}

C'est dans la préparation de la sole au vin que l'ingéniosité des cuisiniers
s'est surtout manifestée. Elle à donné naissance à de nombreuses combinaisons
qui portent chacune un nom particulier ; je ne mentionnerai ici que celles qui
sont classiques : sole au vin blanc, sole normande, sole Cardinal, sole au
gratin, sole Mornay.

En principe, la cuisson d'une sole au vin est faite avec vin et aromates, dans
un plat enduit de beurre\footnote{Dans certaines contrées on emploie l'huile ou
la graisse à la place du beurre.}, ou mieux encore avec un fumet de poisson
aromatisé, au vin. La sole, dressée sur un plat, est masquée par une sauce,
puis servie accompagnée ou non d'une garniture. Or, rien qu'en faisant varier
la nature du vin, depuis les vins blancs et les vins rouges ordinaires
jusqu'aux vins classés et en puisant seulement dans la gamme des vins français,
on a déjà de nombreuses variantes. Mais un certain nombre de vins étrangers
peuvent aussi être employés à cet usage. On peut encore corser les préparations
par une addition d'alcool, telle la fine champagne, par du vermouth, et les
aromatiser avec des liqueurs. En ce qui concerne les condiments, dont on peut
faire usage en proportions variables, il suffira de citer l'essence de
champignons ou de truffes, l’eau d'huîtres ou de moules, les différents
poivres, le cayenne, le paprika, les piments, la muscade, les épices, la
tomate, l'oignon, l'échalote, l'ail, la moutarde, les nombreuses fines herbes
et plantes odoriférantes, le vinaigre, le jus de citron ou d'orange, etc.

La sauce est généralement à base de velouté maigre ; elle est liée à la farine,
à la fécule ou aux jaunes d'œufs ; elle est montée à la crème et au beurre
ordinaire ou à l’un des nombreux beurres composés dont j'ai parlé plus haut ;
mais on peut aussi employer d'autres sauces.

\index{Garniture pour poissons}
Le nombre des garnitures possibles est également très grand : escalopes de
homard ou de langouste ; filets d'anchois ; goujons frits ; brandade de morue ;
corail d'oursins ; corail et œufs de langouste ou de homard ; caviar ;
poutargue ; laitances entières ou en purée présentées ou non dans des
barquettes de croûte fine ; foies de certains poissons, entre autres foies de
lottes ou de raies ; farces et quenelles maigres ; fruits de mer tels que
huîtres, moules, coquilles Saint-Jacques, crevettes, écrevisses ; champignons,
en particulier morilles et truffes ; tomates, épinards, aubergines, courgettes,
concombres, piments doux, fonds d'artichauts, pommes de terre, olives, julienne
de légumes variés ; citron, oranges, noix, noisettes, amandes, raisins secs ;
riz sous la forme de risotto aux truffes blanches, ou de pilaf safrané ou non ;
croûtons frits, pâtes, etc.

\sk

Il est inutile de dire qu'on peut préparer de même des filets de soles au vin.
Ces derniers peuvent aussi être servis froids, par exemple avec de la gelée
maigre, avec une mousse d'écrevisses ou une mousse de laitances au raifort ; en
\index{Aspic de filets de soles}
\index{Aspic de sole}
\index{Filets de soles au vin}
\index{Filets de soles en salade}
aspic avec une sauce chaud-froid maigre aux écrevisses ; en salade avec une
mayonnaise, etc.

\begin{center}
4° \textit{Sole au beurre}\footnote{On peut placer dans ce groupe les préparations
                                   similaires de soles à la crème, à l'huile ou à la graisse.}.
\end{center}

Le type de la sole au beurre est la sole dite « à la meunière » qu'on prépare
de la façon suivante : après avoir assaisonné la sole avec sel et poivre et
l'avoir passée dans de la farine, on la fait cuire dans une poêle avec du
beurre ; on y ajoute un peu de jus de citron ; on sale et on poivre encore ; on
saupoudre le poisson de persil blanchi haché, on l'arrose de beurre noisette et
on sert immédiatement. Ici encore, on peut faire varier les trois éléments de
la préparation : par exemple, en ajoutant à la cuisson de l'essence de
champignons ; en remplacant le beurre ordinaire par un beurre composé et en le
condimentant au goût ; en substituant au beurre noisette une sauce au beurre ;
enfin, en accompagnant le plat de l’une des nombreuses garnitures indiquées
plus haut.

\begin{center}
5° \textit{Sole frite.}
\end{center}

\index{Filets de soles frits}
On fait frire la sole à la graisse\footnote{Dans le Midi, on emploie volontiers
la graisse d'oie.} ou à l'huile, soit après l'avoir passée simplement dans du
lait puis dans de la farine, soit après l'avoir enrobée dans de la pâte ou
après l'avoir panée. Dans les trois cas, on la sert entourée de persil frit et
de tranches de citron.

\index{Filets de soles frits (en white bait)}
On peut faire frire de même une julienne de filets de soles ; on obtiendra
ainsi, jusqu'à un certain point, l'illusion d'une friture de white bait. On
peut encore faire frire la sole après l'avoir enduite d'une sauce, telle que la
sauce Villeroi, et l'avoir enrobée ensuite dans de la pâte.

\sk

La préparation classique connue sous le nom de sole Colbert consiste, après
avoir brisé l'arête avant la cuisson, à la retirer une fois la sole frite et
à la remplacer par un beurre maître d'hôtel.

\sk

Comme variante de la sole Colbert, on peut remplacer le beurre maître d'hôtel
par un beurre composé garni d'écrevisses et de truffes et relevé par de la
purée de crevettes.

\sk

\index{Barbue au vin}
\index{Barbue grillée}
\index{Barbue pochée}
\index{Turbot au vin}
\index{Turbot grillé}
\index{Turbot poché}
Tous ces modes de préparation peuvent être appliqués à d'autres poissons,
notamment à la barbue, au turbotin, aux filets de barbue, de turbotin et de
turbot. Ils conviennent encore aux filets de truite, de lavaret, etc.

\sk

Je vais donner maintenant un certain nombre de formules concrètes de
préparation de soles.

\section*{\centering Sole au vin blanc, sauce à la crème.}
\addcontentsline{toc}{section}{ Sole au vin blanc, sauce à la crème.}
\index{Sole au vin banc, sauce à la crème}

Pour quatre personnes prenez :

\medskip

\setlength\tabcolsep{.15em}
\footnotesize
\begin{longtable}{rrrrp{16em}}
  & 200 & grammes & de & vin blanc,                                                                       \\
  & 125 & grammes & de & crème,                                                                           \\
  & 100 & grammes & de & champignons de couche,                                                           \\
  &  80 & grammes & de & beurre,                                                                          \\
  &  10 & grammes & d' & échalote hachée fin,                                                             \\
  &  10 & grammes & de & chapelure,                                                                       \\
  &   4 & grammes & de & sel,                                                                             \\
  &   3 & grammes & de & persil haché,                                                                    \\
  &   2 & grammes & de & poivre,                                                                          \\
  25 & \multicolumn{2}{r}{centigrammes} & de & paprika,                                                   \\
  &     &         &  4 & belles écrevisses,                                                               \\
  &     &         &  1 & sole pesant 750 grammes,                                                         \\
  &     &         &    & jus de citron.                                                                   \\
\end{longtable}
\normalsize

Faites cuire les écrevisses.

Pelez les champignons, passez les chapeaux dans du jus de citron, hachez fin
les pieds.

Foncez un plat allant au feu avec {\ppp40\mmm} grammes de beurre, mettez dedans la sole
vidée et dépouillée, assaisonnez avec le sel, le poivre et le paprika, ajoutez
le persil, du jus de citron, mouillez avec le vin blanc, couvrez le tout avec
un papier beurré pour conserver à la sole son humidité et l'empêcher de se
colorer ; faites cuire au four pendant une vingtaine de minutes.

Faites fondre dans une casserole le reste du beurre sans le laisser roussir ;
saisissez dedans l'échalote ; mettez les chapeaux des champignons et les pieds
hachés, laissez cuire pendant cinq minutes, ajoutez la chapelure, puis versez
le tout sur la sole et achevez la cuisson qui doit durer encore une douzaine de
minutes. Au dernier moment, mettez la crème, mélangez-la aux éléments de la
sauce, en agitant le plat. Chauffez sans laisser bouillir.

Décorez le plat avec les chapeaux des champignons et les écrevisses ; servez.

\sk

\index{Brochet au vin blanc, sauce à la crème}
\index{Brème au vin blanc, sauce à la crème}
\index{Brochet rôti sauce à la crème}
\index{Brème rôtie sauce à la crème}
On peut préparer de même d'autres poissons, notamment le brochet et la brème,

\section*{\centering Sole normande.}
\addcontentsline{toc}{section}{ Sole normande.}
\index{Sole normande}

Pour huit personnes prenez :

\medskip

\footnotesize
\begin{longtable}{rrrp{16em}}
    300 & grammes & d' & eau,                                                                             \\
    250 & grammes & de & vin blanc,                                                                       \\
    150 & grammes & de & beurre,                                                                          \\
    125 & grammes & de & champignons de couche,                                                           \\
     65 & grammes & de & crème,                                                                           \\
     30 & grammes & de & calvados,                                                                        \\
      2 & grammes & de & poivre en grains,                                                                \\
        & 1 litre & de & moules,                                                                          \\
        &         & 32 & belles crevettes bouquet,                                                        \\
        &         & 16 & huîtres,                                                                         \\
        &         & 16 & éperlans,                                                                        \\
        &         &  2 & soles pesant chacune 600 grammes environ,                                        \\
        &         &  2 & jaunes d'œufs frais,                                                             \\
        &         &  1 & merlan pesant 300 grammes environ,                                               \\
        &         &  1 & bouquet garni,                                                                   \\
        &         &    & légumes de pot-au-feu,                                                           \\
        &         &    & mie de pain rassis tamisée,                                                      \\
        &         &    & pain anglais,                                                                    \\
        &         &    & farine,                                                                          \\
        &         &    & jus de citron,                                                                   \\
        &         &    & huile,                                                                           \\
        &         &    & muscade,                                                                         \\
        &         &    & sel et poivre.                                                                   \\
\end{longtable}
\normalsize

Préparez un court-bouillon avec l'eau, le vin, le calvados, les légumes, le
bouquet garni, le poivre en grains et du sel ; laissez-le cuire pendant vingt
minutes, passez-le.

Pelez les champignons ; réservez les pelures.

Faites cuire en même temps :

\textit{a}) les crevettes, pendant quelques minutes dans le court-bouillon ;

\textit{b}) les champignons dans {\ppp20\mmm} grammes de beurre et du jus de citron.

Faites pocher les huîtres dans leur eau.

Mettez les moules dans une casserole et faites-les s'ouvrir à feu vif.

Tenez au chaud crevettes, champignons, huîtres et moules.

Réunissez la cuisson des crevettes, l'eau des huîtres et celle des moules,
ajoutez le merlan coupé en morceaux, les pelures des champignons, un peu de
muscade, laissez cuire ; vous obtiendrez nn fumet aromatisé ; concentrez-le,
passez-le.

Disposez les soles dans un plat foncé de {\ppp30\mmm} grammes de beurre, mouillez avec
le fumet, couvrez avec un papier beurré et faites cuire au four pendant une
vingtaine de minutes.

Préparez la sauce normande : maniez un peu de farine avec {\ppp10\mmm} grammes de beurre,
mouillez avec la cuisson des soles et celle des champignons, liez avec les
jaunes d'œufs, puis montez la sauce avec {\ppp75\mmm} grammes de beurre et la crème ;
ajoutez un peu de jus de citron et complétez l'assaisonnement s'il y a lieu
avec sel et poivre.

Coupez le pain anglais en huit petites tranches ; faites-les dorer dans le
reste du beurre.

Passez les éperlans dans de la mie de pain rassis tamisée, faites-les frire dans
de l'huile bouillante.

Masquez les soles avec la sauce : garnissez le plat avec les crevettes, les
huîtres, les moules, les éperlans, les champignons et les tranches de pain
anglais ; chauffez pendant un moment au four et servez.

\sk

Certaines personnes font entrer dans la garniture des écrevisses et des goujons ;
il me semble préférable, pour conserver au plat son caractère marin dans toute
sa pureté, de ne pas mettre d'écrevisses et d'employer des éperlans frits,

\sk

On pourra rendre le plat plus riche en y ajoutant des truffes qu'on fera cuire
dans du madère. Les pelures des truffes parfumeront le fumet, le madère de
cuisson corsera la sauce et les truffes, entières ou émincées, entreront dans
la garniture.

\section*{\centering Sole soufflée, sauce hollandaise\footnote{
\index{Définition de la sauce hollandaise}
\index{Sauce hollandaise (Définition de la)}
                    La sauce hollandaise est une émulsion chaude de beurre et de
                    jaunes d'œufs.} vert-pré.}
\addcontentsline{toc}{section}{ Sole soufflée, sauce hollandaise  vert-pré.}
\index{Sole soufflée, sauce hollandaise vert-pré}

Pour douze personnes prenez :

\index{Définition de la sauce hollandaise}
\index{Sauce hollandaise (Définition de la)}

\medskip

1° pour la sole soufflée :

\medskip

\footnotesize
\begin{longtable}{rrrp{16em}}
    250 & grammes & de & bouillon de poisson,                                                             \\
    250 & grammes & de & vin blanc,                                                                       \\
    200 & grammes & de & crème épaisse,                                                                   \\
    100 & grammes & de & beurre,                                                                          \\
    100 & grammes & de & pommes de terre,                                                                 \\
    100 & grammes & de & champignons de couche,                                                           \\
        &         & 48 & moules,                                                                          \\
        &         & 12 & écrevisses,                                                                      \\
        &         &  4 & jaunes d'œufs,                                                                   \\
        &         &  2 & blancs d'œufs,                                                                   \\
        &         &  1 & sole pesant 1 kilogramme environ,                                                \\
        &         &  1 & petit brochet de 125 grammes,                                                    \\
        &         &  1 & petit merlan de  125 grammes,                                                    \\
        &         &  1 & bouquet garni,                                                                   \\
        &         &    & jus de citron,                                                                   \\
        &         &    & sel,                                                                             \\
        &         &    & poivre,                                                                          \\
        &         &    & muscade,                                                                         \\
        &         &    & cayenne ;                                                                        \\
\end{longtable}
\normalsize

\medskip

2° pour la sauce :

\medskip

\footnotesize
\begin{longtable}{rrrp{16em}}
    500 & grammes & de & beurre,                                                                          \\
     60 & grammes & d’ & eau froide,                                                                      \\
        &         &  8 & jaunes d'œufs,                                                                   \\
        &         &    & jus de citron ou vinaigre,                                                       \\
        &         &    & cresson,                                                                         \\
        &         &    & cerfeuil,                                                                        \\
        &         &    & épinards,                                                                        \\
        &         &    & sel et poivre.                                                                   \\
\end{longtable}
\normalsize

Fendez la peau de la sole du côté blanc, dans la direction de l'arête que vous
retirerez sans abimer les filets ; c'est un travail de dissection qui demande
du soin. Réservez les déchets.

Enlevez la tête et l'arête au brochet et au merlan ; réservez-les.

Pilez la chair des deux poissons au mortier ; passez-la au tamis de crin.

Battez les blancs d'œufs en neige.

\index{Farce pour poisson}
Préparez une farce avec les chairs passées du brochet et du merlan, assaisonnez
avec sel, poivre, muscade, cayenne ; travaillez-la sur glace, en y incorporant
la crème, {\ppp4\mmm} jaunes d'œufs et les deux blancs battus.

Emplissez la sole avec cette préparation.

Faites cuire à part : les pommes de terre à l'anglaise, les écrevisses au
court-bouillon, \hyperlink{p0287}{p. \pageref{pg0287}}, les moules au naturel avec
un peu de vin blanc et le bouquet garni ; passez la cuisson des moules et
réservez-la.

Pelez les champignons ; passez-les dans du jus de citron et faites-les cuire
dans {\ppp40\mmm} grammes de beurre.

Tenez au chaud pommes de terre, champignons, écrevisses et moules.

Réunissez le bouillon de poisson, le vin et la cuisson passée des moules ;
faites bouillir dedans les déchets des poissons.

Disposez alors la sole dans un plat foncé de {\ppp60\mmm} grammes de beurre, mouillez
avec le mélange ci-dessus et faites cuire au four pendant {\ppp25\mmm} minutes, en
arrosant fréquemment avec la cuisson.

Servez dans le plat même, en disposant autour de la sole les écrevisses, les
moules, les pommes de terre et les champignons comme garniture. Envoyez en
même temps une saucière de sauce hollandaise vert-pré.

\sk

La sauce hollandaise vert-pré est une sauce hollandaise colorée.

\label{pg0362} \hypertarget{p0362}{}
Pour préparer la sauce hollandaise, coupez le beurre en petits morceaux,
ramollissez-le à la chaleur ; mettez les jaunes d'œufs et l'eau dans un bol
tenu au bain-marie, tournez, ajoutez le beurre par petites quantités et
continuez à tourner ; la sauce doit monter comme des œufs à la neige, et avoir
en même temps de la légèreté et de la cohésion. Comme assaisonnement, le plus
souvent on ne met que du sel, mais on peut aussi ajouter du poivre et un peu de
jus de citron ou de vinaigre, au goût.

Pour obtenir la sauce hollandaise vert-pré, colorez la sauce hollandaise
ordinaire précédente avec un hachis de cerfeuil, cresson et épinards blanchis
et mélangés. La teinte obtenue sera plus claire que si la sauce était colorée
exclusivement avec du vert d'épinards.

\sk

\index{Définition de la sauce hollandaise}
\index{Sauce hollandaise (Définition de la)}
On obtiendra une autre sauce hollandaise vert-pré en prenant, pour douze
personnes :

\medskip

\footnotesize
\begin{longtable}{rrrp{16em}}
    500 & grammes & de & beurre,                                                                          \\
     60 & grammes & d' & eau froide,                                                                      \\
     30 & grammes & de & cresson,                                                                         \\
     20 & grammes & d' & épinards,                                                                        \\
      8 & grammes & d' & estragon,                                                                        \\
        &         &  8 & jaunes d'œufs frais,                                                             \\
        &         &  6 & jaunes d'œufs durs,                                                              \\
        &         &    & jus de citron ou vinaigre,                                                       \\
        &         &    & sel et poivre.                                                                   \\
\end{longtable}
\normalsize

Hachez cresson, épinards et estragon.

Préparez la sauce hollandaise comme ci-dessus ; ajoutez-y le hachis d'herbes,
les jaunes d'œufs durs écrasés, assaisonnez avec sel. poivre et jus de citron
ou vinaigre, au goût, fouettez l'ensemble.

\sk

\index{Asperges, sauce hollandaise vert-pré}
Cette sauce pourra aussi accompagner des asperges en branches.

\sk

Pour compléter ce qui a trait à la sauce hollandaise en général, j'ajouterai
qu'en employant dans sa préparation du beurre de crevettes ou d'écrevisses, ou
du beurre dans lequel on aura incorporé du corail d'oursins, de homard ou de
langouste, au lieu de beurre ordinaire, on aura des sauces colorées et
aromatisées qui trouveront leur emploi dans bien des cas.

\sk

\index{Barbue soufflée, sauce hollandaise vert-pré}
\index{Turbot soufflé, sauce hollandaise vert-pré}
On peut préparer de la même manière d'autres poissons et, en particulier, le
turbotin et la barbue.

\section*{\centering Filets de soles au vin blanc.}
\addcontentsline{toc}{section}{ Filets de soles au vin blanc.}
\index{Filets de soles au vin blanc}
\label{pg0363} \hypertarget{p0363}{}

Pour quatre personnes prenez :

\medskip

\footnotesize
\begin{longtable}{rrrp{16em}}
    250 & grammes & d' & eau,                                                                             \\
    200 & grammes & de & vin blanc,                                                                       \\
    125 & grammes & de & beurre,                                                                          \\
     65 & grammes & de & champignons,                                                                     \\
     50 & grammes & de & crème épaisse,                                                                   \\
      8 & grammes & de & farine,                                                                          \\
        &         &  2 & soles pesant ensemble 750 grammes environ,                                       \\
        &         &  2 & jaunes d'œufs frais,                                                             \\
        &         &  1 & carotte moyenne,                                                                 \\
        &         &  1 & petit oignon,                                                                    \\
        &         &  1 & bouquet garni (persil, thym, laurier, céleri),                                   \\
        &         &  1 & clou de girofle,                                                                 \\
        &         &    & jus de citron,                                                                   \\
        &         &    & sel et poivre.                                                                   \\
\end{longtable}
\normalsize

Levez les filets des soles ; réservez les déchets.

Pelez les champignons ; lavez-les,

\index{Fumet de poisson}
Mettez dans une casserole l'eau, le vin, les champignons, les déchets des
soles, la carotte coupée en morceaux, l'oignon, dans lequel vous aurez piqué le
clou de girofle, le bouquet garni, du sel et du poivre ; faites cuire de
manière à obtenir un fumet ; concentrez-le ; passez-le.

Foncez un plat en porcelaine allant au feu avec {\ppp60\mmm} grammes de beurre, disposez
dedans les filets de soles, assaisonnez avec sel et poivre, mouillez avec quelques
cuillerées de fumet et faites cuire au four doux, sans laisser prendre couleur.
Tenez au chaud.

Préparez la sauce : maniez la farine avec {\ppp25\mmm} grammes de beurre sans la laisser
dorer, mouillez avec le reste du fumet et la cuisson des soles ; liez au fouet
avec les jaunes d'œufs : montez la sauce avec le reste du beurre et la crème,
ajoutez un peu de jus de citron ; chauffez ; goûtez et complétez
l’assaisonnement s'il est nécessaire.

Dressez les filets sur un plat ; masquez-les avec la sauce et servez aussitôt.

\section*{\centering Filets de soles au vin rouge.}
\addcontentsline{toc}{section}{ Filets de soles au vin rouge.}
\index{Filets de soles au vin rouge}

Pour quatre personnes prenez :

\medskip

\footnotesize
\begin{longtable}{rrrp{16em}}
    250 & grammes & d' & eau,                                                                             \\
    200 & grammes & de & vin rouge,                                                                       \\
    125 & grammes & de & beurre,                                                                          \\
     65 & grammes & de & champignons,                                                                     \\
     25 & grammes & de & beurre d'anchois,                                                                \\
     15 & grammes & de & farine,                                                                          \\
        &         &  2 & soles pesant ensemble 750 grammes environ,                                       \\
        &         &  2 & jaunes d'œufs frais,                                                             \\
        &         &  2 & clous de girofle,                                                                \\
        &         &  1 & carotte moyenne,                                                                 \\
        &         &  1 & petit navet,                                                                     \\
        &         &  1 & oignon,                                                                          \\
        &         &  1 & échalote,                                                                        \\
        &         &  1 & gousse d'ail,                                                                    \\
        &         &  1 & bouquet garni (persil, thym, laurier, cerfeuil et céleri),                       \\
        &         &    & têtes et arêtes de poissons,                                                     \\
        &         &    & jus de citron,                                                                   \\
        &         &    & sel, poivre et cayenne.                                                          \\
\end{longtable}
\normalsize


Levez les filets des soles ; réservez les déchets.

Pelez les légumes ; coupez la carotte et le navet en tranches ; hachez les
champignons : piquez les clous de girofle dans l'oignon.

\index{Fumet de poisson}
Mettez dans une casserole eau, vin, champignons, têtes et arêtes de poissons,
déchets des soles, carotte, navet, oignon, échalote, ail, bouquet garni, sel,
poivre et cayenne ; faites cuire de façon à obtenir un fumet relevé et
concentré ; passez-le.

Foncez un plat en porcelaine allant au feu avec {\ppp60\mmm} grammes de beurre ; mettez
dessus les filets de soles ; assaisonnez avec sel et poivre ; mouillez avec quelques
cuillerées de fumet et faites cuire à four doux, sans laisser prendre couleur. Tenez
au chaud.

\label{pg0388} \hypertarget{p0388}{}
Préparez alors la sauce ; faites un roux avec {\ppp50\mmm} grammes de beurre et la
farine ; mouillez avec le reste du fumet : vous aurez un velouté maigre.
Ajoutez la cuisson des soles, achevez la liaison au fouet avec les jaunes
d'œufs, puis montez la sauce avec le reste du beurre et le beurre d’anchois ;
mettez un peu de jus de citron ; chauffez ; goûtez pour l'assaisonnement et
complétez-le, s'il y a lieu, avec sel, poivre et cayenne.

Dressez les filets sur un plat ; masquez-les avec la sauce et servez.

\section*{\centering Filets de soles sur le plat.}
\addcontentsline{toc}{section}{ Filets de soles sur le plat.}
\index{Filets de soles sur le plat}

Les filets de soles seront cuits comme il est dit
\hyperlink{p0363}{p. \pageref{pg0363}}.

La sauce sera obtenue en ajoutant au fumet des fines herbes et du jus de citron.
Le fumet sera suffisamment concentré au préalable, car le mouillement doit être
ici plus court que dans la formule des filets de soles au vin blanc, puisqu'il doit
être complètement réduit.

Pendant la cuisson, on arrosera jusqu'à ce que tout le liquide ait glacé les filets
de soles d'une couche translucide.

On servira dans le plat.

\section*{\centering Filets de soles, garnis.}
\addcontentsline{toc}{section}{ Filets de soles, garnis.}
\index{Filets de soles, garnis}

Levez des filets de soles, aplatissez-les, puis roulez-les autour de mandrins
cylindriques tournés dans un légume, tel que pomme de terre ou carotte.

Foncez une sauteuse avec du beurre parfumé à l'échalote, mettez les filets
roulés, salez, poivrez, mouillez avec du vin blanc ou du bon vermouth et du jus
de citron ; laissez cuire ; puis, retirez les mandrins et réservez les filets.

Faites sauter dans du beurre des tomates coupées en morceaux et assaisonnées,
mettez-les sur un plat, dressez dessus, debout, les filets de soles roulés,
dont vous garnirez l'intérieur, au choix, avec un appareil à base de queues de
crevettes ou d'écrevisses, de champignons, de corail de langouste ou de homard,
d'huîtres pochées dans du vin blanc, de moules préparées avec une sauce liée,
ou encore avec une sauce hollandaise fortement chargée en corail d'oursins,
etc.

Maniez un peu de farine avec du beurre, mouillez avec la cuisson des filets et
des tomates ; faites réduire, puis achevez la liaison, en fouettant, avec des
jaunes d'œufs frais délayés dans de la crème.

Coiffez les filets roulés avec des truffes ou des chapeaux de champignons et
servez, en envoyant en même temps la sauce, dans une saucière.

\section*{\centering Filets de soles au gratin.}
\addcontentsline{toc}{section}{ Filets de soles au gratin.}
\index{Filets de soles au gratin}

Pour préparer des filets de soles au gratin, faites-les cuire comme il est dit
\hyperlink{p0363}{p. \pageref{pg0363}}. Dressez les filets sur un plat ;
masquez-les avec leur cuisson concentrée et liée ; saupoudrez le dessus avec de
la chapelure fine et fraîche sur une épaisseur de {\ppp4\mmm} à {\ppp5\mmm}
millimètres ; arrosez de beurre fondu et faites gratiner au four.

\section*{\centering Filets de soles Mornay.}
\addcontentsline{toc}{section}{ Filets de soles Mornay.}
\index{Filets de soles Mornay}

Faites cuire des filets de soles comme il est dit
\hyperlink{p0363}{p. \pageref{pg0363}} ; tenez-les au chaud ; concentrez le jus de
cuisson.

Préparez une sauce Mornay à base de velouté maigre,
\hyperlink{p0269}{p. \pageref{pg0269}}, dans laquelle vous ferez entrer la cuisson
concentrée des filets.

Foncez un plat allant au feu avec la moitié de cette sauce, mettez dessus les
filets, recouvrez avec le reste de la sauce et poussez au four pour gratiner.

\section*{\centering Filets de soles aux morilles, gratinés.}
\addcontentsline{toc}{section}{ Filets de soles aux morilles, gratinés.}
\index{Filets de soles aux morilles, gratinés}

Pour quatre personnes prenez :

\medskip

\footnotesize
\begin{longtable}{rrrp{16em}}
    500 & grammes & de & vin blanc,                                                                       \\
    500 & grammes & de & légumes (carottes, oignons).                                                     \\
    300 & grammes & d' & eau,                                                                             \\
    250 & grammes & de & morilles,                                                                        \\
    200 & grammes & de & beurre,                                                                          \\
        &         &  2 & soles pesant ensemble 750 grammes environ,                                       \\
        &         &    & mie de pain rassis tamisée,                                                      \\
        &         &    & bouquet garni,                                                                   \\
        &         &    & jus de citron,                                                                   \\
        &         &    & sel et poivre.                                                                   \\
\end{longtable}
\normalsize

Levez les filets des soles ; réservez les déchets.

Mettez dans une casserole le vin, l'eau, les déchets des soles, les légumes
épluchés et émincés, le bouquet garni, du sel, du poivre et faites bouillir
à petit feu de manière à obtenir un bon fumet en quantité suffisante. Passez-le
au chinois.

Faites cuire en même temps : d'une part, les filets de soles dans le fumet
auquel vous aurez ajouté {\ppp25\mmm} grammes de beurre ; d'autre part, les
morilles dans {\ppp50\mmm} grammes de beurre et du jus de citron. Tenez au
chaud filets de soles et morilles.

Réunissez les deux cuissons, concentrez-les jusqu'à consistance convenable
puis, sans laisser bouillir, incorporez {\ppp75\mmm} grammes de beurre en
fouettant ; vous obtiendrez ainsi une sauce très moelleuse. Goûtez-la et
complétez son assaisonnement s'il est nécessaire.

Disposez les filets de soles sur un plat allant au feu, entourez-les avec les
morilles, masquez le tout avec la sauce, saupoudrez avec de la mie de pain
rassis tamisée sur laquelle vous mettrez le reste du beurre coupé en petits
morceaux ; faites dorer au four.

Servez dans le plat.

Cette préparation est remarquable, surtout quand on emploie des morilles
fraîchement récoltées.

\section*{\centering Filets de soles Cardinal.}
\addcontentsline{toc}{section}{ Filets de soles Cardinal.}
\index{Filets de soles Cardinal}

Pour quatre personnes prenez :

\medskip

\footnotesize
\begin{longtable}{rrrrp{16em}}
  &     500 & grammes     & d' & eau,                                                                     \\
  &     250 & grammes     & de & vin blanc,                                                               \\
  &     125 & grammes     & de & beurre,                                                                  \\
  &      30 & grammes     & de & fine champagne,                                                          \\
  &      30 & grammes     & de & mie de pain rassis,                                                      \\
  &      15 & grammes     & de & sel,                                                                     \\
  &      10 & grammes     & de & farine,                                                                  \\
2 & \multicolumn{2}{r}{grammes 1/2}   & de & poivre en grains,                                            \\
  &         &             & 12 & belles écrevisses,                                                       \\
  &         &             &  2 & soles pesant ensemble 350 grammes environ,                               \\
  &         &             &  1 & merlan pesant 200 grammes environ,                                       \\
  &         &             &  1 & œuf frais,                                                               \\
  &         &             &  1 & jaune d'œuf,                                                             \\
  &         &             &  1 & carotte moyenne,                                                         \\
  &         &             &  1 & oignon moyen,                                                            \\
  &         &             &  1 & bouquet garni,                                                           \\
  &         &             &    & lait,                                                                    \\
  &         &             &    & jus de citron,                                                           \\
  &         &             &    & muscade,                                                                 \\
  &         &             &    & cayenne,                                                                 \\
  &         &             &    & carmin.                                                                  \\
\end{longtable}
\normalsize

Mettez les écrevisses à dégorger dans du lait pendant deux heures.

Préparez un court-bouillon, que vous laisserez cuire pendant une demi-heure
environ, avec l'eau, le vin, la fine champagne, la carotte et l'oignon émincés,
le bouquet garni, {\ppp8\mmm} grammes de sel et le poivre en grains.

Enlevez les intestins aux écrevisses ; plongez-les dans le court-bouillon
bouillant ; au bout d'un quart d'heure de cuisson, retirez-les. Réservez la
cuisson.

Séparez les queues des thorax des écrevisses ; videz huit thorax en gardant
intactes les carapaces ; décortiquez les queues, tenez-les au chaud. Réservez
tous les déchets que vous pilerez au mortier avec {\ppp50\mmm} grammes de beurre et que
vous passerez au tamis de crin. Réservez ce beurre d'écrevisses.

\index{Farce pour poisson}
Prélevez sur le merlan {\ppp100\mmm} grammes de chair que vous pilerez au mortier avec la
mie de pain trempée dans du lait et pressée ; ajoutez ensuite {\ppp25\mmm} grammes de
beurre, l'œuf entier, le reste du sel et un peu de muscade ; pilez encore, puis
passez cette farce au tamis de crin.

Levez les filets des soles, garnissez-les avec la farce de merlan, dont vous
mouillerez le dessus avec un peu de jus de citron ; roulez-les en cornets.

Mettez les déchets des soles et du merlan dans le court-bouillon des
écrevisses ; laissez cuire de façon à réduire le liquide de moitié ; vous aurez
ainsi un fumet maigre. Passez-le.

Faites entrer le petit bout des cornets de sole dans les carapaces vides
réservées, placez-les dans une casserole plate, mouillez avec le fumet maigre,
couvrez ; laissez cuire pendant une dizaine de minutes ; tenez au chaud.

Préparez la sauce Cardinal : maniez la farine avec le reste du beurre, mouillez
avec la cuisson des filets, faites bouillir pendant quelques minutes. Éloignez
la casserole du feu, achevez la liaison avec le jaune d'œuf, montez la sauce
avec le beurre d'écrevisses, relevez le tout avec du jus de citron et du
cayenne au goût, et colorez avec un peu de carmin de façon à obtenir une belle
couleur rouge. Chauffez.

Disposez les carapaces d'écrevisses garnies de filets de soles en couronne sur un
plat, mettez au milieu les queues d'écrevisses décortiquées, masquez avec de la
sauce la partie des filets de soles hors des carapaces et les queues d'écrevisses.

Servez en envoyant le reste de la sauce dans une saucière.

\sk

Comme variante, on pourra servir les filets sur un turban de merlan préparé
avec un mélange de chair de merlan, de mie de pain trempée dans du lait et
pressée, de beurre et d'œufs, mis dans un moule en couronne et cuit au
bain-marie. Le turban sera démoulé sur un plat ; on garnira le dessus avec les
filets de soles chevauchés par de grosses crevettes bien rouges, et le centre
avec un salpicon de champignons, de quenelles de brochet et d'écrevisses dans
de la sauce Cardinal.

\sk

On pourra encore garnir les filets de soles avec une farce de merlan aux
anchois. On dressera ces filets farcis en couronne, en intercalant entre eux
des escalopes de homard ; on masquera le tout avec une variante de la sauce
Cardinal, à base de béchamel et de fumet maigre, liée aux jaunes d'œufs, puis
montée au beurre de homard et à la crème, enfin relevée par un peu de cayenne
et parfumée avec de la truffe.

\section*{\centering Filets de soles, sauce homard.}
\addcontentsline{toc}{section}{ Filets de soles, sauce homard.}
\index{Filets de soles, sauce homard}
\label{pg0369} \hypertarget{p0369}{}

Pour six personnes prenez :

\medskip

\footnotesize
\begin{longtable}{rrrp{16em}}
    250 & grammes & de & crevettes grises,                                                                \\
    250 & grammes & de & champignons de couche,                                                           \\
    150 & grammes & de & beurre,                                                                          \\
    100 & grammes & de & vin blanc,                                                                       \\
    100 & grammes & d' & eau,                                                                             \\
     20 & grammes & d' & œufs de homard,                                                                  \\
      5 & grammes & de & farine,                                                                          \\
        &         &  1 & litre de moules,                                                                 \\
        &         &  2 & soles pesant ensemble 750 grammes environ,                                       \\
        &         &  2 & jaunes d'œufs crus,                                                              \\
        &         &  1 & oignon,                                                                          \\
        &         &  1 & bouquet garni,                                                                   \\
        &         &    & jus de citron,                                                                   \\
        &         &    & muscade,                                                                         \\
        &         &    & sel et poivre.                                                                   \\
\end{longtable}
\normalsize

Faites cuire les crevettes, comme d'habitude.

Nettoyez les moules, mettez-les dans une casserole, faites-les s'ouvrir à feu
vif, sortez-les ensuite des coquilles, tenez-les au chaud ; réservez l'eau.

Levez les filets des soles, mettez-les de côté.

Coupez les arêtes en tronçons, faites-les cuire pendant une demi-heure avec
l'eau, le vin blanc, l'eau des moules, l'oignon et le bouquet garni, puis
passez le jus ; réservez-le.

Faites cuire à part les champignons avec un peu de beurre et du jus de citron ;
réservez la cuisson.

Décortiquez les crevettes, mettez de côté les parures ; réunissez les moules,
les queues de crevettes, les champignons dans une petite casserole émaillée et
tenez couvert, au chaud.

Préparez un beurre de crevettes avec {\ppp50\mmm} grammes de beurre et les
parures des crevettes.

Beurrez le plat dans lequel vous servirez, dressez dessus les filets, mouillez
avec le jus mis de côté, couvrez avec un papier beurré et poussez au four
pendant six minutes. Réservez la cuisson.

Maniez {\ppp30\mmm} grammes de beurre avec la farine, laissez cuire pendant
quelques minutes, ajoutez les œufs de homard écrasés et passés, qui parfumeront
la sauce et la coloreront provisoirement en noir, mouillez avec le jus de
cuisson des filets de soles et celui des champignons, donnez un coup de fouet
à la sauce, ajoutez le reste du beurre, le beurre de crevettes, du poivre, de
la muscade et du jus de citron, au goût, liez la sauce avec les jaunes d'œufs ;
chauffez-la : elle deviendra rouge.

Entourez les filets de soles avec les champignons, les moules et les queues de
crevettes, masquez avec la sauce, glacez au four et servez.

Cette façon de préparer les filets de soles est l'une des meilleures que je
connaisse. L'introduction des œufs de homard donne au plat un goût très fin et
permet de le présenter couvert d'un manteau de pourpre cardinalice du plus bel
effet.

\section*{\centering Filets de soles sauce homard, en turban de brochet.}
\addcontentsline{toc}{section}{ Filets de soles sauce homard, en turban de brochet.}
\index{Filets de soles sauce homard, en turban de brochet}

Voici une autre jolie façon de présenter les filets de soles sauce homard.

Après avoir mené la préparation comme dans la formule précédente, dressez les
filets de soles sur les bords d'un turban de brochet préparé comme il est dit
plus loin ; mettez dans l'intérieur du turban les champignons, les moules, les
queues de crevettes avec la sauce et servez.

Ainsi présenté, le plat peut suffire pour huit à dix personnes :

\sk

Pour le turban de brochet prenez :

\medskip

\footnotesize
\begin{longtable}{rrrp{16em}}
    500 & grammes & de & pommes de terre,                                                                 \\
    500 & grammes & de & beurre,                                                                          \\
        &         &  1 & brochet pesant 600 grammes, qui fournira 500 grammes de chair environ,           \\
        &         &    & sel et poivre.                                                                   \\
\end{longtable}
\normalsize

Court-bouillonnez le poisson, enlevez-en la peau et les arêtes ; pilez la
chair.

Faites cuire les pommes de terre à la vapeur ; pelez-les, écrasez-les.

Mélangez beurre, chair de brochet pilée et pommes de terre écrasées,
assaisonnez au goût avec sel et poivre, passez le mélange au tamis, puis
mettez-le dans un moule à turban. Faites cuire au bain-manie.

Démoulez le turban au moment de vous en servir.

\section*{\centering Filets de soles sauce homard, en croustades\footnote{
\index{Croustades (Définition des)}
\index{Croûtes (Définition des)}
\index{Définition des Croustades}
\index{Définition des Croûtes}
D'une façon générale, les croustades sont de petites croûtes, c'est-à-dire des
petites enveloppes faites avec de la mie de pain, du riz, de la semoule, de la
polenta ou de la pâte. Par métonymie, on désigne souvent sous le nom de
croustades et de croûtes ces enveloppes garnies de préparations
alimentaires.}.}
\addcontentsline{toc}{section}{ Filets de soles sauce homard, en croustades.}
\index{Filets de soles sauce homard, en croustades}
\index{Croustade de filets de soles sauce homard}

Les filets de soles sauce homard peuvent aussi être servis en croustades.

La préparation des filets et celle de la sauce restent les mêmes que
\hyperlink{p0369}{p. \pageref{pg0369}}. mais le mets est servi dans des croustades
chaudes que l’on peut confectionner de la façon suivante :

\label{pg0371} \hypertarget{p0371}{}
Faites une pâte à brioche avec :

\medskip

\footnotesize
\begin{longtable}{rrrp{16em}}
    250 & grammes & de & farine tamisée,                                                                  \\
    200 & grammes & de & beurre,                                                                          \\
      7 & grammes & de & levure en hiver, 5 grammes en été,                                               \\
      5 & grammes & de & sel,                                                                             \\
        &         &  4 & œufs entiers.                                                                    \\
\end{longtable}
\normalsize

Délayez la levure dans un peu d'eau tiède. mélangez-la avec {\ppp60\mmm} grammes de
farine de façon à obtenir un levain léger ; travaillez-le, mettez-le en boule,
pratiquez dessus quelques incisions et laissez-le lever à une température
douce.

Formez fontaine avec le reste de la farine, mettez le sel dissous dans un peu
d'eau, cassez dedans deux œufs et travaillez la pâte de façon à lui donner du
corps ; ajoutez ensuite, un à un, les deux œufs restants. Quand la pâte est
bien lisse et sèche, incorporez-y le beurre et le levain.

Laissez lever la pâte à une température douce et rompez-la plusieurs fois avant
de vous en servir.

Prenez autant de moules à brioche que vous avez de filets de soles,
chemisez-les d'une couche de pâte, emplissez l'intérieur avec un corps inerte,
tel que des cailloux de rivière bien lavés, pour en empêcher la déformation ;
faites cuire au four.

Garnissez les croustades avec la préparation, à raison d'un filet par
croustade, et servez sur un plat recouvert d’une serviette.

\section*{\centering Filets de soles Lucullus.}
\addcontentsline{toc}{section}{ Filets de soles Lucullus.}
\index{Filets de soles Lucullus}

Pour six à huit personnes prenez :

\medskip

\footnotesize
\begin{longtable}{rrrp{16em}}
    500 & grammes & de & grand vin blanc : meursault, montrachet, champagne sec, au choix,                \\
    375 & grammes & de & pistaches en coques,                                                             \\
    300 & grammes & de & beurre,                                                                          \\
    200 & grammes & d' & eau,                                                                             \\
    150 & grammes & de & crevettes grises vivantes,                                                       \\
    125 & grammes & de & crème épaisse,                                                                   \\
     75 & grammes & de & carottes,                                                                        \\
     30 & grammes & de & navet,                                                                           \\
     15 & grammes & de & céleri,                                                                          \\
     15 & grammes & de & panais,                                                                          \\
      1 & gramme  & de & poivre en grains,                                                                \\
        &         & 16 & belles huîtres d'Ostende,                                                        \\
        &         & 16 & belles langoustines,                                                             \\
        &         & 16 & petits canapés de feuilletage,                                                   \\
        &         &  8 & gros champignons,                                                                \\
        &         &  4 & grosses truffes noires du Périgord,                                              \\
        &         &  4 & jaunes d'œufs frais,                                                             \\
        &         &  2 & soles pesant chacune 500 grammes environ,                                        \\
        &         &  1 & merlan pesant 300 grammes environ,                                               \\
        &         &    & têtes et arètes de soles,                                                        \\
        &         &    & bouquet garni,                                                                   \\
        &         &    & jus de citron,                                                                   \\
        &         &    & sel, poivre,                                                                     \\
        &         &    & carmin.                                                                          \\
\end{longtable}
\normalsize

Levez les filets des soles, parez-les, réservez les déchets.

Brossez les truffes, cuisez-les dans le vin, pelez-les, réservez les pelures.

Préparez un court-bouillon avec le vin de cuisson des truffes, l'eau, les
légumes, le bouquet garni, le poivre en grains, du sel. Laissez-le cuire
pendant une demi-heure environ.

Passez-le, puis faites cuire dedans les crevettes et les langoustines ;
tenez-les au chaud,

Ouvrez les huîtres et pochez-les dans leur eau ; réservez.

Pelez les champignons, réservez les pelures et les pieds. cuisez les chapeaux
dans {\ppp50\mmm} grammes de beurre et du jus de citron : réservez et tenez au chaud.

Réunissez cuisson des crevettes et des langoustines, eau des huîtres, cuisson
des champignons, ajoutez-y les déchets, les têtes et les arêtes de soles, le
merlan coupé en petits morceaux, les pelures des truffes, les pelures et les
pieds des champignons hachés grossièrement ; laissez cuire de façon à obtenir
un fumet aromatisé suffisamment concentré. Passez-le.

Décortiquez les pistaches, échaudez-les, enlevez la pellicule rouge, pilez les
amandes au mortier et garnissez avec cette pâte les chapeaux de champignons.

Épluchez les crevettes et les langoustines ; réservez les queues des
langoustines,

Pilez les queues des crevettes avec {\ppp100\mmm} grammes de beurre et préparez un beurre
de crustacés avec {\ppp60\mmm} grammes de beurre et les parures des crevettes et des
langoustines.

Mettez les filets des soles dans un plat allant au feu, mouillez avec le fumet
et laissez pocher pendant une dizaine de minutes.

Concentrez la cuisson de façon à la réduire à {\ppp50\mmm} grammes environ.

Préparez une sauce avec le reste du beurre, les jaunes d'œufs et la cuisson des
soles concentrée.

Dressez sur un plat rond en argent les filets des soles, en rosace, la pointe
tournée vers le centre du plat, garnissez les intervalles avec un peu de la
sauce ci-dessus, mettez à la pointe de chaque filet un champignon farci, la
farce en dessus, intercalez dans chaque espace compris entre deux filets de
soles deux queues de langoustines bout à bout, séparées par une demi-truffe ;
à droite et à gauche de chaque filet de sole, au pourtour du plat, dressez une
huître sur un canapé de feuilletage et tenez le tout au chaud.

Incorporez au reste de la sauce le beurre de crevettes, le beurre de crustacés.
du sel et du poivre au goût, colorez avec un peu de carmin et montez, au fouet,
avec la crème la sauce qui doit être légèrement rosée.

Masquez les filets des soles avec une partie de la sauce et versez le reste
dans le turban formé au centre du plat par les champignons farcis.

Servez aussitôt.

Beau plat pour ventres dorés.

\section*{\centering Paupiettes de filets de soles aux truffes.}
\addcontentsline{toc}{section}{ Paupiettes de filets de soles aux truffes.}
\index{Paupiettes de filets de soles aux truffes}
\index{Filets de soles aux truffes (Paupietles de)}

Prenez de belles soles ; levez-en les filets ; réservez les déchets.

Prenez autant de belles truffes que vous avez de filets ; nettoyez-les,
pelez-les, réservez les pelures.

Préparez un court-bouillon avec eau, porto blanc, déchets de soles, pelures de
truffes, carottes, oignons, bouquet garni, sel et poivre. Concentrez-le,
passez-le.

Faites cuire les truffes dans du madère.

Mettez les filets de soles dans un plat beurré, assaisonnez avec sel et poivre,
mouillez avec le court-bouillon ; laissez cuire à moitié, puis retirez les
filets.

Réunissez cuisson des filets et cuisson des truffes, concentrez-les. Au dernier
moment, liez cette sauce avec des jaunes d'œufs, montez-la à la crème et
relevez-la avec du jus de citron.

Enroulez chaque truffe dans un filet de sole, enrobez ces paupiettes dans de la
pâte à frire et finissez-en la cuisson dans de la friture très chaude.
Égouttez-les, dressez-les sur un plat garni d'une serviette ; servez-les en
envoyant en même temps la sauce dans une saucière.

Ces paupiettes sont très succulentes.

\section*{\centering Timbale\footnote{
\index{Définition des timbales}
\index{Timbales (Définition des)}
On désigne sous le nom de timbale des enveloppes en croûte, en
riz, en porcelaine, en terre, en métal, ayant la forme d'un tambour.
\protect\endgraf
On l'emploie aussi pour désigner des préparations culinaires services dans ces
enveloppes.} de filets de soles, sauce homard.}

\addcontentsline{toc}{section}{ Timbale de filets de soles, sauce homard.}
\index{Timbale de filets de soles, sauce homard}
\index{Filets de soles, sauce homard (Timbale de)}

Pour six à huit personnes prenez :

\medskip

\footnotesize
\begin{longtable}{rrrp{16em}}
    250 & grammes & de & crevettes grises,                                                                \\
    250 & grammes & de & champignons de couche,                                                           \\
    150 & grammes & de & beurre,                                                                          \\
    125 & grammes & de & crème,                                                                           \\
    100 & grammes & de & vin blanc,                                                                       \\
    100 & grammes & d' & eau,                                                                             \\
     30 & grammes & d' & œufs de homard crus,                                                             \\
      8 & grammes & de & farine,                                                                          \\
        & 1 litre & de & moules,                                                                          \\
        &         & 12 & belles huîtres grasses,                                                          \\
        &         &  3 & jaunes d'œufs frais,                                                             \\
        &         &  2 & soles moyennes pesant ensemble 750 grammes environ,                              \\
        &         &  1 & carotte moyenne,                                                                 \\
        &         &  1 & oignon moyen,                                                                    \\
        &         &  1 & bouquet garni,                                                                   \\
        &         &    & jus de citron,                                                                   \\
        &         &    & muscade,                                                                         \\
        &         &    & sel et poivre.                                                                   \\
\end{longtable}
\normalsize

Levez les filets des soles, roulez-les sur eux-mêmes, réservez les déchets.

Faites cuire séparément les huîtres et les moules dans leur eau, les crevettes
dans un court-bouillon, les champignons dans du beurre et du jus de citron,
comme il est dit dans les formules précédentes.

Préparez un fond maigre en faisant cuire, dans le vin et l'eau assaisonnés avec
sel et poivre, déchets de soles, carotte, oignon, bouquet garni. Passez-le.

Décortiquez les crevettes, mettez de côté les queues, faites un beurre de
crevettes avec {\ppp50\mmm} grammes de beurre et les parures.

Tenez au chaud huîtres, moules, champignons, queues de crevettes.

Réunissez fond, court-bouillon des crevettes, cuissons des huîtres, des moules
et des champignons, concentrez-les, puis faites cuire dedans les filets de soles
pendant une dizaine de minutes.

Préparez la sauce homard ; maniez la farine avec {\ppp30\mmm} grammes de beurre, laissez
cuire pendant quelques minutes, ajoutez les œufs de homard écrasés et passés,
mouillez avec la cuisson des filets, donnez un coup de fouet ; achevez la
liaison avec les jaunes d'œufs délayés dans la crème, montez la sauce avec le
reste du beurre et le beurre de crevettes, assaisonnez avec poivre, muscade et
jus de citron au goût.

Chauffez.

Mettez dans une timbale en croûte\footnote{Semblable à celle des pâtés.
À Paris, il est facile de s'éviter la peine de la préparer, car on en trouve de
convenable chez la plupart des bons pâtissiers.}, ou en porcelaine chauffée au
préalable au bain-marie, filets de soles, huîtres, moules, champignons et
queues de crevettes, versez dessus la sauce et servez.

Cette timbale est très délicate.

\section*{\centering Timbale de filets de soles au porto, sauce Nantua.}
\addcontentsline{toc}{section}{ Timbale de filets de soles au porto, sauce Nantua.}
\index{Timbale de filets de soles au porto, sauce Nantua}
\index{Filets de soles au porto, sauce Nantua (Timbale}

Levez des filets de soles ; roulez-les sur eux-mêmes.

Prenez une belle darne de saumon ; court-bouillonnez-la comme il est dit
\hyperlink{p0323}{p. \pageref{pg0323}}, en remplaçant le vin rouge par du porto
blanc ; enlevez la peau et les arêtes.

Pilez la chair au mortier ; passez-la au tamis de crin ; ajoutez de la mie de
pain rassis tamisée, des œufs ; mouillez avec plus ou moins de la cuisson du
saumon ; travaillez de façon à obtenir une pâte qui vous permette de mouler des
quenelles.

\bigskip

Préparez un fond de poisson au porto comme il est dit dans la formule
ci-dessous\footnote{\index{Fond de poisson}
Pour faire un litre de fond de poisson, prenez :

\begin{longtable}{rrrrp{16em}}
  & 1 200 & grammes & d' & arêtes, de têtes et de parures de soles, merlans, barbue, etc,                 \\
  &    90 & grammes & de & vin blanc, de porto ou de madère au choix,                                     \\
  &    60 & grammes & de & champignons,                                                                   \\
  &    50 & grammes & d' & oignons,                                                                       \\
  &    10 & grammes & de & persil,                                                                        \\
  &     4 & grammes & de & sel,                                                                           \\
  & \multicolumn{2}{r}{1 gramme 1/2}  & de & poivre en grains,                                            \\
  &       & 1 litre & d’ & eau,                                                                           \\
  &       &         &    & beurre,                                                                        \\
  &       &         &    & jus de citron.                                                                 \\
\end{longtable}

Foncez une casserole avec un peu de beurre, puis mettez champignons, oignons,
persil, débris de poissons, saler, mouillez avec le vin et l’eau ; chauffez,
écumez soigneusement. Fermez hermétiquement la casserole et continuez
l'ébullition doucement et régulièrement pendant une heure environ. Dix minutes
avant la fin, ajoutez le poivre et le jus de citron. Passez au tamis.}.

Faites cuire des écrevisses comme à l'ordinaire ; décortiquez-les ; réservez
les queues. Préparez avec du beurre et les parures un beurre d'écrevisses.

Réunissez le fond de poisson et le reste du court-bouillon de saumon ; faites
cuire dedans les filets des soles. Concentrez la cuisson.

Faites cuire des morilles dans du beurre.

Préparez une béchamel maigre, montez-la à la crème, finissez-la avec le beurre
d'écrevisses, mettez dedans filets de soles, quenelles de saumon, morilles,
queues d'écrevisses, cuisson concentrée, chauffez pendant quelques minutes,

Versez le tout dans une timbale et servez.

\section*{\centering Timbale de paupiettes de filets de soles, à la ravigote.}
\addcontentsline{toc}{section}{ Timbale de paupiettes de filets de soles, à la ravigote.}
\index{Timbale de paupiettes de filets de soles, à la ravigote}
\index{Filets de soles à La ravigote (Timbale de)}

Levez des filets de soles ; mettez sur chaque filet une escalope de queue de
langouste ; roulez en paupiettes.

Faites cuire séparément des champignons dans du beurre avec un peu de jus de
citron ; des truffes dans du madère ; des crevettes dans de l'eau salée.

Décortiquez les crevettes ; mettez de côté les queues.

Faites pocher des huîtres dans leur eau ; tenez-les au chaud ; concentrez la
cuisson ; réservez-la.

Mettez des moules dans une casserole ; faites-les s'ouvrir à feu vif ;
sortez-les des coquilles ; tenez-les au chaud.

Réunissez les cuissons des champignons, des truffes, des huîtres ; ajoutez du
fumet de poisson, du poivre, du sel s'il est nécessaire et faites cuire dedans
les paupiettes de soles.

\index{Beurre de ravigote}
Au dernier moment, montez la sauce au beurre de ravigote, obtenu en incorporant
à du beurre son poids d'un mélange en parties égales de pimprenelle, cresson
alénois, cerfeuil, civette, estragon blanchis, égouttés, pilés au mortier, le
tout passé au tamis, puis mettez les champignons, les truffes, les huîtres, les
moules, les queues de crevettes ; chauffez un instant.

Versez le tout dans une timbale et servez.

\sk

Comme variante, on pourra faire une timbale de paupiettes de filets de soles
avec garniture de quenelles de brochet, champignons et queues d'écrevisses.

\section*{\centering Sole farcie frite.}
\addcontentsline{toc}{section}{ Sole farcie frite.}
\index{Sole farcie frite}

Pour quatre personnes prenez :

\medskip

\footnotesize
\begin{longtable}{rrrp{16em}}
     60 & grammes & de & mie de pain rassis tamisée,                                                      \\
     50 & grammes & de & beurre,                                                                          \\
        &         & 12 & écrevisses,                                                                      \\
        &         &  2 & jaunes d'œufs frais,                                                             \\
        &         &  2 & citrons,                                                                         \\
        &         &  1 & sole pesant 750 grammes environ,                                                 \\
        &         &    & farine,                                                                          \\
        &         &    & fines herbes,                                                                    \\
        &         &    & persil,                                                                          \\
        &         &    & cayenne ou paprika,                                                              \\
        &         &    & sel et poivre.                                                                   \\
\end{longtable}
\normalsize

Faites cuire les écrevisses, décortiquez-les, passez les queues en purée au
tamis ; préparez avec le beurre et les parures un beurre d'écrevisses.

Enlevez l'arête de la sole ; remplacez-la par la purée et le beurre
d'écrevisses, mélangés ensemble et relevés avec des fines herbes, du cayenne ou
du paprika, le tout assaisonné avec sel et poivre.

Passez la sole, ainsi apprêtée, d'abord dans de la farine, ensuite dans les
jaunes d'œufs battus, enfin dans la mie de pain rassis tamisée.

Faites-la frire dans de l'huile bouillante.

Servez-la, avec les citrons coupés en deux, sur un plat garni d'un cordon de
persil frit.

\sk

On aura de nombreuses variantes, soit en changeant la nature des purées, des
aromates et des garnitures, soit en remplaçant les purées par des farces fines
de poissons, de crustacés, de mollusques, de laitances, de champignons, etc.

\section*{\centering Paupiettes de filets de soles frites.}
\addcontentsline{toc}{section}{ Paupiettes de filets de soles frites.}
\index{Paupiettes de filets de soles frites}

Pour six personnes prenez :

\medskip

\footnotesize
\begin{longtable}{rrrrp{16em}}
  & 750 & grammes & de & filets de soles, en huit filets,                                                 \\
  & 300 & grammes & de & filets de merlan,                                                                \\
  & 150 & grammes & de & crevettes grises,                                                                \\
  & 125 & grammes & de & champignons de couche,                                                           \\
  & 110 & grammes & de & beurre,                                                                          \\
  &  50 & grammes & d' & huile d'olive,                                                                   \\
  &  20 & grammes & d' & oignon,                                                                          \\
  &  20 & grammes & de & mie de pain trempée dans du lait et égouttée,                                    \\
  &   5 & grammes & de & persil,                                                                          \\
  &   1 & gramme  & de & thym,                                                                            \\
2 & \multicolumn{2}{r}{décigrammes}   & de & paprika,                                                     \\
  &     &         &  4 & citrons,                                                                         \\
  &     &         &  3 & œufs entiers,                                                                    \\
  &     &         &  1 & feuille de laurier,                                                              \\
  &     &         &    & farine,                                                                          \\
  &     &         &    & mie de pain rassis tamisée,                                                      \\
  &     &         &    & sel et poivre.                                                                   \\
\end{longtable}
\normalsize

Faites cuire les crevettes comme d'ordinaire.

Saupoudrez les filets de soles avec {\ppp10\mmm} grammes de sel et le paprika, mettez-les
à mariner\footnote{
\index{Définition des marinades}
\index{Marinades (Définition des)}
Les marinades ont pour effet d’attendrir les viandes et d'en
masquer la saveur, ce qui peut présenter des avantages dans certains cas. Elles
ont des compositions différentes, elles sont employées crues ou cuites, et les
chairs et les viandes séjournent plus ou moins longtemps suivant leur nature et
le temps dont on dispose.

On trouvera dans ce volume des exemples de marinades pour poissons, pour
viandes de boucherie et pour gibier.} pendant une heure dans {\ppp40\mmm} grammes
d'huile, avec l'oignon émincé, le thym, le laurier, du persil et le jus d'un
citron.

\index{Farce pour poisson}
Préparez la farce : épluchez les crevettes, réservez les queues ; faites un
beurre de crevettes avec les parures et {\ppp60\mmm} grammes de beurre, ajoutez la mie de
pain bien égouttée, le beurre de crevettes, du sel, du poivre, pilez le tout
ensemble ; puis mettez deux œufs entiers et pilez encore de façon à avoir un
mélange homogène et moelleux ; goûtez, complétez l'assaisonnement s'il y a lieu
et, dans ce cas, pilez de nouveau, passez le mélange au tamis de crin.

Épluchez les champignons, émincez-les, passez-les dans du jus de citron,
faites-les cuire dans une casserole avec le reste du beurre et {\ppp2\mmm} grammes de
persil haché, puis étalez le tout sur les filets de soles, mettez par-dessus la
farce et roulez en paupiettes sans ficeler.

Passez les paupiettes d'abord dans de la farine, ensuite dans un œuf battu avec
le reste de l'huile, enfin dans de la mie de pain rassis tamisée ; plongez-les
dans une friture claire, chaude, à une température intermédiaire entre celle de
la graisse légèrement fumante et celle de la graisse franchement fumante,
laissez-les cuire pendant dix minutes en maintenant constamment la température
du bain.

Dressez les paupiettes sur un plat garni d'une serviette, décorez avec des
tranches de citron et du persil frit, aspergez avec un peu de jus de citron et
servez.

Ces paupiettes ont une belle allure et elles sont exquises. La sécheresse
extérieure de la sole frite, légèrement mitigée par le jus de citron, contraste
très agréablement avec le moelleux de la farce, et les deux parties composant
le mets conservent chacune sa saveur propre, dont l'une fait valoir l'autre.

\sk

On peut préparer dans le même esprit des paupiettes d'autres poissons, et
remplacer dans la farce, tour à tour, les crevettes et le beurre de crevettes
par des écrevisses et du beurre d’écrevisses ou par des anchois et du beurre
d'anchois, \hyperlink{p0337}{p. \pageref{pg0337}}. Il y a là toute une série de plats
nuancés qu'on pourra varier en se laissant guider par ses préférences.

\section*{\centering Julienne de filets de soles, panée, sautée.}
\addcontentsline{toc}{section}{ Julienne de filets de soles, panée, sautée.}
\index{Julienne de filets de soles, panée, sautée}

Pour quatre personnes prenez :

\medskip

\footnotesize
\begin{longtable}{rrrp{16em}}
    250 & grammes & de & vin blanc,                                                                       \\
    250 & grammes & de & beurre,                                                                          \\
    200 & grammes & de & chapelure fine et fraîche,                                                       \\
    200 & grammes & d' & eau,                                                                             \\
    100 & grammes & de & champignons,                                                                     \\
     60 & grammes & de & crème épaisse,                                                                   \\
     10 & grammes & de & farine,                                                                          \\
        &         &  2 & soles moyennes pouvant fournir 500 grammes de filets,                            \\
        &         &  2 & jaunes d'œufs frais,                                                             \\
        &         &  1 & carotte moyenne,                                                                 \\
        &         &  1 & oignon moyen,                                                                    \\
        &         &  1 & bouquet garni,                                                                   \\
        &         &    & quatre épices,                                                                   \\
        &         &    & sel et poivre.                                                                   \\
\end{longtable}
\normalsize

Levez les filets des soles ; réservez les déchets.

Mettez dans une casserole vin, eau, déchets de soles, champignons, carotte,
oignon, bouquet garni, sel, poivre et un peu de quatre épices ; laissez cuire
jusqu'à obtention d'un fumet suffisamment concentré, dans lequel vous ferez
cuire les filets de soles. Emincez-les en julienne.

Faites dorer la chapelure dans {\ppp175\mmm} grammes de beurre, ajoutez ensuite la
julienne de poisson et faites sauter le tout ensemble de manière à enrober de
chapelure tous les émincés de soles.

Maniez la farine avec un peu de beurre, mouillez avec la cuisson des filets,
réduite encore s'il est nécessaire, liez cette sauce avec les jaunes d'œufs,
puis montez-la avec le reste du beurre et la crème. Chauffez.

Servez, en envoyant en même temps la sauce dans une saucière, du parmesan
et du gruyère râpés dans un ravier,

\section*{\centering Merlans\footnote{Gadus merlangus, famille des Gadidés.}.}
\addcontentsline{toc}{section}{ Merlans.}
\index{Merlans}

On peut apprêter les merlans de beaucoup de manières : grillés, frits, au vin
blanc, à la meunière, à la dieppoise, à la maître-d'hôtel, à la sauce gratin,
à la sauce Colbert, à la sauce Bercy, à la sauce aux truffes, aux fines herbes, etc.

On les fait cuire entiers, en filets roulés ou non, en paupiettes.

\section*{\centering Filets de merlans garnis.}
\addcontentsline{toc}{section}{ Filets de merlans garnis.}
\index{Filets de merlans garnis}
\index{Filets de merlans garnis, gratinés}

Pour quatre personnes prenez :

\medskip

\footnotesize
\begin{longtable}{rrrp{16em}}
    250 & grammes & d' & eau,                                                                             \\
    200 & grammes & de & vin blanc,                                                                       \\
    150 & grammes & de & champignons,                                                                     \\
    125 & grammes & de & crevettes grises,                                                                \\
    100 & grammes & de & beurre,                                                                          \\
     45 & grammes & de & crème épaisse,                                                                   \\
     25 & grammes & d' & échalotes,                                                                       \\
        &         &  4 & merlans moyens,                                                                  \\
        &         &  2 & laitances de carpes,                                                             \\
        &         &  2 & jaunes d'œufs frais,                                                             \\
        &         &  1 & brochet pesant 250 grammes environ,                                              \\
        &         &  1 & carotte moyenne,                                                                 \\
        &         &    & bouquet garni (persil, thym, laurier, céleri),                                   \\
        &         &    & jus de citron,                                                                   \\
        &         &    & sel et poivre.                                                                   \\
\end{longtable}
\normalsize

Levez les filets des merlans, parez-les.

Enlevez la tête et l'arête du brochet ; passez la chair au tamis.

Décortiquez les crevettes.

Préparez une farce homogène avec la chair du brochet et les queues de crevettes,
et garnissez-en les filets de merlans.

\index{Fumet de poisson}
Faites un fumet de poisson très concentré avec l'eau, le vin blanc, les déchets
des merlans, du brochet et des crevettes, la carotte, les échalotes, {\ppp50\mmm} grammes
de champignons émincés, le bouquet garni, du sel et du poivre ; passez-le.

Beurrez un plat de service allant au feu avec {\ppp40\mmm} grammes de beurre, disposez
dedans les filets de merlans, chauffez, puis mouillez avec une partie du fumet de
poisson ; couvrez avec un papier beurré et effectuez la cuisson au four pendant
une vingtaine de minutes.

Faites cuire : d'une part, le reste des champignons dans {\ppp30\mmm} grammes de beurre
avec un peu de jus de citron ; d'autre part, les laitances de carpes dans le reste du
fumet. Tenez le tout au chaud.

Réunissez jus de cuisson des merlans, des laitances et des champignons, réduisez
suffisamment, puis liez ce jus avec les jaunes d'œufs ; montez cette sauce au fouet
avec le reste du beurre et la crème, relevez-la avec du jus de citron, au goût, du sel
et du poivre s'il est nécessaire.

Masquez les filets avec la sauce, disposez autour les champignons et les laitances
dressées sur des canapés, et servez rapidement.

\sk

Comme variante, on pourra ajouter un peu de tomate dans la cuisson du fumet,
supprimer les jaunes d'œufs et la crème qu'on remplacera par une haison à la
farine et par du fromage (gruyère ou parmesan râpé) et de la mie de pain rassis
tamisée. On parsèmera le dessus de petits morceaux de beurre et on poussera au
four pour gratiner.

On aura ainsi des filets de merlans garnis, gratinés.

\section*{\centering Turbot.}
\addcontentsline{toc}{section}{ Turbot.}
\index{Turbot}
\index{Barbue}

Le turbot, « Rhombus maximus », et la barbue, « Rhombus lævis ». sont des
poissons de la famille des Pleuronectidés. Leur chair est délicate et ils
figurent parmi les relevés de poisson dans les menus des grands dîners. Si la
sole à pu être justement qualifiée de reine des mers, le turbot mérite d'en
être le roi.

Les turbots, les turbotins et les barbues sont généralement cuits et servis
entiers, cependant on peut aussi en apprêter les filets ou des darnes.

Lorsque le turbot doit être présenté en entier, on commence par l'ébarber puis,
au moyen d'un couteau effilé, on détache les filets de l'arête sur une longueur
de {\ppp5\mmm} à {\ppp6\mmm} centimètres, on brise l’arête en pliant le poisson
et on bride la tête. Apprêté de la sorte, le poisson ne se déforme pas à la
cuisson.

\index{Turbot court-bouillonné}
Le plus souvent, on fait cuire le turbot dans un court-bouillon à base d'eau
additionnée de {\ppp100\mmm} grammes de lait, de {\ppp15\mmm} grammes de sel
marin et d'une tranche de citron sans zeste par litre, le tout mis dans une
turbotière qu'on place sur le feu. Lorsque le liquide bout, on écume, puis on
éloigne la turbotière sur le coin du fourneau et on achève la cuisson à liquide
simplement frissonnant. Il faut compter, après avoir écumé, une dizaine de
minutes de cuisson par kilogramme de poisson.

\index{Garniture pour poissons}
On dresse le turbot, rendu brillant par une couche de beurre fondu appliquée au
pinceau, sur un plat garni d'une serviette, on l'entoure de persil et on le
sert en envoyant en même temps, dans un légumier, des pommes de terre cuites
à l'eau ou à la vapeur, et, dans une saucière, du beurre fondu, du beurre
maître-d'hôtel ou une sauce hollandaise, au choix.

\sk

Les filets et les darnes de turbot ainsi que le turbotin peuvent être apprêtés
au moyen des procédés indiqués pour la sole.

\sk

Le turbot peut être présenté aussi en pâté, en salade avec ou sans légumes, etc.

\sk

Le turbot, cuit de la veille, émincé et mélangé avec de la béchamel, du riz et
des œufs durs, le tout relevé par de la muscade, du curry et du cayenne, est un
plat indien qui porte le nom de « kadgeri ».

\sk

\index{Barbue court-bouillonnée}
\index{Barbue en salade}

Tous les procédés de préparation indiqués ci-dessus sont applicables à la
barbue.

\section*{\centering Turbot garni.}
\addcontentsline{toc}{section}{ Turbot garni.}
\index{Turbot garni}

Court-bouillonnez un turbot ; dressez-le sur un plat ; lissez-en la surface
avec du beurre fondu que vous étendrez au pinceau ; entourez le poisson avec
une garniture de croustades de queues d'écrevisses à la sauce allemande
maigre\footnote{La sauce allemande maigre est un velouté maigre,
pp. \hyperlink{p0292}{\pageref{pg0292}}, \hyperlink{p0338}{\pageref{pg0338}}, lié
avec des jaunes d'œufs.}, de quenelles de brochet et de crevettes,
\hyperlink{p0328}{p. \pageref{pg0328}}, de bouchées d'huîtres, sauce Mornay,
gratinées ; masquez avec l'allemande maigre et glacez au four.

\index{Garniture pour poissons}
Servez en envoyant en même temps une saucière de sauce allemande maigre et un
légumier de pommes de terre à l'anglaise.

\sk

\index{Barbue garnie}
On peut apprêter de même une barbue.

\section*{\centering Escalopes de turbot panées.}
\addcontentsline{toc}{section}{ Escalopes de turbot panées.}
\index{Escalopes de turbot panées}

Dépouillez un turbot, retirez-en les arêtes. Escalopez les filets, assaisonnez-les
avec sel, poivre et jus de citron, passez-les dans de la farine, puis dans de l'œuf
battu, enfin dans de la mie de pain rassis tamisée.

Disposez les escalopes, ainsi enrobées, dans un plat en porcelaine allant au feu,
beurré au préalable, et faites cuire au four pendant une dizaine de minutes, en
arrosant avec la cuisson.

Servez dans le plat.

Envoyez en même temps une saucière de sauce italienne maigre, que vous
aurez préparée de la façon suivante :

Faites fondre du beurre dans une casserole, saisissez dedans des échalotes
hachées, puis des champignons hachés ; laissez évaporer l'eau des champignons,
mettez ensuite du persil haché, des tomates concassées ou de la purée de
tomates, du sel, du poivre, du fumet de poisson ; laissez cuire de façon
à avoir une sauce suffisamment concentrée. Passez-la, puis ajoutez-y des fines
herbes hachées.

\sk

\index{Escalopes de filets de soles panées}
\index{Escalopes de merlans panées}
\index{Escalopes de turbot panées}
\index{Escalopes de barbue panées}
On peut préparer de même des escalopes de barbue, des filets de soles, de
merlans, etc.

\section*{\centering Filets de barbue à la portugaise.}
\addcontentsline{toc}{section}{ Filets de barbue à la portugaise.}
\index{Filets de barbue à la portugaise}
\index{Barbue à la portugaise}

Pour quatre personnes prenez :

\medskip

\footnotesize
\begin{longtable}{rrrp{16em}}
    500 & grammes & de & tomates,                                                                         \\
    200 & grammes & de & vin blanc,                                                                       \\
    150 & grammes & de & beurre,                                                                          \\
    125 & grammes & de & champignons de couche,                                                           \\
    100 & grammes & d' & eau,                                                                             \\
        &         &  1 & barbue pesant 1 kilogramme environ,                                              \\
        &         &    & carottes,                                                                        \\
        &         &    & oignons,                                                                         \\
        &         &    & persil,                                                                          \\
        &         &    & thym,                                                                            \\
        &         &    & laurier,                                                                         \\
        &         &    & sel et poivre.                                                                   \\
\end{longtable}
\normalsize

Levez les filets de la barbue, réservez tête, arête et peau.

Mettez dans une casserole les déchets de barbue, des carottes, des oignons, du
persil, du thym, du laurier, du sel et du poivre ; laissez bouillir pendant deux
heures environ. Passez le fumet obtenu au chinois.

Faites cuire au four les filets de barbue dans une partie de ce fumet additionné
de {\ppp50\mmm} grammes de beurre. Tenez-les au chaud.

Réduisez à bonne consistance le reste du fumet auquel vous aurez ajouté le jus
de cuisson de la barbue,

\index{Garniture portugaise}
Préparez la garniture portugaise : pelez, épépinez et coupez en morceaux les
tomates ; pelez les champignons, émincez-les.

Mettez dans une casserole les tomates, les champignons, un oignon moyen, du
persil haché et {\ppp50\mmm} grammes de beurre ; laissez cuire jusqu'à évaporation du
liquide, ce qui demande un quart d'heure environ.

\index{Garniture pour poissons}
Dressez les filets dans un plat ; disposez autour la garniture de tomates et de
champignons, masquez avec le fumet réduit, ajoutez le reste du beurre coupé en
petits morceaux, poussez au four pour quelques minutes et servez.

\sk

\index{Filets de soles à la portugaise}
\index{Filets de turbot à la portugaise}
\index{Filets de turbotin à la portugaise}
On peut préparer de même des filets de soles, de turbotin ou de turbot.

\section*{\centering Barbue aux pointes d'asperges.}
\addcontentsline{toc}{section}{ Barbue aux pointes d'asperges.}
\index{Barbue aux pointes d'asperges}

Préparez des pointes d'asperges à la crème, comme il est dit
\hyperlink{p0760}{p. \pageref{pg0760}}.

Faites cuire une barbue dans un court-bouillon au vin aromatisé et assaisonné
convenablement ; retirez l'arête sans briser le poisson ; farcissez-le avec les
pointes d'asperges ; disposez-le ainsi apprêté dans un plat beurré allant au
feu ; saupoudrez le dessus avec du fromage de Gruyère râpé et faites gratiner
vivement au four. Servez aussitôt.

\sk

\index{Barbue aux épinards}
Comme variante, on peut employer au lieu de pointes d'asperges des épinards
blanchis dans de l'eau salée, non hachés et amalgamés avec de la crème.

\sk

\index{Filets de sole aux épinards}
\index{Turbotins aux épinards}
On peut préparer de même des soles et des turbotins.

\sk

Il est facile de combiner toute une série de plats gratinés de poissons et de
légumes qu'on exécutera d'une manière analogue.

\section*{\centering Barbue marinée, panée.}
\addcontentsline{toc}{section}{ Barbue marinée, panée.}
\index{Barbue marinée}
\index{Barbue marinée, panée}
\label{pg0385} \hypertarget{p0385}{}

Nettoyez et videz le poisson, incisez-le sur le dos, passez-le dans du jus de
citron, puis faites-le mariner pendant trois heures dans du vin blanc
assaisonné avec du sel et du poivre et aromatisé avec un bouquet garni.

Retirez-le de la marinade, trempez-le dans du beurre fondu et enrobez-le de
mie de pain rassis tamisée.

Disposez la barbue ainsi apprêtée dans un plat foncé de beurre, salez, poivrez,
mettez encore quelques petits morceaux de beurre par-dessus ; faites cuire au
four.

\index{Garniture pour poissons}
Décorez le plat avec des tranches de citron et du persil frit.

Servez, en envoyant à part une saucière de sauce hollandaise au beurre
d'écrevisses, \hyperlink{p0363}{p. \pageref{pg0363}}.

\sk

\index{Sole marinée}
\index{Turbot mariné}
\index{Filets de turbot marinés, panés}
\index{Filets de soles marinés, panés}
On peut préparer de même d’autres poissons, notamment des soles et des turbots.

\sk

On peut également, au lieu de poissons entiers, ne prendre que les filets : le
plat est moins présentable, mais il a l'avantage de ne pas contenir d'arêtes.

\section*{\centering Pimentade de filets de daurade\footnote{\textit{Chrysophrys aurata},
                                           poisson de mer de la famille des Sparidés.}.}
\addcontentsline{toc}{section}{ Pimentade de filets de daurade.}
\index{Pimentade de filets de daurade}
\index{Daurade en pimentade}
\index{Définition des pimentades}
\index{Pimentades (Définition des)}

La pimentade est un ragoût de poisson à sauce pimentée longue, qu'on mange
surtout aux Antilles et en Guyane, et qu'on sert au commencement du repas. La
meilleure pimentade que j'ai mangée était une pimentade de têtes
d'\textit{aïmara}, gros poisson de la Guyane, de la dimension du saumon, que
l'on pêche à l'embouchure des rivières et dont les joues sont très délicates.
Mais le plus souvent on emploie des poissons de mer moins fins, notamment le
\textit{mâchoiran blanc}, poisson de vase, qui ne serait guère mangeable
autrement.

Aux colonies, la pimentade est toujours servie accompagnée de \textit{couac}
(semoule de manioc) ou de \textit{cassave} (galette de couac).

Il est bien dificile de faire en France la pimentade des colonies, faute de
matières premières identiques.

Voici cependant une formule de pimentade, préparée avec des filets de daurade,
qui peut donner une idée du plat créole ; elle est intéressante à ce point de
vue.

\medskip

Pour six personnes prenez :

\medskip

\footnotesize
\begin{longtable}{rrrrp{16em}}
  & 500 & grammes     & de & tomates,                                                                     \\
  & 300 & grammes     & de & vin blanc,                                                                   \\
  & 125 & grammes     & de & bouillon de poisson,                                                         \\
  & 125 & grammes     & de & carottes,                                                                    \\
  &  90 & grammes     & de & beurre,                                                                      \\
  &  15 & grammes     & d' & oignon,                                                                      \\
  &  10 & grammes     & de & farine,                                                                      \\
  &   5 & grammes     & de & sel blanc,                                                                   \\
2 & \multicolumn{2}{r}{décigrammes}   & de & paprika,                                                     \\
  &     &             & 10 & petits piments de la Guyane, appelés \textit{cacaral},                       \\
  &     &             &  1 & daurade pesant 1 200 grammes environ,                                        \\
  &     &             &  1 & citron (autant que possible citron vert des Antilles),                       \\
  &     &             &  1 & gousse d'ail,                                                                \\
  &     &             &  1 & bouquet garni.                                                               \\
\end{longtable}
\normalsize

Levez les filets de la daurade, mettez-les dans un vase avec le sel, le
paprika, l'ail émincé et le jus de la moitié du citron ; laissez en contact
pendant une heure.

\index{Court-bouillon pour pimentades}
Préparez un court-bouillon avec le vin blanc, le bouillon de poisson, le
bouquet garni, {\ppp10\mmm} grammes d'oignon, les carottes émincées, les piments, la
moitié du zeste du citron ; faites cuire dedans les arêtes et la tête de la
daurade pendant une demi-heure. Passez-le.

Faites fondre les tomates dans une casserole, passez-les et ajoutez-les au
court-bouillon passé. Réservez.

Mettez dans une sauteuse le beurre, la farine, le reste de l'oignon haché,
faites blondir pendant quelques minutes ; ajoutez ensuite le mélange réservé,
les filets de daurade, le reste du jus de citron ; laissez mijoter doucement
pendant une demi-heure.

\index{Garniture pour poissons}
Disposez les filets dans un plat creux, versez dessus la sauce et servez, en
envoyant à part, pour remplacer le couac, du riz sec, dans un légumier.

\section*{\centering Daurade froide.}
\addcontentsline{toc}{section}{ Daurade froide.}
\index{Daurade froide}

Prenez une belle daurade, nettoyez-la, parez-la, mettez-la dans un plat allant
au feu, couvrez-la avec un mélange d'huile d'olive et d'eau dans la proportion
d'une partie d'huile pour trois parties d'eau, ajoutez du vinaigre et de l'ail
au goût, un peu de persil haché, du jus de citron ; faites cuire au four.

Retirez le poisson lorsqu'il est cuit, dressez-le sur un plat de service,
masquez-le avec une réduction de la cuisson passée. Laissez refroidir.

\sk

On peut apprêter de même des sardines\footnote{Clupea sardina, famille des
Clupéidis.} fraîches et des maquereaux.

\section*{\centering Maquereaux\footnote{Scomber scomber, famille des
Scombéridés. Un savant du \textsc{xviii}\textsuperscript{e} siècle, le
D\textsuperscript{r} Louis Lemery, docteur régent de la Faculté de Médecine de
Paris, membre de l'Académie royale des Sciences, dans son \textit{Traité des
Aliments}, 2\textsuperscript{e} édition, p. {\ppp397\mmm}, paru à Paris, en
{\ppp1705\mmm}, chez Pierre Witte, à l'Ange Gardien, rue Saint-Jacques, dit que
« le nom de maquereau a été donné à ce poisson parce que, aussitôt le printemps
venu, il suit les petites aloses, communément appelées vierges, et il les
conduit aux mâles ».} à la crème.}

\addcontentsline{toc}{section}{ Maquereaux à la crème.}
\index{Maquereaux à la crème}

Prenez des maquereaux moyens ; videz-les ; nettoyez-les ; mettez-les dans une
terrine ; couvrez-les avec du sel gris et laissez-les ainsi pendant deux heures
environ. Secouez-les pour les débarrasser de l'excès de sel et
court-bouillonnez-les ensuite.

Enlevez les têtes, les peaux et les arêtes ; émincez le filets : dressez-les
sur un plat ; laissez-les refroidir.

Assaisonnez-les avec du vinaigre de vin ; masquez-les avec de la crème double
et parsemez le dessus avec un peu de civette hachée.

C'est un hors-d'œuvre agréable.

\section*{\centering Harengs\footnote{Clupea harengus, famille des Clupéidés.}
                     grillés, sauce moutarde maigre.}

\addcontentsline{toc}{section}{ Harengs grillés, sauce moutarde maigre.}
\index{Harengs grillés, sauce moutade maigre}

Pour six personnes prenez :

\medskip

\footnotesize
\begin{longtable}{rrrrp{16em}}
  & 500 & grammes    & d’ & eau,                                                                          \\
  & 125 & grammes    & de & beurre,                                                                       \\
  & 125 & grammes    & de & crevettes grises, cuites,                                                     \\
  &  10 & grammes    & de & farine,                                                                       \\
  & \multicolumn{2}{r}{1 décigramme}  & de & cayenne,                                                     \\
  &     &            &  6 & harengs laités,                                                               \\
  &     &            &  3 & jaunes d'œufs frais,                                                          \\
  &     &            &  2 & citrons,                                                                      \\
  &     &            &  1 & merlan moyen,                                                                 \\
  &     &            &  1 & carotte,                                                                      \\
  &     &            &  1 & blanc de poireau,                                                             \\
  &     &            &  1 & navet,                                                                        \\
  &     &            &    & moutarde\footnote{La proportion de moutarde dépend
                                           de la qualité employée. Il faudra
                                           évidemment mettre moins de moutarde
                                           anglaise, qui est forte, que de
                                           moutarde ordinaire.},                                         \\
  &     &            &    & sel et poivre.                                                               \\
\end{longtable}
\normalsize

Préparez un beurre de crevettes en passant au tamis les crevettes avec 50
grammes de beurre.

Faites revenir dans {\ppp25\mmm} grammes de beurre carotte, blanc de poireau et
navet coupés en morceaux, mouillez avec l'eau, salez, poivrez ; laissez cuire
pendant une heure. Passez le bouillon, pus liez-le avec la farine que vous
aurez fait dorer dans {\ppp25\mmm} grammes de beurre.

Nettoyez le merlan, coupez-le en morceaux, assaisonnez-le avec sel, poivre et
faites-le cuire dans {\ppp25\mmm} grammes de beurre et le jus de la moitié d'un
citron ; passez le tout au tamis, ajoutez l'extrait obtenu au bouillon lié ;
vous aurez ainsi un velouté maigre.

Videz, lavez, essuyez les harengs, incisez-les légèrement à la surface,
salez-les et faites-les griller à feu vif pendant {\ppp3\mmm} à {\ppp4\mmm}
minutes de chaque côté, sur un gril graissé au préalable.

Au moment de servir, achevez la liaison du velouté avec les jaunes d'œufs,
vous aurez ainsi une allemande maigre que vous relèverez avec le cayenne et de
la moutarde au goût, et que vous monterez au fouet avec le beurre de crevettes.

Servez les harengs grillés sur un plat chaud décoré de quartiers de citron, et
la sauce dans une saucière.

\section*{\centering Harengs à la poêle, maître d'hôtel.}
\addcontentsline{toc}{section}{ Harengs à la poêle, maître d'hôtel.}
\index{Harengs à la poêle, maître d'hôtel}

Videz les harengs, lavez-les, essuyez-les, ciselez-les, assaisonnez-les avec
sel et poivre.

Versez dans une poêle autant de cuillerées à bouche d'huile d'olive qu'il
y a de harengs ; lorsqu'elle sera bien chaude, mettez les poissons et laissez-les cuire
pendant {\ppp3\mmm} à {\ppp4\mmm} minutes de chaque côté ; enlevez-les, égouttez-les.

\index{Beurre maître-d'hôtel}
Servez les harengs sur un plat arrosé de beurre fondu dans lequel vous aurez
mis du persil haché, du sel, du poivre, du vinaigre ou du jus de citron, au goût.

Ce procédé de cuisson conserve aux harengs toute leur saveur.

\section*{\centering Harengs salés marinés.}
\addcontentsline{toc}{section}{ Harengs salés marinés.}
\index{Harengs salés marinés}

Pour huit personnes prenez :

\medskip

\footnotesize
\begin{longtable}{rrrp{16em}}
     45 & grammes & de & vinaigre,                                                                        \\
     45 & grammes & de & vin blanc,                                                                       \\
     45 & grammes & d' & huile d'olive,                                                                   \\
        &         &  4 & beaux harengs laités, salés,                                                     \\
        &         &  2 & échalotes hachées fin,                                                           \\
        &         &  2 & clous de girofle,                                                                \\
        &         &    & thym,                                                                            \\
        &         &    & laurier,                                                                         \\
        &         &    & poivre en grains.                                                                \\
\end{longtable}
\normalsize

Faites tremper pendant {\ppp24\mmm} heures les harengs dans de l'eau pour les dessaler,
puis enlevez-en la peau, sortez les laitances et lavez le tout à l'eau froide.
Écrasez les laitances jusqu'à ce qu'elles soient réduites en bouillie ;
passez-les au travers d'une passoire fine ; ajoutez-y le vinaigre, le vin blanc
et l'huile, en tournant pendant quelques minutes, les échalotes, le girofle, le
thym, le laurier et le poivre en grains ; versez le tout sur les harengs et
laissez-les mariner.

On peut servir ces harengs entiers ou en filets, ou encore coupés en petits
morceaux ; c'est un excellent hors-d'œuvre.

\sk

Ces harengs, ainsi que leur sauce passée à la passoire fine, peuvent être
mélangés à une salade de pommes de terre, dont ils relèvent remarquablement
l'assaisonnement.

\section*{\centering Filets de harengs saurs marinés.}
\addcontentsline{toc}{section}{ Filets de harengs saurs marinés.}
\index{Filets de harengs saurs marinés}

Les filets de harengs saurs marinés constituent un hors-d'œuvre très
appétissant. En hiver, la préparation peut se conserver pendant plusieurs jours
sans s'altérer ; lorsqu'il fait chaud, il est préférable de la manger le jour
même.

\medskip

Pour quinze à vingt personnes :

\medskip

\footnotesize
\begin{longtable}{rrrrp{16em}}
   & 500 & grammes & de & lait,                                                                           \\
   & 250 & grammes & d' & huile d'olive,                                                                  \\
   & 250 & grammes & de & crème épaisse,                                                                  \\
   & 100 & grammes & de & vinaigre de vin,                                                                \\
   & 100 & grammes & de & cornichons hachés fin,                                                          \\
   &  70 & grammes & de & moutarde,                                                                       \\
   &   7 & grammes & de & cerfeuil haché fin,                                                             \\
   &   7 & grammes & d' & échalote hachée fin,                                                            \\
   &   5 & grammes & de & sel blanc,                                                                      \\
   &   3 & grammes & de & paprika,                                                                        \\
50 & \multicolumn{2}{r}{centigrammes} & de & poivre fraîchement moulu,                                    \\
   &     &         & 10 & beaux harengs saurs, dont 5 laités et 5 œuvés,                                  \\
   &     &         &  5 & jaunes d'œufs durs.                                                             \\
\end{longtable}
\normalsize

Faites dessaler les harengs dans de l'eau tiède, pendant {\ppp24\mmm} heures ;
levez les filets, les œufs, les laitances et mettez-les à tremper d'abord dans
le lait pendant {\ppp24\mmm} heures, puis dans {\ppp80\mmm} grammes d'huile
d'olive, pendant le même laps de temps.

Retournez-les plusieurs fois dans la marinade.

Retirez de l'huile les filets, les œufs, les laitances, égouttez-les.

Émincez les filets et assaisonnez-les avec une sauce préparée de la façon
suivante.

Écrasez les jaunes d'œufs durs, les laitances et les œufs des harengs,
ajoutez-y le sel, le poivre, le paprika, la moutarde et amalgamez le tout
ensemble : versez alors, par petites quantités et en alternant, le reste de
l'huile et le vinaigre, remuez, mettez les cornichons, le cerfeuil et
l'échalote, mélangez bien ; enfin, ajoutez la crème et homogénéisez le tout.

Les filets de harengs ainsi préparés sont servis en terrine ou dans des raviers.

\section*{\centering Sandwichs aux harengs saurs.}
\addcontentsline{toc}{section}{ Sandwichs aux harengs saurs.}
\index{Sandwichs aux harengs saurs}

Les harengs saurs ordinaires sont généralement trop salés ; aussi, est-il
indispensable de les adoucir par un séjour prolongé dans du lait. Seuls, les
« Yarmouth bloaters » de Yarmouth, dans le comté de Norfolk, peuvent être
employés directement car ils sont peu salés et peu fumés. Mais ils ne se
conservent que peu de jours.

Pour préparer les sandwichs, on opère de la façon suivante : on ébouillante les
poissons, on enlève la peau, on détache les arêtes ; on ne garde que les filets
qu'on fait cuire dans une casserole avec du beurre, un peu de cayenne et de
muscade, pendant une dizaine de minutes. On passe le tout au tamis, puis on
ajoute à la purée du beurre frais, en travaillant de façon à obtenir un mélange
ayant une bonne consistance.

On garnit avec cette purée des tranches minces de pain anglais qu'on couvre
avec d'autres tranches minces non garnies,

C'est un hors-d'œuvre agréable.

\section*{\centering Morue.}
\addcontentsline{toc}{section}{ Morue.}
\index{Morue}

Les morues « Gadus », de la famille des Gadidés, sont de gros poissons voraces
des mers arctiques. Ils ne descendent guère en Europe au-dessous de l'Islande
et de la mer du Nord, en Amérique, au-dessous de Terre-Neuve.

On désigne sous le nom de morue des poissons d'espèces un peu différentes ;
deux seulement méritent d'être retenus en gastronomie : la morue franche « Gadus
morrhua », à peau verdâtre, qui peut atteindre {\ppp1\mmm} mètre de longueur, très
recherchée pour sa chair et pour l'huile qu'on extrait de son foie ; le gade églefin
« Gadus æglefinus », ou morue noire de Saint-Pierre, qui ne diffère de la morue
franche que par sa taille plus petite et par sa couleur gris foncé.

\sk

\index{Cabillaud}
On donne le nom de cabillaud à la morue fraîche ; salée pour la conserve, elle
est appelée morue verte ; séchée, elle est connue sous le nom de merluche, de
stockfisch.

\sk

\index{Églefin}
Le gade églefin, vulgairement églefin, est pêché surtout dans la mer du Nord,
mais aussi dans la mer d'Irlande et même sur les côtes de la Bretagne. Il est
consommé frais, en particulier sur les lieux de pêche, mais souvent on le fume
pour le conserver.

\sk

Voici quelques manières d'accommoder la morue.

On prépare le cabillaud ainsi que l'églefin frais de différentes façons :
court-bouillonnés, frits, grillés, bouillis ou pochés. Dans les trois derniers
modes de préparation, ils sont le plus souvent servis avec des pommes de terre
à l'anglaise, le tout saupoudré de persil haché et masqué avec du beurre fondu.

\sk

On peut aussi, cela va sans dire, les apprêter suivant les formules indiquées
ci-dessous pour la morue et le hadock,

\section*{\centering Morue aux pommes de terre.}
\addcontentsline{toc}{section}{ Morue aux pommes de terre.}
\index{Morue aux pommes de terre}

Pour six personnes prenez :

\medskip

\footnotesize
\begin{longtable}{rrrrp{16em}}
  & 750 & grammes     & de & morue dessalée,                                                              \\
  & 500 & grammes     & de & pommes de terre épluchées,                                                   \\
  & 250 & grammes     & de & beurre,                                                                      \\
  &  10 & grammes     & d’ & ail,                                                                         \\
  &  10 & grammes     & de & persil haché,                                                                \\
2 & \multicolumn{2}{r}{décigrammes} & de & poivre,                                                        \\
  &     &             & le & jus d'un citron.                                                             \\
\end{longtable}
\normalsize

Faites cuire séparément la morue à l’eau, les pommes de terre à la vapeur.

Débarrassez la morue de la peau et des arêtes, émincez-la ; coupez les pommes
de terre en tranches. Tenez au chaud,

Faites cuire à petit feu, pendant cinq minutes, l'ail dans {\ppp125\mmm} grammes de
beurre : passez le beurre aromatisé.

Mettez dans une sauteuse, en alternant, émincés de morue et tranches de pommes
de terre ; arrosez au fur et à mesure avec le beurre aromatisé, poivrez,
saupoudrez avec le persil, faites sauter pendant un instant, sans laisser
dorer, en ajoutant le reste du beurre coupé en petits morceaux ; aspergez de
jus de citron et servez.

Dans cette préparation, la morue et les pommes de terre se confondent
absolument ; la proportion d'ail indiquée parfume très agréablement le plat
sans aucun excès et le jus de citron y apporte discrètement sa note aigrelette.

Les amateurs d'ail pourront en mettre davantage.

\sk

En remplaçant les {\ppp125\mmm} grammes de beurre, ajoutés à la fin, par {\ppp200\mmm} grammes
de crème épaisse, on adoucira le plat et on le rendra plus moelleux,

\section*{\centering Morue aux haricots.}
\addcontentsline{toc}{section}{ Morue aux haricots.}
\index{Morue aux haricots}

Pour six à huit personnes prenez :

\medskip

\footnotesize
\begin{longtable}{rrrp{16em}}
    750 & grammes & de & filets de morue dessalée,                                                        \\
    500 & grammes & de & haricots secs,                                                                   \\
    200 & grammes & de & beurre,                                                                          \\
    150 & grammes & de & crème,                                                                           \\
        &         &  3 & oignons,                                                                         \\
        &         &  2 & carottes,                                                                        \\
        &         &  2 & poireaux moyens,                                                                 \\
        &         &  1 & gousse d'ail,                                                                    \\
        &         &  1 & bouquet garni (persil, thym, laurier),                                           \\
        &         &    & persil blanchi, haché,                                                           \\
        &         &    & sel et poivre.                                                                   \\
\end{longtable}
\normalsize

Mettez les haricots pendant une journée dans de l'eau froide ; ils se ramolliront
et ils gonfleront.

Faites cuire en même temps :

\textit{a}) d'une part, les haricots dans de l'eau salée, avec les poireaux,
les carottes, un oignon et le bouquet garni. Retirez les haricots ; réservez le
bouillon ; tenez le tout au chaud,

\textit{b}) d'autre part, la morue dans de l'eau ; enlevez la peau et les
arêtes ; émincez les filets ; tenez-les au chaud.

Faites revenir, sans laisser prendre couleur, l'ail et les deux oignons restants
dans {\ppp100\mmm} grammes de beurre, mouillez avec la quantité nécessaire de bouillon
de haricots, salez, poivrez et laissez cuire suffisamment pour aromatiser le
liquide.

Passez-le à la passoire fine, donnez-lui du corps avec de la purée obtenue en
passant au tamis quelques cuillerées de haricots cuits. Montez cette sauce au
fouet avec le reste du beurre et la crème ; chauffez, goûtez, complétez
l'assaisonnement s'il y a lieu, de manière à avoir une sauce un peu relevée.

Mettez les émincés de morue et les haricots dans un plat, saupoudrez avec le
persil haché, versez dessus la sauce et servez.

\sk

Comme variante, on pourra exécuter le même plat en ajoutant à la sauce de la
purée de tomates concentrée ({\ppp250\mmm} grammes sont nécessaires) qui sera mise en
même temps que la purée de haricots. On fera cuire de façon à avoir une sauce
tomatée de consistance convenable, qu'on terminera avec le beurre et la crème,

\sk

Comme autre variante, on pourra diminuer de moitié la quantité des haricots,
que l’on remplacera par douze croûtons frits dans du beurre.

\section*{\centering Morue aux haricots, gratinée.}
\addcontentsline{toc}{section}{ Morue aux haricots, gratinée.}
\index{Morue aux haricots, gratinée}

Pour six à huit personnes prenez :

\medskip

\footnotesize
\begin{longtable}{rrrp{16em}}
    750 & grammes & de & lait,                                                                            \\
    600 & grammes & de & morue dessalée,                                                                  \\
    500 & grammes & de & haricots secs,                                                                   \\
    150 & grammes & de & beurre,                                                                          \\
    150 & grammes & de & crème,                                                                           \\
    125 & grammes & de & champignons,                                                                     \\
     90 & grammes & de & carottes,                                                                        \\
     75 & grammes & de & fromage de Gruyère râpé,                                                         \\
     40 & grammes & d' & oignons,                                                                         \\
     20 & grammes & de & farine,                                                                          \\
      5 & grammes & de & céleri,                                                                          \\
      5 & grammes & de & sel blanc,                                                                       \\
      2 & grammes & de & poivre,                                                                          \\
        &         &  2 & bouquets garnis (thym, persil, laurier),                                         \\
        &         &    & quatre épices.                                                                   \\
\end{longtable}
\normalsize

Mettez à tremper les haricots pendant une journée dans de l'eau froide, puis
faites-les cuire, avec la moitié des oignons et un bouquet garni, dans de l'eau
salée. Égouttez-les.

Faites cuire la morue dans de l'eau ; retirez la peau et les arêtes ; émincez
les filets.

Préparez la sauce. Faites bouillir le lait, mettez dedans les champignons, les
carottes, le reste des oignons épluchés et émincés, le céleri, le second
bouquet, le poivre, le sel blanc, un peu de quatre épices au goût ; laissez
cuire pendant une heure environ, de façon à obtenir à peu près {\ppp350\mmm} grammes de
jus. Passez-le.

Maniez la farine avec le beurre, mouillez avec le jus passé, laissez cuire
pendant dix minutes, ajoutez la crème, {\ppp60\mmm} grammes de gruyère, mélangez en
tournant pendant quatre à cinq minutes, de façon à obtenir une crème onctueuse
bien liée. Goûtez et complétez l'assaisonnement s'il est nécessaire.

Mettez les émincés de morue et les haricots, en couches alternées, dans un plat
allant au feu, masquez le tout avec la sauce, saupoudrez la surface avec le
reste du gruyère et poussez au four pour gratiner.

Servez dans le plat.

\section*{\centering Morue à la crème en turban de pilaf au curry.}
\addcontentsline{toc}{section}{ Morue à la crème en turban de pilaf au curry.}
\index{Morue à la crème en turban de pilaf au curry}

Pour six à huit personnes prenez :

\medskip

\footnotesize
\begin{longtable}{rrrp{16em}}
    750 & grammes & de & filets de morue dessalée,                                                        \\
    400 & grammes & de & riz,                                                                             \\
    250 & grammes & de & beurre,                                                                          \\
    200 & grammes & de & crème,                                                                           \\
     10 & grammes & de & persil haché,                                                                    \\
    1/2 & gramme  & de & curry,                                                                           \\
        &         &  1 & petit oignon (facultatif),                                                       \\
        &         &    & jus de citron,                                                                   \\
        &         &    & poivre.                                                                          \\
\end{longtable}
\normalsize

Faites cuire la morue dans de l’eau ; débarrassez-la de la peau et des arêtes ;
tenez-la au chaud. Réservez l’eau.

Mettez dans une casserole la moitié du beurre, l'oignon haché, le riz ; laissez
dorer, puis mouillez avec une partie de l'eau réservée et ajoutez le curry ;
continuez la cuisson jusqu'à évaporation complète du liquide : les grains de
riz doivent rester entiers et non agglomérés.

Garnissez un moule à couronne avec le pilaf ; tenez-le au chaud au bain-marie.

Faites sauter la morue dans le reste du beurre, assaisonnez avec poivre et jus
de citron, saupoudrez avec le persil haché, ajoutez la crème ; mélangez.
Chauffez pendant quelques minutes.

Démoulez le turban de pilaf sur un plat, garnissez-en l'intérieur avec la morue
et sa sauce, puis servez.

\section*{\centering Morue à l’espagnole.}
\addcontentsline{toc}{section}{ Morue à l’espagnole.}
\index{Morue à l'espagnole}

Pour six personnes prenez :

\medskip

\footnotesize
\begin{longtable}{rrrrp{16em}}
  & 600 & grammes & de & morue dessalée,                                                                  \\
  & 600 & grammes & de & pommes de terre,                                                                 \\
  & 500 & grammes & de & tomates,                                                                         \\
  & 400 & grammes & de & piments rouges d'Espagne, frais\footnote{ ou, à défaut,
                                100 grammes de poudre de piment rouge d'Espagne.},                        \\
  & 100 & grammes & d' & huile d'olive,                                                                   \\
  &  40 & grammes & d' & oignons,                                                                         \\
  &  10 & grammes & d' & ail,                                                                             \\
  &  10 & grammes & de & farine,                                                                          \\
2 & \multicolumn{2}{r}{décigrammes} & de & poivre fraîchement moulu,                                      \\
  &     &         &    & bouquet garni (persil, thym, laurier),                                           \\
  &     &         &    & mie de pain rassis tamisée,                                                      \\
  &     &         &    & sel.                                                                             \\
\end{longtable}
\normalsize

Faites cuire la morue dans de l’eau, égouttez-la, retirez-en les arêtes ;
coupez-la en morceaux ; réservez {\ppp200\mmm} grammes de bouillon de cuisson.

Pelez les piments, émincez-les en languettes et saupoudrez-les d'un décigramme
de poivre.

Faites revenir dans l'huile les oignons pelés et hachés, ajoutez les tomates
coupées en morceaux, l'ail, le bouquet garni, le reste du poivre, mouillez avec
le bouillon de morue réservé et laissez cuire pendant dix minutes ; liez
ensuite avec la farine, continuez la cuisson pendant quelques minutes encore,
goûtez et complétez l'assaisonnement avec un peu de sel, s'il y a lieu ; passez
la sauce.

En même temps, faites cuire les pommes de terre à la vapeur, pelez-les et
coupez-les en tranches.

Prenez un plat allant au feu, étalez au fond une couche de tranches de pommes
de terre, mettez dessus une couche de morceaux de morue, par-dessus une couche
de languettes de piment, mouillez avec un peu de sauce et répétez les mêmes
alternances jusqu'à épuisement des substances ; saupoudrez de mie de pain et
faites cuire au four jusqu'à ce que le plat ait pris une consistance onctueuse,
ce qui demande une demi-heure environ et s'obtient lorsque le liquide est
presque complètement évaporé.

C'est une excellente préparation.

\sk

\index{Cabillaud à l'espagnole}
En remplaçant la morue par du cabillaud cuit dans un court-bouillon au vin
blanc, \hyperlink{p0324}{p. \pageref{pg0324}}, on améliore encore le plat.

\section*{\centering Timbale de morue.}
\addcontentsline{toc}{section}{ Timbale de morue.}
\index{Timbale de morue}

Pour six à huit personnes prenez :

\medskip

\footnotesize
\begin{longtable}{rrrrp{16em}}
   & 750 & grammes & de & filets de morue dessalée,                                                       \\
   & 250 & grammes & de & macaroni,                                                                       \\
   & 250 & grammes & de & champignons de couche,                                                          \\
   & 250 & grammes & de & purée de tomates concentrée,                                                    \\
   & 200 & grammes & de & crème,                                                                          \\
   & 175 & grammes & de & beurre,                                                                         \\
   & 125 & grammes & de & fumet de poisson,                                                               \\
   &  50 & grammes & de & câpres au vinaigre,                                                             \\
15 & \multicolumn{2}{r}{centigrammes} & de & cayenne,                                                     \\
   &     &         &    & piment d'Espagne,                                                               \\
   &     &         &    & jus de citron,                                                                  \\
   &     &         &    & sel et poivre.                                                                  \\
\end{longtable}
\normalsize

Pelez les champignons, passez-les dans du jus de citron.

Faites cuire en même temps :

\textit{a}) la morue dans de l'eau ; enlevez la peau et les arêtes ; émincez la chair ;

\textit{b}) les champignons dans du beurre ;

\textit{c}) le macaroni dans de l'eau salée ; égouttez-le.

Tenez le tout au chaud.

Délayez la purée de tomates dans le fumet, ajoutez-y, au goût, plus ou moins de
piment émincé ; chauffez.

Faites sauter les émincés de morue dans {\ppp75\mmm} grammes de beurre,

Passez le macaroni dans le reste du beurre sans lui laisser prendre couleur.

Réunissez fumet tomaté, émincés de morue, macaroni, champignons et leur
cuisson, ajoutez le cayenne, les câpres et la crème, mélangez en évitant de
briser le macaroni, chauffez, goûtez et complétez l'assaisonnement s'il
y a lieu avec sel et poivre, puis versez le tout dans une timbale en porcelaine
chauffée au bain-marie.

\sk

On peut préparer d’une manière analogue un vol-au-vent de morue en remplaçant,
par exemple, le macaroni par des quenelles de brochet et d'écrevisses,
\hyperlink{p0328}{p. \pageref{pg0328}} et en ajoutant des émincés de truffes
cuites dans du madère. On mettra le tout dans une croûte de vol-au-vent.

\sk

Toutes ces formules sont applicables à la merluche.

\section*{\centering Soufflé de morue aux pommes de terre.}
\addcontentsline{toc}{section}{ Soufflé de morue aux pommes de terre.}
\index{Soufflé de morue aux pommes de terre}

Pour six personnes prenez :

\medskip

\footnotesize
\begin{longtable}{rrrrp{16em}}
  &  600 & grammes & de & morue dessalée,                                                                 \\
  &  600 & grammes & de & pommes de terre épluchées,                                                      \\
  &  100 & grammes & de & crème,                                                                          \\
  &   90 & grammes & de & beurre,                                                                         \\
5 & \multicolumn{2}{r}{centigrammes} & de & poivre fraîchement moulu,                                     \\
1 & \multicolumn{2}{r}{centigramme}  & de & cayenne,                                                      \\
1 & \multicolumn{2}{r}{centigramme}  & de & muscade,                                                      \\
  &      &         &  4 & œufs.                                                                           \\
\end{longtable}
\normalsize

Séparez les blancs d'œufs des jaunes.

Faites cuire, séparément, la morue dans de l'eau, les pommes de terre à la
vapeur.

Enlevez la peau et les arêtes de la morue ; émincez la chair ou passez-la en
purée.

Passez les pommes de terre en purée, puis incorporez-y la crème, {\ppp80\mmm}
grammes de beurre, le poivre, le cayenne, la muscade et les jaunes d'œufs ;
mélangez bien. Ajoutez ensuite la morue et mélangez encore.

Fouettez les blancs d'œufs en neige ferme, incorporez-les à l'appareil,
mélangez une dernière fois, puis versez le tout dans un plat allant au feu et
graissé avec le reste du beurre. Mettez au four chaud pendant une vingtaine de
minutes ; la préparation se soufflera et se colorera. Servez dans le plat.

Le soufflé montera d'autant plus que la morue sera émincée plus fin, mais les
amateurs de morue préféreront la trouver en morceaux, au risque d'avoir un
soufflé moins levé.

\section*{\centering Brandade de morue.}
\addcontentsline{toc}{section}{ Brandade de morue.}
\index{Brandade de morue}

La brandade de morue est un plat provençal qui a la réputation d'être lourd, ce
qui n'est exact que lorsqu'il est préparé sans soin, ou avec des matières
premières défectueuses.

Pour six personnes prenez {\ppp1\mmm} kilogramme de morue bien dessalée,
mettez-la dans une casserole avec beaucoup d'eau, chauffez ; au premier
bouillon couvrez et retirez du feu. Laissez-la pocher pendant une dizaine de
minutes, puis enlevez-en les arêtes et émincez-la.

Mettez dans une casserole les émincés de morue et la peau\footnote{On peut ne
pas mettre la peau ; on obtient alors une brandude de morue plus blanche ; je
préfère la conserver, car elle donne du goût et du liant.} avec {\ppp125\mmm}
grammes d'huile d'olive tiède, plus ou moins fruitée, au goût, et
« brandissez » le tout avec une cuiller, c'est-à-dire agitez, travaillez,
d'abord pendant cinq minutes sur le feu, puis au bain-marie et continuez
à travailler. Lorsque l'huile sera complètement absorbée, ajoutez-en encore,
par petites quantités, en travaillant toujours la masse avec la cuiller, de
façon à obtenir une pâte parfaitement lisse.

Ajoutez alors de la crème, mélangez bien, aromatisez et assaisonnez avec de
l'ail pilé, du jus de citron, du zeste râpé, du poivre fraîchement moulu,
goûtez et complétez l’assaisonnement, s'il y a lieu, avec un peu de
sel\footnote{Il est impossible de préciser les proportions des différents
éléments, car tout dépend de la quantité d'huile absorbée par la morue,
quantité qui varie avec la qualité du poisson.}, puis incorporez à la masse un
hachis de truffes crues ; chauffez pendant un instant et tenez au chaud, au
bain-marie, jusqu'au moment de servir.

Dressez la brandade dans des croûtes de bouchées, ou dans une croûte de
vol-au-vent, décorez avec des lames de truffes cuites dans du madère et servez.

\sk

Comme variantes, on peut incorporer de la sauce béchamel ou quelques pommes de
terre cuites à la vapeur, mais on prépare la véritable brandade comme il est
dit plus haut.

\section*{\centering Hadock\footnote{Nom anglais de l'églefin.}}
\addcontentsline{toc}{section}{ Hadock.}
\index{Hadock}

Le hadock arrive à Paris à peine salé et très peu fumé ; il doit donc être
consommé de suite, car il s'altère très vite. Le plus renommé est le hadock
légèrement fumé de Finnan, petit port d'Écosse, non loin d'Aberdeen.

En Angleterre, on sert le hadock grillé ou bouilli, avec du beurre fondu ou
avec une sauce au curry.

À mon avis, il vaut mieux le faire pocher pendant {\ppp5\mmm} à {\ppp10\mmm} minutes, suivant sa
grosseur, et le servir comme le cabillaud ou l'églefin frais pochés.

\section*{\centering Hadock poché, aux pommes de terre sautées, sauce aux œufs durs.}
\addcontentsline{toc}{section}{ Hadock poché, aux pommes de terre sautées, sauce aux œufs durs.}
\index{Hadock poché, aux pommes de terre sautées, sauce aux œufs durs}
\index{Églefin aux pommes de terre, sauce aux œufs durs}

Pour six à huit personnes prenez :

\medskip

\footnotesize
\begin{longtable}{rrrp{16em}}
  1 000 & grammes & de & pommes de terre de Hollande,                                                     \\
    750 & grammes & de & filets de hadock,                                                                \\
    500 & grammes & de & lait,                                                                            \\
    200 & grammes & de & beurre,                                                                          \\
    100 & grammes & de & crème,                                                                           \\
     30 & grammes & de & farine,                                                                          \\
        &         &  2 & œufs durs,                                                                       \\
        &         &  1 & petite carotte,                                                                  \\
        &         &  1 & petit navet,                                                                     \\
        &         &  1 & petit oignon,                                                                    \\
        &         &  1 & bouquet garni,                                                                   \\
        &         &    & persil haché,                                                                    \\
        &         &    & jus de citron,                                                                   \\
        &         &    & muscade,                                                                         \\
        &         &    & sel et poivre.                                                                   \\
\end{longtable}
\normalsize


Faites cuire les pommes de terre dans de l'eau salée, pelez-les, coupez-les en
tranches, puis faites-les sauter dans {\ppp100\mmm} grammes de beurre ; salez.

Faites pocher, à court mouillement, les filets de hadock dans de l'eau ou du
lait salé (le lait destiné à cet usage vient en plus des {\ppp500\mmm} grammes indiqués
au tableau) ; achevez-en la cuisson sur le coin du fourneau en casserole
couverte. La durée totale de la cuisson est d'un quart d'heure.

En même temps, préparez la sauce.

Faites dorer légèrement dans {\ppp50\mmm} grammes de beurre la carotte, le navet,
l'oignon émincés, mettez ensuite la farine ; tournez pendant quelques minutes
sans laisser prendre couleur ; mouillez avec le lait que vous aurez fait
bouillir, ajoutez le bouquet garni, du sel, du poivre, de la muscade et laissez
mijoter jusqu'à obtention d'une sauce aromatisée de bonne consistance, ce qui
demande une heure environ. Passez-la, montez-la au fouet avec le reste du
beurre et la crème ; chauffez. Au dernier moment, mettez dedans du persil haché
et les œufs durs, hachés grossièrement ou passés au gros tamis.

Dressez les filets de hadock sur un plat, entourez-les avec les pommes de terre
sautées et servez en envoyant en même temps la sauce dans une saucière.

\section*{\centering Thon\footnote{Thynnus vulgaris, famille des Scombéridés.} grillé, sauce tomate.}
\addcontentsline{toc}{section}{ Thon grillé, sauce tomate.}
\index{Thon grillé, sauce tomate}

Prenez des darnes de thon d'un centimètre et demi d'épaisseur environ ;
faites-les griller, à feu vif, sur un gril très chaud, salez-les et,
lorsqu'elles seront cuites, servez-les avec une sauce tomate qui les
accompagnera admirablement. La cuisson demande vingt minutes environ ; elle est
à point lorsque la chair se détache facilement de l'arête.

\sk

\label{pg0401} \hypertarget{p0401}{}
Quelques indications sur la sauce tomate.

Pendant sept mois de l'année environ, il est difficile d'avoir à Paris de
bonnes tomates fraîches à des prix abordables ; il est préférable alors de se
servir d'une purée de tomates de conserve, préparée comme je l'ai indiqué
\hyperlink{p0768}{p. \pageref{pg0768}}, ou de conserves de tomates du commerce et,
en particulier, de conserves italiennes.

\medskip

En hiver, si l’on veut obtenir un quart de litre de sauce, soit le contenu d'une
saucière moyenne, on prendra :

\medskip

\footnotesize
\begin{longtable}{rrrp{16em}}
    250 & grammes & de & purée de tomates,                                                                \\
     60 & grammes & de & beurre,                                                                          \\
     10 & grammes & de & farine,                                                                          \\
        &         &    & bouillon ou jus de viande,                                                       \\
        &         &    & sel et poivre.                                                                   \\
\end{longtable}
\normalsize

On fera cuire pendant cinq minutes la farine avec {\ppp40\mmm} grammes de beurre sans
laisser roussir, on mettra ensuite la purée de tomates, par petites quantités,
en mélangeant bien, on assaisonnera, on éclaircira, s'il y a lieu, avec un peu
de bouillon ou de jus de viande, puis on laissera cuire pendant un quart
d'heure ; enfin, on ajoutera, par petites quantités, le reste du beurre, en le
laissant fondre simplement dans la sauce ; on remuera la sauce pour la rendre
bien homogène et on servira.

Cette façon de faire présente le grand avantage de conserver à la sauce tomate
son caractère dans toute sa pureté et d’obtenir, alors que les fruits frais
sont rares, une sauce qui semble en provenir directement.

\sk

En été, on pourra, cela va sans dire, faire la sauce exactement de même avec
des fruits frais, en commençant comme cela à été indiqué dans la formule de la
préparation de la purée de conserve ; mais il n'y a aucun inconvénient à lui
adjoindre quelques autres éléments, tout en lui conservant naturellement la
suprématie dans la préparation.

C'est dans cet esprit que je donne la formule suivante.

\medskip

Pour faire une saucière de sauce tomate prenez :

\medskip

\footnotesize
\begin{longtable}{rrrp{16em}}
  1 000 & grammes & de & belles tomates fraîches,                                                         \\
    200 & grammes & de & vin blanc,                                                                       \\
     60 & grammes & de & beurre,                                                                          \\
     10 & grammes & de & farine,                                                                          \\
        &         &  2 & oignons,                                                                         \\
        &         &  1 & bouquet garni,                                                                   \\
        &         &    & glace de viande,                                                                 \\
        &         &    & girofle,                                                                         \\
        &         &    & sel et poivre.                                                                   \\
\end{longtable}
\normalsize

Épluchez les tomates, coupez-les en morceaux, mettez-les dans une casserole
avec les oignons, le bouquet, du girofle au goût et le vin ; laissez cuire
pendant une heure à petit feu. Passez au tamis.

Faites revenir la farine pendant {\ppp3\mmm} à {\ppp4\mmm} minutes dans
{\ppp40\mmm} grammes de beurre sans laisser roussir, mouillez avec le jus des
tomates, ajoutez la glace de viande, assaisonnez avec sel et poivre, réduisez
à consistance voulue, goûtez, ajoutez le reste du beurre que vous laisserez
simplement fondre, mélangez bien et servez.

\section*{\centering Raie\footnote{Il existe différentes sortes de raies ; les
meilleures sont la raie ponctuée « Raia punctata », appelée raie douce sur le
marché de Paris et la raie bouclée « Raia clavata », de la famille des Raiidés.}
au beurre.}
\addcontentsline{toc}{section}{ Raie au beurre.}
\index{Raie au beurre}

Pour six personnes prenez :

\medskip

\footnotesize
\begin{longtable}{rrrp{16em}}
  1 500 & grammes  & de & raie douce, pris dans le milieu,                                                \\
    250 & grammes  & de & beurre,                                                                         \\
     60 & grammes  & de & câpres au vinaigre,                                                             \\
     30 & grammes  & de & sel gris,                                                                       \\
        & 2 litres & d' & eau,                                                                            \\
        &          &  1 & carotte,                                                                        \\
        &          &  1 & oignon,                                                                         \\
        &          &  1 & bouquet garni,                                                                  \\
        &          &    & vinaigre,                                                                       \\
        &          &    & jus de citron,                                                                  \\
        &          &    & persil haché,                                                                   \\
        &          &    & sel et poivre.                                                                  \\
\end{longtable}
\normalsize

Nettoyez la raie à l'eau froide ; parez-la.

Mettez dans une poissonière l'eau, l'oignon, la carotte émincée, le bouquet
garni, le sel gris, du poivre en grains et du vinaigre au goût ; laissez cuire
pendant une demi-heure ; puis, plongez la raie dans le court-bouillon
bouillant : l'ébulliton s'arrêtera aussitôt. Donnez un bouillon, éloignez la
poissonnière du feu et achevez la cuisson de la raie sans faire bouillir, ce
qui demande une demi-heure environ.

Retirez la raie. dépouillez-la, égouttez-la ; tenez-la au chaud.

Faites fondre le beurre à la couleur noisette ou au noir\footnote{À mon avis,
le beurre noisette est celui qui convient le mieux ; le beurre noir est plutôt
indigeste.}, mais sans le laisser brûler, salez, poivrez, mettez dedans du
persil haché, puis ajoutez câpres, jus de citron et vinaigre au goût, versez le
tout sur la raie et servez, après avoir garni le plat d'un cordon de persil
frit.

\sk

On peut apprêter au beurre noisette ou au beurre noir d'autres poissons,
notamment le thon et la morue.

\section*{\centering Croustades de foie de raie gratinées.}
\addcontentsline{toc}{section}{ Croustades de foie de raie gratinées.}
\index{Croustades de foie de raie gratinées}
\index{Foies de raies gratinés, en croustades}

Pour quatre personnes prenez :

\footnotesize
\begin{longtable}{@{}lp{2em}rrrp{16em}}
\normalsize1°\footnotesize &  & 300 & grammes & de & foie de raie ;                                       \\
\end{longtable}
\normalsize

2° \hspace{.2em}pour le court-bouillon :

\footnotesize
\begin{longtable}{@{}lp{2em}rrrp{16em}}
\normalsize1°\footnotesize &  & 300 & grammes & de & foie de raie ;                                       \kill
   & & 25 & grammes & d' & oignon ciselé,                                                                 \\
   & &    &         &    & eau,                                                                           \\
   & &    &         &    & vinaigre,                                                                      \\
   & &    &         &    & bouquet garni,                                                                 \\
   & &    &         &    & sel ;                                                                          \\
\end{longtable}
\normalsize

3° \hspace{.2em}pour les croustades :

\footnotesize
\begin{longtable}{@{}lp{2em}rrrp{16em}}
\normalsize1°\footnotesize &  & 300 & grammes & de & foie de raie ;                                       \kill
   & &250 & grammes & de & farine,                                                                        \\
   & &100 & grammes & de & beurre,                                                                        \\
   & & 50 & grammes & d’ & eau,                                                                           \\
   & &  8 & grammes & de & sel,                                                                           \\
   & &  6 & grammes & d' & huile d'olive,                                                                 \\
   & &    &         &  1 & jaune d'œuf ;                                                                  \\
\end{longtable}
\normalsize

\medskip

4° \hspace{.2em}pour la sauce :

\medskip

\footnotesize
\begin{longtable}{@{}lp{2em}rrrp{16em}}
\normalsize1°\footnotesize &  & 300 & grammes & de & foie de raie ;                                       \kill
   & &200 & grammes & de & fumet de poisson,                                                              \\
   & &100 & grammes & de & champignons hachés,                                                            \\
   & & 50 & grammes & de & beurre,                                                                        \\
   & & 50 & grammes & d' & anchois,                                                                       \\
   & & 50 & grammes & de & vin blanc,                                                                     \\
   & & 30 & grammes & d' & oignon ciselé,                                                                 \\
   & & 20 & grammes & d' & échalotes hachées,                                                             \\
   & & 20 & grammes & de & câpres au vinaigre hachées,                                                    \\
   & &  8 & grammes & de & farine,                                                                        \\
   & &  4 & grammes & de & persil et estragon hachés,                                                     \\
   & &    &         &    & cayenne,                                                                       \\
   & &    &         &    & sel et poivre ;                                                                \\
\end{longtable}
\normalsize

5° \hspace{.2em}pour le gratin

\footnotesize
\begin{longtable}{@{}lp{2em}rrrp{16em}}
\normalsize1°\footnotesize &  & 300 & grammes & de & foie de raie ;                                       \kill
   & & 30 & grammes & de & parmesan râpé fin.                                                             \\
   & & 20 & grammes & de & beurre.                                                                        \\
\end{longtable}
\normalsize

Mettez à dégorger le foie de raie dans de l'eau.

Préparez un court-bouillon en faisant cuire pendant une heure les {\ppp25\mmm}
grammes d'oignon et un bouquet garni dans une quantité suffisante d'eau salée,
fortement vinaigrée. Lorsque le court-bouillon sera à point, mettez le foie de
raie ; éloignez la casserole du feu et achevez la cuisson sans ébullition
pendant vingt minutes. Laissez-le refroidir dans le liquide ; essuyez-le ;
coupez-le en morceaux.

Avec les éléments du paragraphe {\ppp3\mmm} préparez une pâte homogène ;
faites-en une abaisse carrée que vous couperez en quatre parties égales et que
vous mettrez sur une plaque en tôle. Piquez la pâte avec un couteau en
différents endroits pour éviter qu'elle se boursoufle à la cuisson, collez sur
le pourtour de chaque morceau un petit rebord de pâte et emplissez les
croustades avec des cailloux lavés ou des haricots ; faites cuire au four.

Faites revenir dans {\ppp20\mmm} grammes de beurre : d'abord la farine qui doit
être bien dorée, puis les champignons, les {\ppp30\mmm} grammes d'oignon et les
échalotes, mouillez avec le fumet de poisson et le vin, laissez cuire à petit
feu pendant une heure environ ; goûtez, complétez l'assaisonnement avec sel,
poivre et cayenne, ajoutez câpres, persil et estragon hachés, donnez quelques
bouillons ; passez ou ne passez pas la sauce.

\index{Beurre de crevettes et d'écrevisses}
Préparez un beurre d'anchois, comme il est indiqué
\hyperlink{p0337}{p. \pageref{pg0337}}, avec {\ppp30\mmm} grammes de beurre et les
anchois ; incorporez-le à la sauce, puis mettez le foie de raie coupé ;
chauffez pendant un instant.

Emplissez les croustades avec ce ragoût, saupoudrez avec le parmesan, mettez
dessus les {\ppp20\mmm} grammes de beurre coupé en petits morceaux et poussez
au four pour gratiner.

Dressez les croustades sur un plat couvert d’une serviette et servez.

C'est une excellente entrée de poisson.

\sk

\index{Coquilles de foie de raie, gratinées}
\index{Foies de raies gratinés, en coquilles}
On peut remplacer les croustades de pâte par des croustades de pain, ou encore
par des valves de coquilles Saint-Jacques : dans ce dernier cas, on aura des
coquilles de foie de raie gratinées.

\sk

Comme variante, on peut remplacer la sauce ci-dessus par une béchamel maigre,
\hyperlink{p0269}{p. \pageref{pg0269}}, dans laquelle on fera entrer des
champignons et des câpres.
