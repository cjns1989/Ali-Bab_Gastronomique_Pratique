\sk

\index{Compotes, confitures et marmelades, gelées, pâtes et sirops de fruits}
\begin{center}
\textit{COMPOTES}
\end{center}

\bigskip

\addcontentsline{toc}{section}{ Compotes.}
\index{Compotes}
\index{Compotes (Définition des)}
\index{Définition des compotes}

Les compotes sont des préparations de fruits pas trop mûrs, entiers ou coupés,
cuits ou pochés dans un sirop parfumé léger. Elles sont dressées et présentées
sur des coupes ou dans des compotiers, telles quelles où nappées d'une gelée,
ou encore masquées avec un sirop de fruits très réduit.

Les compotes sont moins cuites et moins sucrées que les confitures et les gelées.

Elles sont d'autant plus belles et plus fines qu'elles ont été préparées avec plus de
soin. On les sert tièdes ou froides.

\section*{\centering Compote de pommes.}
\phantomsection
\addcontentsline{toc}{section}{ Compote de pommes.}
\index{Compote de pommes}

Prenez des calvilles ou des reinettes, de préférence ; pelez-les, videz-les
avec un tube à colonne, passez-les dans du jus de citron si vous voulez les
laisser entières, puis pochez-les doucement dans du sirop à {\ppp20\mmm}°
parfumé à la vanille ; surveillez l'opération. Lorsque les pommes sont à point,
c'est-à-dire quand elles sont devenues diaphanes, retirez-les et dressez-les en
pyramide dans un compotier. Réduisez le sirop de façon à l'amener à peser
{\ppp30\mmm}° ; laissez-le refroidir. Masquez les pommes avec le sirop
refroidi.

\medskip

Si vous opérez avec des pommes coupées par moitiés, en quartiers ou en
rouelles, enlevez-en toutes les parties dures et plongez, au fur et à mesure,
les tranches de pommes dans de l'eau citronnée, pour les empêcher de noircir ;
faites-les cuire ensuite, sans bouillir, dans du sirop à {\ppp20\mmm}° parfumé
à la vanille, jusqu'à ce qu'elles soient diaphanes. Après les avoir égouttées,
dressez-les en couronne dans un compotier ; réduisez le sirop comme ci-dessus
et lorsqu'il est refroidi, versez-le sur les pommes.

\sk

On peut aussi dresser les tranches de pommes en turban, mettre dans l’intérieur
une compote de coings et napper les deux compotes avec les sirops refroidis.

\section*{\centering Compote de coings.}
\phantomsection
\addcontentsline{toc}{section}{ Compote de coings.}
\index{Compote de coings}

Prenez de beaux coings bien mûrs, pelez-les, coupez-les en morceaux. supprimez
les parties dures. Faites cuire les coings dans de l'eau citronnée, à casserole
couverte, jusqu'à ce qu'ils soient bien tendres. Égouttez-les, puis mettez-les
à mariner dans du sirop à {\ppp25\mmm}°, chaud. Laissez en contact pendant
trois heures.

Réduisez le sirop, versez-le de nouveau sur les fruits. Laissez refroidir.

Dressez ensuite la compote.

\section*{\centering Compote de poires.}
\phantomsection
\addcontentsline{toc}{section}{ Compote de poires.}
\index{Compote de poires}
\index{Compote de poires blanches}

Les meilleurs poires pour compotes sont les beurrées, les doyennés, les poires
de rousselet, de Catillac, de messire Jean, d'Angleterre, etc.

Pelez des poires, laissez entières les petites, videz-les à l’aide d'un tube ;
coupez les moyennes et les grosses poires par moitiés ou par quartiers ;
enlevez-en les parties dures.

Si vous employez des poires fondantes, pochez-les directement dans du sirop
à {\ppp26\mmm}° parfumé à la vanille. Égouttez-les et dressez-les dans un
compotier. Réduisez le sirop pour qu'il pèse {\ppp30\mmm}° et, lorsqu'il est
froid, versez-le sur les poires.

S'il s'agit de poires fermes ou de poires à cuire quelconques, frottez-les de
jus de citron, après les avoir pelées, et blanchissez-les pendant quelques
minutes dans de l'eau bouillante additionnée d'un peu de jus de citron, pour
les attendrir. Faites-les cuire ensuite comme les poires fondantes et
dressez-les de même.

Dans ces compotes, les poires restent blanches.

\medskip

\index{Compote de poires rouges}
Lorsqu'on veut obtenir des compotes de poires rouges, on fait cuire les fruits
dans du sirop au vin rouge qu'on parfume avec du zeste de citron et un peu de
cannelle de Ceylan.

\sk

\index{Compote de poires panachées}
On peut aussi servir des compotes de poires panachées, mi-partie blanches,
mi-partie rouges. Après avoir fait cuire séparément des poires de deux espèces
ditférentes, l’une dans du sirop vanillé, l'autre dans du sirop au vin rouge,
on les dresse en les alternant dans un compotier et on les arrose avec les deux
sirops réunis réduits.

\sk

\index{Couronne de poires garnie de fruits}
\index{Compote de poires, garnie d'autres fruits}
On présente enfin des compotes de poires dressées en couronne dont l'intérieur
est garni de compotes d'autres fruits : cerises, abricots, ananas, framboises,
etc. On les masque avec un bon sirop de fruits à {\ppp32\mmm}°.

\section*{\centering Compote de groseilles.}
\phantomsection
\addcontentsline{toc}{section}{ Compote de groseilles.}
\index{Compote de groseilles}

Egrappez de belles groseilles, rouges ou blanches. Passez au tamis une quantité
égale d'autres groseilles, filtrez le jus ; ajoutez‑y même poids de sucre,
chauffez doucement en remuant, puis laissez refroidir. Lorsque le jus est à peu
près pris, mélangez-v les groseilles et versez le tout dans un compotier.

\section*{\centering Compote d’abricots.}
\phantomsection
\addcontentsline{toc}{section}{ Compote d’abricots.}
\index{Compote d’abricots}

Prenez de beaux abricots de plein vent, de préférence, partagez-les en deux,
enlevez les noyaux. Cassez la moitié des noyaux et mettez les amandes
décortiquées à macérer dans un peu de kirsch.

Échaudez vivement les abricots pour en enlever la peau, puis jetez-les dans du
sirop à {\ppp25\mmm}° bouillant ; donnez deux bouillons ; enlevez-les du feu et
laissez-les refroidir dans le sirop.

Dressez les abricots dans un compotier en garnissant le creux de chaque
demi-abricot avec une demi-amande parfumée au kirsch.

Remettez le sirop sur le feu, réduisez-le à peser {\ppp28\mmm}°, ajoutez‑y le
kirsch de macération des amandes. Lorsqu'il est refroidi, masquez la compote
avec ce sirop parfumé,

\section*{\centering Compote de cerises.}
\phantomsection
\addcontentsline{toc}{section}{ Compote de cerises.}
\index{Compote de cerises}

Prenez de belles cerises de Montmorency, raccourcissez les queues, enlevez les
noyaux sans abîmer les cerises ou ne les enlevez pas, au choix.

Faites cuire du sucre au boulé, à raison de {\ppp500\mmm} grammes de sucre pour
{\ppp1\mmm} kilogramme de cerises ; jetez dedans les cerises, donnez un
bouillon ou deux, versez le tout dans une terrine ; laissez refroidir.

Égouttez les cerises, dressez-les dans un compotier. Réduisez le sirop,
parfumez-le avec du kirsch ou du marasquin. Lorsqu'il est refroidi, versez-le
sur les cerises.

\section*{\centering Compote de fraises.}
\phantomsection
\addcontentsline{toc}{section}{ Compote de fraises.}
\index{Compote de fraises}

Choisissez de belles fraises, épluchez-les soignensement.

Faites cuire du sucre au boulé à raison de {\ppp400\mmm} grammes de sucre pour
un kilogramme de fraises ; jetez dedans les fraises, enlevez immédiatement du
feu, mêlez et laissez en contact, à couvert, pendant {\ppp10\mmm}
à {\ppp15\mmm} minutes.

Égouttez les fraises, dressez-les dans une coupe ; concentrez le sirop,
ajoutez‑y un peu de gelée de groseilles ou de framboises, pour donner du corps,
et quelques gouttes de jus de citron, ou parfumez-le à la vanille. Nappez les
fraises avec le sirop refroidi.

\section*{\centering Compote de framboises.}
\phantomsection
\addcontentsline{toc}{section}{ Compote de framboises.}
\index{Compote de framboises}

Prenez de belles framboises bien saines.

Faites cuire du sucre au boulé à raison de {\ppp400\mmm} grammes de sucre pour
{\ppp1\mmm} kilogramme de framboises ; jetez dedans les framboises, enlevez du
feu, laissez en contact pendant quelques minutes.

Égouttez les framboises, dressez-les dans un compotier.

Mettez dans le sirop plus où moins de jus de framboises passé, réduisez, laissez
refroidir. Nappez les framboises avec le sirop refroidi.

\section*{\centering Compote de mirabelles.}
\phantomsection
\addcontentsline{toc}{section}{ Compote de mirabelles.}
\index{Compote de mirabelles}

Prenez des mirabelles saines et bien à point ; piquez-les profondément en
plusieurs endroits ; enlevez les noyaux ou laissez-les, au choix.

Plongez les mirabelles dans du sirop à {\ppp22\mmm}° bouillant et parfumé
à la vanille, où vous les pocherez, sans bouillir, pendant {\ppp10\mmm}
à {\ppp15\mmm} minutes.

Enlevez les mirabelles. disposez-les dans une coupe ou dans un compotier,

Réduisez le sirop, luissez-le refroidir, puis versez-le sur les fruits.

\section*{\centering Compote de prunes Reine-Claude.}
\phantomsection
\addcontentsline{toc}{section}{ Compote de prunes Reine-Claude.}
\index{Compote de prunes Reine-Claude}

Prenez des prunes « Reine-Claude » un peu fermes mais suffisamment mûres,
raccourcissez les queues, enlevez les noyaux ou laissez-les, puis plongez les
fruits dans du sirop à {\ppp20\mmm}° bouillant et parfumé à la vanille ;
enlevez-les aussitôt du feu et laissez-les pocher longuement dans le sirop.

Dressez les prunes en pyramide dans un compotier et masquez-les avec le sirop
réduit et refroidi.

\section*{\centering Compote de pêches.}
\phantomsection
\addcontentsline{toc}{section}{ Compote de pêches.}
\index{Compote de pêches}

Si vous prenez des pêches d'espalier, plongez-les dans de l'eau bouillante pour
en enlever la peau, égouttez-les, séparez-les en deux et retirez les noyaux.
Mettez les demi-pêches dans une terrine, versez dessus du sirop à {\ppp26\mmm}°
bouillant, parfumé à la vanille ou avec de la liqueur de noyaux ; laissez
refroidir.

Dressez les pêches dans un compotier et arrosez-les avec leur sirop froid.

\medskip

On opère un peu différemment lorsqu'on emploie des pêches dures.

Pelez les pêches, séparez-les en deux, enlevez les noyaux. Pochez les fruits
à couvert dans de l'eau bouillante jusqu'à ce qu'ils soient ramollis ;
égouttez-les. Mettez-les dans une terrine, versez dessus du sirop
à {\ppp22\mmm}° bouillant. ajoutez de la vanille et laissez en contact pendant
deux heures à deux heures et demie.

Dressez les pêches, réduisez le sirop et, lorsqu'il est froid, masquez-en les fruits.

\sk

\index{Compote de brugnons}
On peut préparer de méme la compote de brugnons.

\section*{\centering Compote de mûres.}
\phantomsection
\addcontentsline{toc}{section}{ Compote de mûres.}
\index{Compote de mûres}

Choisissez de belles grosses mûres noires, plongez-les dans du sirop
à {\ppp17\mmm}° bouillant, laissez donner un bouillon ou deux, puis égouttez
les fruits et mettez-les dans un compotier.

Remettez le sirop sur le feu, ajoutez-y du bon vin blanc, sauternes ou graves,
du jus de mûres fraîches, du sucre en quantité suffisante et faites cuire de
façon à amener le mélange à peser {\ppp30\mmm}° ; ou parfumez-le avec du
marasquin. Laissez-le refroidir, puis versez-le sur les mûres.

\section*{\centering Compote d'ananas.}
\phantomsection
\addcontentsline{toc}{section}{ Compote d'ananas.}
\index{Compote d'ananas}

Prenez un bon ananas, pelez-le à vif, coupez-le en tranches de {\ppp1\mmm}
centimètre et demi à {\ppp2\mmm} centimètres d'épaisseur, enlevez délicatement
les parties dures du cœur.

Mettez les tranches d'ananas dans un vase, saupoudrez-les de sucre.

Faites cuire les déchets pendant un instant dans du sirop à {\ppp18\mmm}°,
parfumé à la vanille ou au kirsch ; laissez refroidir. Passez le sirop ;
remettez-le sur le feu et, lorsqu'il est bouillant, plongez dedans les tranches
d'ananas. Éloignez le tout du feu et laissez en contact pendant un quart
d'heure.

Dressez les tranches d’ananas en couronne dans un compotier, en les faisant
chevaucher ; masquez-les avec le sirop concentré et refroidi.

\sk

Si l'ananas est de conserve, on le sert simplement avec du sirop à {\ppp30\mmm}°.

\section*{\centering Compote de bananes.}
\phantomsection
\addcontentsline{toc}{section}{ Compote de bananes.}
\index{Compote de bananes}

Prenez de belles bananes, mûres à point, pelez-les, passez-les dans du jus
d'orange.

Préparez un sirop à 28° parfumé avec du rhum ; plongez les bananes
dans le sirop bouillant ; éloignez la casserole du feu et laissez en contact
pendant {\ppp10\mmm} à {\ppp15\mmm} minutes.

Dressez les bananes sur une coupe et masquez-les avec le sirop concentré
à {\ppp32\mmm}° et refroidi.

\section*{\centering Compote de kakis.}
\phantomsection
\addcontentsline{toc}{section}{ Compote de kakis.}
\index{Compote de kakis}

Épluchez des kakis, frottez-les de jus de citron, plongez-les dans du sirop
à {\ppp18\mmm}° bouillant parfumé au curaçao ; éloignez-les du feu et
laissez-les pocher dans le sirop pendant {\ppp20\mmm} à {\ppp30\mmm} minutes.

Dressez les kakis dans un compotier et masquez-les avec le sirop réduit
à {\ppp28\mmm}° et refroidir.

\section*{\centering Compotes composées.}
\phantomsection
\addcontentsline{toc}{section}{ Compotes composées.}
\index{Compotes composées}

On peut préparer des compotes de fruits de plusieurs espèces, soit en les
réunissant après les avoir cuits séparément, soit en les dressant sur des
marmelades ou des purées de fruits ; ou encore en les masquant ou en les
décorant avec des gelées de fruits.

Ces sortes de macédoines ne doivent comporter que peu d'espèces réunies.

\bigskip

\begin{center}
\textit{CONFITURES}
\end{center}

\bigskip

\addcontentsline{toc}{section}{ Confitures.}
\index{Confitures}
\index{Confitures (Définition des)}
\index{Définition des confitures}

Les confitures sont des préparations de fruits mûrs, entiers ou coupés, presque
toujours blanchis dans du sirop et dont la cuisson s'effectue dans un sirop
concentré, transparent.

La quantité de sucre employée est en rapport avec la nature des fruits : elle
est toujours plus grande avec des fruits acides qu'avec des fruits sucrés.
Lorsqu'on met trop de sucre, on atténue l’arome du fruit : si l'on en met trop
peu, il faut faire cuire la confiture plus longtemps et le parfum du fruit
s'évapore, ou bien, si la cuisson n'atteint pas le degré qui convient, la
fermentation se produit bientôt.

\medskip

Le temps de cuisson des confitures ne peut être fixé d'une façon précise. En
général, plus une confiture est cuite vivement, meilleure elle est, le fruit
gardant mieux son goût et sa couleur. Cependant, la cuisson des confitures dans
lesquelles les fruits subsistent ne doit pas être menée à trop grand feu, car
elles seraient susceptibles de brûler ; au contraire, la cuisson des gelées, où
seul le jus des fruits existe, doit être poussée à feu vif pour arriver aussi
vite que possible au degré voulu.

Le degré de cuisson est le même pour toutes les confitures ; c'est la nappe. On
reconnaît pratiquement qu'il est atteint lorsque la vapeur qui s'échappe de la
confiture devient moins dense et le bouillonnement plus serré : l’évaporation
de l'eau des fruits est alors terminée et la cuisson réelle et définitive, très
rapide, est bientôt faite. Aussi doit-on surveiller attentivement et élever
fréquemment l'écumoire hors de la bassine ; on voit la confiture s'en détacher
d'abord très vite ; puis, après un instant, se condenser sur le milieu du
tranchant de l'écumoire et tomber ensuite par larges gouttes plates. A ce
moment, les confitures doivent être retirées du feu. On attend pendant une
dizaine de minutes que la grande chaleur soit atténuée, puis on les verse en
plusieurs fois dans des pots stérilisés et chauffés.

\medskip

On laisse refroidir les confitures à l'abri de la poussière dans un endroit sec
el aéré. Le lendemain, on pose sur chaque pot, à même la confiture, un rond de
papier blanc ou de papier parcheminé imbibé d'alcool ou de cognac légèrement
sucré, ou de glycérine : on couvre avec un double papier fort et on ficelle.

Les pots de confitures doivent être conservés à température peu élevée, au sec
et à l'abri du soleil. Il est bon d'étiqueter chaque pot du nom du fruit qu'il
renferme et d'indiquer la date de sa préparation.

\medskip

Il est préférable de préparer les confitures par petites quantités : elles
cuisent plus vite et conservent mieux leur parfum.

\section*{\centering Confiture d'abricots.}
\phantomsection
\addcontentsline{toc}{section}{ Confiture d'abricots.}
\index{Confiture d'abricots}

Prenez des abricots de plein vent, bien parfumés, sucrés et mûrs à point ;
essuyez-les avec un linge fin ; enlevez les noyaux sans abîmer les fruits ;
laissez les abricots entiers ou partagez-les en deux. Passez au tamis ceux qui
seraient défectueux, filtrez le jus au travers d'un linge. Cassez les noyaux.
ébouillantez les amandes afin de pouvoir facilement les monder, séparez-les en
deux : trempez-les dans un peu de kirsch.

Prenez, par kilogramme de fruits et de jus prêts à être employés, {\ppp750\mmm}
grammes de sucre en morceaux, {\ppp150\mmm} grammes d'eau et le jus d'un citron
moyen. Mettez dans une bassine en cuivre non étamée le sucre et l'eau, faites
cuire au grand boulé, écumez soigneusement, puis plongez dedans une partie des
abricots, juste ce qu'il faut pour couvrir la surface du sirop, retournez-les
sans les briser afin qu'ils cuisent de tous côtés ; lorsqu'ils sont devenus
translucides, enlevez-les avec une écumoire en cuivre. Renouvelez l'opération
jusqu'à ce que tous les fruits soient cuits. Ajoutez alors au sirop le jus
d'abricots et le jus de citron, écumez s'il est nécessaire, cuisez-le à la
nappe, mettez les abricots et les amandes, donnez un bouillon. Enlevez du feu,
laissez refroidir un peu, puis emplissez les pots en répartissant également les
abricots dans le sirop.

\section*{\centering Confiture de mirabelles.}
\phantomsection
\addcontentsline{toc}{section}{ Confiture de mirabelles.}
\index{Confiture de mirabelles}

Prenez des mirabelles bien saines et mûres à point, piquez-les profondément en
plusieurs endroits, enlevez les queues et les noyaux en laissant les fruits
entiers ; recueillez le jus qui en découle. Pesez fruits et jus.

Mettez dans une bassine poids égal de sucre en morceaux avec {\ppp200\mmm}
grammes d'eau par kilogramme de sucre ; faites cuire au petit boulé, écumez ;
plongez les prunes dans le sirop par petites quantités afin qu'elles cuisent
rapidement ; enlevez-les avec l'écumoire dès qu'elles sont translucides.
Incorporez au sirop le jus rendu par les mirabelles et du jus d'orange (une
orange pour trois livres de fruits), cuisez à la nappe, puis mettez les
mirabelles ; laissez frémir pendant quelques instants ; retirez du feu.

Emplissez des pots, en plusieurs fois, avec la confiture un peu refroidie, en
répartissant bien les fruits,

\section*{\centering Confiture de prunes Reine-Claude.}
\phantomsection
\addcontentsline{toc}{section}{ Confiture de prunes Reine-Claude.}
\index{Confiture de prunes Reine-Claude}

Prenez des prunes saines et mûres à point, piquez-les en plusieurs endroits ;
enlevez les noyaux et les queues sans abîmer les fruits. Passez au tamis une
petite quantité de prunes très mûres, filtrez le jus au travers d'un linge.

Pesez fruits et jus. Cuisez au petit boulé même poids de sucre avec
{\ppp200\mmm} grammes d'eau par kilogramme de sucre, écumez ; plongez dedans
les prunes, donnez un ou deux bouillons ; retirez les fruits, faites-les
égoutter sur un tamis.

Recuisez le sirop au boulé ; jetez dedans les prunes égouttées, par petites
quantités, pour qu'elles cuisent rapidement. Dès qu'elles sont devenues
translucides, enlevez-les. Remettez le sirop sur le feu, incorporez-y le jus de
prunes, amenez-le à la nappe. Ajoutez les prunes, donnez un bouillon ; enlevez
du feu. Emplissez convenablement des pots avec la confiture légèrement
refroidie.

\section*{\centering Confiture de fraises\footnote{Pour les confitures de
fruits rouges, il est préférable de se servir de bassines émaillées qui ne
communiquent aux fruits auçun goût d'âcreté, comme cela arrive souvent avec les
bassinés de cuivre non étamées.}.}
\phantomsection
\addcontentsline{toc}{section}{ Confiture de fraises.}
\index{Confiture de fraises}

Choisissez de belles fraises (Héricart ou quatre saisons), saines et mûres à
point ; ne les lavez que si c'est absolument nécessaire ; épluchez-les.

Prenez par kilogramme de fraises {\ppp800\mmm} grammes de sucre et
{\ppp200\mmm} grammes d'eau que vous ferez cuire au grand boulé ; écumez
soigneusement. Jetez les fraises dans le sirop ; éloignez un peu la bassine du
feu ; laissez pocher les fraises dans le sirop non bouillant pendant
{\ppp8\mmm} à {\ppp10\mmm} minutes. Égouttez-les sur un tamis.

Remettez le sirop sur feu vif ; dès qu'il atteint {\ppp32\mmm}° au pèse-sirop,
plongez dedans les fraises et cuisez à la nappe, ce qui demande {\ppp5\mmm}
à {\ppp6\mmm} minutes. Retirez la confiture du feu et versez-la dans des pots
en plusieurs fois pour bien répartir les fraises et éviter qu'elles montent
à la surface.

\sk

Prenez des fraises Héricart ou quatre saisons, ou encore des fraises ananas pas
trop grosses, épluchez-les, pesez-les. Disposez-les dans une terrine, par
couches, avec même poids de sucre en poudre interposé ; laissez en contact
jusqu'à ce que le sucre soit complètement fondu. Versez le tout dans une
bassine, chauffez, donnez deux bouillons, puis enlevez la confiture du feu et
mettez-la à refroidir dans une terrine. Égouttez ensuite les fraises.

Préparez une gelée de pommes en faisant cuire du jus de pommes filtré avec
poids égal de sucre, à raison de {\ppp1\mmm} kilogramme du mélange par
kilogramme de fraises, ajoutez le jus rendu par les fraises, écumez avec soin
dès le début de la cuisson et amenez le sirop à la nappe. Remettez alors les
fraises, laissez-les cuire pendant {\ppp5\mmm} à {\ppp6\mmm} minutes ; enlevez
la bassine du feu, laissez un peu refroidir et coulez la confiture dans des
pots comme il est dit plus haut.

\medskip

On peut, au goût, faire entrer un peu de jus de citron dans la cuisson de ces
confitures,

\section*{\centering Confiture de cerises.}
\phantomsection
\addcontentsline{toc}{section}{ Confiture de cerises.}
\index{Confiture de cerises}

\index{Confiture de cerises au naturel}
\textit{Au naturel}. — Prenez des cerises (royales) bien en chair, sucrées et
mûres à point, enlevez-en les noyaux et les queues sans abîmer les fruits.
Pesez les cerises et le jus qu'elles ont rendu ; mettez dans une bassine même
poids de sucre, mouillez avec un peu d'eau et cuisez au petit cassé ; écumez
bien, puis plongez les cerises dans le sirop ; éloignez un peu la bassine du
feu et continuez la cuisson jusqu'à ce que le sirop marque {\ppp32\mmm}°,
c'est-à-dire lorsqu'il est à la nappe ; écumez encore pendant cette dernière
opération.

Enlevez la confiture du feu. laissez-la un peu refroidir, puis versez-la dans
des pots, en plusieurs fois, pour éviter que les cerises surnagent.

Avec des cerises moins sucrées, on devra recommencer la cuisson au moins une
fois.

\sk

\index{Confiture de cerises au marasquin}
\textit{Au marasquin}. — Prenez de belles cerises (royales), supprimez les
noyaux et les queues ; laissez les cerises entières ; pesez-les avec le jus
quelles ont rendu.

Faites cuire au petit cassé même poids de sucre mouillé avec un peu d'eau,
écumez ; jetez dedans les cerises, donnez un ou deux bouillons ; versez le tout
dans une terrine ; laissez refroidir.

Égouttez les cerises, remettez le sirop sur le feu, cuisez-le au boulé, ajoutez
les cerises et du marasquin, au goût, écumez encore. Achevez la cuisson à la
nappe. Laissez légèrement refroidir et mettez en pots comme ci-dessus.

\sk

\index{Confiture de cerises au jus de groseilles}
\textit{Au jus de groseilles}. — Prenez des cerises mûres (griottes ou
Montmorency, de préférence) ; enlevez noyaux et queues en laissant les cerises
entières ; pesez les fruits, pesez même poids de sucre.

Mettez le sucre dans une bassine, avec un peu d'eau pour l'humecter, faites
cuire au boulé en écumant bien, puis ajoutez les cerises ; donnez deux
bouillons ; versez dans une terrine ; laissez refroidir.

Quatre heures après, égouttez les cerises, remettez le sirop sur le feu,
additionnez-le de {\ppp300\mmm} grammes de jus de groseilles et même poids de
sucre par kilogramme de fruits. Cuisez au boulé, écumez encore, ajoutez les
cerises. Achevez la cuisson de l’ensemble à la nappe. Mettez en pots comme il
est dit plus haut.

\sk

\index{Confiture de cerises au jus de framboises}
\textit{Au jus de framboises}. — Prenez des cerises (griottes ou Montmorency),
supprimez noyaux et queues ; laissez les fruits entiers. Pesez-les et prenez
même poids de sucre. Cuisez le sucre au petit cassé, écumez-le ; jetez dedans
les cerises, donnez deux ou trois bouillons, puis versez le tout dans une
terrine ; laissez refroidir pendant quatre heures. Égouttez alors les cerises
sur un tamis.

Mettez une partie du sirop dans une bassine, cuisez-le au petit cassé, ajoutez
les cerises. donnez deux ou trois bouillons, laissez refroidir. Égouttez les
cerises.

Remettez tout le sirop de cerises dans la bassine ; ajoutez‑y {\ppp250\mmm}
grammes de jus de framboises mélangé à {\ppp200\mmm} grammes de sucre, cuisez
à la nappe, écumez, ajoutez de nouveau les cerises, donnez quelques bouillons.
Enlevez du feu. Mettez en pots comme dans la première formule.

\section*{\centering Confiture de groseilles de Bar.}
\phantomsection
\addcontentsline{toc}{section}{ Confiture de groseilles de Bar.}
\index{Confiture de groseilles de Bar}

Prenez de très belles groseilles blanches peu mûres, enlevez-en les pépins du
côté de la queue avec la pointe d'une plume d’oie.

Égrappez des groseilles blanches mûres à point, écrasez-les, passez le jus au
travers d'un linge.

Pesez {\ppp1\mmm} {\ppp500\mmm} grammes de sucre par kilogramme de groseilles
épépinées et de jus, mouillez-le avec un peu d'eau et faites-le cuire au grand
boulé ; écumez avec soin, pendant la cuisson ; ajoutez alors doucement les
groseilles épépinées ; éloignez un peu la bassine du feu et laissez pocher les
groseilles, à liquide frémissant, pendant {\ppp5\mmm} minutes. Versez le tout
dans une terrine, laissez refroidir. Reprenez le sirop, additionnez-les de
{\ppp250\mmm} grammes de jus de groseilles par litre de sirop ; faites cuire au
boulé, puis ajoutez les groseilles ; achevez vivement la cuisson à la nappe.

Mettez la confiture en petits pots tronconiques en répartissant bien les
groseilles dans le sirop et en évitant qu'elles surnagent.

\section*{\centering Confiture de groseilles et de framboises.}
\phantomsection
\addcontentsline{toc}{section}{ Confiture de groseilles et de framboises.}
\index{Confiture de groseilles et de framboises}

Prenez des groseilles rouges ou blanches et des framboises de même couleur,
à raison de {\ppp250\mmm} grammes de framboises pour {\ppp1\mmm} kilogramme de
groseilles.

Égrappez les groseilles, épluchez les framboises ; écrasez un cinquième des
framboises ; filtrez le jus. Pesez même poids de sucre, mouillez-le d'un peu
d'eau, faites-le cuire au grand soufflé, écumez, jetez dedans les groseilles et
les framboises, donnez un bouillon ; laissez refroidir dans une terrine.
Égouttez les fruits.

Prenez le sirop, ajoutez-y le jus de framboises, cuisez au grand soufflé,
écumez, remettez les fruits ; achevez rapidement la cuisson à la nappe.

Mettez en pots lorsque la confiture est un peu refroidie.

\section*{\centering Confiture de raisin.}
\phantomsection
\addcontentsline{toc}{section}{ Confiture de raisin.}
\index{Confiture de raisin}

Prenez des raisins blancs, gros, fermes et mûrs à point ; épépinez les plus
beaux grains, à l’aide d'une plume d'oie taillée en pointe, sans les abîmer ;
écrasez et pressez les autres avec le double de raisin muscat blanc : passez le
jus au travers d'un linge.

Pesez grains de raisin et jus ; faites cuire même poids de sucre avec un peu
d'eau, au grand boulé, mettez doucement les grains de raisin dans le sirop,
donnez deux ou trois bouillons : éloignez la bassine du feu et laissez pocher
les fruits pendant {\ppp8\mmm} à {\ppp10\mmm} minutes, à liquide frissonnant.
Versez le tout dans une terrine.

Cinq heures après, égouttez les fruits, cuisez de nouveau le sucre au boulé,
mettez dedans les raisins, donnez deux ou trois bouillons, laissez pocher, puis
faites refroidir en terrine comme précédemment. Au bout de cinq heures, égouttez
de nouveau les raisins, remettez le sirop dans la bassine avec le jus des raisins,
cuisez à la nappe, ajoutez les fruits entiers, donnez quelques bouillons, enlevez
du feu. Lorsque la confiture est un peu refroidie, mettez-la en pots en prenant les
précautions nécessaires pour que les grains ne surnagent pas.

\section*{\centering Confiture d’oranges.}
\phantomsection
\addcontentsline{toc}{section}{ Confiture d’oranges.}
\index{Confiture d’oranges}

Prenez de belles oranges bien mûres, lourdes et fraîchement cueillies, à écorce
un peu épaisse et souple. Piquez l'écorce assez profondément avec un petit
bâton pointu ou un poinçon en os ; puis plongez les oranges dans de l'eau
bouillante dans laquelle vous les laisserez cuire pendant une demi-heure.
Mettez-les ensuite à rafraîchir, dans de l'eau froide souvent renouvelée,
pendant {\ppp24\mmm} heures, afin d'en enlever l’amertume.

Égouttez les oranges, séchez-les dans un linge, coupez-les en quartiers ou en
rouelles, recueillez le jus qui en découle, enlevez les pépins. Pesez fruits et
jus, faites, avec poids égal de sucre et un peu d'eau, un sirop au grand boulé,
écumez, mettez dedans les tranches d'oranges, laissez cuire pendant
{\ppp15\mmm} à {\ppp25\mmm} minutes, suivant leur épaisseur ; les fruits doivent
être translucides. Disposez les tranches d'oranges dans des pots. Ajoutez au
sirop le jus rendu par les oranges et plus ou moins de bon jus de pommes
reinettes avec même poids de sucre ; cuisez à la nappe. Finissez d'emplir les
pots avec le sirop en soulevant les tranches d’oranges pour les répartir comme
il faut.

\section*{\centering Confiture d’ananas.}
\phantomsection
\addcontentsline{toc}{section}{ Confiture d’ananas.}
\index{Confiture d’ananas}

Prenez de beaux ananas mûrs à point, pelez-les à vif. Coupez-les en deux dans
le sens de la longueur, enlevez les parties dures du centre, puis détaillez en
tranches chaque demi-ananas. Pesez-les, faites avec un même poids de sucre un
sirop à {\ppp26\mmm}° ; lorsqu'il est un peu refroidi, versez-le sur les
tranches d'ananas placées dans une terrine, laissez en contact pendant trois
heures.

Passez au tamis les parures des ananas, filtrez le jus au travers d'un linge,
étendez-le avec du jus d'orange ({\ppp2\mmm} oranges pour {\ppp1\mmm} ananas),
ajoutez-y même poids de sucre.

Égouttez les tranches d'ananas ; faites cuire le sirop à la plume, jetez dedans
les tranches de fruits, donnez un bouillon ; versez dans une terrine, laissez
refroidir. Reprenez le sirop, ajoutez‑y le jus d'ananas et d'oranges, du bon
jus de pommes reinettes, cuisez à la nappe, mettez ensuite les tranches
d'ananas et achevez la cuisson de l’ensemble. Les tranches d'ananas doivent
être translucides. Lorsque la confiture est un peu refroidie, coulez-la en
pots.

\medskip

On pourra. suivant le goût, parfumer très légèrement cette confiture avec du
kirsch ou de la vanille.

\section*{\centering Confiture de pastèques.}
\phantomsection
\addcontentsline{toc}{section}{ Confiture de pastèques.}
\index{Confiture de pastèques}

Coupez en tranches minces des pastèques blanches ou melons d'eau, à chair
ferme ; pelez-les, enlevez les graines. Blanchissez les tranches de fruits dans
de l'eau bouillante légèrement salée jusqu'à ce qu'elles plient sous le doigt.
Séchez-les dans un linge.

Cuisez au grand perlé même poids de sucre avec {\ppp350\mmm} grammes d’eau par
kilogramme de sucre, mettez dedans les tranches de pastèques, ajoutez du jus de
citron et le zeste ébouillanté ({\ppp1\mmm} citron pour {\ppp3\mmm} livres de
fruits) ; laissez cuire à demi et faites refroidir.

Le lendemain, égouttez les pastèques sur un tamis, remettez le sirop sur le
feu, amenez-le à la nappe et finissez dedans la cuisson des tranches de melon.

Laissez refroidir à moitié ; emplissez des pots.

\section*{\centering Confiture de cassis.}
\phantomsection
\addcontentsline{toc}{section}{ Confiture de cassis.}
\index{Confiture de cassis}

Prenez du beau cassis mûr, mais un peu ferme et non mouillé ; égrappez-le,
enlevez-en les pépins comme il est indiqué pour les confitures de groseilles de
Bar, ou laissez-les, au choix.

Pesez les fruits. Prenez même poids de sucre ; mettez-le dans une bassine avec
{\ppp300\mmm} grammes d'eau par kilogramme de sucre ; cuisez-le à la plume ;
écumez soigneusement. Jetez dedans les grains de cassis, donnez quelques
bouillons, puis égouttez les fruits sur un tamis. Continuez la cuisson du sirop
à la plume, additionnez-le de {\ppp300\mmm} grammes de jus de framboises mêlé
à {\ppp300\mmm} grammes de sucre par kilogramme de cassis, écumez encore,
remettez les grains de cassis et achevez vivement la cuisson à la nappe.

Lorsque la confiture est un peu refroidie, versez-la dans des pots.

\section*{\centering Confiture de dattes.}
\phantomsection
\addcontentsline{toc}{section}{ Confiture de dattes.}
\index{Confiture de dattes}

Prenez de grosses dattes d'un beau jaune, pelez-les et faites-les blanchir dans
juste ce qu'il faut d'eau pour les couvrir. Quand elles sont ramollies, fendez
chaque datte sur le côté et enlevez le noyau que vous remplacerez par une
amande ou une pistache mondée. Pesez les fruits ainsi apprêtés. Prenez
{\ppp750\mmm} grammes de sucre et {\ppp250\mmm} grammes d'eau par kilogramme de
fruits ; cuisez-le au boulé, mettez dedans les dattes et laissez cuire jusqu'à
ce que le liquide soit à la nappe.

Mettez en pots chauffés.

\section*{\centering Confiture de quatre fruits.}
\phantomsection
\addcontentsline{toc}{section}{ Confiture de quatre fruits.}
\index{Confiture de quatre fruits}

Prenez poids égaux de cerises, fraises des quatre saisons, framboises et groseilles
rouges. Enlevez les noyaux des cerises, épluchez les fraises et les framboises,
égrappez les groseilles.

Écrasez groseilles et framboises, passez-les au tamis ; filtrez le jus à la
serviette. Pesez fruits et jus. Prenez même poids de sucre ; faites-le cuire
avec un peu d'eau, au petit cassé, dans une bassine émaillée, écumez. Plongez
dedans les cerises ; éloignez un peu la bassine du feu, pour permettre au sucre
de se dissoudre dans le jus des cerises et continuez la cuisson jusqu'à la
nappe. Enlevez les cerises avec l'écumoire ; faites-les égoutter sur un tamis.

Continuez à faire cuire le sirop ; lorsqu'il est au petit boulé, mettez dedans
les fraises, laissez-les cuire pendant une dizaine de minutes ; retirez-les,
égouttez-les.

Réunissez dans la bassine jus de groseilles et de framboises, sirop de cerises
et de fraises ; amenez le mélange à la nappe ; remettez dedans cerises et
fraises, donnez quelques bouillons, puis mettez en pots lorsque la confiture
sera suffisamment refroidie.

\section*{\centering Confiture de fruits mélangés.}
\phantomsection
\addcontentsline{toc}{section}{ Confiture de fruits mélangés.}
\index{Confiture de fruits mélangés}

Prenez, au choix, abricots, ananas, cerises, fraises, framboises, groseilles,
pêches, poires, pommes, prunes, raisin, etc. en assortissant le mieux possible
nuances et espèces.

Faites cuire chaque sorte de fruits séparément dans du sirop au petit cassé,
les petits entiers, les gros coupés en cubes ou tournés à la cuiller.
Égouttez-les ensuite sur un tamis.

Mondez des amandes douces et des pistaches, séparez-les en deux.

Épluchez les amandes des abricots, blanchissez-les dans du kirsch.

Préparez, au choix, une gelée de framboises, de grenades, de groseilles ou de
pommes ; cuisez-la à la nappe, mettez dedans les fruits, les amandes douces et
les pistaches, les amandes d'abricots, donnez un bouillon ; retirez la bassine
du feu et laissez en contact jusqu'à refroidissement presque complet.

Emplissez des pots avec cette macédoine ; ne les couvrez que lorsque la
confiture est bien refroidie.

\bigskip

\begin{center}
\textit{MARMELADES}
\end{center}

\bigskip

\addcontentsline{toc}{section}{ Marmelades}
\index{Marmelades}
\index{Définition des marmelades}

Les marmelades sont des préparations de fruits ou de pulpe de fruits cuits avec
du sucre ou du sirop et réduits plus ou moins en purée.

\section*{\centering Marmelade de fraises.}
\phantomsection
\addcontentsline{toc}{section}{ Marmelade de fraises.}
\index{Marmelade de fraises}

Épluchez des fraises sucrées, saines et mûres ; pesez-les.

Prenez {\ppp750\mmm} grammes de sucre par kilogramme de fruits, mettez-le dans
une bassine émaillée, humectez-le d'un peu d’eau et faites-le cuire au grand
perlé ; écumez ; ajoutez alors les fraises et cuisez à la nappe sur feu vif en
remuant constamment avec l'écumoire et sans quitter un instant la préparation.

Laissez refroidir à moitié et mettez en pots.

\sk

Passez des fraises au tamis ; pesez la purée.

Prenez {\ppp800\mmm} grammes de sucre par kilogramme de purée, faites-le cuire
avec un peu d'eau au grand perlé ; écumez ; ajoutez la purée de fraises et
cuisez à la nappe.

\section*{\centering Marmelade de framboises.}
\phantomsection
\addcontentsline{toc}{section}{ Marmelade de framboises.}
\index{Marmelade de framboises}

Prenez des framboises saines et mûres ; épluchez-les ; pesez-les. Pesez même poids
de sucre. Laissez les fruits entiers ou passez-les au tamis, au choix,

Mettez le sucre, les fruits ou la purée dans une bassine émaillée ; faites cuire
à la nappe en remuant constamment avec l'écumoire pour éviter que les fruits
s’attachent au fond de la bassine ; écumez pendant la cuisson.

Laissez refroidir à moitié et mettez en pots ou en bocaux.

\section*{\centering Marmelade de cerises.}
\phantomsection
\addcontentsline{toc}{section}{ Marmelade de cerises.}
\index{Marmelade de cerises}

Prenez des cerises (griottes, anglaises ou Montmorency) ; enlevez queues et
noyaux ; pesez les fruits ; prenez même poids de sucre.

Disposez les cerises dans une terrine, par couches alternées avec le sucre ;
laissez en contact.

Le lendemain, mettez le contenu de la terrine dans une bassine émaillée,
ajoutez un peu de vanille et faites cuire pendant {\ppp30\mmm} à {\ppp40\mmm}
minutes jusqu'au lissé\footnote{Appelé aussi « filé ».}, en remuant
constamment. A ce moment, un peu de sirop refroidi, pris entre le pouce et
l'index, doit s'étendre comme un fil lorsqu'on écarte les doigts.

Enlevez alors la bassine du feu, laissez refroidir un peu, puis versez dans des
pots.

\sk

Mettez {\ppp750\mmm} grammes de sucre, par kilogramme de fruits, dans une
bassine émaillée avec un peu d'eau ; faites bouillir légèrement, ajoutez les
cerises et cuisez à la nappe en remuant et en écumant pendant une dizaine de
minutes. Laissez refroidir.

Le lendemain, faites bouillir de nouveau la marmelade pendant {\ppp5\mmm}
minutes, en la remuant ; enlevez les fruits avec une écumoire et mettez-les
dans une terrine. Continuez la cuisson du jus jusqu'à ce qu'il arrive à la glu.
Remettez alors les cerises, ajoutez au tout même volume de gelée de groseilles
cuite à {\ppp32\mmm}°, donnez quelques bouillons.

Enlevez du feu, laissez refroidir à moitié, puis mettez en pots.

\section*{\centering Marmelade d’abricots.}
\phantomsection
\addcontentsline{toc}{section}{ Marmelade d’abricots.}
\index{Marmelade d’abricots}

Prenez des abricots mûrs, frais et parfumés, enlevez les noyaux, extrayez les
amandes, blanchissez-les et mondez-les.

Coupez les fruits en morceaux ; mettez-les dans une bassine avec un peu d'eau,
chauffez et faites-les fondre en les remuant avec une spatule ; passez-les
ensuite au tamis.

Pesez la purée ; prenez de {\ppp750\mmm} à {\ppp1\mmm} {\ppp000\mmm} grammes de
sucre par kilogramme de purée suivant qu'elle est plus ou moins sucrée ;
cuisez-le au petit cassé en écumant bien ; ajoutez la purée d’abricots ;
éloignez un peu la bassine du feu pour permettre au sucre de se dissoudre dans
le jus, tournez sans arrêt, puis faites cuire à la nappe en remuant. Quelques
minutes avant la fin, mettez les amandes blanchies.

Versez aussitôt la marmelade dans des vases préalablement chauffés. Laissez
refroidir avant de couvrir les pots.

\sk

Apprêtez des abricots comme il est dit ci-dessus.

Pesez les fruits ; placez-les dans une bassine par couches alternées avec
{\ppp750\mmm} à {\ppp1 000\mmm} grammes de sucre par kilogramme de fruits,
suivant leur degré d'acidité ; mouillez avec quelques cuillerées d'eau. Quand
le sucre est dissous, mettez la bassine sur le feu, chauffez en tournant et
cuisez la marmelade à la grande nappe\footnote{\index{Cuisson des sirops de fruits à la grande nappe}
                                               \index{Cuisson des sirops de fruits à la goutte}
                                               \index{Cuisson des sirops de fruits à la nappe}
Lorsque le sirop est arrivé à la nappe, si on le laisse donner quelques
bouillors de plus, au lieu de s'étaler en larges gouttes plates, le sirop tombe
en gouttes rondes, gardant cette forme. C'est ce qu'on appelle la grande nappe
ou goutte.} ou au lissé, en la remuant constamment sans la quitter. Ajoutez les
amandes des abricots quelques minutes avant la fin.

Mettez en pots comme ci-dessus.

\section*{\centering Marmelade de prunes Reine-Claude.}
\phantomsection
\addcontentsline{toc}{section}{ Marmelade de prunes Reine-Claude.}
\index{Marmelade de prunes Reine-Claude}

Choisissez des prunes bien mûres, enlevez noyaux et queues. Coupez les fruits
en morceaux ; pesez-les. Prenez {\ppp750\mmm} grammes de sucre par kilogramme
de fruits, faites-le cuire avec un peu d'eau au petit cassé, ajoutez les prunes
et le jus qu'elles ont rendu et cuisez à la nappe.

Versez dans des pots chauffés au préalable ; laissez refroidir.

\sk

Mettez sucre et prunes dans une bassine, mouillez avec un peu d'eau et faites
cuire, sur feu vif, à la nappe, ce qui demande une demi-heure environ. Remuez
constamment la marmelade sans la quitter pendant la cuisson.

Versez-la aussitôt dans des pots chauffés.

\sk

Faites fondre les prunes avec un peu d'eau et de sucre, sur le feu, en les
remuant avec une spatule ; passez-les au tamis. Pesez la purée. Prenez
{\ppp750\mmm} grammes de sucre par kilogramme de purée ; cuisez-le au boulé ;
ajoutez la purée ; cuisez à la nappe.

Mettez en pots comme ci-dessus.

\section*{\centering Marmelade de mirabelles.}
\phantomsection
\addcontentsline{toc}{section}{ Marmelade de mirabelles.}
\index{Marmelade de mirabelles}

Prenez des fruits sains bien mûrs. Enlevez noyaux et queues ; divisez les
mirabelles en deux ; mettez-les avec quelques cuillerées d'eau dans une
bassine ; faites-les fondre à petit feu ; passez-les au tamis. Pesez la purée.

Prenez {\ppp750\mmm} grammes de sucre par kilogramme de purée, cuisez-le au boulé ;
écumez ; ajoutez la purée de mirabelles et continuez la cuisson jusqu'à la nappe.

\sk

Mettez dans une bassine des mirabelles coupées en deux et privées de leurs
noyaux, par couches alternées avec {\ppp750\mmm} grammes de sucre par
kilogramme de fruits ; laissez en contact en bassine couverte pendant
{\ppp8\mmm} à {\ppp10\mmm} heures, cuisez le tout à la nappe avec un peu de
vanille ou de jus de citron.

Coulez ces préparations en pots préalablement chauffés.

\section*{\centering Marmelade de pommes.}
\phantomsection
\addcontentsline{toc}{section}{ Marmelade de pommes.}
\index{Marmelade de pommes}

Prenez des reinettes ou des calvilles ; pelez-les, coupez-les, en lames assez
minces, enlevez les pépins. Mettez les tranches de pommes dans une bassine avec
un peu d'eau et du jus de citron et faites-les cuire doucement jusqu'à ce
qu'elles cèdent sous la pression des doigts. Passez au tamis. Pesez la purée.

Prenez {\ppp750\mmm} grammes de sucre, concassé fin, par kilogramme de purée.
Faites cuire ensemble sucre et purée de pommes à la nappe. en remuant
constamment, puis versez la marmelade dans des pots chauffés.

\sk

Préparez, comme ci-dessus, une purée de pommes ; pesez-la. Prenez {\ppp750\mmm}
grammes de sucre par kilogramme de purée ; cuisez-le avec un peu d'eau au petit
cassé ; écumez bien ; ajoutez la purée de pommes ; éloignez un peu la bassine
du feu, puis finissez la cuisson à la nappe.

La marmelade préparée ainsi est plus belle.

\medskip

On peut parfumer la marmelade de pommes avec de la vanille ou du citron, au
goût.

\section*{\centering Marmelade de coings.}
\phantomsection
\addcontentsline{toc}{section}{ Marmelade de coings.}
\index{Marmelade de coings}

Chauffez de l'eau à {\ppp40\mmm}° ; mettez dedans des coings entiers ;
laissez-les pocher pendant une demi-heure. Retirez-les et laissez-les
refroidir. Pelez-les ensuite, coupez-les en tranches régulières assez minces,
supprimez toutes les parties dures.

Faites cuire les tranches de coings doucement, en casserole fermée, dans de
l'eau légèrement acidulée avec du jus de citron, afin de les conserver blanches
et entières, mais en les tenant un peu fermes. Égouttez-les sur un tamis. Pesez
coings et jus de cuisson.

Prenez {\ppp800\mmm} grammes de sucre et {\ppp250\mmm} grammes de jus de pommes
par kilogramme de fruits et de jus. Faites cuire le sucre au petit cassé,
ajoutez coings, jus de cuisson et jus de pommes, et achevez de cuire à la
nappe.

Mettez en pots chauffés au préalable.

\sk

Faites cuire des coings coupés en tranches jusqu'à ce qu'ils cèdent sous les
doigts ; passez-les au tamis ; recueillez la purée ; pesez-la.

Prenez {\ppp850\mmm} grammes de sucre et {\ppp250\mmm} grammes de jus de pommes
par kilogramme de purée ; mettez le tout dans une bassine et cuisez à la nappe.

\medskip

On peut parfumer les marmelades de coings avec de la vanille, de la cannelle
ou du zeste de citron, au goût.

\section*{\centering Marmelade de poires.}
\phantomsection
\addcontentsline{toc}{section}{ Marmelade de poires.}
\index{Marmelade de poires}

Prenez des poires fondantes bien saines (beurrés ou doyennés de préférence).
Pelez-les, coupez-les en quartiers, enlevez les pépins et les parties dures.
Pesez les fruits ainsi apprêtés.

Prenez {\ppp650\mmm} à {\ppp750\mmm} grammes de sucre par kilogramme de fruits
prêts à être employés, suivant qu'ils sont plus ou moins sucrés ; mettez
tranches de poires et sucre dans une bassine, couvrez-la et laissez en contact
pendant quelques heures. Faites cuire ensuite à la nappe avec un peu de vanille
ou de cannelle.

\sk

Faites cuire les tranches de poires avec la quantité d'eau juste nécessaire,
pour que la pulpe s'écrase sous les doigts ; passez-les au tamis.

Prenez {\ppp700\mmm} grammes de sucre par kilogramme de purée, un peu de jus de
citron et de vanille. Faites cuire le sucre avec un peu d'eau au petit cassé,
écumez, ajoutez purée de poires, jus de citron et vanille ; cuisez à la nappe.

Mettez ces marmelades en pots chauffés.

\section*{\centering Marmelade de pêches.}
\phantomsection
\addcontentsline{toc}{section}{ Marmelade de pêches.}
\index{Marmelade de pêches}

Prenez des pêches mûres de plein vent (pavies ou persèques) ; pelez-les,
coupez-les en tranches et mettez-les dans une bassine par couches alternées
avec {\ppp750\mmm} ou {\ppp800\mmm} grammes de sucre par kilogramme de fruits
net. Lorsque le sucre est fondu, portez la bassine sur feu modéré et faites
cuire jusqu'à la grande nappe.

\sk

Si les pêches sont dures, faites-les bouillir pendant quelques minutes dans un
peu d'eau ; passez-les au tamis. Prenez {\ppp800\mmm} grammes de sucre par
kilogramme de purée, cuisez-le au petit cassé ; ajoutez la purée de pêches et
finissez à la grande nappe ou au lissé.

Versez la marmelade dans des pots chauffés.

\medskip

On peut parfumer légèrement ces marmelades avec du kirsch, de la liqueur de
noyaux ou de la vanille.

\sk

On prépare de même la marmelade de brugnons ou nectarines.

\section*{\centering Marmelade de figues.}
\phantomsection
\addcontentsline{toc}{section}{ Marmelade de figues.}
\index{Marmelade de figues}

Prenez des figues mûres et entières, encore un peu fermes, piquez-les avec une
aiguille à brider, blanchissez-les pendant quelques minutes dans un peu d'eau
bouillante. Égouttez-les sur un tamis. Faites cuire au grand perlé du sucre
à raison de {\ppp750\mmm} grammes de sucre par kilogramme de fruits : mettez
dedans les figues une à une, un peu de vanille ou du zeste de citron et achevez
la cuisson à la nappe.

\section*{\centering Marmelade d’oranges.}
\phantomsection
\addcontentsline{toc}{section}{ Marmelade d’oranges.}
\index{Marmelade d’oranges}

Pelez des oranges bien mûres ; émincez une partie du zeste ; faites-le blanchir
longuement afin d'en enlever l'amertume ; rafraîchissez-le, égouttez-le et
versez dessus du sirop au perlé, juste ce qu'il en faut pour le couvrir ;
laissez en contact.

Passez au tamis la pulpe des oranges ; pesez la purée obtenue. Prenez
{\ppp1\mmm} {\ppp200\mmm} grammes de sucre et {\ppp400\mmm} grammes de jus de
pommes (reinette ou calville) par kilogramme de purée. Cuisez le sucre au
boulé, écumez, ajoutez-y la purée d'oranges, le jus de pommes, le zeste
d'orange et son sirop, cuisez à la nappe en remuant constamment.

Mettez en pots lorsque la marmelade est un peu refroidie.

\section*{\centering Marmelade d’oranges à l'anglaise.}
\phantomsection
\addcontentsline{toc}{section}{ Marmelade d’oranges à l'anglaise.}
\index{Marmelade d’oranges à l'anglaise}

Prenez des oranges amères, pelez-les à vif, puis faites-les cuire dans un peu
d'eau pendant trois quarts d'heure environ ; passez-les au tamis.

Blanchissez les écorces pendant une heure et demie dans de l'eau bouillante ;
rafraîchissez-les ensuite dans de l'eau froide renouvelée, afin d'en enlever
l'amertume, puis plongez-les dans du sirop bouillant à {\ppp25\mmm}° ; laissez
cuire.

Prenez {\ppp700\mmm} grammes de sucre par kilogramme de purée d'oranges,
faites-le cuire à la nappe ; ajoutez purée d'oranges, écorces et leur sirop ;
continuez la cuisson jusqu'à ce qu'elle atteigne la grande nappe. Mélangez en
remuant pendant la cuisson.

Mettez en pots après refroidissement partiel.

\section*{\centering Marmelade de pastèques.}
\phantomsection
\addcontentsline{toc}{section}{ Marmelade de pastèques.}
\index{Marmelade de pastèques}

Prenez des fruits mûrs à point, c'est-à-dire ni trop aqueux, ni passés ;
pelez-les en laissant sur la pulpe une mince pellicule du blanc de l'écorce ;
coupez-les en tranches minces ; pesez-les.

Prenez {\ppp750\mmm} grammes de sucre en poudre par kilogramme de fruits ;
disposez par couches les tranches de pastèques dans une bassine en les
saupoudrant avec le sucre ; tenez dans un endroit frais. Au bout de {\ppp3\mmm}
à {\ppp4\mmm} heures de contact, faites cuire le tout à feu vif, en remuant,
jusqu'au degré de la nappe. Quelques minutes avant la fin, ajoutez, au goût,
plus ou moins de zeste de citron émincé.

Versez la préparation dans des vases chauffés et laissez refroidir.

\section*{\centering Marmelade de pistaches.}
\phantomsection
\addcontentsline{toc}{section}{ Marmelade de pistaches.}
\index{Marmelade de pistaches}

Prenez des pistaches recouvertes de leur coque, décortiquez-les, mondez-les,
puis écrasez-les légérement.

Pesez même poids de sucre et {\ppp400\mmm} grammes d'eau par kilogramme de
sucre ; faites-le cuire au perlé ; ajoutez-y les pistaches, un peu de jus de
citron et très peu de vanille. Donnez quatre à cinq bouillons, puis éloignez la
bassine du feu et mélangez en tournant jusqu'à ce que le produit devienne épais
au point qu'il soit difficile de le remuer.

Coulez en pots chauffés au préalable.

\sk

On prépare de même des marmelades de pin pignon ou pignons doux.

\bigskip

\begin{center}
\textit{GELÉES DE FRUITS}
\end{center}

\bigskip

\addcontentsline{toc}{section}{ Gelées de fruits}
\index{Gelées de fruits}
\index{Définition des gelées de fruits}
Les gelées de fruits sont des préparations de jus de fruits filtrés cuits
rapidement avec du sucre, à un degré donné.

\section*{\centering Gelée de groseilles.}
\phantomsection
\addcontentsline{toc}{section}{ Gelée de groseilles.}
\index{Gelée de groseilles}

Prenez des groseilles rouges et blanches, en parties égales ou deux tiers de
rouges et un tiers de blanches, mûres et cueillies par temps sec ; égrappez-les
avec une fourchette dans une terrine, puis mettez-les dans une bassine avec
{\ppp200\mmm} grammes d'eau par kilogramme de fruits ; chauffez en remuant au
fond avec une écumoire de cuivre rouge ; il est très important que le jus de
groseilles ne bouille pas. Dès que les groseilles sont fondues, c'est-à-dire au
bout d'un quart d'heure environ, versez-les sur un tamis placé au-dessus d'une
terrine, laissez égoutter en appuyant seulement mais sans presser pour que le
jus ne soit pas trouble ; filtrez-le à la chausse si vous voulez que la gelée
soit très limpide. Pesez le jus.

Prenez même poids de sucre en morceaux. Mettez dans une bassine sucre et jus de
groseilles ; quand le sucre est dissous, portez la bassine sur le feu et faites
cuire à feu vif jusqu'à la nappe, ce qui demande à peu près {\ppp6\mmm}
à {\ppp7\mmm} minutes, écumez soigneusement pendant la cuisson. Éloignez la
bassine du feu.

Versez la gelée dans des pots en grès ou dans des vases en verre préalablement
chauffés.

\sk

Prenez des groseilles rouges et des groseilles blanches dans les proportions
indiquées ci-dessus ; égrappez-les, écrasez les grains dans une terrine, puis
mettez grains et jus dans un torchon et exprimez-en le jus en tordant sans trop
presser ; filtrez-le ou ne le filtrez pas ; pesez-le.

Mettez poids égal de sucre cassé en morceaux dans une bassine avec le jus ; dès
qu'il est fondu. cuisez le tout à la nappe, écumez bien, puis versez dans des
pots chauffés.

\section*{\centering Gelée de groseilles framboisée.}
\phantomsection
\addcontentsline{toc}{section}{ Gelée de groseilles framboisée.}
\index{Gelée de groseilles framboisée}

Prenez des groseilles mûres ({\ppp2\mmm}/{\ppp3\mmm} de rouges,
{\ppp1\mmm}/{\ppp3\mmm} de blanches) et {\ppp200\mmm} grammes de framboises
rouges par kilogramme de groseilles. Écrasez dans une terrine groseilles et
framboises ; mettez le tout dans un torchon et tordez pour extraire le jus.
Pesez-le. Pesez même poids de sucre. Cuisez dans une bassine jus et sucre à feu
vif jusqu'à la nappe ; écumez bien pendant la cuisson.

Mettez en pots chauffés.

\section*{\centering Gelée de groseilles blanches.}
\phantomsection
\addcontentsline{toc}{section}{ Gelée de groseilles blanches.}
\index{Gelée de groseilles blanches}

Prenez des groseilles blanches bien mûres et {\ppp200\mmm} grammes de
framboises blanches par kilogramme de groseilles. Égrappez les groseilles,
épluchez les framboises ; pesez les fruits et poids égal de sucre.

Mettez le sucre dans une bassine avec {\ppp200\mmm} grammes d'eau par
kilogramme de sucre ; faites cuire au petit boulé en écumant soigneusement.
Jetez dedans groseilles et framboises ; éloignez la bassine sur le coin du
fourneau pendant une dizaine de minutes pour que le jus sorte des fruits, puis
cuisez à feu vif jusqu'à la nappe en écumant toujours.

Versez le tout sur un tamis placé au-dessus d'une terrine et laissez passer
le liquide. Mettez en pots aussitôt.

\medskip

Les gelées de groseilles sont particulièrement belles quand elles sont préparées
par petites quantités.

\section*{\centering Gelée de cerises.}
\phantomsection
\addcontentsline{toc}{section}{ Gelée de cerises.}
\index{Gelée de cerises}

Choisissez des cerises mûres, enlevez queues et noyaux. Écrasez les fruits et
passez le jus à la serviette ou filtrez-le. Pesez-le.

Mettez dans une bassine émaillée le jus de cerises, ajoutez‑y un tiers de bon
jus de pommes et {\ppp1\mmm} kilogramme de sucre par kilogramme de jus. Cuisez
à la nappe en écumant avec soin.

Laissez un peu refroidir et versez la gelée dans des pots.

\section*{\centering Gelée de framboises.}
\phantomsection
\addcontentsline{toc}{section}{ Gelée de framboises.}
\index{Gelée de framboises}

Prenez des framboises mûres à point et non mouillées ; exprimez-en le jus
au travers d'une serviette en tordant ; pesez-le.

Cuisez poids égal de sucre en morceaux avec un peu d'eau au grand soufflé,
ajoutez le jus de framboises et continuez la cuisson jusqu'à la nappe.

\sk

Mettez le jus de framboises avec le sucre dans une bassine émaillée ; laissez
fondre ; ajoutez {\ppp200\mmm} grammes de jus de pommes par kilogramme de jus
de framboises et faites cuire à la nappe en écumant bien.

Mettez la gelée de framboises dans des pots dès qu'elle est un peu refroidie.

\section*{\centering Gelée de cassis.}
\phantomsection
\addcontentsline{toc}{section}{ Gelée de cassis.}
\index{Gelée de cassis}

Choisissez du cassis bien mûr ; égrappez-le dans une bassine ; ajoutez‑y
{\ppp100\mmm} grammes d'eau par kilogramme de fruits ; chauffez et laissez
fondre sur feu doux sans bouillir. Lorsque les grains ont éclaté et rendu leur
jus, passez le tout au tamis. Ou bien, plus simplement, écrasez les fruits et
tordez-les dans une serviette.

Dans les deux cas, additionnez le jus de cassis de {\ppp200\mmm} grammes de jus
de groseilles blanches par kilogramme de jus de cassis. Pesez le mélange et
prenez autant de fois {\ppp900\mmm} grammes de sucre en morceaux qu'il y a de
kilogrammes de jus. Faites cuire le sucre au petit boulé avec un peu d'eau ;
écumez, ajoutez le jus des fruits ; tenez le tout un instant hors du feu pour
laisser dissoudre le sucre, puis cuisez à feu vif jusqu'à la nappe en écumant
pendant la cuisson.

Coulez en pots lorsque la gelée est un peu refroidie.

\sk

Gertaines personnes font entrer dans la préparation du jus de cerises ou du jus
de framboises ; c'est affaire de goût.

\section*{\centering Gelée de mûres.}
\phantomsection
\addcontentsline{toc}{section}{ Gelée de mûres.}
\index{ Gelée de mûres}

On prépare la gelée de mûres comme la gelée de groseilles ou la gelée de
framboises.

\section*{\centering Gelée d'abricots.}
\phantomsection
\addcontentsline{toc}{section}{ Gelée d'abricots.}
\index{Gelée d'abricots}

Choisissez des abricots bien mûrs, coupez-les en petits morceaux, retirez les
noyaux. Mettez les abricots coupés dans une bassine avec juste ce qu'il faut
d'eau pour les couvrir et faites-les fondre sur feu doux. Passez-les ensuite au
tamis sans presser et recueillez le jus. Pesez-le ; mélangez-y moitié de son
poids de bon jus de pommes, la moitié du jus d'un citron et {\ppp850\mmm}
grammes de sucre par kilogramme de jus. Faites cuire à la nappe en écumant.

Versez dans des pots chauffés au préalable.

\section*{\centering Gelée de pommes.}
\phantomsection
\addcontentsline{toc}{section}{ Gelée de pommes.}
\index{Gelée de pommes}

Prenez de bonnes pommes reinettes, pelez-les, enlevez les parties dures et les
pépins. Coupez-les en tranches et jetez-les au fur et à mesure dans de l'eau
additionnée de jus de citron pour les empêcher de noircir. Mettez-les ensuite
dans une bassine avec juste ce qu'il faut d'eau pour les couvrir et faites-les
cuire à petit feu, en casserole couverte, sans les toucher, jusqu'à ce qu'elles
soient fondues. Renversez-les sur une serviette tenue au-dessus d’une terrine
ou sur un tamis ; laissez égoutter, puis décantez le jus et filtrez-le.
Pesez-le ensuite.

Prenez, par kilogramme de jus, {\ppp900\mmm} grammes de sucre en morceaux,
faites-le fondre dans le jus ; ajoutez un peu de jus de citron, au goût, puis
cuisez le tout, à feu très vif, en remuant et en écumant bien, jusqu'à la
nappe. Versez dans des vases en verre chauffés au préalable et tenez-les
à l'étuve tiède pendant une demi-heure au moins.

\medskip

On peut parfumer la gelée de pommes avec de la vanille ou avec du jus d'oranges.

\section*{\centering Gelée de coings.}
\phantomsection
\addcontentsline{toc}{section}{ Gelée de coings.}
\index{Gelée de coings}

Prenez des coings mûrs et parfumés, pelez-les ou essuyez-les pour en enlever le
duvet ; coupez-les en morceaux, extrayez les pépins et les parties dures.
Mettez les coings dans une casserole avec du jus de citron, au goût, et juste
assez d'eau pour qu'ils trempent. Faites-les cuire en casserole couverte sans
les toucher. Aussitôt qu'ils sont bien tendres et qu'ils commencent à fondre,
versez-les sur un tamis tenu au-dessus d'une terrine ; laissez passer le jus,
décantez-le, filtrez-le à la chausse ; ou bien mettez les coings dans un
torchon suspendu au-dessus d'une terrine et laissez-les égoutter.

Faites cuire le jus de coings avec poids égal de sucre, à la nappe, en remuant et
en écumant bien.

Versez dans des pots chauffés que vous tiendrez à l'étuve tiède pendant une
demi-heure au moins.

\sk

On peut faire de la gelée de coings avec {\ppp3\mmm}/{\ppp4\mmm} de jus de
coings et {\ppp1\mmm}/{\ppp4\mmm} de jus de pommes.

\medskip

On peut la parfumer avec de la vanille, de la cannelle ou du zeste de citron.

\section*{\centering Gelée d'oranges.}
\phantomsection
\addcontentsline{toc}{section}{ Gelée d'oranges.}
\index{Gelée d'oranges}

Choisissez des oranges mûres et lourdes, pelez-les, retirez les pépins, pressez
la pulpe à fond, filtrez le jus.

Prenez, par kilogramme de jus d'oranges, {\ppp250\mmm} grammes de jus de pommes
et {\ppp750\mmm} grammes de sucre en morceaux ; mettez le tout dans une bassine
et, lorsque le sucre est fondu, cuisez à la nappe.

Laissez refroidir pendant {\ppp10\mmm} minutes et coulez la gelée dans des
pots.

\section*{\centering Gelée de grenades.}
\phantomsection
\addcontentsline{toc}{section}{ Gelée de grenades.}
\index{Gelée de grenades}

Prenez des grenades mûres, passez la pulpe au tamis ; laissez reposer, puis
décantez et filtrez le jus.

Préparez du bon jus de pommes reinettes. Ajoutez {\ppp300\mmm}  grammes de jus
de pommes par kilogramme de jus de grenades.

Pesez de {\ppp1\mmm} {\ppp500\mmm} à {\ppp1\mmm} {\ppp750\mmm} grammes de sucre
par kilogramme de jus suivant qu'il est plus où moins aigrelet ; mettez-le dans
une bassine avec un peu d'eau, cuisez-le au grand perlé, puis versez dedans les
jus de pommes et de grenades réunis, écumez soigneusement et achevez la cuisson
à la nappe.

Laissez refroidir pendant un quart d'heure ; puis coulez en pots.

\section*{\centering Gelée de citrons.}
\phantomsection
\addcontentsline{toc}{section}{ Gelée de citrons.}
\index{Gelée de citrons}

Prenez de belles pommes reinettes bien mûres ; faites-les cuire avec de l'écorce
de citrons râpée à la râpe plate, de façon à obtenir un jus bien parfumé. Filtrez-le.

Choisissez des citrons mûrs à point, à raison de {\ppp6\mmm} citrons par
kilogramme de jus de pommes ; pressez-les ; filtrez le jus.

Faites cuire du sucre au boulé ({\ppp1\mmm} {\ppp200\mmm} grammes de sucre par
kilogramme de jus), écumez bien. Versez dedans le jus de pommes et celui des
citrons et amenez à la nappe.

Laissez refroidir pendant un quart d'heure et coulez en pots.

\bigskip

\begin{center}
\textit{PÂTES DE FRUITS}
\end{center}

\bigskip

\addcontentsline{toc}{section}{ Pâtes de fruits.}
\index{Pâtes de fruits}
\index{Définition des pâtes de fruits}

Les pâtes de fruits sont des marmelades de purées de fruits très consistantes,
aromatisées ou non, cuites à la grande nappe, qu'on coule en couches plus ou
moins épaisses pour les faire refroidir et qu'on sèche ensuite à l'étuve.

On leur donne aussi le nom de « Pains de fruits ».

\section*{\centering Pâte de pommes.}
\phantomsection
\addcontentsline{toc}{section}{ Pâte de pommes.}
\index{Pâte de pommes}

Prenez de bonnes pommes reinettes, pelez-les, coupez-les en morceaux, enlevez
pépins et parties dures. Cuisez les pommes avec un peu d'eau en casserole
couverte, sans les toucher, jusqu'à ce qu'elles soient fondues. Passez-les au
tamis. Pesez la purée.

Mettez dans une bassine la purée avec poids égal de sucre pilé, et de la
vanille ; faites cuire à la grande nappe.

Coulez la marmelade, en couche d'un centimètre d'épaisseur, sur des plaques
poudrées de sucre fin, saupoudrez la surface de la pâte de sucre glace et
mettez à l'étuve chaude pendant {\ppp12\mmm} heures au moins. Aussitôt que la
pâte a une consistance suffisante pour être maniée, renversez-la sur d'autres
plaques. Laissez-la refroidir à la température ambiante en la retournant
fréquemment.

\section*{\centering Pâte de coings.}
\phantomsection
\addcontentsline{toc}{section}{ Pâte de coings.}
\index{Pâte de coings}

Prenez des coings bien mûrs, épluchez-les, émincez-les ; disposez-les ensuite
dans une casserole avec de l'eau légèrement additionnée de jus de citron et en
quantité juste suffisante pour qu'ils trempent ; couvrez la casserole et faites
cuire. Lorsque les fruits sont bien tendres, passez-les au tamis. Pesez la
purée.

Mettez dans une bassine la purée de coings avec poids égal de sucre pilé ;
réduisez la marmelade en la remuant constamment. Lorsqu'elle est à la grande
nappe, colorez-la un peu en rose avec de la cochenille liquide, puis coulez-la,
à {\ppp3\mmm}/{\ppp4\mmm} ou {\ppp1\mmm} centimètre d'épaisseur, dans des
boîtes ou sur des plaques poudrées de sucre, lissez la surface au couteau,
saupoudrez-la de sucre fin. Tenez la pâte à l'étuve chaude pendant {\ppp12\mmm}
heures ; laissez-la refroidir à l'air libre en la retournant souvent.

\section*{\centering Pâte de poires.}
\phantomsection
\addcontentsline{toc}{section}{ Pâte de poires.}
\index{Pâte de poires}

Préparez une bonne purée de poires, additionnez-la d'un cinquième de son poids
de purée de pommes reinettes ; pesez le tout.

Prenez même poids de sucre ; cuisez-le avec un peu d’eau au petit boulé,
écumez ; ajoutez ensuite la purée de fruits et achevez la cuisson à la grande
nappe en remuant constamment ; colorez légèrement en rouge avec un peu de
carmin liquide. On reconnait que la pâte est à point lorsqu'elle présente,
prise entre deux doigts, une certaine résistance et qu'elle produit un léger
claquement lorsqu'on écarte les doigts.

Versez-la sur des plaques poudrées de sucre, étalez-la à un centimètre
d'épaisseur environ, lissez la surface et mettez à l'étuve chaude pour sécher.
Laissez-la ensuite refroidir à l'air.

\medskip

Un peut parfumer cette pâte à la vanille.

\section*{\centering Pâte de cerises.}
\phantomsection
\addcontentsline{toc}{section}{ Pâte de cerises.}
\index{Pâte de cerises}

Prenez de belles cerises mûres et bien en chair, retirez noyaux et queues ;
passez la pulpe au tamis ; mettez-la dans une bassine, chauffez doucement et
réduisez jusqu'à moitié du volume primitif en remuant sans arrêt. Pesez la
préparation.

Faites cuire au boulé poids égal de sucre, écumez, ajoutez la purée réduite et
continuez la cuisson en remuant avec une spatule jusqu'à ce que la marmelade se
détache du fond.

Couchez la pâte dans un moule à l'épaisseur d'un centimètre environ, saupoudrez
de sucre et mettez à l’étuve chaude pendant {\ppp20\mmm} à {\ppp22\mmm} heures.
Laissez refroidir, démoulez et saupoudrez de sucre.

\sk

On pourra préparer dans le même esprit de la pâte de fraises.

\section*{\centering Pâte de groseilles.}
\phantomsection
\addcontentsline{toc}{section}{ Pâte de groseilles.}
\index{Pâte de groseilles}

Choisissez de belles groseilles fraîchement cueillies et bien mûres ;
faites-les crever sur feu doux avec {\ppp50\mmm} grammes d'eau par kilogramme
de groseilles, donnez cinq à six bouillons, puis passez le tout au tamis posé
sur une terrine, appuyez simplement sans presser. Lorsque tout le jus est
passé, faites-le réduire sur feu doux à moitié de son volume. Pesez-le.

Faites cuire au petit cassé même poids de sucre avec un peu d'eau, mélangez‑y
la purée de groseilles, cuisez jusqu'à la grande nappe.

Versez la pâte sur des plaques en fer-blanc, saupoudrez de sucre et séchez à
l'étuve chaude pendant une vingtaine d'heures.

\sk

On pourra préparer de même de la pâte de framboises,

\section*{\centering Pâte d'abricots.}
\phantomsection
\addcontentsline{toc}{section}{ Pâte d’abricots.}
\index{Pâte d’abricots}

Prenez des abricots bien mûrs, échaudez-les pour en enlever la peau ; ôtez les
noyaux.

Mettez les fruits dans une casserole avec très peu d'eau et un peu de sucre ;
chauffez pour faire fondre les fruits en surveillant l'opération. Passez au
tamis. Pesez la purée.

Prenez {\ppp850\mmm} à {\ppp1\mmm} {\ppp000\mmm} grammes de sucre par
kilogramme de purée, suivant que les fruits sont plus ou moins sucrés ;
faites-le cuire avec un peu d'eau soit au grand boulé, soit au petit cassé,
écumez, ajoutez la purée et continuez la cuisson en remuant sans cesse avec une
spatule jusqu'à ce que la marmelade soit à la grande nappe. Elle est à point
lorsqu'elle se détache du fond de la bassine et qu'on entend le léger
claquement caractéristique produit par l'écartement brusque des doigts
contenant un peu de pâte. Évitez que la préparation s'attache au fond de la
bassine.

Couchez la pâte en une épaisseur de {\ppp1\mmm} centimètre au plus sur des
plaques en fer-blanc, saupoudrez de sucre glace. Séchez-la à l'étuve chaude
pendant {\ppp12\mmm} heures ; retournez-la ensuite sur d'autres plaques,
garnies de papier blanc.

\medskip

On pourra parfumer, au goût, la pâte d'abricots avec un peu de kirsch.

\section*{\centering Pâte de prunes Reine-Claude.}
\phantomsection
\addcontentsline{toc}{section}{ Pâte de prunes Reine-Claude.}
\index{Pâte de prunes Reine-Claude}

La préparation est la même que celle de la pâte d'abricots.

\section*{\centering Pâte de goyaves.}
\phantomsection
\addcontentsline{toc}{section}{ Pâte de goyaves.}
\index{Pâte de goyaves}

Prenez des goyaves blanches de préférence, pelez-les et faites-les fondre sur
feu doux dans un peu d’eau ; passez la pulpe au tamis. Pesez la purée.

Faites cuire au grand boulé poids égal de sucre de canne avec un peu d'eau,
ajoutez la purée et achevez la cuisson à la grande nappe.

Coulez la pâte, sur une épaisseur de deux centimètres environ, dans des boîtes
de fer-blanc, saupoudrez de sucre de canne pilé et faites sécher à l’étuve
pendant {\ppp28\mmm} à {\ppp30\mmm} heures ; retournez-la, saupoudrez-la de
sucre et laissez-la refroidir.

\sk

On conserve les pâtes de fruits dans des petites caisses en bois, entre deux
feuilles de papier saupoudrées de sucre fin, où mieux encore dans des boîtes
à conserve en fer-blanc.

\medskip

Pour l'usage, on découpe ces pâtes avec des emporte-pièces, ou on les débite en
morceaux plus ou moins grands, en tablettes, en pastilles, etc.

\bigskip

\begin{center}
\textit{SIROPS DE FRUITS}
\end{center}

\bigskip

\addcontentsline{toc}{section}{ Sirops de fruits.}
\index{Sirops de fruits}
\index{Définition des sirops de fruits}

Les sirops de fruits sont des préparations liquides résultant de la
concentration de jus de fruits filtrés cuits avec du sucre ; en général, ils
marquent {\ppp35\mmm}° à l'aréomètre de Baumé, lorsqu'ils sont refroidis.

\sk

Les jus de fruits qui entrent dans la composition des sirops sont obtenus
différemment suivant les fruits. Voici comment il faut opérer.

Les groseilles, cerises, framboises, fraises, cassis, mûres, raisins, etc. sont
simplement écrasés. Les abricots, pêches, prunes, oranges, citrons, grenades,
etc. sont débarrassés de leurs écorces, pépins ou noyaux, puis pressés. Les
coings, poires, pommes, etc. sont pelés, puis râpés. On laisse macérer la
pulpe ; on la presse ensuite.

On agite d'abord ces sucs de fruits, on les laisse ensuite reposer au frais et
couverts pendant un temps variable (de {\ppp12\mmm} à {\ppp48\mmm} heures et
même davantage) pour qu'ils s'éclaircissent, enfin on les filtre.

\medskip

La préparation des sirops doit s'effectuer en bassines de fer émaillées ou dans
du bi-métal ; les bassines de cuivre étamées ou non ne doivent jamais être
employées à cet usage ; les premières changent la coloration des sirops, les
secondes leur communiquent une saveur métallique.

\medskip

Les sirops, complètement refroidis, doivent être mis dans des bouteilles
parfaitement sèches, bouchées avec des bouchons secs de premier choix. On
tiendra les bouteilles dans un endroit sombre, frais et aéré.

\section*{\centering Sirop de pommes.}
\phantomsection
\addcontentsline{toc}{section}{ Sirop de pommes.}
\index{Sirop de pommes}

Prenez des pommes reinettes bien mûres, pelez-les, râpez-les, mettez la pulpe
à macérer pendant {\ppp10\mmm} à {\ppp12\mmm} heures, pressez-la ensuite,
recueillez le jus, laissez-le reposer au frais et couvert pendant {\ppp24\mmm}
heures. Décantez-le, filtrez-le.

Pesez {\ppp1\mmm} {\ppp200\mmm} à {\ppp1\mmm} {\ppp700\mmm} grammes de sucre
par litre de jus de pommes, suivant que les fruits sont plus ou moins sucrés ;
mettez-le dans une bassine avec le jus de pommes ; faites cuire, en écumant,
pendant un quart d'heure environ, jusqu'au moment où le sirop est à la grande
nappe ; passez-le ensuite au blanchet. Laissez refroidir.

\medskip

On peut aromatiser le sirop de pommes avec de la cannelle.

\section*{\centering Sirop de coings.}
\phantomsection
\addcontentsline{toc}{section}{ Sirop de coings.}
\index{Sirop de coings}

Prenez des coings jaunes bien mûrs, essuyez-les pour en enlever le duvet,
pelez-les ou ne les pelez pas ; râpez-les jusqu'au cœur. Mettez la pulpe
à macérer pendant {\ppp6\mmm} à {\ppp8\mmm} heures ; pressez-la ensuite.
Laissez reposer le jus pendant {\ppp24\mmm} heures, décantez-le, filtrez-le.

Pesez {\ppp1\mmm} {\ppp700\mmm} à {\ppp2\mmm} {\ppp000\mmm} grammes de sucre
par litre de jus et finissez la préparation comme il est dit pour le sirop de
pommes.

\section*{\centering Sirop d'abricots.}
\phantomsection
\addcontentsline{toc}{section}{ Sirop d'abricots.}
\index{Sirop d'abricots}

Choisissez des abricots de plein vent bien mûrs ; enlevez-en les noyaux ;
passez les fruits au tamis ; pressez la pulpe ; ou bien faites fondre les
abricots sur feu doux avec un peu d'eau, pressez-les. Recueillez le jus,
laissezle reposer pendant {\ppp15\mmm} à {\ppp20\mmm} heures au frais et
couvert ; filtrez-le.

Mettez-le dans une bassine avec {\ppp1\mmm} kilogramme de sucre par litre de
jus et cuisez jusqu'à ce que le sirop marque {\ppp32\mmm}°. Passez-le au blanchet.

Laissez refroidir.

\section*{\centering Sirop de cerises.}
\phantomsection
\addcontentsline{toc}{section}{ Sirop de cerises.}
\index{Sirop de cerises}

Prenez des cerises mûres à point, enlevez-en les queues et les noyaux. Écrasez
les fruits ; laissez reposer le tout au frais, en terrine couverte, pendant
{\ppp24\mmm} heures, passez ensuite la pulpe au tamis en pressant, puis filtrez
le jus à la chausse.

Pesez {\ppp1\mmm} {\ppp700\mmm} à {\ppp2\mmm} {\ppp000\mmm} grammes de sucre
par litre de jus, mettez-le dans une bassine avec le jus de cerises et faites
cuire jusqu'à la grande nappe en écumant soigneusement. Filtrez au blanchet.

Laissez refroidir.

\medskip

On peut parfumer le sirop de cerises avec de la vanille ou de la cannelle.

\section*{\centering Sirop de fraises.}
\phantomsection
\addcontentsline{toc}{section}{ Sirop de fraises.}
\index{Sirop de fraises}

Prenez de belles fraises des quatre saisons, épluchez-les. Pesez le double de
leur poids de sucre en morceaux, mettez-le dans une bassine avec un tiers de
son poids d'eau. Cuisez au grand soufflé. Jetez dedans les fraises, donnez
quelques bouillons et versez de suite sur un blanchet.

Laissez refroidir.

\section*{\centering Sirop de framboises.}
\phantomsection
\addcontentsline{toc}{section}{ Sirop de framboises.}
\index{Sirop de framboises}

On peut préparer le sirop de framboises comme le sirop de fraises ou le sirop de
cerises.

\section*{\centering Sirop de groseilles.}
\phantomsection
\addcontentsline{toc}{section}{ Sirop de groseilles.}
\index{Sirop de groseilles}

Égrappez des groseilles bien mûres, écrasez-les, exprimez-en le jus au travers
d'uu linge, ou faites-les crever sur feu doux et passez le jus au tamis en
pressant. Laissez reposer le jus pendant {\ppp24\mmm} heures dans un endroit
frais en terrine couverte ; filtrez-le ensuite.

Faites cuire du sucre, à raison de {\ppp2\mmm} kilogrammes de sucre par litre
de jus, au grand boulé, écumez, ajoutez le jus de groseilles et achevez la
cuisson, en écumant bien, jusqu'à ce que le sirop soit à la grande nappe.
Filtrez au blanchet. Laissez refroidir.

\section*{\centering Sirop de groseilles et de cerises.}
\phantomsection
\addcontentsline{toc}{section}{ Sirop de groseilles et de cerises.}
\index{Sirop de groseilles et de cerises}

Prenez {\ppp9\mmm} parties de groseilles pour {\ppp1\mmm} partie de cerises
aigres ; égrappez les groseilles ; enlevez les noyaux et les queues des
cerises. Écrasez les fruits à la main dans une terrine ; tenez au frais
à couvert pendant {\ppp24\mmm} heures, puis filtrez le jus à la chausse.

Pesez {\ppp2\mmm} kilogrammes de sucre en morceaux par litre de jus ; mettez
sucre et jus dans une bassine, laissez fondre, puis cuisez jusqu'à
{\ppp32\mmm}°. Passez de nouveau à la chausse et ajoutez {\ppp100\mmm} grammes
de sirop de framboises par kilogramme de sirop de groseilles et de cerises.

\section*{\centering Sirop de cassis.}
\phantomsection
\addcontentsline{toc}{section}{ Sirop de cassis.}
\index{Sirop de cassis}

On prépare le sirop de cassis comme le sirop de groseilles.

\section*{\centering Sirop de mûres.}
\phantomsection
\addcontentsline{toc}{section}{ Sirop de mûres.}
\index{Sirop de mûres}

Prenez des mûres entières, écrasez-les, laissez-les fermenter pendant
{\ppp24\mmm} heures, puis pressez le jus, filtrez-le.

Pesez {\ppp1\mmm} {\ppp000\mmm} à {\ppp1\mmm} {\ppp500\mmm} grammes de sucre
par litre de jus suivant la douceur des fruits ; mettez-le dans une bassine
avec le jus de mûres et cuisez jusqu'à la nappe, en remuant et en écumant,
Filtrez au blanchet,

Laissez refroidir.

\section*{\centering Sirop d'oranges.}
\phantomsection
\addcontentsline{toc}{section}{ Sirop d'oranges.}
\index{Sirop d'oranges}

Prenez de belles oranges lourdes et mûres à point et des citrons, dans la
proportion de {\ppp12\mmm} oranges pour {\ppp2\mmm} citrons ; râpez les écorces
des fruits à la râpe plate. Versez dessus un litre de sirop de sucre bouillant
et laissez en contact pendant {\ppp48\mmm} heures au moins ; filtrez le sirop
au blanchet.

Pressez les oranges et les citrons au presse-citron ; recueillez le jus ;
laissez-le reposer pendant {\ppp18\mmm} à {\ppp20\mmm} heures, à couvert.
Filtrez-le,

Pesez {\ppp1\mmm} {\ppp200\mmm} à {\ppp1\mmm} {\ppp500\mmm} grammes de sucre
par litre de jus ; faites-le cuire au perlé avec un peu d'eau ; ajoutez le jus
des fruits ; cuisez jusqu à {\ppp32\mmm}°. Passez le sirop au blanchet.

Réunissez les deux sirops ; mélangez bien ; laissez refroidir.

\section*{\centering Sirop de mandarines.}
\phantomsection
\addcontentsline{toc}{section}{ Sirop de mandarines.}
\index{Sirop de mandarines}

Prenez de belles mandarines, parfumées et mûres à point ; enlevez les zestes ;
écrasez la pulpe ; pressez-la dans un linge ; recueillez le jus ; laissez-le
reposer à couvert ; décantez-le.

Faites un sirop au perlé avec un peu d'eau et {\ppp1\mmm} {\ppp200\mmm} grammes
de sucre par litre de jus ; ajoutez le jus des mandarines et continuez la
cuisson jusqu'à la nappe.

Versez le sirop bouillant sur plus ou moins de zeste de mandarines, au goût,
et filtrez au blanchet,

Laissez refroidir.

\label{pg0988} \hypertarget{p0988}{}
\section*{\centering Sirop de grenades.}
\phantomsection
\addcontentsline{toc}{section}{ Sirop de grenades.}
\index{Sirop de grenades}

Ouvrez des grenades bien mûres, sortez-en les grains. Mettez-les dans un poêlon
sur le feu ; chauffez doucement. Lorsque les grains sont crevés, versez-les
dans une serviette ; tordez fortement pour en extraire le jus. Laissez-le
reposer au frais et couvert ; filtrez-le.

Prenez de {\ppp1\mmm} {\ppp500\mmm} à {\ppp1\mmm} {\ppp700\mmm} grammes de
sucre en morceaux par litre de jus, suivant qu'il est plus ou moins aigrelet ;
faites-le cuire au grand soufflé avec un peu d'eau. Mettez dedans le jus de
grenades et continuez la cuisson jusqu'à ce que le sirop pèse {\ppp35\mmm} ou
{\ppp36\mmm}°. Passez le sirop au blanchet.

Laissez-le refroidir.

\section*{\centering Sirop d'ananas de conserve.}
\phantomsection
\addcontentsline{toc}{section}{ Sirop d'ananas de conserve.}
\index{Sirop d'ananas de conserve}

Prenez du jus d’ananas de conserve, même volume d'eau et {\ppp700\mmm} grammes
de sucre par litre de jus d'ananas.

Faites cuire le sucre avec l'eau au grand boulé ; écumez bien, ajoutez le jus
d'ananas et concentrez le tout jusqu'à la grande nappe.

\medskip

On peut parfumer ce sirop, au goût, avec un peu de kirsch.

\section*{\centering Sirop d'ananas frais.}
\phantomsection
\addcontentsline{toc}{section}{ Sirop d'ananas frais.}
\index{Sirop d'ananas frais}

Prenez de la pulpe d'ananas, ajoutez-y même poids d'alcool à {\ppp90\mmm}° :
laissez macérer le tout pendant {\ppp10\mmm} jours. Filtrez.

Pelez un ananas mûr à point, passez la pulpe au tamis, laissez reposer le suc
pendant plusieurs heures ; décantez le jus ; filtrez-le.

Pesez {\ppp1\mmm} {\ppp500\mmm} à {\ppp1\mmm} {\ppp700\mmm} grammes de sucre de
canne par litre de jus ; mettez-le dans une bassine avec le jus d’ananas ;
cuisez en écumant jusqu'à ce que le sirop soit à la grande nappe. Éloignez la
bassine du feu, ajoutez au sirop plus ou moins d'alcoolature d'ananas, au goût,
pour renforcer son parfum ; mélangez bien.

Mettez le sirop à l'étuve tiède et faites évaporer l'alcool en remuant. Filtrez
à la chausse.

\label{pg0989} \hypertarget{p0989}{}
\section*{\centering Sirop d'orgeat.}
\phantomsection
\addcontentsline{toc}{section}{ Sirop d'orgeat.}
\index{Sirop d'orgeat}

Prenez les éléments suivants. dans les proportions indiquées :

\footnotesize
\begin{longtable}{rrrp{16em}}
  1 500 & grammes & de & sucre en morceaux,                                                               \\
    800 & grammes & d' & eau,                                                                             \\
    250 & grammes & d' & amandes douces,                                                                  \\
    125 & grammes & d' & eau de fleurs d'oranger,                                                         \\
     75 & grammes & d' & amandes amères,                                                                  \\
     15 & grammes & de & gomme arabique pulvérisée.                                                       \\
\end{longtable}
\normalsize

Échaudez les amandes, mondez-les, mettez-les dans un mortier et pilez-les avec
{\ppp375\mmm} grammes de sucre et {\ppp65\mmm} grammes d'eau, de façon à en
faire une pâte que vous délayerez avec le reste de l'eau et que vous passerez
ensuite à la serviette en tordant fortement. Ajoutez à l'émulsion d'amandes le
reste du sucre et la gomme ; faites cuire doucement au bain-marie, entre
{\ppp38\mmm} et {\ppp40\mmm}° C.

Lorsque la préparation est à point, incorporez l'eau de fleurs d'oranger et
filtrez au blanchet. Tenez toute la préparation à couvert pour éviter qu'il se
forme une pellicule à la surface.

\section*{\centering Sirop de pistaches.}
\phantomsection
\addcontentsline{toc}{section}{ Sirop de pistaches.}
\index{Sirop de pistaches}

On prépare le sirop de pistaches de la même manière que le sirop d'orgeat en
remplacant dans la préparation les amandes par un même poids de pistaches
mondées.
