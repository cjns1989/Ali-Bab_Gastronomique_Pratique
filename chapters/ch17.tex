\section*{\centering Gras-double\footnote{Par métonymie, on désigne à Paris
sous le nom de gras-double ce qu'on appelle ailleurs les tripes. Les tripes
sont composées non seulement du gras-double proprement dit, qui est la panse ou
le premier estomac des ruminants, mais encore du bonnet ou deuxième estomac, du
feuillet ou troisième estomac et de la caillette ou franche-mulle qui est le
quatrième estomac,} au vin blanc.}
\index{Définition des issues}
\index{Issues (Définition des)}
\index{Définition du gras-double}
\index{Gras-double (Définition du)}

\addcontentsline{toc}{section}{ Gras-double au vin blanc.}
\index{Gras-double au vin blanc}
\label{pg0406} \hypertarget{p0406}{}

Pour dix à douze personnes prenez :

\medskip

\footnotesize
\begin{longtable}{rrrrp{18em}}
  & \multicolumn{2}{r}{2 kilogrammes} & de & gras-double nettoyé et blanchi, tel qu'on le vend à Paris,   \\
  & 650 & grammes & de & sauternes,                                                                       \\
  & 250 & grammes & de & carottes coupées en rondelles,                                                   \\
  & 250 & grammes & d' & oignons,                                                                         \\
  & 200 & grammes & d' & eau,                                                                             \\
  & 100 & grammes & de & poireaux,                                                                        \\
  &  60 & grammes & de & sel gris,                                                                        \\
  &  50 & grammes & de & fine champagne,                                                                  \\
  &  40 & grammes & de & céleri,                                                                          \\
  &   8 & grammes & de & poivre en grains,                                                                \\
  &   4 & grammes & de & thym,                                                                            \\
  &   4 & grammes & de & feuilles de laurier,                                                             \\
  &     &         &  4 & clous de girofle,                                                                \\
  &     &         &  1 & fort pied de veau désossé et coupé en morceaux,                                  \\
  &     &         &    & muscade râpée.                                                                   \\
\end{longtable}
\normalsize

Préparez : d’une part, deux bouquets garnis avec les poireaux, le céleri, le
thym et le laurier ; d'autre part, deux sachets en mousseline contenant chacun
la moitié des oignons, du poivre en grains et des clous de girofle.

Lavez le gras-double à l'eau chaude ; couper-le en morceaux de
{\ppp7\mmm} à {\ppp8\mmm} centimètres de côté.

Mettez dans une cocote en porcelaine allant au feu, la moitié des carottes, un
bouquet et un sachet, puis le gras-double, le pied de veau, ensuite le reste
des carottes, le deuxième bouquet et le deuxième sachet, le sel, de la muscade
au goût ; mouillez avec l'eau, le sauternes et la fine champagne. Faites cuire
au four à feu doux et en marmite fermée, pendant douze heures au moins.

Retirez les bouquets et les sachets.

Servez sur assiettes chaudes.

\sk

Dans le Midi, on met souvent deux gousses d'ail.

\sk

On obtiendra les tripes à la périgourdine en ajoutant des truffes coupées en
tranches, qui cuisent avec les tripes.

\sk

On aura les tripes à la mode de Caen en remplacant le vin par du cidre pur jus
de pommes et la fine champagne par du calvados.

\sk

Comme autre variante intéressante, on fera le mouillement avec {\ppp350\mmm}
grammes de vin blanc, {\ppp350\mmm} grammes de cidre, {\ppp200\mmm} grammes de
bon bouillon et {\ppp50\mmm} grammes d'eau-de-vie de Châteauneuf-du-Pape ; de
plus, on ajoutera aux éléments ci-dessus énumérés {\ppp125\mmm} grammes de
tomates concassées et épépinées.

\newpage

\section*{\centering Gras-double au lard.}
\addcontentsline{toc}{section}{ Gras-double au lard.}
\index{Gras-double au lard}

Pour dix à douze personnes prenez :

\smallskip

\footnotesize
\begin{longtable}{rrrrp{18em}}
  & \multicolumn{2}{r}{2 kilogrammes} & de & gras-double blanchi et tendre,                               \\
  & 750 & grammes & de & bon bouillon,                                                                    \\
  & 300 & grammes & de & lard de poitrine,                                                                \\
  & 250 & grammes & de & fine champagne,                                                                  \\
  & 250 & grammes & de & carottes coupées en rondelles,                                                   \\
  & 200 & grammes & d' & oignons,                                                                         \\
  & 125 & grammes & de & beurre,                                                                          \\
  & 100 & grammes & de & poireaux,                                                                        \\
  &  50 & grammes & de & persil,                                                                          \\
  &  40 & grammes & de & céleri,                                                                          \\
  &  20 & grammes & de & sel gris,                                                                        \\
  &   4 & grammes & de & thym,                                                                            \\
  &   2 & grammes & de & laurier,                                                                         \\
  &   2 & grammes & de & poivre fraîchement moulu,                                                        \\
  &     &         &  4 & clous de girofle qu'on piquera chacun dans un oignon,                            \\
  &     &         &  3 & jaunes d'œufs,                                                                   \\
  &     &         &  1 & fort pied de veau désossé et coupé en morceaux,                                  \\
  &     &         &    & muscade,                                                                         \\
  &     &         &    & quatre épices.                                                                   \\
\end{longtable}
\normalsize

Préparez deux bouquets garnis avec les poireaux, le persil, le céleri, le thym
et le laurier.

Lavez le gras-double à l'eau chaude, essuyez-le et coupez-le en morceaux.

Faites dorer dans un peu de graisse le lard coupé en tranches minces.

Faites flamber la fine champagne.

Mettez dans une marmite en porcelaine allant au feu le beurre et le
gras-double, faites revenir à feu vif, pendant un quart d'heure, sans laisser
prendre couleur ; mouillez avec le bouillon et la fine champagne brûlée, puis
ajoutez pied de veau, lard revenu, carottes, oignons, sel, poivre, muscade,
quatre épices, les deux bouquets garnis, l’un au fond, l'autre au-dessus ;
donnez un bouillon, diminuez le feu, et laissez cuire, au four, à feu doux et
en casserole fermée, pendant douze heures.

Une heure ou deux avant la fin de la cuisson, si la sauce est trop longue,
poussez le feu pour la faire réduire ; goûtez et complétez l'assaisonnement
s'il est nécessaire.

Au moment de servir, enlevez le gras-double, le lard, le pied de veau et les
carottes, que vous dresserez sur un plat tenu au chaud ; dégraissez la sauce,
passez-la, liez-la avec les jaunes d'œufs, puis versez-la dans le plat et
servez. C'est, à mon avis, l’un des procédés qui conserve le mieux au
gras-double son goût propre, tout en lui donnant un moelleux qu'il ne peut
avoir lorsqu'il est simplement cuit au naturel.

\section*{\centering Gras-double à l'échalote.}
\addcontentsline{toc}{section}{ Gras-double à l'échalote.}
\index{Gras-double à l'échalote}

Ce plat doit être préparé trois jours à l'avance.

\medskip

Pour dix à douze personnes prenez :

\medskip

\footnotesize
\begin{longtable}{rrrrp{18em}}
  & \multicolumn{2}{r}{2 kilogrammes} & de & gras-double blanchi et tendre,                               \\
  & 500 & grammes & de & bon bouillon,                                                                    \\
  & 300 & grammes & de & bon vin blanc sec,                                                               \\
  & 250 & grammes & de & lard de poitrine,                                                                \\
  & 250 & grammes & de & carottes,                                                                        \\
  & 175 & grammes & de & bon cognac ou de fine champagne,                                                 \\
  & 125 & grammes & de & beurre,                                                                          \\
  & 100 & grammes & de & poireaux,                                                                        \\
  & 100 & grammes & d’ & oignons,                                                                         \\
  & 100 & grammes & d’'& échalotes,                                                                       \\
  &  50 & grammes & de & persil,                                                                          \\
  &  40 & grammes & de & céleri,                                                                          \\
  &  20 & grammes & de & sel gris,                                                                        \\
  &   4 & grammes & de & thym,                                                                            \\
  &   2 & grammes & de & laurier,                                                                         \\
  &   2 & grammes & de & poivre fraîchement moulu,                                                        \\
  &     &         &  4 & clous de girofle,                                                                \\
  &     &         &  1 & fort pied de veau désossé et coupé en morceaux,                                  \\
  &     &         &    &  muscade,                                                                        \\
  &     &         &    &  quatre épices.                                                                  \\
\end{longtable}
\normalsize

Préparez : d'une part, deux bouquets contenant chacun la moitié des poireaux,
du persil, du céleri, du thym et du laurier ; d'autre part, deux sachets
renfermant chacun la moitié des oignons, des échalotes et des clous de girofle.

Lavez le gras-double à l'eau chaude ; essuyez-le ; coupez-le en morceaux.

Faites dorer dans un peu de graisse le lard coupé en tranches minces.

Mettez, dans une marmite en porcelaine allant au feu, le beurre et le
gras-double ; faites revenir à feu vif pendant un quart d'heure sans laisser
prendre couleur ; flambez au cognac ou à la fine champagne ; mouillez avec le
bouillon et le vin, glissez au fond de la marmite un bouquet et un sachet, puis
ajoutez le pied de veau, le lard revenu, les carottes, le sel, le poivre, de la
muscade et des quatre épices au goût, le second bouquet et le second sachet ;
amenez à ébullition ; donnez quelques bouillons, diminuez le feu et laissez
cuire, en marmite fermée, pendant six heures.

Laissez refroidir le tout dans la marmite.

Le lendemain, réchauffez le gras-double et faites-le cuire, comme précédemment,
pendant six heures ; laissez-le de nouveau refroidir.

Le troisième jour, remettez la marmite avec son contenu sur le feu et faites
cuire à nouveau, toujours dans les mêmes conditions, pendant cinq heures.

Enlevez ensuite les deux bouquets et les deux sachets.

Dressez le gras-double, le pied de veau et les carottes sur un plat tenu au chaud,
dégraissez la sauce, réduisez-la un peu s'il est nécessaire, versez-la dans le plat et
servez aussitôt.

Cette préparation, par son moelleux, son fondant, son parfum et son goût
exquis, est l'une des meilleures que je connaisse.

\section*{\centering Gras-double au gratin.}
\addcontentsline{toc}{section}{ Gras-double au gratin.}
\index{Gras-double au gratin}

Pour dix à douze personnes prenez :

\medskip

\footnotesize
\begin{longtable}{rrrp{18em}}
  2 500 & grammes & de & gras-double tel qu'on le vend à Paris,                                           \\
    750 & grammes & de & bouillon,                                                                        \\
    500 & grammes & de & crème,                                                                           \\
    250 & grammes & de & fromage de Gruyère râpé,                                                         \\
    250 & grammes & de & parmesan râpé,                                                                   \\
    125 & grammes & de & mie de pain rassis tamisée,                                                      \\
     75 & grammes & de & carottes coupées en rondelles,                                                   \\
     75 & grammes & d' & oignons coupés en rondelles,                                                     \\
     50 & grammes & de & beurre,                                                                          \\
     50 & grammes & de & poireaux,                                                                        \\
     10 & grammes & de & céleri en branche,                                                               \\
      6 & grammes & de & paprika,                                                                         \\
      5 & grammes & de & persil,                                                                          \\
        &         &    & bouquet garni,                                                                   \\
        &         &    & vinaigre de vin\footnote{Il est bien difficile d'indiquer
                                                  exactement la quantité de vinaigre
                                                  nécessaire ; tout dépend de sa force :
                                                  avec un vinaigre doux de vin, que je
                                                  prépare moi-même, il en faut 30 grammes.},              \\
        &         &    & sel.                                                                             \\
\end{longtable}
\normalsize

Lavez le gras-double à l'eau chaude ; émincez-le en languettes de la grosseur
de bâtons de macaroni.

Hachez le persil et le céleri.

Mettez dans une casserole le beurre, le persil et le céleri hachés, faites
revenir, puis ajoutez le gras-double, les légumes, le bouquet garni ; mouillez
avec le bouillon et laissez cuire en casserole couverte jusqu'à ce que le
gras-double soit bien tendre. Quatre heures de cuisson suffisent généralement.

Découvrez alors la casserole, faites évaporer le liquide qui reste, retirez le
gras-double, disposez-le dans une terrine, un plat creux ou un légumier allant
au feu, en couches alternées de gras-double, mie de pain, gruyère et parmesan
râpés, assaisonnez au fur et à mesure avec le paprika ; terminez par du fromage.
Mélangez bien, ajoutez la crème, aigrie au préalable avec un peu de vinaigre,
mélangez encore, goûtez, complétez l'assaisonnement s'il y a lieu ; mettez au
four pendant vingt minutes. Tous les éléments doivent être amalgamés entre eux
et le dessus du plat doré.

Servez aussitôt.

C'est absolument fondant.

\section*{\centering Gras-double sauté aux poireaux.}
\addcontentsline{toc}{section}{ Gras-double sauté aux poireaux.}
\index{Gras-double aux poireaux}

Pour quatre personnes prenez :

\medskip

\footnotesize
\begin{longtable}{rrrp{18em}}
    750 & grammes & de & gras-double blanchi et tendre,                                                   \\
    250 & grammes & de & poireaux (le blanc seulement) coupés en rondelles minces,                        \\
    150 & grammes & de & beurre,                                                                          \\
    150 & grammes & de & vin blanc,                                                                       \\
     50 & grammes & d' & huile,                                                                           \\
     20 & grammes & de & sel blanc,                                                                       \\
      5 & grammes & de & poivre fraîchement moulu,                                                        \\
      5 & grammes & de & persil haché,                                                                    \\
        &         &    & le jus d'un citron.                                                              \\
\end{longtable}
\normalsize

Coupez le gras-double en languettes de {\ppp1\mmm} centimètre de largeur sur
{\ppp4\mmm} centimètres de longueur.

Chauffez, dans une grande poêle, l'huile avec {\ppp40\mmm} grammes de beurre ;
mettez dedans le gras-double et faites-le sauter, en l'assaisonnant avec sel et
poivre, jusqu'à ce qu'il soit d'une couleur dorée et légèrement croquant. Il
faut vingt minutes environ pour arriver à ce résultat, en opérant sur un feu
vif.

En même temps, mais à part, faites dorer les poireaux dans le reste du beurre ;
amenez-les à la même couleur que le gras-double.

Mélangez le tout ensemble, mouillez avec le vin, saupoudrez avec le persil,
laissez cuire encore un peu, arrosez avec le jus de citron et servez,

Cette formule, qui diffère de la formule classique du gras-double à la
lyonnaise par la substitution du poireau à l'oignon, donne un produit beaucoup
plus digestible.

Il y à là un principe culinaire peu connu, dont je donnerai plusieurs
applications dans la suite, et sur lequel je désire attirer l'attention des
gastronomes,

\section*{\centering Gras-double braisé au jus de poireaux.}
\addcontentsline{toc}{section}{ Gras-double braisé au jus de poireaux.}
\index{Gras-double au jus de poireaux}

Pour douze personnes prenez :

\footnotesize
\begin{longtable}{rrrrp{18em}}
  & 2 500 & grammes & de & gras-double,                                                                  \\
  &   500 & grammes & de & couenne de lard,                                                              \\
  &    50 & grammes & de & sel gris,                                                                     \\
  &    25 & grammes & de & poivre blanc concassé,                                                        \\
  &    20 & grammes & de & persil,                                                                       \\
  &     5 & grammes & d' & estragon,                                                                     \\
  & \multicolumn{2}{r}{5 milligrammes} & de & cayenne,                                                   \\
  &       &         &  5 & poireaux,                                                                     \\
  &       &         &  5 & échalotes,                                                                    \\
  &       &         &  5 & petites gousses d'ail,                                                        \\
  &       &         &  2 & clous de girofle,                                                             \\
  &       &         &  1 & pied de bœuf,                                                                 \\
  &       &         &  1 & pied de veau,                                                                 \\
  &       &         &    & muscade.
\end{longtable}
\normalsize

Foncez avec la couenne une marmite en porcelaine épaisse allant au feu ; puis
mettez le gras-double coupé en gros morceaux et les pieds ; saupoudrez avec le
sel et le cayenne, ajoutez le persil, l'estragon et les poireaux réunis en
bouquet et, dans un sachet, le poivre, les échalotes, l'ail, les clous de
girofle et de la muscade au goût.

Lutez la marmite avec de la pâte, mettez au four et faites cuire pendant sept
heures, à température modérée. Ouvrez la marmite, retirez le sachet, le bouquet
de persil, estragon et poireaux, enlevez les os des pieds, mélangez pour
homogénéiser et servez sur assiettes chaudes.

\section*{\centering Gras-double aux tomates.}
\addcontentsline{toc}{section}{ Gras-double aux tomates.}
\index{Gras-double aux tomates}

Pour six personnes prenez :

\medskip

\footnotesize
\begin{longtable}{rrrp{18em}}
  1 500 & grammes & de & gras-double,                                                                     \\
    600 & grammes & de & bon consommé,                                                                    \\
    100 & grammes & de & purée de tomates concentrée,                                                     \\
     50 & grammes & de & beurre,                                                                          \\
     10 & grammes & de & farine,                                                                          \\
        &         &  2 & os à moelle de 16 centimètres de longueur,                                       \\
        &         &  1 & bouquet de persil et céleri,                                                     \\
        &         &    & chapelure,                                                                       \\
        &         &    & marjolaine,                                                                      \\
        &         &    & sel et poivre.                                                                   \\
\end{longtable}
\normalsize

Lavez le gras-double à l'eau chaude ; coupez-le en languettes : mettez-le dans
le consommé bouillant et laissez-le cuire à tout petits bouillons pendant le
temps nécessaire pour le bien attendrir, ce qui demande au moins trois heures.

Faites pocher les os à moelle, obturés aux extrémités, dans de l'eau salée.
Retirez la moelle, coupez-la en morceaux.

Préparez un roux avec {\ppp25\mmm} grammes de beurre et la farine : ajoutez le
gras-double, la moelle, la purée de tomates, le bouquet de persil et de céleri,
de la marjolaine au goût ; laissez mijoter ensemble pendant une demi-heure ;
goûtez et complétez l'assaisonnement avec sel et poivre, s'il est nécessaire.

Retirez le bouquet, versez le gras-double dans un plat, saupoudrez de chapelure,
arrosez avec le reste du beurre que vous aurez fait fondre et servez.

\section*{\centering Friture de gras-double.}
\addcontentsline{toc}{section}{ Friture de gras-double.}
\index{Friture de gras-double}

Pour dix à douze personnes prenez :

\medskip

\footnotesize
\begin{longtable}{rrrrp{18em}}
  2 & \multicolumn{2}{r}{kilogrammes} & de & gras-double blanchi et tendre,                               \\
    & 250 & grammes & de & carottes émincées,                                                             \\
    & 250 & grammes & d' & oignons émincés,                                                               \\
    & 100 & grammes & de & poireaux,                                                                      \\
    &  60 & grammes & de & sel gris,                                                                      \\
    &  40 & grammes & de & céleri,                                                                        \\
    &  10 & grammes & d' & ail,                                                                           \\
    &   8 & grammes & de & poivre en grains,                                                              \\
    &   4 & grammes & de & thym,                                                                          \\
    &   4 & grammes & de & laurier,                                                                       \\
 25 &\multicolumn{2}{r}{centigrammes} & de & paprika,                                                     \\
    &     &         &  1 & bouteille de sauternes,                                                        \\
    &     &         &  1 & fort pied de veau désossé et coupé en morceaux,                                \\
    &     &         &  4 & œufs entiers.                                                                  \\
    &     &         &  4 & clous de girofle, qu'on piquera dans les oignons,                              \\
    &     &         &    & muscade râpée,                                                                 \\
    &     &         &    & mie de pain rassis tamisée.                                                    \\
\end{longtable}
\normalsize

Coupez le gras-double en {\ppp24\mmm} morceaux.

Faites-le cuire pendant douze heures avec le vin, le pied de veau, les légumes,
les aromates et les condiments, moins le paprika, comme il est indiqué
\hyperlink{p0406}{p. \pageref{pg0406}}.

Retirez-le ensuite et égouttez-le.

Battez les œufs avec le paprika ; passez successivement les morceaux de
gras-double d'abord dans les œufs battus, ensuite dans la mie de pain ; puis,
lorsqu'ils seront convenablement panés, faites-les frire pendant {\ppp5\mmm}
à {\ppp6\mmm} minutes dans un bain de graisse à la température de la graisse
qui commence à fumer.

Servez-les sur une serviette et envoyez en même temps une saucière de sauce
tomate, \hyperlink{p0401}{p. \pageref{pg0401}} ou de sauce béarnaise,
\hyperlink{p0434}{p. \pageref{pg0434}}.

\section*{\centering Friture de gras-double.}
\addcontentsline{toc}{section}{ Friture de gras-double.}
\index{Friture de gras-double}
\index{Friture de gras-double (autre formule)}

\begin{center}
\textit{(Autre formule).}
\end{center}

\medskip

Pour quatre personnes prenez :

\medskip

\footnotesize
\begin{longtable}{rrrp{18em}}
    750 & grammes & de & gras-double,                                                                     \\
    500 & grammes & de & lait,                                                                            \\
     50 & grammes & d' & oignons,                                                                         \\
     10 & grammes & de & sel gris,                                                                        \\
      2 & grammes & de & poivre,                                                                          \\
        &         &  2 & œufs,                                                                            \\
        &         &  2 & citrons,                                                                         \\
        &         &  1 & bouquet garni,                                                                   \\
        &         &    & mie de pain rassis tamisée.                                                      \\
\end{longtable}
\normalsize

Pelez les oignons et coupez-les en rondelles.

Faites cuire pendant quatre heures, dans le lait bouillant, le gras-double,
coupé en morceaux carrés de six centimètres de côté, avec les oignons, le
bouquet garni, le sel et le poivre. Laissez le gras-double refroidir dans sa
cuisson ; puis retirez-le et enlevez avec un couteau une partie de la sauce
figée qui le recouvre.

Battez les œufs ; trempez dedans un à un les morceaux de gras-double ;
passez-les ensuite dans la mie de pain : faites-les frire dans un bain de
friture chaude : saupoudrez-les de sel blanc au sortr de la friture.

Servez sur une serviette, avec les citrons coupés par la moitié.

\section*{\centering Langue de bœuf salée\footnote{La salaison, qui s'effectue soit à sec, soit
                              dans une saumure, est un procédé très ancien
                              de conservation des aliments.}.}
\addcontentsline{toc}{section}{ Langue de bœuf salée.}
\index{Langue de bœuf salée}

Pour six personnes prenez :

\medskip

\footnotesize
\begin{longtable}{rrrrp{18em}}
  2 &  \multicolumn{2}{r}{kilogrammes} & de & sel gris,                                                   \\
    & 20 & grammes & de & salpêtre en poudre,                                                             \\
    & 12 & grammes & de & poivre fraîchement moulu,                                                       \\
    & 12 & grammes & de & poudre de curry,                                                                \\
    & 12 & grammes & de & quatre épices,                                                                  \\
    & 12 & grammes & de & paprika,                                                                        \\
    &  5 & grammes & de & thym mondé,                                                                     \\
    &    &         & 10 & oignons moyens coupés en rondelles,                                             \\
    &    &         & 10 & échalotes,                                                                      \\
    &    &         &  6 & gousses d'ail,                                                                  \\
    &    &         &  6 & feuilles de laurier,                                                            \\
    &    &         &  1 & langue de bœuf,                                                                 \\
    &    &         &    & vin blanc.                                                                      \\
\end{longtable}
\normalsize

Incisez la peau de la langue.

Frottez la langue avec le salpêtre, puis mettez-la dans un vase en porcelaine
avec les oignons, les échalotes, l'ail, le laurier, le thym, saupoudrez avec
poivre, curry, quatre épices, paprika, et couvrez le tout avec le sel gris.
Gardez au frais, en vase couvert, pendant quinze jours.

Retirez alors la langue, secouez-la et faites-la cuire pendant trois heures
dans du vin blanc. Laissez-la refroidir dans sa cuisson, puis sortez-la et
essuyez-la : elle est prête à être servie.

La langue de bœuf préparée ainsi est de beaucoup supérieure à tous les produits
similaires du commerce,

\section*{\centering Langue de bœuf fumée\footnote{\textit{
\index{Fumage des viandes et des poissons} 
Fumage}. — Le
fumage, ou exposition des viandes et des poissons à l'action de la fumée, est
un excellent procédé de conservation qui permet d'aromatiser ces substances de
différentes façons, suivant le combustible employé. Les substances à fumer sont
d'abord salées, essuyées et enfin exposées dans une cheminée ou mieux dans une
chambre à fumée, suffisamment loin du foyer pour que le feu ne les atteigne
pas, et suffisamment près pour que la fumée les pénètre bien. Le fumage doit se
faire progressivement et les matières combustibles doivent être fraîches, de
façon à donner le maximum de fumée. Les plus employées sont le bouleau, le
chêne, le hêtre, le peuplier et le mélèze, mais le combustible de choix
consiste en branches vertes de genévrier chargées de baies.
\protect\endgraf
On termine souvent l'opération du fumage par l'incinération d'aromates, tels
que laurier, romarin, thym, pruneaux secs, bois de réglisse, clous de girofle,
etc.
\protect\endgraf
Pour être convenablement fumées, les pièces de bœuf ou de mouton.ne doivent pas
peser plus de {\ppp4\mmm} kilogrammes. Leur fumage demande sept semaines environ, dont
une semaine de fumage intensif au genévrier et six semaines de fumage lent au
bois ordinaire, Au bout de six semaines de séchage, ces viandes fumées peuvent
être mangées crues ; elles peuvent également être préparées avec des légumes.
\protect\endgraf
Les jambons sont fréquemment mis à macérer dans de l'eau-de-vie aromatisée de
baies de genévrier, pendant quelques heures, au sortir de la saumure et avant
d'être soumis au fumage. La durée du fumage d'un jambon moyen est de quinze
jours environ.
\protect\endgraf
\index{Andouilles fumées}
\index{Boudins fumés}
\index{Saucisses fumées}
\index{Saucissons fumés}
On peut traiter de même les saucisses, les saucissons, les andouilles et les
boudins.
\protect\endgraf
\index{Poitrine d'oie fumée}
On fume aussi certaines parties de volaille et, en particulier, les poitrines
d'oie.
\protect\endgraf
\index{Poissons fumés}
\index{Anguilles fumées}
\index{Harengs fumés}
\index{Saumon fumé}
\index{Brochet fumé}
On fume également les poissons. La durée de l'opération est de {\ppp24\mmm} heures pour
les harengs, de {\ppp4\mmm} jours pour les brochets et les anguilles, de {\ppp3\mmm} semaines pour
les saumons.
\protect\endgraf
D'une façon générale, les substances à fumer doivent être suspendues pendant
l'opération successivement par les deux bouts, et les substances fumées doivent
être conservées pendues dans un endroit sec.}.}

\addcontentsline{toc}{section}{ Langue de bœuf fumée.}
\index{Langue de bœuf fumée}

La langue, salée au préalable, comme il vient d'être dit, puis essuyée, est
exposée pendant huit jours entiers à la fumée de branches vertes de genévrier
chargées de baies, dont on augmente progressivement l'intensité. Lorsqu'elle
est fumée, elle doit être suspendue dans un endroit sec.

Au bout de six semaines de séchage on peut la manger crue ou cuite, chaude
ou froide.

Lorsqu'elle devra être servie chaude, on la fera cuire simplement dans de l'eau
pendant trois heures ; mais, si on doit la servir froide, on prendra de
préférence comme liquide de cuisson un court-bouillon au vin, dans lequel on la
laissera refroidir.

\section*{\centering Tête de veau en aspic, sauce douce.}
\addcontentsline{toc}{section}{ Tête de veau en aspic, sauce douce.}
\index{Tête de veau en aspic, sauce douce}
\index{Aspic de tête de veau}
\label{pg0415} \hypertarget{p0415}{}
\label{pg0416} \hypertarget{p0416}{}

Pour dix à douze personnes prenez :

\medskip
\setlength\tabcolsep{.1em}
\footnotesize
% \begin{longtable}{@{}lp{1em}rrrp{18em}}
% \begin{longtable}{@{}lrrrp{16em}}
%   \begin{longtable}{@{} | l | r | r | r | p{18em}}
% \begin{longtable}{@{}lrrrp{18em}}
% \begin{tabular}{@{} | l | r | r | r | p{18em}}
\begin{tabular}{@{}lrrrp{18em}}
\normalsize1°\footnotesize  & & & 1 & belle tête de veau entière,                                         \\
    & & & 1 & grosse carotte et 4 plus petites,                                                           \\
    & & & 1 & poireau,                                                                                    \\
    & & & 1 & oignon,                                                                                     \\
    & & & 2 & clous de girofle qu'on piquera dans l'oignon,                                               \\
    & & &   & bouquet garni,                                                                              \\
    & & &   & sel gris.                                                                                   \\
    & & &   &                                                                                             \\
\normalsize 2° & \multicolumn{4}{l}{\normalsize   pour un litre de sauce :}                               \\
\footnotesize
    & & &   &                                                                                             \\
& 30 & grammes & d' & un mélange en parties égales de civette, estragon, cerfeuil et persil hachés,       \\
& 10 & grammes & de & sucre en poudre,                                                                    \\
&    &         &  4 & jaunes d'œufs durs,                                                                 \\
&    &         &    & huile, vinaigre,                                                                    \\
&    &         &    & sel.                                                                                \\
% \end{longtable}
\end{tabular}
\normalsize

\medskip

Lavez la tête, mettez-la dans une marmite en porcelaine allant au feu avec la
grosse carotte, le poireau, l'oignon, les clous de girofle, le bouquet garni,
de l'eau en quantité suffisante pour la couvrir et du sel gris, à raison de 15
grammes par litre d'eau ; faites bouillir, écumez, puis continuez la cuisson,
à liquide frissonnant comme pour un pot-au-feu, pendant trois heures.

En même temps, mettez à cuire dans le bouillon les quatre autres carottes, sans
les laisser ramollir.

Retirez la tête, désossez-la, coupez-la en petits morceaux. Prenez un moule,
décorez-en le fond et les parois avec des rondelles de carottes, puis mettez
les différents morceaux de tête en les mélangeant ; versez dessus le jus de
cuisson. Laissez prendre en gelée,

Préparez la sauce : pétrissez les jaunes d'œufs avec les fines herbes, puis
faites une sauce ordinaire à l'huile et au vinaigre salée au goût et
incorporez-y le sucre en poudre, les jaunes d'œufs et les fines herbes.

La proportion de sucre indiquée convient généralement, mais il est préférable
de s'en rapporter à son goût personnel et de n'ajouter le sucre que par petites
quantités, en remuant chaque fois la sauce de façon à l'homogénéiser
complètement.

C'est le sucre qui donne à la sauce son cachet spécial ; il ne faut pas
cependant qu'elle ait une saveur sucrée : il suffit qu'elle soit douce,

Démoulez l'aspic et servez, en envoyant en même temps la sauce dans une
saucière.

\sk

Comme variante, on peut servir la tête de veau en aspic, avec une sauce
moutarde froide qu'on obtiendra de la façon suivante : on délaiera des jaunes
d'œufs avec du jus de citron ou du vinaigre et de la moutarde, on ajoutera
ensuite de l'huile d'olive en tournant constamment ; enfin, on terminera
l'assaisonnement au goût avec sel, poivre et estragon haché.

\section*{\centering Tête de veau frite, sauce tomate.}
\addcontentsline{toc}{section}{ Tête de veau frite, sauce tomate.}
\index{Tête de veau frite, sauce tomate}

Préparez six heures à l'avance une pâte bien lisse, homogène et légèrement
fluide, avec du lait, de la farine et du sel.

Faites cuire la tête de veau comme il est dit
\hyperlink{p0425}{p. \pageref{pg0425}}. Coupez-la en morceaux, trempez chaque
morceau dans la pâte, puis plongez-les ainsi enrobés dans de la friture
chaude ; retirez-les dès que la pâte est cuite et servez aussitôt avec une
sauce tomate, \hyperlink{p0401}{p. \pageref{pg0401}}.

\sk
\index{Cervelle de veau frite, sauce tomate}
On peut préparer de même la cervelle de veau. La seule différence consiste en
ce qu'il ne faut la court-bouillonner que pendant une dizaine de minutes
seulement.

\section*{\centering Langue de veau, sauce aux raisins et aux amandes.}
\addcontentsline{toc}{section}{ Langue de veau, sauce aux raisins et aux amandes.}
\index{Langue de veau, sauce aux raisins et aux amandes}

Pour six personnes prenez :

\medskip

\footnotesize
\begin{longtable}{rrrrrp{18em}}
  & \hspace{2em}  & 60 & grammes & de & beurre,                                                           \\
  &    & 40 & grammes & de & raisins de Smyrne,                                                           \\
  &    & 25 & grammes & d' & amandes,                                                                     \\
  &    & 20 & grammes & de & farine,                                                                      \\
  &    & 20 & grammes & de & glace de viande,                                                             \\
  &    & 10 & grammes & de & jus de citron,                                                               \\
  &    &  5 & grammes & de & sucre en poudre,                                                             \\
  & \multicolumn{3}{r}{2 décigrammmes} & de & poivre,                                                     \\
  &    &    &         & 1 & langue de veau,                                                               \\
  &    &    &         & 1 & grosse carotte,                                                               \\
  &    &    &         & 1 & poireau,                                                                      \\
  &    &    &         & 1 & oignon piqué de 2 clous de girofle,                                           \\
  &    &    &         &   & bouquet garni,                                                                \\
  &    &    &         &   & sel blanc.                                                                    \\
\end{longtable}
\normalsize

Faites cuire la langue avec les légumes et le bouquet garni, pendant une heure
et demie en moyenne, comme il est dit pour la tête de veau,
\hyperlink{p0416}{p. \pageref{pg0416}}, et tenez-la au chaud. Concentrez le
bouillon, dégraissez-le.

Ébouillantez les amandes, pelez-les et émincez-les.

Triez les grains de raisin et lavez-les.

Faites roussir la farine dans le beurre, mouillez avec une quantité suffisante
de bouillon de cuisson, mettez la glace de viande, le sucre, le poivre, le jus
de citron, les raisins, les amandes émincées, goûtez et ajoutez, s'il y a lieu,
un peu de sel.

Dépouillez la langue, escalopez-la ; mettez les tranches dans la sauce,
laissez-les mijoter pendant une heure à petit feu et servez.

Envoyez en même temps un plat de riz sec ou de riz à l’étouffée au beurre clarifié.

\sk

\index{Carpe court-bouillonnée, sauce aux raisins et aux amandes}
La sauce aux raisins et aux amandes peut aussi accompagner certains poissons
et, en particulier, la carpe court-bouillonnée.

\section*{\centering Ris de veau au jus.}
\phantomsection
\addcontentsline{toc}{section}{ Ris de veau au jus.}
\index{Ris de veau au jus}

Pour six personnes prenez :

\medskip

\footnotesize
\begin{longtable}{rrrp{18em}}
    100 & grammes & de & beurre,                                                                          \\
     60 & grammes & de & carottes émincées,                                                               \\
     60 & grammes & d’ & oignons émincés,                                                                 \\
     50 & grammes & de & bon jus,                                                                         \\
     10 & grammes & de & persil en bouquet,                                                               \\
        &         &  1 & beau ris de veau,                                                                \\
        &         &    & sel et poivre.                                                                   \\
\end{longtable}
\normalsize

Enlevez le cornet, faites dégorger le ris pendant une heure au moins dans de
l'eau fraîche légèrement salée avec du sel gris. puis retirez-le et essuyez-le.
On recommande généralement, dans les traités classiques, de faire
blanchir\footnote{À cet effet, certaines personnes ébouillantent le ris ;
d'autres le plongent dans de l'eau froide, font chauffer et retirent le ris au
premier bouillon.} le ris à l'eau avant de le faire cuire ; à mon avis, c'est
une erreur, car on lui conserve mieux la finesse de son goût en ne lui faisant
pas subir cette première opération.

Voici comment je conseille d'opérer.

Mettez dans une casserole le beurre et le ris dégorgé, faites-le revenir
pendant un quart d'heure, de façon à le dorer légèrement, enlevez-le et
remplacez-le par les oignons, les carottes. le bouquet de persil ; faites
pincer\footnote{
\index{Définition du mot pincer au sens culinaire}
Au sens culinaire, le mot pincer veut dire attacher au fond de
la casserole.} les légumes pendant un quart d'heure, remettez le ris, mouillez
avec le jus, salez, poivrez et achevez la cuisson à petit feu pendant une
demi-heure.

Dégraissez, passez la sauce.

Servez le ris, masqué avec la sauce, sur un plat de légumes : petits pois, épinards,
chicorée ou oseille, par exemple.

\section*{\centering Escalopes de ris de veau panées, sauce suprême\footnote{Je
donne ici la formule de la sauce suprême simple. On peut, à volonté, la garnir
de fines herbes hachées, de champignons, de truffes.}.}
\phantomsection
\addcontentsline{toc}{section}{ Escalopes de ris de veau panées, sauce suprême.}
\index{Escalopes de ris de veau panées, sauce suprême}

Pour six personnes prenez :

\medskip

\footnotesize
\begin{longtable}{rrrp{18em}}
    250 & grammes & de & gelée de veau et de volaille\footnote{
\index{Fond de veau et volaille}
                   \protect
                   \label{pg0418} \hypertarget{p0418}{}
                   Pour préparer un litre de gelée de veau et de volaille, prenez :
                   \protect\endgraf
                   \begin{tabular}{rrrl}
                   \hspace{5em} 1 500 & grammes  & de & jarret et bas morceaux de veau,             \\
                   \hspace{5em}   300 & grammes  & de & gîte de bœuf,                               \\
                   \hspace{5em}   150 & grammes  & de & carottes,                                   \\
                   \hspace{5em}    50 & grammes  & d’ & oignons,                                    \\
                   \hspace{5em}    30 & grammes  & de & sel gris,                                   \\
                   \hspace{5em}    10 & grammes  & de & céleri,                                     \\
                   \hspace{5em}       & 2 litres & d' & eau,                                        \\
                   \hspace{5em}       &          &  3 & abatis de poulardes,                        \\
                   \hspace{5em}       &          &  2 & blancs d'œufs,                              \\
                   \hspace{5em}       &          &  1 & fort pied de veau,                          \\
                   \hspace{5em}       &          &  1 & poireau moyen,                              \\
                   \hspace{5em}       &          &  1 & petit bouquet de persil, thym et laurier,   \\
                   \hspace{5em}       &          &    & poivre au goût.                             \\
                   \end{tabular}                                                                    \\
                   \protect\endgraf
                  Mettez le tout dans une marmite, faites bouillir, écumez et continuez
                  la cuisson comme pour un pot-au feu jusqu'à réduction à un litre de liquide,
                  ce qui demande cinq heures environ, Passez, clarifiez avec les blancs
                  d'œufs ; laissez refroidir.}                                                            \\
    140 & grammes & de & beurre,                                                                          \\
    100 & grammes & de & crème,                                                                           \\
     60 & grammes & de & vin de Sauternes,                                                                \\
     15 & grammes & de & farine,                                                                          \\
        &         &  2 & œufs frais,                                                                      \\
        &         &  2 & jaunes d'œufs frais,                                                             \\
        &         &  1 & beau ris de veau,                                                                \\
        &         &    & mie de pain rassis tamisée,                                                      \\
        &         &    & jus de citron,                                                                   \\
        &         &    & sel et poivre.                                                                   \\
\end{longtable}
\normalsize

Parez le ris, mettez-le à dégorger pendant une heure dans de l'eau froide
légèrement salée. Escalopez-le ; cette opération demande un peu de soin.

Battez les deux œufs entiers.

Passez, à plusieurs reprises, les escalopes de ris d'abord dans les œufs
battus, puis dans la mie de pain, de façon à les paner convenablement.
Faites-les cuire à la poêle, dans {\ppp100\mmm} grammes de beurre, sans aucun
assaisonnement, pendant un quart d'heure.

\label{pg0419} \hypertarget{p0419}{}
Pendant leur cuisson, préparez la sauce suprême : mettez dans une casserole
{\ppp30\mmm} grammes de beurre et la farine, tournez sans laisser roussir,
délayez avec la glace de veau et volaille, mouillez avec le sauternes ; faites
cuire pendant un quart d'heure au moins, ajoutez ensuite le reste du beurre
coupé en petits morceaux ; achevez la liaison de l’ensemble avec les jaunes
d'œufs et la crème, mélangez bien ; chauffez sans faire bouillir, goûtez et
complétez l'assaisonnement avec sel, poivre et jus de citron.

Servez les escalopes de ris de veau masquées avec la sauce, ou servez-les
seules sur un plat et la sauce, à part, dans une saucière.

\section*{\centering Escalopes de ris de veau sautées, au piment.}
\phantomsection
\addcontentsline{toc}{section}{ Escalopes de ris de veau sautées, au piment.}
\index{Escalopes de ris de veau sautées, au piment}

Pour six personnes prenez :

\medskip

\footnotesize
\begin{longtable}{rrrp{18em}}
    250 & grammes & de & fond de veau, \hyperlink{p0426}{p. \pageref{pg0426}},                            \\
    250 & grammes & de & purée de tomates,                                                                \\
    125 & grammes & de & champignons de couche,                                                           \\
    125 & grammes & de & beurre,                                                                          \\
     50 & grammes & de & vin blanc,                                                                       \\
        &         &  3 & piments doux d'Espagne,                                                          \\
        &         &  2 & jaunes d'œufs,                                                                   \\
        &         &  1 & beau ris de veau,                                                                \\
        &         &    & sel et poivre.                                                                   \\
\end{longtable}
\normalsize

Parez le ris ; faites-le dégorger dans de l'eau fraîche.

Escalopez-le. Cette opération est un peu délicate aussi, étant donnée la
difficulté pour beaucoup de personnes de découper le ris cru, on pourra le
faire blanchir dans de l'eau salée, ce qui lui donnera un peu de fermeté. On le
laissera refroidir sous presse ; on l'escalopera facilement ensuite.

Pelez les champignons ; hachez-les et faites-les cuire dans {\ppp30\mmm} grammes de
beurre jusqu'à siccité, de manière à les griller. Cessez l'opération aussitôt
que vous sentirez se dégager l'odeur aromatique qui les caractérise.

Faites cuire les piments dans du fond de veau ; émincez-les en languettes ;
réservez la cuisson.

Chauffez la purée de tomates ; mettez dedans champignons grillés et piments
émincés. Tenez au chaud.

Faites sauter les escalopes de ris dans le reste du beurre, assaisonnez avec
sel et poivre. Énlevez-les et tenez-les au chaud, au bain-marie. Déglacez avec
le vin, ajoutez la cuisson des piments et le reste du fond de veau. Concentrez
la sauce, puis liez-la, hors du feu, avec les jaunes d'œufs.

Foncez un plat de service avec la purée de tomates, disposez dessus les escalopes
de ris, masquez-les avec la sauce et servez.

\sk

On peut augmenter la proportion de piments doux et même ajouter du piment rouge
de cayenne écrasé, si l'on aime la cuisine très relevée.

\sk

\index{Agneau (Ris et rognons d')}
\index{Agneau au piment}
\index{Brochettes de ris de veau ot de bacon}
\index{Brochettes de ris de veau, de lard et de champignons}
\index{Filets de soles au piment}
\index{Filets de turbot au piment}
\index{Filots de barbue au piment}
\index{Foie de veau au piment}
Ce mode de préparation est applicable à bien des substances : agneau et ris
d'agneau, veau, rognon de veau et foie de veau, gras-double, poulet, filets de
barbue, de turbot, de sole, etc.

\section*{\centering Brochettes de ris de veau et de bacon\footnote{Lard fumé anglais.}.}
\phantomsection
\addcontentsline{toc}{section}{ Brochettes de ris de veau et de bacon.}
\index{Brochettes de ris de veau et bacon}

Pour quatre personnes prenez :

\medskip

\footnotesize
\begin{longtable}{rrrp{18em}}
    125 & grammes & de & bacon,                                                                           \\
     30 & grammes & de & beurre,                                                                          \\
     25 & grammes & de & mie de pain rassis tamisée,                                                      \\
        &         &  1 & ris de veau moyen,                                                               \\
        &         &    & jus de citron,                                                                   \\
        &         &    & poivre fraîchement moulu,                                                        \\
        &         &    & sel.                                                                             \\
\end{longtable}
\normalsize

Parez le ris ; mettez-le à dégorger dans de l’eau froide salée ; essuyez-le ;
enveloppez-le d'un linge et mettez-le sous une planchette chargée d'un poids de
{\ppp2\mmm} à {\ppp3\mmm} kilogrammes. Laissez-le ainsi pendant une heure.

Coupez le ris en tranches carrées ou rectangulaires de {\ppp4\mmm}
à {\ppp6\mmm} centimètres de côté et de {\ppp1\mmm} centimètre d' épaisseur.

Coupez le bacon en tranches carrées ou rectangulaires de même surface, mais
d'une épaisseur de quelques millimètres seulement.

Faites sauter le bacon et le ris, pendant quelques minutes, dans une partie du
beurre ; puis enfilez tranches alternées de ris et de bacon sur quatre
brochettes ; roulez-les dans la mie de pain et faites-les griller sur un feu
doux de braise, de façon à les bien dorer.

Arrosez les brochettes garnies avec le reste du beurre, que vous aurez fait
fondre, aspergez-les de jus de citron, saupoudrez-les de poivre et servez.

Envoyez en même temps, par exemple, soit des petits pois ou des épinards à la
crème, soit une purée d’oseille ou une purée de cresson, ou encore une purée de
pommes de terre et de haricots verts.

\sk

Comme variante, on pourra présenter des brochettes de lard de poitrine, ris et
champignons.

\sk

\index{Brochettes de ris et de rognon de veau}
On peut préparer dans le même esprit des brochettes de ris et de rognons de
veau.

\sk

\index{Brochettes de ris d'agneau et de bacon}
\index{Brochettes de ris d'agneau, de lard et de champignons}
\index{Brochettes de ris et de rognons d'agneau}
On peut apprêter de même des rognons et des ris d'agneau. On prendra en moyenne
{\ppp5\mmm} à {\ppp6\mmm} ris d'agneau par personne.

\section*{\centering Timbale de ris de veau.}
\phantomsection
\addcontentsline{toc}{section}{ Timbale de ris de veau.}
\index{Timbale de ris de veau}

Les croûtes de timbale sont généralement préparées avec de la pâte brisée ;
mais on peut les faire aussi avec d'autres pâtes. En voici un exemple ayant
donné un très bon résultat.

Pour huit personnes prenez :

\medskip

1° pour la croûte :

\footnotesize
\begin{longtable}{rrrp{18em}}
    400 & grammes & de & farine,                                                                          \\
    250 & grammes & de & beurre,                                                                          \\
    100 & grammes & d' & eau,                                                                             \\
     10 & grammes & de & sel,                                                                             \\
        &         &  2 & œufs frais ;                                                                     \\
\end{longtable}
\normalsize

\medskip

2° pour la garniture :

\footnotesize
\begin{longtable}{rrrp{18em}}
    500 & grammes & de & quenelles de volaille,                                                           \\
    500 & grammes & de & rognons de coq,                                                                  \\
    250 & grammes & de & petits champignons de couche,                                                    \\
        & 1 litre & de & fond de veau concentré,                                                          \\
        &         &  8 & foies de volaille,                                                               \\
        &         &  1 & beau ris de veau,                                                                \\
        &         &    & truffes à volonté,                                                               \\
        &         &    & beurre,                                                                          \\
        &         &    & farine,                                                                          \\
        &         &    & madère,                                                                          \\
        &         &    & jus de citron,                                                                   \\
        &         &    & sel et poivre.                                                                   \\
\end{longtable}
\normalsize

Préparez une pâte homogène avec la farine, le beurre, les œufs, l'eau et le
sel ; laissez-la reposer jusqu'au lendemain.

Parez le ris ; mettez-le à dégorger dans de l'eau fraîche un peu salée, pendant
une heure environ.

Abaissez la pâte ; réservez-en une partie pour le couvercle ; chemisez avec le
reste un moule à timbale de {\ppp12\mmm} à {\ppp15\mmm} centimètres de diamètre et {\ppp20\mmm} centimètres
de hauteur environ.

Découpez dans la pâte réservée un disque du diamètre de la timbale ;
décorez-le, dorez-le au jaune d'œuf.

Emplissez la timbale avec des haricots ou des cailloux lavés et faites-la cuire
au four ainsi que le couvercle.

Pendant la cuisson de la croûte, préparez la garniture.

Épluchez les champignons, passez-les dans du jus de citron ; lavez et brossez
les truffes ; faites cuire les champignons dans du beurre, les truffes dans du
madère.

Émincez les truffes ; réservez le madère de cuisson.

Passez le ris et les foies dans du beurre, sans leur laisser prendre couleur ;
escalopez le ris. Faites cuire ris et foies dans le fond de veau. Quelques
minutes avant la fin ajoutez les quenelles et les rognons de coq après les
avoir fait blanchir et achevez la cuisson de l’ensemble.

Faites un roux avec de la farine et du beurre, mouillez avec le jus de cuisson
du ris, donnez quelques bouillons ; mettez dans la sauce escalopes de ris de
veau, quenelles et foies de volaille, rognons de coq, champignons, truffes et
madère réservé, chauffez, goûtez pour l'assaisonnement, puis garnissez-en la
croûte.

Couvrez la timbale et servez aussitôt.

\section*{\centering Cervelles\footnote{
\index{Bœuf (Cervelle de), sauce hollandaise à la ravigote}
\index{Cervelle de bœuf, sauce hollandaise à la ravigote}
\index{Cervelle de mouton, sauce hollandaise à la ravgote}
\index{Cervelle de porc, sauce hollandaise à Ia ravigote}
\index{Cervelle de veau, sauce hollandaise à la ravigote}
Les préparations indiquées pour la
cervelle de veau sont applicables aux cervelles de tous les animaux de
boucherie et à celle de porc.} de veau, sauce hollandaise à la ravigote.}
\phantomsection
\addcontentsline{toc}{section}{ Cervelles de veau, sauce hollandaise à la ravigote.}
\index{Cervelles de veau, sauce hollandaise à la ravigote}

Pour six personnes prenez :

\medskip

\footnotesize
\begin{longtable}{rrrp{15em}}
    250 & grammes & d' & un mélange en parties égales de cerfeuil,
                         estragon, pimprenelle, civette, cresson alénois,                                 \\
    250 & grammes & de & beurre,                                                                          \\
     30 & grammes & d' & eau,                                                                             \\
      8 & grammes & d' & huile d'olive,                                                                   \\
      4 & grammes & de & vinaigre à l’estragon,                                                           \\
        &         &  4 & jaunes d'œufs,                                                                   \\
        &         &  2 & cervelles de veau,                                                               \\
        &         &    & légumes tournés,                                                                 \\
        &         &    & bouillon,                                                                        \\
        &         &    & sel et poivre.                                                                   \\
\end{longtable}
\normalsize

Épluchez les herbes, lavez-les, blanchissez-les, égouttez-les et pilez-les au
mortier ; ajoutez ensuite l'huile, le vinaigre, le beurre ; pilez encore, puis
passez le tout au tamis de crin. Vous aurez ainsi un beurre de ravigote.

Faites blanchir les cervelles dans de l’eau bouillante salée et vinaigrée, puis
achevez leur cuisson dans du bon bouillon en les laissant mijoter à tout petit feu
pendant à un quart d'heure. Égouttez-les ; tenez-les au chaud.

En même temps, préparez la sauce avec les jaunes d'œufs, l'eau, du sel, du
poivre et le beurre de ravigote, en opérant comme pour une sauce hollandaise
ordinaire, \hyperlink{p0362}{p. \pageref{pg0362}}.

Servez les cervelles entières, très chaudes et masquées par la sauce, sur un
plat garni avec des légumes tournés, tels que pommes de terre, carottes,
navets, etc., cuits à l'anglaise.

\section*{\centering Cervelles de veau en coquilles.}
\phantomsection
\addcontentsline{toc}{section}{ Cervelles de veau en coquilles.}
\index{Cervelles de veau en coquilles}

Faites cuire des cervelles de veau dans de l'eau salée, vinaigrée et aromatisée
avec des légumes et de l'oignon ; passez-les au tamis et tenez-les au chaud.

Préparez une sauce suprême,
pp. \hyperlink{p0419}{\pageref{pg0419}}, \hyperlink{p0510}{\pageref{pg0510}} ; incorporez-y
des queues de crevettes ou d'écrevisses cuites comme il est dit
\hyperlink{p0287}{p. \pageref{pg0287}}, des champignons cuits dans du beurre avec
du jus de citron et émincés, des rondelles de truffes noires cuites au porto,
puis ajoutez la purée de cervelles et disposez cette préparation dans des
coquilles ; saupoudrez avec un peu de parmesan râpé et de mie de pain rassis
tamisée, mettez par-dessus du beurre coupé en petits morceaux et faites
gratiner au four,

Servez avec des tranches de citron.

\sk

Comme variante, on peut remplacer la sauce suprême par une sauce béchamel
grasse, \hyperlink{p0566}{p. \pageref{pg0566}}.

\sk

\index{Cervelle de veau en bouchées}
\index{Cervelle de veau en vol-au-vent}
\index{Coquilles de cervelle de veau}
On peut aussi, avec la préparation ci-dessus, garnir des croûtes de bouchées ou
une croûte de vol-au-vent : on aura ainsi des bouchées ou un vol-au-vent à la
cervelle.

\sk

\index{Cervelle de veau en croquettes}
On peut encore, avec la même préparation, mais en la tenant plus serrée, faire
des croquettes qu'on passera successivement dans de l'œuf battu et dans de la
mie de pain rassis tamisée, et qu'on fera frire à pleine friture.

On servira, après avoir décoré le plat avec du persil frit et des tranches de
citron.

\section*{\centering Fraise de veau au naturel.}
\phantomsection
\addcontentsline{toc}{section}{ Fraise de veau au naturel.}
\index{Fraise de veau au naturel}
\label{pg0425} \hypertarget{p0425}{}

On désigne vulgairement sous le nom de « fraise », chez le veau et l'agneau, la
partie du péritoine appelée mésentère, membrane séreuse qui, par ses replis,
maintient dans leurs positions respectives les différentes parties de
l'intestin, particulièrement l'intestin grêle.

La fraise a une saveur fine, mais elle n'est réellement bonne que lorsqu'elle
est extrêmement fraîche.

Voici la meilleure façon de l’accommoder.

Mettez la fraise à dégorger dans de l’eau froide pendant une heure ;
blanchissez-la ensuite dans de l'eau bouillante pendant un quart d'heure.

\index{Blanc}
Préparez un blanc, c'est-à-dire délayez de la farine dans de l'eau, à raison de
{\ppp20\mmm} grammes de farine par litre d'eau, ajoutez {\ppp6\mmm} grammes de
sel, {\ppp25\mmm} grammes de vinaigre ou {\ppp35\mmm} grammes de jus de citron
par litre d'eau ; faites bouillir ; mettez ensuite oignons, clous de girofle,
bouquet garni et poivre au goût, la fraise, de la graisse de rognon de veau
hachée (qui a surtout pour but d'entraîner à la surface du liquide les
impuretés provenant de la cuisson et de protéger la substance à cuire du
contact de l'air, ce qui lui permet de rester bien blanche) et faites cuire
pendant trois heures environ.

Retirez la fraise, égouttez-la.

Servez-la telle quelle, sur un plat garni de persil frit, avec accompagnement de
l'une quelconque des sauces indiquées pour les issues.

\sk

\index{Agneau (Fraise d')}
\index{Fraise d'agneau au naturel}
On peut préparer de même la tête de veau et la fraise d'agneau.

\newpage
\section*{\centering Rognons de veau sautés, sauce au vin.}
\phantomsection
\addcontentsline{toc}{section}{ Rognons de veau sautés, sauce au vin.}
\index{Rognons de veau sautés, sauce au vin}

Pour quatre personnes prenez :

\medskip

\label{pg0426} \hypertarget{p0426}{}
\footnotesize
\begin{tabular}{@{}lrrrp{18em}}
\normalsize1°\footnotesize & 200 & grammes & de & vin blanc : porto, madère ou champagne,                 \\
   &     &         &    &    par exemple,                                                                 \\
   & 125 & grammes & de & rognons de coq, blancs, fermes, non crevés,                                     \\
   & 100 & grammes & de & beurre,                                                                         \\
   &     &         &  2 & rognons de veau,                                                                \\
   &     &         &  2 & belles truffes,                                                                 \\
   &     &         &    & farine,                                                                         \\
   &     &         &    & jus de citron,                                                                  \\
   &     &         &    & sel et poivre ;                                                                 \\
   &     &         &    &                                                                                 \\
\normalsize 2° & \multicolumn{4}{l}{\normalsize   pour le fond de veau :}                                 \\
\footnotesize
   &     &         &    &                                                                                 \\
   & 500 & grammes & de & jarret de veau,                                                                 \\
   &     &         &  1 & pied de veau,                                                                   \\
   &     &         &  1 & morceau carré de couenne maigre de                                              \\
   &     &         &    &   15 centimètres de côté,                                                       \\
   &     &         &  1 & bouquet garni,                                                                  \\
   &     &         &  1 & clou de girofle,                                                                \\
   &     &         &    & carottes,                                                                       \\
   &     &         &    & oignons.                                                                        \\
   &     &         &    & sel.                                                                            \\
\end{tabular}
\normalsize

\medskip

\index{Fond de veau}
Préparez le fond de veau la veille : mettez dans une casserole pied et jarret
de veau, couenne maigre, bouquet garni, girofle, de l'eau en quantité
suffisante, des carottes, des oignons et du sel, au goût ; faites cuire pendant
cinq heures ; passez le fond.

Le lendemain, réduisez-le de façon à obtenir {\ppp200\mmm} grammes de fond de veau
concentré.

Nettoyez les truffes : pilez-en une avec {\ppp20\mmm} grammes de beurre ; faites cuire
l'autre dans le vin ; coupez-la en rondelles.

Mettez dans le vin un peu de sel et de poivre ; réduisez-le aux trois quarts de
son volume.

Clarifiez le reste du beurre.

Coupez les rognons de veau en deux dans le sens de leur longueur ; enlevez-en
les parties nerveuses ; escalopez-les en tranches minces que vous assaisonnerez
avec sel et poivre, et que vous ferez sauter pendant cinq minutes dans le
beurre clarifié. Égouttez-les dans une passoire ; puis, remettez-les dans la
sauteuse, sans beurre ; saupoudrez-les avec un peu de farine ; faites-les
encore sauter pendant deux minutes ; mouillez ensuite avec le vin réduit ;
laissez cuire pendant une minute. Passez la cuisson. Tenez les rognons au
chaud.

En même temps, faites cuire, sans bouillir, les rognons de coq dans le fond de
veau aromatisé avec du jus de citron ; retirez-les dès qu'ils seront fermes.
Tenez-les au chaud.

Réunissez les cuissons des rognons de veau et des rognons de coq, ajoutez la
truffe pilée avec le beurre ; concentrez la sauce.

Foncez un plat avec cette sauce, disposez dessus les escalopes de rognons de
veau, décorez avec les rognons de coq et les rondelles de truffe. Servez.

Comme boisson, un vieux bourgogne blanc, sec, est de circonstance.

\section*{\centering Rognons de veau sautés, sauce à la crème.}
\phantomsection
\addcontentsline{toc}{section}{ Rognons de veau sautés, sauce à la crème.}
\index{Rognons de veau sautés, sauce à la crème}
\index{Farce pour rognons}

Pour quatre personnes prenez :

\medskip

\footnotesize
\begin{longtable}{rrrp{18em}}
    125 & grammes & de & crème,                                                                           \\
    125 & grammes & de & champignons de couche,                                                           \\
    100 & grammes & de & beurre,                                                                          \\
    100 & grammes & de & fond de veau,                                                                    \\
     30 & grammes & d’ & oignon haché,                                                                    \\
     30 & grammes & de & raifort râpé,                                                                    \\
      5 & grammes & de & persil haché,                                                                    \\
      2 & grammes & de & sel,                                                                             \\
      2 & grammes & de & paprika,                                                                         \\
        &         &  2 & rognons de veau parés, pesant ensemble 500 grammes environ,                      \\
        &         &    & vinaigre.                                                                        \\
\end{longtable}
\normalsize

Faites revenir l'oignon dans {\ppp20\mmm} grammes de beurre sans qu'il prenne couleur :
mouillez avec le fond de veau : laissez cuire,

Coupez les rognons en deux dans le sens de leur longueur, retirez-en les parties
nerveuses, escalopez-les ensuite en tranches minces, assaisonnez-les avec le sel et
le paprika ; faites-les sauter pendant cinq minutes dans le reste du beurre, puis
égouttez-les dans une passoire et tenez-les au chaud.

Épluchez les champignons, émincez-les et faites-les sauter dans le beurre qui a
servi pour les rognons. Réservez et tenez au chaud.

Mettez dans le fond de veau aromatisé le beurre de cuisson des champignons. la
crème additionnée d'un peu de vinaigre, au goût, mélangez, puis ajoutez le
persil, le raifort, les champignons et les rognons ; chauffez ensemble pendant
un instant, sans laisser bouillir.

Servez, en envoyant en même temps un plat de riz sauté au beurre,
\hyperlink{p0710}{p. \pageref{pg0710}}.

Ces rognons à la crème sont excellents : tous les éléments de la sauce sont
parfaitement fondus et aucun ne domine.

\section*{\centering Rognons de veau sautés, sauce à la fine champagne.}
\phantomsection
\addcontentsline{toc}{section}{ Rognons de veau sautés, sauce à la fine champagne.}
\index{Rognons de veau sautés, sauce à la fine champagne}

Pour quatre personnes prenez :

\medskip

\footnotesize
\begin{longtable}{rrrp{18em}}
    250 & grammes & de & morilles,                                                                        \\
    200 & grammes & de & fond de veau,                                                                    \\
    100 & grammes & de & beurre,                                                                          \\
     75 & grammes & de & fine champagne,                                                                  \\
        &         &  4 & tranches de pain,                                                                \\
        &         &  2 & rognons de veau,                                                                 \\
        &         &    & farine,                                                                          \\
        &         &    & sel et poivre.                                                                   \\
\end{longtable}
\normalsize


Nettoyez soigneusement les morilles.

Apprêtez les rognons, escalopez-les.

Préparez un roux avec un peu de farine et {\ppp20\mmm} grammes de beurre,
mouillez avec le fond de veau ; concentrez la sauce.

Faites revenir les morilles dans {\ppp30\mmm} grammes de beurre ; ajoutez-les
à la sauce. Tenez le tout au chaud.

Faites sauter vivement, pendant quelques minutes, les escalopes de rognons dans
{\ppp30\mmm} grammes de beurre bien chaud ; salez, poivrez ; égouttez-les.

Mettez-les ensuite dans une casserole, flambez-les à la fine champagne, chauffée
au préalable, puis ajoutez le ragoût de morilles et le reste du beurre. Chauffez
pendant quelques minutes.

Faites griller les tranches de pain : dressez-les sur un plat ; versez dessus le
contenu de la casserole et servez.

\sk

Comme variantes, on pourra remplacer la fine champagne par du rhum, du
calvados, du gin, etc., et les morilles par des champignons de couche, des
cèpes, des gyroles, des truffes, etc.

\section*{\centering Rognons de veau sautés, sauce indienne}
\phantomsection
\addcontentsline{toc}{section}{ Rognons de veau sautés, sauce indienne}
\index{Rognons de veau sautés, sauce indienne}

Pour six personnes prenez :

\medskip

\footnotesize
\begin{longtable}{rrrrp{18em}}
  & 600 & grammes & de & fond de veau,                                                                    \\
  & 250 & grammes & de & champignons de couche,                                                           \\
  & 200 & grammes & de & crème,                                                                           \\
  & 100 & grammes & de & beurre,                                                                          \\
  &  30 & grammes & de & farine,                                                                          \\
  & \multicolumn{2}{r}{6 décigrammes} & de & curry,                                                       \\
  &     &         &  3 & rognons de veau moyens,                                                          \\
  &     &         &  2 & jaunes d'œufs,                                                                   \\
  &     &         &    & sel et poivre.                                                                   \\
\end{longtable}
\normalsize

Apprêtez les rognons : enlevez la graisse et la peau ; dénervez-les.

Pelez les champignons et faites-les cuire pendant une heure environ dans
{\ppp30\mmm} grammes de beurre et {\ppp50\mmm} grammes de fond de veau : vous
aurez ainsi une essence de champignons ; passez-la.

Faites blondir la farine dans {\ppp40\mmm} grammes de beurre ; mouillez avec le reste du
fond de veau ; laissez cuire pendant une demi-heure ; dépouillez la sauce
pendant la cuisson.

Mettez dans une sauteuse le reste du beurre et les rognons ; laissez-les dorer
de tous côtés pendant {\ppp12\mmm} à {\ppp15\mmm} minutes, en les sautant pour
éviter qu'ils s'attachent ; salez, poivrez.

Réunissez sauce et essence de champignons, ajoutez-y curry, crème et jaunes
d'œufs ; chauffez sans laisser bouillir. Goûtez et complétez l'assaisonnement,
s'il est nécessaire, avec sel et poivre.

Dressez les rognons sur un plat : masquez-les avec la sauce et envoyez en même
temps un légumier de pommes de terre duchesse,
\hyperlink{p0725}{p. \pageref{pg0725}}, ou bien mettez les pommes duchesse dans la
sauce au moment de servir.

\section*{\centering Rognons de veau en cocote.}
\phantomsection
\addcontentsline{toc}{section}{ Rognons de veau en cocote.}
\index{Rognons de veau sautés en cocote}

Pour quatre personnes prenez :

\medskip

\footnotesize
\begin{longtable}{rrrp{18em}}
    500 & grammes & de & légumes variés, tels que pointes d'asperges,
                         petits pois, fonds d'artichauts, petits champignons,
                         petits oignons, etc.                                                             \\
    160 & grammes & de & beurre,                                                                          \\
    100 & grammes & de & fond de veau,                                                                    \\
    100 & grammes & de & vin blanc,                                                                       \\
        &         &  2 & rognons de veau recouverts d'une mince couche de
                         graisse et pesant ensemble 500 grammes environ,                                  \\
        &         & 10 & grains de genièvre,                                                              \\
        &         &    & sel et poivre.                                                                   \\
\end{longtable}
\normalsize

Mettez dans une cocote en porcelaine {\ppp60\mmm} grammes de beurre, chauffez, puis
ajoutez les rognons, le genièvre, du sel, du poivre\footnote{Les proportions de
sel et de poivre dépendent de l’assaisonnement du fond de veau.} : laissez
cuire à très petit feu, en cocote à moitié couverte, pendant trois quarts
d'heure. Retournez les rognons de temps en temps.

Faites blanchir dans de l'eau salée bouillante les pointes d'asperges, les petits
pois et les fonds d'artichauts. Égouttez-les.

Faites blondir légèrement les oignons dans le reste du beurre, ajoutez ensuite
les champignons et les légumes blanchis ; achevez leur cuisson ; tenez-les au
chaud.

Dix minutes avant la fin, dégraissez et passez la cuisson des rognons, mettez-la
dans une casserole avec le fond de veau et le vin ; réduisez la sauce jusqu'à
consistance convenable ; goûtez et complétez l'assaisonnement, s'il y a lieu.

Coupez chaque rognon en deux dans le sens de leur longueur, remettez-les dans
la cocote, disposez les légumes autour comme garniture, versez dessus la sauce
et servez.

\section*{\centering Rognons de veau panés, aux cèpes, sauce
Colbert\footnote{La sauce Colbert est un composé de glace de viande et de
beurre manié de persil, aromatisé au goût.}.}

\phantomsection \addcontentsline{toc}{section}{ Rognons de veau panés, aux cèpes, sauce Colbert.}
\index{Rognons de veau panés, aux cèpes, sauce Colbert}

Voici deux formules de rognons de veau panés, aux cèpes, l'une avec des cèpes
frais grillés, l'autre avec un ragoût de cèpes secs.

\medskip

\begin{center}
\textit{Première formule.}
\end{center}

\medskip

Pour quatre personnes prenez :

\medskip

\footnotesize
\begin{longtable}{rrrp{18em}}
    225 & grammes & de & beurre,                                                                          \\
     30 & grammes & de & glace de viande,                                                                 \\
     20 & grammes & de & farine,                                                                          \\
     15 & grammes & de & persil haché,                                                                    \\
        &         &  4 & beaux cèpes frais,                                                               \\
        &         &  4 & gousses d'ail,                                                                   \\
        &         &  2 & beaux rognons de veau,                                                           \\
        &         &  1 & œuf frais,                                                                       \\
        &         &    & chapelure,                                                                       \\
        &         &    & madère,                                                                          \\
        &         &    & jus de citron,                                                                   \\
        &         &    & muscade,                                                                         \\
        &         &    & sel et poivre.                                                                   \\
\end{longtable}
\normalsize

Clarifiez {\ppp100\mmm} grammes de beurre : mettez le beurre dans une casserole,
laissez-le fondre, écumez-le et filtrez-le.

Coupez chaque gousse d'ail en quatre ; piquez chaque chapeau de cèpe de quatre
morceaux d'ail, puis mettez-les à mariner dans le beurre clarifié tenu tiède.

Parez les rognons, coupez-les en tranches d'un centimètre d'épaisseur environ,
passez-les d'abord dans la farine, ensuite dans de l'œuf battu, enfin dans de
la chapelure.

En même temps :

d'une part, faites sauter les rognons pendant {\ppp3\mmm} à {\ppp4\mmm} minutes
dans {\ppp30\mmm} grammes de beurre ; salez et poivrez pendant la cuisson ;

d'autre part, mettez sur un gril, chauffé au préalable, les chapeaux des cèpes
et faites-les griller, à feu dessus, pendant {\ppp5\mmm} minutes de chaque
côté ; pendant l'opération, salez, poivrez et arrosez avec le beurre de la
marinade ; saupoudrez avec {\ppp5\mmm} grammes de persil haché ; enlevez
ensuite les morceaux d'ail.

Hachez les pieds des cèpes et faites-les cuire dans {\ppp20\mmm} grammes de
beurre.

Tenez rognons et cèpes au chaud.

Préparez la sauce : maniez le reste du beurre avec le reste du persil. Faites
fondre la glace de viande, ajoutez-y, par petites quantités, le beurre manié de
persil et du jus de citron, du sel, du poivre et de la muscade ; chauffez sans
laisser bouillir. Au dernier moment, mettez un peu de madère pour aromatiser.

Foncez un plat chaud avec le hachis de cèpes, disposez dessus les émincés de
rognons et les chapeaux de cèpes, masquez le tout avec la sauce. Servez aussitôt.

\medskip

\begin{center}
\textit{Deuxième formule.}
\end{center}

\medskip

Pour quatre personnes prenez :

\medskip

\footnotesize
\begin{longtable}{rrrp{18em}}
    200 & grammes & de & fond de veau,                                                                   \\
    135 & grammes & de & beurre,                                                                         \\
    100 & grammes & de & cèpes secs,                                                                     \\
     30 & grammes & de & glace de viande,                                                                \\
     30 & grammes & de & farine,                                                                         \\
     10 & grammes & de & persil haché,                                                                   \\
        &         &  2 & beaux rognons de veau,                                                          \\
        &         &  1 & œuf frais,                                                                      \\
        &         &    & chapelure,                                                                      \\
        &         &    & madère,                                                                         \\
        &         &    & jus de citron,                                                                  \\
        &         &    & muscade,                                                                        \\
        &         &    & cayenne,                                                                        \\
        &         &    & sel et poivre.                                                                  \\
\end{longtable}
\normalsize

Mettez la veille les cèpes à tremper dans de l’eau froide. Essuyez-les,
émincez-les, faites-les sauter dans {\ppp30\mmm} grammes de beurre, saupoudrez-les de 10
grammes de farine, mouillez avec le fond de veau, assaisonnez avec sel, poivre
et cayenne ; laissez cuire de façon à avoir un ragoût de consistance serrée.

Préparez les rognons et la sauce comme précédemment.

Disposez sur un plat chaud le ragoût de cèpes et les émincés de rognons,
masquez avec la sauce et servez,

\section*{\centering Rognons de veau farcis.}
\phantomsection
\addcontentsline{toc}{section}{ Rognons de veau farcis.}
\index{Rognons de veau farcis}

Pour quatre personnes prenez :

\medskip

\footnotesize
\begin{longtable}{rrrp{18em}}
    150 & grammes & de & fond de veau concentré,                                                          \\
    125 & grammes & de & bacon,                                                                           \\
    125 & grammes & de & champignons de couche, de cèpes, de morilles ou de truffes, au goût,             \\
     80 & grammes & de & beurre,                                                                          \\
     50 & grammes & de & madère, porto, xérès ou malvoisie, au choix,                                     \\
      5 & grammes & de & persil,                                                                          \\
        &         &  4 & rognons de veau,                                                                 \\
        &         &  1 & cervelle de mouton,                                                              \\
        &         &  1 & œuf frais,                                                                       \\
        &         &    & muscade,                                                                         \\
        &         &    & sel et poivre.                                                                   \\
\end{longtable}
\normalsize

Hachez séparément le bacon, les champignons, la cervelle et le persil.

Faites revenir le bacon dans {\ppp40\mmm} grammes de beurre, ajoutez ensuite les
champignons, laissez-les dorer légèrement, puis mettez la cervelle et le
persil ; chauffez pendant quelques instants ; éloignez ensuite la casserole du
feu et liez le tout avec l'œuf ; salez, poivrez, relevez avec un peu de
muscade ; mélangez bien de façon à obtenir une farce homogène.

Ouvrez les rognons, dénervez-les, étalez-les, garnissez la face interne de deux
rognons avec la farce, couvrez les rognons garnis avec ceux qui ne le sont pas ;
ficelez.

Faites revenir les rognons de tous côtés dans le reste du beurre pendant un
quart d'heure et à feu assez vif ; mouillez avec le fond de veau et le vin, puis
finissez la cuisson au four doux pendant une demi-heure, en arrosant les rognons
fréquemment. Concentrez la sauce.

Enlevez les ficelles ; dressez les rognons sur un plat, masquez-les avec la
sauce, entourez-les avec des pommes de terre Chip et servez.

\sk

Comme variante, on pourra barder les rognons de lard et les faire rôtir en les
arrosant pendant leur cuisson avec du fond de veau additionné de vin. On
retirera le reste des bardes avant de servir.

\section*{\centering Oreilles de veau farcies, sauce béarnaise\footnote{La
sauce béarnaise est une sauce hollandaise relevée par une réduction de vinaigre
et d'échalotes, et aromatisée avec de l'estragon.}.}

\phantomsection
\addcontentsline{toc}{section}{ Oreilles de veau farcies, sauce béarnaise.}
\index{Oreilles de veau farcies, sauce béarnaise}
\label{pg0433} \hypertarget{p0433}{}

Pour six personnes prenez :

\medskip

\footnotesize
\begin{tabular}{@{}lrrrp{18em}}
\normalsize1°\footnotesize & 200 & grammes & de & crème,                                                  \\
  & 185 & grammes & de & beurre,                                                                          \\
  & 125 & grammes & de & truffes cuites dans du madère,                                                   \\
  & 120 & grammes & de & blanc de poulet rôti,                                                            \\
  & 120 & grammes & de & ris de veau cuit au beurre,                                                      \\
  &  30 & grammes & de & farine,                                                                          \\
  &  30 & grammes & de & fond de veau,                                                                    \\
  &   4 & grammes & de & poivre fraîchement moulu,                                                        \\
  &     &         &  6 & oreilles de veau,                                                                \\
  &     &         &  6 & écrevisses cuites,                                                               \\
  &     &         &  3 & œufs frais,                                                                      \\
  &     &         &  1 & carotte,                                                                         \\
  &     &         &  1 & oignon,                                                                          \\
  &     &         &    & vin blanc sec,                                                                   \\
  &     &         &    & bouillon,                                                                        \\
  &     &         &    & mie de pain rassis tamisée,                                                      \\
  &     &         &    & bouquet garni,                                                                   \\
  &     &         &    & citron,                                                                          \\
  &     &         &    & sel et poivre ;                                                                  \\
  &     &         &    &                                                                                  \\
\normalsize 2° & \multicolumn{4}{l}{\normalsize   pour la sauce :}                                        \\
\footnotesize
  &     &         &    &                                                                                  \\
  & 250 & grammes & de & beurre,                                                                          \\
  & 100 & grammes & de & vinaigre de vin de force moyenne,                                                \\
  &  30 & grammes & d’ & eau froide,                                                                      \\
  &     &         &  4 & jaunes d'œufs frais,                                                             \\
  &     &         &  2 & échalotes,                                                                       \\
  &     &         &    & estragon,                                                                        \\
  &     &         &    & sel et poivre.                                                                   \\
\end{tabular}
\normalsize

\medskip

Les oreilles doivent être préparées quelques heures à l'avance.

Faites blanchir les oreilles dans de l’eau salée, lavez-les ensuite à l'eau
froide, frottez-les avec du citron, puis enveloppez chaque oreille à part dans
un linge blanc que vous coudrez.

\index{Court-bouillon pour issues}
Préparez un court-bouillon avec du vin blanc et du bouillon en parties égales,
l'oignon, la carotte coupée en morceaux, le bouquet garni, du sel et du poivre.

Mettez les oreilles dans ce court-bouillon ; laissez-les cuire pendant trois
heures en ajoutant, au fur et à mesure de l'évaporation, quantité équivalente
du mélange vin blanc et bouillon, de façon que les oreilles baignent
constamment dans le liquide, puis sortez-les du court-bouillon et enlevez les
linges.

\index{Farce pour oreilles de veau}
Pendant la cuisson des oreilles, préparez la farce.

Coupez en petits morceaux le blanc de poulet, le ris de veau et les queues
d'écrevisses ; mélangez bien le tout.

Préparez un beurre d'écrevisses avec {\ppp75\mmm} grammes de beurre et les parures.

Faites blondir la farine dans {\ppp50\mmm} grammes de beurre, mouillez avec le fond de
veau, mettez une partie de la crème, le beurre d'écrevisses, les truffes
coupées en tranches, le poivre et du sel au goût, ajoutez le mélange de blanc
de poulet, ris de veau et queues d’écrevisses ; laissez cuire le tout pendant
un moment.

Retirez la préparation du feu, liez avec le reste de la crème et deux jaunes
d'œufs ; laissez refroidir.

Lorsque la farce sera froide, garnissez-en les pavillons des oreilles.

Battez en neige le dernier œuf entier et les deux blancs ; passez dedans les
oreilles ; roulez-les ensuite dans de la mie de pain tamisée, le tout à deux
reprises, de façon à les bien paner.

\label{pg0434} \hypertarget{p0434}{} 
Préparez la sauce béarnaise de la façon suivante : avec
le beurre, les jaunes d'œufs, l'eau froide, du sel et du poivre faites une
sauce hollandaise, comme il est dit \hyperlink{p0362}{p. \pageref{pg0362}},
à laquelle vous incorporerez de l'estragon haché, au goût, et une réduction
faite avec le vinaigre et les échalotes, que vous passerez.

Un quart d'heure avant de servir, faites dorer au four, dans le reste du
beurre, les oreilles farcies et panées ; égouttez-les et servez-les, présentées
debout sur un plat. Envoyez en même temps la sauce dans une saucière.

\section*{\centering Foie de veau au naturel.}
\phantomsection
\addcontentsline{toc}{section}{ Foie de veau au naturel.}
\index{Foie de veau au naturel}

Le foie de veau, au naturel, peut être grillé ou sauté.

Pour le griller : mettez sur un gril, chauffé au préalable et beurré ou
graissé, autant de tranches de foie qu'il y a de convives ; poussez la cuisson
d'autant plus rapidement que les tranches sont plus minces, de manière que
l'intérieur reste rosé ; assaisonnez avec sel et poivre.

Dressez les tranches de foie sur un plat chaud : versez dessus du beurre maître
d'hôtel et servez aussitôt.

Pour le sauter : enrobez les tranches de foie dans de la farine. Faites fondre
du beurre ou de la graisse de rôti dans une poêle et saisissez dedans les
tranches de foie, d'autant plus rapidement qu'elles sont plus minces.
Sortez-les, égouttez-les, assaisonnez-les avec sel et poivre.

Mettez-les sur un plat chaud, arrosez-les avec du beurre maître d'hôtel et
servez immédiatement.

\sk

Comme variante, on peut remplacer le beurre maître d'hôtel par un beurre Bercy,
\hyperlink{p0453}{p. \pageref{pg0453}}, ou une sauce italienne aux fines herbes,
\hyperlink{p0490}{p. \pageref{pg0490}}.

\sk

Les légumes qui accompagnent le mieux le foie de veau au naturel sont les
pommes de terre et les pâtes, notamment les pommes de terre sautées et les
nouilles sautées panées, \hyperlink{p0681}{p. \pageref{pg0681}}.

\sk

Les plats de foie de veau au naturel garnis ou non ne doivent attendre sous
aucun prétexte.

\sk

On peut apprêter de même le rognon de veau.

\section*{\centering Foie de veau braisé.}
\phantomsection
\addcontentsline{toc}{section}{ Foie de veau braisé.}
\index{Foie de veau braisé}

Pour six personnes prenez :

\medskip

\footnotesize
\begin{longtable}{rrrp{18em}}
    250 & grammes & de & fond de veau,                                                                    \\
    250 & grammes & de & lard de poitrine,                                                                \\
    250 & grammes & de & lard à piquer,                                                                   \\
    100 & grammes & de & vin blanc sec,                                                                   \\
    100 & grammes & de & madère,                                                                          \\
     50 & grammes & de &  glace de viande,                                                                \\
        &         &  1 & foie de veau,                                                                    \\
        &         &    & truffes à volonté,                                                               \\
        &         &    & crépine,                                                                         \\
        &         &    & beurre,                                                                          \\
        &         &    & sel et poivre.                                                                   \\
\end{longtable}
\normalsize

Coupez en lardons le lard à piquer ; assaisonnez-le.

Coupez en petits morceaux le lard de poitrine.

Piquez le foie avec les lardons assaisonnés ; enveloppez-le dans la crépine.

Faites revenir le lard de poitrine avec un peu de beurre dans une cocote en
porcelaine allant au feu ; mettez ensuite le foie enveloppé ; mouillez avec le
fond de veau, le vin blanc et le madère, ajoutez la glace de viande, les
truffes, couvrez et faites cuire au four doux pendant deux heures.

Retirez le foie ; enlevez ce qui reste de crépine ; dégraissez la sauce ;
complétez l'assaisonnement s'il est nécessaire,

Mettez le foie sur un plat de service, versez dessus la sauce et servez, en
envoyant en même temps un légumier de riz sauté,
\hyperlink{p0710}{p. \pageref{pg0710}}.

\section*{\centering Foie de veau sauté, au bacon.}
\phantomsection
\addcontentsline{toc}{section}{ Foie de veau sauté, au bacon.}
\index{Foie de veau sauté, au bacon}

Pour quatre personnes prenez :

\medskip

\footnotesize
\begin{longtable}{rrrp{18em}}
    800 & grammes & de & foie de veau, en quatre tranches semblables,                                     \\
    250 & grammes & de & bacon,                                                                           \\
     50 & grammes & de & beurre,                                                                          \\
        &         &    & persil,                                                                          \\
        &         &    & sel et poivre.                                                                   \\
\end{longtable}
\normalsize

Émincez le bacon et faites-le revenir dans une sauteuse avec {\ppp10\mmm}
grammes de beurre, pendant quelques minutes.

Enlevez-le, tenez-le au chaud. Remplacez-le dans la sauteuse par les tranches de
foie que vous ferez sauter sans aucun assaisonnement, pendant quelques minutes
de chaque côté.

En même temps, préparez un beurre manié avec le reste du beurre, du persil, du
sel et du poivre. Chauffez.

Dressez rapidement les tranches de foie sur un plat de service chaud, décorez
avec les émincés de bacon, masquez le tout avec le beurre manié fondu.

Servez aussitôt, en envoyant en même temps des pommes paille,
\hyperlink{p0714}{p. \pageref{pg0714}}, ou des pommes Chip,
\hyperlink{p0715-2}{p. \pageref{pg0715-2}}.

\section*{\centering Foie de veau en crépinettes.}
\phantomsection
\addcontentsline{toc}{section}{ Foie de veau en crépinettes.}
\index{Foie de veau sauté en crépinettes}
\index{Crépinattes de foie de veau}
\index{Foie de veau en crépinettes}

Pour six personnes prenez :

\medskip

\footnotesize
\begin{longtable}{rrrp{18em}}
    450 & grammes & de & foie de veau paré et coupé en trois tranches semblables,                         \\
    150 & grammes & de & veau maigre, sans os,                                                            \\
    150 & grammes & de & jambon cru, entrelardé, sans os,                                                 \\
    125 & grammes & de & champignons de couche,                                                           \\
     60 & grammes & de & beurre,                                                                          \\
     10 & grammes & de & glace de viande,                                                                 \\
     10 & grammes & d' & échalotes,                                                                       \\
     10 & grammes & de & sel blanc,                                                                       \\
      2 & grammes & de & persil,                                                                          \\
      2 & grammes & de & poivre fraîchement moulu,                                                        \\
    1/2 &  gramme & de & quatre épices,                                                                   \\
        &         &  2 & jaunes d'œufs,                                                                   \\
        &         &    & jus de citron,                                                                   \\
        &         &    & crépine de porc.                                                                 \\
\end{longtable}
\normalsize

Pelez les champignons ; passez-les dans du jus de citron.

Hachez ensemble veau, jambon, champignons, échalotes, persil ; assaisonnez avec
sel, poivre, quatre épices ; liez le hachis avec les jaunes d'œufs ; mélangez
bien. Enduisez avec le mélange les deux côtés des tranches de foie de veau ;
enveloppez isolément chaque tranche ainsi apprêtée dans de la crépine de porc.

Faites cuire à petit feu, en casserole couverte, les crépinettes dans le
beurre, pendant trois quarts d'heure, puis dégraissez le jus de cuisson,
corsez-le avec la glace de viande, goûtez pour l'assaisonnement et complétez-le
s'il y a lieu.

Servez, en envoyant en même temps un légumier de petits pois.

\sk

\index{Crépinettes de cervelle de veau}
\index{Cervelle de veau en crépinettes}
On peut préparer la cervelle de veau d'une façon analogue : on la servira sur
une purée de chicorée à la crème que l'on apprêtera comme la purée d'épinards à
la crème, \hyperlink{p0745}{p. \pageref{pg0745}}.

\section*{\centering Foie de veau en aspic.}
\phantomsection
\addcontentsline{toc}{section}{ Foie de veau en aspic.}
\index{Foie de veau sauté en aspic}
\index{Foie de veau en aspic}
\index{Aspic de foie de veau}

Pour six personnes prenez :

\medskip

\footnotesize
\begin{longtable}{rrrp{18em}}
  1 000 & grammes & de & foie de veau,                                                                    \\
    150 & grammes & de & vin blanc de Bourgogne,                                                          \\
    150 & grammes & de & madère,                                                                          \\
      5 & grammes & de & sel,                                                                             \\
      2 & grammes & de & poivre,                                                                          \\
      2 & grammes & de & quatre épices,                                                                   \\
        & 1 litre & de & bouillon corsé, auquel on aura ajouté du jarret de veau,                         \\
        &         &  1 & bouquet garni,                                                                   \\
        &         &  1 & barde de lard.                                                                   \\
\end{longtable}
\normalsize

Assaisonnez le foie avec le sel, le poivre et les quatre épices ; mettez-le
à mariner pendant deux heures dans le madère ; bardez-le ensuite.

Versez dans une cocote en porcelaine allant au feu le bouillon, le vin blanc et
le madère de la marinade ; ajoutez le bouquet ; faites bouillir de façon
à réduire le volume du liquide à un litre.

Plongez le foie bardé dans le liquide bouillant ; mettez le tout au four
pendant {\ppp20\mmm} à {\ppp30\mmm} minutes, suivant que vous aimez le foie plus ou moins cuit ;
laissez refroidir un peu le foie dans sa cuisson afin d'éviter qu'il se
racornisse.

Dégraissez, concentrez et clarifiez le jus de cuisson.

Coulez au fond d'un moule une partie de ce jus ; laissez-le prendre en gelée ;
mettez ensuite le foie, versez par-dessus le reste du jus ; laissez bien
refroidir.

Démoulez et servez.

Le foie de veau ainsi préparé n'a certes pas la prétention de rivaliser avec le
foie gras d'oie en aspic, \hyperlink{p0596}{p. \pageref{pg0596}} : mais il peut
accompagner honorablement une salade dans un repas de famille.

\section*{\centering Ragoût d'issues de veau.}
\phantomsection
\addcontentsline{toc}{section}{ Ragoût d'issues de veau.}
\index{Ragoût d'issues de veau}

Pour huit personnes prenez :

\medskip

\footnotesize
\begin{longtable}{rrrp{18em}}
  1 000 & grammes & de & tête de veau avec oreille,                                                       \\
    500 & grammes & de & foie de veau,                                                                    \\
    500 & grammes & d’ & amourettes (moelle épinière du veau),                                            \\
    250 & grammes & de & vin blanc sec,                                                                   \\
        &         &  2 & rognons de veau,                                                                 \\
        &         &  1 & beau ris de veau,                                                                \\
        &         &  1 & langue de veau,                                                                  \\
        &         &  1 & cervelle de veau,                                                                \\
        &         &    & bouillon de veau très parfumé,                                                   \\
        &         &    & carottes,                                                                        \\
        &         &    & oignons,                                                                         \\
        &         &    & tomates,                                                                         \\
        &         &    & olives verdales,                                                                 \\
        &         &    & beurre,                                                                          \\
        &         &    & farine,                                                                          \\
        &         &    & jus de citron,                                                                   \\
        &         &    & sel, poivre, épices.                                                             \\
\end{longtable}
\normalsize

Faites blanchir la tête, la langue, la cervelle, le ris et les amourettes dans
le bouillon de veau.

Faites revenir ensemble dans du beurre le foie et les rognons après les avoir
émincés ; retirez-les ; tenez-les au chaud.

Mettez dans le même beurre quelques petits oignons entiers et quelques carottes
coupées en rouelles minces ; saupoudrez de farine ; laissez pincer ; déglacez
avec le vin ; puis versez le tout dans une marmite en porcelaine allant au
feu : ajoutez d'abord la tête de veau coupée en cubes de {\ppp4\mmm} centimètres de côté,
l'oreille émincée en languettes, la langue coupée en tranches d'un
demi-centimètre d'épaisseur, le foie et, au-dessus, les amourettes tronconnées
en morceaux de {\ppp2\mmm} centimètres de longueur, les rognons, la cervelle et le ris
escalopés, plus ou moins de tomate pelée, concassée et épépinée, assaisonnez
avec sel, poivre et épices, mouillez avec {\ppp250\mmm} grammes environ de bouillon de
cuisson, mettez quelques olives dont vous aurez enlevé les noyaux, couvrez et
faites mijoter au four doux pendant {\ppp4\mmm} heures.

Dégraissez à fond, goûtez et complétez l'assaisonnement avec un peu de jus de
citron.

Servez, en envoyant en même temps soit des pâtes à la poche, soit une purée de
pommes de terre.

\section*{\centering Pieds de veau, sauce au safran.}
\phantomsection
\addcontentsline{toc}{section}{ Pieds de veau, sauce au safran.}
\index{Pieds de veau, sauce au safran}

Faites cuire des pieds de veau dans un court-bowillon au vin blanc, relevé,
additionné de légumes et aromatisé avec du safran, au goût. Retirez-les, désossez-
les, coupez-les en morceaux ; tenez-les au chaud.

Passez le court-bouillon à la passoire fine, concentrez-le, ajoutez-y ensuite
des jaunes d'œufs durs écrasés, des câpres. des cornichons hachés, un peu de
sucre, au goût ; donnez un bouillon. puis acidulez avec du jus de citron.

Masquez les pieds avec cette sauce et servez.

\section*{\centering Friture mélangée grasse.}
\phantomsection
\addcontentsline{toc}{section}{ Friture mélangée grasse.}
\index{Friture mélangée grasse}
\index{Fritto misto}

Comme la friture mélangée maigre, \hyperlink{p0315}{p. \pageref{pg0315}}, ce
« fritto misto » de viandes blanches, d'issues et de légumes est un plat de la
cuisine italienne.

\medskip

Pour quatre personnes prenez :

\footnotesize
\begin{longtable}{rrrp{18em}}
    500 & grammes & de & chou-fleur, fonds d'artichauts, aubergines et
                         courgettes en parties égales, le tout blanchi,                                   \\
    500 & grammes & de & sauce tomate,                                                                    \\
    125 & grammes & de & blanc de poulet rôti, émincé en languettes,                                      \\
    125 & grammes & de & ris de veau blanchi, coupé en morceaux,                                          \\
    125 & grammes & de & foie de veau sauté, émincé en petites tranches,                                  \\
    125 & grammes & de & moelle de bœuf blanchie, coupée en rondelles,                                    \\
        &         &  2 & citrons,                                                                         \\
        &         &    & pâte à frire,                                                                    \\
        &         &    & persil.                                                                          \\
\end{longtable}
\normalsize

Enrobez séparément les différents éléments dans de la pâte à frire. Faites-les
cuire dans de la friture chaude composée d'un mélange de graisses de porc et de
rognon de veau, en parties égales.

Disposez-les sur un plat garni d’une serviette et décorez avec un cordon de
persil frit et les citrons coupés par la moitié.

Servez, en envoyant en même temps la sauce tomate dans une saucière.

\section*{\centering Langues de mouton braisées.}
\phantomsection
\addcontentsline{toc}{section}{ Langues de mouton braisées.}
\index{Langues de mouton braisées}
\index{Agneau (Langues d')}

Prenez autant de langues qu'il y a de convives. Parez-les, mettez-les à dégorger
pendant une heure dans de l'eau tiède, puis faites-les blanchir pendant {\ppp2\mmm} à
{\ppp3\mmm} minutes dans de l’eau salée bouillante. Dépouillez-les.

Faites-les dorer dans du beurre, flambez-les à la fine champagne et laissez
tomber à glace. Couvrez les langues avec du fond de veau, puis faites-les braiser
en four doux pendant deux heures à deux heures et demie.

Retirez-les de la braisière ; tenez-les au chaud,

Corsez le jus de cuisson avec de la glace de viande, montez-le au fouet avec du
beurre que vous incorporerez par petites quantités ; goûtez et complétez
l'assaisonnement, sil y a lieu.

Dressez les langues sur un plat de purée de pois frais, par exemple, et servez,
en envoyant en même temps la sauce dans une saucière.

\sk

\index{Garnitures pour langues braisées}
On pourra préparer de même des langues d'agneau, de veau, de bœuf, de porc,
avec des garnitures variées, telles que des purées de céleri, de pommes de terre,
de marrons. de lentilles au lard, du riz, etc.

\section*{\centering Rognons de mouton grillés.}
\phantomsection
\addcontentsline{toc}{section}{ Rognons de mouton grillés.}
\index{Rognons de mouton grillés}

Pour quatre personnes prenez :

\medskip

\footnotesize
\begin{longtable}{rrrp{18em}}
    125 & grammes & de & beurre,                                                                          \\
        &         &  8 & rognons de mouton,                                                               \\
        &         &    & huile d'olive,                                                                   \\
        &         &    & persil haché,                                                                    \\
        &         &    & jus de citron,                                                                   \\
        &         &    & sel et poivre.                                                                   \\
\end{longtable}
\normalsize

Parez les rognons, c’est-à-dire enlevez la pellicule qui les enveloppe et
supprimez les filaments.

Fendez les rognons par le milieu et aux trois quarts de leur épaisseur dans le
sens de leur grand axe ; enfilez-les, ouverts, deux par deux, sur des
brochettes. du côté de la charnière qui maintient ensemble les deux moitiés ;
assaisonnez-les avec sel et poivre, huilez-les légèrement, puis faites-les
griller sur feu vif pendant trois à trois minutes et demie de chaque côté.

En même temps, préparez un beurre maître d'hôtel avec le beurre, du persil
haché, du jus de citron, du sel et du poivre.

Disposez les brochettes de rognons sur un plat ; arrosez-les avec le beurre
maître d'hôtel, entourez-les avec des pommes paille et servez.

\sk

Comme variantes, on peut remplacer le beurre maître d'hôtel par d'autres
beurres composés : beurre d'ail, d'échalote ou d'estragon ; beurre de
ravigote ; beurre marchand de vin ; beurre Colbert ; beurre Bercy, etc.

\sk

On peut faire griller de la même manière des rognons de veau, mais cela
demandera un peu plus de temps : cinq à six minutes de chaque côté.

\section*{\centering Rognons de mouton, sauce au citron et à la
moutarde\footnote{ Les proportions de jus de citron et de moutarde varient avec
la nature des matières employées. Pour fixer les idées, j'indiquerai comme
proportions le jus de deux citrons moyens et deux cuillerées à café de moutarde
à l'estragon.}.}
\phantomsection
\addcontentsline{toc}{section}{ Rognons de mouton, sauce au citron et à la moutarde.}
\index{Rognons de mouton, sauce au citron et à la moutarde}

Pour quatre personnes prenez :

\medskip

\footnotesize
\begin{longtable}{rrrp{18em}}
     80 & grammes & de & beurre,                                                                          \\
        &         &  8 & rognons de mouton,                                                               \\
        &         &    & jus de citron,                                                                   \\
        &         &    & moutarde,                                                                        \\
        &         &    & sel et poivre.                                                                   \\
\end{longtable}
\normalsize

Parez les rognons : coupez-les en deux.

Mettez dans une casserole {\ppp40\mmm} grammes de beurre, chauffez, ajoutez les rognons,
laissez-les cuire pendant deux minutes de chaque côté ; salez, poivrez ; puis
retirez les rognons. Tenez-les au chaud.

Déglacez la casserole avec du jus de citron, ajoutez de la moutarde, du sel et
du poivre ; mélangez.

Montez la sauce avec le reste du beurre, versez-la sur les rognons et servez
avec accompagnement de pommes de terre sautées, dans un légumier.

\sk

La sauce au citron et à la moutarde convient très bien aussi aux rognons de
veau et aux ris de veau, ainsi qu'aux viandes blanches cuites au beurre.

\section*{\centering Brochettes de rognons de mouton et de foie de veau.}
\phantomsection
\addcontentsline{toc}{section}{ Brochettes de rognons de mouton et de foie de veau.}
\index{Brochettes de rognons de mouton et de foie de veau}

Coupez des rognons de mouton parés et du foie de veau en morceaux de
{\ppp4\mmm} à {\ppp6\mmm} centimètres de côté et de {\ppp1\mmm} centimètre d'épaisseur environ ; enfilez-les
sur des brochettes en les alternant et en les séparant les unes des autres par
des tranches de lard maigre et de lard gras de même surface, mais de quelques
millimètres d'épaisseur seulement ; assaisonnez avec sel et poivre.

Passez les brochettes, ainsi apprêtées, d'abord dans du beurre fondu contenant
du persil haché, puis dans de la mie de pain rassis tamisée.

Faites griller les brochettes, pendant un quart d'heure environ, à la braise ou
au charbon de bois, feu dessus. Arrosez-les de beurre fondu aromatisé avec des
fines herbes et relevé par du jus de citron.

Servez aussitôt.

C'est un excellent plat de déjeuner.

\sk

\index{Brochettes de rognons de veau et de foies de volaille}
Comme variante. on pourra faire de la même manière des brochettes de rognon
de veau et de foies de volaille.

\section*{\centering Ragoût de rognons de mouton et de jambon.}
\phantomsection
\addcontentsline{toc}{section}{ Ragoût de rognons de mouton et de jambon.}
\index{Ragoût de rognons de mouton et de jambon}

Pour quatre personnes prenez :

\medskip

\footnotesize
\begin{longtable}{rrrp{18em}}
    300 & grammes & de & jambon de Bayonne cru,                                                           \\
    300 & grammes & de & fond de veau,                                                                    \\
    100 & grammes & de & beurre,                                                                          \\
     80 & grammes & de & porto,                                                                           \\
     20 & grammes & de & farine,                                                                          \\
     20 & grammes & d' & échalotes,                                                                       \\
        &         &  6 & rognons de mouton,                                                               \\
        &         &    & sel et poivre,                                                                   \\
        &         &    & cayenne.                                                                         \\
\end{longtable}
\normalsize

Faites revenir le jambon, coupé en dés, dans {\ppp30\mmm} grammes de beurre ;
retirez-le ; remplacez-le par la farine et les échalotes hachées
grossièrement ; laissez roussir, puis mouillez avec le fond de veau et le
porto ; remettez le jambon et continuez la cuisson pendant trois quarts d'heure
environ, de manière à amener la sauce à bonne consistance.

Apprêtez les rognons : coupez-les en quatre et faites-les sauter vivement dans
{\ppp40\mmm} grammes de beurre. Assaisonnez avec sel, poivre et cayenne.

Égouttez-les, mettez-les dans la sauce, ajoutez le reste du beurre coupé en
petits morceaux ; chauffez, goûtez et complétez l’assaisonnement s'il y a lieu.

Servez en envoyant en même temps un légumier de riz sauté au beurre,
\hyperlink{p0710}{p. \pageref{pg0710}}.

\sk

Ce mode de préparation peut être appliqué au rognon de veau.

\section*{\centering Paquets marseillais.}
\phantomsection
\addcontentsline{toc}{section}{ Paquets marseillais.}
\index{Paquets marseillais}

En Provence, on désigne sous le nom de paquets marseillais des abats de mouton
roulés que l'on fait cuire dans du bouillon plus ou moins additionné de vin.

En voici une excellente formule qui ne diffère essentiellement de la formule
primitive que par l'addition de tomates et de fine champagne et par l'emploi d'un
fond de veau en place de bouillon.

\medskip

Pour douze personnes prenez :

\medskip

\footnotesize
\begin{longtable}{rrrp{18em}}
  1 500 & grammes & de & fond de veau,                                                                    \\
  1 000 & grammes & de & vin blanc sec,                                                                   \\
    125 & grammes & de & fine champagne,                                                                  \\
    125 & grammes & de & jambon ou de petit salé,                                                         \\
    125 & grammes & de & lard gras,                                                                       \\
     50 & grammes & de & beurre,                                                                          \\
        &         & 12 & pieds de mouton,                                                                 \\
        &         &  4 & tomates,                                                                         \\
        &         &  3 & oignons,                                                                         \\
        &         &  2 & tripes de mouton, comprenant le gras-double, la fraise et les boyaux gras,       \\
        &         &  2 & carottes,                                                                        \\
        &         &  1 & poireau,                                                                         \\
        &         &  1 & bouquet garni,                                                                   \\
        &         &    & fines herbes,                                                                    \\
        &         &    & ail,                                                                             \\
        &         &    & girofle en poudre,                                                               \\
        &         &    & sel et poivre.                                                                   \\
\end{longtable}
\normalsize

Nettoyez soigneusement les tripes et les pieds.

Faites blanchir séparément dans de l'eau bouillante :

d'une part, les pieds de mouton ; désossez-les ;

d'autre part, les tripes : coupez le gras-double en morceaux carrés de
{\ppp8\mmm} centimètres de côté environ ; réservez la fraise et les boyaux gras.

Hachez grossièrement le jambon ou le petit salé, la fraise et une partie des
boyaux gras, assaisonnez au goût avec ail et fines herbes hachées, sel et
poivre. Mélangez.

Disposez ce hachis sur les carrés de gras-double ; roulez en paupiettes et
ficelez avec le reste des boyaux.

Faites revenir dans une poêle d'abord le lard gras, puis les oignons, les carottes
et le poireau coupé en morceaux.

Prenez une cocote en porcelaine allant au feu ; mettez au fond une soucoupe
également en porcelaine afin d'éviter toute adhérence, ensuite les légumes et
le lard revenus, les paquets de tripes, les pieds, les tomates pelées,
épépinées et coupées en morceaux, le bouquet garni, un peu d'ail râpé et de
girofle en poudre, au goût ; mouillez avec le fond de veau, le vin et la fine
champagne ; couvrez et laissez cuire à petit feu pendant {\ppp7\mmm}
à {\ppp8\mmm} heures.

Retirez pieds et paquets ; tenez-les au chaud.

Dégraissez la sauce : passez-la, réduisez-la à bonne consistance, goûtez et
complétez l’assaisonnement s'il y a lieu, puis montez-la au beurre.

Dressez les paquets sur un plat chaud, intercalez entre eux les pieds ; masquez
le tout avec la sauce et servez.

\sk

On peut aussi faire entrer les pieds dans les paquets, mais il sera
indispensable de prendre à cet effet des carrés de tripes de mouton de
dimensions plus grandes.

\newpage
\section*{\centering Pieds de mouton, sauce poulette.}
\phantomsection
\addcontentsline{toc}{section}{ Pieds de mouton, sauce poulette.}
\index{Pieds de mouton, sauce poulette}
\label{pg0445} \hypertarget{p0445}{}

Pour quatre personnes prenez :

\medskip

\footnotesize
\begin{tabular}{@{}lrrrp{18em}}
\normalsize1°\footnotesize & & & 16 & pieds de mouton parés et échaudés,                                  \\
  &     &          &    &     tels qu'on les vend à Paris,                                                \\
  &     &          &  5 & citrons ;                                                                       \\
  &     &          &    &                                                                                 \\
\normalsize 2° & \multicolumn{4}{l}{\normalsize   pour la cuisson :}                                      \\
\footnotesize
  &     &          &    &                                                                                 \\
  & 200 & grammes  & de & farine,                                                                         \\
  & 200 & grammes  & de & graisse de rognon de veau,                                                      \\
  & 100 & grammes  & d' & oignons,                                                                        \\
  &  40 & grammes  & de & sel,                                                                            \\
  &  20 & grammes  & de & poivre en grains,                                                               \\
  &     & 6 litres & d' & eau,                                                                            \\
  &     &          &  3 & clous de girofle,                                                               \\
  &     &          &  1 & gousse d'ail,                                                                   \\
  &     &          &  1 & fort bouquet garni ;                                                            \\
  &     &          &    &                                                                                 \\
\normalsize 3° & \multicolumn{4}{l}{\normalsize   pour la sauce :}                                        \\
\footnotesize
  &     &         &    &                                                                                  \\
  & 500 & grammes & de & fond de veau,                                                                    \\
  & 250 & grammes & de & champignons de couche,                                                           \\
  & 150 & grammes & de & crème épaisse,                                                                   \\
  &  25 & grammes & de & beurre,                                                                          \\
  &  10 & grammes & de & fécule,                                                                          \\
  &   5 & grammes & de & persil haché,                                                                    \\
  &     &         &  3 & jaunes d'œufs,                                                                   \\
  &     &         &    & jus de citron.                                                                   \\
\end{tabular}
\normalsize

\medskip

Faites tremper les pieds dans de l'eau froide pendant une heure ; nettoyez-les
bien ; essuyez-les. Passez-les dans le jus des cinq citrons ; réservez ce qui
restera de jus de citron.

Avec les éléments du paragraphe 2, préparez un blanc,
\hyperlink{p0425}{p. \pageref{pg0425}} ; mettez dedans les pieds et le jus de
citron réservé ; laissez cuire doucement pendant quatre heures environ, temps
nécessaire pour les bien attendrir.

Sortez les pieds, enlevez-en les os qui se détachent facilement, faites-les mijoter
ensuite, à tout petit feu, dans le fond de veau, pendant un quart d'heure environ,
en évitant qu'ils se colorent.

En même temps, faites cuire les champignons avec le beurre et du jus de citron
pendant un quart d'heure.

Retirez les pieds de la cuisson ; tenez-les au chaud.

Délayez la fécule dans un peu de fond de veau refroidi, chauffez en tournant,
ajoutez le reste du fond de veau, en tournant toujours : vous aurez un velouté.
Liez ce velouté avec la crème et les jaunes d'œufs sans laisser bouillir : vous
aurez une sauce suprême ; mettez dans cette sauce les champignons et leur
cuisson, le persil haché : vous obtiendrez ainsi une excellente sauce poulette
de couleur jaune clair. Ajoutez alors les pieds, chauffez encore, puis servez.

Les pieds de mouton poulette, considérés comme un plat plutôt vulgaire,
acquièrent, préparés ainsi, une grande finesse et ils peuvent fort bien figurer
dans un déjeuner sans cérémonie.

\section*{\centering Pieds de mouton, sauce mayonnaise à la ravigote.}
\phantomsection
\addcontentsline{toc}{section}{ Pieds de mouton, sauce mayonnaise à la ravigote.}
\index{Pieds de mouton, sauce mayonnaise à la ravigote}

Pour quatre personnes prenez :

\medskip

\footnotesize
\begin{longtable}{rrrp{18em}}
      1 & bouteille & de & vin blanc sec,                                                                 \\
        &           & 12 & pieds de mouton échaudés, tels qu'on les vend à Paris,                         \\
        &           &  1 & grosse gousse d'ail entière,                                                   \\
        &           &  1 & belle carotte coupée en rondelles,                                             \\
        &           &  1 & oignon moyen coupé en rondelles,                                               \\
        &           &  1 & bouquet garni composé de 5 grammes de persil,
                           1 brindille de thym et 1/4 de feuille de laurier,                              \\
        &           &    & huile,                                                                         \\
        &           &    & jaunes d'œufs frais,                                                           \\
        &           &    & cerfeuil,                                                                      \\
        &           &    & estragon,                                                                      \\
        &           &    & pimprenelle,                                                                   \\
        &           &    & civette,                                                                       \\
        &           &    & cresson alénois,                                                               \\
        &           &    & câpres,                                                                        \\
        &           &    & cornichons hachés,                                                             \\
        &           &    & muscade,                                                                       \\
        &           &    & poivre blanc,                                                                  \\
        &           &    & piment,                                                                        \\
        &           &    & sel,                                                                           \\
        &           &    & jus de citron ou vinaigre.                                                     \\
\end{longtable}
\normalsize

Faites cuire les pieds, pendant une heure et demie environ dans un
court-bouillon préparé avec le vin, la carotte, l'ail, l'oignon, le bouquet
garni ; laissez-les refroidir dans le liquide.

Retirez les pieds, désossez-les, passez le jus de cuisson, puis remettez-les
dans le jus passé pendant le temps suffisant pour que tout le liquide soit
absorbé par eux.

Préparez la sauce : avec de l'huile, des jaunes d'œufs et du sel, faites
d'abord une mayonnaise ordinaire, \hyperlink{p0323-2}{p. \pageref{pg0323-2}} ;
ajoutez-y du cerfeuil, de l'estragon, de la pimprenelle, de la civette, du
cresson alénois blanchis, égouttés, pressés, pilés et passés ; relevez-la avec
des câpres, des cornichons hachés, de la muscade, du poivre blanc, du piment et
du jus de citron ou du vinaigre, au goût.

Mélangez cette sauce avec les pieds et laissez en contact pendant plusieurs
heures avant de servir.

\sk

Comme variantes, on peut introduire dans la ravigote de la ciboulette, du
fenouil, de l'ail, des filets d'anchois, des œufs durs hachés, etc.

\section*{\centering Pieds de porc bouillis.}
\phantomsection
\addcontentsline{toc}{section}{ Pieds de porc bouillis.}
\index{Pieds de porc bouillis}

Prenez des pieds de porc frais ; grattez-les, échaudez-les, enlevez-en les
ongles ; lavez-les ensuite dans de l’eau froide.

Préparez un court-bouillon avec eau, vin blanc, oignon, carottes, ail, échalote,
clou de girofle, bouquet garni composé de thym, laurier, persil et sauge, poivre
en grains et sel.

Faites cuire les pieds dans ce court-bouillon pendant cinq heures, à liquide
frissonnant, comme pour un pot-au-feu, en ayant soin qu'ils baignent
constamment dans le liquide,

Retirez-les ; tenez-les au chaud.

Passez le jus de cuisson, réduisez-le, ajoutez-y de la crème de riz délayée
dans une petite partie de la cuisson refroidie, laissez cuire, puis liez la
sauce avec des jaunes d'œufs frais et versez-la sur les pieds.

Servez sur assiettes chaudes.

\sk

Comme variantes, on pourra désosser les pieds, les couper en morceaux, les
enrober dans de la pâte à frire ou dans de la mie de pain rassis tamisée et les
faire frire dans de la graisse, à pleine friture.

\sk

Les pieds de porc bouillis pourront aussi être servis avec les sauces
suivantes : sauce poulette, ayant pour base le fond de cuisson des pieds ;
sauce tartare ; sauce aux câpres ; sauce hollandaise à la ravigote ; sauce
béarnaise ; sauce tomate ; sauce douce ; sauce vinaigrette.

\section*{\centering Pieds de porc panés, grillés.}
\phantomsection
\addcontentsline{toc}{section}{ Pieds de porc panés, grillés.}
\index{Pieds de porc panés, grillés}

On prépare à Sainte-Menehould des pieds de porc qui ont une réputation
mondiale. En voici la formule.

Grattez, échaudez des pieds, enlevez-en les ongles, rafraîchissez-les dans de
l'eau froide. Attachez-les ensuite sur des planchettes pour éviter qu'il se
déforment et mettez-les pendant {\ppp48\mmm} heures dans de la saumure.

Préparez un court-bouillon avec de l'eau additionnée d'un cinquième de son
volume de vin blanc sec, des légumes de pot-au-feu, des aromates, un bouquet
garni et du sel.

Mettez dedans les pieds ; faites bouillir pendant trois quarts d'heure, puis
continuez la cuisson à liquide simplement frissonnant pendant cinq heures.

Passez la cuisson ; laissez refroidir les pieds dans la cuisson passée.

Coupez les pieds en deux, dans leur longueur ; passez-les successivement dans
des jaunes d'œufs frais battus puis dans de la mie de pain rassis tamisée et
faites-les griller en les arrosant d'un peu de beurre fondu.

Servez-les tels quels sans aucune garniture, ou servez-les accompagnés d'une
saucière de sauce Robert\footnote{Pour préparer la sauce Robert, faites fondre
dans du beurre des oignons coupés en rondelles, sans les laisser roussir ;
mouillez avec un bon jus corsé par de la glace de viande ; faites réduire ;
puis relevez la sauce avec du poivre et finissez-la avec de la moutarde.
Passez-la.}.

\section*{\centering Pieds de porc braisés au jus.}
\phantomsection
\addcontentsline{toc}{section}{ Pieds de porc braisés au jus.}
\index{Pieds de porc braisés au jus}
\index{Fond de porc}
Préparez un fond de porc corsé, comme il est dit dans la formule du fond de
veau, \hyperlink{p0426}{p. \pageref{pg0426}}, en remplaçant le veau par du porc,
ou bien prenez un fond de cuisson de jambon,
\hyperlink{p0537}{p. \pageref{pg0537}}, ces deux fonds étant aromatisés par du vin
blanc.

Apprêtez les pieds comme d'ordinaire : enveloppez-les dans une mousseline et
ficelez-les pour les empêcher de se déformer à la cuisson ; mettez-les dans une
braisière ; mouillez avec le fond de porc ou la cuisson de jambon et faites
cuire au four, à feu doux, à liquide frissonnant, pendant une dizaine d'heures,
laissez-les refroidir dans le liquide.

Peu de temps avant de servir, égouttez-les, sortez-les de leur enveloppe,
disposez-les sur un plat de service et passez-les au four de façon à les
présenter très chauds, croustillants à l'extérieur, moelleux à l'intérieur.

Servez avec des citrons coupés en quatre.

\sk

Les pieds braisés, au jus, sont naturellement plus savoureux que lorsqu'ils ont
élé simplement cuits à l'eau.

\section*{\centering Crépinettes de pieds de porc et de foie gras, sauce Périgueux.}
\phantomsection
\addcontentsline{toc}{section}{ Crépinettes de pieds de porc et de foie gras, sauce Périgueux.}
\index{Crépinettes de pieds de porc et de foie gras, sauce Périgueux}

Pour douze personnes prenez :

\medskip

\footnotesize
\begin{longtable}{rrrp{18em}}
    375 & grammes & de & madère,                                                                          \\
    300 & grammes & de & jambon de Bayonne,                                                               \\
    200 & grammes & de & bon jus ou de glace de viande dissoute dans un peu de bouillon,                  \\
     40 & grammes & de & beurre fin,                                                                      \\
     30 & grammes & de & farine,                                                                          \\
        &         &  6 & pieds de porc frais entiers,                                                     \\
        &         &  2 & oignons,                                                                         \\
        &         &  1 & foie gras d'oie,                                                                 \\
        &         &  1 & échalote,                                                                        \\
        &         &    & truffes à volonté,                                                               \\
        &         &    & légumes de pot-au-feu,                                                           \\
        &         &    & vin blanc,                                                                       \\
        &         &    & bouquet garni,                                                                   \\
        &         &    & crépine de porc,                                                                 \\
        &         &    & quatre épices,                                                                   \\
        &         &    & sel et poivre.                                                                   \\
\end{longtable}
\normalsize

Brossez et lavez les truffes ; faites-les cuire dans {\ppp175\mmm} grammes de madère avec
du sel et un peu de quatre épices, au goût. Laissez-les refroidir dans leur
cuisson ; enlevez-les.

Mettez à mariner, pendant {\ppp24\mmm} heures, le foie gras dans le madère de cuisson
des truffes. Escalopez-le ensuite en douze tranches ; réservez les déchets.

Pelez les truffes, coupez-en une partie en rondelles, réservez les autres pour la
sauce. Hachez les pelures et les déchets.

Nettoyez les pieds, échaudez-les, rafraîchissez-les.

Préparez un court-bouillon avec de l’eau et du vin blanc, dans les proportions
d'un cinquième de vin et quatre cinquièmes d'eau, des légumes de pot-au-feu, un
oignon, un bouquet garni, du sel, du poivre ; ajoutez-y les pieds ; faites
bouillir pendant une bonne demi-heure, puis continuez la cuisson à liquide
frissonnant pendant cinq heures.

Retirez les pieds, désossez-les, hachez-les grossièrement avec les déchets de
foie gras ; mélangez hachis de pieds et de foie et hachis de truffes.

Coupez la crépine en douze morceaux rectangulaires de {\ppp10\mmm} centimètres sur 20
centimètres environ ; mettez sur chaque morceau de crépine quelques rondelles
de truffe, au-dessus une couche de hachis de pieds truffé, ensuite une escalope
de foie gras que vous couvrirez par une autre couche de hachis truffé sur
lequel vous placerez quelques rondelles de truffe. Fermez les crépinettes, puis
faites-les griller doucement des deux côtés, à feu dessus.

En même temps, préparez la sauce Périgueux avec le beurre, la farine,
l'échalote, le second oignon coupé en rouelles, le jambon coupé en petits
morceaux, le jus, le reste du madère et celui de la marinade, comme il est dit,
\hyperlink{p0540}{p. \pageref{pg0540}}.

Au dernier moment, hachez les truffes réservées, mettez-les dans la sauce ;
chauffez.

Servez les crépinettes sur un plat décoré avec du persil frit, et la sauce dans une
saucière.

Ces crépinettes font bonne figure même dans un déjeuner d'apparat.

\section*{\centering Boudin noir.}
\phantomsection
\addcontentsline{toc}{section}{ Boudin noir.}
\index{Boudin noir}

Le boudin noir est un mets très ancien, d'origine assyrienne.

A défaut de la formule assyrienne primitive, en voici une autre plus moderne,
qui n'est pas sans mérite.

Prenez :

\medskip

\footnotesize
\begin{longtable}{rrrp{18em}}
  1 000 & grammes & de & panne,                                                                           \\
    250 & grammes & d' & oignons,                                                                         \\
    250 & grammes & de & truffes du Périgord,                                                             \\
    250 & grammes & de & crème épaisse,                                                                   \\
     30 & grammes & de & vieux rhum,                                                                      \\
     30 & grammes & de & sel,                                                                             \\
      5 & grammes & de & quatre épices,                                                                   \\
      2 & grammes & de & poivre,                                                                          \\
        & 1 litre & de & sang de porc, pur,                                                               \\
        &         &  1 & bouquet garni, composé de 30 grammes de persil, 2 grammes de thym,
                         et 2 grammes de laurier,                                                         \\
        &         &    & madère,                                                                          \\
        &         &    & boyau de porc.                                                                   \\
\end{longtable}
\normalsize

Nettoyez, grattez et lavez soigneusement le boyau.

Brossez, lavez les truffes ; faites-les cuire dans du madère ; hachez-les.

Faites fondre la panne ; mettez dedans les oignons émincés et le bouquet
garni ; laissez cuire à feu modéré, sans prendre couleur, pendant une
demi-heure environ.

Éloignez la casserole du feu, versez dedans le sang, petit à petit, en remuant
sans interruption, ajoutez le rhum, le sel, le poivre et les quatre épices ;
mélangez bien. Passez le tout au tamis à l’aide d'un pilon, chauffez sur feu
doux, mettez la crème et les truffes hachées, mélangez encore.

Coulez le mélange, au moyen d'un entonnoir à large goulot, dans le boyau fermé
à une extrémité par un nœud de ficelle. Lorsque le boyau sera empli aux neuf
dixièmes, fermez l’autre extrémité : le vide réservé ayant pour but de
permettre la dilatation.

Faites pocher le boudin dans de l’eau bouillante salée et aromatisée, pendant
quinze à vingt minutes. On reconnaît qu'il est cuit à point lorsque, après
l'avoir piqué, on voit sorti du jus au lieu de sang. Sortez le boudin de l'eau,
essuyez-le doucement, mettez-le sur une claie, couvrez-le avec un linge pour
l'empêcher de se racornir, laissez-le refroidir.

Coupez-le en morceaux, piquez légèrement chaque morceau avec une aiguille
à brider pour éviter leur éclatement à la cuisson finale et faites-les griller
à feu modéré pendant douze à quinze minutes.

Servez aussitôt, en envoyant en même temps de la purée de reinettes ou de
pommes du Canada\footnote{Pour préparer la purée, prenez de belles reinettes ou
des pommes du Canada ; pelez-les, coupez-les en morceaux, épépinez-les et
faites-les cuire dans un peu d’eau, très légèrement salée, en quantité
suffisante pour qu'elles ne s'attachent pas au fond de la casserole.
Écrasez-les, ajoutez du beurre frais, laissez-le fondre, mélangez bien.
\protect\endgraf
Tenez la purée au chaud jusqu'au moment de servir.}, qui accompagnera
parfaitement le boudin.

\sk

On peut simplifier la formule, qui est celle d’un boudin de luxe, en supprimant
les truffes.

\section*{\centering Boudin blanc.}
\phantomsection
\addcontentsline{toc}{section}{ Boudin blanc.}
\index{Boudin blanc}

Prenez :

\medskip

\footnotesize
\begin{longtable}{rrrp{18em}}
    500 & grammes & de & lait,                                                                            \\
    100 & grammes & de & panne,                                                                           \\
    100 & grammes & de & blanc de volaille rôtie, haché,                                                  \\
    100 & grammes & de & porc frais, cru, haché,                                                          \\
     30 & grammes & de & riz,                                                                             \\
        &         &  2 & gros oignons hachés,                                                             \\
        &         &  1 & œuf,                                                                             \\
        &         &    & truffes cuites, à volonté,                                                       \\
        &         &    & sel, poivre, muscade.                                                            \\
\end{longtable}
\normalsize

Faites cuire le riz dans le lait.

Faites fondre la panne ; mettez à cuire dedans les oignons, sans les laisser
roussir et le hachis de porc ; ajoutez le riz, la volaille, les truffes, du
sel, du poivre, de la muscade ; liez avec l'œuf battu.

Entonnez le tout dans un boyau et achevez l'opération de la cuisson finale
comme cela a été dit pour le boudin noir.
