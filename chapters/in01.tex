La deuxième édition de ma « Gastronomie pratique » a été épuisée il y a
quelques années déjà. Des obstacles matériels en ont empêché la réimpression.

Aujourd'hui, malgré les difficultés inhérentes en ce temps à toute publication,
je présente à mes lecteurs, en les remerciant de leur bienveillance, une édition
nouvelle remaniée de fond en comble.

Ils y verront d'abord un chapitre sur le service des repas depuis les temps les
plus reculés jusqu'à nos jours, une étude succincte sur les champignons comestibles 
de France et un article documenté sur le vin qui, complétant ce qui a paru
comme généralités dans la précédente édition, font du tout un véritable Prodrome
de Gastronomie ; ils y trouveront ensuite beaucoup de formules nouvelles, ainsi
que de nombreuses indications de variantes et de créations.

L'ouvrage est devenu très volumineux, mais le serait bien davantage encore
qu'il ne pourrait contenir toutes les combinaisons culinaires qui sont
innombrables. Il n'y a pas, il n'y aura jamais de livre de cuisine complet.
