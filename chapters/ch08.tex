Le vin naturel est le produit de la fermentation du jus de raisins frais.

L'apparition de la vigne sur la terre remonte à l'époque tertiaire. L'homme
primitif la trouva à l'état sauvage. Dès la plus haute antiquité,
instinctivement poussé à la recherche de boissons fortes, il tira du raisin un
breuvage plus ou moins alcoolisé en pressant simplement des grains fermentés au
soleil. Quelque imparfait qu'ait été ce premier vin, il possédait déjà des
qualités tellement séduisantes qu'il fut considéré comme un don du Ciel et
qu'on lui rendit des honneurs divins. Osiris en Égypte, Bacchus chez les Grecs
et dans l'Inde, Saturne chez les Latins, Noé chez les Hébreux personnifient les
premiers propagateurs du vin dans la préhistoire.

La culture de la vigne semble avoir été introduite sur notre côte
méditerranéenne par les Phéniciens, plus de six siècles avant notre ère. Au fur
et à mesure du développement de la civilisation, on détermina les conditions
les plus favorables à sa culture, on créa ses divers cépages, on découvrit des
procédés perfectionnés pour fabriquer le vin, mais aussi, hélas ! des moyens
scientifiques de le sophistiquer. L'un des effets du Christianisme fut
l'amélioration de la viticulture et de la vinification, la plupart des couvents
ayant leurs clos propres et fabriquant chacun son vin de messe. Nombre de crus
très réputés aujourd'hui ont été créés par des monastères.

La vigne couvre actuellement la quinzième partie de la surface de la France :
l'industrie vinicole y occupe plusieurs millions de personnes et la production
du vin y atteint de nos jours, en moyenne, une cinquantaine de millions
d'hectolitres représentant une valeur totale de plusieurs milliards de francs.

Je nai pas l'intention de parler ici de ce qui a trait aux règles de
l'œnologie : je me bornerai à donner quelques indications sur la composition
chimique des vins, sur leurs différentes qualités qui dépendent du cépage, du
terrain, du climat, du degré de maturité des raisins employés, des soins
apportés à la préparation des vins et de leur âge ; je dirai quelques mots sur
leur dégustation ; je fournirai des détails sur les principaux crus de France
et leur classement ; enfin, je terminerai par quelques mots sur le service des
vins et la question de la cave.

\section*{\centering Composition chimique des vins.}

La composition chimique des vins est assez variable. D'une façon générale, les
éléments les plus importants en poids sont : l'eau, qui va de {\ppp718\mmm}
à {\ppp936\mmm} grammes par litre ; différents alcools, de {\ppp45\mmm}
à {\ppp135\mmm} grammes ; la glycérine, de {\ppp4\mmm} à {\ppp13\mmm} grammes ;
les tartrates, de {\ppp1\mmm} gramme
à {\ppp3\mmm}\textsuperscript{gr} ,{\ppp75\mmm} ; le sucre, de {\ppp1\mmm}
gramme à {\ppp3\mmm} grammes (et même davantage dans certains vins très sucrés,
tels que le vin de Malaga qui contient jusqu'à {\ppp146\mmm} grammes de sucre
par litre) ; les matières colorantes, de
{\ppp0\mmm}\textsuperscript{gr}, {\ppp6\mmm} à {\ppp3\mmm} grammes. L'ensemble
des autres composants n'atteint que {\ppp9\mmm} à {\ppp13\mmm} grammes par
litre : ce sont des éthers, des essences, des aldéhydes, des matières
pectiques, des principes albuminoïdes, des gommes, des dextrines, des acides
variés et des sels (sels de fer, de magnésie, d'alumine, etc.) ; enfin des gaz,
des ferments et des matières fermentescibles.

On appelle \textit{extrait sec} ou \textit{matières extractives} d'un vin la
totalité de ses éléments non volatiles à la température de {\ppp100\mmm}° C. La
proportion de l'extrait sec varie généralement de {\ppp14\mmm} grammes
à {\ppp90\mmm} grammes par litre, sauf dans certains vins très sucrés où elle
peut atteindre jusqu'à {\ppp190\mmm} grammes. En général, dans les meilleurs
vins rouges de France, elle varie de {\ppp18\mmm} à {\ppp26\mmm} grammes.

\section*{\centering Propriétés et qualités du vin.}

Le vin, que les poètes appellent le sang de la vigne, est un liquide vivant ; il
peut devenir malade, il vieillit et il meurt.

Il constitue un aliment par ses hydrocarbures, par ses matières albumineuses et
gélatineuses et par ses sels ; il répand la chaleur dans le corps et stimule
l'appétit par son alcool et ses éthers ; il est tonique et fortifiant, en
partie par le fer qu'il contient ; il est digestif par ses ferments. Pris en
quantité modérée, il est réellement bienfaisant.

Les vins blancs jeunes, de même que les vins blancs mousseux, sont plus ou
moins laxatifs, à cause de leur acide carbonique ; les vins blancs légèrement
acidulés ont des propriétés diurétiques ; les vins mousseux et principalement
ceux de Champagne calment les muqueuses irritées.

L'usage prolongé du vin a certainement contribué à la formation et au
développement des qualités fondamentales de la race française : cordialité,
franchise, gaîté, esprit, goût, qui la différencient si profondément des
peuples grands buveurs de bière.

\medskip

Les qualités principales du vin sont : la couleur, le goût et le parfum.

Au point de vue de la couleur, les vins se divisent en vins rouges et en vins
blancs.

Les vins rouges sont le produit de la fermentation des moûts de raisins noirs.
Leur couleur, plus ou moins nuancée de violet, suivant qu'ils sont corsés ou
légers, varie comme teinte depuis le rose jusqu'au rouge marron, en passant par
le rouge cinabre, désigné aussi sous le nom de rouge pelure d'oignon, qui est
caractéristique des vins vieux.

Les vins blancs sont le plus souvent le produit de la fermentation des moûts de
raisins blancs ; mais on peut en obtenir aussi avec des raisins noirs,
exception faite pour ceux dits « teinturiers », en les égrappant et les
pressant aussitôt après la cueillette. Leur couleur varie comme teinte depuis
le jaune paille clair jusqu'au jaune ambré ou doré, caractéristique des vins
vieux doux, très alcoolisés. Ceux qui proviennent de raisins blancs donnent par
transparence un reflet verdâtre ; ceux qui proviennent de raisins noirs ont par
transparence un reflet rosé.

Les vins gris, qui rentrent dans la catégorie des vins blancs, sont le produit
de raisins roses, ou de vins blancs passés sur la râpe noire, c'est-à-dire sur
la partie ligneuse des grappes de raisins noirs, ou encore d'un mélange de vins
de coupage rouges et blancs.

La couleur des vins est souvent désignée sous le nom de \textit{robe}. Elle est
due aux œnotannins qui ont aussi pour effet de tempérer l'action de l'alcool
sur le système nerveux, ce qui explique pourquoi les vins blancs, qui
contiennent moins d'œnotannins que les vins rouges, sont plus excitants.

Le goût spécial dit \textit{de terroir} tient le plus souvent à la nature du
sol sur lequel a poussé la vigne qui a produit le vin et parfois aussi aux
engrais employés.

On dit qu'un vin est \textit{fruité} quand il a une saveur franche de raisin ;
qu'il est \textit{équilibré} quand aucun de ses éléments ne domine et qu'il
a un goût franc de vin. Ici, comme ailleurs, l'équilibre parfait est une
qualité relativement rare.

Un vin \textit{bourru} est un vin jeune chargé de particules solides, au sortir
de la cuve.

Un vin est \textit{dépouillé} quand il est débarrassé par le repos des
particules solides qui troublaient sa limpidité.

On appelle \textit{vin doux} un vin qui a peu fermenté. On ne consomme comme
vins doux que des vins blancs, la douceur étant un défaut dans le vin rouge.

Un vin est \textit{fort}, \textit{chaud}, quand il est chargé en alcool ; on
dit encore qu'il a de la \textit{vinosité}, du \textit{feu}. Il est
\textit{généreux} quand il produit, pris même en très petite quantité, une
sensation de bien-être, un effet tonique. Il est dit \textit{capiteux} quand il
monte à la tête ; tous les vins capiteux sont riches en matières spiritueuses.
Quand un vin est trop capiteux, on le qualifie souvent de \textit{fumeux}
(faisant monter à la tête des fumées, des vapeurs) : on dit encore que c'est un
\textit{casse-tête}.

Un vin est \textit{léger} quand sa teneur en alcool est relativement faible ;
\textit{frais} lorsque, à la température ambiante, il donne par sa saveur
légèrement acidulée une sensation de fraîcheur agréable au palais, qui est due
à une harmonie convenable entre sa teneur en alcool, en acides et en mucilages.

Un vin est \textit{corsé}, \textit{étoffé} quand il est largement pourvu
d'alcool, d'extrait sec, de matières colorantes, qu'il donne dans la bouche une
sensation spéciale de consistance : on dit qu'il \textit{emplit la bouche},
qu'il \textit{a de la chair}, qu'il \textit{a de la mâche}.

Un vin \textit{liquoreux} est un bon vin, plus ou moins capiteux, ayant une
saveur douce, sucrée, agréable. Cette qualification s'applique surtout aux vins
blancs : comme type de vin liquoreux, on peut citer le vin de Château Yquem. On
désigne quelquefois les vins liquoreux sous le nom de \textit{vins de paille}
parce qu'on emploie pour leur fabrication des raisins dans lesquels on
a concentré le sucre en les faisant plus ou moins sécher au soleil, au
préalable, sur de la paille.

Les vins de \textit{liqueur}, qui sont rouges ou blancs, se distinguent des
vins liquoreux par une douceur plus prononcée due à un soleil très ardent ou
à une cuisson du moût ; dans ce dernier cas, on les appelle encore \textit{vins
cuits} : tel est le vin de Frontignan.

Un vin est dit \textit{vif} quand il impressionne vivement le palais sans avoir
aucune saveur acide. Les vins vifs sont généralement caractérisés par une robe
brillante.

Il est dit \textit{nerveux} quand il a à la fois de la chair et de la vivacité.

Un vin est dit \textit{friand} quand on le boit toujours avec plaisir. Le type
des vins friands est le vin de Chablis.

Un vin \textit{moelleux} flatte le palais et chatouille agréablement les
papilles par sa saveur fondue qui est due à la glycérine et aux gommes. On dit
parfois d'un vin moelleux qu'il est \textit{coulant}, qu'il est
\textit{tendre}.

Un vin moelleux qui a de la chair est un vin \textit{gras}.

Un vin blanc est \textit{onctueux} quand il joint le moelleux à la douceur. Les
bons vins de Sauternes ont beaucoup d'onctuosité.

L'\textit{arome} d'un vin, c'est-à-dire l'impression qu'il produit sur
l'odorat, tient le plus souvent aux essences toutes formées qu'il contient et
aussi à des composés odorants produits par le dédoublement d'éléments inodores.
L'arome d’un vin est souvent désigné par le mot \textit{bouquet pour les vins
rouges}, par le mot \textit{parfum} pour les vins blancs.

Le \textit{cachet} d'un vin (tout vin cacheté n'a pas forcément de cachet) est
sa marque caractéristique.

La \textit{sève} d'un vin est sa qualité vitale (les poètes disent son âme) ;
elle provient de son cépage. On la perçoit à l'arrière-bouche, au premier
contact. Quand un vin devient trop vieux, il perd sa sève, il meurt, il est
passé.

Les vins \textit{fins} se distinguent par la délicatesse de leur sève,
l'agrément de leur arome, la netteté de leur goût et de leur couleur.

Un vin qui plait au goût et qui a de la délicatesse est \textit{distingué,}
disent les Bordelais ; il est \textit{savoureux} quand il a une sève abondante
et agréable, \textit{suave} quand il produit une impression douce et dégage un
charme irrésistible ; on dit alors qu'il fait dans la bouche \textit{la queue
de paon}..

Un vin \textit{velouté} est à la fois fin et moelleux. En Bourgogne, on dit
d'un vin velouté qu'il a de l'\textit{amour}.

On désigne sous le nom de \textit{grands vins} ceux qui, par l’ensemble de
leurs qualités, ont une supériorité incontestée et incontestable.

On appelle vin \textit{faible}, \textit{mince} ou \textit{maigre} un vin très
léger, manquant de corps, d'extrait sec et de couleur. Quand on déguste un vin
faible, il semble vraiment qu'il n'y ait rien entre la langue et le palais.

Un vin \textit{dur} est un vin qui manque de moelleux.

Un vin rouge est \textit{sec} quand il manque à la fois de chair et de
moelleux ; le goût d'un pareil vin est légèrement astringent ; il a souvent
perdu de ses matières extractives. Certains vins rouges sèchent en vieillissant

Un vin blanc \textit{sec} est un bon vin qui n'est pas liquoreux ; il chauffe
la langue et excite vivement le système nerveux. Le type des vins blancs secs
est le vin de Chablis.

Un vin est \textit{vert} lorsqu'il a une saveur astringente due au manque de
maturité du raisin. Les vins verts contiennent un excès de tartrate de potasse
et de tannin.

Un vin \textit{âpre} au goût passe difficilement dans la gorge ; il est dit
\textit{acerbe} quand il joint à l'âpreté la saveur qui caractérise les acides
végétaux ; il agace les dents.

Un vin est \textit{mou} ou \textit{plat} quand il manque de corps et que la
saveur des substances mucilagineuses domine toutes les autres.

Un vin est \textit{lourd} quand il est chargé à l'excès de couleur et de
matières extractives. Les vins lourds contiennent beaucoup trop de tannin ; ils
sont d'une digestion difficile.

Un \textit{gros} vin ou un vin \textit{bleu} est un vin monté en couleur, âpre
et peu corsé.

On désigne sous le nom de \textit{petit bleu} le vin rouge de Suresnes,

On appelle \textit{piccolo} les petits vins sans prétention.

\medskip

Les meilleures années au point de vue vinicole sont celles où, après un hiver
normal, le printemps a été lumineux, sec et tiède, l'été chaud, faiblement
nébuleux, avec alternances de pluie et de chaleur, et que le temps a été sec
pendant les vendanges. Cependant, bien des vins ne tiennent pas, quand ils ont
de la bouteille, ce qu'ils promettaient quand ils étaient jeunes, tels certains
enfants précoces.

\index{Dégustation du vin}
\section*{\centering Dégustation.}

On reconnait les différentes qualités des vins par la dégustation, qui consiste
à les soumettre aux impressions successives des différents organes susceptibles
de les apprécier : l'œil, le nez, la langue, le palais, la gorge et l'estomac.
Pour les vins fins, elle ne porte le plus souvent que sur les cinq premiers et
cela suffit largement, car un bon vin, pris en quantité modérée, n'a jamais
fait de mal à personne.

Voici comment on procède. Après avoir amené le vin, au sortir de la cave, à la
température la plus convenable : {\ppp17\mmm}° à {\ppp18\mmm}° C. pour les
bordeaux rouges, {\ppp12\mmm}° à {\ppp13\mmm}° pour les bourgognes rouges,
{\ppp10\mmm}° pour les vins blancs non mousseux et {\ppp8\mmm}° ou au-dessous
pour les vins mousseux, versez-en une petite quantité dans un grand verre
mousseline\footnote{Beaucoup de dégustateurs professionnels emploient comme
vase d'essai une tasse plate en argent qui, d'après eux, permettrait de mieux
apprécier par réflexion la couleur du vin. A mon avis, rien ne vaut un grand
verre de cristal fin. Le seul avantage que je reconnaisse à la tasse plate est
d'être très portative et cela peut avoir un certain intérêt pour les
dégustateurs de profession qui ont à se déplacer. Ils ont souvent à déguster
une telle quantité de vins dans une matinée qu'ils en sont réduits à supprimer
la dernière partie de l'opération, qui n’est cependant pas la moins agréable.
Au lieu d'avaler le vin, ils le rejettent et se rincent la bouche après chaque
dégustation, de peur d'en arriver à ne plus avoir qu'une sensation confuse.} et
examinez sa couleur ; puis, imprimez au verre quelques mouvements giratoires
pour favoriser l'expansion de l'arome et sentez le bouquet ; mouillez ensuite
la pointe de la langue qui vous renseignera parfaitement sur certaines
propriétés, notamment sur l'acidité et sur l'astringence. Si aucun parfum,
aucun goût ne dominent, vous pouvez conclure déjà que le vin est équilibré.
Prenez-en alors une bonne gorgée, promenez-la dans la bouche et retenez-la un
certain temps à l'entrée du pharynx, où les sensations sont particulièrement
nettes, cela vous permettra de constater les qualités de sève, de corps, de
finesse et de moelleux du breuvage. Enfin, avalez le liquide lentement en
faisant une large aspiration ; vous aurez ainsi une impression d'ensemble et
vous serez renseigné sur l'arrière-goût du vin, ce qui a son importance. Tout
cela est facile ; ce qui l'est moins, c'est de déterminer par cet essai le cru
et l'âge du sujet. Seuls, des gourmets d'élite, naturellement doués et
entraînés par une longue pratique sur des vins authentiques, sont en état de
résoudre la question. Mais la plupart des gens, ayant tant soit peu de goût,
reconnaîtront au moins approximativement si le vin est corsé, léger, sec, doux
ou liquoreux, si c'est un bourgogne ou un bordeaux, s'il est jeune, vieux ou
passé.

Pour les vins ordinaires d'un usage courant, la meilleure façon pratique de les
éprouver consiste, après les avoir goûtés comme il vient d'être dit, à en faire
usage pendant un certain temps. S'ils sont bien digérés, s'ils n'occasionnent
aucune aigreur, aucune lourdeur de tête, ils sont convenables.
 
\newpage
\vspace*{4\baselineskip}
\centerline{\large\textbf{LES MEILLEURS VINS DE FRANCE}}
\bigskip
\bigskip
\sk
\bigskip
\bigskip

\index{Classification des vins de France}
Estimant suffisante, pour les vins étrangers, la mention que j'en ai faite dans
les chapitres traitant des cuisines étrangères, je ne parlerai ici que des
meilleurs vins de France : vins de Bordeaux, de Bourgogne, du Beaujolais et du
Lyonnais, vins de Champagne, vins des côtes du Rhône, vins du Jura, de la
Touraine et de l'Anjou et vins de quelques autres régions dignes d'être
mentionnés\footnote{Je crois que la plupart de mes lecteurs partageront ma
manière de voir concernant la suprématie des vins de France ; tout le monde
n'est pourtant pas de cet avis et je m'en voudrais de ne pas signaler à ce
sujet l'opinion complètement différente d'un « konnaisseur » allemand, William
Bruchner : « Les vins de France, dit-il, sont des vins sans pensée (?) ; on les
boit parce qu'ils ont bon goût, mais ce n’est que lorsqu'on boit du vin du Rhin
qu'on pense » !}.

\section*{\centering Vins de Bordeaux.}

La culture de la vigne, très ancienne dans le Bordelais\footnote{La région du
Bordelais est formée par le département de la Gironde et par une partie de
l'arrondissement de Bergerac (Dordogne).}, reçut une grande impulsion sous le
règne de l'empereur Probus, au \textsc{iii}\textsuperscript{e} siècle de notre
ère. Dès le \textsc{iv}\textsuperscript{e} siècle Les vins du Médoc étaient
très prisés à Rome.

La réputation des vins de Bordeaux commença à se répandre en France au
\textsc{xiii}\textsuperscript{e} siècle, mais pendant longtemps ils ne furent
pas estimés à leur valeur. C’est seulement sous le règne de Louis XV, après
leur introduction à la Cour par le maréchal de Richelieu, qu'ils eurent la
vogue. Depuis, leur renommée alla sans cesse en grandissant ; elle est
universelle aujourd'hui.

Les bons vins de Bordeaux ont une belle couleur, une sève riche, une
distinction suprême, une finesse, une suavité et un velouté exquis ; ils sont
le type des vins bien équilibrés ; ils ont du corps et du moelleux ; ils sont
généreux sans être capiteux ; ils possèdent un bouquet délicieux ; on peut les
transporter sans inconvénient, ce qui a son importance\footnote{Le transport en
bateau active même leur vieillissement.}, et les proportions d'alcool et de
tannin qu'ils renferment leur permettent de vieillir sans sécher. Ils sont
toniques, reconstituants et digestifs ; ce sont les seuls vins qui conviennent
aux personnes débilitées et aux vieillards.

Le sol du Bordelais appartient à la formation secondaire. Les raisins blancs
y sont cultivés sur un terrain graveleux, non ferrugineux, reposant sur des
calcaires et sur des argiles jaunes ; les raisins noirs sont cultivés sur des
alluvions reposant sur des marnes et sur des calcaires ferrugineux plus ou
moins rouges.

Les principaux cépages noirs sont les \textit{cabernets} : le \textit{sémillon}
est le principal cépage blanc.

Les vins de Bordeaux peuvent être classés, suivant leur provenance, en cinq
groupes :

1° les vins du Médoc, sur la rive gauche de la Garonne, au nord de Bordeaux ;

2° les vins de Graves et de Sauternes, au Nord-Ouest, à l'Ouest et au Sud-Est
de cette ville ;

3° les vins de côtes, sur les coteaux argileux qui bordent la Gironde et la
Dordogne ;

4° la région de l'Entre-deux-Mers, entre la Garonne et la Dordogne, qui donne
des vins rouges grand ordinaire et des vins de coupage ;

5° les Palus, sur les alluvions des vallées de la Dordogne, qui ne donnent que
des vins communs, servant surtout au coupage.

\subsection*{\centering \small\sc Vins rouges.}

Voici un tableau des cinq premiers crus des vins rouges de Bordeaux
officiellement classés, avec l'indication de leurs qualités fondamentales,
étant entendu que les vins d’un même lieu participent d'une façon générale aux
mêmes qualités, à un degré décroissant avec leur classe.

Les meilleures années, depuis {\ppp1865\mmm}, ont été : {\ppp1865\mmm} (la plus
belle année du \textsc{xix}\textsuperscript{e} siècle), {\ppp1869\mmm},
{\ppp1870\mmm}, {\ppp1874\mmm}, {\ppp1875\mmm}, {\ppp1878\mmm}, {\ppp1887\mmm},
{\ppp1888\mmm}, {\ppp1891\mmm}, {\ppp1893\mmm}, {\ppp1895\mmm}, {\ppp1897\mmm},
{\ppp1898\mmm}, {\ppp1899\mmm}, {\ppp1904\mmm}, {\ppp1906\mmm}, {\ppp1907\mmm},
{\ppp1908\mmm}, {\ppp1909\mmm}, {\ppp1911\mmm}, {\ppp1913\mmm}, {\ppp1914\mmm}.

\subsection*{\centering \textit{Médoc}.}

\paragraph{Premiers crus.}

\setlength{\tabcolsep}{1pt}
\scriptsize
\begin{longtable}{m{14em}m{8em}m{14em}}                                                    
  \makecell{\textsc{désignation}}       & \makecell{\textsc{communes}} & \makecell{\textsc{observations}}              \\
  \makecell{—}                          & \makecell{—}                 & \makecell{—}                                  \\
  \nohyphens{Château Lafite‑Rothschild.}& \makecell{Pauillac.} & Les vins de Pauillac ont une belle robe 
                                                                 rubis et un bouquet exquis ; ils sont 
                                                                 corsés, très généreux, moelleux et pleins 
                                                                 de sève. Le domaine de Château Lafite, 
                                                                 créé en {\ppp1355\mmm}, a été vendu en 
                                                                 {\ppp1888\mmm} 
                                                                 {\ppp4\mmm} {\ppp500\mmm} {\ppp000\mmm} 
                                                                 francs.                                               \\
                                        &                      &                                                       \\
  Château Margaux.                      & \makecell{Margaux.}  & Les vins de Margaux, moins corsés que 
                                                                 ceux de Pauillac, sont plus parfumés et 
                                                                 leur finesse est exceptionnelle.                      \\
                                        &                      &                                                       \\
  Château La‑Tour.                      & \makecell{Pauillac.} &                                                       \\
\end{longtable}
\normalsize

\paragraph{Deuxième crus.}

\scriptsize
\begin{longtable}{m{14em}m{8em}m{14em}}                                                    
 \nohyphens{Château Mouton‑Rothschild.}           & \makecell{Pauillac.}      &                                        \\
 \nohyphens{Château Rauzan‑Ségla.}                & \makecell{Margaux.}       &                                        \\
 \nohyphens{Château Rauzan‑Gassies.}              & \makecell{—}              &                                        \\ 
                                                  &                           &                                        \\ 
 \nohyphens{Château Léoville‑Lascases.}           & \makecell{Saint‑Julien.}  & \multirow{3}{10em}{Légers, jolie        
                                                                                robe, pleins de sève délicieux.}       \\
 \nohyphens{Château Léoville‑Poyferré.}           & \makecell{—}              &                                        \\ 
 \nohyphens{Château Léoville‑ Barton.}            & \makecell{—}              &                                        \\   
                                                  &                           &                                        \\ 
 \nohyphens{Château Durfort‑Vivens.}              & \makecell{Margaux.}       &                                        \\
 \nohyphens{Château Lascombes.}                   & \makecell{—}              &                                        \\ 
 \nohyphens{Château Gruaud‑Larose‑Sarget.}        & \makecell{Saint‑Julien.}  &                                        \\
 \nohyphens{Château Gruaud‑Larose.}               & \makecell{—}              &                                        \\
                                                  &                           &                                        \\ 
                                                  &                           &                                        \\ 
 \nohyphens{Château Brane‑Cantenac.}              & \makecell{Cantenac.}      & \multirow{2}{10em}{Moelleux, bouquetés, 
                                                                                analogues aux vins de Margaux.}        \\ 
                                                  &                           &                                        \\
                                                  &                           &                                        \\
 \nohyphens{Château Pichon‑Longueville.}          & \makecell{Pauillac.}      &                                        \\
 \nohyphens{Château Pichon‑Longueville‑Lalande.}  & \makecell{—}              &                                        \\
 \nohyphens{Château Ducru‑Beaucaillou.}           & \makecell{Saint‑Julien.}  &                                        \\
                                                  &                           &                                        \\
 \nohyphens{Château Cos d’Estournel.}             & \makecell{Saint‑Estèphe.} & \multirow{2}{10em}{Légers, moelleux,    
                                                                                aromatisés.}                           \\
 \nohyphens{Château Montrose.}                    & \makecell{—}              &                                        \\ 
\end{longtable}
\normalsize

\paragraph{Troisièmes crus.}

\scriptsize
\begin{longtable}{m{14em}m{8em}m{14em}}                                                    
 \nohyphens{Château Kirwan.}                    & \makecell{Cantenac.}      &                                          \\
 \nohyphens{Château Issan.}                     & \makecell{—}              &                                          \\
 \nohyphens{Château Lagrange.}                  & \makecell{Saint‑Julien.}  &                                          \\
 \nohyphens{Château Langoa.}                    & \makecell{—}              &                                          \\
 \nohyphens{Château Giscours.}                  & \makecell{Labarde.}       & Fins et bouquetés.                       \\
 \nohyphens{Château Malescot‑Saint‑Exupéry.}    & \makecell{Margaux.}       &                                          \\
 \nohyphens{Château Cantenac‑Brown.}            & \makecell{Cantenac.}      &                                          \\
 \nohyphens{Château Palmer.}                    & \makecell{—}              &                                          \\
 \nohyphens{Château La Lagune.}                 & \makecell{Ludon.}         & Corsés, sève particulière.               \\
 \nohyphens{Château Desrmirail.}                & \makecell{Margaux.}       &                                          \\
 \nohyphens{Château Dubignon.}                  & \makecell{—}              &                                          \\
 \nohyphens{Château Calon‑Ségur.}               & \makecell{Saint‑Estèphe}  &                                          \\
 \nohyphens{Château Ferrière.}                  & \makecell{Margaux.}       &                                          \\
 \nohyphens{Château Marquis d'Alesme‑Becker.}   & \makecell{—}              &                                          \\
\end{longtable}
\normalsize

\paragraph{Quatrièmes crus.}

\scriptsize
\begin{longtable}{m{14em}m{8em}m{14em}}                                                    
 \nohyphens{Château Saint‑Pierre‑Sevaistre.}    & \makecell{Saint‑Julien.}  &                                          \\
 \nohyphens{Château St‑Pierre‑Bontemps‑Dubarry.}& \makecell{—}              &                                          \\
 \nohyphens{Château Duluc‑Branaire‑Ducru.}      & \makecell{—}              &                                          \\
 \nohyphens{Château Talbot‑Marquis d'Aux.}      & \makecell{—}              &                                          \\
 \nohyphens{Château Duhart‑Milon‑Castéja.}      & \makecell{Pauillac.}      &                                          \\
 \nohyphens{Château Pouget‑la‑Salle.}           & \makecell{Cantenac.}      &                                          \\
 \nohyphens{Château Pouget}                     & \makecell{—}              &                                          \\
 \nohyphens{Château La Tour‑Carnet.}            & \makecell{Saint-Laurent.} & Bouquet prononcé                         \\
 \nohyphens{Château Lafon‑Rochet.}              & \makecell{Saint-Estèphe.} &                                          \\
 \nohyphens{Château Beychevelle.}               & \makecell{Saint-Julien.}  &                                          \\
 \nohyphens{Château Le Prieuré.}                & \makecell{Cantenac.}      &                                          \\
 \nohyphens{Château Marquis de Thermes.}        & \makecell{Margaux.}       &                                          \\
\end{longtable}
\normalsize

\paragraph{Cinquième crus.}

\scriptsize
\begin{longtable}{m{14em}m{8em}m{14em}}                                                    
 \nohyphens{Château Pontet‑Canet.}              & \makecell{Pauillac.}      &                                          \\
 \nohyphens{Château Batailley.}                 & \makecell{—}              &                                          \\
 \nohyphens{Château Grand‑Puy‑Lacoste.}         & \makecell{—}              &                                          \\
 \nohyphens{Château Grand‑Puy‑Ducasse.}         & \makecell{—}              &                                          \\
 \nohyphens{Château Lynch‑Bages.}               & \makecell{—}              &                                          \\
 \nohyphens{Château Lynch‑Moussas.}             & \makecell{—}              &                                          \\
 \nohyphens{Château Dauzac.}                    & \makecell{Labarde.}       &                                          \\
 \nohyphens{Château Mouton‑d'Armailhacq.}       & \makecell{Pauillac.}      &                                          \\
 \nohyphens{Château Le Tertre.}                 & \makecell{Arsac.}         &  Analogue au vin de Cantenac.            \\
 \nohyphens{Château Haut‑Bages‑Libéral.}        & \makecell{Pauillac.}      &                                          \\
 \nohyphens{Château Pédesclaux.}                & \makecell{—}              &                                          \\
 \nohyphens{Château Belgrave.}                  & \makecell{Saint‑Laurent.} &                                          \\
 \nohyphens{Château Camensac.}                  & \makecell{—}              &                                          \\
 \nohyphens{Château Cos Labory.}                & \makecell{Saint‑Estèphe.} &                                          \\
 \nohyphens{Château Clerc‑Milon.}               & \makecell{Pauillac.}      &                                          \\
 \nohyphens{Château Croizet‑Bages.}             & \makecell{—}              &                                          \\
 \nohyphens{Château Cantemerle.}                & \makecell{Macau.}         &  Bonne tenue, du corps et du bouquet.    \\
\end{longtable}
\normalsize

\newpage
\subsection*{\centering \textit{Graves}.}

\paragraph{Premier cru.}

\scriptsize
\begin{longtable}{m{14em}m{8em}m{14em}}                                                    
Château Haut‑Brion.                             & \makecell{Pessac.}        & Seul grand vin rouge de Graves. Robe 
                                                                              vive et brillante, beaucoup de corps ; 
                                                                              rivalise avec les crus du Château Lafite, 
                                                                              du Château Margaux du Château La‑Tour, 
                                                                              tout en étant un un peu moins bouqueté. 
                                                                              Considéré cependant par certains gourmets 
                                                                              comme le premier cru rouge de Bordeaux.  \\
\end{longtable}
\normalsize

\subsection*{\centering \small\sc Vins blancs.}

Les meilleures années pour les vins blancs de Bordeaux ont été depuis
{\ppp1865\mmm} : {\ppp1865\mmm}, {\ppp1869\mmm}, {\ppp1850\mmm},
{\ppp1871\mmm}, {\ppp1874\mmm} (notamment pour le Château Yquem),
{\ppp1875\mmm}, {\ppp1878\mmm}, {\ppp1892\mmm}, {\ppp1893\mmm}, {\ppp1895\mmm},
{\ppp1897\mmm}, {\ppp1898\mmm}, {\ppp1899\mmm}, {\ppp1900\mmm}, {\ppp1904\mmm},
{\ppp1906\mmm}, {\ppp1908\mmm}, {\ppp1912\mmm}, {\ppp1914\mmm}, {\ppp1916\mmm}.

\subsection*{\centering \textit{Graves}.}

Les vins blancs de Graves se différencient des vins du Médoc par une sève et un
moelleux particuliers ; quelques‑uns sont distingués, mais il n'existe pas de
vin blanc de Graves classé.

\newpage
\subsection*{\centering \textit{Sauternes}.}

\paragraph{Grand premier cru.}

\scriptsize
\begin{longtable}{m{14em}m{8em}m{14em}}                                                    
  Château Yquem.                               & \makecell{Sauternes.}     & Grand vin, considéré comme le premier vin 
                                                                             liquoreux du monde. Robe ambrée ; très 
                                                                             parfumé, moelleux, suave. Un tonneau de 
                                                                             ce vin de {\ppp1847\mmm} a été vendu 
                                                                             {\ppp20\mmm} {\ppp000\mmm} fr. au 
                                                                             Grand‑Duc Constantin, en {\ppp1858\mmm}.  \\
\end{longtable}
\normalsize

\paragraph{Premiers crus.}

\scriptsize
\begin{longtable}{m{14em}m{8em}m{14em}}                                                   
  \nohyphens{Château La Tour‑Blanche.}         & \makecell{Bommes.}        &                                           \\
  \nohyphens{Château Peyraguey.}               & \makecell{—}              & Ressemblent aux précédents, avec un petit 
                                                                             parfum de muscat.                         \\
  \nohyphens{Château Lafaurie‑Peyraguey.}      & \makecell{—}              &                                           \\
  \nohyphens{Château Rayne‑Vigneau.}           & \makecell{—}              &                                           \\
  \nohyphens{Château Suduiraut.}               & \makecell{Preignac.}      & Moins liquoreux que les vins de 
                                                                             Sauternes.                                \\
                                               &                           &                                           \\
  \nohyphens{Château Coutet.}                  & \makecell{Barsac.}        & Capiteux et parfumés.                     \\
  \nohyphens{Château Climens.}                 & \makecell{—}              &                                           \\
                                               &                           &                                           \\
  \nohyphens{Château Guiraud.}                 & \makecell{Sauternes.}     &                                           \\
  \nohyphens{Château Rieussec.}                & \makecell{Fargues.}       & Analogue aux vins de Sauternes.           \\
  \nohyphens{Château Sigalas‑Rabaud.}          & \makecell{Bommes.}        &                                           \\
  \nohyphens{Château Rabaud‑Promis.}           & \makecell{—}              &                                           \\
\end{longtable}
\normalsize

\paragraph{Deuxième crus.}

\scriptsize
\begin{longtable}{m{14em}m{8em}m{14em}}                                                   
 \nohyphens{Château de Mirat.}                 & \makecell{Barsac.}        &                                           \\
 \nohyphens{Château Doisy-Daens.}              & \makecell{—}              &                                           \\
 \nohyphens{Château Doisy-Graves.}             & \makecell{—}              &                                           \\
 \nohyphens{Château Doisy-Védrines.}           & \makecell{—}              &                                           \\
 \nohyphens{Château d'Arche-Lafaurie.}         & \makecell{Sauternes.}     &                                           \\
 \nohyphens{Château d'Arche-Lacoste.}          & \makecell{—}              &                                           \\
 \nohyphens{Château Filhot.}                   & \makecell{—}              &                                           \\
 \nohyphens{Château Broustet.}                 & \makecell{Barsac.}        &                                           \\
 \nohyphens{Château Caillou.}                  & \makecell{—}              &                                           \\
 \nohyphens{Château Suau.}                     & \makecell{—}              &                                           \\
 \nohyphens{Château de Malle.}                 & \makecell{Preignac.}      &                                           \\
 \nohyphens{Château Romer.}                    & \makecell{—}              &                                           \\
 \nohyphens{Château Lamothe.}                  & \makecell{Sauternes.}     &                                           \\
\end{longtable}    
\normalsize

\newpage
\subsection*{\centering \textit{Quelques crus du Bordelais non classés}.}

En dehors de la classification officielle précédente, il existe dans le
Bordelais beaucoup de crus de valeur qui auraient pu parfaitement être classés.
Je me bornerai à en citer quelques-uns.

\subsubsection*{\centering \small\sc Graves rouges }

\scriptsize
\begin{longtable}{m{14em}m{8em}m{14em}}                                                   
 \nohyphens{Château Pape-Clément.}             &                           & Cru célèbre, créé en {\ppp1300\mmm} 
                                                                             par le Pape Clément V.                    \\ 
 \nohyphens{Château de la Mission-Haut-Brion.} &                           &                                           \\
 \nohyphens{Château des Carmes Haut-Brion.}    &                           &                                           \\
 \nohyphens{Château Monballon.}                &                           &                                           \\
 \nohyphens{Château Hermitage Haut-Brion.}     &                           &                                           \\
 \nohyphens{Château Raba.}                     &                           &                                           \\
 \nohyphens{Château des Templiers.}            &                           & Tous fins et délicats.                    \\
 \nohyphens{Château Haut-Bailly.}              &                           &                                           \\
 \nohyphens{Château Carbonnieux.}              &                           &                                           \\
 \nohyphens{Château Haut-Gardère.}             &                           &                                           \\
 \nohyphens{Château Brown-Léognan.}            &                           &                                           \\
 \nohyphens{Château Smith-Haut-Lafite.}        &                           &                                           \\
\end{longtable} 
\normalsize

\subsubsection*{\centering \small\sc Graves blancs }

\scriptsize
\begin{longtable}{m{14em}m{8em}m{14em}}                                                   
  Château Carbonnieux.                         &                           & \multirow{4}{10em}{Moins capiteux et moins 
                                                                             liquoreux que les vins de Sauternes, mais 
                                                                             très délicats.}                           \\
  Château Pontac.                              &                           &                                           \\
  Château Monplaisir                           &                           &                                           \\
  Château Smith-Haut-Lafite.                   &                           &                                           \\
\end{longtable} 
\normalsize

\subsubsection*{\centering \small\sc Sauternes}

\scriptsize
\begin{longtable}{m{14em}m{8em}m{14em}}                                                   
  Château Roumieux.                            &                           & \multirow{6}{10em}{Valent les deuxièmes 
                                                                             crus classés.}                            \\
  Château Cantegril.                           &                           &                                           \\
  Chäteau Piada.                               &                           &                                           \\
  Château Pernaud.                             &                           &                                           \\  
  Château du Closiot.                          & \makecell{Barsac.}        &                                           \\
  Château La Montagne.                         & \makecell{Preignac.}      &                                           \\
\end{longtable} 
\normalsize

\newpage
\subsection*{\centering \textit{ Vins de côtes}.}

Les vins de côtes, qui ne sont pas classés, et dont le type est le vin de
Saint-Émilion, méritent un petit chapitre.

La région de Saint-Émilion comprend des coteaux parallèles à la Dordogne,
s'étendant sur {\ppp7\mmm} à {\ppp8\mmm} kilomètres de longueur et {\ppp3\mmm}
kilomètres de largeur.

Les vins de Saint-Émilion ont une jolie robe foncée, brillante et veloutée, du
corps, une sève agréable, de la générosité et un bouquet spécial avec un léger
cachet d'amertume qui flatte le palais. On les appelle souvent les bourgognes
de la Gironde. Ils acquièrent le maximum de leurs qualités après {\ppp10\mmm}
à {\ppp20\mmm} ans de bouteille ; certains peuvent se conserver jusqu'à
{\ppp50\mmm} ans.

\medskip

Les meilleurs crus de Saint-Émilion sont :

\scriptsize
\begin{longtable}{m{14em}m{8em}m{14em}}                                                   
  Château Ausone.                     &                 & Le plus réputé.                                              \\
  Château Magdelaine.                 &                 & Très fin ; bouquet distingué.                                \\
  Château Canon.                      &                 & Vieille renommée.                                            \\
  Château Pavie.                      &                 & Généreux et moelleux.                                        \\
                                      &                 &                                                              \\
  \makecell[l]{Château Saint-Georges \\ 
  \hspace{1em}(côte Pavie).}          &                 & Cépage provenant du vignoble de Romanée-Conti.               \\
                                      &                 &                                                              \\
  Château Soutard.                    &                 & Du corps et de la finesse.                                   \\
  Château Sansonnet.                  &                 &                                                              \\
  Clos de l'Angelus.                  &                 & Généreux et distingué.                                       \\
  Château Palat Saint-Georges.        &                 & Analogue au vin de Château Pavie,                            \\
                                      &                 &                                                              \\
  Château Cheval-Blanc.               &                 & \multirow{2}{10em}{Deux crus situés sur un sol 
                                                          graveleux ; cachet spécial rappelant les vins 
                                                          du Médoc.}                                                   \\
                                      &                 &                                                              \\
  Château Figeac.                     &                 &                                                              \\
                                      &                 &                                                              \\
  Château de la Tour-du-Pin-Figeac.   &                 &                                                              \\
  Clos des Cordeliers.                &                 & Très bon champagnisé.                                        \\
\end{longtable}     
\normalsize

Enfin, comme vin de côtes, il faut mentionner le vin de Pommerol, des croupes
graveleuses situées entre la plaine de Libourne et les coteaux de
Saint-Émilion, vin fin, moelleux et bouqueté, tenant autant des vins du Médoc
que des vins de Saint-Émilion.

\medskip

Les meilleurs crus de Pommerol sont :

\scriptsize
\begin{longtable}{m{14em}m{8em}m{14em}}                                                   
  Vieux Château Certan.               &                 &                                                              \\
  Château Petrus.                     &                 &                                                              \\
\end{longtable}                                                                                             
\normalsize

Certains crus non classés du Bordelais sont souvent désignés sous les
dénominations suivantes : bourgeois, bourgeois du Bas-Médoc, paroisses
supérieures, artisans, palus et paysans. Ils sont vendus dans le commerce sous
les noms génériques de Médoc, Fronsac, Cistrac, Saint-Estèphe, Saint-Émilion,
Saint-Julien, Sauternes, Barsac, etc.

\section*{\centering Vins de Bourgogne.}

Les vins de Bourgogne sont aussi anciens que les vins de Bordeaux. Tacite en
parle. Les meilleurs crus de la contrée étaient déjà très appréciés dès le
\textsc{iii}\textsuperscript{e} siècle. Mais leur vogue ne remonte guère qu'aux
environs du \textsc{xiii}\textsuperscript{e} siècle, époque où les vins de
Pommard et de Volnay étaient en grande réputation. Au
\textsc{xiv}\textsuperscript{e} siècle, le vin de Beaune, qui seul avait le
privilège de paraître sur la table royale le jour du sacre, était considéré
comme le meilleur vin de France.

Les vins de Bourgogne des bons crus sont chauds, parfumés, corsés, généreux,
moelleux, délicieux ; ils ont un bouquet spécial très agréable ; ils sont
stimulants ; ils activent la digestion ; ils donnent de la verve et de la
vivacité d'esprit. Mais ils sont capiteux et ils montent facilement à la tête.
Ils conviennent aux gens bien portants et actifs : les nerveux, les
pléthoriques, les malades et les vieillards doivent s'en abstenir.

Les vins de Bourgogne doivent leurs qualités au sol, à l'excellence du cépage,
le pineau, et au climat de la région, régulièrement froid en hiver et chaud en
été. Les différentes compositions du terrain et son morcellement, qui
conduisent fréquemment les propriétaires à faire des mélanges souvent heureux,
ont produit en Bourgogne un très grand nombre de variétés de vins, dont la
classification libre, car il n’en existe pas d'officielle, est autrement
difficile que celle des vins de Bordeaux. Depuis le phylloxéra, la plupart des
vins de Bourgogne, les vins des Hospices de Beaune exceptés, sont « procédés »,
c'est-à-dire sucrés ; la proportion de sucre ajouté varie de {\ppp5\mmm}
à {\ppp15\mmm} kilogrammes par barrique de {\ppp228\mmm} litres.

Le sol de la Bourgogne appartient à la formation secondaire, depuis l'étage du
lias jusqu'à la craie. Il se compose de calcaires plus ou moins siliceux, de
grès, de calcaires à entroques, de calcaires marneux et de craie contenant des
rognons de silex en quantité plus ou moins grande, le tout recouvert en
certains endroits par des alluvions. Les éléments minéralogiques qui semblent
concourir aux qualités des vins de Bourgogne sont la chaux, le fer, l'alumine
et la silice.

Les vins de Bourgogne peuvent être classés géographiquement en trois zones :

1° la Haute-Bourgogne (Côte-d'Or),

2° la Basse-Bourgogne (Yonne), qu'on appelle aussi le bouquet de la Bourgogne,

3° la Bourgogne, sans désignation spéciale, comprenant le département de
Saône-et-Loire jusqu'au sud de Mâcon, à peu près à la limite de la commune de
la Chapelle-de-Guinchay, où le terrain Jurassique fait place aux terrains
primitifs et où le vin passe du type bourguignon au type des vins du
Beaujolais.

Les meilleures années pour les vins de Bourgogne ont été depuis
{\ppp1865\mmm} : {\ppp1865\mmm}, {\ppp1870\mmm}, {\ppp1873\mmm},
{\ppp1874\mmm}, {\ppp1877\mmm}, {\ppp1878\mmm} (surtout pour les vins de
Nuits), {\ppp1884\mmm}, {\ppp1886\mmm}, {\ppp1887\mmm}, {\ppp1889\mmm},
{\ppp1893\mmm}, {\ppp1898\mmm}, {\ppp1904\mmm}, {\ppp1906\mmm}, {\ppp1911\mmm},
{\ppp1915\mmm}.

\subsection*{\centering \small\sc Vins rouges.}

\subsubsection*{\centering \textit{ Haute-Bourgogne}.}

Les vins de la Haute-Bourgogne peuvent être subdivisés en trois groupes qui
sont, par ordre d'importance : ceux de la côte de Nuits, de Gevrey à Corgolin ;
ceux de la côte de Beaune, de Ladois-Serrigny à Decize (Saône-et-Loire) ; ceux
de la côte de Dijon, depuis les coteaux de la rive gauche de l'Ouche, jusqu'à
l'extrémité de la commune de Gevrey-Chambertin.

\paragraph{\normalsize\sc Côte de Nuits}

Les vins de la côte de Nuits sont réputés les meilleurs de la Bourgogne ; ils
sont fins, corsés, bouquetés ; ils atteignent généralement l'apogée de leurs
qualités vers l'âge de {\ppp12\mmm} ans ; on peut cependant conserver ceux des
grandes années jusqu'à {\ppp30\mmm} ans. Les gourmets leur trouvent un
arrière-goût de cassis.

\subparagraph{Grands premiers Crus.}

\scriptsize
\begin{longtable}{m{12em}m{9em}m{13em}}                                                    
  Romanée-Conti.        & \makecell{Vosne.}      & La perle des vins rouges de Bourgogne, d'un moelleux et 
                                                   d'une finesse véritablement extraordinaires. Accaparé 
                                                   depuis quelques années par des acheteurs étrangers.                 \\
                        &                        &                                                                     \\
  Clos Vougeot.         & \makecell{Vougeot.}    & Ancien clos très réputé, créé par les Cisterciens au 
                                                   \textsc{xii}\textsuperscript{e} siècle. Morcelé en 
                                                   {\ppp1889\mmm} en nombreuses parcelles. Vin absolument 
                                                   remarquable par son bouquet.                                        \\
                        &                        &                                                                     \\
  Musigny.              & \makecell{Chambolle.}  & L'un des vins les plus délicats de la Côte-d'Or.                    \\ 
                        &                        &                                                                     \\
\end{longtable}                                                                                             
\normalsize

\subparagraph{Premiers Crus.}

\scriptsize
\begin{longtable}{m{12em}m{9em}m{13em}}                                                    
  Romanée Saint-Vivant. & \makecell{Vosne.}     & Excellent, très bouqueté.                                            \\
  Saint-Georges.        & \makecell{Nuits.}     & Analogue au précédent.                                               \\
                        &                       &                                                                      \\
  Richebourg.           & \makecell{Vosne.}     & \multirow{2}{12em}{Très fins et très bouquetés.}                     \\
  La Tâche.             & \makecell{—}          &                                                                      \\
                        &                       &                                                                      \\
  Clos de Tart.         & \makecell{Morey.}     &                                                                      \\
  Bonnes Mares.         & \makecell{—}          & Riches en alcool et en fer.                                          \\
  Clos de la Roche.     & \makecell{—}          &                                                                      \\
  Lambrey.              & \makecell{—}          &                                                                      \\
\end{longtable}                          
\normalsize

\subparagraph{Deuxièmes Crus.}

\scriptsize
\begin{longtable}{m{12em}m{9em}m{13em}}                                                    
  Beaux Monts.          & \makecell{Vosne.}     &                                                                      \\
  Malconsort.           & \makecell{—}          &                                                                      \\
                        & \makecell{ }          &                                                                      \\
  Échézeaux,            & \makecell{Flagey.}    & Les grands Échézeaux se rapprochent des Romanée.                     \\
                        & \makecell{ }          &                                                                      \\
  Baudets.              & \makecell{Nuits.}     &                                                                      \\
  Cailles.              & \makecell{—}          &                                                                      \\
  Cras.                 & \makecell{—}          &                                                                      \\
  Murgers.              & \makecell{—}          &                                                                      \\
  Perrets.              & \makecell{—}          &                                                                      \\
  Prulière.             & \makecell{—}          &                                                                      \\
  Thorey.               & \makecell{—}          &                                                                      \\
  Vaucraine,            & \makecell{—}          &                                                                      \\
\end{longtable}                          
\normalsize

\paragraph*{\centering \small\sc Côte de Beaune}

Les vins de la côte de Beaune ont comme caractéristique un parfum qui semble
être une combinaison délicieuse de la violette et de la framboise et une saveur
qui rappelle à la fois celles de la pêche et de la reine-claude. Ils sont
francs et colorés, moelleux, pleins de feu et de bouquet. Ils se conservent
assez longtemps.

\subparagraph{ Premiers crus.}

\scriptsize
\begin{longtable}{m{12em}m{9em}m{13em}}                                                    
  Corton.               & \makecell{Aloxe.}     & \multirow{6}{12em}{Beaucoup de corps,de fermeté 
                                                  et de bouquet. Dans les meilleures années,ils 
                                                  représentent comme qualité, sur la côte de Beaune, 
                                                  les vins de Saint-Georges de la côte de Nuits.}                      \\
  Clos du Roi.          & \makecell{—}          &                                                                      \\
  Brossandes.           & \makecell{—}          &                                                                      \\
  Renardes-Corton.      & \makecell{—}          &                                                                      \\
                        &                       &                                                                      \\
                        &                       &                                                                      \\
                        &                       & \multirow{6}{12em}{Les vins de Corton des grandes 
                                                  années peuvent se conserver jusqu'à {\ppp20\mmm} ans.}               \\
                        &                       &                                                                      \\
  Les Chaumes.          & \makecell{Aloxe.}     &                                                                      \\
                        &                       &                                                                      \\
                        &                       &                                                                      \\
\end{longtable}                                                                                             
\normalsize

\subparagraph{ Deuxièmes crus.}

\scriptsize
\begin{longtable}{m{12em}m{9em}m{13em}}                                                    
  La Barre.             & \makecell{Volnay.}    &                                                                      \\
  Caillerets.           & \makecell{—}          &                                                                      \\
  Champans.             & \makecell{—}          &                                                                      \\
  Chevret.              & \makecell{—}          &                                                                      \\
  Premier.              & \makecell{—}          &                                                                      \\
  Bousse d'Or.          & \makecell{—}          &  \multirow{3}{12em}{Les crus de Volnay sont très anciens. Vins 
                                                   homogènes, fermes et moelleux.}                                     \\
  Les Angles.           & \makecell{—}          &                                                                      \\
  Carelle-sur-Chapelle. & \makecell{—}          &                                                                      \\
  Rougiet.              & \makecell{—}          &                                                                      \\
  Les Mitans.           & \makecell{—}          &                                                                      \\
  L'Ormeau.             & \makecell{—}          &                                                                      \\
                        & \makecell{ }          &                                                                      \\
  Santenet.             & \makecell{Meursault.} & \multirow{3}{12em}{Se rapprochent du chambertin.}                    \\
  Gras.                 & \makecell{—}          &                                                                      \\
  Pelures.              & \makecell{—}          &                                                                      \\
                        & \makecell{ }          &                                                                      \\
  Grèves Enfant-Jésus   & \makecell{Beaune.}    &                                                                      \\
  Fèves et Grèves.      & \makecell{—}          &                                                                      \\
  Clos de la Mousse.    & \makecell{—}          &                                                                      \\
  Avaux.                & \makecell{—}          & Bons vins.                                                           \\
  Biessandes-Aigrets    & \makecell{—}          &                                                                      \\
  Clos du Roi.          & \makecell{—}          &                                                                      \\
  Beaune Hospices.      &                       & Les vins des Hospices de Beaune constituent plutôt une excellente 
                                                  marque de vin de la côte de Beaune qu'un cru spécial. Les Hospices 
                                                  de Beaune possèdent, en effet, d'importants vignobles entre 
                                                  Aloxe-Corton et Meursault, sur les territoires de Beaune, Pommard, 
                                                  Volnay, Corton et Meursault, dont les vins, très réputés,sont vendus 
                                                  chaque année aux Hospices.Leurs prix servent de base aux transactions                       
                                                  annuelles de la région.                                              \\
                        &                       &                                                                      \\
  Pommard.              &                       & Les vins de Pommard, très parfumés, sont un peu moins fins que ceux 
                                                  de Volnay ; ils supportent très bien le transport.                   \\
                        &                       &                                                                      \\
  Claveillon.           & \makecell{Puligny.}   &                                                                      \\
  Clos Morgeot.         & \makecell{Chassagne.} &                                                                      \\
  Clos Saint-Jean.      & \makecell{Chassagne.} &                                                                      \\
  Clos Pitois.          & \makecell{—}          &                                                                      \\
  Corvées.              & \makecell{Prépeaux.}  &                                                                      \\
  Didiers.              & \makecell{—}          &                                                                      \\
  Portes.               & \makecell{—}          &                                                                      \\
  Clos Cavannes         & \makecell{Santenay.}  & Joli bouquet.                                                        \\
\end{longtable}                                                                                             
\normalsize

\paragraph*{\centering \small\sc Côte de Dijon}

Les vins de la côte de Dijon ont moins de bouquet que ceux de la côte de Nuits,
mais ils ont beaucoup de corps et de couleur.

\subparagraph{Grands premiers crus.}

\scriptsize
\begin{longtable}{m{12em}m{9em}m{13em}}                                                    
  Chambertin            & \makecell{Gevrey-\\Chambertin.} & Le vin de Chambertin est l’un des plus réputés de la Bourgogne pour le corps et la couleur.             \\
  Clos de Bèze.         & \makecell{—}          &                                                                      \\
                        &                       &                                                                      \\
\end{longtable}                                                                                             
\normalsize

\subparagraph{Premier cru.}

\scriptsize
\begin{longtable}{m{12em}m{9em}m{13em}}                                                    
  La Perrière.          & \makecell{Fixin.}                & Créé par les Cisterciens.                                 \\
\end{longtable}                                                                                             
\normalsize

\subsubsection*{\centering \textit{Basse-Bourgogne.}}

Les vins de la Basse-Bourgogne sont le plus souvent fins, bouquetés et
relativement légers. Malheureusement, un certain nombre des anciens crus
n'existent plus et nous ne les mentionnerons que pour mémoire.

\newpage
\scriptsize
\begin{longtable}{m{12em}m{9em}m{13em}}                                                    
  Migraine.                   &                              & \multirow{3}{12em}{N'existent plus ou presque plus.}    \\
  La Chaînette.               &                              &                                                         \\
  Queutard.                   &                              &                                                         \\
                              &                              &                                                         \\
  Judas.                      & \makecell{Auxerre.}          &                                                         \\
  Boivin.                     & \makecell{—}                 &                                                         \\
  Irancy.                     & \makecell{—}                 & Fin, mais rare. Possède toutes ses qualités à 
                                                               {\ppp6\mmm} ans.                                        \\
  \makecell[l]{Épineuil \\ 
\hspace{1em}(Les Perrières).} & \makecell{—}                 & Beaucoup d'arome, mais peu de corps.                    \\
  Les Olivettes.              & \makecell{Tonnerre.}         & Se conserve peu.                                        \\
\end{longtable}                                                                                             
\normalsize

\subsubsection*{\centering \textit{Bourgogne. }}

Les vins dits simplement de Bourgogne comprennent ceux des côtes châlonnaise et
mâconnaise.

Comme vins rouges intéressants, il n'y a guère que ceux de Mercurey, dans la
côte châlonnaise, légers à l'estomac, fins, bouquetés, d'un goût agréable.

\subsection*{\centering \small\sc Vins blancs.}

Tous les grands vins blancs de Bourgogne sont produits par le cépage
\textit{chardonay}. Ils sont extrêmement stimulants et, à moins d'une grande
accoutumance, il convient de n'en faire usage qu'avec modération.

Les principaux crus sont le montrachet et les meursault dans la
Haute-Bourgogne, les chablis dans la Basse-Bourgogne et le pouilly dans la
région qualifiée Bourgogne sans autre désignation.

\subsubsection*{\centering \textit{Haute-Bourgogne.}}

\paragraph*{\centering \small\sc CÔTE DE BEAUNE}

\scriptsize
\begin{longtable}{m{12em}m{9em}m{13em}}                                                    
  Montrachet   & \makecell{Puligny \\ et \\ Chassagne.}     & La perle des vins blancs de Bourgogne ; 
                                                              le rival, en son genre du vin de Château Yquem. 
                                                              Corsé, moelleux, et bouquet suave, à cheval 
                                                              entre celui du meursault et celui du chablis ; 
                                                              léger goût de noisette. Peut être conservé très 
                                                              longtemps.                                               \\
               &                                            &                                                          \\
  Meursault    & \makecell{Meursault.}                      &  La perle des vins blancs de Bourgogne ;                 \\
\end{longtable}                                                                                             
\normalsize

Voici comment on peut classer les différents crus de vin de Meursault :

\scriptsize
\begin{longtable}{m{12em}m{9em}m{13em}}                                                    
                                   &                     &                                                             \\
 Meursault Perrières.              &                     &  Incontestablement le meilleur : fin, parfum 
                                                            individuel, carré, comme on dit en Bourgogne ; 
                                                            se rapproche du vin de Montrachet.                         \\
                                                                                                                       \\
  Meursault Charmes.               &                     &                                                             \\
  Meursault Combette.              &                     &                                                             \\
  Meursault Genevrières.           &                     &                                                             \\
  Meursault Goutte-d'Or.           &                     &                                                             \\
  Meursault les Poruset.           &                     &                                                             \\
  Meursault les Chevalières.       &                     & Les vins de Meursault qui ont, comme 
                                                           les vins de Montrachet, un petit goût 
                                                           de noisette, sont plus légers que ces 
                                                           derniers.                                                   \\
  Meursault les Terres blanches.   &                     &                                                             \\
  Meursault les Lurôles.           &                     &                                                             \\
  Meursault les Tessons.           &                     &                                                             \\
  Meursault les Bouchères.         &                     &                                                             \\
  Meursault Santenet.              &                     &                                                             \\
  Meursault la Désirée.            &                     &                                                             \\
                                   &                     &                                                             \\
  Musigny blanc. Chambolle.        &                     & Rarissime ; grand vin presque aussi fin 
                                                           que le montrachet avec lequel il rivalise. 
                                                           Extrêmement moelleux, goût particulier, 
                                                           bouquet admirable, sève vigoureuse, couleur 
                                                           dorée très belle.                                           \\
  Corton blanc.                    & \makecell{Aloxe.}   & Réputé.                                                     \\
  Charlemagne-Corton.              & \makecell{Pernaud.} &                                                             \\
  Bâtard-Montrachet.               & \makecell{Puligny.} &                                                             \\
  Chevalier-Montrachet.            & \makecell{-}        &                                                             \\
                                   & \makecell{-}        &                                                             \\
\end{longtable}                                                                                             
\normalsize

\subsubsection*{\centering \textit{Basse-Bourgogne.}}

\scriptsize
\begin{longtable}{m{12em}m{9em}m{13em}}                                                    
                                   &                     &                                                             \\
  Chablis.                         & \makecell{Chablis.} & Le vin de Chablis a une robe très claire au 
                                                           début, qui prend avec l'âge une teinte ambrée. 
                                                           Il est vif, friand. Son bouquet charmant éclate 
                                                           sur le palais en un véritable feu d'artifice, 
                                                           disent les Bourguignons, qui le mettent au-dessus 
                                                           de tous les vins blancs.                                    \\
                                   &                     &                                                             \\
\end{longtable}                                                                                             
\normalsize

Les meilleurs crus de Chablis sont les suivants :

\scriptsize
\begin{longtable}{m{12em}m{9em}m{13em}}                                                    
                                   &                    &                                                              \\
  Chablis moutonne.                &                    & On considère généralement ce qu'on appelle le chablis 
                                                          moutonne comme le meilleur cru de Chablis, mais en 
                                                          réalité le vocable « moutonne » n'est pas, comme on 
                                                          pourrait le croire, un terme générique indiquant un 
                                                          lieu dit ; c'est le surnom donné à certaines parcelles 
                                                          de terre plus ou moins isolées et dont la qualité du vin 
                                                          a fait la réputation. On en abuse quelquefois.               \\
                                   &                    &                                                              \\
  Chablis les Clos.                &                    &                                                              \\
  Chablis Vaudésir.                &                    &                                                              \\
  Chablis les Grenouilles.         &                    &                                                              \\
  Chablis Mont de Milieu.          &                    &                                                              \\
  Chablis Valmur.                  &                    &                                                              \\
  Chablis Chamlot,.                &                    &                                                              \\
                                   &                    &                                                              \\
  Milly.                           &                    & Bon vin, assez léger, parfumé, mais
                                                          inférieur au chablis,                                        \\
                                   &                    &                                                              \\
  Côte Saint-Jacques.              & \makecell{Joigny.} & Vin gris, parfumé et généreux ; très rare.                   \\ 
                                   &                    &                                                              \\
\end{longtable}                                                                                             
\normalsize

\subsubsection*{\centering \textit{Bourgogne.}}

\scriptsize
\begin{longtable}{m{12em}m{9em}m{13em}}                                                    
  Pouilly                         & \makecell{Fuissé 
                                    \\ et        
                                    \\ Solutré.}        & Vin capiteux, fin, bouqueté, presque incolore, 
                                                              de bonne conservation.                                   \\
                                  &                     &                                                              \\
  Rully.                          & \makecell{Côte     
                                     \\ châlonnaise}    & Réputé pour sa finesse.                                      \\
\end{longtable}                                                                                             
\normalsize

\section*{\centering Vins du Lyonnais.}

Le Lyonnais, qui s étend de la Bourgogne au Velay et de l'Auvergne au Dauphiné,
est partagé géographiquement en trois régions : le Beaujolais, le Lyonnais
proprement dit et le Forez ; seules les deux premières produisent des vins de
marque.

\subsubsection*{\centering \textit{ Vins du Beaujolais.}}

Le Beaujolais confine au Mâconnais, mais il en diffère complètement au point de
vue de son ossature géologique, il comprend une partie du département de la
Saône, le Rhône et une partie du département de Saône-et-Loire, jusqu'à la
vallée du Gien.

Le sol se compose essentiellement de terrains primitifs : granites, gneiss,
schistes et porphyres.

Les vins rouges du Beaujolais sont renommés. Ils sont produits par le cépage,
le \textit{gamay} et les vignes qui le fournissent sont localisées dans les
communes de Beaujeu et de Belleville. Ces vins ont une belle robe ; ils sont
frais, légers, très agréables, avec un goût de terroir caractéristique. Ils ont
le maximum de leurs qualités au bout de {\ppp4\mmm} à {\ppp5\mmm} ans : mais il
ne faut pas les conserver trop longtemps, car ils passent assez vite. Ils
supportent difficilement le transport.

\medskip
Parmi les crus principaux, on peut citer :

\paragraph{Premiers crus.}

\scriptsize
\begin{longtable}{m{12em}m{9em}m{13em}}                                                    
  Thorins.             & \makecell{(Saône-et-Loire.) 
                         \\ Canton  
                         \\ de la Chapelle 
                         \\ de-Guinchay.}             & Sur le territoire de la commune de Romanèche-Thorins. 
                                                        Les vins de Romanèche sont très appréciés.                     \\
                       &                              &                                                                \\
  Moulin-à-Vent,       & \makecell{—}                 & Le vin de Moulin-à-Vent, qui provient de vignes situées 
                                                        sur l'affleurement d'un filon manganésifère, a un goût 
                                                        de terroir particulier.                                        \\
                       &                              &                                                                \\
  Fleurie.             & \makecell{(Rhône.)                                       
                         \\ Canton                  
                         \\ de Beaujeu.}              & Vins légers, délicats, ayant de la sève, très agréables 
                                                        et très friands.                                               \\
\end{longtable}                                                                                             
\normalsize

\medskip
Parmi les meilleurs crus de Fleurie citons :
\medskip

\scriptsize
\begin{longtable}{m{12em}m{9em}m{13em}}                                                    
  La Chapelle des Bois.                 &                     &                                                        \\
  Le Garant.                            &                     &                                                        \\
  Les Morière.                          &                     &                                                        \\
  Poncié.                               &                     &                                                        \\
  La Roilette.                          &                     &                                                        \\
  Le Vivier.                            &                     &                                                        \\
  Le Point du Jour.                     &                     &                                                        \\
  \multicolumn{2}{l}{Grande Cour (ancien Fleurie-Lacour).}    &                                                        \\
  Le Bourg.                             &                     &                                                        \\
  Quatre-Vents.                         &                     &                                                        \\
  Les Rocheux.                          &                     &                                                        \\
                                        &                     &                                                        \\
  Villié-Morgon.                        & \makecell{(Rhône.)                                       
                                          \\ Canton                  
                                          \\ de Beaujeu.}     & Vins très réputés et se conservant bien.               \\
\end{longtable}                                                                                                         
\normalsize

\medskip
Les crus les plus appréciés de Villié-Morgon sont :

\medskip

\scriptsize
\begin{longtable}{m{12em}m{9em}m{13em}}                                                    
  Les Pierres.                         &                     &                                                        \\
  Le Pis.                              &                     &                                                        \\
  Les Plâtres.                         &                     &                                                        \\
  La Verchère.                         &                     &                                                        \\
  Les Varennes.                        &                     &                                                        \\
                                       &                     &                                                        \\
  Chénas.                              & \makecell{(Rhône.)                                       
                                          \\ Canton                  
                                          \\ de Beaujeu.}    & Vins généreux et bouquetés.                            \\
\end{longtable}                            
\normalsize

\medskip
Parmi les meilleurs crus de Chénas, citons :

\medskip
\scriptsize
\begin{longtable}{m{12em}m{9em}m{13em}}                                                    
  Les Caves.                           &                     &                                                        \\
  Roche-Grès,                          &                     &                                                        \\
  La Rochelle.                         &                     &                                                        \\
                                       &                     &                                                        \\
  Saint-Lager.                         & \makecell{(Rhône.) 
                                         \\ Canton             
                                         \\ de Belleville.}  &                                                        \\
\end{longtable}
\normalsize

\medskip
Les meilleurs crus de Saint-Lager sont :
\medskip

\scriptsize
\begin{longtable}{m{12em}m{9em}m{13em}}                                                    
  Brouilly.                            &                     & \multirow{5}{12em}{On fabrique à Saint-Lager, avec les  
                                                                meilleurs crus, des vins mousseux très appréciés.}     \\
  L'Écluse.                            &                     &                                                         \\
  L'Éronde.                            &                     &                                                         \\                     
  Godefroid.                           &                     &                                                         \\
  Les Maisons-Neuves.                  &                     &                                                         \\
\end{longtable} 
\normalsize

\paragraph{Deuxièmes crus.}

\scriptsize
\begin{longtable}{m{12em}m{9em}m{13em}}                                                    
                                       &                     &                                                        \\
  La Chapelle-de-Guinchay.             & \makecell{(Saône-
                                          et-Loire.)}        &                                                        \\
                                       &                     &                                                        \\
  Juliénas.                            & \makecell{(Rhône.)                                       
                                          \\ Canton                  
                                          \\ de Beaujeu.}    &                                                        \\
\end{longtable}
\normalsize

\subsubsection*{\centering \textit{ Vins du Lyonnais proprement dit.}}

\subsection*{\centering \small\textsc{vins rouges.}}

\scriptsize
\begin{longtable}{m{12em}m{9em}m{13em}}                                                    
                                       &                     &                                                         \\
                                       & (Rhône.)            &  Belle robe, généreux, beaucoup de bouquet              \\
  Côte-Rôtie.                          & Canton              &  et de sève, d'une grande finesse. Gagnent              \\
                                       & de Condrieu.        &  beaucoup en vieillissant. Cépage                       \\
                                       &                     &  \textit{sérine}.                                       \\
                                       &                     &                                                         \\
\end{longtable}
\normalsize

\subsection*{\centering \small\textsc{vins blancs.}}

\scriptsize
\begin{longtable}{m{12em}m{9em}m{13em}}                                                    
  Château grillé.                      &                     &  Très généreux, pétillants, parfumés     
                                                                et de bonne conservation. Cépage \textit{voignier.}    \\
                                       &                     &                                                         \\
  Condrieu.                            &                     &                                                         \\
\end{longtable}
\normalsize

\section*{\centering Vins de Champagne.}

La connaissance des vins de Champagne remonte assez loin dans l'Histoire.
L'empereur Probus, de même qu'il l'avait fait en Bourgogne, fit planter des
vignes en Champagne : la première date de l'an {\ppp280\mmm}.

Saint Rémy, patron de Reims, mort en {\ppp530\mmm}, et, plus tard, les prêtres
de son diocèse constituèrent dans la contrée de magnifiques propriétés
vinicoles.

En {\ppp1397\mmm}, Venceslas, roi de Bohême et empereur d'Allemagne, qui était
venu à Reims dans le but de négocier un traité avec Charles VI, prit un tel
goût au vin du pays que, lorsque les envoyés du roi de France vinrent le
chercher pour l'introduire auprès de leur maître, ils le trouvèrent ivre-mort,
ce qui facilita beaucoup les négociations.

Agnès Sorel, qui raffolait du vin de Champagne, lui trouvait un goût de pêche.

Au \textsc{xvi}\textsuperscript{e} siècle, François I\textsuperscript{er},
Charles-Quint, Léon X de Médicis, Henri VIII d'Angleterre acquirent des
vignobles à Ay ; et, au sacre de Henri III, le vin de Champagne prit la place
du vin de Beaune.

\medskip

On ne connaissait alors que les vins naturels de Champagne, rouges ou blancs,
qui joignaient à la sève des vins de Bourgogne une grande finesse de goût et un
remarquable bouquet. Ce ne fut qu'à la fin du \textsc{xvii}\textsuperscript{e}
siècle, vers {\ppp1\mmm} {\ppp695\mmm}, que Dom Pérignon, moine de l'abbaye
d'Hautevillers, près d'Épernay, et fin gourmet, eut l'idée de les rendre
mousseux. Il fixa expérimentalement leur procédé de fabrication, qui ne fut
scientifiquement complété qu'en {\ppp1\mmm} {\ppp836\mmm} par un chimiste de
Reims, François, lequel détermina les proportions de sucre les plus convenables
pour obtenir les meilleurs vins mousseux.

Aujourd'hui, à part quelques vins rouges mentionnés plus loin et que l'on peut
encore trouver tels quels dans le commerce, il n'est guère possible de goûter
aux vins naturels de Champagne que chez des propriétaires de vignes. La plupart
sont gazéifiés pour fournir le vin de Champagne mousseux, dont la renommée est
mondiale.

\medskip

Le sol de la Champagne appartient au système crétacé et, en bien des endroits,
les terrains vinicoles présentent à leur surface des cailloux siliceux et des
cailloux calcaires.

Les raisins noirs viennent dans des terrains colorés, les raisins blancs dans des
terrains gris ou jaunâtres.

Les cépages des raisins noirs sont le \textit{pineau noir}, le
\textit{meunier}, le \textit{vert doré d'Ay} ou \textit{morillon d'Épernay} et
le \textit{pineau gris}. Ceux des raisins blancs sont le \textit{gamay}, le
\textit{mestier} et le \textit{pineau blanc chardonay d'Avize}.

\subsection*{\centering \small\sc Vins rouges.}

On fabrique encore en Champagne, pour de rares amateurs, des cuvées de vin
rouge. Voici les quelques vins rouges qui ont survécu :

\scriptsize
\begin{longtable}{m{10em}m{12em}m{12em}}                                                    
                           &                     &                                                                     \\
  Cru Riceys.              & \makecell{(Aube.)}  & Vin rosé, ayant du corps et de la finesse.                          \\
                           & \makecell{ }        &                                                                     \\
  Bouzy.                   & \makecell{(Marne.)} &                                                                     \\
  Ambonnay.                & \makecell{—}        & Cuvées rouges. Vins bouquetés, mais                                 \\
  Mailly.                  & \makecell{—}        & supportant mal le transport.                                        \\
  Cumières.                & \makecell{—}        &                                                                     \\
                           &                     &                                                                     \\
\end{longtable}
\normalsize

\medskip
\subsection*{\centering \small\textsc{vins blancs.}}

Les vins blancs sont récoltés dans trois régions de la Marne : la montagne de
Reims, la vallée de la Marne et les collines d'Avize. Les premiers ont beaucoup
de corps et de fraîcheur : les seconds, tous produits avec des raisins noirs
vinifiés en blanc, sont très moelleux et possèdent un bouquet extraordinaire :
les troisièmes sont très délicats.

\subsubsection*{\centering \textit{ Essai de classification. }}

Toute classification de crus, par ordre de mérite, est difficile et n'a rien
d'absolu. Celle qui suit a été dressée avec le concours d'amateurs très compétents
et elle repose sur des appréciations sérieuses : c'est tout ce qu’on peut en dire.

\paragraph{ Grands crus. }

\scriptsize
\begin{longtable}{m{10em}m{12em}m{12em}}                                                    
  Cramant.                 & \makecell{Colline d'Avize.} & \multirow{4}{12em}{Raisins blancs.}                         \\
  Avize.                   & \makecell{—}                &                                                             \\
  Oger.                    & \makecell{—}                &                                                             \\
  Mesnil-sur-Oger.         & \makecell{—}                &                                                             \\
\end{longtable}
\normalsize

\paragraph{ Grands premiers crus.}

\scriptsize
\begin{longtable}{m{10em}m{12em}m{12em}}                                                    
  Ay.                      & \makecell{Vallée de la 
                             \\ Marne.}                  & \multirow{8}{12em}{Raisins noirs.}                          \\
  Mareuil.                 & \makecell{—}                &                                                             \\
  Dizy.                    & \makecell{—}                &                                                             \\
  Bouzy.                   & \makecell{—}                &                                                             \\
  Ambonnay.                & \makecell{—}                &                                                             \\
  Verzy.                   & \makecell{Montagne    
                             \\ de Reims.}               &                                                             \\
  Verzenay.                & \makecell{—}                &                                                             \\
  Mailly.                  & \makecell{—}                &                                                             \\
\end{longtable}
\normalsize

\paragraph{ Premiers crus.}

\scriptsize
\begin{longtable}{m{10em}m{12em}m{12em}}                                                     
  Ludes.                   & \makecell{Montagne de 
                             \\ Reims.}                  & \multirow{5}{12em}{Raisins noirs.}                          \\
  Chigny.                  & \makecell{—}                &                                                             \\
  Rilly.                   & \makecell{—}                &                                                             \\
  Pierry.                  & \makecell{Colline d'Avize.} &                                                             \\
  Avenay.                  & \makecell{—}                &                                                             \\
                           & \makecell{ }                &                                                             \\
\end{longtable}
\normalsize

\paragraph{ Deuxièmes crus.}

\scriptsize
\begin{longtable}{m{10em}m{12em}m{12em}}                                                     
  Chouilly,                & \makecell{Vallée 
                             \\de la Marne.}             & \multirow{3}{12em}{Raisins blancs.}                         \\
  Cuis.                    & \makecell{Colline 
                             \\d'Avize.}                 &                                                             \\
  Grauves.                 & \makecell{—}                &                                                             \\
                           &                             &                                                             \\
  Champillon.              & \makecell{Montagne 
                             \\de Reims.}                & \multirow{6}{12em}{Raisins noirs.}                          \\
  Hautvillers.             & \makecell{—}                &                                                             \\
  Cumières,                & \makecell{Vallée 
                             \\de la Marne.}             &                                                             \\
  Monthelon.               & \makecell{—}                &                                                             \\
  Vertus.                  & \makecell{Colline d'Avize.} &                                                             \\
  Trépail.                 & \makecell{Montagne                                                                         
                             \\ de Reims.}               &                                                             \\
\end{longtable}
\normalsize

\paragraph{ Troisièmes crus.}

\scriptsize
\begin{longtable}{m{10em}m{12em}m{12em}}                                                    
  Trépail.                 & \makecell{Montagne 
                             \\ de Reims.}               & \multirow{2}{12em}{Raisins blancs.}                         \\
  Villers-sur-Marmerv.     & \makecell{—}                &                                                             \\
                           &                             &                                                             \\
  Moussv                   & \makecell{Vallée 
                             \\de la Marne.}             & \multirow{6}{12em}{Raisins noirs.}                          \\
  Sacy.                    & \makecell{Montagne 
                             \\ de Reims.}               &                                                             \\
  Coulommes.               & \makecell{—}                &                                                             \\
  Écueil.                  & \makecell{—}                &                                                             \\
  Pargny.                  & \makecell{—}                &                                                             \\
  Villedommange.           & \makecell{—}                &                                                             \\
\end{longtable}
\normalsize

On peut assurément champagniser tous les vins (c'est une industrie très
répandue aujourd'hui) ; mais, sans les crus de Champagne, il est impossible
d'obtenir des vins mousseux comparables aux vins de Champagne proprement
dits\footnote{ Pour éviter les fraudes, la loi n'autorise l'inscription du mot
« Champayne » sur l'étiquette et sur le bouchon que pour les vins récoltés et
traités en Champagne.}.

Les véritables vins de Champagne, d'une finesse inouïe, délicieux au goût,
frais, transparents, sont stimulants et digestifs. Ils doivent être servis
frais, comme tous les vins blancs, mais non glacés\footnote{L'habitude, que
beaucoup de personnes ont de boire le champagne frappé, n'est justifiée que
lorsqu'on a affaire à des vins trop verts et qui ont reçu une forte addition de
sucre ; c'est notamment le cas des tisanes. En effet, le froid empêche la
dissociation désagréable de la sensation de verdeur ou d'amertume du vin de
celle de la douceur du sucre.}, le froid faisant disparaître ou atténuant tout
au moins leur délicat arome. Ils jouissent de la propriété charmante de mettre
en liesse, de faire voir la vie en rose. Combien de malades leur ont dû leurs
dernières illusions ! L'ivresse produite par leur abus est gaie, mais cet abus
a pour inconvénient d'augmenter sérieusement la tension artérielle. La plupart
des accidents survenus en parties fines leur sont imputables, et presque tous
les grands buveurs de champagne sont des hypertendus. On s'empoisonne même avec
des fleurs !

Au point de vue pratique, la classification des crus importe relativement peu
pour le consommateur ordinaire. Chaque marque commerciale a ses formules, ses
dosages, l'art du fabricant consistant à combiner les cuvées avec des mélanges
de raisins de crus variés qui apportent chacun sa note propre de façon
à concourir à la production d’un ensemble aussi harmonieux que possible. Il
suffit de distinguer les marques et les années ; au point de vue des marques,
les goûts varient beaucoup ; au point de vue des années, les meilleures depuis
{\ppp1884\mmm} ont été :

\medskip
\begin{itemize}
\scriptsize
\item[ ]{\ppp1\mmm} {\ppp884\mmm}, léger, élégant, très grand vin.
\item[ ]{\ppp1\mmm} {\ppp889\mmm}, grand vin corsé avec beaucoup de sucre naturel. À vieilli.
\item[ ]{\ppp1\mmm} {\ppp892\mmm}, très fin et assez corsé. L'un des meilleurs du siècle. Resté assez frais.
\item[ ]{\ppp1\mmm} {\ppp893\mmm}, moins élégant et moins fin que le précédent, mais très plein.
\item[ ]{\ppp1\mmm} {\ppp895\mmm}, analogue au vin de {\ppp1\mmm} {\ppp893\mmm}, mais n'a pas réussi partout. Il a pris rapidement le goût du vieux.
\item[ ]{\ppp1\mmm} {\ppp898\mmm}, léger, élégant, fruité, grande qualité. Il ne s'est développé que lentement.
\item[ ]{\ppp1\mmm} {\ppp900\mmm}, très fruité, très moelleux, à cause de sa grande teneur en sucre naturel.
\item[ ]{\ppp1\mmm} {\ppp904\mmm}, bonne qualité. Parfois un peu dur, mais élégant, A quelquefois vieilli assez vite.
\item[ ]{\ppp1\mmm} {\ppp906\mmm}, très bonne année ; moelleux.
\item[ ]{\ppp1\mmm} {\ppp911\mmm}, un peu dur ; se corrigera probablement en bouteille.
\item[ ]{\ppp1\mmm} {\ppp914\mmm}, bonne qualité, mais rare.
\item[ ]{\ppp1\mmm} {\ppp915\mmm}, certaines cuvées excellentes. Sélection à faire.
\end{itemize}

\medskip

Ici encore, les indications qui précèdent n'ont pas une valeur absolue, car
dans certaines bonnes années des fabricants ont pu moins bien réussir leurs
mélanges que dans d’autres.

Parmi les vieux vins de Champagne, introuvables aujourd'hui et qui, du reste,
seraient probablement passés, je tiens à signaler un Cliquot sec
{\ppp1\mmm} {\ppp869\mmm} que j'ai goûté il y à une vingtaine d'années. Je
crois vraiment n'avoir jamais rencontré son pareil ; il m'a laissé un souvenir
inoubliable.

La plupart des maisons classent leurs produits en champagnes brut, extra-dry ou
dry (goût dit anglais ou américain), champagnes secs ou demi-secs (goût dit
français), champagnes doux et tisanes.

Peu de crus, même dans les meilleures années, possèdent assez de sucre naturel
pour donner des vins gazéifiés parfaits et si certains « brut » naturels sont
excellents, ils sont fréquemment durs. Les différentes classes de champagne
s'obtiennent généralement avec des vins plus ou moins capiteux additionnés
d'une proportion variable de sucre candi cristallisé dissous dans du vieux vin
de Champagne ou, quelquefois, dans de la vieille eau-de-vie de Champagne. Ce
sont naturellement les vins les plus secs, ceux dont le sucrage artificiel est
le moindre qui conservent le mieux les qualités fondamentales des vins de
Champagne naturels. Les vins doux et les tisanes sont produits avec des vins
légers des cuvées les moins bonnes. Aussi, certains vins de Vouvray et de
Saumur sont-ils préférables à certaines tisanes de Champagne.

\section*{\centering Vins des côtes du Rhône.}

La partie des côtes du Rhône qui produit les meilleurs vins de la région
s'étend du nord de l'Ardèche et du Lez, à droite et à gauche du Rhône.

Le sol est en partie granitique, en partie calcaire, ce dernier appartenant à
l'étage jurassique.

Les cépages sont : la \textit{petite syrah} pour les raisins rouges, la
\textit{roussanne} pour les raisins blancs.

\subsection*{\centering \small\sc Vins rouges.}

\scriptsize
\begin{longtable}{m{10em}m{12em}m{12em}}                                                    
  Ermitage.                & \makecell{(Drôme.)}      & Vignoble créé au \textsc{xiii}\textsuperscript{e} siècle. 
                                                        Le vin de l'Ermitage a une robe vive et éclatante ; il est 
                                                        très généreux ; un peu amer quand il est jeune, il prend en 
                                                        vieillissant du moelleux et un bouquet délicat.                          
                                                        (A été chanté par Boileau.)                                    \\
                           &                          &                                                                \\
  Cornas.                  & \makecell{(Ardèche.)}    & \multirow{3}{12em}{Ces trois vins ont une certaine analogie 
                                                        avec le précédent.}                                            \\              
  Maures.                  & \makecell{—}             &                                                                \\
  Saint-Joseph.            & \makecell{—}             &                                                                \\
\end{longtable}
\normalsize
                                                                                                        
\subsection*{\centering \small\sc Vins blancs.}

\scriptsize
\begin{longtable}{m{10em}m{12em}m{12em}}                                                    
  Ermitage.                & \makecell{(Drôme.)}     & \multirow{3}{12em}{Vins corsés, liquoreux, rappelant certains 
                                                       des meilleurs vins d'Espagne.}                                  \\
  Recoulès.                & \makecell{—}            &                                                                 \\
  Murets,                  & \makecell{—}            &                                                                 \\
                           &                         &                                                                 \\
  Saint-Péray.             & \makecell{(Ardèche.)}   & Spiritueux, champagnisé.                                        \\
\end{longtable}
\normalsize

\section*{\centering Vins du Jura.}

Le sol du Jura, dans ses grandes lignes, est un sol marneux liasique. Le climat
est plutôt dur et humide. Cependant, certains vins du Jura sont intéressants.

Les cépages qui fournissent les meilleurs vins sont : le \textit{poulsard
noir}, le \textit{pineau blanc chardonay}, le \textit{savagnin blanc} et le
\textit{trousseau}.

\subsection*{\centering \small\sc Vins rouges.}

\scriptsize
\begin{longtable}{m{10em}m{12em}m{12em}}                                                    
  Arbois.                  &                     & Considéré comme le meilleur des vins rouges du Jura. 
                                                   Un peu vert quand il est jeune, il prend du moelleux en                  
                                                   vieillissant. On peut le conserver jusqu'à {\ppp50\mmm} ans. 
                                                   Il a été étudié par Pasteur.                                        \\
                           &                     &                                                                     \\
  Poligny.                 &                     & Rappelle un peu, quand il a vieilli, certains vins des côtes du 
                                                   Rhône.                                                              \\
\end{longtable}
\normalsize

\subsection*{\centering \small\sc Vins blancs.}

\scriptsize
\begin{longtable}{m{10em}m{12em}m{12em}}                                                    
  Arbois.                  &                     & Robe jaune ; très bouqueté ; appelé quelquefois le madère du Nord.  \\
                           &                     &                                                                     \\
  Château-Châlon.          &                     & Très corsé, très bouqueté, a un goût de noisette rappelle le vin 
                                                   de Johannisberg. On peut le conserver cent ans.                     \\
\end{longtable}
\normalsize

On fabrique aussi dans le Jura beaucoup de vins mousseux dits « de l'Étoile »,
du nom de l’un des crus utilisés, qui sont très agréables, et un vin de paille,
qui n'a guère son pareil, mais dont la production est malheureusement très
limitée,

\section*{\centering Vins de Touraine.}

Les vignes produisant les vins rouges de Touraine sont cultivées au sud de la
Loire, dans les vallées, sur des alluvions anciennes, ou sur des coteaux
crétacés : celles qui produisent les vins blancs sont cultivées dans des
terrains siliceux ou argileux, au nord de la Loire.

Les vins rouges de choix proviennent d'un cépage du Bordelais, le
\textit{cabernet franc} ou \textit{breton} ; ils rappellent le vin de Bordeaux ;
mais, s'ils ont plus de vivacité et de fraîcheur, ils ont aussi moins de corps
et de bouquet.

Les vins blancs proviennent du \textit{pineau blanc} ; ils sont frais et
parfumés ; leur parfum rappelle celui du coing.

\medskip

Parmi les vins les plus estimés, il faut noter :

\subsection*{\centering \small\sc Vins rouges.}

\scriptsize
\begin{longtable}{m{10em}m{12em}m{12em}}                                                    
  Bourgueil.               & \makecell{Chinonais.}          & \multirow{2}{12em}{Les vins de Chinon rappellent 
                                                              le médoc avec un parfum framboisé spécial.}              \\
  Saint-Nicolas.           & \makecell{—}                   &                                                          \\
\end{longtable}
\normalsize

\subsection*{\centering \small\sc Vins blancs.}

\scriptsize
\begin{longtable}{m{10em}m{12em}m{12em}}                                                    
  Vouvray.                 & \makecell{Pays 
                             \\ de 
                             \\ Tours.}                     & Les vins de Vouvray sont récoltés sur les coteaux 
                                                              de Vouvray, de Rochecorbeau et de Montlouis. Secs et 
                                                              pétillants ou liquoreux, suivant les années, ils sont 
                                                              capiteux, parfumés et ont ce qu'on appelle « un goût 
                                                              de pierre à fusil ». Ils ont été célébrés par Rabelais 
                                                              qui les adorait et leur trouvait « un moelleux de 
                                                              taffetas ».                                              \\
\end{longtable}
\normalsize

\section*{\centering Vins d'Anjou.}

Les vins d'Anjou ont pour terroir des coteaux schisteux.

La réputation des vins d' Anjou est surtout due à ses vins mousseux qui sont
pétillants et frais, mais qui manquent en réalité de moelleux et de saveur. Ils
ne sauraient, en aucune manière, être comparés aux vins de Champagne avec
lesquels leurs admirateurs fanatiques les mettent parfois en parallèle.

Parmi les vins non mousseux, rouges ou blancs, quelques-uns sont dignes
d'attention.

Les meilleurs vins rouges d'Anjou proviennent du cépage \textit{breton} ; les
meilleurs vins blancs, du \textit{chenin} blanc.

Comme vins rouges, on peut citer :

\scriptsize
\begin{longtable}{m{10em}m{12em}m{12em}}                                                    
  Clos des Cordeliers.     & \makecell{Environs 
                             \\ de Saumur.}                 &                                                          \\
  Château-Parnay.          & \makecell{—}                   &                                                          \\
\end{longtable}
\normalsize

Comme vins blancs :

\scriptsize
\begin{longtable}{m{10em}m{12em}m{12em}}                                                    
  Savenniéres.             & \makecell{Environs             
                             \\ d'Angers.}                  & \multirow{4}{12em}{Comparable certaines années aux 
                                                              grand sauternes.}                                        \\
  Coulée-de-Serrant        & \makecell{—}                   &                                                          \\
  Espiré.                  & \makecell{—}                   &                                                          \\
  Roche-aux-Moines.        & \makecell{—}                   &                                                          \\
                           &                                &                                                          \\
  Brézé                    & \makecell{Environs            
                             \\ de Saumur.}                 & Remarquable.                                             \\
\end{longtable}
\normalsize

\section*{\centering Vins d'Alsace.}

L'Alsace produit surtout des vins blancs qui sont très appréciés dans le pays.

\medskip

Les plus estimés sont :

\scriptsize
\begin{longtable}{m{10em}m{12em}m{12em}}                                                    
  Riesling de Riquewihr.   &                     & Le plus parfumé, le plus léger et le plus frais des vins d'Alsace. 
                                                   Il rappelle certains vins du Rhin.                                  \\
                           &                     &                                                                     \\
  Riesling de Ribeauvillé. &                     & Analogue au précédent.                                              \\
                           &                     &                                                                     \\
  Kitterlé de Guebwiller.  &                     & \multirow{2}{12em}{Vins blancs secs, ayant un petit 
                                                   goût de noisette.}                                                  \\
  Rangwein de Thann.       &                     &                                                                     \\
                           &                     &                                                                     \\
  Brand de Turkheim.       &                     & Analogue aux deux précédents.                                       \\
                           &                     &                                                                     \\
  Finkenwein d'Avolsheim.  &                     & Vin blanc corsé, très agréable.                                     \\
\end{longtable}
\normalsize

On fabrique aussi en Alsace des vins de paille qui rappellent le vin de Tokay.
                           
\section*{\centering Vins de Lorraine.}

La Lorraine produit des vins rouges et des vins blancs.

\subsection*{\centering \small\sc Vins rouges.}

\scriptsize
\begin{longtable}{m{10em}m{12em}m{12em}}                                                    
  Thiaucourt.              &                        & \multirow{3}{12em}{Possèdent les qualités des grands vins :
                                                      bouquet, alcool, couleur.}                                       \\
  Pagny.                   &                        &                                                                  \\
  Guentrange.              &                        &                                                                  \\
\end{longtable}
\normalsize

\subsection*{\centering \small\sc Vins blancs.}

\scriptsize
\begin{longtable}{m{10em}m{12em}m{12em}}                                                       
  Guentrange.              &                        & Excellent.                                                       \\
                           &                        &                                                                  \\
  Dormot.                  &                        & Vin exceptionnel, rappelant le chablis ; provient de 
                                                      l'\textit{auxerrois} blanc.                                      \\
                           &                        &                                                                  \\
  Klang.                   &                        & Cépage \textit{riesling}.                                        \\
\end{longtable}
\normalsize

\section*{\centering Quelques vins intéressants d'autres régions.}

\scriptsize
\begin{longtable}{m{10em}m{12em}m{12em}}                                                    
  Saint-Pourçain.          & \makecell{(Allier.)}   & Vin blanc, sec, pétillant.                                       \\
                           &                        &                                                                  \\
  Blanquette de Limoux.    & \makecell{(Aude.)}     & Vin blanc, très doux, très bouqueté.                             \\
                           &                        &                                                                  \\
  Miserey.                 & \makecell{(Doubs.)}    & Vin blanc, saveur exquise, robe brillante. Analogue aux 
                                                      vins blancs d'Arbois.                                            \\
                           &                        &                                                                  \\
  Tavel.                   & \makecell{(Gard.)}     & Vin rouge, peu coloré, corsé, très agréable.                     \\
                           &                        &                                                                  \\
  Lunel.                   & \makecell{(Hérault.)}  & Traité autrefois comme vin doux ; vendu naturel 
                                                      actuellement.                                                    \\
                           &                        &                                                                  \\
  Jurançon.                & \makecell{(Basses- 
                             \\ Pyrénées)}          & Vin blanc renommé dans la contrée, rosé, capiteux, 
                                                      ayant un parfum de truffe. L'histoire nous apprend 
                                                      que, lorsque Henri IV vint au monde, on lui frotta 
                                                      les lèvres avec une gousse d'ail et on les lui 
                                                      mouilla avec du vin de Jurançon.                                 \\ 
                           &                        &                                                                  \\
  Banyuls.                 & \makecell{(Pyrénées-   
                             \\ Orientales.)}       & Vin rouge de liqueur, stimulant et tonique. Très 
                                                      foncé quand il est jeune, il prend en vieillissant 
                                                      une couleur tirant sur le jaune et un bouquet spécial 
                                                      dit de \textit{rancio}.                                          \\
                           &                        &                                                                  \\
  Rivesaltes.              & \makecell{(Pyrénées-                       
                             \\ Orientales.)}       & Vin blanc de liqueur, fin, généreux et parfumé.                  \\
                           &                        &                                                                  \\
  Clos des Altesses.       & \makecell{(Savoie.)}   & Liquoreux et pétillant.                                          \\
                           &                        &                                                                  \\
\end{longtable}
\normalsize

\section*{\centering Service des vins.}

On a écrit des volumes sur le service des vins. Certains auteurs ont émis la
prétention d'imposer des crus déterminés pour les différentes phases des repas.
Il me paraît excessif d'aller jusque-là, d'autant plus que le conseil est plus
facile à donner qu'à suivre. Ce qui est certain, c'est qu'il faut absolument
éviter des antinomies fâcheuses. L'hôte qui servirait du vin de Banyuls avec
des huîtres, sous prétexte que son banyuls est bon, un vin liquoreux avec du
gibier ou un fin bourgogne avec une crème ferait preuve d'une grande ignorance
gastronomique.

Bien des personnes considèrent le madère, le porto, etc., comme devant être
servis après le potage. A mon avis, c'est une véritable hérésie. Pris au
commencement du repas, ils empâtent la bouche et empêchent d'apprécier les
autres vins. Un Château Yquem un peu sec, un Château Suduiraut, un barsac, un
montrachet ou un meursault seraient là tout à fait à leur place. Les vins
blancs secs conviennent avec les huîtres et le poisson ; les bordeaux rouges
avec les viandes de boucherie, la volaille et les légumes ; les bourgognes
rouges et les côtes du Rhône avec les mets très relevés, le gibier et le
fromage ; les vins de liqueur et les vins mousseux avec les desserts. Mais il
est entendu que l'on peut faire un excellent repas avec un seul vin,
à condition que ce ne soit pas un vin sucré. J'ai souvenance d'un excellent
déjeuner au champagne et d’un dîner remarquable arrosé du commencement à la fin
d'un vieux vin de Pontet-Canet. Cependant, il est bien certain qu'avec un même
vin, à la longue, la langue se sature.

Dans un repas d'amateurs modestes, on pourra s'en tirer, par exemple, avec un
chablis, un meursault ou un suduiraut comme vins blancs ; un Saint-Émilion, un
{\ppp2\mmm}\textsuperscript{e} ou un {\ppp3\mmm}\textsuperscript{e} cru de
bordeaux, une côte-rôtie ou un corton, comme vins rouges.

Dans un repas de fins gourmets, on pourra servir quelques-unes des perles de
notre écrin : vins de Haut-Brion, de Margaux, de Lafite, de La Tour d'Yquem, de
l'Ermitage, de Romanée-Conti, de Chambertin, de Musigny, de Montrachet.

Mais, quels que soient les vins, les bourgognes rouges doivent être servis au
sortir de la cave ; les bordeaux rouges doivent être chambrés, puis servis
décantés en carafe ; tous les vins blancs doivent être servis frais.

En ce qui concerne les vins ordinaires, je ne saurais les admettre médiocres et
je préfère boire tout simplement de l'eau pendant le repas, quitte à finir, si
je le puis, par un petit verre de bon vin, plutôt que d'absorber des gros vins
ou des vins de synthèse étendus d'eau. Comme vins de table, je considère comme
très recommandables un petit chablis, un beaujolais rouge ou un Saint-Julien
non classé. En province, beaucoup de vins de pays, généralement
intransportables, conviennent très bien.

\section*{\centering La question de la cave.}

Doit-on se faire une cave, c'est-à-dire acheter du vin en pièces, le soigner
jusqu'au moment de la mise en bouteilles et le laisser vieillir, ou l'acheter
au fur et à mesure des besoins, quitte à le payer un peu plus cher ?

À moins d'avoir une fortune permettant d'entretenir un sommelier connaissant
parfaitement son affaire et de mettre à sa disposition un budget annuel sérieux
et une cave parfaite\footnote{Une cave parfaite doit être, avant tout, à l'abri
des trépidations, comme le sous-sol d'un observatoire astronomique ; elle doit
être aérée sans l'être trop, sans ouvertures autres que vers le Nord ou vers
l'Est ; sa température doit être aussi constante que possible : ne jamais
descendre au-dessous de {\ppp10\mmm}° C, ni monter au-dessus de
{\ppp16\mmm}° C. ; d'une façon générale, elle doit être sèche ; toutes
conditions difficiles à réaliser.

Si la cave est imparfaitement sèche, il est indispensable de soigner tout
particulièrement le cachetage des bouteilles pour éviter l’action néfaste de
l'humidité sur les bouchons.}, le plus simple et le plus sûr est de s'adresser
à des fournisseurs compétents et consciencieux (il en existe), qui peuvent
livrer, en temps utile, des vins naturels ordinaires honnêtement préparés,
possédant toutes les qualités hygiéniques que l'on est en droit de leur
demander, et des vins fins, à point, de crus authentiques et d'années
déterminées\footnote{ Un jour que nous avions à dîner un médecin éminent et
dégustateur renommé, je fis servir des vins que j'étais allé chercher chez mon
fournisseur et que j'avais décantés moi-même pour éviter toute indiscrétion.
Sans la moindre hésitation ni la moindre erreur, notre ami détermina les crus
et les années.

Cette expérience fait autant d'honneur au dégustateur qu'au commerçant.}. On
évitera de la sorte le désagrément de voir quelquefois se perdre toute une
pièce de vin et l’on aura ainsi, à sa disposition, un choix de vins impeccables
qu'il est à peu près impossible d'avoir autrement. Cette manière de faire m'a
été suggérée par l'ami dont il est question dans la note ci-dessous. A cause
des cahots des voitures et de la chaleur des calorifères, il a absolument
renoncé à avoir une cave. Je m'appuie sur sa haute autorité et sur mon
expérience personnelle pour recommander cette méthode qui est la solution
pratique de la question pour les habitants des villes. Seuls, les œnophiles
vivant à la campagne peuvent, raisonnablement, avoir une cave chez eux.

