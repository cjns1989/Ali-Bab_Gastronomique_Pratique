\sk

\bigskip

L'obésité est l'état du corps humain dans lequel la proportion de graisse des
différents organes se trouve en excès notable sur la moyenne normale.

On a beaucoup écrit sur la question et, comme il arrive souvent en pareil cas,
on a émis les opinions les plus contradictoires.

Ce qui paraît incontestable, c'est l'existence de plusieurs variétés d'obésité,
suivant les causes qui les produisent.

Bien souvent, le mal est le résultat d’un vice de nutrition. Cependant, on voit
fréquemment des gens, paraissant jouir d'un excellent estomac et d'intestins en
parfait état, dont toutes les fonctions semblent s’accomplir à merveille, qui,
mangeant trop, deviennent obèses, à l'exemple des bêtes primées dans les
concours d'animaux gras.

C'est exclusivement de cette variété d'obésité, obésité des gourmands, dont je
m'occuperai ici.

Étudions d'abord la marche des phénomènes.

Les candidats à l'obésité, souvent très maigres dans leur jeune âge,
engraissent généralement tôt, mais ils peuvent ne pas se faire remarquer par
des proportions anormales avant l’âge de {\ppp25\mmm} ou {\ppp30\mmm} ans ;
pourtant, il est bien rare qu'ils n'aient pas manifesté leur tendance
à l'engraissement avant la trentaine. Ils commencent par devenir d'abord
simplement grassouillets, la graisse se développant, au début, surtout dans les
tissus sous-cutanés ; comme ils sont encore jeunes et que tout va bien à la
jeunesse, cela ne leur messied pas trop ; ils font presque envie. Plus tard,
ils s'épaississent et ils deviennent lourds ; la graisse se loge autour de tous
les viscères et elle envahit les enveloppes de la cavité abdominale ; alors les
premiers symptômes inquiétants apparaissent.

Les gens gras soufflent et transpirent au moindre effort ; ils s'enrhument avec
la plus grande facilité ; ils perdent progressivement de leur activité, la
marche leur devient pénible et même odieuse ; ils dorment après leurs repas et
ils finissent par ne plus avoir la moindre énergie. Cependant, ils conservent
encore, au moins en partie, leurs facultés intellectuelles et leur humeur
enjouée. Ils produisent, sans rien faire pour cela, un effet comique
irrésistible ; ils font rire. Ils se laissent aller de plus en plus à leur
péché mignon et la graisse s'infiltre dans les cellules mêmes de leurs tissus.
De gras qu'ils étaient, ils se transforment en véritables obèses ; leur
circulation se fait de plus en plus mal, leurs facultés intellectuelles
s'émoussent notablement, ils perdent leur bonne humeur, ils deviennent
impotents, infirmes et ils n'inspirent plus que la pitié.

Un de mes meilleurs amis, âgé de {\ppp51\mmm} ans, mesurant
{\ppp1\mmm}\textsuperscript{m}, {\ppp80\mmm} à la toise, ayant toujours joui
d'une bonne santé, d'un excellent estomac et d'un appétit à la hauteur de son
estomac, s'était laissé aller, tout jeune, à la gourmandise. Du poids de
{\ppp80\mmm} kilogrammes qu'il avait à {\ppp18\mmm} ans, il était passé
à {\ppp120\mmm} kilogrammes à {\ppp27\mmm} ans, et à {\ppp49\mmm} ans il avait
acquis, tout nu, le poids formidable de {\ppp150\mmm} kilogrammes, avec
{\ppp1\mmm}\textsuperscript{m}, {\ppp48\mmm} de tour de ceinture. C'était un
type d’obèse gros mangeur.

À plusieurs reprises, dans le courant de sa vie, il avait essayé de réagir
contre cet état : il avait consulté des maîtres, il avait même suivi leurs
conseils, mais, malgré tout, son poids n'avait jamais diminué d’une façon
notable et durable.

L'un de ses consultants, savant grave à lunettes, après l'avoir soigneusement
examiné et palpé des pieds à la tête, lui déclara d'abord qu'il le considérait
comme un dilaté et un accéléré, ce qui, paraît-il, expliquait tout ; puis,
après avoir étudié la composition de ses urines, il pencha pour le classer
plutôt parmi les rétrécis et les ralentis, ce qui, du reste, devait également
jeter la lumière sur les phénomènes observés.

En fin de compte, il lui fit, sans rire, l'ordonnance suivante :

\textit{a}) On favorisera les mutations nutritives par des modificateurs
généraux.

\textit{b}) On prendra tous les matins, à jeun, un verre d'eau de Carlsbad
chauffée à {\ppp37\mmm}° C. au bain-marie (on fera également chauffer le
verre).

Un autre l'envoya dans une station thermale, où à force de purgations, de
massages, d'exercices, de marches forcées en terrain accidenté et au soleil,
sans boire ni manger, il arriva à perdre quelques kilogrammes qu'il rattrapa
bien vite, du reste, dès son retour à Paris,

Un essai loyal de régime sec lui valut un bel accès de goutte.

Il prit des pilules dites fondantes, essaya la médication thyroïdienne et la
médication iodurée qui heureusement ne lui firent pas de mal, mais n'eurent
aucune action sur son obésité.

Il était tout à fait découragé, lorsqu'il eut l'idée de me demander conseil.

J'avais fort heureusement et depuis longtemps médité sur la question ; aussi je
pus, en me donnant des airs de savant, lui faire \textit{ex abrupto} le petit
discours suivant.

Il paraît à peu près établi d'une façon expérimentale qu un homme de carrure
moyenne, ayant {\ppp1\mmm}\textsuperscript{m}, {\ppp75\mmm} à la toise et pesant
{\ppp75\mmm} kilogrammes, consomme par jour, au repos, pour le travail de son
cœur, ses échanges gazeux respiratoires, etc. {\ppp2\mmm} {\ppp200\mmm}
calories : on peut en déduire plus ou moins exactement le nombre correspondant
de calories pour un homme d'une taille et d'un poids donnés.

De plus, on connaît la composition de la chair musculaire et celle des
différents aliments ; on connaît aussi plus ou moins les équivalents caloriques
de la plupart d’entre eux ; dans ces conditions, il est facile de constituer un
régime alimentaire ayant une composition moyenne élémentaire analogue à celle
de la ration d'entretien\footnote{{\ppp72\mmm} pour {\ppp100\mmm} d'hydrates de
carbone, {\ppp18\mmm},{\ppp50\mmm} pour {\ppp100\mmm} d'albumine,
{\ppp9\mmm},{\ppp90\mmm} pour {\ppp100\mmm} de graisse.} et dont l'intégralité
représente un nombre de calories inférieur au nombre de calories nécessaires.

Il y aurait ainsi dans la balance un déficit de calories que l'organisme serait
obligé de prendre sur lui-même.

Il est bien certain, comme l'a écrit spirituellement un médecin, en faisant la
critique de certains régimes prescrits contre l'obésité, que si l'on pouvait se
contenter le matin d'un verre d'eau avec rien et le soir du même régime sec, on
maigrirait. Succi et Merlatti l'ont démontré péremptoirement, mais ils ont
montré en même temps l'absurdité du système.

Tout régime excessif manque son but, la fatigue exagérée épuise, les exercices
violents augmentent les sensations de faim et de soif, ce qu'il importe
d'éviter. Il convient donc d'adopter un régime modéré, basé sur des données
scientifiques, les unes théoriques, les autres expérimentales, et combiné de
façon qu'on puisse le suivre sans dégoût et sans trop de peine.

Le corps humain est loin d'être une cornue inerte et, dans l'équation
alimentaire, nous ne connaissons à peu près que le premier terme et le dernier,
comprenant l'un la composition chimique centésimale des produits ingérés,
l'autre celle des excreta. L'expérience, en contradiction sur ce point avec la
théorie, montre que les farineux en général et, en particulier, le pain
favorisent l'engraissement beaucoup plus que ne devrait le faire craindre la
quantité correspondante de calories absorbées.

Un certain nombre de savants attribuent le phénomène à l'indigestibilité du
pain, mais cette explication ne me satisfait pas, car je connais des obèses,
gros mangeurs de pain, des gaillards ayant de véritables estomacs de casoar,
pour qui le mot « indigeste » est vide de sens.

Quoi qu'il en soit, il résulte de ces faits qu'il importe de réduire la
consommation des farineux et surtout celle du pain.

Toutefois, pour éviter une sensation de faim trop pénible, il est bon de
permettre un volume relativement considérable de nourriture, en prenant comme
base de l'alimentation les légumes verts et les fruits, qui nourrissent peu. Il
convient, bien entendu, d'y adjoindre des aliments carnés, de façon que
l'organisme n'ait à emprunter que le moins possible aux muscles, mais il faut
en limiter la quantité.

En ce qui concerne la boisson, contrairement à l’opinion des partisans du
régime sec, j'estime qu'il faut boire, de façon à bien laver les reins,
condition indispensable au bon fonctionnement de notre machine. Reste seulement
à savoir à quel moment on doit boire.

Un fait d'observation banale montre que l'on mange moins lorsqu'on ne boit pas
en mangeant ; c'est là une raison suffisante pour boire seulement entre les
repas. Les physiologistes attribuent à l'absorption des liquides pendant le
repas une action défavorable sur la digestion, parce que, disent-ils, le suc
gastrique dilué perd de son activité ; c'est assez vraisemblable. Certains
savants assurent que l'avantage des repas secs est de diminuer notablement la
putréfaction intestinale azotée ; c'est bien possible. Au point de vue qui nous
occupe, la première raison me suffit complètement.

En vue de réduire l'absorption des liquides à table, il convient également de
limiter l'usage des potages.

Voici, en termes généraux, la ligne de conduite à suivre.

1° \textit{Alimentation}. — Il faut réduire le plus possible la quantité des
farineux ; diminuer progressivement la ration de pain et les pâtisseries ;
finalement en supprimer l'usage.

Les pâtes, les pommes de terre et le riz engraissant relativement moins que le
pain, tu pourras en manger quelquefois, mais modérément.

Il faut mâcher soigneusement ; ne pas rester plus d'une demi-heure à table et
la quitter toujours sans être rassasié.

Ces réserves faites, tu peux manger de toutes les viandes de boucherie ou de
porc, rôties, braisées ou bouillies, du jambon, de la volaille, du gibier (en
évitant cependant le gibier trop avancé qui est indigeste), des œufs, de temps
en temps, mas pas trop souvent, du poisson de mer, du poisson d'eau douce, des
coquillages, etc.

Tous les légumes verts sont permis et même recommandés. Du moment que tu les
digères bien, tu peux manger haricots verts, petits pois, asperges, artichauts,
différentes variétés de chou, chou vert, chou rouge, chou frisé, chou cabus,
chou-fleur, brocoli, choux de Bruxelles, chou confit, choucroute, oignons,
poireaux, carottes, navets, radis, concombres, raves, salsifis, cardons,
rhubarbe, céleri, fenouil, épinards, tomates, aubergines, betteraves,
champignons, etc. Toutefois, il est préférable de ne pas abuser de l'oseille,
à cause de la forte proportion d'acide oxalique qu'elle contient.

Toutes les salades crues ou cuites sont permises ; les salades crues
accommodées de préférence au jus de citron conviennent très bien.

Enfin, tu peux manger de lous les fruits crus ou cuits.

Il faut réduire le plus possible la boisson à table, supprimer les potages, et
t'habituer à ne pas boire du tout en mangeant, ce qui est facile.

Il faut réduire également au minimum la consommation de l'alcool, du sucre et
du laitage.

Mais il est essentiel de boire ; tu boiras donc seulement entre les repas,
d'abord le matin, au réveil, puis une heure ou deux après chacun des deux
principaux repas, enfin le soir en te couchant. La quantité de liquide
nécessaire pour te désaltérer pourra varier entre un et deux litres par jour.
Comme boissons chaudes, tu donneras la préférence à des infusions légères de
thé ou à des tisanes quelconques ; comme boissons froides, tu prendras des eaux
minérales, de l'eau pure ou aromatisée, de la citronnade, de l'orangeade,
etc. ; jamais de bière.

2° \textit{Sommeil}. — Sept à huit heures de sommeil en moyenne doivent
suffire.

\index{Exercices prescrits aux obèses gros mangeurs voulant maigrir}
3° \textit{Exercices}. — Le défaut de la plupart des régimes est de prescrire,
dès le début, des exercices pénibles. En procédant ainsi, on décourage
immédiatement des gens dont la volonté est généralement chancelante. Il ne faut
pas perdre de vue, en effet, qu'un gros obèse est matériellement incapable de
fournir sans entraînement un travail physique tant soit peu important, et qu'il
faut absolument lui éviter en même temps qu'une fatigue trop grande un
accroissement exagéré de l'appétit.

Le meilleur exercice est incontestablement la marche, mais il convient d'en
augmenter la durée progressivement : en agissant ainsi, tu arriveras en peu de
temps à faire avec plaisir et grand profit huit à dix kilomètres à pied, par
jour.

Dix minutes ou un quart d'heure d'exercices modérés et gradués faits, de
préférence, au moment du coucher, soit avec un appareil de gymnastique de
chambre, soit sans aucun appareil, produisent un excellent effet au point de vue
de l’assouplissement général du corps.

Le massage et surtout l’auto-massage des régions plus particulièrement envahies
par la graisse est également un bon exercice, à condition qu'il soit pratiqué avec
modération.

\begin{center}
\sc
Recommandations particulières.
\end{center}

Il faudra enfin :

\textit{a)} Surveiller tes urines, qui doivent toujours rester claires et de
couleur jaune citrin. En cas d'urines troubles ou rougeâtres, commence par
augmenter la quantité de liquide ingéré entre les repas ; si le phénomène
persiste, fais analyser les urines et consulte ton médecin.

\textit{b)} Surveiller la liberté du ventre. En cas d'irrégularité dans les
selles, fais usage de laxatifs, par exemple d'eau de Marienbad, source
Kreuzbrunn, à la dose d'un à quatre verres, le matin à jeun, à des intervalles
de dix minutes, en prenant avec chaque verre un cachet de {\ppp2\mmm} grammes
de sel de Marienbad. En cas de constipation réelle, consulte ton médecin.

\textit{c)} Prendre chaque jour, autant que possible dans les mêmes conditions de
vacuité du corps, ton poids, ton tour de ceinture et ton volume, cette dernière
mesure pouvant être obtenue facilement au moyen d'une baignoire graduée, et
porter tous ces éléments sur un tableau graphique qui te permettra de te rendre
compte d'un seul coup d'œil de la façon dont marche la cure.

Ces principes posés, si un homme de {\ppp75\mmm} kilogrammes consomme au repos
{\ppp2\mmm} {\ppp200\mmm} calories par jour, il paraît légitime d'admettre
qu'un homme de {\ppp150\mmm} kilogrammes en consomme au moins {\ppp3000\mmm}.

Dans ces conditions, voici, avec le coefficient de sécurité que donneront les
exercices prescrits, un exemple concret de régime sévère de début\footnote{Il
est bien entendu qu'il faudra diminuer plus ou moins les quantités d'aliments
indiquées, au fur et à mesure de l'amaigrissement.} pour un obèse pesant
{\ppp150\mmm} kilogrammes.

\medskip

\index{Exemple concret de régime alimentaire pour obèses gros mangeurs voulant maigrir}
\begin{center}
\textit{Déjeuner du matin.}
\end{center}

\medskip

{\ppp250\mmm} grammes de thé léger sans sucre et un petit croissant.


\medskip

\begin{center}
\textit{Déjeuner de midi.}
\end{center}

\medskip

{\ppp200\mmm} grammes de viande quelconque, préparée n'importe comment.

{\ppp300\mmm} grammes de légumes verts, blanchis à l'eau et sautés au beurre,
ou arrosés de beurre fondu.

Une salade quelconque.

{\ppp250\mmm} grammes de fruits crus, ou cuits sans sucre.

Vers deux heures de l'après-midi, une tasse ou deux de thé, d'infusion de
camomille, de tilleul, etc, et, au besoin, à cinq heures un verre d'eau ou de
limonade.

\medskip

\begin{center}
\textit{Dîner.}
\end{center}

\medskip

Deux œufs préparés d'une manière quelconque, ou un membre de poulet, ou un
petit poisson, ou, pendant la saison de la chasse, un petit gibier : une grive.
une caille, deux alouettes, la moitié d'un perdreau, etc.

{\ppp200\mmm} grammes de légumes verts sautés au beurre, ou en salade.

{\ppp250\mmm} grammes de fruits crus, ou cuits sans sucre.

\medskip

Dans la soirée, une tasse ou deux de thé ou d'une infusion quelconque, à la
rigueur un grog, et, en se couchant, un verre d'eau.

\medskip

Pas plus de {\ppp100\mmm} grammes de beurre par jour, en tout, au maximum.

\sk

Mon ami a suivi mes conseils et il a été pleinement récompensé de ses efforts.
Seuls, les trois premiers mois furent un peu pénibles ; il se plaignait
d'éprouver presque constamment une sensation de faim. Mais quel résultat ! Dès
le premier mois, il avait perdu
{\ppp7\mmm}\textsuperscript{kgr}, {\ppp500\mmm} ; il perdit
{\ppp5\mmm}\textsuperscript{kgr}, {\ppp500\mmm} le mois suivant
{\ppp4\mmm}\textsuperscript{kgr}, {\ppp500\mmm} le troisième, puis la diminution
fut un peu moins rapide. Quoi qu'il en soit, au bout de dix mois, tout en se
relâchant un peu de la rigueur du régime, il avait perdu {\ppp40\mmm}
kilogrammes et {\ppp40\mmm} centimètres de tour de ceinture.

Un peu plus tard, son poids diminua de {\ppp5\mmm} kilogrammes encore, son tour
de ceinture de {\ppp5\mmm} centimètres, et bien que la diminution de son poids
ait été relativement faible dans cette deuxième période, mon ami eut le plaisir
de constater une réduction considérable dans son volume. Ses muscles s'étaient
développés et ses formes s'étaient affinées,

Il est habitué maintenant à manger relativement peu, ou plutôt il s'est
déshabitué de manger trop ; il ne souffre plus du tout de la faim, il respire
sans aucune gêne et il ne transpire plus d'une façon anormale. La marche est
devenue pour lui un plaisir et douze kilomètres par jour ne lui font pas peur ;
il lui est même arrivé d'en faire trente dans une journée, sans éprouver trop
de fatigue. C'est une transformation et un véritable rajeunissement à tous les
points de vue.

J'ai eu l’occasion de faire suivre le même traitement à plusieurs autres obèses ;
tous s'en sont bien trouvés.

Je l'ai conseillé également, en le mitigeant un peu, à des dames qui, sans être
positivement obèses, étaient grasses et empâtées : Je les ai engagées
à diminuer simplement la quantité de pain, à supprimer les pâtisseries, les
potages et la boisson à table. Grâce à ces seules modifications dans leur
régime alimentaire, elles ont diminué en huit mois de {\ppp12\mmm}
à {\ppp15\mmm} kilogrammes de poids, de {\ppp12\mmm} à {\ppp15\mmm} centimètres
de tour de ceinture, ce qui a suffi à leur rendre le charme qu'elles avaient
perdu. Je citerai en particulier le cas d'une dame copieusement fessière, comme
aurait pu dire Rabelais, qui, au bout de dix-huit mois de régime bien observé,
a vu son tour de taille passer de {\ppp93\mmm} à {\ppp70\mmm} centimètres et
son tour de bassin de {\ppp1\mmm}\textsuperscript{m}, {\ppp40\mmm}
à {\ppp0\mmm}\textsuperscript{m}, {\ppp98\mmm} !

Si l'on examine d'un peu près les courbes des poids et des tours de ceinture,
on voit, d'une part, qu'elles varient parallèlement et que, le plus souvent,
à une diminution de poids d'un kilogramme correspond une diminution de tour de
ceinture d'un centimètre ; d'autre part, l'amaigrissement ne se produit jamais
d'une manière continue, mais par à-coups, séparés les uns des autres par des
oscillations.

Les variations diurnes atteignent normalement quelques centaines de grammes ;
exceptionnellement, elles peuvent dépasser {\ppp2\mmm} kilogrammes, mais alors
l'élévation du chiffre tient, en partie au moins, à une grande variation de
l’état hygrométrique de l'air.

La diminution de volume est toujours proportionnellement plus grande que la
diminution de poids, surtout lorsqu'il y a longtemps qu'on suit le régime.

\sk

Voilà les résultats de l'expérience, et je devrais peut-être m'en tenir là.
Mais, si je ne faisais pas au moins un essai de théorie, j'aurais l'air d’être
incomplet et de ne rien comprendre au mécanisme intime du phénomène, ce qui,
tout en étant rigoureusement vrai, est plutôt vexant. Le démon de la
métaphysique me tente et, s'il me faut une excuse pour me faire pardonner mon
imprudence, la meilleure est que ma théorie concorde assez bien avec les faits
observés : il est impossible, à la vérité, d'en démontrer l'exactitude, mais il
me semble impossible aussi de prouver qu'elle est fausse.

Pour moi, l'obésité des gourmands ne correspond pas à un vice organique de
nutrition. Tout dans le développement de l'affection et dans sa disparition se
passe comme si l'appareil régulateur du poids était simplement faussé sous
l'influence de la suralimentation et comme si le régime agissait en permettant
à cet appareil de reprendre son fonctionnement normal. L'obésité des gourmands
ne serait donc qu'une névrose ; il faut bien convenir que sa curabilité sans
médicaments est en faveur de cette hypothèse.

\sk

Mon régime est-il toujours efficace ? Débarrassera-t-il dans tous les cas les
obèses gros mangeurs de leur obésité ?

Ici, je dois faire quelques réserves. S'il s'agit d’un sujet dont les fonctions
nutritives ne sont pas profondément altérées et chez lequel la seule cause
d'engraissement est la gourmandise, l'obésité disparaîtra fatalement. Mais s'il
s'agit d'une obésité organique chez un gros mangeur, association qui peut fort
bien se rencontrer car les vices de nutrition ne vaccinent pas contre la
gourmandise, seul l'effet de la suralimentation disparaîtra avec elle, et il
faudra, pour combattre l'élément organique, recourir aux conseils d'un médecin
expérimenté.

Ces réserves faites, je ne saurais trop engager tous les gros mangeurs,
goinfres et gloutons obèses, à suivre mon régime. Tous s'en trouveront bien,
tant au point de vue de leur obésité qu'à celui de leur santé générale et, ce
qui n'est pas à dédaigner, ils s'en trouveront également bien au point de vue
de l'accroissement des jouissances gastronomiques qu'ils éprouveront, car on
n'est guère en état d'apprécier convenablement la cuisine quand on dévore.
Or, s'il est indécent de vivre pour manger, il convient, tout en mangeant pour
vivre, de s'acquitter de cette tâche comme de toutes les autres, de son mieux,
avec plaisir.
