\section*{\centering Pâtes alimentaires italiennes.}
\phantomsection
\addcontentsline{toc}{section}{ Pâtes alimentaires italiennes.}
\index{Pâtes alimentaires italiennes}

On désigne sous le nom générique de pâtes italiennes des pâtes alimentaires
fabriquées sous des formes diverses : macaroni moulé en tubes ou en baguettes,
nouilles en rubans minces, lazagnes en rubans larges, coquilles, etc. Parmi les
macaroni en tubes, il faut citer, par ordre de grosseurs décroissantes : les
\textit{caneloni} destinés surtout à être farcis et qui atteignent un diamètre
de plusieurs centimètres, les \textit{maccheroni}, du diamètre moyen de
5 millimètres et dont les variétés. en allant des plus gros aux plus minces,
sont : les \textit{zita}, les \textit{mezzani} et les \textit{macheroncelli}.
Parmi les macaroni en baguettes, il convient de noter : les \textit{spaghetti}
de Naples, de deux millimètres et demi environ, les \textit{vermicelli} de
Sicile, d'un millimètre et demi, les vermicelles proprement dits, d'un diamètre
ne dépassant guère un millimètre et dont les plus fins sont les
\textit{capellini} de Sicile.

Toutes ces pâtes diffèrent de qualité suivant leur composition ; aussi, est-il
préférable de préparer soi-même au moins celles de forme simple, telles que les
nouilles et les lazagnes, plutôt que de les acheter toutes prêtes. Leur goût,
pour une composition donnée, diffère, non seulement avec le mode de
préparation, mais aussi avec la forme du produit.

\sk

Une bonne pâte à nouilles se compose essentiellement de farine de blé dur et
d'œufs ; on peut y ajouter accessoirement plus ou moins de beurre.

Le blanc de l'œuf a pour effet de donner à la pâte du liant et de la fermeté,
le jaune lui donne de la finesse, de la couleur et la rend plus nourrissante.
\label{pg0680} \hypertarget{p0680}{}

Une pâte à nouilles de qualité moyenne, suffisante pour les usages courants,
s'obtient par le mélange des éléments suivants dans les proportions de
{\ppp100\mmm} grammes de farine pour un œuf et 3 grammes de sel, mais on peut
avec avantage augmenter la quantité des jaunes d'œufs,

\footnotesize
\begin{longtable}{rrrp{16em}}
    500 & grammes & de & farine de blé dur,                                                               \\
     20 & grammes & de & beurre,                                                                          \\
     15 & grammes & de & sel,                                                                             \\
        &         & 10 & jaunes d'œufs,                                                                   \\
        &         &  2 & blancs d'œufs.                                                                   \\
\end{longtable}
\normalsize

donnent une composition considérée généralement comme très bonne.

\medskip

On pourra même supprimer complètement les blancs et mettre autant de
jaunes que la farine pourra en absorber.

\footnotesize
\begin{longtable}{rrrp{16em}}
    500 & grammes & de & farine de blé dur,                                                               \\
     20 & grammes & de & beurre,                                                                          \\
     15 & grammes & de & sel,                                                                             \\
        &         & 18 & jaunes d'œufs.                                                                   \\
\end{longtable}
\normalsize

donnent une pâte à nouilles absolument exquise et d'une finesse incomparable.

Ces quantités peuvent servir pour 8 à {\ppp10\mmm} personnes.

Quelle que soit la composition employée, la préparation de la pâte reste la
même, et voici comment on doit opérer.

Mélangez les éléments et travaillez de façon à avoir un ensemble homogène ;
laissez reposer la pâte pendant quelques heures ; divisez-la ensuite, pour la
commodité du travail, en quatre parties que vous étendrez successivement au
rouleau et dont vous ferez des abaisses minces, sans déchirures ; saupoudrez de
farine les abaisses ainsi obtenues, roulez-les et débitez-les en tranches de
3 millimètres pour les nouilles, de {\ppp15\mmm} millimètres pour les lazagnes.
Lorsque la pâte a été préparée un peu d'avance et que les abaisses ont été bien
farinées, les nouilles ou les lazagnes se déroulent facilement,

\sk

\index{Garnitures pour potages}
Toutes les pâtes italiennes peuvent être employées soit comme garniture de
potages ou de plats de viande, soit comme entremets de légumes.

\section*{\centering Nouilles sautées.}
\phantomsection
\addcontentsline{toc}{section}{ Nouilles sautées.}
\index{Nouilles sautées}

Pour quatre personnes prenez :

\footnotesize
\begin{longtable}{rrrp{16em}}
    250 &  grammes & de & nouilles fraîchement préparées,                                                 \\
     50 &  grammes & de & parmesan râpé,                                                                  \\
     30 &  grammes & de & beurre,                                                                         \\
     10 &  grammes & de & sel gris,                                                                       \\
      1 &  gramme  & de & poivre,                                                                         \\
        & 2 litres & d' & eau.                                                                            \\
\end{longtable}
\normalsize

Mettez dans une casserole l'eau, le sel et le poivre, faites bouillir, semez
dedans les nouilles et empêchez-les de s'agglutiner en les remuant légèrement.
Laissez-les cuire pendant cinq à six minutes\footnote{Avec des nouilles sèches,
il faudrait un peu plus de temps.}, égouttez-les, puis faites-les sauter dans
une sauteuse avec le beurre, saupoudrez de parmesan. Mélangez bien le tout sans
briser les nouilles et servez.

\section*{\centering Nouilles sautées, panées.}
\phantomsection
\addcontentsline{toc}{section}{ Nouilles sautées, panées.}
\index{Nouilles sautées, panées}
\label{pg0681} \hypertarget{p0681}{}

Pour quatre personnes prenez :

\footnotesize
\begin{longtable}{rrrp{16em}}
    250 & grammes  & de & nouilles,                                                                       \\
    125 & grammes  & de & beurre,                                                                         \\
    100 & grammes  & de & pain,                                                                           \\
     75 & grammes  & de & parmesan râpé,                                                                  \\
     75 & grammes  & de & gruyère râpé,                                                                   \\
     30 & grammes  & de & sel,                                                                            \\
      1 & gramme   & de & poivre,                                                                         \\
        & 2 litres & d' & eau.                                                                            \\
\end{longtable}
\normalsize

Faites cuire les nouilles dans l'eau salée et poivrée ; égouttez-les.

Séchez et faites griller le pain au four ; broyez-le et passez-le au tamis fin.

Mettez le beurre dans une sauteuse ; amenez-le à la couleur noisette ; ajoutez
d'abord la chapelure, puis les nouilles ; mélangez bien en faisant sauter le
tout ensemble.

Servez en envoyant en même temps, dans un ravier, les deux fromages mélangés.

Les nouilles préparées ainsi font partie de la cuisine slave. Ce sont les
nouilles sautées à la polonaise.

\section*{\centering Nouilles mi-sautées, mi-frites.}
\phantomsection
\addcontentsline{toc}{section}{ Nouilles mi-sautées, mi-frites.}
\index{Nouilles mi-sautées, mi-frites}

Faites sauter les trois quarts des nouilles comme il est dit à l'article
« Nouilles sautées ».

Faites frire le reste dans du beurre fondu ; au bout de quelques minutes, elles
seront croustillantes ; retirez-les,

Mettez dans un plat les nouilles sautées, disposez dessues les nouilles frites
et servez.

Le contraste des deux préparations donne une sensation nouvelle et agréable.

\section*{\centering Nouilles à la crème.}
\phantomsection
\addcontentsline{toc}{section}{ Nouilles à la crème.}
\index{Nouilles à la crème}

\enlargethispage{10mm}
Pour quatre personnes prenez :

\footnotesize
\begin{longtable}{rrrp{16em}}
    250 & grammes  & de & nouilles sèches\footnote{Si, au lieu de nouilles sèches, on emploie
                          des nouilles fraîches, la cuisson se fera comme d'ordinaire sans être
                          suivie de pochage. La quantité de beurre sera la même, mais on réduira
                          de moitié environ celle de la crème.
                          \protect\endgraf
                          Pour un même poids de nouilles, le plat sera moins copieux avec des
                          nouilles fraîches.},                                                            \\
    200 & grammes  & de & bonne crème épaisse,                                                            \\
     50 & grammes  & de & beurre,                                                                         \\
     10 & grammes  & de & sel gris,                                                                       \\
      1 & gramme   & de & poivre fraîchement moulu,                                                       \\
        & 2 litres & d' & eau.                                                                            \\
\end{longtable}
\normalsize

Mettez dans l'eau le sel et le poivre ; amenez à ébullition ; plongez les
nouilles dans le liquide. Faites-les cuire pendant deux minutes ; éloignez la
casserole du feu et laissez-les pocher pendant trois quarts d'heure environ,
à liquide simplement frissonnant.

Égouttez les nouilles ; disposez-les ensuite dans un plat de service allant au
feu, ajoutez le beurre, coupé en petits morceaux, et la crème ; chauffez sans
faire bouillir.

Servez dans le plat.

Les nouilles à la crème, excellentes comme entremets de légumes, accompagnent
parfaitement les viandes blanches.

\section*{\centering Nouilles aux cèpes.}
\phantomsection
\addcontentsline{toc}{section}{ Nouilles aux cèpes.}
\index{Nouilles aux cèpes}

Pour six personnes prenez :

\footnotesize
\begin{longtable}{rrrp{16em}}
    250 & grammes  & de & nouilles,                                                                       \\
    125 & grammes  & de & beurre,                                                                         \\
    125 & grammes  & de & cèpes secs,                                                                     \\
    125 & grammes  & de & purée de tomates,                                                               \\
     60 & grammes  & de & gruyère, en lames minces,                                                       \\
     60 & grammes  & de & parmesan, en lames minces,                                                      \\
     50 & grammes  & de & glace de viande,                                                                \\
     30 & grammes  & d' & oignon haché,                                                                   \\
     10 & grammes  & de & sel gris,                                                                       \\
      1 & gramme   & de & poivre,                                                                         \\
        & 2 litres & d' & d'eau.                                                                          \\
\end{longtable}
\normalsize

Plongez les cèpes dans de l'eau pendant une heure pour les ramollir.

Faites cuire les nouilles avec le sel gris et le poivre dans l'eau bouillante ;
égouttez-les.

Mettez dans une casserole {\ppp50\mmm} grammes de beurre ; lorsqu'il sera
chaud, saisissez dedans l'oignon haché, sans le laisser roussir, puis ajoutez
les cèpes, la purée de tomates, la glace de viande et laissez cuire ensemble
pendant dix minutes.

Faites sauter les nouilles dans {\ppp50\mmm} grammes de beurre ; mélangez-v les
trois quarts des deux fromages.

Disposez dans un plat une couche de nouilles, au-dessus le ragoût de cèpes, une
autre couche de nouilles, le reste des deux fromages, arrosez avec le reste du
beurre que vous aurez fait fondre et servez.

C'est exquis.

\sk

Ou peut préparer de même des nouilles aux morilles.

\section*{\centering Nouilles au jambon.}
\phantomsection
\addcontentsline{toc}{section}{ Nouilles au jambon.}
\index{Nouilles au jambon}

Pour six personnes prenez :

\footnotesize
\begin{longtable}{rrrp{16em}}
    250 & grammes  & de & nouilles,                                                                       \\
    200 & grammes  & de & jambon maigre, coupé en languettes,                                             \\
    100 & grammes  & de & fond de veau,                                                                   \\
     40 & grammes  & de & parmesan râpé,                                                                  \\
     25 & grammes  & de & beurre,                                                                         \\
      5 & grammes  & de & sel gris,                                                                       \\
      1 & gramme   & de & poivre.                                                                         \\
        & 2 litres & d' & eau.                                                                            \\
\end{longtable}
\normalsize

Faites bouillir l'eau, salée et poivrée, jetez dedans les nouilles ;
laissez-les cuire ; égouttez-les.

Mettez dans une sauteuse le fond de veau, chauffez dedans le jambon, puis
retirez-le.

Disposez ensuite dans la sauteuse des couches alternées de nouilles, de jambon
et de fromage jusqu'à leur épuisement ; faites sauter, ajoutez le beurre coupé en
petits morceaux, faites sauter encore juste assez pour fondre le beurre et servez.

\sk

On préparera dans le même esprit des nouilles truffées, en remplaçant dans la
formule précédente le jambon par des émincés de truffes cuites dans du madère ;
{\ppp75\mmm} grammes de truffes pour {\ppp250\mmm} grammes de nouilles peuvent
suffire, mais la proportion des truffes variera naturellement avec le goût.

\section*{\centering Lazagnes gratinées, aux épinards.}
\phantomsection
\addcontentsline{toc}{section}{ Lazagnes gratinées, aux épinards.}
\index{Lazagnes gratinées, aux épinards}

Pour six personnes prenez :

\footnotesize
\begin{longtable}{rrrp{16em}}
    500 & grammes  & d' & épinards,                                                                       \\
    400 & grammes  & de & bon jus,                                                                        \\
    250 & grammes  & de & lazagnes,                                                                       \\
    100 & grammes  & de & parmesan râpé,                                                                  \\
     75 & grammes  & de & sel gris,                                                                       \\
     25 & grammes  & de & beurre,                                                                         \\
      1 & gramme   & de & poivre,                                                                         \\
        & 5 litres & d' & eau.                                                                            \\
\end{longtable}
\normalsize

Blanchissez les épinards dans 3 litres d'eau additionnée de {\ppp45\mmm}
grammes de sel gris ; rafraîchissez-les ensuite.

Faites bouillir 2 litres d'eau dans laquelle vous mettrez le poivre et le reste
du sel gris, jetez dedans les lazagnes, laissez-les cuire pendant dix
minutes\footnote{La durée de cuisson indiquée s'applique à des lazagnes qui ne
sont pas fraîchement préparées ; autrement, {\ppp5\mmm} à {\ppp6\mmm} minutes
suffiront}.

Mettez lazagnes et épinards dans une casserole, mouillez avec le jus, mélangez,
laissez cuire à petit feu pendant un quart d'heure ; puis ajoutez le beurre, le
fromage et faites gratiner au four pendant une dizaine de minutes.

Les lazagnes ainsi préparées constituent un entremets de légumes qui sort de la
banalité.

\section*{\centering Spaghetti à la napolitaine\footnote{Les spaghetti
constituent le plat national napolitain.
\protect\endgraf
C'est un spectacle vraiment pittoresque de voir, dans les restaurants de
Naples, les indigènes enrouler avec dextérité sur leurs fourchettes ces
ficelles de pâte de près de {\ppp50\mmm} centimètres de longueur, pour en faire
des boules qu'ils portent à leur bouche où elles disparaissent comme par
enchantement.}}
\phantomsection
\addcontentsline{toc}{section}{ Spaghetti à la napolitaine.}
\index{Spaghetti à la napolitaine}

Pour quatre personnes prenez :

\footnotesize
\begin{longtable}{rrrp{16em}}
    250 & grammes  & de & spaghetti de Naples,                                                            \\
    150 & grammes  & de & purée de tomates de Naples,                                                     \\
    125 & grammes  & de & parmesan râpé,                                                                  \\
     60 & grammes  & de & beurre,                                                                         \\
     30 & grammes  & de & sel gris,                                                                       \\
      1 & gramme   & de & poivre,                                                                         \\
        & 2 litres & d' & eau.                                                                            \\
\end{longtable}
\normalsize

Mettez le sel et le poivre dans l'eau, faites bouillir, plongez dedans les
spaghetti sans les casser, laissez-les cuire pendant douze minutes, puis
retirez-les et égouttez-les.

Faites fondre le beurre dans une casserole, ajoutez la purée de tomates,
chauffez, mettez les spaghetti, le fromage ; mélangez bien.

Servez en envoyant en même temps un ravier de parmesan râpé.

\section*{\centering Spaghetti garnis.}
\phantomsection
\addcontentsline{toc}{section}{ Spaghetti garnis.}
\index{Spaghetti garnis}

Pour quatre personnes prenez :

\footnotesize
\begin{longtable}{rrrp{16em}}
    500 & grammes  & de & fond de veau,                                                                   \\
    250 & grammes  & de & spaghetti,                                                                      \\
    125 & grammes  & de & jambon de Parme,                                                                \\
    100 & grammes  & de & parmesan râpé fin,                                                              \\
     60 & grammes  & de & cèpes secs,                                                                     \\
     40 & grammes  & de & beurre,                                                                         \\
     30 & grammes  & de & glace de viande,                                                                \\
     30 & grammes  & de & sel gris,                                                                       \\
     10 & grammes  & de & farine,                                                                         \\
      1 & gramme   & de & poivre,                                                                         \\
        & 2 litres & d' & eau,                                                                            \\
        &          &  6 & foies de poulets.                                                               \\
\end{longtable}
\normalsize

Ramollissez les cèpes dans de l'eau et coupez-les en morceaux.

Faites un roux avec {\ppp20\mmm} grammes de beurre et la farine ; mouillez avec
le fond de veau ; mettez les cèpes et la glace de viande ; laissez cuire.

Faites cuire les spaghetti à l'eau salée et poivrée, comme précédemment ;
égouttez-les ; tenez-les au chaud.

Émincez le jambon.

Coupez les foies de poulets en morceaux, faites-les revenir dans du beurre ;
ajoutez-les ainsi que le jambon au ragoût de cèpes ; chauffez pendant un
instant.

Dressez les spaghetti sur un plat chaud, disposez dessus le salpicon de cèpes,
foies et jambon ; servez en envoyant en même temps le parmesan râpé dans un
ravier.

\section*{\centering Timbale de spaghetti au ris de veau et au jambon.}
\phantomsection
\addcontentsline{toc}{section}{ Timbale de spaghetti au ris de veau et au jambon.}
\index{Timbale de spaghetti au ris de veau et au jambon}

Pour six personnes prenez :

\footnotesize
\begin{longtable}{rrrp{16em}}
    250 & grammes & de  & spaghetti,                                                                      \\
    250 & grammes & de  & champignons,                                                                    \\
    150 & grammes & de  & jambon de Bayonne,                                                              \\
    100 & grammes & de  & beurre,                                                                         \\
     60 & grammes & de  & parmesan,                                                                       \\
      5 & grammes & de  & farine,                                                                         \\
        & 1 litre & de  & fond de veau,                                                                   \\
        &         & 1/2 & ris de veau,                                                                    \\
        &         &     & sel et poivre.                                                                  \\
\end{longtable}
\normalsize

Faites pocher les spaghetti pendant une vingtaine de minutes dans de l'eau
salée bouillante ; égouttez-les, puis mettez-les dans une casserole avec la
moitié du fond de veau et continuez doucement la cuisson à tout petit feu.

Nettoyez et blanchissez le ris ; coupez-le en dés, ainsi que le jambon.

Pelez les champignons, émincez-les, dorez-les légèrement dans {\ppp50\mmm}
grammes de beurre, saupoudrez ensuite avec la farine, mouillez avec le reste du
fond de veau. ajoutez le ris, le jambon et achevez la cuisson de l'ensemble de
manière à réduire de moitié la sauce du salpicon.

Incorporez aux spaghetti, sans les briser, le reste du beurre et le parmesan,
mélangez-y doucement le salpicon, goûtez et complétez l'assaisonnement, s'il
est nécessaire, avec sel et poivre.

Servez dans une croûte de timbale ou de vol-au-vent.

\section*{\centering Vermicelli à la sicilienne.}
\phantomsection
\addcontentsline{toc}{section}{ Vermicelli à la sicilienne.}
\index{Vermicelli à la sicilienne}

Pour six personnes prenez :

\footnotesize
\begin{longtable}{rrrp{16em}}
    250 & grammes  & de & vermicelli de Palerme,                                                          \\
    150 & grammes  & de & purée de tomates de Palerme,                                                    \\
    150 & grammes  & de & hachis de viandes, telles que poulet, veau, jambon,                             \\
    150 & grammes  & de & parmesan râpé,                                                                  \\
     60 & grammes  & de & beurre,                                                                         \\
     30 & grammes  & de & sel gris,                                                                       \\
      1 & gramme   & de & poivre,                                                                         \\
        & 2 litres & d' & eau,                                                                            \\
        &          &  1 & aubergine,                                                                      \\
        &          &    & huile d'olive.                                                                  \\
\end{longtable}
\normalsize

Faites cuire les vermicelli comme les spaghetti, dans l'eau bouillante salée et
poivrée ; égouttez-les.

Mettez le beurre et la purée de tomates dans une casserole, chauffez, ajoutez
les vermicelli, {\ppp125\mmm} grammes de parmesan ; mélangez bien.

Coupez l'aubergine en tranches, faites-les frire dans de l'huile d'olive et
garnissez-les avec le hachis chaud.

Dressez les vermicelli sur un plat, disposez dessus les tranches d'aubergine
garnies, saupoudrez avec le reste du parmesan, faites gratiner pendant un
moment au four et servez.

\section*{\centering Timbale de macaroni au jus.}
\phantomsection
\addcontentsline{toc}{section}{ Timbale de macaroni au jus.}
\index{Timbale de macaroni au jus}

Pour six personnes prenez :

\footnotesize
\begin{longtable}{rrrrrp{18em}}
  & \hspace{2em} & 250 & grammes  & de & macaroni,                                                        \\
  & \hspace{2em} & 100 & grammes  & de & gruyére râpé,                                                    \\
  & \hspace{2em} & 100 & grammes  & de & parmesan râpé,                                                   \\
  & \hspace{2em} &  70 & grammes  & de & beurre,                                                          \\
  & \hspace{2em} &  15 & grammes  & de & farine,                                                          \\
  & \multicolumn{3}{r}{5 décigrammes}  & de & paprika,                                                    \\
  & \multicolumn{3}{r}{1 litre 1/2}    & de & jus de viande,                                              \\
  & \hspace{2em} &     &          &  1 & croûte de timbale,                                               \\
  & \hspace{2em} &     &          &    & truffes à volonté.                                               \\
\end{longtable}
\normalsize

Si le marconi est frais, souple, on peut l'employer tel que ; s'il est sec,
plongez-le d'abord dans de l'eau bouillante salée pendant le temps strictement
nécessaire pour le ramollir.

Faites bouillir le jus de viande, mettez dedans le macaroni, éloignez la
casserole du feu, mais tenez-la au chaud sans laisser bouillir, pendant une
heure environ, de façon que le macaroni absorbe tout le jus.

Faites cuire la farine dans le beurre sans qu'elle prenne couleur, ajoutez
ensuite macaroni, fromages râpés, paprika, truffes cuites au naturel et
hachées, mélangez bien, puis mettez le tout dans la croûte de timbale ou dans
une croûte de pâté de foie gras, saupoudrez le dessus de truffes hachées et
servez.

\section*{\centering Turban de macaroni aux champignons farcis, sauce tomate.}
\phantomsection
\addcontentsline{toc}{section}{ Turban de macaroni aux champignons farcis, sauce tomate.}
\index{Turban de macaroni aux champignons farcis, sauce tomate}

Pour six personnes prenez :

\footnotesize
\begin{longtable}{rrrrrp{18em}}
  & \hspace{2em}  &  250 & grammes & de & macaroni,                                                       \\
  & \hspace{2em}  &  250 & grammes & de & purée de tomates,                                               \\
  & \hspace{2em}  &  250 & grammes & de & fromage de Gruyère en lamelles,                                 \\
  & \hspace{2em}  &  150 & grammes & de & beurre,                                                         \\
  & \hspace{2em}  &  125 & grammes & de & fromage de Parme en lamelles,                                   \\
  & \hspace{2em}  &  125 & grammes & de & champignons,                                                    \\
  & \hspace{2em}  &   15 & grammes & de & sel gris,                                                       \\
  & \hspace{2em}  &   10 & grammes & de & farine,                                                         \\
  & \multicolumn{3}{r}{3 décigrammes} & de & muscade en poudre,                                           \\
  & \hspace{2em}  &      & 1 litre & de & lait,                                                           \\
  & \hspace{2em}  &      &         & 12 & petits champignons de couche semblables et de belle forme,      \\
  & \hspace{2em}  &      &         &  6 & gros champignons de couche semblables et bien ronds,            \\
  & \hspace{2em}  &      &         &  1 & jaune d'œuf,                                                    \\
  & \hspace{2em}  &      &         &    & mie de pain rassis tamisée,                                     \\
  & \hspace{2em}  &      &         &    & jus de viande,                                                  \\
  & \hspace{2em}  &      &         &    & persil haché,                                                   \\
  & \hspace{2em}  &      &         &    & jus de citron,                                                  \\
  & \hspace{2em}  &      &         &    & curry,                                                          \\
  & \hspace{2em}  &      &         &    & sel et poivre.                                                  \\
\end{longtable}
\normalsize

Cassez le macaroni en morceaux de {\ppp10\mmm} centimètres de longueur.

Faites bouillir le lait, mettez dedans le sel gris, la muscade et le macaroni ;
donnez quelques bouillons, éloignez un peu la casserole du feu, mais
maintenez-la au chaud pendant le temps nécessaire à la cuisson du macaroni qui
doit avoir absorbé tout le lait et être gonflé à point. Une heure suffit
généralement.

Égouttez le macaroni sans le laisser refroidir, incorporez-y aussitôt le
gruyère et le parmesan râpés, moins {\ppp15\mmm} grammes de chaque que vous
réserverez. Lorsqu'il sera suffisamment filant, mettez-le dans un moule
annulaire beurré légèrement ({\ppp15\mmm} grammes de beurre) ; tenez le moule
au chaud au bain-marie.

En même temps, préparez une sauce tomate épaisse : maniez la farine avec
{\ppp40\mmm} grammes de beurre, laisssz-la cuire pendant cinq minutes sans
qu'elle prenne couleur, ajoutez la purée de tomates, un peu de jus de viande,
salez, poivrez ; mélangez bien et continuez la cuisson pendant un quart
d'heure. Finissez la sauce en y incorporant {\ppp20\mmm} grammes de beurre
frais coupé en morceaux.

Pelez les champignons, passez-les dans du jus de citron ; enlevez les pieds des
gros champignons, réservez les chapeaux ; réservez les {\ppp12\mmm} petits
champignons.

Hachez ensemble le reste des champignons avec les pieds des gros, ajoutez-y de
la mie de pain rassis tamisée imbibée de jus de viande, du persil haché,
{\ppp30\mmm} grammes de beurre, le jaune d'œuf, du sel, du poivre et du curry
au goût ; mélangez bien.

Farcissez avec ce mélange les chapeaux des gros champignons, disposez-les, le
creux en dessus, dans un plat foncé de {\ppp15\mmm} grammes de beurre,
aspergez-les d'un peu de jus de citron, saupoudrez avec le reste des fromages
mélangés, mettez par-dessus {\ppp15\mmm} grammes de beurre coupé en petits
morceaux et faites cuire au four doux pendant une vingtaine de minutes.

Faites cuire les {\ppp12\mmm} petits champignons dans le reste du beurre
additionné d'un peu de jus de citron, ajoutez-les à la sauce tomate.

Démoulez le turban sur un plat, garnissez-en l'intérieur avec la sauce tomate,
dressez autour les gros champignons farcis et servez.

C'est confortable et joli.

\section*{\centering Macaroni au gratin.}
\phantomsection
\addcontentsline{toc}{section}{ Macaroni au gratin.}
\index{Macaroni au gratin}

A. — \textit{Macaroni au gratin, sans crème}.

\medskip

Pour cinq personnes prenez :

\footnotesize
\begin{longtable}{rrrp{16em}}
    250 & grammes  & de & macaroni,                                                                       \\
     80 & grammes  & de & beurre,                                                                         \\
     60 & grammes  & de & gruyère râpé,                                                                   \\
     60 & grammes  & de & parmesan râpé,                                                                  \\
     45 & grammes  & de & sel gris,                                                                       \\
        & 3 litres & d' & eau.                                                                            \\
\end{longtable}
\normalsize

Faites bouillir l'eau, plongez dedans le macaroni, l'ébullition s'arrêtera,
ajoutez le sel, donnez quelques bouillons, puis éloignez la casserole sur le
coin du fourneau, de façon que l'eau reste simplement très chaude. Laissez
pocher le maçaroni pendant le temps nécessaire pour qu'il soit cuit
convenablement, ce qui demande {\ppp40\mmm} minutes à une heure, suivant la
fraîcheur du produit et l'épaisseur de la pâte. Égouttez-le, puis disposez dans
un plat, foncé d'un peu de beurre et allant au feu, des couches alternées de
macaroni, de fromages râpés et de beurre, terminez par du beurre.

Faites gratiner au four.

Le macaroni ainsi préparé est très croustillant.

\medskip

B. — \textit{Macaroni au gratin, à la crème}.

\medskip

Pour six personnes prenez :

\footnotesize
\begin{longtable}{rrrp{16em}}
    250 & grammes  & de & macaroni,                                                                       \\
    250 & grammes  & de & crème,                                                                          \\
     60 & grammes  & de & beurre,                                                                         \\
     60 & grammes  & de & gruyère râpé,                                                                   \\
     60 & grammes  & de & parmesan râpé,                                                                  \\
     45 & grammes  & de & sel gris,                                                                       \\
     20 & grammes  & de & farine,                                                                         \\
        & 3 litres & d' & eau,                                                                            \\
        &          &  2 & clous  de girofle,                                                              \\
        &          &  1 & oignon,                                                                         \\
        &          &    & un peu de muscade.                                                              \\
\end{longtable}
\normalsize

Mettez dans l'eau l'oignon et les clous de girofle enveloppés dans une
mousseline, le sel, amenez à ébullition, plongez dedans le macaroni et
laissez-le cuire comme dans la formule précédente.

Faites blondir la farine avec la moitié du beurre, aromatisez avec de la muscade
au goût, ajoutez la crème, tout le gruyère et la moitié du parmesan ; mélangez ;
chauffez sans laisser bouillir.

Foncez un plat avec {\ppp15\mmm} grammes de beurre, mettez dedans des couches
alternées de macaroni et de sauce, terminez par le reste du parmesan et le
reste du beurre ; faites gratiner.

Ce macaroni, très moelleux sous sa couche gratinée, diffère absolument du
précédent.

\section*{\centering Pâtes à la poche.}
\phantomsection
\addcontentsline{toc}{section}{ Pâtes à la poche.}
\index{Pâtes à la poche}

Pour six personnes prenez :

\footnotesize
\begin{longtable}{rrrp{16em}}
    200 & grammes & de & gruyère ou de parmesan râpé,                                                     \\
    200 & grammes & de & fond de veau et volaille très concentré,                                         \\
        &         &  6 & œufs frais,                                                                      \\
        &         &    & farine de blé dur,                                                               \\
        &         &    & lait non écrémé,                                                                 \\
        &         &    & sel.                                                                             \\
\end{longtable}
\normalsize

Cassez les œufs dans une terrine et incorporez-y autant de farine qu'ils
pourront en absorber ; salez légèrement, mouillez avec du lait et travaillez de
façon à obtenir une pâte coulante.

Mettez-la dans une poche percée de trous de {\ppp7\mmm} à {\ppp8\mmm}
millimètres de diamètre environ au travers desquels vous passerez des baguettes
de pâte d'une dizaine de centimètres de longueur, que vous ferez tomber dans
une marmite placée sur le feu et contenant soit du bouillon bouillant, soit,
à défaut, de l'eau bouillante un peu salée\footnote{Quand ces pâtes doivent
servir de garniture pour potages, on les fait cuire directement dedans.}.

Aussitôt cuites, les pâtes monteront à la surface du liquide.

Enlevez-les avec une écumoire ; égouttez-les bien ; disposez-les ensuite dans
un plat allant au feu, par couches alternées avec du fromage râpé, en terminant
par ce dernier ; versez dessus le fond de veau et volaille ; faites gratiner au
four et servez dans le plat.

Ces pâtes sont délicieuses.

\sk

Comme variantes, on pourra servir les pâtes à la poche simplement arrosées de
beurre fondu, ou encore sautées dans du beurre avec de la mie de pain rassis
tamisée, dorée au préalable.

\section*{\centering Ravioli.}
\phantomsection
\addcontentsline{toc}{section}{ Ravioli.}
\index{Ravioli}

Les ravioli sont des cachets de pâte renfermant une farce.

Voici une formule de ravioli soignés.

\medskip

Pour six personnes prenez :

\medskip

1° pour la pâte :

\footnotesize
\begin{longtable}{rrrrrp{18em}}
  & \hspace{2em} &  250 & grammes & de & farine de blé dur de Hongrie,                                    \\
  & \hspace{2em} &  100 & grammes & de & lait,                                                            \\
  & \hspace{2em} &    5 & grammes & de & sel,                                                             \\
  & \hspace{2em} &      &         &  1 & œuf frais entier,                                                \\
  & \hspace{2em} &      &         &  1 & jaune d'œuf frais ;                                              \\
\end{longtable}
\normalsize

\index{Farce pour raviolis}
2° pour la farce :

\footnotesize
\begin{longtable}{rrrrrp{18em}}
  & \hspace{2em} &  125 & grammes & de & blanc de poulet rôti et haché,                                   \\
  & \hspace{2em} &  125 & grammes & de & jambon d'York haché,                                             \\
  & \hspace{2em} &  125 & grammes & de & ris de veau cuit au beurre et haché,                             \\
  & \hspace{2em} &   50 & grammes & de & crème épaisse,                                                   \\
  & \hspace{2em} &   50 & grammes & de & truffes cuites au madère et hachées,                             \\
  & \hspace{2em} &      &         &  2 & jaunes d'œufs,                                                   \\
  & \hspace{2em} &      &         &    &  sel et poivre ;                                                 \\
\end{longtable}
\normalsize

3° pour la cuisson et pour l'apprêt :

\footnotesize
\begin{longtable}{rrrrrp{18em}}
  & \hspace{2em} &  125 & grammes & de & blanc de poulet rôti et haché,                                   \kill
  & \hspace{2em} &      &         &    & bouillon,                                                        \\
  & \hspace{2em} &      &         &    & beurre,                                                          \\
  & \hspace{2em} &      &         &    & parmesan râpé,                                                   \\
  & \hspace{2em} &      &         &    & sauce tomate,                                                    \\
  & \hspace{2em} &      &         &    & jus de viande.                                                   \\
\end{longtable}
\normalsize

Disposez sur une planche la farine en cratère de volcan, cassez l'œuf dans le
cratère, ajoutez le jaune, le lait, le sel et faites du tout une pâte
parfaitement liée. Laissez-la reposer pendant une heure au moins, abaissez-la
ensuite à un millimètre et demi d'épaisseur, puis découpez dedans, avec un
coupe-pâte, {\ppp48\mmm} ronds de {\ppp6\mmm} centimètres de diamètre.

Préparez une farce avec le blanc de poulet, le jambon, le ris de veau, la
crème, les truffes, assaisonnez avec du sel et du poivre et liez avec les
jaunes d'œufs.

Partagez cette farce en {\ppp48\mmm} petites noisettes.

Mettez une noisette de farce au milieu de chacun des ronds de pâte, mouillez-en
les bords, fermez-les en croissants, soudez les joints à la pince.

Faites bouillir du bouillon, jetez dedans les ravioli, laissez-les cuire
pendant une dizaine de minutes, éloignez un peu la casserole du feu et
laissez-les pocher pendant à peu près autant de temps dans le bouillon presque
bouillant. Égouttez-les.

Dressez les ravioli sur un plat, versez dessus du beurre fondu couleur
noisette, saupoudrez de parmesan râpé et servez.

Envoyez en même temps, dans une saucière, de la sauce tomate corsée avec du jus
de viande.

\sk

\index{Garnitures pour potages}
Les ravioli destinés à garnir des potages sont préparés moitié moins gros.

\sk

Les \textit{tortellini}, spécialité de Bologne, sont des ravioli à farce moins
fine que la précédente, souvent même simple farce de légumes ; leur
caractéristique morphologique est d'être tordus sur eux-mêmes.

\section*{\centering Rissoles.}
\phantomsection
\addcontentsline{toc}{section}{ Rissoles.}
\index{Rissoles}

Les rissoles peuvent être considérées comme une variété de ravioli frits.

La pâte employée pour les rissoles est généralement la pâte feuilletée, mais on
se sert aussi de pâte brisée\footnote{Pour préparer la pâte brisée, prenez :
        \protect\endgraf
        \begin{tabular}{rrrl}
        \hspace{4em} 150 & grammes & de & farine,                                                         \\
        \hspace{4em} 100 & grammes & de & beurre,                                                         \\
        \hspace{4em}  70 & grammes & d' & eau,                                                            \\
        \hspace{4em}   3 & grammes & de & sel.                                                            \\
        \end{tabular}
        \protect\endgraf

Mélangez et pétrissez bien les éléments, ne fraisez pas ; laissez reposer la
pâte, puis donnez-lui deux tours de feuilletage.}, de pâte à foncer fine et de
pâte à brioche ordinaire.

\index{Farce pour rissoles}
Quelle que soit la pâte choisie, abaissez-la à une épaisseur de {\ppp3\mmm}
à {\ppp4\mmm} millimètres ; découpez dans cette abaisse des ronds d'un diamètre
plus ou moins grand ; mettez sur ces ronds de pâte des petites quenelles de
farce ou des noix de salpicon à sauce très liée, ou encore des garnitures
fines, mouillez les bords des ronds et fermez soit en pliant les ronds sur
eux-mêmes en demi-cercles, soit avec des ronds de pâte non garnis.

Faites frire les rissoles telles quelles ou après les avoir passées
successivement dans de l'œuf battu et dans de la mie de pain rassis tamisée.
Dès qu'elles auront pris une belle couleur, égouttez-les et servez-les, comme
hors-d'œuvre chaud ou comme entrée\footnote{ On peut également faire des
rissoles pour entremets sucrés en employant comme garniture des marmelades de
fruits ou des confitures.}, sur un plat garni d'une serviette et décoré avec du
persil frit.

\medskip

\index{Garnitures pour rissoles}
\index{Garnitures pour potages}
Voici quelques exemples de garnitures :

\medskip

\textit{a)} huîtres masquées par de la purée de homard liée avec une sauce allemande
au poisson, relevée par du curry ;

\textit{b)} queues de crevettes liées avec une sauce homard très consistante ;

\textit{c)} salpicon de homard et de truffe, à sauce homard très consistante ;

\textit{d)} quenelles de poisson préparées avec chair de brochet, de merlan ou
de saumon, par exemple, du beurre, de la panade faite avec de la mie de pain et
du bouillon de poisson, et de la sauce allemande au poisson très réduite ;

\textit{e)} petites noisettes de moelle de bœuf pochées dans du bouillon, puis
enrobées dans de la purée épaisse de gibier ;

\textit{f)} salpicon de ris d'agneau braisé, de morilles et de pointes
d'asperges, lié avec de la sauce Béchamel ou de la sauce Soubise, ou encore un
mélange des deux corsés avec de la glace de viande ;

\textit{g)} salpicon de riz de veau braisé et de champignons lié avec une sauce
serrée à la crème ;

\textit{h)} quenelles de volaille préparées avec volaille, tétine de veau,
panade et sauce allemande très réduite ;

\textit{i)} rognons de coq cuits au champagne, enrobés dans un coulis de
perdreau lié avec de la glace de perdreau, et roulés dans un hachis de truffes
cuites au madère ;

\textit{j)} salpicon de foies de volaille, de crêtes de coq, de rognons de coq
et de truffes lié avec de la glace de volaille ;

\textit{k)} quenelles de purée de faisan, de bécasse, de foie gras d'oie et de
truffes liée avec de la glace de gibier ;

\section*{\centering Rissoles de ris de veau et de champignons.}
\phantomsection
\addcontentsline{toc}{section}{ Rissoles de ris de veau et de champignons.}
\index{Rissoles de ris de veau et de champignons}

Prenez :

\medskip

1° pour la pâte :

\footnotesize
\begin{longtable}{rrrrrp{18em}}
  & \hspace{2em} &  300 & grammes & de & farine,                                                          \\
  & \hspace{2em} &  300 & grammes & de & beurre fin,                                                      \\
  & \hspace{2em} &   50 & grammes & d' & eau,                                                             \\
  & \hspace{2em} &    6 & grammes & de & sel ;
\end{longtable}
\normalsize

2° pour le remplissage :
\footnotesize
\begin{longtable}{rrrrrp{18em}}
  & \hspace{2em} &  250 & grammes & de & champignons,                                                     \\
  & \hspace{2em} &  200 & grammes & de & fond de veau aromatisé, corsé avec de la glace de viande,        \\
  & \hspace{2em} &  150 & grammes & de & crème épaisse,                                                   \\
  & \hspace{2em} &   50 & grammes & de & beurre fin,                                                      \\
  & \hspace{2em} &   20 & grammes & de & farine,                                                          \\
  & \hspace{2em} &      &         &  1 & beau ris de veau,                                                \\
  & \hspace{2em} &      &         &    & fines herbes,                                                    \\
  & \hspace{2em} &      &         &    & persil,                                                          \\
  & \hspace{2em} &      &         &    & jus de citron,                                                   \\
  & \hspace{2em} &      &         &    & sel et poivre.                                                   \\
\end{longtable}
\normalsize

Préparez une pâte à foncer fine ; laissez-la reposer pendant quelques heures,
puis faites-en une abaisse aussi mince que possible.

Nettoyez le ris de veau ; pochez-le pendant quelques minutes dans de l'eau
salée bouillante, puis faites-le braiser pendant une demi-heure dans le fond de
veau.

Retirez-le, coupez-le en petits cubes ; enlevez les parties nerveuses,
Concentrez la cuisson à 40 grammes.

Épluchez les champignons, coupez-les en lamelles, passez-les dans du jus de
citron ; faites-les cuire dans le beurre.

Réunissez champignons et ris de veau, salez, poivrez, saupoudrez avec la
farine, mélangez sans laisser roussir, ajoutez le fond concentré, la crème, des
fines herbes hachées, au goût ; faites cuire un instant ; laissez refroidir.

Suivant que vous voudrez servir les rissoles comme hors-d'œuvre chaud ou comme
entrée, divisez la masse en vingt ou en dix parties égales que vous enroberez
séparément dans de la pâte en lui donnant la forme qui vous plaira, et
faites-les frire dans un bain de friture très chaude.

Servez sur un plat garni d'une serviette et décorez avec du persil frit.

Les proportions indiquées suffiront pour dix personnes comme hors-d'œuvre,
pour quatre à cinq personnes comme entrée.

\section*{\centering Gnocchi.}
\phantomsection
\addcontentsline{toc}{section}{ Gnocchi.}
\index{Gnocchi}

Les gnocchi sont des pâtes au parmesan.

En voici une formule qui donne un plat faisant bonne figure comme entremets de
légumes dans un repas de famille.

\medskip

Pour six personnes prenez :

\medskip

1° pour la pâte :

\footnotesize
\begin{longtable}{rrrrrp{18em}}
  & \hspace{2em} & 150 & grammes   & de & farine de blé dur,                                              \\
  & \hspace{2em} & 150 & grammes   & de & parmesan râpé,                                                  \\
  & \hspace{2em} &  20 & grammes   & de & beurre,                                                         \\
  & \hspace{2em} &   5 & grammes   & de & sel gris,                                                       \\
 & \multicolumn{3}{r}{5 décigrammes} & de & poivre,                                                       \\
  & \hspace{2em} &     & 1 litre   & de & lait,                                                           \\
  & \hspace{2em} &     & 1/2 litre &  d'& eau,                                                            \\
  & \hspace{2em} &     &           &  3 & œufs ;                                                          \\
\end{longtable}
\normalsize

2° pour la sauce :

\footnotesize
\begin{longtable}{rrrrrp{18em}}
  & \hspace{2em} & 100 & grammes   & de & crème,                                                          \\
  & \hspace{2em} &  50 & grammes   & de & parmesan râpé,                                                  \\
  & \hspace{2em} &  40 & grammes   & de & beurre,                                                         \\
  & \hspace{2em} &  25 & grammes   & de & farine,                                                         \\
  & \hspace{2em} &   5 & grammes   & de & sel blanc.                                                      \\
\end{longtable}
\normalsize

\textit{Préparation de la pâte}. — Mettez dans une casserole l'eau, le beurre,
le sel et le poivre, faites bouillir, jetez la farine en pluie dans le liquide,
tournez, ajoutez le parmesan, mélangez bien pendant une minute environ ;
éloignez la casserole du feu et délayez dedans les œufs, l'un après l'autre, en
mélangeant continuellement.

Divisez la pâte ainsi obtenue en petites billes.

\medskip

\textit{Cuisson des billes}. — Faites bouillir le lait, plongez dedans les
billes de pâte ; l'ébullition s'arrêtera aussitôt ; enlevez la casserole du feu
et tenez-la sur le coin du fourneau de façon que le lait reste très chaud sans
bouillir. Laissez cuire les billes pendant cinq minutes, sortez-les du liquide,
égouttez-les.

Passez le lait, dont le volume sera réduit à peu près de moitié ; réservez-le.

\medskip

\textit{Préparation de la sauce}. — Mettez dans une casserole {\ppp30\mmm}
grammes de beurre avec la farine, faites cuire sans laisser roussir, mouillez
avec le lait réservé, ajoutez la crème, le sel et continuez la cuisson
à liquide non bouillant pendant une dizaine de minutes.

\medskip

\textit{Finissage}. — Graissez avec le reste du beurre un plat allant au feu, disposez
dedans des couches alternées de billes, de sauce et de parmesan râpé, en finissant
par du fromage.

Faites dorer au four.

Comme boisson, un meursault goutte-d'or me paraît indiqué.

\section*{\centering Gnocchi à la semoule.}
\phantomsection
\addcontentsline{toc}{section}{ Gnocchi à la semoule.}
\index{Gnocchi à la semoule}

Pour six personnes prenez :

\footnotesize
\begin{longtable}{rrrp{16em}}
    250 & grammes & de & semoule,                                                                         \\
    200 & grammes & de & beurre,                                                                          \\
    150 & grammes & de & parmesan râpé,                                                                   \\
      5 & grammes & de & sel,                                                                             \\
        & 1 litre & de & lait,                                                                            \\
        &         &  3 & œufs frais.                                                                      \\
\end{longtable}
\normalsize

Faites bouillir le lait, jetez dedans la semoule en pluie, mettez {\ppp100\mmm}
grammes de beurre et le sel ; laissez cuire. Remuez souvent.

Éloignez la casserole du feu, ajoutez {\ppp125\mmm} grammes de parmesan et les
œufs que vous aurez battus comme pour une omelette ; achevez la cuisson sans
faire bouillir.

Beurrez une plaque de marbre ou de tôle ; étalez dessus l'appareil en une
couche de deux centimètres d'épaisseur environ ; laissez refroidir.

Découpez la pâte en morceaux, les uns rectangulaires, les autres en forme de
cœur, disposez les morceaux par couches dans un plat beurré allant au feu,
saupoudrez chaque couche avec un peu de parmesan, arrosez avec le reste du
beurre et poussez au four chaud pour gratiner.

Servez dans le plat que vous doublerez, pour la commodité du service, d'un
autre plat froid.

On peut servir les gnocchi seuls comme entremets de légumes, ou comme garniture
à certains plats de viande, en particulier avec les escalopes de veau au jambon
qu'ils accompagnent très bien.

\sk

On préparera de même des gnocchni à la farine de maïs.

\section*{\centering Polenta.}
\phantomsection
\addcontentsline{toc}{section}{ Polenta.}
\index{Polenta}

La polenta, mets national italien, est une bouillie de farine de maïs.

Simplement cuite à l'eau, c'est une nourriture grossière.

On l'améliore beaucoup en y ajoutant du beurre, du parmesan râpé et quelquefois
aussi de la sauce tomate.

Pour les entremets sucrés, on la prépare de préférence au lait.

\index{Beignets de polenta}
\index{Croustades de polenta}
\index{Soufflés de polenta}
La polenta accompagne le plus souvent des plats à sauce, des ragoûts, des
salmis, des civets, mais parfois elle est servie seule. On en fait aussi des
croustades, des beignets, des soufflés très appréciés en Italie.

\section*{\centering Croustades de polenta.}
\phantomsection
\addcontentsline{toc}{section}{ Croustades de polenta.}
\index{Croustades de polenta}

Pour six personnes prenez :

\footnotesize
\begin{longtable}{rrrp{16em}}
  1 200 & grammes & d' & eau,                                                                             \\
    400 & grammes & de & farine de maïs fraîchement moulue,                                               \\
    150 & grammes & de & parmesan râpé,                                                                   \\
    100 & grammes & de & beurre,                                                                          \\
     15 & grammes & de & sel gris,                                                                        \\
        &         &  2 & œufs frais,                                                                      \\
        &         &    & huile d'olive.                                                                   \\
\end{longtable}
\normalsize

Faites bouillir l'eau salée avec le sel gris, jetez dedans la farine de maïs en
pluie, laissez cuire pendant une demi-heure environ en remuant constamment avec
une cuiller en bois pour éviter la formation de grumeaux ; ajoutez ensuite le
parmesan et le beurre en tournant toujours sans interruption jusqu'à obtention
d'un mélange de bonne consistance. Laissez refroidir.

Battez les œufs.

Moulez avec la pâte des croustades d'un diamètre de {\ppp6\mmm} centimètres
environ, passez-les dans les œufs battus, puis faites-les frire dans de l'huile
bouillante.

Enlevez sur chaque croustade une partie superficielle mince destinée à faire
couvercle, puis creusez le reste en réservant des parois suffisamment solides.
Garnissez les croustades avec un salpicon maigre ou gras au choix, par
exemple : soit avec un salpicon de laitances et foies de poissons, moules,
huîtres, queues d'écrevisses et de crevettes, lié avec une sauce italienne ;
soit avec un salpicon de foie et de blanc de volaille, langue à l'écarlate, ris
et cervelle de veau, lié avec une sauce suprême ou avec une sauce Périgueux ;
ou encore avec un salpicon de gibier et de truffes lié avec une sauce
demi-glace au fumet de gibier.

Fermez les croustades avec les couvercles ; mettez le tout pendant un instant
au four.

Dressez les croustades sur un plat garni d'une serviette et servez très chaud.

\section*{\centering Kolduny.}
\phantomsection
\addcontentsline{toc}{section}{ Kolduny.}
\index{Kolduny}

Les kolduny, d'origine tartare, sont, comme les ravioli, dont ils ont été les
précurseurs, des cachets de pâte renfermant de la farce ; mais ici la pâte est
moins fine et la farce, simplement pochée en même temps que cuit la pâte qui
l'enrobe, est composée de viande et de graisse crues.

\medskip

Pour faire des kolduny prenez :

\medskip

1° pour la pâte :

\footnotesize
\begin{longtable}{rrrp{16em}}
    750 & grammes & de & farine de blé dur,                                                               \\
     50 & grammes & de & beurre,                                                                          \\
     15 & grammes & de & sel,                                                                             \\
        &         &  2 & œufs frais,                                                                      \\
        &         &    & eau en quantité suffisante pour avoir une pâte de consistance convenable ;       \\
\end{longtable}
\normalsize

\index{Farce pour kolduny}
2° pour la farce :

\footnotesize
\begin{longtable}{rrrp{16em}}
  1 000 & grammes & de & rump-steak juteux,                                                               \\
    500 & grammes & de & graisse de rognon de bœuf,                                                       \\
     20 & grammes & de & beurre,                                                                          \\
     20 & grammes & de & sel,                                                                             \\
      7 & grammes & de & marjolaine en poudre,                                                            \\
      5 & grammes & de & poivre en poudre,                                                                \\
        &         &  1 & oignon.                                                                          \\
\end{longtable}
\normalsize

Avec les éléments contenus dans le premier paragraphe, préparez une pâte
élastique ; fraisez-la ; faites-en une abaisse très mince que vous découperez
en ronds de {\ppp7\mmm} centimètres de diamètre.

Coupez la viande en morceaux gros comme des têtes d'épingle (je recommande de
couper la viande et de ne pas la hacher au hachoir pour éviter de perdre du
sang), hachez la graisse.

Hachez l'oignon et faites-le revenir dans le beurre sans le laisser roussir.

Mélangez viande, graisse, oignon et sa cuisson, sel, poivre et marjolaine, de
façon à obtenir une farce moelleuse\footnote{Si la farce n'était pas
suffisamment moelleuse avec la composition indiquée, ajoutez-y un peu de jus de
viande et de moelle de bœuf.}.

Préparez des noisettes de farce, disposez-les sur les ronds de pâte, puis fermez
les kolduny comme des ravioli, en laissant suffisamment de vide pour que la
dilatation puisse se produire sans que l'enveloppe éclate à la cuisson.

Pour cuire les kolduny, plongez-les dans une grande marmite d'eau bouillante
salée, de façon qu'ils ne se gênent pas les uns les autres ; ils tomberont
d'abord au fond, puis ils remonteront à la surface de l'eau. Lorsqu'ils seront
remontés, attendez quelques minutes, retirez-les, égouttez-les et servez-les
aussitôt, brûlants, sur assiettes chaudes.

Les kolduny doivent être mangés à la cuiller ; ils rendent un jus qui parfume
la bouche et l'on perdrait cette sensation agréable si on les blessait avec une
fourchette.

Les kolduny, très appréciés en Lithuanie, passent pour être un peu lourds,
mais les amateurs à estomac robuste en raffolent et les proportions indiquées
ci-dessus suffiraient à peine pour deux Lithuaniens d'appétit moyen.

En Lithuanie, les kolduny sont servis au début du repas, à peu près comme des
hors-d'œuvre ; chaque convive boit aussitôt après un verre de vieille
eau-de-vie (starka) ; ensuite vient le barszcz très chaud, puis on attaque les
plats de résistance.

\sk

On peut aussi faire cuire les kolduny à pleine friture dans de la graisse fraîche,
pendant quelques minutes.

\section*{\centering Kolduny au gratin.}
\phantomsection
\addcontentsline{toc}{section}{ Kolduny au gratin.}
\index{Kolduny au gratin}

Mettez des kolduny crus dans une sauteuse avec du beurre et faites-les dorer
sans leur permettre de s'attacher.

Dressez-les dans un plat allant au feu, mouillez avec suffisamment de bonne
crème, masquez d'une couche épaisse de parmesan râpé et de mie de pain revenue
dans du beurre, mettez au four pendant une vingtaine de minutes, puis servez.

\section*{\centering Yorkshire pudding.}
\phantomsection
\addcontentsline{toc}{section}{ Yorkshire pudding.}
\index{Yorkshire pudding}

Le Yorkshire padding est très apprécié en Angleterre comme accompagnement
de rôtis de bœuf, notamment avec l'aloyau rôti.

Pour six personnes prenez :

\footnotesize
\begin{longtable}{rrrp{16em}}
    250 & grammes & de & farine,                                                                          \\
     75 & grammes & de & graisse de rognon de bœuf au de rognon de veau, ou encore les deux mélangées,    \\
     15 & grammes & de & sel,                                                                             \\
        & 1 litre & de & lait,                                                                            \\
        &         &  4 & œufs frais,                                                                      \\
        &         &    & graisse de rôti ou saindoux,                                                     \\
        &         &    & muscade.                                                                         \\
\end{longtable}
\normalsize

Préparez une pâte homogène avec tous les éléments moins la graisse de rôti ou
le saindoux ; mettez-la sur une plaque creuse à rebord contenant de lq graisse
de rôti ou du saindoux bien chaud et faites-la cuire au four pendant
{\ppp25\mmm} minutes environ.

Placez ensuite le pudding dans la lèchefrite de la rôtissoire où cuit la pièce
de viande, de façon qu'il s'imprègne du jus qui en découle.

Au moment de servir, découpez le pudding en morceaux carrés, triangulaires ou
losangiques et disposez-les autour de la viande.

\sk

Le Yorkshire pudding peut aussi être servi seul comme entremets de légumes.

\section*{\centering Boulettes au fromage blanc.}
\phantomsection
\addcontentsline{toc}{section}{ Boulettes au fromage blanc.}
\index{Boulettes au fromage blanc}

Pour quatre personnes prenez :

\enlargethispage{10mm}
\footnotesize
\begin{longtable}{rrrp{16em}}
    300 & grammes & de & fromage blanc parfaitement égoutté\footnote{Le fromage blanc est
                         naturellement d'autant meilleur que le lait dont il dérive est
                         meilleur lui-même. Il est indispensable de l'égoutter complètement
                         de façon à obtenir une masse ayant une consistance ferme et cassante,
                         autrement les boulettes risqueraient de se désagréger à la cuisson.
                         On peut, il est vrai, remédier jusqu'à un corlain point à cet
                         inconvénient et donner à la pâte une consistance convenable en
                         augmentant la proportion de farine, mais alors les boulettes sont
                         moins moelleuses.\protect\endgraf},                                              \\
    150 & grammes & de & beurre,                                                                          \\
    100 & grammes & de & chapelure blande,                                                                \\
     15 & grammes & de & sel blanc,                                                                       \\
        &         &  3 & jaunes d'œufs frais,                                                             \\
        &         &  2 & blancs d'œufs frais.                                                             \\
\end{longtable}
\normalsize

Mettez dans une terrine fromage, farine, sel, jaunes et blancs d'œufs, mélangez
bien ; ajoutez {\ppp50\mmm} grammes de beurre fondu au bain-marie ; triturez de
façon à obtenir une pâte homogène, de consistance un peu fluide et sans
grumeaux.

Faites bouillir de l'eau salée à raison de {\ppp15\mmm} grammes de sel par
litre ; prenez de la pâte avec une cuiller à soupe ou à entremets et plongez-la
dans l'eau salée bouillante, chaque cuillerée se prendra en boulette.

Retirez les boulettes au fur et à mesure qu'elles montent à la surface du
liquide, égouttez-les dans une passoire et séparez-les les unes des autres ;
passez-les ensuite dans la chapelure.

Chauffez le reste du beurre dans un plat allant au feu, mettez dedans les
boulettes, faites-les dorer des deux côtés et servez dans le plat.

Ces boulettes, bien préparées, doivent être fermes à la surface, moelleuses à
l'intérieur.

Elles peuvent figurer sur un menu à la place d'un entremets de légumes.

\sk

\index{Boulettes au fromage blanc gratinées}
Comme variante, on pourra remplacer les {\ppp100\mmm} grammes de chapelure par
un mélange de {\ppp50\mmm} grammes de mie de pain rassis tamisée et
{\ppp50\mmm} grammes de parmesan. La préparation ne demandera alors que
{\ppp125\mmm} grammes de beurre et {\ppp10\mmm} grammes de sel.

Le manuel opératoire sera le même pour toute la première partie de
l'apération ; mais les boulettes, une fois enrobées dans le mélange mie de pain
et parmesan, seront mises dans un plat foncé avec {\ppp30\mmm} grammes de
beurre et gratinées au four pendant une demi-heure, durant laquelle on les
arrosera avec le reste du beurre.

\medskip

Ces préparations portent dans leur pays d'origine, la Pologne, le nom de
« Kluski ».

\section*{\centering Galette au fromage.}
\phantomsection
\addcontentsline{toc}{section}{ Galette au fromage.}
\index{Galette au fromage}

Pour six personnes prenez :

\footnotesize
\begin{longtable}{rrrp{16em}}
    400 & grammes & de & farine,                                                                          \\
    350 & grammes & de & beurre,                                                                          \\
    200 & grammes & d' & eau,                                                                             \\
    150 & grammes & de & gruvère ou de parmesan,                                                          \\
     12 & grammes & de & sel,                                                                             \\
        &         &    & jaune d'œuf.                                                                     \\
\end{longtable}
\normalsize

Avec le beurre, la farine, l'eau et le sel, préparez une pâte feuilletée, mais après
chacun des deux derniers tours saupoudrez-la de fromage finement râpé.

Abaissez-la ensuite au rouleau, à une épaisseur de {\ppp1\mmm} centimètre et
demi environ, donnez-lui la forme que vous voudrez, ronde ou carrée ; faites
des entailles sur la face supérieure, dorez-la au jaune d'œuf délayé dans un
peu d'eau, puis mettez-la au four chaud. Une demi-heure de cuisson suffit
généralement pour obtenir un beau feuilleté bien doré.

Il est important de ne pas ouvrir le four avant la fin de la cuisson ;
autrement, le feuilleté s'aplatirait.

\sk

\index{Allumettes au fromage}
Comme variante, on peut, après avoir abaissé la pâte à 1 centimètre ou
à {\ppp3\mmm}/{\ppp4\mmm} de centimètre d'épaisseur, la couper en baguettes.

Ces baguettes, désignées souvent sous le nom d'allumettes au fromage, sont
mises séparées les unes des autres, sur une plaque en tôle et cuites au four
comme la galette.

\medskip

Les allumettes au fromage sont servies comme accompagnement de consommés
clairs où comme entrée.

\section*{\centering Gougère bourguignonne.}
\phantomsection
\addcontentsline{toc}{section}{ Gougère bourguignonne.}
\index{Gougère bourguignonne}

Pour huit personnes prenez :

\footnotesize
\begin{longtable}{rrrp{16em}}
  1 250 & grammes & d' & eau,                                                                             \\
    750 & grammes & de & farine,                                                                          \\
    250 & grammes & de & beurre,                                                                          \\
    250 & grammes & de & gruyère,                                                                         \\
        &         &  8 & œufs frais,                                                                      \\
        &         &    & sel.                                                                             \\
\end{longtable}
\normalsize

Coupez le fromage en petits cubes de {\ppp1\mmm} centimètre de côté environ ;
ne le râpez pas.

Mettez dans une casserole l'eau, le beurre et du sel, amenez à ébullition ;
jetez dedans la farine en pluie, en remuant ; continuez la cuisson à feu
modéré, en tournant toujours, jusqu'à ce que tout goût de farine ait disparu,
ce qui demande une demi-heure environ.

Éloignez la casserole du feu, laissez un peu refroidir, puis cassez dedans les
œufs un à un, en malaxant sans cesse. Dès que la température de la pâte le
permet, achevez le pétrissage à la main ; il faut que la pâte soit bien lisse
sans être trop épaisse. Alors seulement, ajoutez le fromage et mélangez encore.

Disposez la pâte en couronne sur une plaque en tôle sans rebord et poussez au
four chaud. Trois quarts d'heure à une heure de cuisson suffisent.

Il est essentiel pour que la cuisson soit bonne de ne pas ouvrir le four avant
que la pâte soit montée complètement, autrement elle retomberait ; toute la
réussite du plat dépend de là.

La gougère est servie chaude ou froide.

\section*{\centering Gruau de sarrasin.}
\phantomsection
\addcontentsline{toc}{section}{ Gruau de sarrasin.}
\index{Gruau de sarrasin}

Le gruau de sarrasin, peu employé en France, constitue une nourriture très saine.

Il peut servir dans l'alimentation au même titre que le riz, mais il est essentiel
de l'échauder à l'eau bouillante avant de l'employer.

Il existe beaucoup de façons de le préparer : en voici quelques-unes.

\medskip

\textit{A l'eau ou au bouillon}. — Après avoir ébouillanté le gruau, mettez-le dans
un poêlon en terre, couvrez-le d'eau ou de bouillon gras, ou encore de bouillon
de légumes bouillant additionné de plus ou moins de beurre, de sel et de poivre
et faites-le cuire au four doux jusqu'à évaporation du liquide, en évitant que
le gruau s'attache aux parois du vase.

\medskip

\textit{Au beurre}. — Faites cuire le gruau comme précédemment, de façon
à obtenir des grains isolés et bien secs, arrosez-le de beurre fondu ou
faites-le sauter avec du beurre. Le gruau cuit ainsi sera servi seul ou avec du
sucre, de la crème, du lait ; dans ce dernier cas, il pourra constituer un
potage qu'on liera avec des jaunes d'œufs.

\medskip

\textit{Au lard}. — Après avoir ébouillanté le gruau, mettez-le dans une
casserole, ajoutez-y du lard revenu et sa cuisson, plus ou moins de beurre, de
sel et de poivre, mouillez avec de l'eau ou mieux avec du bouillon et faites
cuire au four en casserole couverte.

\medskip

\textit{Au parmesan}. — Après avoir fait cuire le gruau à l'eau ou dans du
bouillon, disposez-le dans un plat beurré allant au feu, par couches
successives saupoudrées de parmesan et arrosées de beurre fondu et faites dorer
au four.

\medskip

\textit{Aux champignons de toutes natures}. — Mettez dans une casserole le
gruau ébouillanté, ajoutez de l'eau, du bouillon ou du lait, au choix, du sel,
du poivre, des aromates au goût et laissez cuire jusqu'à évaporation du
liquide, de façon à obtenir des grains isolés et secs.

Faites cuire doucement des champignons dans du beurre avec plus ou moins
d'oignon, assaisonnez-les ; hachez-les ensuite. Réunissez le gruau, les
champignons et leur cuisson, ajoutez-y des œufs frais, battus, mélangez bien,
chauffez sans faire bouillir ; remuez pour éviter la formation de grumeaux.
Versez le tout dans un plat beurré allant au feu, mettez sur le dessus plus ou
moins de beurre coupé en petits morceaux, du parmesan râpé. Faites dorer au
four doux.

\medskip

\index{Crêpes de gruau de sarrasin}
\index{Galettes de gruau de sarrasin}
Enfin, on peut, avec la farine de gruau de sarrasin, faire des bouillies, des
crêpes, des galettes, etc.

\section*{\centering Couscous.}
\phantomsection
\addcontentsline{toc}{section}{ Couscous.}
\index{Couscous}
\label{pg0704} \hypertarget{p0704}{}

Le mot arabe couscous désignait à l'origine les graines mondées de maïs et de
houlque en épi, très employées toutes deux comme aliment dés la plus haute
antiquité\footnote{ Ces graines, en temps normal, étaient simplement
bouillies ; mais lorsque, par suite d'invasions de sauterelles, les récoltes se
trouvaient en partie détruites, les indigènes mélangeaient les sauterelles avec
ce qu'elles avaient laissé de grain pour se rattraper un peu, et les raffinés
faisaient fermenter le mélange avant de s'en régaler. Aujourd'hui, les amateurs
de sauterelles se font rares. Tout passe.
\protect\endgraf
Dans certains pays, on fait fermenter le couscous de millet, puis on le fait
cuire à la vapeur après l'avoir relevé avec des aromates et des épices et
y avoir ajouté des raisins, qui remplaçent les sauterelles d'antan.}. On donne
aujourd'hui ce nom à de toutes petites boulettes de semoule de millet, de maïs
ou de blé, faites en roulant la semoule dans une sorte de sébile, avec la paume
de la main légèrement mouillée.

Pour cuire le couscous, il est bon d'avoir un appareil spécial composé d'un
vase en terre vernissée remplissant l'office de marmite et d'un autre vase, appelé
passe-couscous, également en terre, dont le fond est percé de petits trous, et qui
peut s'emboîter sur le premier. L'appareil complet est vendu couramment dans
tous les pays où l'on apprête fréquemment ce mets ; il est très commode, mais
non indispensable. On peut le remplacer par une simple marmite en terre et une
passoire fine d'un diamètre convenable pour lui permettre de s'emboîter dans le
col de la marmite.

On prépare le couscous de bien des manières ; voici une formule de couscous
soigné.

\medskip

Pour huit personnes prenez :

\medskip

\footnotesize
\begin{tabular}{@{}lrrrp{16em}}
\normalsize1°\footnotesize \hspace{2em} & 1 000 & grammes  & de & couscous de froment,                    \\
\hspace{2em} & 1 000 & grammes  & de & poitrine de mouton,                                                \\
\hspace{2em} &   400 & grammes  & de & beurre,                                                            \\
\hspace{2em} &   250 & grammes  & de & petites courges d'Algérie, de la grosseur de pommes
                                       de terre de Hollande moyennes,                                     \\
\hspace{2em} &   150 & grammes  & de & fèves vertes,                                                      \\
\hspace{2em} &   150 & grammes  & de & petits pois,                                                       \\
\hspace{2em} &   150 & grammes  & de & pois chiches,                                                      \\
\hspace{2em} &       & 3 litres & de & bouillon,                                                          \\
\hspace{2em} &       &          &  6 & oignons moyens coupés en rondelles,                                \\
\hspace{2em} &       &          &  1 & poule,                                                             \\
\hspace{2em} &       &          &    & un peu de safran (facultatif) ;                                    \\
\hspace{2em} &       &          &    &                                                                    \\
\normalsize 2° & \multicolumn{4}{l}{\normalsize   pour la sauce :}                                        \\
\footnotesize
\hspace{2em} &       &          &    &                                                                    \\
\hspace{2em} &       &          &    & sauce tomate,                                                      \\
\hspace{2em} &       &          &    & piment doux d'Espagne,                                             \\
\hspace{2em} &       &          &    & poivre arabe de Ras-el-Hamont,                                     \\
\hspace{2em} &       &          &    & paprika,                                                           \\
\hspace{2em} &       &          &    & cayenne,                                                           \\
\hspace{2em} &       &          &    & poivre ordinaire.                                                  \\
\end{tabular}
\normalsize

\medskip

Faites tremper pendant {\ppp24\mmm} heures les pois chiches dans de l'eau.

Versez le couscous dans une terrine, mouillez-le légèrement avec du bouillon en
évitant la formation de grumeaux, mettez-le dans le passe-couscous.

Coupez le mouton en morceaux et faites-le revenir pendant {\ppp20\mmm} minutes
dans une sauteuse, de façon à le bien dégraisser.

Faites dorer dans {\ppp100\mmm} grammes de beurre la poule coupée en morceaux
et les oignons (cette opération demande une vingtaine de minutes) ; puis mettez
le tout dans la marmite.

Ajoutez le mouton, les pois chiches ; mouillez avec le bouillon.

Posez le passe-couscous sur la marmite, fermez hermétiquement le joint au moyen
d'un linge mouillé ; faites bouillir.

Lorsque la vapeur aura passé largement au travers du couscous pendant un quart
d'heure, arrêtez l'opération, enlevez le passe-couscous, videz-le dans la
terrine et mouillez de nouveau le couscous avec un peu de bouillon.

Mettez dans la marmite les courges, les fèves et les petits pois, du safran si
vous l'aimez, remettez le couscous dans le passe-couscous, remontez l'appareil
et refaites le joint.

Reprenez la cuisson ; elle sera achevée lorsque la vapeur aura passé de nouveau
pendant un quart d'heure au travers du couscous.

Préparez la sauce et faites-en deux saucières contenant toutes deux les mêmes
éléments, mais l'une plus relevée que l'autre. de façon à satisfaire tous les
goûts. A cet effet, concentrez le jus de cuisson du couscous, dépouillez-le,
dégraissez-le, réservez-en une partie, ajoutez au reste de la sauce tomate,
relevez au goût avec du piment doux d' Espagne, du poivre arabe de
Ras-el-Hamont, du paprika, du cayenne et du poivre ordinaire : vous aurez ainsi
une sauce très renommée, connue sous le nom de sauce Marga.

Au moment de servir, finissez le couscous avec le reste du beurre coupé en
morceaux et le jus de cuisson réservé.

Dressez le couscous sur un plat, disposez au milieu mouton et poule et
garnissez avec les légumes. La préparation complète du couscous demande en tout
trois heures environ.

\sk

\index{Couscous (autre formule)}
Quand on n'a pas de petites courges d'Afrique, de fèves vertes et de petits pois
frais, on peut remplacer ces légumes par :

\footnotesize
\begin{longtable}{rrrp{16em}}
    150 & grammes & de & cardons,                                                                         \\
    150 & grammes & de & haricots blancs frais,                                                           \\
    100 & grammes & de & petits champignons de couche ou 50 grammes de cèpes secs,                        \\
        &         &  4 & fonds d'artichauts.                                                              \\
\end{longtable}
\normalsize

Les pois chiches doivent toujours figurer dans le plat.

\medskip

Le couscous est un aliment très digestible et très assimilable. On lui
a attribué le développement invraisemblable que présentent par derrière comme
par devant certaines femmes d'Orient\footnote{ Ces énormes personnes, qui
donnent l'impression d'oies engraissées pour la production de foies gras, ont,
comme tous les obèses, en même temps qu'un poids global considérable, un poids
spécifique inférieur à la moyenne ; c'est ainsi que j'en ai vu flotter à la
surface de l'eau, dans le plus simple appareil, comme de monstrueuses
vessies.}, mais il est juste de dire que l'abus des sucreries et l'inaction
systématique qui complètent leur régime agissent comme lipogènes plus encore
que le couscous. Quoi qu'il en soit, on arrive facilement, par l'usage de cet
aliment, à garnir les creux, et il m'a paru utile de signaler ici cette
propriété thérapeutique qui peut avoir des applications intéressantes et
devenir la base d'une cure de la maigreur.

\section*{\centering Riz sec.}
\phantomsection
\addcontentsline{toc}{section}{ Riz sec.}
\index{Riz sec}

Quel est le meilleur riz ?

Les opinions sont partagées. Le riz de Novare est excellent, ceux de
Charleston, de Calcutta sont également très bons, les autres viennent ensuite.

En France, on prise généralement peu le riz, parce qu'il est souvent mal
préparé ; en effet, il n est pas rare de voir servir sous ce nom une subslance
agglutinée, colloïdale qui, en réalité, n'a rien d'appétissant,

\label{pg0707} \hypertarget{p0707}{}
Voici comment il faut le préparer.

Lavez d'abord le riz à l'eau tiède, sans le laisser tremper, puis jetez-le en
pluie dans une grande casserole emplie aux trois quarts d'eau salée en
ébullition ({\ppp3\mmm} litres d'eau et {\ppp30\mmm} grammes de sel gris pour
{\ppp250\mmm} grammes de riz).

La cuisson doit être faite en casserole découverte et durer de {\ppp10\mmm}
à {\ppp30\mmm} minutes suivant la nature du riz employé (le riz du Piémont est
l'un de ceux qui cuisent le plus vite), et l'eau doit bouillir suffisamment
fort pendant toute la durée de la cuisson pour soulever et agiter les grains de
riz, afin de les empêcher de s'agglutiner entre eux et de s'attacher à la
casserole. Pour arriver à reconnaître le moment psychologique où il est
à point, il faut, à mesure que la cuisson avance et surtout pendant les
dernières minutes, prendre de temps en temps quelques grains de riz sur une
cuiller et essayer leur résistance sous la dent. La pratique permettra bientôt,
du reste, de se rendre compte du temps nécessaire et suffisant pour atteindre,
avec le riz employé, le résultat voulu.

Quand le riz est cuit, c'est-à-dire lorsqu'il est tendre, bien qu'encore un peu
ferme, mais non croquant, égouttez-le rapidement, étalez-le dans un grand plat,
sur une épaisseur qui ne dépasse pas deux centimètres et faites-le sécher
pendant quelques minutes à l'entrée d'un four, en le remuant une fois ou deux
pour assurer la dessiccation de tous les grains. Le riz doit être sec, non
rôti, ni croustillant, ce qui arriverait si on le laissait trop longtemps
devant le four ; deux ou trois minutes en plus ou en moins ont une grande
influence sur le résultat.

Dressez légèrement le riz en pyramide, en vous servant d'une spatule en bois
que vous glisserez sous les grains pour les prendre, de façon à ne les tasser
ni les casser.

Le riz ainsi apprêté accompagne très bien les plats à sauce.

Lorsqu'il est servi seul, on envoie en même temps, dans une saucière, soit du
beurre fondu clair ou coloré, au goût, soit de la crème douce ou de la crème
plus ou moins aigrie,

\section*{\centering Riz aux cèpes.}
\phantomsection
\addcontentsline{toc}{section}{ Riz aux cèpes.}
\index{Riz aux cèpes}

Pour quatre personnes prenez :

\footnotesize
\begin{longtable}{rrrp{16em}}
    250 & grammes & de & riz,                                                                             \\
    125 & grammes & de & beurre,                                                                          \\
     60 & grammes & de & cèpes secs,                                                                      \\
        &         &    & sel, poivre.                                                                     \\
\end{longtable}
\normalsize

Faites cuire le riz comme il est dit dans la formule du riz sec.
\label{pg0708-1} \hypertarget{p0708-1}{}

Lavez les cèpes à l'eau froide, laissez-les tremper pendant une heure ;
pressez-les pour en exprimer tout excès d'eau et coupez-les en morceaux.
Mettez-les avec le beurre dans une casserole, laissez-les cuire pendant un
quart d'heure, ajoutez ensuite le riz, salez, poivrez au goût, mélangez bien en
secouant pour éviter que le riz prenne couleur, puis servez.

Le riz aux cèpes est une excellente garniture convenant parfaitement avec le
ragoût de mouton, les paupiettes de bœuf braisées, etc.

\sk

On apprêtera de même du riz aux morilles.

\section*{\centering Riz au gras.}
\phantomsection
\addcontentsline{toc}{section}{ Riz au gras.}
\index{Riz au gras}

Pour six personnes prenez :

\footnotesize
\begin{longtable}{rrrp{16em}}
  1 125 & grammes & de & bon bouillon de bœuf ou de veau,                                                 \\
    250 & grammes & de & riz,                                                                             \\
    125 & grammes & de & beurre,                                                                          \\
    125 & grammes & de & moelle de bœuf,                                                                  \\
     50 & grammes & de & fromage de Gruyère râpé (facultatif),                                            \\
        &         &  2 & échalotes moyennes,                                                              \\
        &         &  1 & oignon moyen,                                                                    \\
        &         &    & sel et poivre.                                                                   \\
\end{longtable}
\normalsize

\label{pg0708} \hypertarget{p0708}{}
Hachez les échalotes et l'oignon.

Lavez le riz, laissez-le tremper ensuite dans de l'eau froide pendant une
heure, essuyez-le, séchez-le dans un linge.

Mettez dans une casserole le beurre, la moelle, les échalotes, l'oignon ;
faites cuire sans laisser roussir, passez au tamis, puis ajoutez le riz,
mélangez bien, étalez le tout au fond de la casserole, arrosez avec quelques
cuillerées de bouillon et chauffez en casserole découverte.

Dès que le bouillon sera absorbé, agitez de façon à détacher les grains de riz
qui auratent pu adhérer au fond de la casserole ; mouillez de nouveau avec du
bouillon et continuez ainsi jusqu'à ce que tout le liquide soit absorbé.

Pendant la cuisson, assaisonnez avec sel et poivre après avoir goûté, car
l'assaisonnement à ajouter dépend de celui du bouillon employé.

Au dernier moment, ajoutez le fromage râpé, mélangez bien et servez.

Le riz doit être moelleux et les grains entiers, non crevés.

\sk

\index{Cailles rôties, au riz}
Ce riz au gras, qui est excellent par lui-même, peut être servi avec une
garniture de truffes blanches, de morilles fraîches, de tomates farcies de
champignons grillés, etc. Il accompagne très bien les cailles rôties ; il est
parfait avec un salpicon de foies de volaille, crêtes et rognons de coq,
champignons et truffes noires.

\section*{\centering Riz au maigre.}
\phantomsection
\addcontentsline{toc}{section}{ Riz au maigre.}
\index{Riz au maigre}

Le riz au maigre doit être préparé d'une façon analogue à celle qui a été
indiquée dans la formule précédente, mais avec les différences suivantes :
laitances de poissons au lieu de moelle de bœuf, huile d'olive au lieu de
beurre, bouillon de poisson au lieu de bouillon de bœuf ou de veau.

\sk

\index{Coquillages au riz}
\index{Moules au riz}
\index{Écrevisses au riz}
\index{Queues d'écrevisses au riz}
\index{Escalopes de homard au riz}
\index{Escalopes de langouste au riz}
Le riz au maigre se marie bien avec les moules, les coquillages, les queues
d'écrevisses, les escalopes de homard et de langouste.

\medskip

On peut également servir au centre d'une couronne de riz au maigre un
ragoût de coquillages et de crustacés.

\section*{\centering Riz au beurre clarifié, à l'étouffée.}
\phantomsection
\addcontentsline{toc}{section}{ Riz au beurre clarifié, à l'étouffée.}
\index{Riz au beurre clarifié, à l'étouffée}

Pour quatre personnes prenez :

\footnotesize
\begin{longtable}{rrrp{16em}}
    250 & grammes  & de & riz,                                                                            \\
    100 & grammes  & de & beurre,                                                                         \\
     45 & grammes  & de & sel gris,                                                                       \\
        & 3 litres & d' &  eau,                                                                           \\
        &          &    &  jus de citron (facultatif),                                                    \\
        &          &    &  sel blanc, poivre.                                                             \\
\end{longtable}
\normalsize

Lavez le riz à l'eau froide, puis faites-le blanchir pendant une dizaine de
minutes dans l'eau salée avec le sel gris et aromatisée où non avec du jus de
citron. Rafraîchissez-le, égouttez-le.

\index{Beurre clarifié}
Mettez le beurre dans une casserole, laissez-le fondre, écumez-le et filtrez-le
pour le clarifier ; remettez-le sur le feu, chauffez, ajoutez le riz, salez et
poivrez au goût ; achevez la cuisson au four doux, en casserole fermée, ce qui
demande vingt-cinq minutes environ.

Le riz, ainsi préparé, se présente en grains isolés, plus où moins gonflés, non
dorés.

\medskip

Le riz au beurre clarifié, à l'étouffée, peut être servi seul ou comme
garniture. Il accompagne on ne peut mieux le bifteck à la poêle.

\section*{\centering Riz sauté au beurre noisette.}
\phantomsection
\addcontentsline{toc}{section}{ Riz sauté au beurre noisette.}
\index{Riz sauté au beurre noisette}
\label{pg0710} \hypertarget{p0710}{}

Pour six personnes prenez :

\footnotesize
\begin{longtable}{rrrp{16em}}
    350 & grammes  & de & riz,                                                                            \\
    200 & grammes  & de & beurre,                                                                         \\
     45 & grammes  & de & sel gris,                                                                       \\
        & 3 litres & d' & eau,                                                                            \\
        &          &    & sel blanc, poivre.                                                              \\
\end{longtable}
\normalsize

Lavez le riz sans le laisser tremper ; blanchissez-le pendant une dizaine de
minutes dans l'eau salée avec le sel gris ; rafraîchissez-le, égouttez-le.

Mettez le beurre dans une sauteuse, laissez-le fondre, amenez-le à la couleur
noisette, ajoutez le riz, du sel blanc et du poivre au goût et faites-le sauter
pendant une dizaine de minutes. Tout le beurre doit alors être absorbé.

Le riz sauté au beurre noisette se présente en grains non gonflés, fermes sans
être croustillants, luisants et colorés par le beurre sans être dorés.

Il est excellent,

\section*{\centering Riz à la vapeur.}
\phantomsection
\addcontentsline{toc}{section}{ Riz à la vapeur.}
\index{Riz à la vapeur}

On prépare le riz à la vapeur dans un appareil semblable à celui dans lequel
on cuit le couscous.

On met dans le compartiment du bas du consommé bien assaisonné, de la viande de
boucherie et de la volaille coupées en morceaux et revenues. Le compartiment du
haut reçoit le riz lavé à l'eau froide et égoutté que l'on arrose de beurre
fondu ou de bonne graisse de rôti ; on colore avec un peu de safran : on couvre
l'appareil et on laisse cuire le r1z à la vapeur du liquide pendant une heure.

On sert en même temps, mais à part, le riz et le ragoût.

\section*{\centering Pain de riz.}
\phantomsection
\addcontentsline{toc}{section}{ Pain de riz.}
\index{Pain de riz}

Pour quatre personnes prenez :

\footnotesize
\begin{longtable}{rrrp{16em}}
    200 & grammes & de & riz,                                                                             \\
    200 & grammes & de & truffes,                                                                         \\
     15 & grammes & de & sel,                                                                             \\
    1 & litre 1/2 & de & lait,                                                                            \\
        &         &  4 & œufs frais,                                                                      \\
        &         &    & beurre,                                                                          \\
        &         &    & madère.                                                                          \\
\end{longtable}
\normalsize

Lavez le riz à l'eau froide, puis faites-le blanchir pendant cinq minutes dans
de l'eau bouillante ; rafraîchissez-le, égouttez-le.

Faites bouillir le lait, mettez dedans le riz, le sel et laissez cuire
doucement pendant une heure environ de façon que les grains de riz restent
entiers et en évitant qu'ils s'attachent à la casserole.

Brossez, lavez, séchez les truffes ; faites-les cuire dans du madère ;
émincez-les en julienne.

Cassez les œufs, séparez les blancs des jaunes ; battez les blancs en neige.

Incorporez au riz, hors du feu, la julienne de truffes et les jaunes d'œufs, puis
les blancs.

Mettez cet appareil dans un moule beurré, sans l'emplir.

Faites cuire au bain-marie au four, pendant une demi-heure environ.

Démoulez le pain de riz sur un plat et servez avec une sauce Périgueux.

\sk

On peut faire un pain de riz plus simple en supprimant les truffes et en
remplaçant la sauce Périgueux par une sauce tomate.

\section*{\centering Risotto.}
\phantomsection
\addcontentsline{toc}{section}{ Risotto.}
\index{Risotto}
\label{pg0712-2} \hypertarget{p0712-2}{}

Le risotto est un plat de riz à l'italienne qu'on prépare de façons plus ou
moins analogues à celles indiquées pour le riz au gras ou le riz au maigre et
pour l'épaule de pré-salé au risotto ; mais le fromage qui entre dans sa
composition est toujours du parmesan.

Toutes les combinaisons signalées dans ces recettes lui sont applicables.

\sk

Le risotto à la milanaise est un risotto aromatisé avec du safran et additionné
de champignons.

\sk

Le risotto à la napolitaine est un risotto à la purée de tomates\footnote{Les
purées de tomates d'Italie sont remarquables ; les meilleures sont celles de
Naples et de Palerme.} ({\ppp200\mmm} grammes de purée de tomates pour
{\ppp250\mmm} grammes de riz).

\sk

Le risotto à la piémontaise est un risotto aux truffes du Piémont.

\section*{\centering Pilaf.}
\phantomsection
\addcontentsline{toc}{section}{ Pilaf.}
\index{Pilaf}
\label{pg0712} \hypertarget{p0712}{}

Le pilaf est une préparation de riz au gras, d'origine égyptienne, qui d'Égypte
parvint en Perse et fut introduite en Europe par les Turcs.

Le pilaf primitif était apprêté au bouillon et à la graisse de mouton ; c'est
ainsi que les gens du peuple le font encore dans la péninsule balkanique.

Dans la cuisine soignée, on remplace la graisse de mouton par du beurre et
le bouillon de mouton par un bouillon fait avec du bœuf, du mouton et de la
volaille.

Le riz employé est généralement du riz d'Égypte, à grains très allongés.

Voici une formule concrète de préparation d'un pilaf simple.

\medskip

Pour six personnes prenez :

\footnotesize
\begin{longtable}{rrrp{16em}}
    500 & grammes & de & bouillon de bœuf, mouton et volaille, dûment assaisonné,                         \\
    250 & grammes & de & riz d'Égypte,                                                                    \\
    175 & grammes & de & beurre.                                                                          \\
\end{longtable}
\normalsize

Lavez le riz à plusieurs reprises dans de l'eau chaude (le riz d'Égypte est
généralement sale).

Faites bouillir le bouillon, jetez dedans le riz, laissez-le cuire jusqu'à
absorption complète du liquide, sans qu'il s'attache à la casserole.

Faites fondre le beurre à la couleur noisette, versez-le sur le riz, couvrez,
maintenez la casserole à une température modérée pendant une vingtaine de
minutes, remuez de temps en temps pour éviter l'agglutination des grains, puis
servez.

Les grains de riz doivent être isolés les uns des autres comme des grains de
sable.

\sk

Les personnes qui aiment l'oignon pourront aromatiser le pilaf en faisant
revenir dans le beurre un peu d'oignon.

\sk

On peut encore faire entrer dans le pilaf des aubergines, des tomates, des
gombos, des petits pois, des piments, de l'ail, des raisins de Smyrne, etc.

\sk

Le pilaf est fréquemment servi avec de la viande de boucherie, surtout du
mouton, ou avec de la volaille, du gibier, des crustacés, des mollusques.

Lorsqu'il est servi avec de la viande on y met généralement moins de beurre
que lorsqu'il est servi seul,

\section*{\centering Croquettes de riz au fromage blanc.}
\phantomsection
\addcontentsline{toc}{section}{ Croquettes de riz au fromage blanc.}
\index{Croquettes de riz au fromage blanc}
\index{Croquettes de riz, au fromage}

Prenez du bon fromage blanc, égouttez-le, pressez-le dans un linge pour en
exprimer tout liquide. Mettez-le dans une terrine, ajoutez-y des œufs, du jambon
maigre et du persil hachés, du parmesan râpé, salez et poivrez au goût. Mélangez
de façon à obtenir une pâte compacte que vous partagerez en boulettes d'un centi-
mètre et demi de diamètre environ.

Faites cuire du riz au gras dans lequel vous incorporerez du parmesan râpé ;
laissez-le refroidir.

Enrobez chaque boulette de fromage dans une couche de riz, roulez en
croquettes. Passez ces croquettes successivement dans de la farine, dans de
l'œuf battu et dans de la mie de pain rassis tamisée ; faites-les frire dans de
la friture chaude.

Dressez les croquettes sur un plat couvert d'une serviette, décorez avec du
persil frit et servez.

\section*{\centering Pommes de terre frites.}
\phantomsection
\addcontentsline{toc}{section}{ Pommes de terre frites.}
\index{Pommes de terre frites}
\label{pg0714} \hypertarget{p0714}{}

La préparation des pommes de terre frites est l'enfance de l'art.

Les meilleures pommes de terre à frire sont les pommes de terre dites de
Hollande.

Après les avoir épluchées, on les coupe, à volonté, en quartiers plus ou moins
gros, en rondelles plus ou moins épaisses, en julienne plus ou moins fine (pommes
de terre paille), en copeaux (pommes de terre Chip) ; on les lave, on les essuie
dans un linge, puis on les plonge dans un grand bain de friture chaude de graisse
ou d'huile. Au bout de quelques minutes de cuisson, on agite le bain pour
régulariser la température ; dès que les pommes de terre surnagent, elles sont
cuites ; lorsqu'elles sont dorées à souhait, on les retire avec une écumoire, on
les saupoudre de sel et on les sert aussitôt.

La durée de la cuisson varie de huit à douze minutes, suivant que les pommes de
terre sont coupées en morceaux plus ou moins minces et qu'on les aime plus ou
moins rissolées et croquantes.

\medskip

Les pommes de terre frites peuvent être servies seules, avec ou sans sauce, ou
comme garniture.

\section*{\centering Pommes de terre Chip.}
\phantomsection
\addcontentsline{toc}{section}{ Pommes de terre Chip.}
\index{Pommes de terre Chip}
\label{pg0715-2} \hypertarget{p0715-2}{}

Pelez des pommes de terre, débitez-les, au rabot, en tranches aussi minces que
possible, lavez-les à l'eau courante, en les remuant, jusqu'à ce que l'eau
reste claire.

Égouttez-les, séchez-les dans une serviette, puis plongez-les par dix ou douze
tranches, pas davantage, dans un grand bain de friture bouillante ; elles cuiront
presque instantanément et deviendront croustillantes sans se colorer.

Retirez-les de la friture avec une écumoire, salez-les et servez-les aussitôt.

Par suite de leur peu d'épaisseur et par le fait du lavage qu'elles ont subi,
les pommes de terre Chip ont perdu une forte proportion de fécule et le peu qui
en reste s'est transformé en dextrine où amidon soluble, d'où leur grande
digestibilité.

\medskip

Les pommes de terre Chip ont un goût exquis : elles constituent une jolie
garniture pour toutes sortes de viandes. C'est un aliment de choix pour
diabétiques.

\section*{\centering Pommes de terre soufflées.}
\phantomsection
\addcontentsline{toc}{section}{ Pommes de terre soufflées.}
\index{Pommes de terre soufflées}
\label{pg0715} \hypertarget{p0715}{}

Les pommes de terre soufflées paraissent rarement sur les tables de famille,
leur exécution ayant la réputation d'être compliquée et difficile à réaliser
autre part que dans les cuisines des restaurants. En réalité, leur préparation
est facile.

Les conditions nécessaires et suffisantes pour les réussir sont les suivantes :

1° employer des pommes de terre de Hollande à peau très fine ;

2° avoir deux bains de friture fraîche, à des températures différentes ;

3° faire l'opération en trois temps : un temps de cuisson et deux temps de
soufflage.

L'outillage se compose d'une poêle à anses, d'une poêle à manche ou poêle
à frire, dite coupe lyonnaise, d'une écumoire et d'un panier à manche en fils
métalliques.

Comme friture, on peut prendre du saindoux ; je préfère un mélange de
{\ppp50\mmm} pour {\ppp100\mmm} de saindoux et {\ppp50\mmm} pour {\ppp100\mmm}
de graisse de rognon de bœuf.

Passons à l'exécution.

Pelez les pommes de terre, coupez-les uniformément en tranches de l'épaisseur
d'une pièce de {\ppp5\mmm} francs en argent, essuyez-les,

Pour {\ppp500\mmm} grammes de pommes de terre prêtes à être employées, mettez,
dans une poêle à anses d'une capacité de {\ppp5\mmm} litres, {\ppp2\mmm}
kilogrammes de friture que vous amènerez à la température de la graisse qui
commence à fumer légèrement, et, dans une poêle à manche d'une capacité de
{\ppp2\mmm} litres, {\ppp1\mmm} kilogramme de friture que vous amènerez à la
température plus élevée de la graisse franchement fumante.

Plongez une à une les tranches de pommes de terre dans le grand bain de
friture, en évitant qu'elles se touchent, laissez-les cuire en les agitant
doucement dans le bain et en entretenant toujours la même température. Au bout
de {\ppp7\mmm} à {\ppp8\mmm} minutes, elles monteront à la surface,
retournez-les, laissez-les cuire encore pendant une minute, enlevez-les alors
une à une avec l'écumoire et jetez-les aussitôt dans la poêle à manche. Elles
se souffleront presque instantanément, mais cette soufflure n'est que
transitoire et, si l'on continuait l'opération pour la consolider, les parties
minces brûleraient. Pour éviter cet inconvénient, retirez-les avec l'écumoire
l'une après l'autre dès qu'elles ont subi cette première soufflure, mettez-les
ensemble à égoutter dans le panier ; elles ne tarderont pas à se dégonfler et
à s'agglutiner.

Ces deux premiers temps de l'opération peuvent être exécutés d'avance sans
aucun inconvénient.

Quelques minutes avant de servir, réunissez les deux bains de graisse dans la
poêle à anses et portez la friture à la température de la graisse fumante.
Plongez dedans le panier, retournez-le, les tranches de pommes de terre se
sépareront les unes des autres ; vous les y aiderez du reste au besoin en les
tourmentant avec l'écumoire à plat ; elles se gonfleront à nouveau et monteront
à la surface, soufflées définitivement ; c'est le troisième et dernier temps de
l'opération.

Sortez-les de la friture, égouttez-les, salez-les et servez-les sur un plat
garni d'une serviette.

\medskip

Les pommes de terre soufflées sont une garniture de choix pour viandes grillées.

Les proportions indiquées sont suffisantes pour trois personnes.

\medskip

Théoriquement, deux temps sembleraient devoir suffire, le premier pour la
cuisson, le deuxième pour le soufflage, comprenant la concentration de la
vapeur d'eau, le gonflement des tranches sous l'influence de la pression de
cette vapeur et la solidification de la croûte ; mais il faudrait pour cela que
toutes les tranches fussent absolument identiques les unes aux autres. Or,
pratiquement, cela ne peut avoir lieu que si les pommes de terre sont équarries
et débitées à la machine,

Avec des pommes de terre coupées à la main, si l'on opère en deux temps, on
a le plus souvent des inégalités dans la cuisson et dans la soufflure. C'est
pourquoi il est bon de faire le soufflage en deux temps. Dans le premier temps,
on donne à la surface une certaine imperméabilité tout en lui conservant une
élasticité suffisante et l'on brise les adhérences moléculaires en faisant
souffler la pomme de terre une première fois. Puis, pour protéger les parties
minces, on retire les tranches du bain dès qu'elles ont été soufflées une
première fois, on les laisse se reposer, se ramollir un peu ; elles sont alors
prêtes à supporter uniformément le deuxième temps pendant lequel elles se
gonflent à nouveau et se solidifient à la surface, ce qui assure la permanence
de la soufflure ; on n'a ainsi ni incuits, ni parties brûlées.

\section*{\centering Pommes de terre sautées.}
\phantomsection
\addcontentsline{toc}{section}{ Pommes de terre sautées.}
\index{Pommes de terre sautées}
\label{pg0717} \hypertarget{p0717}{}

Les procédés employés pour la préparation des pommes sautées peuvent être
groupés en deux classes, suivant qu'on opère avec des pommes de terre cuites ou
avec des pommes de terre crues. Lorsqu'on emploie des pommes de terre cuites,
soit à l'eau, soit à la vapeur, il est préférable de les faire sauter alors qu'elles sont
chaudes, car elles absorbent mieux les corps gras.

On peut faire sauter des pommes de terre cuites ou crues dans du beurre, de
l'huile, de la graisse, soit seules, soit avec d'autres substances, notamment des
oignons (pommes de terre à la lyonnaise), des champignons, des truffes, etc.

Voici une formule de pommes de terre sautées à la graisse, avec des poireaux.

\medskip

Pour quatre personnes prenez :

\footnotesize
\begin{longtable}{rrrp{16em}}
    600 & grammes & de & pommes de terre de Hollande, épluchées et taillées
                         en morceaux pouvant fournir des tranches régulières,                             \\
    300 & grammes & de & poireaux (le blanc seulement),                                                   \\
    130 & grammes & de & beurre,                                                                          \\
    125 & grammes & de & panne fraîche,                                                                   \\
     10 & grammes & de & sel gris,                                                                        \\
      5 & grammes & de & sel blanc,                                                                       \\
      5 & grammes & de & persil haché.                                                                    \\
\end{longtable}
\normalsize

Lavez les pommes de terre, coupez-les en tranches minces avec un couteau
à légumes, mettez-les dans un torchon épais avec le sel gris pilé, secouez-les
légèrement, laissez-les en contact avec le sel pendant une dizaine de minutes
pour leur faire rendre tout excès d'eau ; essuyez-les ensuite.

Coupez le blanc des poireaux en lames très minces.

Faites fondre la panne dans une poêle ou dans une sauteuse, sur un feu doux ;
passez la graisse ; activez le feu et, lorsque la graisse sera suffisamment
chaude, mettez dedans les pommes de terre et faites-les sauter de façon à les
amener à être uniformément dorées et flexibles sous le doigt. Le temps moyen
nécessaire pour arriver à ce résultat est une vingtaine de minutes, pendant
lesquelles on fait sauter les pommes de terre six fois.

Faites dorer les poireaux dans {\ppp100\mmm} grammes de beurre, de façon à leur
donner une teinte semblable à celle des pommes sautées.

Mélangez poireaux et pommes de terre, ces dernières ne doivent pas attendre ;
égouttez tout excès de graisse et de beurre, assaisonnez avec le sel blanc,
saupoudrez de persil haché, ajoutez le reste du beurre, laissez-le fondre et
servez.

\section*{\centering Pommes de terre rissolées au beurre.}
\phantomsection
\addcontentsline{toc}{section}{ Pommes de terre rissolées au beurre.}
\index{Pommes de terre rissolées au beurre}

Les pommes de terre rissolées sont des pommes de terre sautées, cuites à petit
feu.

\medskip

Pour trois personnes prenez :

\footnotesize
\begin{longtable}{rrrp{16em}}
    500 & grammes & de & pommes de terre nouvelles, petites et épluchées,                                 \\
    125 & grammes & de & beurre,                                                                          \\
        &         &    & sel.                                                                             \\
\end{longtable}
\normalsize

Prenez une sauteuse suffisamment large pour que les pommes de terre puissent
y tenir toutes sur une seule couche ; mettez dedans le beurre ; lorsqu'il est
chaud, ajoutez les pommes de terre, laissez-les cuire pendant une heure à petit
feu, à découvert, sans que le beurre noircisse, et en les faisant fréquemment
sauter. Salez à la fin de la cuisson.

Au bout d'une heure, tout le beurre doit être absorbé. Servez aussitôt.

Ces pommes de terre, imbibées de beurre, croustillantes à l'extérieur,
moelleuses jusqu'au cœur, sont délicieuses.

\section*{\centering Pommes de terre sautées ou pommes de terre rissolées, a la sauce.}
\phantomsection
\addcontentsline{toc}{section}{ Pommes de terre sautées ou pommes de terre rissolées, a la sauce.}
\index{Pommes de terre sautées ou pommes de terre rissolées, a la sauce}

Pour six personnes prenez :

\footnotesize
\begin{longtable}{rrrp{16em}}
  1 000 & grammes & de & pommes de terre,                                                                 \\
    210 & grammes & de & beurre,                                                                          \\
    200 & grammes & de & crème épaisse,                                                                   \\
     15 & grammes & de & vinaigre de vin,                                                                 \\
     10 & grammes & de & fines herbes hachées,                                                            \\
     10 & grammes & de & sel blanc,                                                                       \\
        &         &    & ail.                                                                             \\
\end{longtable}
\normalsize

Faites sauter ou rissoler les pommes de terre dans {\ppp150\mmm} grammes de
beurre, ne les salez pas.

En même temps, préparez la sauce.

Mettez le reste du beurre dans une casserole, laissez-le fondre, ajoutez-y plus
ou moins d'ail haché, au goût, chauffez doucement pendant quelques instants,
passez ; puis incorporez au beurre aromatisé la crème, le vinaigre, les fines
herbes hachées, le sel, chauffez encore, mélangez bien et versez cette sauce
sur les pommes de terre sautées ou rissolées, dressées dans un plat.

\sk

\index{Artichauts sautés, à la sauce}
\index{Salsifis sautés, à la sauce}
\index{Céleri-rave à la sauce}
\index{Fonds d'artichauts sautés, à la sauce}
On peut accommoder de même les fonds d'artichauts, les raves et les salsifis
sautés.

\section*{\centering Pommes de terre à la crème.}
\phantomsection
\addcontentsline{toc}{section}{ Pommes de terre à la crème.}
\index{Pommes de terre à la crème}

Faites cuire des pommes de terre à petit feu dans un diable Rousset, en
retournant fréquemment le diable ; pelez-les, salez-les et servez-les telles
quelles, en faisant passer en même temps une saucière de belle crème épaisse et
froide.

Chaque bouchée de pomme de terre doit être accompagnée d'un peu de crème ; le
mélange produit une sensation agréable.

La préparation est particulièrement bonne avec des pommes de terre nouvelles
fraîchement récoltées.

\section*{\centering Pommes de terre Anna.}
\phantomsection
\addcontentsline{toc}{section}{ Pommes de terre Anna.}
\index{Pommes de terre Anna}

Les pommes de terre Anna sont des pommes de terre au beurre, cuites au four,
à l'étouffée.

\medskip

Pour six personnes prenez :

\footnotesize
\begin{longtable}{rrrp{16em}}
  1 000 & grammes & de & pommes de terre\footnote{Les meilleures pommes de terre
                                        pour cette préparation sont les pommes de terre
                                        dites de Cavaillon ; les pommes de terre dites de
                                        Hollande conviennent aussi.},                                     \\
    350 & grammes & de & beurre,                                                                          \\
     20 & grammes & de & sel blanc,                                                                       \\
      2 & grammes & de & poivre en poudre.                                                                \\
\end{longtable}
\normalsize

Pelez les pommes de terre, coupez-les en tranches minces au rabot ou avec un
couteau à légumes, lavez-les dans de l'eau froide, égouttez-les, séchez-les
dans un linge.

Faites fondre le beurre et clarifiez-le en enlevant l'écume qui surnage.

Prenez un moule cylindrique en cuivre, à parois et à fond épais, ayant
{\ppp155\mmm} millimètres de diamètre et {\ppp75\mmm} millimètres de hauteur,
beurrez-le au pinceau, mettez au fond une rondelle de papier parcheminé bien
beurré, cela aidera au démoulage ; disposez dans le moule les tranches de
pommes de terre en les faisant chevaucher les unes sur les autres (comme les
tranches de pomme dans le dessus des flans messins), en commençant par le fond,
continuant par les parois, le long desquelles vous placerez les tranches
verticalement, et terminant par l'intérieur. Arrosez au fur et à mesure de
l'opération avec les trois quarts du beurre clarifié, salez et poivrez.

Couvrez le moule avec son couvercle, mettez-le au four doux. Quand la masse se
sera un peu affaissée, ajoutez le reste du beurre en plusieurs fois\footnote{Il
est bon de mettre le beurre en plusieurs fois, autrement une partie déborderait
pendant la cuisson.}, jusqu'à la fin de la cuisson, qui dure en tout une heure.
Démoulez, enlevez le papier et servez.

\medskip

Les pommes de terre Anna se présentent sous la forme d'un gâteau ferme et doré
à la surface, tendre et fondant à l'intérieur. Elles forment un très joli
entremets de légumes. Elles peuvent également être servies comme garniture avec
une pièce de bœuf rôti, une selle de mouton, un gigot d'agneau, etc.

\sk

En remplaçant les tranches de pommes de terre par des pommes de terre paille,
on obtient un gâteau plus croustillant encore, qui a reçu le nom de pommes de
terre Champs-Élysées.

\sk

En ajoutant du fromage râpé aux poumes de terre Anna, on obtient les pommes de
terre Voisin.

\sk

En mélangeant aux pommes de terre Champs-Elysées une julienne de truffes, on
aura les pommes de terre des gourmands.

\section*{\centering Pommes de terre Georgette.}
\phantomsection
\addcontentsline{toc}{section}{ Pommes de terre Georgette.}
\index{Pommes de terre Georgette}

Les pommes de terre Georgette sont des pommes de terre farcies au maigre.

Prenez de belles pommes de terre de Hollande, pelez-les, lavez-les, essuyez-les
et faites-les cuire au four.

Fendez-les par le côté au quart de leur épaisseur pour former couvercle ;
évidez-les en laissant seulement quelques millimètres de pulpe, puis emplissez
les creux avec un salpicon de queues d'écrevisses, huîtres, champignons,
truffes et sauce Nantua serrée. Rabattez le couvercle sur la garniture ;
dressez les pommes de terre sur un plat garni d'une serviette et servez.

C'est un délicieux hors-d'œuvre chaud.

\section*{\centering Pommes de terre Léontine.}
\phantomsection
\addcontentsline{toc}{section}{ Pommes de terre Léontine.}
\index{Pommes de terre Léontine}

Les pommes de terre Léontine sont des pommes de terre farcies au gras et
gratinées,

Apprêtez des pommes de terre comme dans la formule précédente ; emplissez les
vides avec un salpicon de rognons de coq, blanc de volaille, ris de veau,
langue à l'écarlate et sauce suprême endeuillée d'un hachis de truffes.
Rabattez le couvercle sur la garniture ; dressez les pommes de terre dans un
plat de service allant au feu. saupoudrez-les de fromage râpé et faites-les
gratiner au four.

\medskip

Les pommes de terre Léontine, excellent hors-d'œuvre chaud, peuvent aussi
être servies comme entremets de légumes.

\section*{\centering Pommes de terre au jus.}
\phantomsection
\addcontentsline{toc}{section}{ Pommes de terre au jus.}
\index{Pommes de terre au jus}

Pour trois personnes prenez :

\footnotesize
\begin{longtable}{rrrp{16em}}
    600 & grammes & de & belles pommes de terre de Hollande,                                              \\
    250 & grammes & de & jus de viande deux fois plus salé que du bouillon ordinaire.                     \\
    100 & grammes & de & beurre.                                                                          \\
\end{longtable}
\normalsize

Pelez les pommes de terre, laissez-les entières si elles sont petites, coupez-les
en tranches si elles sont grosses, puis mettez-les avec le jus dans un plat allant
au feu et de dimensions convenables pour qu'elles baignent dans le jus.

Faites cuire au four, en arrosant fréquemment.

Lorsque les pommes de terre ne sont plus couvertes par le jus, ajoutez le
beurre par petits morceaux et continuez la cuisson.

Au bout d'une heure un quart à une heure et demie la cuisson est complète ;
les pommes de terre, souples au toucher, sont moelleuses à l'intérieur, dorées et
croustillantes à l'extérieur : elles ont absorbé tout le jus et tout le beurre.

\section*{\centering Pommes de terre au gratin.}
\phantomsection
\addcontentsline{toc}{section}{ Pommes de terre au gratin.}
\index{Pommes de terre au gratin}

Voici trois formules très différentes de pommes de terre au gratin,

\medskip

1° \textit{Au bouillon}.

\medskip

Pour trois personnes prenez :

\footnotesize
\begin{longtable}{rrrp{18em}}
    500 & grammes & de & pommes de terre pelées et émincées en tranches,                                  \\
    400 & grammes & de & bouillon,                                                                        \\
    200 & grammes & de & fromage de Gruyère en lames,                                                     \\
     80 & grammes & de & beurre,                                                                          \\
      5 & grammes & de & sel,                                                                             \\
      1 & gramme  & de & poivre,                                                                          \\
        &         &    & chapelure.                                                                       \\
\end{longtable}
\normalsize

Beurrez un plat allant au feu, mettez dedans des couches alternées de tranches
de pommes de terre crues, de lames de gruyère et de petits morceaux de beurre,
salez, poivrez, terminez par du fromage ; mouillez avec le bouillon sans
couvrir la couche supérieure de fromage, saupoudrez de chapelure et faites
cuire au four, à petit feu, jusqu'à ce que tout le liquide soit absorbé, ce qui
demande une heure environ.

\medskip

2° \textit{À la vapeur et à la crème}.

\medskip

Pour trois personnes prenez :

\footnotesize
\begin{longtable}{rrrp{18em}}
    500 & grammes & de & pommes de terre pelées,                                                          \\
    200 & grammes & de & crème,                                                                           \\
     80 & grammes & de & beurre,                                                                          \\
     50 & grammes & de & parmesan,                                                                        \\
     50 & grammes & d' & oignons,                                                                         \\
     30 & grammes & de & glace de viande,                                                                 \\
     15 & grammes & de & farine,                                                                          \\
      5 & grammes & de & sel,                                                                             \\
      5 & grammes & de & persil haché,                                                                    \\
      1 & gramme  & de & poivre,                                                                          \\
        &         &    & mie de pain rassis tamisée.                                                      \\
\end{longtable}
\normalsize

Faites cuire les pommes de terre à la vapeur.

Épluchez les oignons, hachez-les, faites-les fondre doucement dans {\ppp50\mmm}
grammes de beurre, puis ajoutez la farine, mélangez sans laisser prendre
couleur ; mettez ensuite la crème, la glace de viande, le sel, le poivre, le
persil et laissez mijoter pendant cinq minutes.

Lorsque les pommes de terre sont cuites, coupez-les en tranches, disposez-les
par couches dans un plat beurré, en séparant chaque couche de ses voisines par
un peu de parmesan râpé, arrosez avec la sauce, terminez par une couche de
fromage, saupoudrez de mie de pain, mettez dessus le reste du beurre coupé en
petits morceaux et faites gratiner au four pendant une vingtaine de minutes.

\medskip

3° \textit{Aux œufs}.

\medskip

Pour trois personnes prenez :

\footnotesize
\begin{longtable}{rrrp{18em}}
    500 & grammes & de & pommes de terre pelées et coupées au rabot,                                      \\
    300 & grammes & de & lait,                                                                            \\
    100 & grammes & de & crème,                                                                           \\
     80 & grammes & de & beurre,                                                                          \\
     10 & grammes & de & sel,                                                                             \\
      1 & gramme  & de & poivre (facultatif).                                                             \\
        &         &  3 & œufs frais.                                                                      \\
\end{longtable}
\normalsize

Faites bouillir le lait.

Beurrez un plat de service allant au feu, disposez dedans les pommes de terre,
en couches séparées les unes des autres par de petits morceaux de beurre,
salez, poivrez, mouillez avec le lait et faites cuire au four, à petit feu,
jusqu'à ce que le liquide soit complètement absorbé.

Battez les œufs avec la crème, versez le tout sur les pommes de terre et
finissez la cuisson doucement au four.

Servez dans le plat.

\sk

Dans le gratin dauphinois, on fait cuire les pommes de terre avec du lait dans
lequel on a battu des œufs.

\section*{\centering Pommes de terre aux harengs salés, à la crème.}
\phantomsection
\addcontentsline{toc}{section}{ Pommes de terre aux harengs salés, à la crème.}
\index{Pommes de terre aux harengs salés, à la crème}

Pour six personnes prenez :

\footnotesize
\begin{longtable}{rrrp{16em}}
    750 & grammes & de & pommes de terre\footnote{Les meilleures pommes de terre
                         pour cet usage sont les pommes de terre dites « marchand
                         de vin ». }                                                                      \\
    500 & grammes & de & crème épaisse,                                                                   \\
    250 & grammes & de & beurre,                                                                          \\
        &         &  2 & beaux harengs de Hollande laités, salés,                                         \\
        &         &    & lait,                                                                            \\
        &         &    & sel et poivre.                                                                   \\
\end{longtable}
\normalsize

Mettez les harengs à dessaler dans du lait, puis levez-en les filets et
coupez-les en dés.

Travaillez les laitances avec la crème.

Faites blanchir les pommes de terre dans de l'eau salée bouillante, coupez-les
en tranches pendant qu'elles sont encore chaudes.

Disposez les tranches de pommes de terre dans un plat foncé de {\ppp125\mmm}
grammes de beurre, placez sur chacune d'elles un morceau de hareng, masquez-les
avec la crème, mettez au-dessus le reste du beurre coupé en petits morceaux et
faites gratiner au four pendant une heure.

\section*{\centering Croquettes de pommes de terre.}
\phantomsection
\addcontentsline{toc}{section}{ Croquettes de pommes de terre.}
\index{Croquettes de pommes de terre}

Pour six personnes prenez :

\footnotesize
\begin{longtable}{rrrp{16em}}
  1 000 & grammes & de & pommes de terre à chair jaune,                                                   \\
    125 & grammes & de & beurre,                                                                          \\
        &         &  6 & jaunes d'œufs frais,                                                             \\
        &         &  1 & blanc d'œuf frais,                                                               \\
        &         &    & mie de pain rassis tamisée,                                                      \\
        &         &    & sel, poivre,                                                                     \\
        &         &    & muscade, au goût.                                                                \\
\end{longtable}
\normalsize

Faites cuire les pommes de terre, bien lavées, en robe de chambre dans de
l'eau salée, séchez-les à l'entrée du four, pelez-les et passez-les au tamis.

Mettez la purée dans une casserole avec le beurre, du sel, du poivre au goût,
ajoutez un peu de muscade si vous l'aimez, chauffez en tournant, puis
incorporez-y, hors du feu, les jaunes et le blanc d'œufs.

Divisez la masse en parties égales, du poids de {\ppp50\mmm} à {\ppp60\mmm}
grammes, moulez-les en forme de bouchon, de poire ou de boule un peu aplatie ;
passez-les dans du blanc d'œuf battu, puis roulez-les dans de la mie de pain
rassis tamisée et faites-les frire à pleine friture très chaude pendant
{\ppp5\mmm} à {\ppp6\mmm} minutes. Servez-les aussitôt cuites.

\sk

Comme variante, on pourra mélanger à l'appareil ci-dessus, par kilogramme,
{\ppp250\mmm} à {\ppp300\mmm} grammes de pâte à pets de nonne ou de pâte
à chou, très ferme, sans sucre ; on fera de l'ensemble des croquettes allongées
qu'on finira comme précédemment.

\section*{\centering Pommes de terre duchesse.}
\phantomsection
\addcontentsline{toc}{section}{ Pommes de terre duchesse.}
\index{Pommes de terre duchesse}
\label{pg0725} \hypertarget{p0725}{}

Les pommes de terre duchesse sont d'excellentes croquettes de pommes de terre.

Pour quatre personnes prenez :

\footnotesize
\begin{longtable}{rrrp{16em}}
  1 000 & grammes & de & pommes de terre,                                                                 \\
    300 & grammes & de & beurre,                                                                          \\
     60 & grammes & de & crème épaisse,                                                                   \\
     40 & grammes & de & parmesan râpé,                                                                   \\
        &         &  4 & jaunes d'œufs frais,                                                             \\
        &         &    & sel et poivre.                                                                   \\
\end{longtable}
\normalsize

Pelez les pommes de terre, coupez-les en morceaux et faites-les cuire à l'eau
salée sans qu'elles se défassent. Égouttez-les, séchez-les, passez-les au
tamis : ajoutez-y {\ppp80\mmm} grammes de beurre, la crème, le parmesan et les
jaunes d'œufs ; salez et poivrez au goût ; mélangez bien.

Avec cet appareil, moulez des boulettes de la grosseur d'une noisettge, que
vous ferez cuire dans le reste du beurre de façon qu'elles soient
croustillantes et bien dorées.

\medskip

Les pommes de terre duchesse sont servies comme entremets de légumes ou comme
garniture.

\section*{\centering Beignets de pommes de terre.}
\phantomsection
\addcontentsline{toc}{section}{ Beignets de pommes de terre.}
\index{Beignets de pommes de terre}

Pelez des pommes de terre, râpez-les. Égouttez bien la pulpe, puis ajoutez-y du
persil haché, du sel et du poivre.

Préparez une pâte à frire légère ({\ppp250\mmm} grammes de pâte pour
{\ppp1\mmm} kilogramme de pulpe égouttée), incorporez-la à la pulpe de pommes
de terre ; mélangez bien.

Prenez des petites boulettes du mélange avec une cuiller (une petite cuiller
ronde convient parfaitement) ; jetez-les dans de la friture bouillante ;
laissez-les cuire pendant une dizaine de minutes.

Enlevez les boulettes avec une écumoire, dressez-les entourées d'un cordon de
persil frit sur un plat garni d'une serviette et servez.

\medskip

Ces beignets sont un excellent entremets de légumes et une très bonne garniture
pour pièces de viande rôtie ou grillée.

\section*{\centering Soufflé de pommes de terre.}
\phantomsection
\addcontentsline{toc}{section}{ Soufflé de pommes de terre.}
\index{Soufflé de pommes de terre}

Pour six personnes prenez :

\footnotesize
\begin{longtable}{rrrp{16em}}
    750 & grammes & de & pommes de terre farineuses,                                                      \\
    175 & grammes & de & crème épaisse,                                                                   \\
     65 & grammes & de & beurre,                                                                          \\
     15 & grammes & de & sel,                                                                             \\
        &         &  4 & œufs frais,                                                                      \\
        &         &    & farine ou mie de pain rassis tamisée.                                            \\
\end{longtable}
\normalsize

Lavez les pommes de terre, essuyez-les. faites-les cuire au four ou au diable
Rousset.

Cassez les œufs, séparez les blancs des jaunes ; battez les blancs en neige
ferme.

Beurrez un moule cylindrique de {\ppp17\mmm} centimètres de diamètre avec
{\ppp25\mmm} grammes de beurre, saupoudrez avec un peu de farine ou de mie de
pan rassis tamisée.

Videz les pommes de terre cuites avec une cuiller, passez rapidement la pulpe
au tamis dans une casserole et travaillez-la sur le coin du fourneau avec le
reste du beurre, ajoutez ensuite le sel, les jaunes d'œufs et la crème ;
travaillez bien la pâte, puis incorporez-y d'abord la moitié des blancs battus
en soulevant simplement la masse et en la déplaçant avec une large spatule,
ensuite l'autre moitié en opérant toujours de la même manière pour que
l'appareil ne tombe pas.

Versez l'appareil dans le moule et faites cuire au four doux. Le soufflé
montera beaucoup au début, puis il se tassera en prenant de la consistance.

La cuisson demande une demi-heure environ. On reconnaît qu'elle est à point
lorsque les bords du soufflé se détachent du moule ; à ce moment, il résiste
à la pression du doigt tout en présentant une certaine élasticité.

Sortez le moule du four, attendez pendant quelques minutes : cela facilitera le
démoulage.

Démoulez le soulllé sur un plat et servez aussitôt.

\section*{\centering Purée de pommes de terre.}
\phantomsection
\addcontentsline{toc}{section}{ Purée de pommes de terre.}
\index{Purée de pommes de terre}

La meilleure formule pour préparer la purée de pommes de terre est, à mon avis,
la suivante :

\medskip

Pour trois personnes prenez :

\footnotesize
\begin{longtable}{rrrp{16em}}
    500 & grammes & de & pommes de terre farineuses, pelées,                                              \\
    500 & grammes & de & beurre frais, très fin,                                                          \\
     10 & grammes & de & sel.                                                                             \\
\end{longtable}
\normalsize

Saupoudrez les pommes de terre avec le sel, faites-les cuire à la vapeur,
passez-les au presse-purée dans un plat tenu au chaud, incorporez-y le beurre,
mélangez intimement, goûtez, ajoutez du sel s'il est nécessaire et servez
aussitôt.

\sk

Comme variante, on pourra préparer de la façon suivante une purée de pommes de
terre moins coûteuse et cependant très bonne,

\medskip

Pour trois personnes prenez :

\footnotesize
\begin{longtable}{rrrp{16em}}
    500 & grammes & de & pommes de terre farineuses, pelées,                                              \\
    250 & grammes & de & lait,                                                                            \\
    150 & grammes & de & beurre,                                                                          \\
     10 & grammes & de & sel.                                                                             \\
\end{longtable}
\normalsize

Faites cuire les pommes de terre à la vapeur, comme ci-dessus, mettez-les dans
une casserole sur le feu et écrasez-les comme il faut avec une fourchette en
incorporant le beurre, ajoutez ensuite, par petites quantités, le lait que vous
aurez fait bouillir, tout en fouettant la purée jusqu'à ce qu'elle soit devenue
parfaitement homogène et mousseuse. Goûtez, salez encore un peu si vous le
jugez bon et servez.

\section*{\centering Purée de pommes de terre et de cerfeuil bulbeux.}
\phantomsection
\addcontentsline{toc}{section}{ Purée de pommes de terre et de cerfeuil bulbeux.}
\index{Purée de pommes de terre et de cerfeuil bulbeux}
\index{Cerfeuil bulbeux en purée}

Le cerfeuil bulbeux, à saveur aromatique et sucrée, se marie bien avec la pomme
de terre.

\medskip

Voici un exemple d'association de ces deux légumes,

\medskip

Pour dix à douze personnes prenez :

\footnotesize
\begin{longtable}{rrrp{16em}}
  1 500 & grammes & de & cerfeuil bulbeux,                                                                \\
    750 & grammes & de & pommes de terre,                                                                 \\
    500 & grammes & de & beurre,                                                                          \\
    250 & grammes & de & crème épaisse,                                                                   \\
        &         &    & sel.                                                                             \\
\end{longtable}
\normalsize

Pelez et lavez les pommes de terre et le cerfeuil bulbeux ; faites-les cuire
séparément à la vapeur ; passez-les ensemble au presse-purée ; incorporez-y le
beurre, la crème et du sel ; travaillez bien : chauffez sans faire bouillir,
puis versez la purée dans un légumier et servez.

\medskip

La purée de pommes de terre et de cerfeuil bulbeux est servie comme entremets
de légumes et comme garniture, notamment avec des rôtis de venaison. Elle
remplace avantageusement la classique et monotone purée de marrons.

\sk

On peut varier au goût les proportions de pommes de terre et de cerfeuil
bulbeux, ainsi que celle du beurre et de la crème.

\sk

On peut aussi supprimer complètement les pommes de terre et ne préparer qu'une
purée de cerfeuil bulbeux : on aura ainsi une purée plus douce, également très
bonne.

\section*{\centering Purée de pommes de terre et de haricots verts.}
\phantomsection
\addcontentsline{toc}{section}{ Purée de pommes de terre et de haricots verts.}
\index{Purée de pommes de terre et de haricots verts}

Pour six personnes prenez :

\footnotesize
\begin{longtable}{rrrp{16em}}
    500 & grammes & de & pommes de terre,                                                                 \\
    500 & grammes & de & haricots verts fins,                                                             \\
    275 & grammes & de & beurre,                                                                          \\
    250 & grammes & de & lait,                                                                            \\
     15 & grammes & de & sel,                                                                             \\
      1 & gramme  & de & poivre.                                                                          \\
\end{longtable}
\normalsize

Épluchez les haricots verts, lavez-les, faites-les cuire au four, à l'étouffée,
avec 125 grammes de beurre, la moitié du sel et du poivre ; passez-les en purée.

En même temps, faites cuire à la vapeur les pommes de terre pelées, passez-les
en purée, ajoutez-y le reste du beurre, le reste du sel et du poivre, mélangez
bien ; mouillez avec le lait que vous aurez fait bouillir et que vous
incorporerez à la purée par petites quantités, en la fouettant jusqu'à ce
qu'elle soit devenue mousseuse.

Réunissez purée de pommes de terre et purée de haricots ; mélangez-les
intimement au fouet et servez.

C'est original et bon.

\section*{\centering Purée de pommes de terre et de céleri-rave, à la mayonnaise.}
\phantomsection
\addcontentsline{toc}{section}{ Purée de pommes de terre et de céleri-rave, à la mayonnaise.}
\index{Purée de pommes de terre et de céleri-rave, à la mayonnaise}
\index{Céleri-rave en purée}

La purée de pommes de terre et de céleri-rave à la mayonnaise peut être servie
soit comme entremets de légumes, soit comme salade.

Dans les deux cas, la proportion des légumes n'a rien d'absolu et, suivant le
goût, on peut mettre plus ou moins de l'un ou de l'autre. Deux tiers de céleri
pour un tiers de pommes de terre donnent un mélange qui plaît généralement ;
dans le cas où il paraîtrait trop doux, on augmenterait la proportion des
pommes de terre.

\medskip

Pour quatre personnes prenez, par exemple :

\footnotesize
\begin{longtable}{rrrp{16em}}
    600 & grammes  & de & céleri-rave,                                                                    \\
    300 & grammes  & de & pommes de terre,                                                                \\
     20 & grammes  & de & sel gris,                                                                       \\
        & 2 litres & d' & eau,                                                                            \\
        &          &  1 & œuf dur,                                                                        \\
        &          &    & sauce mayonnaise au citron,                                                     \\
        &          &    & sel blanc,                                                                      \\
        &          &    & poivre fraîchement moulu.                                                       \\
\end{longtable}
\normalsize

Faites cuire le céleri dans l'eau assaisonnée avec le sel gris, les pommes de
terre à la vapeur.

Coupez l'œuf en petits morceaux, puis passez au tamis, à l'aide d'un pilon en
bois, céleri, pommes de terre et œuf : vous obtiendrez ainsi une purée dont
vous complèterez l’assaisonnement à votre goût avec du sel blanc et du poivre.

Laissez refroidir la purée et masquez-la avec une bonne mayonnaise au jus de
citron.

\medskip

Comme entremets de légumes, la purée peut être servie telle quelle, ou décorée
avec des rondelles de truffes cuites au naturel.

Comme salade, il convient d'y ajouter quelques cœurs de laitue passés au jus de
citron.

Dans les deux cas, le plat n'est pas banal et il est bon.

\section*{\centering Patate.}
\phantomsection
\addcontentsline{toc}{section}{ Patate.}
\index{Patate}

La patate, originaire de l'Inde, est aujourd'hui l'objet d'une culture
importante dans tous les pays chauds et en France même, dans la région
méditerranéenne.

C'est un tubercule de la famille des Convolvulacées, à chair tendre et un peu
sucrée. Il en existe plusieurs variétés : rouge, blanche, jaune, violette.

Tous les procédés de cuisson de la pomme de terre peuvent s'appliquer à la
patate.

Les variétés rouge et violette conviennent spécialement pour les entremets
sucrés.

\section*{\centering Topinambours.}
\phantomsection
\addcontentsline{toc}{section}{ Topinambours.}
\index{Topinambours}

Les topinambours sont des plantes originaires de l'Amérique du Sud, de la
famille des Composées.

On emploie en cuisine leurs rhizomes tuberculeux qui ont une saveur douce
avec un goût rappelant celui de l'artichaut.

Le procédé de cuisson le plus recommandable pour les topinambours est la
cuisson au four qui leur enlève leur excès d’eau.

On peut les apprêter au beurre, à la crème, à la béchamel, au jus, en purée. les
faire frire, etc.

\section*{\centering Purée de marrons.}
\phantomsection
\addcontentsline{toc}{section}{ Purée de marrons.}
\index{Purée de marrons}

Pour six personnes prenez :

\footnotesize
\begin{longtable}{rrrp{16em}}
    750 & grammes & de & marrons,                                                                         \\
    150 & grammes & de & crème épaisse,                                                                   \\
    100 & grammes & de & beurre,                                                                          \\
     15 & grammes & de & sel,                                                                             \\
        & 1 litre & de & lait ou d'eau, au goût.                                                          \\
\end{longtable}
\normalsize

Échaudez les marrons, épluchez-les, faites-les cuire dans le lait ou dans l’eau
avec le sel ; la cuisson demande une demi-heure en moyenne. Passez-les au tamis
dans une casserole tenue au chaud, ajoutez-y le beurre et la crème par petites
quantités, sans faire bouillir et en remuant constamment. Goûtez et servez.

\medskip

La purée de marrons est une excellente garniture pour viandes blanches telles
que l'agneau, le veau, pour l'oie, la dinde et les venaisons.

\section*{\centering Fèves fraîches.}
\phantomsection
\addcontentsline{toc}{section}{ Fèves fraîches.}
\index{Fèves fraîches}

Les fèves fraîches sont un excellent légume qu'on peut présenter :

\index{Fèves en hors-d'œuvre}
1° en hors-d'œuvre. Toutes jeunes, crues et fraîchement écossées, servies avec
du beurre frais.

\index{Fèves à l'anglaise}
2° à l'anglaise. Fraîchement écossées, cuites à l'eau salée, égouttées et
servies telles quelles, sous une serviette, comme des pommes de terre en robe
de chambre. Accompagnement de beurre frais.

\index{Fèves au beurre}
3° au beurre. Écossées et débarrassées de leur peau au moment de les apprêter,
on les fait cuire dans de l’eau salée, parfumée avec un bouquet de sarriette,
Lorsquelles sont cuites, on les égoutte bien, puis on leur incorpore du beurre
fin qu'on laisse simplement fondre et on saupoudre de sarriette hachée.

\index{Fèves à la crème}
4° à la crème. Après avoir fait cuire les fèves comme ci-dessus et les avoir
bien égouttées, les additionner de bonne crème épaisse au lieu de beurre et les
saupoudrer de sarriette hachée.

\index{Fèves en purée}
5° en purée. Après les avoir fait cuire à l’eau salée parfumée avec de la
sarriette, les accommoder comme une purée de pommes de terre.

\section*{\centering Petits pois.}
\phantomsection
\addcontentsline{toc}{section}{ Petits pois.}
\index{Petits pois}

Les procédés classiques de préparation des petits pois sont les suivants :

à l'anglaise, en les arrosant ensuite de beurre fondu ;

à la française, cuits avec de l'oignon dans de l'eau salée, puis sucrés plus ou
moins et servis avec une sauce au beurre liée à la farine ;

aux laitues, avec liaison à la crème et au jaune d'œuf ;

au lard :

à l'étouffée, au beurre, comme dans la potée fermière.

\medskip

Voici trois nouvelles et délicieuses façons de les accommoder.

\medskip

A. — Pour quatre personnes prenez :

\footnotesize
\begin{longtable}{rrrrrp{18em}}
  & \multicolumn{3}{r}{2 kilogrammes} & de & petits pois en cosses, jeunes et de bonne qualité,
                                            pouvant donner environ un litre de petits pois écossés,       \\
  & \hspace{2em} &  250 & grammes & de & carottes pelées (le rouge seulement),                            \\
  & \hspace{2em} &  100 & grammes & de & beurre,                                                          \\
  & \hspace{2em} &  100 & grammes & de & crème,                                                           \\
  & \hspace{2em} &    5 & grammes & de & sel blanc,                                                       \\
  & \hspace{2em} &      &         &  1 & cœur de laitue haché,                                            \\
  & \hspace{2em} &      &         &  1 & oignon,                                                          \\
  & \hspace{2em} &      &         &    & bouquet de persil,                                               \\
  & \hspace{2em} &      &         &    & sucre en poudre.                                                 \\
\end{longtable}
\normalsize

Écossez les pois, réservez {\ppp100\mmm} grammes de cosses vertes, tendres,
sans taches, débarrassées de leur pellicule intérieure, coriace, et des
extrémités.

Mettez dans une casserole le beurre, les carottes coupées en petits morceaux,
les cosses réservées, couvrez et faites cuire à petit feu pendant un quart
d'heure ; ajoutez ensuite les petits pois, la laitue, l'oignon, le bouquet de
persil, le sel, du sucre au goût et continuez la cuisson à feu doux pendant une
heure.

Enlevez l'oignon, le bouquet, liez la sauce avec la crème et servez.

Lorsque les petits pois sont de très bonne qualité, les carottes et les cosses
suffisent généralement pour assurer le degré voulu de sucrage.

\medskip

B. — Pour quatre personnes prenez :

\footnotesize
\begin{longtable}{rrrrrp{18em}}
  & \hspace{2em} & 250 & grammes & de & carottes,                                                         \\
  & \hspace{2em} & 200 & grammes & de & crème,                                                            \\
  & \hspace{2em} & 100 & grammes & de & beurre,                                                           \\
  & \hspace{2em} &   5 & grammes & de & sel blanc,                                                        \\
  & \hspace{2em} &     & 1 litre & de & petits pois fraîchement écossés,                                  \\
  & \hspace{2em} &     &         &  2 & bottillons de pointes d'asperges,                                 \\
  & \hspace{2em} &     &         &  1 & oignon,                                                           \\
  & \hspace{2em} &     &         &    & bouquet de persil,                                                \\
  & \hspace{2em} &     &         &    & sucre en poudre.                                                  \\
\end{longtable}
\normalsize

Mettez dans une casserole le beurre, les carottes coupées en gros morceaux,
couvrez et faites cuire à petit feu pendant un quart d'heure ; ajoutez ensuite
les petits pois, l'oignon, le bouquet, le sel, du sucre au goût ; continuez la
cuisson à feu doux pendant une heure.

En même temps, faites cuire les pointes d'asperges dans de l’eau salée ;
égouttez-les ; passez-les en purée.

Retirez les carottes, l'oignon, le bouquet de la casserole, mélangez aux petits
pois la purée d’asperges et la crème, chauffez sans laisser bouillir, puis
servez dans un légumier.

\medskip

C.— Enfin, je signalerai les petits pois à la graisse d'ortolans.

S'il vous arrive un jour de faire rôtir une douzaine d'ortolans enveloppés
pudiquement chacun dans une simple feuille de vigne et enfilés par côté sur une
baguette de coudrier, recueillez soigneusement le jus de la lèchefrite et
faites cuire dedans pendant quelques minutes seulement {\ppp250\mmm} grammes de
petits pois nouveaux, très jeunes et très fins. La graisse des ortolans leur
communiquera un parfum inoubliable.

Ce n'est pas un procédé courant, je n'ai pas besoin de le dire, mais essayez-le
à l'occasion. C'est exquis.

\section*{\centering Pois chiches aux épinards.}
\phantomsection
\addcontentsline{toc}{section}{ Pois chiches aux épinards.}
\index{Pois chiches aux épinards}

Pour quatre personnes prenez :
\footnotesize
\begin{longtable}{rrrp{16em}}
    500 & grammes & de & pois chiches d'Espagne,                                                          \\
    500 & grammes & d' & épinards,                                                                        \\
        &         &  8 & amandes ou 8 noisettes,                                                          \\
        &         &  1 & belle tomate,                                                                    \\
        &         &  1 & œuf dur,                                                                         \\
        &         &    & huile d'olive,                                                                   \\
        &         &    & ail,                                                                             \\
        &         &    & persil,                                                                          \\
        &         &    & safran où cannelle,                                                              \\
        &         &    & poivre ou paprika,                                                               \\
        &         &    & sel.                                                                             \\
\end{longtable}
\normalsize

Mettez les pois à tremper pendant 24 heures dans une assez grande quantité
d'eau salée. Égouttez-les ; puis, après les avoir mis dans de l'eau froide
salée, portez l'eau à l'ébullition et laissez cuire à petits bouillons.

Faites revenir dans de l'huile d'olive la tomate concassée, l'ail et le
persil ; versez le tout dans la casserole où cuisent les pois.

Pilez dans un mortier les amandes ou les noisettes avec le jaune de l'œuf dur,
ajoutez du safran ou de la cannelle au goût, du paprika ou du poivre. Hachez le
blanc de l'œuf et incorporez-le avec le mélange pilé aux pois dont vous
achèverez la cuisson.

Passez le jus de cuisson des pois, faites cuire dedans les épinards,
égouttez-les, concentrez le liquide.

Disposez dans un plat des couches alternées de pois chiches et d'épinards,
arrosez avec un peu de jus de cuisson concentré et servez.

\sk

On pourra faire gratiner le plat. Il suffira pour cela d'en saupoudrer la
couche supérieure avec de la mie de pain rassis tamisée ou avec un mélange de
mie de pain et de fromage de Gruyère râpé, de placer par-dessus quelques petits
morceaux de beurre, puis de pousser au four.

\section*{\centering Haricots verts.}
\phantomsection
\addcontentsline{toc}{section}{ Haricots verts.}
\index{Haricots verts}

Les procédés classiques de préparation des haricots verts sont les suivants :

à l'anglaise, arrosés de beurre fondu ;

sautés au beurre, rissolés ou non, après avoir été blanchis dans de l’eau salée
bouillante ;

à la poulette ;

au jus ;

à l'étouffée, au beurre, comme dans la potée fermière ;

en salade.

\section*{\centering Haricots verts à la crème.}
\phantomsection
\addcontentsline{toc}{section}{ Haricots verts à la crème.}
\index{Haricots verts à la crème}

On peut accommoder les haricots verts à la crème de différentes manières.

\medskip

A. — A la crème fraîche.

\medskip

Pour quatre personnes prenez :

\footnotesize
\begin{longtable}{rrrp{16em}}
    500 & grammes & de & haricots verts fins,                                                             \\
    150 & grammes & de & crème,                                                                           \\
     60 & grammes & de & beurre,                                                                          \\
      5 & grammes & de & persil haché,                                                                    \\
      5 & grammes & de & sel blanc,                                                                       \\
      3 & grammes & de & poivre.                                                                          \\
\end{longtable}
\normalsize

Faites cuire les haricots à découvert dans {\ppp3\mmm} litres d'eau additionnée
de {\ppp15\mmm} grammes de sel gris ; retirez-les dès qu'ils sont flexibles
sous les doigts ; égouttez-les.

Mettez dans une casserole le beurre, les haricots, la crème, le persil, le sel,
le poivre, chauffez, mélangez bien, puis servez.

\medskip

B. — A la crème aigrie.

\medskip

Remplacer dans la formule précédente la crème fraîche par de la crème
légèrement aigrie et ajouter de la glace de viande.

\medskip

C. — A la crème, à l'étouffée.

\medskip

Pour quatre personnes prenez :

\footnotesize
\begin{longtable}{rrrp{16em}}
    500 & grammes & de & haricots verts fins,                                                             \\
    150 & grammes & de & bouillon,                                                                        \\
    100 & grammes & de & crème,                                                                           \\
     90 & grammes & de & beurre,                                                                          \\
     10 & grammes & d' & oignon haché fin,                                                                \\
      7 & grammes & de & farine,                                                                          \\
      5 & grammes & de & persil haché fin,                                                                \\
        &         &    & sel et poivre.                                                                   \\
\end{longtable}
\normalsize

Mettez dans une marmite en porcelaine, allant au feu, {\ppp60\mmm} grammes de
beurre. les haricots, le persil et l'oignon ; mélangez ; fermez la marmite et
faites cuire au four, à l'étouffée, pendant un quart d'heure. Mouillez ensuite
avec le bouillon ; assaisonnez avec sel et poivre en tenant compte du degré
d'assaisonnement du bouillon ; continuez la cuisson à tout petit feu, comme
précédemment, pendant une heure et demie. Maniez le reste du beurre avec la
farine, ajoutez-le aux haricots et laissez cuire encore pendant une demi-heure.
Quelques minutes avant la fin, incorporez la crème et server.

\sk

On pourra apprêter de mêmes manières des haricots mange-tout.

\section*{\centering Haricots écossés.}
\phantomsection
\addcontentsline{toc}{section}{ Haricots écossés.}
\index{Haricots écossés}
\index{Flageolets}

Les haricots écossés (blancs, rouges, panachés, etc. et les flageolets) sont
employés frais ou secs. Lorsqu'ils sont frais, on les fait cuire directement
dans de l'eau bouillante assaisonnée et aromatisée. Lorsqu'ils sont secs, on
les fait tremper d'abord pendant plusieurs heures dans de l’eau fraîche, puis
on les met dans de l'eau froide et l'on fait cuire.

Il existe de nombreuses façons d'accommoder les haricots :

au beurre simplement ;

à la crème ;

à la maître d'hôtel : ils sont d'abord sautés au beurre, puis additionnés de
beurre frais, de persil haché et de jus de citron ;

au jus ;

au velouté ;

au gras, avec graisse et lard ;

au maigre, avec oignons, tomates, sauce blanche ;

aux tomates avec du lard ;

à la lyonnaise, avec oignons émincés et revenus dans du beurre ;

à la bretonne, avec émincés d'oignons tombés à glace, jus de viande et
persil ;

à la provençale, avec huile, ail et anchois pilés ;

à la moutarde, avec jus lié, beurre et moutarde ;

à l'étuvée ;

au gratin ;

en croquettes ;

en salade.

\section*{\centering Haricots blancs à la crème, gratinés.}
\phantomsection
\addcontentsline{toc}{section}{ Haricots blancs à la crème, gratinés.}
\index{Haricots blancs à la crème, gratinés}

Pour quatre personnes prenez :

\footnotesize
\begin{longtable}{rrrrrp{18em}}
  & \hspace{2em} & 700 & grammes & de & haricots de Soissons fraîchement écossés,                         \\
  & \hspace{2em} & 600 & grammes & de & lait,                                                             \\
  & \hspace{2em} & 125 & grammes & de & beurre,                                                           \\
  & \hspace{2em} & 125 & grammes & de & créme épaisse,                                                    \\
  & \hspace{2em} & 100 & grammes & de & champignons de couche pelés,                                      \\
  & \hspace{2em} &  75 & grammes & de & carottes pelées,                                                  \\
  & \hspace{2em} &  60 & grammes & de & gruyère râpé,                                                     \\
  & \hspace{2em} &  20 & grammes & d' & oignon,                                                           \\
  & \hspace{2em} &  20 & grammes & de & sel gris,                                                         \\
  & \hspace{2em} &  15 & grammes & de & farine,                                                           \\
  & \hspace{2em} &  10 & grammes & de & persil,                                                           \\
  & \hspace{2em} &   5 & grammes & de & céleri,                                                           \\
  & \hspace{2em} &   5 & grammes & de & sel blanc,                                                        \\
  & \hspace{2em} &   1 & gramme  & de & poivre,                                                           \\
  & \multicolumn{3}{r}{5 décigrammes} & de & thym,                                                        \\
  & \multicolumn{3}{r}{1 décigramme}  & de & laurier,                                                     \\
  & \hspace{2em} &    & 1 litres & d' & eau,                                                              \\
  & \hspace{2em} &    &          &  1 & petit clou de girofle.                                            \\
\end{longtable}
\normalsize

Mettez dans une casserole l'eau et le sel gris, amenez-la à ébullition, plongez
dedans les haricots, préalablement lavés, et laissez-les cuire à feu assez vif
pendant une heure environ en les conservant entiers.

En même temps, préparez la sauce. Faites bouillir le lait, mettez dedans
champignons, carottes et oignon émincés, céleri, persil, thym, laurier, clou de
girofle, sel blanc et poivre ; laissez cuire à petit feu pendant une heure
environ de façon à obtenir à peu près {\ppp225\mmm} grammes de lait concentré
et aromatisé ; passez-le.

Mettez le beurre avec la farine dans une casserole, tournez sans laisser
prendre couleur, mouillez avec le lait passé ; laissez cuire pendant dix
minutes, puis ajoutez la crème et la moitié du fromage ; tournez encore sans
faire bouillir pendant quatre à cinq minutes de manière à obtenir une crème
onctueuse de consistance sirupeuse.

Égouttez les haricots ; disposez-les dans un plat en porcelaine allant au feu,
versez dessus la crème, saupoudrez avec le reste du fromage, faites gratiner au
four pendant quelques minutes et servez.

\medskip

Les haricots à la crème gratinés peuvent être servis seuls comme entremets de
légumes. Ils accompagnent parfaitement les grillades de mouton.

\sk

On pourra préparer de même des haricots blancs secs après les avoir fait
tremper pendant une douzaine d'heures dans de l’eau fraîche. On les mettra
ensuite dans de l'eau froide et on portera le tout à l'ébullition. Le reste de
la préparation se fera comme ci-dessus ; mais le plat sera naturellement moins
fin.

{\ppp350\mmm} grammes de haricots secs suffisent pour quatre personnes.

\section*{\centering Purée de haricots blancs.}
\phantomsection
\addcontentsline{toc}{section}{ Purée de haricots blancs.}
\index{Purée de haricots blancs}

Pour quatre personnes prenez :

\footnotesize
\begin{longtable}{rrrp{16em}}
    500 & grammes & de & haricots blancs secs\footnote{Si, au lieu de haricots secs,
                         on emploie des haricots frais, il faudra en prendre davantage
                         et il sera inutile de les faire tremper.},                                       \\
    300 & grammes & de & légumes de pot-au-feu, épluchés,                                                 \\
    100 & grammes & de & crème,                                                                           \\
     50 & grammes & de & beurre,                                                                          \\
        &         &    & sel et poivre.                                                                   \\
\end{longtable}
\normalsize

Mettez les haricots dans de l'eau fraîche ; laissez-les tremper pendant une
douzaine d'heures.

Faites-les cuire ensuite, avec les légumes de pot-au-feu, pendant trois heures
dans un peu d'eau salée et poivrée ; au bout de ce temps, le liquide doit être
presque complètement absorbé ou évaporé.

Passez le tout au tamis fin ; ajoutez la crème et le beurre ; fouettez
l'ensemble ; goûtez et complétez l'assaisonnement avec sel et poivre, s'il est
nécessaire.

Cette purée, très moelleuse, accompagne on ne peut mieux le mouton et le porc
grillés.

\section*{\centering Cassoulet.}
\phantomsection
\addcontentsline{toc}{section}{ Cassoulet.}
\index{Cassoulet}


Le cassoulet est un plat renommé du midi de la France (Languedoc). C'est une
plantureuse étuvée de haricots servie dans le récipient même où s'est achevée
sa cuisson, sorte de marmite de forme spéciale en terre rougeâtre nommée
elle-même « cassoulet » ou « cassolet ».

Il existe trois variétés classiques de cassoulet : celui de Castelnaudary,
celui de Carcassonne et celui de Toulouse qui différent entre eux par les
viandes qui entrent dans leur composition. A Castelnaudary, c'est la poitrine
de mouton et le porc frais qui dominent ; à Carcassonne, on emploie le gigot et
la perdrix braisée ; à Toulouse, le confit d’oie ou de canard donne la note
originale. Cependant, il n'y a là rien d'absolu, les préparations empiétant
souvent l'une sur l'autre.

\medskip

Voici trois façons de préparer le cassoulet.

\medskip

\index{Cassoulet au mouton et au confit d'oie}
Au mouton et au confit d'oie.

\medskip

Pour dix à douze personnes prenez :

\footnotesize
\begin{longtable}{rrrp{18em}}
  1 250 & grammes & de & filet de mouton\footnote{On peut remplacer le filet
                                         de mouton par de l'épaule ou du gigot.},                         \\

    500 & grammes & de & lard de poitrine demi-sel,                                                       \\
    250 & grammes & de & couenne maigre légèrement salée,                                                 \\
    250 & grammes & de & saucisse de Toulouse,                                                            \\
    125 & grammes & de & purée de tomates,                                                                \\
    100 & grammes & de & graisse de canard ou de saindoux,                                                \\
    100 & grammes & de & bouillon,                                                                        \\
     80 & grammes & d' & oignons,                                                                         \\
     30 & grammes & d' & échalotes,                                                                       \\
        & 1 litre & de & haricots blancs secs,                                                            \\
        &         &  1 & cuisse de confit d'oie,                                                          \\
        &         &  1 & saucisson à l'ail pesant 300 grammes environ,                                    \\
        &         &    & bouquet garni (persil, thym, laurier),                                           \\
        &         &    & chapelure (facultative),                                                         \\
        &         &    & ail, au goût,                                                                    \\
        &         &    & sel et poivre.                                                                   \\
\end{longtable}
\normalsize

Désossez le mouton, réservez les os. Assaisonnez la viande avec sel et poivre,
ficelez-la en lui donnant une bonne forme.

Triez, lavez les haricots ; mettez-les dans de l’eau froide et faites-les
bouillir pendant une dizaine de minutes ; versez le tout dans une terrine,
couvrez et laissez les haricots tomber au fond ; rejetez ceux qui surnagent,
ils ne cuiraient pas bien.

Mettez dans une grande casserole trois à quatre litres d’eau, faites-la
bouillir, ajoutez-y les haricots égouttés, le lard que vous aurez échaudé, la
couenne roulée en paquet et ficelée, {\ppp30\mmm} grammes de graisse de canard
ou de saindoux, {\ppp50\mmm} grammes d'oignons, {\ppp10\mmm} grammes
d'échalotes et le bouquet mis dans un sachet. Écumez, puis laissez cuire à feu
vif.

Au bout d'une heure de cuisson, ajoutez le saucisson, un peu de poivre.
Continuez la cuisson pendant une heure encore, mais de façon à conserver les
haricots entiers.

En même temps, faites revenir le mouton dans {\ppp40\mmm} grammes de graisse de
canard ou de saindoux ; égouttez la graisse ; ajoutez au mouton les os
réservés, le reste des oignons, le reste des échalotes et de l'ail hachés ;
faites braiser au four doux pendant deux heures. Une demi-heure avant la fin,
mettez la purée de tomates et le bouillon. Achevez la cuisson.

Faites cuire la saucisse dans un peu de graisse de canard ou de saindoux.

Sortez de la marmite le lard, la couenne, le saucisson ; égouttez les haricots ;
réservez leur cuisson. Coupez en morceaux réguliers lard, confit d'oie, couenne,
saucisson et saucisse de Toulouse ; taillez le mouton en tranches.

Disposez dans la marmite spéciale appelée cassoulet ou cassolet des couches
alternées, régulières, de mouton, lard, confit d'oie, couenne, saucisson et
saucisse, séparées par des haricots ; terminez par des rondelles de saucisson
et des morceaux de couenne ; mouillez avec le jus dégraissé et passé du mouton
et suffisamment de bouillon des haricots, saupoudrez ou non de chapelure,
arrosez avec le reste de la graisse de canard ou de saindoux, puis faites
mijoter doucement au four pendant une heure.

Servez dans la marmite.

\sk

\index{Cassoulet au mouton et an porc frais}
Au mouton et au porc frais.

\medskip

Pour dix à douze personnes prenez :

\footnotesize
\begin{longtable}{rrrp{18em}}
  1 000 & grammes & de & gigot ou d'épaule de mouton.                                                     \\
  1 000 & grammes & de & filet de porc ou de jambon frais,                                                \\
    250 & grammes & de & couenne maigre, légérement salée,                                                \\
    250 & grammes & de & bonne sauce tomate,                                                              \\
    100 & grammes & d' & oignons,                                                                         \\
        & 1 litre & de & haricots blancs secs,                                                            \\
        &         &  3 & queues de cochon,                                                                \\
        &         &  2 & oreilles de cochon.                                                              \\
        &         &  1 & petit jambonneau derni-sel,                                                      \\
        &         &  1 & saucisson à l'ail pesant 300 grammes environ,                                    \\
        &         &  1 & gousse d'ail (facultatif),                                                       \\
        &         &    & bouquet garni (persil, thym, laurier),                                           \\
        &         &    & sel et poivre.                                                                   \\
\end{longtable}
\normalsize

Salez et poivrez la veille le filet de porc ou le jambon.

Désossez le mouton ; assaisonnez la viande avec sel et poivre ; ficelez-la ;
réservez les os.

Triez, lavez et blanchissez les haricots comme ci-dessus ; puis mettez-les dans
{\ppp3\mmm} à 4 litres d'eau bouillante avec le jambonneau échaudé, la couenne
roulée en paquet et ficelée, 60 grammes d'oignons et le bouquet dans un sachet,
écumez ; faites cuire à feu vif pendant une heure et demie. Ajoutez ensuite le
saucisson, les queues et les oreilles de cochon, un peu de poivre. Continez la
cuisson pendant une heure encore, à feu régulier, de façon à garder les
haricots entiers,

Faites revenir le mouton avec le reste des oignons dans une braisière, égouttez
la graisse, ajoutez les os réservés, l'ail haché, mettez au four doux et
laissez braiser pendant deux heures ; une demi-heure avant la fin, ajoutez la
sauce tomate.

Faites rôtir ou braiser au four le filet de porc ou le jambon.

Retirez de la marmite le jambonneau, la couenne, le saucisson, les queues et
les oreilles de cochon ; égouttez les haricots ; réservez le liquide de
cuisson.

Coupez en morceaux réguliers jambonneau, couenne, saucisson et queues de
cochon ; émincez en languettes les oreilles ; détaillez en tranches le mouton
et le porc.

Disposez le tout dans le cassoulet, en couches régulières et alternées,
séparées par des haricots ; terminez par des rondelles de saucisson et des
morceaux de couenne, mouillez avec la cuisson dégraissée du mouton, celle du
porc et suffisamment de bouillon des haricots ; achevez la cuisson doucement au
four pendant trois quarts d'heure environ.

Servez dans la marmite.

\sk

\index{Cassoulet au mouton et aux perdrix}
Au mouton et aux perdrix.

\medskip

Pour dix à douze personnes prenez :

\footnotesize
\begin{longtable}{rrrp{18em}}
    500 & grammes & de & mouton maigre,                                                                   \\
    400 & grammes & de & jambon frais,                                                                    \\
    250 & grammes & de & lard maigre peu salé,                                                            \\
    250 & grammes & de & couenne maigre légèrement salée,                                                 \\
    200 & grammes & de & foie de canard,                                                                  \\
    100 & grammes & de & fond de gibier ou de fond de volaille,                                           \\
     80 & grammes & d' & oignons,                                                                         \\
        & 1 litre & de & haricots blancs secs,                                                            \\
        &         &  3 & perdrix,                                                                         \\
        &         &  2 & carottes,                                                                        \\
        &         &  2 & échalotes,                                                                       \\
        &         &  1 & saucisson à l'ail pesant 250 grammes environ,                                    \\
        &         &    & beurre,                                                                          \\
        &         &    & persil,                                                                          \\
        &         &    & bouquet garni,                                                                   \\
        &         &    & girofle en poudre,                                                               \\
        &         &    & sel et poivre.                                                                   \\
\end{longtable}
\normalsize

Plumez, videz, flambez les perdrix ; réservez les foies et les gésiers ;
nettoyez-les. Hachez ensemble jambon, foie de canard, foies et gésiers de
perdrix, échalotes et persil ; assaisonnez cette farce avec sel, poivre et
girofle au goût ; mélangez bien.

Triez, lavez et blanchissez les haricots comme précédemment ; mettez-les dans
{\ppp3\mmm} à {\ppp4\mmm} litres d'eau bouillante avec le lard, la couenne
roulée et ficelée, les carottes, les oignons et le bouquet garni ; écumez, puis
laissez cuire à feu vif pendant deux heures et demie, en ayant soin que les
haricots restent entiers. Une heure avant la fin, ajoutez le saucisson et du
poivre,

Ouvrez les perdrix par le dos, désossez-les, farcissez-les avec le mélange
ci-dessus, ficelez-les en leur donnant une bonne forme, puis faites-les revenir
dans un peu de beurre. Mouillez ensuite avec le fond de gibier ou de volaille,
ajoutez les os des perdrix et mettez à braiser au four doux pendant deux
heures.

Faites rôtir le mouton.

Détaillez en morceaux le lard, la couenne et le saucisson. Coupez le mouton et
les perdrix en tranches. Égouttez les haricots.

Disposez dans la marmite spéciale des couches alternées de perdrix, mouton,
lard, couenne et saucisson, séparées par des haricots, terminez par du
saucisson et de la couenne ; mouillez avec la cuisson des perdrix et du
bouillon des haricots en quantité suffisante, puis achevez la cuisson au four
doux pendant une demi-heure environ.

Servez dans le récipient.

\sk

\index{Cassoulet du Périgord}
Dans le Périgord, on prépare un cassoulet dont la caractéristique consiste dans
l’adjonction de cous d'oies farcis de truffes.

\sk

\index{Cassoulet d'Alsace}
En Alsace, on remplace, dans la préparation du cassoulet, le mouton par du
jarret et du pied de veau ; on y fait encore entrer de l'oie et aussi des
saucisses, du saucisson et du jambon fumés.

\section*{\centering Haricots rouges à l’étuvée.}
\phantomsection
\addcontentsline{toc}{section}{ Haricots rouges à l’étuvée.}
\index{Haricots rouges à l’étuvée}

Pour six personnes prenez :

\footnotesize
\begin{longtable}{rrrp{16em}}
    300 & grammes & de & selle de mouton, dégraissée et sans os,                                          \\
    250 & grammes & de & lard de poitrine,                                                                \\
    250 & grammes & de & bon vin rouge,                                                                   \\
    200 & grammes & de & carottes pelées,                                                                 \\
     75 & grammes & de & beurre,                                                                          \\
     40 & grammes & d' & oignons épluchés,                                                                \\
     30 & grammes & de & navet pelé,                                                                      \\
     20 & grammes & de & farine,                                                                          \\
     15 & grammes & de & sel gris,                                                                        \\
     12 & grammes & d' & échalotes épluchées,                                                             \\
      1 & gramme  & de & poivre fraîchement moulu,                                                        \\
        & 1 litre & de & haricots rouges, frais écossés,                                                  \\
        &         &    & bouquet garni (persil, thym, laurier),                                           \\
        &         &    & quatre épices.                                                                   \\
\end{longtable}
\normalsize

Mettez dans une cocote {\ppp60\mmm} grammes de beurre et la farine, faites
roussir ; ajoutez ensuite carottes, navet, oignons, échalotes et bouquet
garni ; tournez pour leur faire prendre couleur ; poivrez, puis mouillez avec
le vin. Laissez cuire doucement pendant trois quarts d'heure environ.

En même temps, faites cuire les haricots dans {\ppp2\mmm} litres d'eau salée
avec le sel gris, de façon à les conserver entiers. Arrêtez la cuisson dès
qu'ils cèdent sous la pression des doigts.

Tenez le tout au chaud.

Coupez le lard en tranches minces ou en petits cubes, au choix, blanchissez-le
pendant deux minutes dans de l'eau bouillante pour lui enlever son excès de
sel ; séchez-le dans un linge, puis faites-le revenir dans le reste du beurre.

Faites griller ou braiser la selle de mouton ; coupez-la en six tranches.

Retirez de la cocote les carottes, le navet, les oignons, les échalotes et le
bouquet, mettez à leur place les haricots que vous aurez égouttés, ajoutez
{\ppp250\mmm} grammes de bouillon de haricots, le lard et sa cuisson, le mouton
et sa cuisson dégraissée, un peu de quatre épices, au goût, et laissez mijoter
à tout petit feu pendant une heure.

Servez ensuite.

C'est un bon plat substantiel.

\section*{\centering Purée de haricots rouges au vin.}
\phantomsection
\addcontentsline{toc}{section}{ Purée de haricots rouges au vin.}
\index{Purée de haricots rouges au vin}

Pour six personnes prenez :

\footnotesize
\begin{longtable}{rrrp{16em}}
    750 & grammes & de & haricots rouges,                                                                 \\
    500 & grammes & de & vin rouge,                                                                       \\
    375 & grammes & de & lard de poitrine peu salé, non fumé,                                             \\
    150 & grammes & de & beurre,                                                                          \\
     20 & grammes & de & farine,                                                                          \\
        &         &    & carotte,                                                                         \\
        &         &    & oignon,                                                                          \\
        &         &    & bouquet garni,                                                                   \\
        &         &    & sel et poivre.                                                                   \\
\end{longtable}
\normalsize

Mettez à tremper les haricots dans de l’eau pendant {\ppp12\mmm} heures ; puis
faites-les cuire dans de l’eau en quantité suffisante, avec du sel et du
poivre. Lorsque la cuisson est complète, retirez-les et passez-les en purée.

Faites une réduction avec le vin, de la carotte, de l'oignon, au goût, et un
bouquet garni ; passez-la.

Coupez le lard en tranches que vous ferez griller à la poêle dans un peu de
beurre.

Faites un roux avec du beurre et la farine ; mouillez avec la réduction du vin,
ajoutez la purée de haricots, le beurre de cuisson des grillades et du bouillon
de haricots en quantité suffisante ; mélangez. Au dernier moment, mettez le
reste du beurre frais, que vous laisserez simplement fondre.

Versez la purée dans un plat, garnissez-la avec les tranches de lard grillé et
servez.

Les haricots rouges en purée peuvent être servis comme entremets de légumes
ou accompagner une viande grillée, du gibier en particulier.

Lorsque les haricots rouges devront accompagner du gibier grillé, il faudra
d'abord faire cuire dans le vin les débris de la viande et, si la viande a été
marinée, ajouter aussi le vin de la marinade, en observant que la proportion
totale de vin employé ne doit pas dépasser un demi-litre pour {\ppp750\mmm}
grammes de haricots. Tout le reste de la préparation sera exécuté ensuite comme
il est dit plus haut.

\section*{\centering Soja au jus.}
\phantomsection
\addcontentsline{toc}{section}{ Soja au jus.}
\index{Soja au jus}

Pour quatre personnes prenez :

\footnotesize
\begin{longtable}{rrrp{16em}}
  1 000 & grammes & de & germes de soja,                                                                  \\
    100 & grammes & de & jus de veau ou de volaille,                                                      \\
     40 & grammes & de & beurre,                                                                          \\
     40 & grammes & de & vin blanc,                                                                       \\
        &         &    & sel et poivre.                                                                   \\
\end{longtable}
\normalsize

Lavez les germes à l'eau froide, égouttez-les.

Faites fondre le beurre dans une casserole, mettez dedans les germes, tournez
pendant quelques instants, mouillez avec le jus, le vin blanc, assaisonnez au
goût avec sel et poivre ; laissez cuire à feu doux pendant une vingtaine de
minutes.

Servez.

\section*{\centering Lentilles.}
\phantomsection
\addcontentsline{toc}{section}{ Lentilles.}
\index{Lentilles}

Les lentilles peuvent être accommodées comme les haricots. Toutes les
combinaisons culinaires concernant les haricots leur conviennent parfaitement.

\section*{\centering Lentilles aux saucisses.}
\phantomsection
\addcontentsline{toc}{section}{ Lentilles aux saucisses.}
\index{Lentilles aux saucisses}

Pour quatre personnes prenez :

\footnotesize
\begin{longtable}{rrrp{16em}}
    500 & grammes & de & belles lentilles,                                                                \\
    500 & grammes & de & jambon frais,                                                                    \\
    250 & grammes & de & saucisses (longues, plates ou chipolata),                                        \\
    100 & grammes & de & carottes,                                                                        \\
     60 & grammes & d' & oignons,                                                                         \\
     50 & grammes & de & navet,                                                                           \\
        &         &    & bouquet garni,                                                                   \\
        &         &    & sel et poivre.                                                                   \\
\end{longtable}
\normalsize

Nettoyez soigneusement les lentilles, triez-les, ne laissez pas de cailloux ;
lavez-les.

Mettez les lentilles dans une casserole avec de l'eau froide, faites bouillir.
ajoutez carottes, oignons, navet et bouquet garni, sel, poivre ; laissez cuire.

En même temps, faites braiser le jambon assaisonné avec sel et poivre ;
passez-le en purée au tamis.

Enlevez carottes, oignons, navet et bouquet ; égouttez les lentilles ;
mélangez-les avec la purée de jambon et sa cuisson ; laissez mijoter ensemble
pendant une demi-heure.

Faites griller les saucisses.

Dressez les lentilles sur un plat, garnissez avec les saucisses et servez.

\sk

On pourra apprêter de même des lentilles avec du marcassin.

\section*{\centering Épinards à la crème.}
\phantomsection
\addcontentsline{toc}{section}{ Épinards à la crème.}
\label{pg0745} \hypertarget{p0745}{}
\index{Épinards à la crème}

Pour quatre personnes prenez :

\footnotesize
\begin{longtable}{rrrp{16em}}
  1 500 & grammes & d' & épinards d'hiver\footnote{Si l’on emploie des épinards nouveaux
                                                qui sont plus tendres et plus aqueux, ils
                                                absorberont moins de beurre à poids égal
                                                et ils fourniront moins : on devra donc
                                                en prendre une plus grande quantité pour
                                                le même nombre de personnes.},                            \\
    125 & grammes & d' & oseille,                                                                         \\
    125 & grammes & de & beurre,                                                                          \\
    100 & grammes & de & crème,                                                                           \\
        &         &    & sel et poivre.                                                                   \\
\end{longtable}
\normalsize

Épluchez les épinards et l'oseille, c'est-à-dire enlevez les queues, les
parties avariées, les pailles, etc., lavez-les à grande eau à plusieurs
reprises et faites-les cuire dans une grande bassine découverte contenant
quatre litres d'eau bouillante salée à raison de {\ppp10\mmm} grammes de sel
gris par litre d'eau ; écumez au fur et à mesure de la production de l'écume ;
dix à quinze minutes de cuisson suffisent\footnote{Avec des épinards nouveaux
la cuisson est généralement complète au bout d'une à deux minutes.}.

Dès qu'ils cèdent sous la pression des doigts, retirez-les, égouttez-les au
travers d'une passoire, rafraîchissez-les à l'eau froide, pressez-les ensuite
pour exprimer l'excédent d'eau et passez-les ou hachez-les grossièrement
à volonté.

Mettez-les dans une casserole avec {\ppp70\mmm} grammes de beurre, laissez
mijoter pendant un quart d'heure, salez et poivrez au goût. Cinq minutes avant
de servir, ajoutez la crème et le reste du beurre par petits morceaux.

Les épinards cuits peuvent être réchauffés, à condition d'y ajouter chaque fois
qu'on les réchauffe du beurre et de la crème.

Les amateurs d'épinards gras arrivent à faire absorber jusqu'à {\ppp375\mmm}
grammes de beurre et {\ppp300\mmm} grammes de crème à {\ppp1\mmm} kilogramme
d'épinards.

Les épinards sont généralement servis garnis de croûtons frits.

\sk

\index{Chicorée à la crème}
On peut préparer de même la chicorée.

\section*{\centering Épinards au jus.}
\phantomsection
\addcontentsline{toc}{section}{ Épinards au jus.}
\index{Épinards au jus}

Pour quatre personnes prenez :

\footnotesize
\begin{longtable}{rrrp{16em}}
  1 500 & grammes & d' & épinards,                                                                        \\
    300 & grammes & de & bon jus de viande,                                                               \\
     60 & grammes & de & beurre,                                                                          \\
     25 & grammes & de & farine,                                                                          \\
        &         &    & sel et poivre.                                                                   \\
\end{longtable}
\normalsize

Épluchez, lavez et faites cuire les épinards comme il est dit à l'article
précédent.

Mettez dans une casserole {\ppp30\mmm} grammes de beurre et la farine, laissez
cuire pendant trois minutes, ajoutez les épinards, tournez pendant quelques
instants, puis mouillez en trois fois avec le jus de viande ; mélangez bien
après chaque addition de liquide. Goûtez, salez et poivrez s'il est nécessaire,
continuez la cuisson à petit feu pendant une dizaine de minutes, puis mettez le
reste du beurre que vous laisserez simplement fondre.

Servez avec des croûtons frits comme entremets de légumes, sans croûtons
comme accompagnement de viande,

\section*{\centering Épinards à l'italienne.}
\phantomsection
\addcontentsline{toc}{section}{ Épinards à l'italienne.}
\index{Épinards à l'italienne}

Pour quatre personnes prenez :

\footnotesize
\begin{longtable}{rrrrrp{18em}}
  & \multicolumn{3}{r}{2 kilogrammes} & d' & épinards, jeunes et fraîchement cueillis,                    \\
  & \hspace{2em} & 150 & grammes & de & beurre,                                                           \\
  & \hspace{2em} & 100 & grammes & d' & huile d'olive non fruitée,                                        \\
  & \hspace{2em} &  15 & grammes & de & sel,                                                              \\
  & \multicolumn{3}{r}{1 décigramme}  & de & poivre,                                                      \\
  & \hspace{2em} &     &         &  1 & oignon.                                                           \\
\end{longtable}
\normalsize

Épluchez les épinards, lavez-les, séchez-les dans un linge.

Faites fondre le beurre dans une sauteuse ; ajoutez l'huile, l'oignon ;
chauffez, puis mettez les épinards par petites quantités en les saupoudrant, au
fur et à mesure, avec le sel et le poivre. Faites-les cuire à petit feu pendant
trois quarts d'heure environ jusqu'à ce qu'ils aient rendu toute leur eau et
absorbé le beurre et l'huile. Retournez-les fréquemment pendant la cuisson,
retirez l'oignon et servez.

Les épinards préparés de cette manière conservent mieux leur goût que
lorsqu'ils ont été blanchis avant d'être cuits.

\section*{\centering Épinards et champignons gratinés.}
\phantomsection
\addcontentsline{toc}{section}{ Épinards et champignons gratinés.}
\index{Épinards et champignons gratinés}

Pour six personnes prenez :

\footnotesize
\begin{longtable}{rrrrrp{18em}}
  & \multicolumn{3}{r}{2 kilogrammes} & d' & épinards,                                                    \\
  &  \hspace{2em} & 250 & grammes & de & champignons de couche,                                           \\
  &  \hspace{2em} & 200 & grammes & de & beurre,                                                          \\
  &  \hspace{2em} & 200 & grammes & de & fromage de Gruyère râpé,                                         \\
  &  \hspace{2em} &     &         &    & sel et poivre.                                                   \\
\end{longtable}
\normalsize

Épluchez et lavez les épinards ; blanchissez-les dans de l’eau bouillante
salée, rafraîchissez-les, pressez-les, puis faites-les sauter à feu vif avec
{\ppp100\mmm} grammes de beurre, de manière à les sécher.

Épluchez les champignons, coupez-les en morceaux, mettez-les dans une casserole
avec {\ppp50\mmm} grammes de heurre, assaisonnez-les avec sel et poivre ;
laissez-les cuire.,

Mélangez les épinards, les champignons, {\ppp100\mmm} grammes de fromage râpé
et {\ppp45\mmm} grammes de beurre ; mettez ce mélange dans un plat légèrement
beurré avec le reste du beurre et allant au feu, saupoudrez avec le reste du
fromage, faites gratiner vivement.

Servez dans le plat.

\medskip

Ces épinards peuvent être servis seuls ; ils accompagnent très bien aussi le
veau et le mouton.

\section*{\centering Soufflé aux épinards.}
\phantomsection
\addcontentsline{toc}{section}{ Soufflé aux épinards.}
\index{Soufflé aux épinards}
\index{Épinards en soufflé}

Pour cinq personnes prenez :

\footnotesize
\begin{longtable}{rrrp{16em}}
    500 & grammes & de & sauce Béchamel,                                                                  \\
    500 & grammes & d' & épinards,                                                                        \\
    100 & grammes & de & parmesan râpé,                                                                   \\
        &         &  5 & œufs frais,                                                                      \\
        &         &    & beurre,                                                                          \\
        &         &    & sel et poivre.                                                                   \\
\end{longtable}
\normalsize

Épluchez les épinards, faites-les cuire dans de l'eau salée bouillante,
égouttez-les, passez-les au tamis,

Cassez les œufs ; séparez les jaunes des blancs ; fouettez les blancs en neige.

Incorporez intimement aux épinards d'abord la béchamel, puis les jaunes d'œufs
et le fromage ; salez, poivrez ; l'assaisonnement doit être plutôt relevé ;
ajoutez enfin les blancs.

Beurrez un moule à soufflé ; emplissez-le aux trois quarts avec l'appareil
ci-dessus et poussez au four chaud pendant {\ppp10\mmm} à {\ppp12\mmm} minutes.

Le soufflé d'épinards servi seul constitue un excellent entremets de légumes.

\sk

\index{Chicorée en soufflé}
\index{Oseille en soufflé}
\index{Choux-fleurs en soufflé}
\index{Pommes de terre en soufflé}
\index{Choux-fleurs en soufflé}
\index{Salade cuites en soufflé}
On peut préparer de même, avec ou sans fromage, toute une série de soufflés de
légumes : pommes de terre, oseille, choux-fleurs, salades cuites, etc.

\section*{\centering Oseille.}
\phantomsection
\addcontentsline{toc}{section}{ Oseille.}
\index{Oseille}

L'oseille peut être préparée au jus ou sans jus.

\medskip

\begin{center}
\textit{Formule sans jus.}
\end{center}

Pour six personnes prenez :

\footnotesize
\begin{longtable}{rrrrrp{18em}}
  & \multicolumn{3}{r}{2 kilogrammes} & d' & oseille,                                                     \\
  &  \hspace{2em} & 125 & grammes & de & beurre,                                                          \\
  &  \hspace{2em} &  50 & grammes & de & crème,                                                           \\
  &  \hspace{2em} &     &         &  3 & jaunes d'œufs frais,                                             \\
  &  \hspace{2em} &     &         &    & sel et poivre.                                                   \\
\end{longtable}
\normalsize

Épluchez l'oseille en retirant entièrement les queues, lavez-la soigneusement
à grande eau à plusieurs reprises ; égouttez-la ; mettez-la ensuite dans une
casserole sur le feu pour la faire fondre et lui faire suer son eau, si elle
est jeune ; ou, si elle est vieille, faites-la blanchir dans de l'eau salée
bouillante ; laissez donner deux ou trois bouillons, puis versez l'oseille sur
un tamis pour l'égoutter.

Mettez dans une casserole {\ppp100\mmm} grammes de beurre, l'oseille égouttée,
laissez mijoter pendant une demi-heure ; puis ajoutez le reste du beurre coupé
en petits morceaux, laissez-le simplement fondre, salez et poivrez au
goût\footnote{Les personnes qui trouveraient trop acide l'oseille ainsi
préparée pourront en adoucir le goût en ajoutant un peu de sucre en poudre.}.

Mêlez dans un bol les trois jaunes d'œufs et la crème, liez l'oseille avec ce
mélange, chauffez sans faire bouillir pendant cinq minutes et servez.

\begin{center}
\textit{Formule au jus.}
\end{center}

Pour six personnes prenez :

\footnotesize
\begin{longtable}{rrrrrp{18em}}
  & \multicolumn{3}{r}{2 kilogrammes} & d' & oseille,                                                     \\
  &  \hspace{2em} & 125 & grammes & de & bon jus,                                                         \\
  &  \hspace{2em} & 100 & grammes & de & beurre,                                                          \\
  &  \hspace{2em} &  50 & grammes & de & crème,                                                           \\
  &  \hspace{2em} &  15 & grammes & de & farine,                                                          \\
  &  \hspace{2em} &     &         &  3 & jaunes d'œufs frais,                                             \\
  &  \hspace{2em} &     &         &    & sel et poivre.                                                   \\
\end{longtable}
\normalsize

Commencez la préparation comme il est dit ci-dessus. Pendant que l'oseille
égouttée mijote, faites cuire la farine pendant trois minutes dans {\ppp60\mmm}
grammes de beurre, mouillez avec le jus, ajoutez le tout à l’oseille, salez,
poivrez, continuez la cuisson pendant une quinzaine de minutes encore, enfin
liez avec la crème et les jaunes d'œufs.

Ici la farine est presque indispensable pour donner du corps, étant donnée la
proportion relativement grande de liquide.

\medskip

L'oseille est servie soit comme entremets de légumes, soit comme garniture.

\medskip

Comme entremets de légumes, on la garnit généralement avec des croûtons frits
ou avec des galettes feuilletées.

\medskip

\index{Garniture pour le veau}
\index{Garniture pour œufs}
Comme garniture, elle accompagne bien les œufs durs, les œufs sur le plat, les
œufs pochés et les œufs frits. Elle s'associe parfaitement avec le veau.

\section*{\centering Oxalis.}
\phantomsection
\addcontentsline{toc}{section}{ Oxalis.}
\index{Oxalis}

Les oxalis ou surelles, de la famille des Oxalidées, dont les espèces sont
nombreuses, poussent sous toutes les latitudes.

Les rhizomes tubéreux, translucides, de quelques-unes d'entre elles sont
comestibles : \textit{oxalis crenala}, à tubercules jaunes ; \textit{oxalis
tuberosa} ou \textit{Oca du Pérou}, à tubercules jaunes et rouges ;
\textit{oxalis carnosa} ; \textit{oxalis esculenta}, etc.

\medskip

On peut apprêter les oxalis de bien des manières : à l'étouffée, sautées,
frites, au jus, en purée, farcies, au gratin, etc.

On fait aussi des potages crème d'oxalis.

Enfin, les jeunes pousses peuvent être accommodées en salade.

\section*{\centering Purée de cresson.}
\phantomsection
\addcontentsline{toc}{section}{ Purée de cresson.}
\index{Purée de cresson}
\index{Cresson (Purée de)}

Pour six personnes prenez :

\footnotesize
\begin{longtable}{rrrp{16em}}
    250 & grammes & de & crème épaisse,                                                                   \\
    100 & grammes & de & beurre,                                                                          \\
     10 & grammes & de & farine,                                                                          \\
        &         & 10 & bottes de cresson,                                                               \\
        &         &    & sel et poivre.                                                                   \\
\end{longtable}
\normalsize

L'opération se fait en deux temps.

\medskip

1° La veille du jour où vous voudrez servir ce plat, épluchez le cresson,
enlevez-en les grosses côtes, lavez-le à plusieurs reprises, égouttez-le et
faites-le blanchir pendant une douzaine de minutes dans de l'eau salée
bouillante, rafraîchissez-le et passez-le ensuite au tamis. Laissez-le reposer
jusqu'au lendemain : il rendra une certaine quantité d'eau que vous enlèverez.

2° Le lendemain, lorsque le moment de l’apprêter sera venu, mettez dans une
casserole le beurre et la farine, faites cuire sans laisser roussir, puis
ajoutez la purée de cresson bien égouttée, la crème, assaisonnez au goût et
chauffez à petit feu, en remuant, pendant le temps nécessaire pour rendre le
tout bien homogène et l'amener à une consistance convenable.

\medskip

\index{Garniture pour le veau}
\index{Garniture pour l'agneau}
Cette purée est excellente avec le veau ou l'agneau.

\medskip

Servie seule, elle devra être garnie de croûtons frits dans du beurre.

\section*{\centering Endives au beurre.}
\phantomsection
\addcontentsline{toc}{section}{ Endives au beurre.}
\index{Endives au beurre}

Pour quatre personnes prenez :

\footnotesize
\begin{longtable}{rrrrrp{18em}}
  & \hspace{2em}  & 1 000 & grammes & d' & endives,                                                       \\
  & \hspace{2em}  &   150 & grammes & de & beurre,                                                        \\
  & \hspace{2em}  &     6 & grammes & de & sel,                                                           \\
  & \multicolumn{3}{r}{1 décigramme}  & de & poivre.                                                      \\
\end{longtable}
\normalsize

Épluchez les endives et, si elles ne renferment pas de sable, essuyez-les sans les
laver ; autrement, lavez-les rapidement sans les laisser tremper, pour éviter de leur
donner de l'amertume, et séchez-les dans un linge avant de les faire cuire.

Faites fondre le beurre dans une casserole, mettez les endives, salez, poivrez,
placez par-dessus un papier beurré et laissez mijoter à tout petit feu, en
casserole ouverte, pendant deux heures.

Ce mode de préparation des endives est surtout recommandable au début de la
saison.

\sk

On peut préparer de méme des laitues et des scaroles.

\section*{\centering Laitues braisées.}
\phantomsection
\addcontentsline{toc}{section}{ Laitues braisées.}
\index{Laitues braisées}

Pelez et coupez en tranches des oignons doux d'Espagne.

Coupez des tomates en morceaux ; enlevez-en les pépins.

Faites revenir oignons et tomates dans du beurre ; passez-les en purée.

Blanchissez pendant quelques instants, dans de l’eau salée bouillante, des laitues
nettoyées et lavées ; égouttez-les.

Disposez la purée d'oignons et de tomates dans un plat beurré allant au feu,
mettez dessus les laitues, masquez avec du fond de veau et faites braiser très
doucement au four pendant deux heures.

\section*{\centering Scaroles au jus.}
\phantomsection
\addcontentsline{toc}{section}{ Scaroles au jus.}
\index{Scaroles au jus}

Pour trois personnes prenez :

\footnotesize
\begin{longtable}{rrrp{16em}}
    300 & grammes & de & jus de viande,                                                                   \\
    100 & grammes & de & beurre,                                                                          \\
        &         &  6 & belles scaroles,                                                                 \\
        &         &  6 & petites bardes de lard,                                                          \\
        &         &  6 & petits oignons épluchés, pesant ensemble 25 grammes environ,                     \\
        &         &    & sel et poivre.                                                                   \\
\end{longtable}
\normalsize

Nettoyez et lavez soigneusement les scaroles, faites-les blanchir dans de l’eau
salée bouillante, égouttez-les sur un tamis, puis enveloppez chaque cœur dans
une barde que vous attacherez avec une ficelle.

Mettez dans une braisière {\ppp60\mmm} grammes de beurre, faites suer dedans,
pendant vingt minutes, les scaroles bardées et les oignons, mouillez ensuite
avec le jus de viande, laissez cuire pendant une heure trois quarts ;
concentrez alors la sauce, dégraissez-la bien, ajoutez le reste du beurre coupé
en petits morceaux, goûtez et complétez l'assaisonnement, s'il y a lieu, avec
sel et poivre.

Retirez les ficelles, les bardes et les oignons ; dressez les scaroles sur un
plat, masquez-les avec la sauce et servez.

\sk

\index{Laitues au jus}
\index{Endives au jus}
On peut préparer de même des laitues et des endives.

\section*{\centering Chicorée au jus.}
\phantomsection
\addcontentsline{toc}{section}{ Chicorée au jus.}

Prenez des chicorées frisées, plutôt vertes, bien saines et bien fraîches ;
épluchez-les soigneusement feuille par feuille afin de ne laisser ni ver, ni
paille, ni partie avariée, lavez-les à plusieurs reprises dans beaucoup d'eau,
puis faites-les cuire dans de l’eau salée bouillante, ce qui demande
{\ppp20\mmm} à {\ppp25\mmm} minutes environ.

Rafraîchissez-les ensuite à l'eau froide, égouttez-les, pressez-les, hachez-les
grossièrement.

Faites blondir de la farine dans du beurre, mouillez avec du jus de viande, de
préférence avec du jus de veau et volaille ; laissez cuire pendant quelques
minutes, puis ajoutez les chicorées ; mélangez bien, couvrez avec un papier
beurré et poussez au four doux.

Au bout d'une demi-heure de cuisson, goûtez, salez et poivrez s'il y a lieu.
aromatisez avec un peu de muscade si vous l’aimez et ajoutez du beurre frais
par petits morceaux que vous laisserez simplement fondre.

Servez avec ou sans croûtons frits.

\medskip

\index{Garnitures pour viandes blanches}
La chicorée au jus accompagne on ne peut mieux les viandes blanches.

\sk

\index{Scaroles au jus}
\index{Épinards au jus}
On peut préparer de même la scarole, les épinards, etc.

\section*{\centering Pain de chicorée.}
\phantomsection
\addcontentsline{toc}{section}{ Pain de chicorée.}
\index{Pain de chicorée}
\index{Chicorée (Pain de)}

Pour huit personnes prenez :

\footnotesize
\begin{longtable}{rrrp{16em}}
    750 & grammes & de & crème,                                                                           \\
    250 & grammes & de & beurre,                                                                          \\
     20 & grammes & de & farine,                                                                          \\
        &         &  6 & belles chicorées d'hiver pouvant fournir 2 kilogrammes
                         environ de chicorée cuite,                                                       \\
        &         &  4 & œufs,                                                                            \\
        &         &    & sel et poivre.                                                                   \\
\end{longtable}
\normalsize

Épluchez les chicorées, lavez-les soigneusement à l’eau froide.

Faites-les cuire dans de l'eau salée bouillante, égouttez-les, rafraîchissez-les
dans de l'eau froide, mettez-les ensuite dans un torchon et tordez pour exprimer
l'eau en excès. Hachez-les.

Mettez dans une casserole {\ppp60\mmm} grammes de beurre et la chicorée hachée,
laissez cuire pendant cinq minutes, versez dessus {\ppp100\mmm} grammes de
crème, tournez, ajoutez encore {\ppp60\mmm} grammes de beurre et {\ppp125\mmm}
grammes de crème par petites portions, cn continuant à tourner sans faire
bouillir ; cette opération dure de quinze à vingt minutes.

Éloignez la casserole du feu, cassez dedans les œufs, salez, poivrez, mêlez
intimement.

Beurrez un moule, surtout au fond ({\ppp40\mmm} à {\ppp50\mmm} grammes de
beurre), versez dedans la chicorée et faites cuire au bain-marie pendant une
heure et demie,

Servez avec une sauce préparée de la façon suivante : mettez dans une casserole
le reste du beurre et la farine, tournez sans laisser prendre couleur ajoutez
{\ppp250\mmm} grammes de crème, mélangez bien ; dès que la sauce épaissit
remettez de la crème, salez, poivrez et faites cuire pendant une demi-heure
à feu très doux en tournant constamment. À la fin, si la sauce est trop
épaisse, ajoutez tout ou partie de ce qui reste de crème.

\section*{\centering Fenouil.}
\phantomsection
\addcontentsline{toc}{section}{ Fenouil.}
\index{Fenouil}
\index{Finocchi}
\index{Fenouil en hors-d'œuvre}
\index{Fenouil an salade}

Le fenouil est une plante aromatique de la famille des Ombellifères.

Peu employé dans la cuisine française, son usage est très répandu dans d'autres
contrées. Dans les pays slaves, on emploie surtout les feuilles hachées, comme
condiment dans les potages et dans les cuissons, un peu comme le persil chez
nous. En Italie, sous le nom de \textit{finocchi}, on sert les bourgeons de
fenouil, crus en hors-d'œuvre ou en salade, cuits au beurre ou au jus, comme
légumes. On les fait aussi gratiner : ils sont alors présentés avec une sauce
à la crème ou avec une sauce tomate.

Les finocchi accompagnent souvent une pièce de veau.

\medskip

Voici trois formules de préparation :

\medskip

\index{Fenouil au jus}
A. \textit{ Au jus}. — Coupez les bourgeons de fenouil en deux ou en quatre
morceaux dans le sens de leur longueur, faites-les blanchir pendant une
quinzaine de minutes environ dans de l'eau salée, puis achevez doucement la
cuisson dans du bon jus. Il faut un demi-litre de jus bien aromatisé pour six
bourgeons. Dressez le fenouil sur un plat, concentrez le jus, goûtez pour
l’assaisonnement et versez-le sur les finocchi.

\medskip

\index{Fenouil gratiné, sauce à la crème}
B. \textit{ Gratiné, sauce à la crème}. — Coupez et faites blanchir le fenouil ;
préparez ensuite une sauce blanche à la crème. Mettez dans un plat beurré
allant au feu des couches alternées de fenouil cuit, de sauce et de parmesan
râpé, en finissant par du fromage ({\ppp40\mmm} grammes de parmesan environ pour
{\ppp6\mmm} bourgeons de fenouil), puis achevez la cuisson au four.

\medskip

\index{Fenouil gratiné, sauce tomate}
C. \textit{ Gratiné, sauce tomate}. — Apprêtez les finocchi comme dans la formule
précédente, mais en remplaçant la sauce à la crème par de la sauce tomate.

\sk

\index{Céleri au jus}
\index{Céleri gratiné, sauce tomate}
\index{Céleri gratiné, sauce à la crème}
On pourra préparer d'une façon analogue du céleri en branches.

\section*{\centering Artichauts farcis.}
\phantomsection
\addcontentsline{toc}{section}{ Artichauts farcis.}
\index{Artichauts farcis}

Pour six personnes prenez :

\footnotesize
\begin{longtable}{rrrp{18em}}
    600 & grammes & de & bouillon,                                                                        \\
    600 & grammes & de & vin blanc,                                                                       \\
    500 & grammes & de & tomates,                                                                         \\
    250 & grammes & de & champignons de couche,                                                           \\
    200 & grammes & de & blanc de poulet rôti,                                                            \\
    200 & grammes & de & couenne,                                                                         \\
    200 & grammes & de & carottes,                                                                        \\
    125 & grammes & de & jambon de Bayonne,                                                               \\
    125 & grammes & de & glace de viande,                                                                 \\
    125 & grammes & de & beurre,                                                                          \\
    125 & grammes & d' & oignons,                                                                         \\
     30 & grammes & de & fine champagne,                                                                  \\
     10 & grammes & d' & échalote,                                                                        \\
      5 & grammes & de & farine,                                                                          \\
      4 & grammes & de & persil,                                                                          \\
        &         &  6 & artichauts de Paris, de grosseur moyenne,                                        \\
        &         &  6 & bardes de lard, rondes, du diamètre des artichauts,                              \\
        &         &  1 & bouquet garni (persil, thym, laurier),                                           \\
        &         &    & sel, poivre, paprika\footnote{Il est impossible d'indiquer exactement
                                      les proportions de ces différents éléments, car
                                      tout dépend de l'assaisonnement du bouillon et de
                                      la glace de viande employés. Pour fixer les idées,
                                      j'indiquerai les chiffres suivants : 10 grammes de
                                      sel blanc, 6 décigrammes de poivre,
                                      2 décigrammes de paprika.}.                                         \\
\end{longtable}
\normalsize

Parez les artichauts, lavez-les, blanchissez-les suffisamment pour pouvoir
arracher les feuilles du milieu et enlever le foin au moyen d'une cuiller ;
laissez en place toutes les belles feuilles du pourtour.

Faites un hachis avec le blanc de poulet et le jambon ; assaisonnez-le.

Hachez séparément les échalotes, les champignons et le persil.

Pelez, épépinez et hachez fin les tomates.

\index{Appareil d'Uxel}
Préparez un appareil d'Uxel de la façon suivante : mettez {\ppp55\mmm} grammes
de beurre dans une casserole, chauffez fortement, saisissez dedans les
échalotes hachées, puis les champignons hachés, laissez cuire jusqu'à complète
évaporation de l'eau rendue par les champignons, saupoudrez ensuite de persil
haché et assaisonnez avec sel, poivre et paprika. Ajoutez à l'appareil les
tomates, le hachis de poulet et de jambon, la farine maniée avec {\ppp15\mmm}
grammes de beurre, amenez le tout à l'état d'une farce de consistance
moelleuse.

Emplissez les artichauts avec cette farce et couvrez chacun avec une barde de
lard.

Foncez une sauteuse avec la couenne, mettez dedans les carottes et les oignons
émincés, le bouquet garni, {\ppp55\mmm} grammes de beurre, laissez pincer,
c'est-à-dire laissez la couenne et les légumes s'attacher au fond de la
sauteuse, cela donnera de la couleur et du goût ; puis ajoutez les artichauts,
mouillez ensuite avec le vin, la fine champagne et le bouillon, dans lequel
vous aurez fait dissoudre la glace de viande ; les artichauts doivent baigner
dans le liquide jusqu'à mi-hauteur.

Faites braiser au four, à petit feu, pendant une heure.

Retirez alors les artichauts, dressez-les sur un plat tenu au chaud, concentrez
la sauce, dégraissez-la. Glacez les artichauts avec la sauce et servez.

Ces artichauts sont véritablement très fins.

\sk

\index{Fonds d'artichauts farcis}
On peut farcir de la même manière des fonds d’artichauts : le plat est
certainement moins joli à l'œil, mais il a l'avantage d’être plus facile
à manger.

\section*{\centering Artichauts froids, à l'huile.}
\phantomsection
\addcontentsline{toc}{section}{ Artichauts froids, à l'huile.}
\index{Artichauts froids, à l'huile}

Pour six personnes prenez :

\footnotesize
\begin{longtable}{rrrp{16em}}
    400 & grammes & d' & eau,                                                                             \\
    300 & grammes & d' & huile d'olive,                                                                   \\
     20 & grammes & de & sel,                                                                             \\
      1 & gramme  & de & poivre,                                                                          \\
        &         &  6 & artichauts jeunes, tendres et charnus,                                           \\
        &         &  3 & oignons,                                                                         \\
        &         &  1 & citron,                                                                          \\
        &         &  1 & bouquet garni.                                                                   \\
\end{longtable}
\normalsize

Lavez les artichauts, enlevez les premières feuilles dures et coupez le bout
des autres.

Mettez dans une casserole l'eau, l'huile, le jus du citron et un peu du zeste,
les oignons, le sel, le poivre, le bouquet garni ; faites bouillir. Plongez les
artichauts dans le liquide bouillant, couvrez et continuez la cuisson jusqu'à
évaporation complète de l'eau, phénomène qui se manifeste par un petit
sifflement.

Laissez refroidir.

\section*{\centering Fonds d'artichauts au jambon.}
\phantomsection
\addcontentsline{toc}{section}{ Fonds d'artichauts au jambon.}
\index{Artichauts au jambon}
\index{Fonds d'artichauts au jambon}

Faites cuire les artichauts dans de l'eau salée, détachez les feuilles,
enlevez-en la chair avec un couteau, hachez-la et garnissez-en les fonds.

Rangez les fonds d'artichauts dans une sauteuse foncée de beurre, placez sur
chacun une tranche mince de jambon de Bayonne de même diamètre, mouillez
avec du bon jus de viande et mettez au four pendant dix minutes.

\sk

Comme variante, garnissez les fonds avec une farce obtenue en mélangeant du
jambon haché, la chair des feuilles hachée, des champignons hachés et grillés,
le tout lié avec des jaunes d'œufs ; saupoudrez de mie de pain rassis tamisée,
mettez dessus un peu de beurre, faites dorer au four et servez avec une sauce
demi-glace ou de la sauce tomate.

\section*{\centering Fonds d'artichauts aux champignons.}
\phantomsection
\addcontentsline{toc}{section}{ Fonds d'artichauts aux champignons.}
\index{Fonds d'artichauts aux champignons}
\index{Artichauts aux champignons}

Cuisez des artichauts dans de l'eau salée, égouttez-les, enlevez les feuilles
et le foin ; parez les fonds. Retirez et réservez la chair des feuilles.

Faites dorer dans une sauteuse, avec du beurre et un peu d'huile d'olive,
d'abord des échalotes hachées, puis des champignons hachés également,
saupoudrez de quelques pincées de farine, tournez, mouillez avec une petite
quantité de fond de veau ; achevez la cuisson. Laissez un peu refroidir,
ajoutez ensuite du persil haché, la chair des feuilles, des jaunes d'œufs
frais ; goûtez et complétez l'assaisonnement avec sel et poivre, s'il est
nécessaire.

Disposez les fonds d'artichauts dans un plat légèrement beurré, allant au feu ;
garnissez-les avec le mélange, saupoudrez de mie de pain rassis tamisée, mettez
par-dessus quelques petits morceaux de beurre et faites gratiner au four.

Au dernier moment, masquez la préparation avec du fond, très concentré, de
veau et volaille corsé de glace de viande et servez.

\sk

On peut apprêter de même d'autres légumes.

\section*{\centering Fonds d’artichauts aux pointes d'asperges.}
\phantomsection
\addcontentsline{toc}{section}{ Fonds d’artichauts aux pointes d'asperges.}
\index{Fonds d’artichauts aux pointes d'asperges}
\index{Artichauts aux pointes d'asperges}

Prenez de beaux fonds d'artichauts cuits et enlevez dans chaque fond une
rondelle, de façon à les transformer en couronnes.

Parmi des asperges cuites comme il convient, choisissez les plus belles et
faites avec les pointes d'une longueur de {\ppp6\mmm} à {\ppp7\mmm} centimètres
des bottillons, du diamètre des rondelles enlevées, que vous insérerez debout
dans les couronnes.

Passez au tamis les rondelles de fonds d'artichauts et les pointes d’asperges
les moins belles, ajoutez des jaunes d'œufs, assaisonnez au goût, puis, avec le
mélange, garnissez en forme de cône l'intervalle compris entre le bord des
fonds et les bottillons d'asperges, de manière que l'extrémité seule des
pointes dépasse.

\medskip

Ce plat peut être servi chaud ou froid.

Lorsque le plat est servi chaud, on masque la préparation soit simplement
avec une sauce hollandaise, soit avec une sauce béchamel épaisse saupoudrée de
fromage de Gruyère râpé, le tout gratiné au four.

Froid, le plat peut être servi avec une sauce à l'huile et au vinaigre, où bien
encore une sauce mayonnaise au citron,

\medskip

Les fonds d'artichauts aux pointes d'asperges constituent un plat de légumes
peu banal, agréable à l'œil et d'un goût délicat.

\section*{\centering Cardons à la moelle.}
\phantomsection
\addcontentsline{toc}{section}{ Cardons à la moelle.}
\index{Cardons à la moelle}
\index{Bouchées à la moelle}

Prenez des cardons bien blancs et bien pleins ; enlevez les parties ligneuses
et faites blanchir les parties tendres dans de l'eau bouillante, salée et
acidulée. Rafraîchissez-les ensuite, frottez-les avec un linge pour en enlever
la peau, coupez-les en tronçons et passez-les au jus de citron.

Disposez les cardons ainsi apprêtés sur la grille d'une braisière et laissez-les
cuire à tout petit feu pendant une heure un quart à une heure et demie dans du
fond de veau, en évitant qu'ils prennent l'air, ce qui les noircirait.

Pendant leur cuisson, préparez une réduction de vin blanc avec échalotes
hachées, thym et laurier ; étendez-la avec la cuisson des cardons. Concentrez
cette sauce, passez-la à l'étamine, montez-la au beurre frais, relevez-la au
goût avec du jus de citron et liez-la avec des jaunes d'œufs.

Faites pocher de la moelle de bœuf dans du bouillon ; assaisonnez-la avec sel
et poivre.

Préparez des petites bouchées feuilletées ou des petites tartines de mie de pain
anglais dorées dans du beurre ; garnissez de moelle les bouchées ou les tartines,
masquez-les avec un peu de sauce et glacez-les rapidement au four.

Dressez les cardons en pyramide sur un plat ; versez dessus le reste de la
sauce, disposez autour les bouchées ou les tartines à la moelle ; servez
vivement.

\sk

On peut supprimer les bouchées ou les tartines et disposer dans un plat les
cardons coupés en tronçons revêtus chacun d'une tranche de moelle ; c'est plus
simple, mais c'est moins joli.

\section*{\centering Asperges, sauce mousseline.}
\phantomsection
\addcontentsline{toc}{section}{ Asperges, sauce mousseline.}
\index{Asperges, sauce mousseline}
\index{Asperges, sauce hollandaise ou sauce mousseline au jus d'orange sanguine}

Grattez les asperges, lavez-les, faites-les cuire dans de l'eau salée
bouillante de façon qu'elles conservent une certaine fermeté, égouttez-les et
dressez-les sur un plat garni d'une serviette.

Servez, en envoyant en même temps une saucière de sauce mousseline.

\sk

\index{Sauce mousseline}
\label{pg0759} \hypertarget{p0759}{}
La sauce mousseline est une variante de la sauce hollandaise. Voici comment on
la prépare : commencez par faire une sauce hollandaise, mais tenez-la plus
serrée que d'ordinaire en doublant le nombre des jaunes d'œufs qui entrent dans
sa composition ; incorporez-y, hors du feu, de la crème fouettée fraîche.

\medskip

Proportions pour six personnes :

\footnotesize
\begin{longtable}{rrrp{16em}}
    250 & grammes & de & beurre,                                                                          \\
     60 & grammes & de & crème fouettée,                                                                  \\
     30 & grammes & d' & eau froide,                                                                      \\
        &         &  8 & jaunes d'œnfs,                                                                   \\
        &         &    & sel.                                                                             \\
\end{longtable}
\normalsize

L'addition de crème fouettée donne à la sauce mousseline une consistance
particulièrement moelleuse.

\sk

On peut aussi servir les asperges avec d'autres sauces, notamment avec une
sauce hollandaise parfumée avec du jus d'orange sanguine et relevée avec un peu
de jus de citron.

\section*{\centering Asperges vertes à la crème.}
\phantomsection
\addcontentsline{toc}{section}{ Asperges vertes à la crème.}
\index{Asperges vertes à la crème}
\label{pg0760} \hypertarget{p0760}{}

Pour six personnes prenez :

\footnotesize
\begin{longtable}{rrrp{16em}}
  1 000 & grammes  & des & parties tendres de petites asperges vertes, appelées communément « balais »,   \\
    250 & grammes  & de  & crème,                                                                         \\
     60 & grammes  & de  & sel gris,                                                                      \\
     20 & grammes  & de  & beurre,                                                                        \\
     20 & grammes  & de  & farine,                                                                        \\
      4 & grammes  & de  & poivre fraîchement moulu,                                                      \\
        & 2 litres & d'  & eau.                                                                           \\
\end{longtable}
\normalsize

Coupez les asperges en morceaux de {\ppp2\mmm} à {\ppp3\mmm} centimètres de
longueur, faites-les cuire dans l’eau additionnée de {\ppp50\mmm} grammes de
sel gris jusqu'à ce qu'ils cèdent sous les doigts : dix minutes suffisent
généralement pour les morceaux de tiges, cinq minutes pour les pointes.

Faites blondir la farine dans le beurre, ajoutez la crème, le reste du sel, le
poivre, laissez cuire sur un feu doux, mettez ensuite les morceaux d'asperges,
continuez la cuisson pendant un instant et servez.

\section*{\centering Aubergines frites.}
\phantomsection
\addcontentsline{toc}{section}{ Aubergines frites.}
\index{Aubergines frites}

Pelez des aubergines, coupez-les en tranches d'épaisseur uniforme,
saupoudrez-les de sel gris, laissez-les dégorger pendant une heure,
assaisonnez-les avec du poivre.

On peut faire frire les aubergines dans du beurre ou dans de l'huile. Si vous
les faites frire dans du beurre, roulez-les au préalable dans de la farine ; si
vous les faites frire dans de l'huile, enduisez-les d'abord de blanc d'œuf et
passez-les ensuite dans de la mie de pain rassis tamisée.

Disposez dans un légumier les tranches d'aubergines frites, les unes sur les
autres, sans les presser, et servez,

\sk

\index{Courgettes frites}
On peut apprêter de même des courgettes.

\section*{\centering Aubergines et tomates gratinées.}
\phantomsection
\addcontentsline{toc}{section}{ Aubergines et tomates gratinées.}
\index{Aubergines et tomates gratinées}

Pour quatre personnes prenez :

\footnotesize
\begin{longtable}{rrrp{16em}}
  1 000 & grammes & de & tomates,                                                                         \\
    500 & grammes & d' & aubergines,                                                                      \\
    125 & grammes & de & fromage de Gruyère râpé,                                                         \\
    100 & grammes & de & beurre,                                                                          \\
     20 & grammes & de & chapelure,                                                                       \\
     20 & grammes & de & sel blanc,                                                                       \\
      2 & grammes & de & poivre fraîchement moulu.                                                        \\
\end{longtable}
\normalsize

Pelez les aubergines et les tomates, coupez-les en tranches, retirez-en les graines.

Foncez avec {\ppp50\mmm} grammes de beurre un plat creux allant au feu,
disposez dedans des couches alternées d'aubergine et de tomate sur lesquelles
vous mettrez, au fur et à mesure, du fromage râpé, du sel et du poivre ;
terminez par du fromage, saupoudrez avec la chapelure, mettez par-dessus le
reste du beurre, coupé en petits morceaux, et faites cuire au four doux pendant
une heure.

L'ensemble, cuit ainsi et gratiné, constitue un excellent plat pouvant être
servi dans les repas sans cérémonie,

\section*{\centering Moussaka.}
\phantomsection
\addcontentsline{toc}{section}{ Moussaka.}
\index{Moussaka}

La moussaka est un plat oriental d'aubergines.

Pour quatre personnes prenez :

\footnotesize
\begin{longtable}{rrrrrp{18em}}
  & \hspace{2em}  &  400 & grammes & d' & aubergines,                                                     \\
  & \hspace{2em}  &  250 & grammes & de & tomates,                                                        \\
  & \hspace{2em}  &  250 & grammes & de & tranche de bœuf,                                                \\
  & \hspace{2em}  &  200 & grammes & de & beurre,                                                         \\
  & \hspace{2em}  &   60 & grammes & de & graisse de rognon de veau,                                      \\
  & \hspace{2em}  &   60 & grammes & de & bouillon,                                                       \\
  & \hspace{2em}  &   30 & grammes & d' & oignon,                                                         \\
  & \hspace{2em}  &    5 & grammes & de & sel blanc,                                                      \\
  & \multicolumn{3}{r}{2 décigrammes} & de & poivre fraîchement moulu,                                    \\
  & \multicolumn{3}{r}{2 décigrammes} & de & paprika,                                                     \\
  & \hspace{2em}  &      &         &    & farine.                                                         \\
  & \hspace{2em}  &      &         &    & sel gris.                                                       \\
\end{longtable}
\normalsize

Pelez les aubergines, coupez-les en tranches de l'épaisseur d'une pièce de cinq
francs en argent, saupoudrez-les de sel gris, laissez-les dégorger pendant une
heure.

Hachez ensemble le bœuf et la graisse de rognon de veau.

Hachez l'oignon.

Menez de front les quatre opérations suivantes.

Mettez les tomates coupées en morceaux dans une casserole, laissez-les fondre à
petit feu, passez le jus au tamis.

Faites roussir l'oignon dans {\ppp50\mmm} grammes de beurre, mouillez avec le
bouillon et chauffez à siccité.

Faites revenir dans {\ppp25\mmm} grammes de beurre le hachis de bœuf et de
graisse de veau, assaisonnez avec le sel blanc, le poivre et le paprika, mettez
au four, laissez mijoter pendant une dizaine de minutes, puis ajoutez le hachis
d'oignon ; mélangez bien.

Lavez les tranches d'aubergines à plusieurs reprises dans de l’eau fraîche,
roulez-les dans de la farine et faites-les dorer dans le reste du beurre.

Disposez dans un plat creux allant au feu des couches alternées d'aubergine et
de hachis, mouillez avec le jus de tomates, mettez au four et continuez la cuisson,
à feu doux, jusqu'à ce que le liquide soit évaporé, ce qui demande une heure au
moins.

Le dessus doit être gratiné, l'intérieur moelleux.

\section*{\centering Aubergines aux tomates, froides.}
\phantomsection
\addcontentsline{toc}{section}{ Aubergines aux tomates, froides.}
\index{Aubergines aux tomates, froides}

Pour six personnes prenez :

\footnotesize
\begin{longtable}{rrrp{16em}}
    750 & grammes & de & tomates,                                                                         \\
    200 & grammes & de & fond de veau,                                                                    \\
     50 & grammes & de & beurre,                                                                          \\
        &         &  6 & aubergines moyennes,                                                             \\
        &         &  4 & gousses d'ail moyennes,                                                          \\
        &         &    & huile d'olive,                                                                   \\
        &         &    & persil,                                                                          \\
        &         &    & sel et poivre.                                                                   \\
\end{longtable}
\normalsize

Épluchez partiellement les aubergines, en laissant des lambeaux de peau.

Emincez l'ail en languettes, réservez-en un huitième et piquez les aubergines
avec le reste.

Faites cuire les tomates avec le beurre, l'ail réservé et un peu de persil ;
passez le jus et tenez-le au chaud,

Faites revenir les aubergines dans de l'huile d'olive ; lorsqu'elles seront
bien dorées de tous les côtés, égouttez-les, mettez-les dans un plat allant au
feu, assaisonnez-les avec sel et poivre, mouillez avec le fond de veau et le
jus des tomates, couvrez et laissez cuire pendant trois quarts d'heure.

Disposez les aubergines sur un plat, masquez-les avec la cuisson ; laissez
refroidir.

\section*{\centering Aubergines farcies, froides.}
\phantomsection
\addcontentsline{toc}{section}{ Aubergines farcies, froides.}
\index{Aubergines farcies, froides}

Pour cinq personnes prenez :

\footnotesize
\begin{longtable}{rrrp{16em}}
     75 & grammes & d' & huile d'olive,                                                                   \\
        &         &  5 & aubergines ayant peu de pépins,                                                  \\
        &         &  4 & gros oignons,                                                                    \\
        &         &  3 & belles tomates,                                                                  \\
        &         &  2 & gousses d'ail,                                                                   \\
        &         &    & sel et poivre.                                                                   \\
\end{longtable}
\normalsize

Enlevez aux aubergines trois bandes de pelure sur toute leur longueur ; ce qui
en restera suffira pour les empêcher de se défaire. Coupez les queues sans les
détacher complètement et pratiquez des incisions profondes, longitudinales.

Faites chauffer l'huile dans une poêle ; lorsqu'elle commencera à chanter,
jetez dedans les oignons et l'ail hachés ; laissez prendre couleur, puis
ajoutez les tomates pelées, épépinées et coupées en tranches. Continuez la
cuisson pendant deux minutes, salez, poivrez, mélangez hien et insérez cette
masse dans les incisions des aubergines.

Disposez les aubergines ainsi farcies dans une sauteuse, couvrez d'eau, faites
cuire jusqu'à ce que toute l'eau soit évaporée ; dressez les aubergines sur un
plat ; laissez-les refroidir.

\medskip

Ce plat, qui appartient à la cuisine byzantine, est connu à Constantinople sous
la dénomination pittoresque de \textit{Imam bayeldi}, ce qui veut dire
littéralement : l'Iman en a roté ! Était-ce réellement pour témoigner sa
satisfaction d'une manière à la fois délicate et expressive, comme le
soutiennent les fanatiques du plat, ou le phénomène ne s'est il pas plutôt
produit à cause d'une proportion excessive d'ail, d'oignon et d'huile ? Ce
point d'histoire restera probablement longtemps obscur.

Quoi qu'il en soit, le plat présente incontestablement un caractère original,
et à ce titre, il mérite d'être mentionné.

\section*{\centering Courgettes farcies, au maigre.}
\phantomsection
\addcontentsline{toc}{section}{ Courgettes farcies, au maigre.}
\index{Courgettes farcies, au maigre}

Pour six personnes prenez :

\footnotesize
\begin{longtable}{rrrp{16em}}
    500 & grammes & de & champignons de couche,                                                           \\
    250 & grammes & de & lait,                                                                            \\
    125 & grammes & de & beurre,                                                                          \\
     75 & grammes & de & carotte,                                                                         \\
     50 & grammes & d' & oignon haché,                                                                    \\
     45 & grammes & de & farine,                                                                          \\
     35 & grammes & de & navet,                                                                           \\
        &         &  6 & courgettes moyennes,                                                             \\
        &         &  4 & jaunes d'œufs frais,                                                             \\
        &         &    & mie de pain rassis tamisée,                                                      \\
        &         &    & bouquet de persil, thym et laurier,                                              \\
        &         &    & muscade,                                                                         \\
        &         &    & sel et poivre.                                                                   \\
\end{longtable}
\normalsize

Épluchez la carotte et le navet, coupez-les en petits morceaux, faites-les
cuire avec le bouquet garni dans le lait de façon à le parfumer ; passez-le.

Pelez les champignons, hachez-les et cuisez-les dans le lait.

Pelez les courgettes, coupez-les dans le sens de leur longueur en deux parties
symétriques ; creusez-les ; réservez la chair que vous hacherez. Blanchissez
les demi-courgettes dans de l’eau salée et poivrée légèrement.

Faites dorer l'oignon haché dans {\ppp90\mmm} grammes de beurre, saupoudrez
avec la farine à laquelle vous laisserez prendre couleur, ajoutez les
champignons et leur cuisson, du sel, du poivre et de la muscade au goût ;
concentrez la sauce. Incorporez-y ensuite la chair des courgettes et les jaunes
d'œufs ; mélangez bien, goûtez et complétez l'assaisonnement s'il y a lieu.
Cela constituera la farce.

Dressez les courgettes sur un plat légèrement beurré, allant au feu,
garnissez-en les creux avec la farce, saupoudrez de mie de pain rassis tamisée,
parsemez le dessus avec le reste du beurre coupé en petits morceaux et faites
gratiner au four.

\sk

Comme variantes, on pourra mettre dans la farce moitié champignons et moitié
petits pois frais.

On pourra, cela va sans dire. remplacer les champignons de couche par d'autres
espèces : cèpes, mousserons, pratelles, lépiotes, lactaires, etc.

\sk

\index{Aubergines verts farcis, au maigre}
\index{Artichauts farcis, au maigre}
\index{Piments verts farcis, au maigre}
\index{Pommes de terre farcies, au maigre}
On préparera de même d'autres légumes : aubergines, piments verts, fonds
d'artichauts, pommes de terre, etc.

\section*{\centering Courgettes au fromage, frites.}
\phantomsection
\addcontentsline{toc}{section}{ Courgettes au fromage, frites.}
\index{Courgettes au fromage, frites}

Prenez de préférence de jeunes courgettes d'Italie, à chair blanche. Pelez-les,
coupez-les en tranches, faites-les dégorger dans du sel gris, épongez-les
ensuite, passez-les dans de la farine et faites-les frire dans de l'huile.

Réunissez ces tranches deux par deux en les accolant l'une à l'autre avec un
mélange de jaunes d'œufs et de fromage de Gruyère râpé.

Au moment de servir, passez ces groupes de tranches dans de la mie de pain
rassis tamisée ou dans de la pâte à frire, plongez-les dans de l'huile bouillante,
retirez-les avec une écumoire et dressez-les sur un plat garni de persil frit.

\sk

\index{Aubergines au fromage, frites}
On peut préparer de même des aubergines.

\section*{\centering Courgettes frites sauce tomate, au gratin.}
\phantomsection
\addcontentsline{toc}{section}{ Courgettes frites sauce tomate, au gratin.}
\index{Courgettes frites sauce tomate, au gratin}

Pour quatre personnes prenez :

\footnotesize
\begin{longtable}{rrrp{16em}}
  1 000 & grammes & de & petites courgettes,                                                              \\
    200 & grammes & de & bonne sauce tomate serrée,                                                       \\
    125 & grammes & de & fromage de Gruyère râpé,                                                         \\
    100 & grammes & de & beurre,                                                                          \\
     30 & grammes & de & mie de pain rassis tamisée,                                                      \\
     10 & grammes & de & sel,                                                                             \\
      2 & grammes & de & poivre,                                                                          \\
        &         &    & huile d'olive.                                                                   \\
\end{longtable}
\normalsize

Pelez les courgettes, laissez-les entières ou coupez-les en morceaux suivant
leur grosseur ; faites-les frire rapidement dans de l'huile d'olive bien
chaude ; égouttez-les.

Foncez un plat creux allant au feu avec {\ppp50\mmm} grammes de beurre,
disposez dedans des couches de courgettes séparées par de la sauce tomate, sur
lesquelles vous mettrez, au fur et à mesure, du fromage râpé, du sel et du
poivre ; terminez par du fromage, saupoudrez avec la mie de pain, ajoutez le
reste du beurre coupé en petits morceaux et poussez au four pendant {\ppp5\mmm}
à {\ppp10\mmm} minutes pour gratiner.

\sk

On peut apprêter de méme des aubergines,

\section*{\centering Tomates farcies de champignons grillés.}
\phantomsection
\addcontentsline{toc}{section}{ Tomates farcies de champignons grillés.}
\index{Tomates farcies de champignons grillés}
\label{pg0766} \hypertarget{p0766}{}

La tomate ou pomme d'amour, délicieux et poétique légume-fruit qui joue
surtout, en cuisine, le rôle de garniture, mérite de fixer l'attention des
amateurs. Fine et jolie à travailler, elle a véritablement droit à des égards ;
aussi doit-elle bien souffrir la pauvrette de se sentir farcir par des Barbares
avec du hachis de bœuf bouilli et de la chair à saucisses. Elle vaut réellement
mieux que cela.

Pour lui conserver son parfum, pour ne pas profaner son arome délicat, il
convient de ne lui associer que des farces distinguées, de bonne compagnie, et
l'une des meilleures, à mon avis, est une simple farce aux champignons grillés.

Voici comment il faut opérer.

\medskip

Pour six personnes prenez :

\footnotesize
\begin{longtable}{rrrp{16em}}
    500 & grammes & de & champignons de couche,                                                           \\
    180 & grammes & de & beurre,                                                                          \\
    150 & grammes & de & sauce tomate épaisse,                                                            \\
     10 & grammes & de & sel blanc,                                                                       \\
      1 & gramme  & de & poivre fraîchement moulu,                                                        \\
        &         &  6 & belles tomates.                                                                  \\
        &         &  1 & jaune d'œuf frais,                                                               \\
        &         &    & jus de citron,                                                                   \\
        &         &    & mie de pain rassis tamisée.                                                      \\
\end{longtable}
\normalsize

Ébouillantez les tomates, pelez-les, retirez-en les graines.

Pelez les champignons, coupez-les en petits morceaux, passez-les dans du jus de
citron qui les imprégnera et les empêchera de noircir ; faites-les cuire dans
{\ppp120\mmm} grammes de beurre jusqu'à siccité, de manière à les griller ;
arrêtez l'opération dès que vous sentirez se dégager des champignons le parfum
de l'huile aromatique qui les caractérise. Retirez-les alors du feu, ajoutez la
sauce tomate, le jaune d'œuf, le sel et le poivre, mélangez intimement.

Farcissez les tomates avec ce mélange, saupoudrez la surface de mie de pain
rassis tamisée, mettez dessus du beurre, à raison de {\ppp10\mmm} grammes par
tomate faites dorer au four.

Préparées de la sorte, les tomates farcies sont délicieuses, légères et
parfumées ; elles font la joie des gourmands.

\sk

Comme variante, on pourra servir ces tomates farcies masquées avec du fond de
veau et volaille tomaté, très concentré.

\section*{\centering Tomates en surprise.}
\phantomsection
\addcontentsline{toc}{section}{ Tomates en surprise.}
\index{Tomates en surprise}

Pour six personnes prenez :

\index{Enveloppe paur tomates}
1° pour l'enveloppe :

\medskip

\footnotesize
\begin{longtable}{rrrp{16em}}
    650 & grammes & de & gelée de veau et volaille,                                                       \\
    350 & grammes & de & purée de tomates,                                                                \\
     35 & grammes & de & vinaigre,                                                                        \\
        &         &    & sel et poivre ;                                                                  \\
\end{longtable}
\normalsize

\index{Garniture pour tomates}
2° pour la garniture :

\footnotesize
\begin{longtable}{rrrp{16em}}
    375 & grammes & de & blanc de volaille,                                                               \\
    375 & grammes & de & langue à l'écarlate,                                                             \\
    150 & grammes & de & truffes cuites au naturel,                                                       \\
    125 & grammes & de & jambon d'York ou de Bayonne,                                                     \\
        &         &    & purée de tomates épaisse,                                                        \\
        &         &    & cayenne,                                                                         \\
        &         &    & sel et poivre ;                                                                  \\
\end{longtable}
\normalsize

3° pour la décoration :

\footnotesize
\begin{longtable}{rrrp{16em}}
    650 & grammes & de & gelée de veau et volaille,                                                       \kill
        &         &    & gelée de veau et volaille,                                                       \\
        &         &    & beurre,                                                                          \\
        &         &    & vert végétal.                                                                    \\
\end{longtable}
\normalsize

Préparez un salpicon avec le blanc de volaille, la langue à l'écarlate, le
jambon et les truffes coupés en petits morceaux, assaisonnez avec cayenne, sel
et poivre au goût.

Mettez dans une casserole la gelée de veau et volaille, la purée de tomates et
le vinaigre figurant au premier paragraphe, salez, poivrez, mélangez,
concentrez. Passez le mélange au chinois, laissez-le refroidir un peu et
coulez-le dans {\ppp12\mmm} moules en forme de tomates ; emplissez-les bien,
mettez-les à la glace.

Lorsque l'extérieur des pseudo-tomates sera solidifié, creusez l'intérieur,
réservez les parties enlevées, emplissez incomplètement les vides avec le
salpicon, finissez le remplissage avec de la purée de tomates épaisse, coulez
dessus un peu du mélange réservé qui a servi à préparer les enveloppes. Fermez
les moules ; faites prendre sur glace.

Foncez un plat avec de la gelée de veau et volaille hachée, démoulez dessus les
tomates en surprise, décorez avec des feuilles simulées de tomate faites avec du
beurre coloré par du vert végétal. Servez.

C'est un excellent hors-d'œuvre.

\section*{\centering Conserve de purée de tomates.}
\phantomsection
\addcontentsline{toc}{section}{ Conserve de purée de tomates.}
\index{Conserve de purée de tomates}
\label{pg0768} \hypertarget{p0768}{}

Pendant la saison des tomates, rien ne vaut le fruit frais pour toutes les
préparations qui nécessitent son emploi ; mais, en hiver, on n'a guère à sa
disposition que les conserves du commerce, rarement fabriquées avec tout le
soin voulu.

Voici une formule pratique permettant de faire chez soi, au moment de la
saison, une purée aromatisée parfaite, se conservant facilement pendant tout
l'hiver, et de beaucoup supérieure à la plupart des conserves commerciales.

Pour préparer environ quatre litres de purée, prenez {\ppp10\mmm} kilogrammes
de belles tomates bien mûres, lavez-les, séchez-les, puis mettez-les sur le feu
dans une bassine en cuivre, écrasez-les à la main, jusqu'à ce que toute la
pulpe soit détachée des peaux et que la température ne vous permette plus de
les triturer. Faites bouillir pendant une heure en remuant avec une cuiller,
passez ensuite au travers d'une passoire fine afin d'éliminer les graines et
laissez reposer jusqu'au lendemain.

Enlevez alors avec précaution l'eau qui surnage, ajoutez {\ppp1\mmm} gramme
d'acide salicylique par litre de jus (cela ne donne absolument aucun goût et
suffit pour éviter toute fermentation ultérieure), mélangez intimement.

Mettez en flacons avec un doigt d'huile d'olive par-dessus, qui constituera une
fermeture hermétique, bouchez comme vous voudrez, simplement pour éviter la
poussière, et conservez les flacons debout, dans un endroit frais.

\section*{\centering Carottes.}
\phantomsection
\addcontentsline{toc}{section}{ Carottes.}
\index{Carottes}
\index{Carottes (Différentes manières d'accommoder les)}
\index{Carottes en ragoût}
\index{Carottes à la béchamel}
\index{Carottes à la crème}
\index{Carottes à la maître d'hôtel}
\index{Carottes à la sauce poulette}
\index{Carottes braisées au madère}
\index{Carottes nouvelles à la Vichy}
\index{Carottes à la sauce hollandaise }
\index{Carottes à la sauce veloutée}
\index{Carottes en purée}
\index{Carottes en pudding}
\index{Carottes au lard}

Les procédés de préparation des carottes sont les suivants :

au jus ;

glacées ;

à la maitre d'hôtel ;

au lait ou à la crème ;

au beurre (à la Vichy) ;

au lard ;

frites ;

en ragoût, seules ou accompagnées d'autres légumes (petits pois, champignons,
par exemple), avec sauces Béchamel, poulette, veloutée, hollandaise, etc.

On les sert aussi en purée, en pudding.

On en fait des potages.

\section*{\centering Carottes à la maître-d'hôtel.}
\phantomsection
\addcontentsline{toc}{section}{ Carottes à la maître-d'hôtel.}
\index{Carottes à la maître-d'hôtel}

Épluchez ou tournez des petites carottes nouvelles, mettez-les dans une
sauteuse avec du beurre, du sel et du sucre en poudre, au goût ; mouillez avec
du consommé de volaille ; laissez cuire.

Réduisez le mouillement à glace ; éloignez la casserole du feu, ajoutez du
velouté de volaille, du beurre manié avec du persil haché, relevez avec du jus
de citron ou un peu de vinaigre, chauffez pendant quelques minutes et servez.

\section*{\centering Carottes à la crème.}
\phantomsection
\addcontentsline{toc}{section}{ Carottes à la crème.}
\index{Carottes à la crème}

Pour quatre personnes prenez :

\footnotesize
\begin{longtable}{rrrp{16em}}
  1 000 & grammes & de & carottes nouvelles,                                                              \\
    400 & grammes & de & crème épaisse,                                                                   \\
     20 & grammes & de & sel,                                                                             \\
     10 & grammes & de & sucre en poudre,                                                                 \\
        &         &    & eau.                                                                             \\
\end{longtable}
\normalsize

Mettez dans une casserole une quantité suffisante d'eau pour que les carottes
puissent baigner dedans, ajoutez le sel et le sucre ; faites bouillir ; plongez
dans l'eau bouillante les carottes pelées ou tournées et laissez-les cuire
pendant {\ppp30\mmm} à {\ppp40\mmm} minutes, suivant leur grosseur.

Concentrez le liquide de cuisson ; ajoutez la crème ; donnez quelques bouillons
et servez.

\medskip

Les carottes à la crème sont particulièrement recommandables avec l'agneau,

\section*{\centering Carottes braisées au madère.}
\phantomsection
\addcontentsline{toc}{section}{ Carottes braisées au madère.}
\index{Carottes braisées au madère}

Prenez de belles carottes Crécy, épluchez-les, coupez-les en morceaux ou
tournez-les.

Faites-les dorer légèrement à la casserole, dans du beurre, avec de l'oignon au
goût.

En même temps. faites revenir du lard de poitrine après l'avoir blanchi,
mettez-le avec les carottes, mouillez avec du bouillon ou du fond de veau,
ajoutez un bouquet garni, un peu de quatre épices, du poivre ; laissez cuire.

Vingt minutes avant la fin, corsez avec du madère et réduisez suffisamment la
cuisson. Goûtez et complétez l'assaisonnement s'il y a lieu.

Dressez les carottes sur un plat, décorez avec le lard et masquez avec la sauce
passée,

\section*{\centering Carottes nouvelles à la Vichy.}
\phantomsection
\addcontentsline{toc}{section}{ Carottes nouvelles à la Vichy.}
\index{Carottes nouvelles à la Vichy}
\label{pg0770} \hypertarget{p0770}{}

Pour quatre personnes prenez :

\footnotesize
\begin{longtable}{rrrp{16em}}
  1 000 & grammes & de & carottes nouvelles, fraîchement cueillies,                                       \\
    250 & grammes & de & beurre,                                                                          \\
     60 & grammes & de & fine champagne,                                                                  \\
     10 & grammes & de & sucre en poudre,                                                                 \\
     10 & grammes & de & sel.                                                                             \\
\end{longtable}
\normalsize

Épluchez les carottes, coupez-les, avec un couteau à légumes, en rondelles très
minces.

Faites fondre le beurre dans une casserole sans le laisser roussir, ajoutez le
sucre et le sel, mélangez bien, puis saisissez dedans les carottes ; mouillez ensuite
avec la fine champagne, couvrez la casserole et laissez cuire au four, à feu doux,
sans remuer, pendant une heure. Surveillez la cuisson. Au bout de ce temps, les
carottes doivent avoir absorbé presque tout le beurre et les rondelles doivent
être entières, non racornies.

\medskip

Les carottes à la Vichy peuvent être servies seules. Elles accompagnent aussi
tres bien le veau braisé.

\section*{\centering Navets.}
\phantomsection
\addcontentsline{toc}{section}{ Navets.}
\index{Navets}

On prépare les navets de différentes manières :
au jus ;
glacés ;
à la crème ;
farcis ;
en purée ;
en potages ; etc.

\section*{\centering Navets glacés.}
\phantomsection
\addcontentsline{toc}{section}{ Navets glacés.}
\index{Navets glacés}

Tournez des navets nouveaux ; faites-les blanchir dans de l'eau salée ;
séchez-les.

Dorez-les dans du beurre ; saupoudrez-les de sucre ; mouillez à mi-hauteur avec
un fond corsé de viande ou de volaille ; laissez tomber le mouillement à glace.

Dressez les navets sur un plat ; déglacez avec du jus, très réduit et lié, de
viande ou de volaille ; masquez-en les navets et servez.

\section*{\centering Navets farcis.}
\phantomsection
\addcontentsline{toc}{section}{ Navets farcis.}
\index{Navets farcis}

Prenez de beaux navets ronds nouveaux ; pelez-les, creusez-les avec une curette ;
mettez-les dans un plat beurré ; mouillez avec un bon jus, corsé, de viande ou de
volaille et faites cuire au four.

En même temps, préparez une farce avec de la semoule, cuite dans du jus, et du
parmesan râpé.

Garnissez les navets avec cette farce ; couronnez-les avec un peu de beurre
frais ; remettez-les au four pendant quelques minutes encore et servez.

\sk

On peut aussi farcir des navets avec des purées de légumes : purées de pommes
de terre et de cerfeuil bulbeux, d'épinards, de pointes d'asperges, par exemple.

\section*{\centering Céleri-rave.}
\phantomsection
\addcontentsline{toc}{section}{ Céleri-rave.}

On peut apprêter le céleri-rave de différentes façons.

\medskip

\index{Céleri-rave sauté}
1° \textit{Sauté}. — Pour quatre personnes prenez un céleri-rave pesant
{\ppp500\mmm} grammes environ ; épluchez-le, coupez-le en tranches minces de
{\ppp3\mmm} à {\ppp4\mmm} millimètres, que vous ferez blanchir pendant cinq
minutes dans de l'eau salée bouillante, puis faites-les sauter dans une poêle
avec 100 grammes de beurre, pendant une demi-heure, de manière à les bien
dorer. Au moment de servir, saupoudrez de feuilles vertes de céleri hachées.

Le céleri-rave sauté peut être servi tel quel ou accompagner une grillade.

\medskip

\index{Céleri-rave frit}
\index{Crosnes frits}
2° \textit{Frit}. — Coupez, comme précédemment, le céleri épluché ; mettez les
tranches, pendant un quart d'heure, dans de l'eau froide vinaigrée,
égouttez-les ensuite, puis enrobez-les une à une dans de la pâte à frire
légère, et faites-les cuire dans une abondante friture de graisse de rognon de
veau.

\medskip

\index{Céleri-rave au jus}
\index{Fonds d'artichauts au jus}
3° \textit{Au jus}. — Coupez le céleri épluché en tranches épaisses de
{\ppp6\mmm} millimètres, que vous plongerez dans de l'eau salée bouillante ;
laissez donner deux bouillons, retirez les tranches, égouttez-les et achevez
leur cuisson dans du bon jus, en les faisant mijoter tout doucement pendant une
heure et demie.

\medskip

\index{Céleri-rave à la béchamel}
\index{Crosnes à la béchamel}
\index{Fonds d'artichauts à la béchamel}
4° \textit{À la béchamel}. — Coupez le céleri comme précédemment ; faites cuire
les tranches pendant six minutes dans de l’eau salée bouillante, égouttez-les,
puis mélangez-les avec de la sauce Béchamel grasse, chaude, saupoudrez de
feuilles vertes de céleri hachées et servez dans un légumier.

\medskip

\index{Céleri-rave au gratin}
5° \textit{Au gratin}. — Remplacez dans la formule précédente la béchamel par
une mornay, mettez la préparation dans un plat allant au feu, saupoudrez avec
un mélange de mie de pain rassis tamisée et de fromage de Gruyère râpé, en
volumes égaux ; faites gratiner au four.

\medskip

\index{Céleri-rave en salade}
6° \textit{En salade}. — Émincez en julienne un céleri-rave cru et épluché,
mettez-le à mariner pendant une heure dans du vin blanc de Sauternes, puis
assaisonnez-le au goût avec huile, vinaigre, sel, poivre, moutarde et feuilles
vertes de céleri hachées.

Préparé ainsi, le céleri-rave est servi en hors-d'œuvre.

\sk

Toutes ces préparations sont applicables aux salsifis, aux fonds d'artichauts et
aux crosnes.

\section*{\centering Potée fermière.}
\phantomsection
\addcontentsline{toc}{section}{ Potée fermière.}
\index{Potée fermière}
\label{pg0772} \hypertarget{p0772}{}

La potée fermière est une macédoine de légumes cuits sans eau, à l'étouffée.
Lorsqu'elle est faite avec des légumes nouveaux fraîchement cueillis et du
beurre fin, c'est un plat exquis. Voici comment on la prépare.

Prenez une marmite en porcelaine allant au feu et munie d'un couvercle, mettez
dedans du beurre et des oignons coupés en rondelles, laissez dorer, ajoutez
ensuite des cœurs de laitues, des petits pois, des haricots verts, des
carottes, des navets, des pommes de terre, des pointes d'asperges, etc., en
proportions indéterminées, à votre goût, puis du beurre en morceaux dans la
proportion de {\ppp125\mmm} grammes de beurre pour {\ppp500\mmm} grammes de
légumes ; couvrez la marmite et laissez cuire au four, à petit feu pendant deux
heures. Un quart d'heure avant la fin, salez, poivrez et achevez la cuisson.

Il est difficile d'avoir un meilleur entremets de légumes,

\section*{\centering Macédoine de légumes à la crème.}
\phantomsection
\addcontentsline{toc}{section}{ Macédoine de légumes à la crème.}
\index{Macédoine de légumes à la crème}
\label{pg0773} \hypertarget{p0773}{}

Pour quatre personnes prenez :

\footnotesize
\begin{longtable}{rrrp{16em}}
    500 & grammes & de & lait,                                                                            \\
    500 & grammes & de & petits pois en cosses,                                                           \\
    350 & grammes & de & carottes,                                                                        \\
    350 & grammes & de & navets,                                                                          \\
    300 & grammes & de & pommes de terre,                                                                 \\
    250 & grammes & de & haricots verts,                                                                  \\
    250 & grammes & de & crème,                                                                           \\
     50 & grammes & de & beurre,                                                                          \\
     10 & grammes & de & farine,                                                                          \\
        &         &  1 & bottillon de pointes d'asperges,                                                 \\
        &         &    & sel et poivre.                                                                   \\
\end{longtable}
\normalsize

Écossez les pois ; épluchez les autres légumes.

Coupez les haricots verts et les pointes d'asperges en morceaux ; émincez les
carottes et les navets en julienne.

Faites cuire les haricots verts et les pointes d'asperges dans de l'eau salée ;
les carottes, les navets et les petits pois dans le lait ; les pommes de terre
à la vapeur.

Lorsque les pommes de terre sont cuites, coupez-les en dés.

Faites cuire la farine dans {\ppp25\mmm} grammes de beurre sans lui laisser
prendre couleur ; mouillez avec ce qui reste du lait de cuisson ; ajoutez les
légumes et la crème ; assaisonnez avec sel et poivre ; chauffez sans laisser
bouillir ; enfin incorporez le reste du beurre et servez.

\sk

On peut, il va sans dire, faire entrer dans la préparation d'autres légumes que
ceux indiqués ci-dessus, par exemple du chou. du chou-fleur, des fonds
d'artichauts, etc.

\medskip

Toutes ces macédoines, excellentes comme entremets de légumes, peuvent servir
de garniture pour viandes grillées ou sautées.

\section*{\centering Piments verts farcis, braisés.}
\phantomsection
\addcontentsline{toc}{section}{ Piments verts farcis, braisés.}
\index{Piments verts farcis, braisés}

Pour six personnes prenez :

\footnotesize
\begin{longtable}{rrrp{16em}}
    450 & grammes & de & bouillon,                                                                        \\
    300 & grammes & de & tomates,                                                                         \\
    150 & grammes & de & blanc de poulet rôti,                                                            \\
    125 & grammes & de & champignons,                                                                     \\
    100 & grammes & de & riz,                                                                             \\
    100 & grammes & de & jus de viande,                                                                   \\
     75 & grammes & de & jambon gras, fumé,                                                               \\
     35 & grammes & de & gruyère râpé,                                                                    \\
     25 & grammes & de & beurre,                                                                          \\
        &         &  6 & gros piments verts,                                                              \\
        &         &  2 & jaunes d'œufs,                                                                   \\
        &         &    & jus de citron,                                                                   \\
        &         &    & paprika,                                                                         \\
        &         &    & sel et poivre.
\end{longtable}
\normalsize

Faites cuire le riz dans le bouillon jusqu'à ce qu'il ait absorbé tout le
liquide, mais en conservant les grains entiers, non crevés ; incorporez-y le
gruyère râpé.

Hachez le blanc de poulet.

Pelez les champignons, passez-les dans du jus de citron, émincez-les, faites-les
cuire dans le beurre.

Videz les piments au moyen d'un emporte-pièce, en les laissant entiers,
blanchissez-les pendant deux minutes dans de l'eau salée bouillante.

Mélangez ensemble riz, poulet haché, champignons ; liez avec les jaunes d'œufs,
assaisonnez avec paprika, sel, poivre, goûtez et farcissez les piments avec ce
mélange.

Foncez une casserole avec le jambon fumé, disposez dedans les piments, mouillez
avec le jus de viande, ajoutez les tomates que vous aurez écrasées ; faites
braiser pendant vingt minutes. Concentrez la sauce.

Servez les piments masqués avec la sauce passée.

\sk

\index{Aubergines farcies, braisées}
\index{Courgettes farcies, braisées}
On peut préparer de même des paprikas verts, des aubergines et des courgettes.

\section*{\centering Piments verts farcis, gratinés.}
\phantomsection
\addcontentsline{toc}{section}{ Piments verts farcis, gratinés.}
\index{Piments verts farcis, gratinés}

Pour six personnes prenez :

\footnotesize
\begin{longtable}{rrrrp{16em}}
  &     100 & grammes & de & veau maigre, cru, paré et haché,                                             \\
  &     100 & grammes & de & jambon de Bayonne, paré et haché,                                            \\
  &      75 & grammes & de & beurre,                                                                      \\
  &      75 & grammes & de & bon jus de viande,                                                           \\
  &      50 & grammes & de & parmesan râpé,                                                               \\
  &      30 & grammes & de & vin blanc,                                                                   \\
  &       2 & grammes & de & persil haché,                                                                \\
  & \multicolumn{2}{r}{1 décigramme} & de & muscade râpée,                                                \\
  &         &         &  6 & gros piments verts,                                                          \\
  &         &         &  2 & jaunes d'œufs durs,                                                          \\
  &         &         &    & mie de pain rassis tamisée,                                                  \\
  &         &         &    & sel et poivre.                                                               \\
\end{longtable}
\normalsize

Coupez les piments en deux dans leur longueur, enlevez-en les graines.

Blanchissez les piments ainsi apprêtés dans de l'eau salée bouillante, pendant
deux minutes.

Préparez une farce avec le veau, le jambon, les jaunes d'œufs, le persil, la
muscade, {\ppp30\mmm} grammes de parmesan, le vin blanc, du sel, du poivre et
garnissez-en les piments.

Foncez avec le beurre un plat allant au feu, disposez dedans les piments, la
farce en l'air, saupoudrez-les avec le reste du fromage et de la mie de pain
rassis tamisée, mettez au four, laissez cuire doucement pendant une heure et
demie, en faisant gratiner le dessus.

Servez dans le plat et envoyez en même temps le jus de viande chaud dans une
saucière.

L'enveloppe du piment donne à la préparation une saveur particulière très
agréable,

\sk

On peut apprêter de même des paprikas verts ; mais alors, il est bon de
remplacer, dans l’assaisonnement de la farce, la muscade par du paprika en
poudre.

\sk

\index{Aubergines farcies gratinées}
\index{Courgettes farcies, gratinées}
On peut aussi préparer d'une façon analogue des aubergines, des courgettes ;
mais, comme ces dernières n'ont par elles-mêmes que peu de goût et ne servent
guère que d'enveloppe, on devra relever la farce avec un peu plus de poivre, de
paprika ou de piment, au goût.

\section*{\centering Salsifis au gratin.}
\phantomsection
\addcontentsline{toc}{section}{ Salsifis au gratin.}
\index{Salsifis au gratin}

Pour six personnes prenez :

\footnotesize
\begin{longtable}{rrrp{16em}}
  1 000 & grammes & de & salsifis,                                                                        \\
    350 & grammes & de & sauce Béchamel grasse on maigre, au choix,                                       \\
    250 & grammes & de & gruyère ou de parmesan râpé, ou un mélange des deux fromages,                    \\
     40 & grammes & de & beurre,                                                                          \\
     30 & grammes & de & mie de pain rassis tamisée,                                                      \\
        &         &    & vinaigre,                                                                        \\
        &         &    & sel.                                                                             \\
\end{longtable}
\normalsize

Grattez les salsifis ; mettez-les dans de l'eau légèrement vinaigrée pour les
empêcher de noircir ; lavez-les ensuite, coupez-les en morceaux et faites-les
cuire dans de l'eau salée bouillante.

Incorporez à la béchamel {\ppp150\mmm} grammes de fromage, chauffez cette
mornay.

Étendez sur un plat en porcelaine allant au feu une partie de la sauce,
disposez dessus les salsifis, couvrez avec le reste de la sauce Mornay,
saupoudrez avec le reste du fromage mélangé avec la mie de pain rassis tamisée,
mettez par-dessus le beurre coupé en petits morceaux et faites gratiner au
four.

\sk

\index{Asperges au gratin}
\index{Betteraves au gratin}
\index{Cardons au gratin}
\index{Carottes au gratin}
\index{Chou-fleur au gratin}
\index{Chou-rave au gratin}
\index{Choux de Bruxelles au gratin}
\index{Crosnes au gratin}
\index{Endives au gratin}
\index{Cerfeuil bulbeux au gratin}
\index{Oignons au gratin}
\index{Champignons au gratin}
\index{Marrons au gratin}
\index{Haricots verts au gratin}
\index{Haricots secs au gratin}
\index{Fonds d'artichauts au gratin}
\index{Pommes de terre  au gratin}
\index{Lentilles au gratin}
\index{Navets au gratin}
\index{Topinambours au gratin}
\index{Pois au gratin}
\index{Pointes d'asperges au gratin}
On peut préparer dans le même esprit la plupart des légumes : pommes de terre,
haricots verts ou secs, pois, lentilles, choux-fleurs, choux de Bruxelles,
pointes d'asperges, fonds d'artichauts, topinambours, raves, cerfeuil bulbeux,
cardons, endives, crosnes, betteraves, carottes, navets, oignons, champignons
et aussi le riz, les pâtes, les marrons.

On aura ainsi un grand nombre de plats agréables.

\section*{\centering Salsifis, sauce au vin et au mirepoix.}
\phantomsection
\addcontentsline{toc}{section}{ Salsifis, sauce au vin et au mirepoix.}
\index{Salsifis, sauce au vin et au mirepoix}

Pour six personnes prenez :

\footnotesize
\begin{longtable}{rrrp{16em}}
    900 & grammes & de & salsifis épluchés,                                                               \\
    250 & grammes & de & bon bouillon,                                                                    \\
    125 & grammes & de & vin blanc,                                                                       \\
    100 & grammes & de & jambon salé, non fumé, coupé en petits morceaux,                                 \\
    100 & grammes & de & noix de veau, coupée en petits morceaux,                                         \\
    100 & grammes & de & beurre,                                                                          \\
     20 & grammes & d' & oignon haché fin,                                                                \\
     20 & grammes & de & carolte hachée fin,                                                              \\
     10 & grammes & de & fines herbes hachées fin,                                                        \\
      2 & grammes & de & poivre,                                                                          \\
        &         &  1 & feuille de laurier,                                                              \\
        &         &    & le jus d'un citron.                                                              \\
\end{longtable}
\normalsize

Préparez d'abord la sauce.

Foncez une casserole avec {\ppp60\mmm} grammes de beurre, faites revenir dedans
très légèrement le veau, le jambon, la carotte, l'oignon, ajoutez ensuite les
fines herbes, le laurier et le poivre ; mouillez cet appareil Mirepoix avec le
vin et le bouillon, amenez à ébullition, puis laissez cuire doucement pendant
deux heures. Passez à la serviette.

En même temps, faites cuire les salsifis dans de l'eau salée ; égouttez-les.

Mettez les salsifis cuits dans la sauce, laissez mijoter le tout ensemble
pendant une demi-heure, ajoutez le jus de citron, le reste du beurre coupé en
petits morceaux, laissez-le simplement fondre, goûtez (il faut quelquefois un
peu de sel, mais le plus souvent celui du jambon suffit) et servez.

\sk

\index{Salsifis, sauce au vin et au mirepoix}
\index{Cerfeuil bulbeux, sauce au vin et au mirepoix}
\index{Crosnes, sauce au vin el au mirepoix}
\index{Céleri-rave, sauce au vin et au mirepoix}
\index{Fonds d'artichauts, sauce au vin et au mirepoix}

On peut préparer de même d’autres légumes, notamment le céleri-rave, le
cerfeuil bulbeux, les fonds d'artichauts, les crosnes ; et aussi des viandes
blanches, du gras-double, etc.

\section*{\centering Crosnes.}
\phantomsection
\addcontentsline{toc}{section}{ Crosnes.}
\index{Crosnes}

Les crosnes sont les rhizomes tuberculeux comestibles d'un stachys originaire
du Japon, famille des Labiées.

Ils ont été introduits en France en {\ppp1858\mmm} par M. Pallieux et semés
pour la première fois à Crosnes (Seine-et-Oise), lieu de sa naissance, d'où
leur nom.

Les crosnes ont un goût intermédiaire entre ceux du salsifis et de l’artichaut.

\index{Crosnes (Différentes manières d'accommoder les)}
\index{Crosnes à la crème}
\index{Crosnes au velouté}
\index{Crosnes au jus}
\index{Crosnes au jus, avec fromage}
\index{Crosnes au jus, sans fromage}
\index{Crosnes au gratin}
\index{Crosnes en purée}
\index{Crosnes en beignets}
\index{Crosnes en croquettes}
\index{Crosnes au mirepoix}
\index{Crosnes en ragoût}
\index{Crosnes à la béchamel}
\index{Crosnes sautés au beurre}
\index{Crosnes sautés en salade}
On accommode les crosnes de différentes manières : à la crème ; au velouté ; au
jus, avec ou sans fromage ; au gratin ; en ragoût ; à la béchamel ; au
mirepoix ; en purée ; en beignets ; en croquettes ; on les fait encore sauter
au beurre et on les présente aussi en salade,

\section*{\centering Crosnes au piment et à la crème.}
\phantomsection
\addcontentsline{toc}{section}{ Crosnes au piment et à la crème.}
\index{Crosnes au piment et à la crème}
\index{Crosnes à la crème}

Prenez de beaux crosnes bien blancs et bien frais ; débarrassez-les de leurs
radicelles, lavez-les, vannez-les ensuite dans un linge avec du sel gris pour
enlever la pellicule qui les recouvre ; lavez-les de nouveau ; séchez-les.

Faites-les cuire aux trois quarts dans du beurre, à l'étouffée avec des languettes
de piment d'Espagne ; salez ; mouillez ensuite avec de la crème épaisse et achevez
la cuisson à tout petit feu. Au dernier moment, ajoutez un peu de crème fraîche,
chauffez simplement et servez.

\section*{\centering Purée de crosnes et de pommes de terre.}
\phantomsection
\addcontentsline{toc}{section}{ Purée de crosnes et de pommes de terre.}
\index{Purée de crosnes et de pommes de terre}
\index{Crosnes en purée}

Pour quatre personnes prenez :

\footnotesize
\begin{longtable}{rrrp{16em}}
    500 & grammes & de & crosnes,                                                                         \\
    250 & grammes & de & pommes de terre,                                                                 \\
    150 & grammes & de & crème épaisse,                                                                   \\
    100 & grammes & de & beurre,                                                                          \\
        &         & 12 & petites tranches de pain de mie,                                                 \\
        &         &  2 & jaunes d'œufs frais,                                                             \\
        &         &    & sel.                                                                             \\
        &         &    & muscade.                                                                         \\
\end{longtable}
\normalsize

Nettoyez et lavez les crosnes comme il est dit ci-dessus.

Pelez les pommes de terre, lavez-les, coupez-les en petits morceaux.

Faites cuire les pommes de terre dans de l'eau salée, les crosnes dans un
blanc. Passez le tout au tamis.

Mettez la purée dans une casserole, séchez-la légèrement sur le feu en la
remuant constamment, ajoutez-y {\ppp50\mmm} grammes de beurre, la crème, les
jaunes d'œufs, de la muscade au goût et du sel s'il est nécessaire. Chauffez
doucement ; mélangez bien.

Faites dorer le pain dans le reste du beurre ; versez la purée dans un plat,
garnissez avec les tranches de pain et servez,

\section*{\centering Oignons farcis.}
\phantomsection
\addcontentsline{toc}{section}{ Oignons farcis.}
\index{Oignons farcis}

Prenez de beaux oignons nouveaux ou des oignons doux d'Espagne ; enlevez sur
chacun d'eux, à la partie supérieure, une rondelle destinée à faire couvercle ;
creusez ce qui reste en laissant seulement quelques millimètres d'épaisseur.

Hachez l'oignon extrait.

\index{Farce pour oignons}
Faites cuire, d'une part, l'oignon haché avec des champignons et du persil
hachés, dans du beurre ; d'autre part, de la semoule dans du fond de veau et
volaille. Réunissez le tout, mélangez bien et garnissez le creux des oignons
avec ce mélange.

Disposez les oignons farcis dans un plat allant au feu ; mettez dessus les
couvercles ; mouillez avec du fond de veau et volaille corsé ou avec du bon jus
de viande concentré ; laissez cuire pendant une heure à tout petit feu ;
arrosez fréquemment pendant la cuisson.

À la fin de l'opération, le jus doit être réduit à l'état de demi-glace.

\medskip

Les oignons nouveaux ainsi préparés constituent une excellente garniture pour
pièce de viande.

\medskip

Les oignons d'Espagne farcis sont un excellent entremets de légumes.

\section*{\centering Purée d'oignons à la crème.}
\phantomsection
\addcontentsline{toc}{section}{ Purée d'oignons à la crème.}
\index{Purée d'oignons à la crème}

Pour quatre personnes prenez :

\footnotesize
\begin{longtable}{rrrp{16em}}
  1 000 & grammes & d' & oignons épluchés,                                                                \\
    300 & grammes & de & beurre,                                                                          \\
    250 & grammes & de & crème,                                                                           \\
     40 & grammes & de & farine,                                                                          \\
        &         &    & sucre en poudre,                                                                 \\
        &         &    & sel.                                                                             \\
\end{longtable}
\normalsize

Coupez les oignons en rondelles et faites-les cuire dans {\ppp250\mmm} grammes
de beurre, pendant {\ppp20\mmm} à {\ppp30\mmm} minutes, à feu doux, sans les
laisser roussir. Passez-les au tamis.

Faites un roux avec le reste du beurre et la farine, mouillez avec la crème,
ajoutez la purée passée, du sucre et du sel au goût, chauffez pendant un moment
et servez.

Le plat est incontestablement meilleur avec des oignons nouveaux, les vieux
étant toujours plus âcres.

\medskip

Cette purée accompagne admirablement le porc braisé et les côtelettes de
mouton braisées.

\section*{\centering Betteraves.}
\phantomsection
\addcontentsline{toc}{section}{ Betteraves.}
\label{pg0780} \hypertarget{p0780}{}
\index{Betteraves}
\index{Betteraves à la crème}
\index{Betteraves au lard}
\index{Betteraves au beurre}
\index{Betteraves à la béchamel}

Pour préparer les betteraves de façon à en faire une garniture (qui accompagne
on ne peut mieux le veau), faites-les cuire au four, pelez-les, hachez-les ou,
si vous le préférez, passez-les au tamis un peu gros.

Amalgamez-les soit avec une sauce Béchamel, soit avec une sauce à la crème,
soit, plus simplement, avec de la crème épaisse ou avec du beurre frais. Dans
ces deux derniers cas, saupoudrez les betteraves avec du persil haché fin, au
moment de servir. Assaisonnez au goût.

\sk

Comme entremets de légumes, on peut apprêter les betteraves au lard.

\section*{\centering Betteraves confites, au raifort.}
\phantomsection
\addcontentsline{toc}{section}{ Betteraves confites, au raifort.}
\index{Betteraves confites, au raifort}

Faites cuire des betteraves au four, pelez-les, coupez-les en tranches.

Râpez de la racine de raifort.

Mettez dans un pot des couches alternées de betterave et de raifort râpé ;
noyez le tout dans du bon vinaigre à l’estragon. La préparation sera à point au
bout de quelques jours.

Excellent condiment, en particulier avec le bœuf bouilli.

\section*{\centering Concombres.}
\phantomsection
\addcontentsline{toc}{section}{ Concombres.}
\index{Concombres}
\index{Concombres blancs en hors-d'œuvre}
\index{Concombres verts en saumure}

On apprête les concombres de différentes manières : les concombres blancs sont
excellents comme hors-d'œuvre ; les concombres verts sont très bons en saumure.

\sk

Pour préparer des concombres en hors-d'œuvre, prenez de beaux concombres
blancs, pelez-les, coupez-les en tranches minces, retirez-en les graines ;
saupoudrez de sel gris, laissez dégorger. Assaisonnez les concombres dégorgés
soit avec de l'huile, du vinaigre, du sel et du poivre, soit avec de la crème,
du vinaigre, du sel et du poivre, où bien encore avec une mayonnaise,

Ces trois préparations qui ne différent que par l'emploi de l'huile, de la
crème et de la mayonnaise constituent trois hors-d'œuvre appétissants ayant
chacun un goût particulier.

\sk

Pour préparer des concombres en saumure, prenez des concombres verts de
dimensions moyennes, de {\ppp10\mmm} à {\ppp12\mmm} centimètres de longueur,
brossez-les, essuyez-les, coupez les extrémités et taillez en croix les parties
à vif dans le but d'éviter que la fermentation les fasse éclater,

Disposez dans une terrine profonde, beaucoup plus haute que large, des couches
successives de feuilles de vigne, de cerisier, de raifort, ou, à leur défaut,
un peu de racine de raifort râpée, du fenouil, du sel gris et une partie des
concombres préparés comme il vient d'être dit. Continuez les alternances
d'aromates et de concombres jusqu'à sept ou huit centimètres de la partie
supérieure de la terrine, noyez le tout dans de l'eau et mettez dessus une
petite planchette en bois que vous chargerez d'un poids de façon que tout soit
couvert par le liquide ; laissez la terrine découverte, tenez-la au frais.
Suivant la saison, les concombres seront prêts au minimum en huit jours, au
maximum en trois semaines environ ; il est bon de surveiller l'opération pour
l'arrêter quand elle est à point.

\medskip

Les concombres saumurés accompagnent admirablement la viande froide, en
particulier le bœuf bouilli.

\section*{\centering Cornichons, petits melons et oignons confits au vinaigre.}
\phantomsection
\addcontentsline{toc}{section}{ Cornichons, petits melons et oignons confits au vinaigre.}
\index{Cornichons, petits melons et oignons confits au vinaigre}

Choisissez de préférence des cornichons petits et de même grosseur, brossez-les
un à un avec une brosse dure, coupez-en les extrémités, puis mettez-les dans un
torchon avec leur poids de sel gris ; mélangez bien sel et cornichons,
suspendez le torchon par les quatre coins au-dessus d'un vase et laissez
dégorger pendant une douzaine d'heures. Essuyez-les ensuite un à un, placez-les
dans un bocal sur une couche d'estragon en branches, ajoutez {\ppp40\mmm}
grains de poivre, plus ou moins de perce-pierre (erithmum maritimum) et deux
petits piments rouges par kilogramme de cornichons ; finissez par une couche
d'estragon ; mouillez avec du vinaigre d'Orléans fort, en quantité suffisante
pour couvrir le tout. Bouchez hermétiquement ; laissez confire pendant quatre
à cinq mois.

\sk

On peut encore préparer les cornichons de la façon suivante : Brossez et faites
dégorger, comme précédemment, {\ppp2\mmm} kilogrammes de cornichons,
essuyez-les. Mettez au fond d'un vase en grès plus ou moins d’estragon, au
goût, un peu de thym, une feuille de laurier, deux petits piments rouges, une
petite gousse d'ail, cinq ou six petits oignons épluchés ; au-dessus une couche
de cornichons, puis cinq petits oignons, deux petites échalotes, une nouvelle
couche de cornichons ; au-dessus cinq petits oignons, deux petites échalotes,
une autre couche de cornichons et terminez par cinq ou six petits oignons, une
petite gousse d'ail, deux petits piments rouges, deux clous de girofle, trente
grains de poivre, une feuille de laurier, un peu de thym ; couvrez avec plus ou
moins d'estragon et noyez le tout dans du bon vinaigre.

Laissez confire.

\sk

Pour faire des petits melons confits, prenez des melons de la grosseur d'une
noix. La préparation est identique à celle des cornichons.

\sk

Les oignons peuvent être préparés d'une façon analogue, mais on ne les fait pas
dégorger dans du sel.

Ils seront choisis tout petits et épluchés délicatement.

\section*{\centering Chou-rave farci.}
\phantomsection
\addcontentsline{toc}{section}{ Chou-rave farci.}
\index{Chou-rave farci}
\index{Farce pour raves}

Prenez un beau chou-rave, enlevez dessus une calotte destinée à servir de
couvercle.

Creusez l'intérieur du chou. blanchissez-le avec le couvercle dans de l'eau
salée bouillante. Séchez-le ensuite au four.

Faites braiser du filet de porc, assaisonnez-le avec sel, poivre et épices au goût.

Hachez fin le porc braisé, mélangez-le avec son jus ; garnissez l'intérieur du
chou-rave avec cette farce, fermez avec le couvercle et achevez la cuisson du
chou farci dans de la sauce Soubise.

Servez.

\section*{\centering Choux-fleurs.}
\phantomsection
\addcontentsline{toc}{section}{ Choux-fleurs.}
\index{Choux-fleurs}
\index{Choux-fleurs (Différentes manières  d'accommoder, les)}
\index{Choux-fleurs au beurre}
\index{Choux-fleurs au fromage}
\index{Choux-fleurs en hors-d'œuvre}
\index{Choux-Îeurs en salade}
\index{Choux-fleurs à diverses sauces}
\index{Brocoli}

Les choux-fleurs, après qu'ils ont été cuits dans de l’eau salée, peuvent être
accommodés de bien des manières. On peut les servir chauds ou froids.

Les procédés classiques de préparation pour les choux-fleurs chauds sont les
suivants :

au beurre ;

sautés ;

au fromage (gratinés où non) ;

en friture, enrobés dans de la pâte ;

masqués de sauces diverses : béchamel, mornay (gratinés ou non), hollandaise
ou mousseline, sauce tomate, sauces à la crème ou au beurre ;

en soufflés ; etc.

Froids, on pourra les présenter :

au naturel, sur une serviette, avec quelques feuilles tendres, ou avec un
cordon de persil ;

en hors-d'œuvre, masqués d'une mayonnaise, d'une sauce à la moutarde ou
d'une sauce à la crème ;

en salade.

\sk

Ces divers procédés sont applicables aux brocoli.

\section*{\centering Chou-fleur, sauce hollandaise ou sauce mousseline aux pointes d'asperges.}
\phantomsection
\addcontentsline{toc}{section}{ Chou-fleur, sauce hollandaise ou sauce mousseline aux pointes d'asperges.}
\index{Chou-fleur, sauce hollandaise ou sauce mousseline aux pointes d'asperges}
\index{Chou-fleur, sauce hollandaise où sauce mousseline}

Faites cuire le chou-fleur dans de l'eau salée ; égouttez-le.

Faites blanchir pendant cinq minutes des pointes d'asperges dans de l'eau
salée ; égouttez-les ; achevez leur cuisson dans du beurre ou dans de la
crème ; passez-les au travers d'une passoire à l'aide d'un pilon. Tenez au
chaud.

Préparez une sauce hollandaise, \hyperlink{p0362}{p. \pageref{pg0362}} ou une
sauce mousseline, \hyperlink{p0759}{p. \pageref{pg0759}} ; incorporez-y en
fouettant la purée de pointes d'asperges ; complétez l'assaisonnement au goût
avec sel, poivre et jus de citron.

Dressez le chou-fleur sur un plat, masquez-le avec la sauce et servez.

\sk

\index{Céleri-rave, sauce hollandaise ou sauce mousseline aux pointes d'asperges}
\index{Fonds d'artichauts, sauce hollandaise ou sauce mousseline aux pointes d'asperges.}
\index{Artichauts, sauce hollandaise ou sauce mousseline aux pointes d'asperges}
Ces sauces exquises peuvent accompagner d'autres légumes : fonds d'artichauts,
céleri-rave, asperges, etc.

\section*{\centering Chou-fleur sauté.}
\phantomsection
\addcontentsline{toc}{section}{ Chou-fleur sauté.}
\index{Chou-fleur sauté}

Pour quatre personnes prenez :

\footnotesize
\begin{longtable}{rrrp{16em}}
  1 000 & grammes & de & chou-fleur épluché et séparé en petits bouquets,                                 \\
    175 & grammes & de & beurre frais,                                                                    \\
        &         &    & sel et poivre.                                                                   \\
\end{longtable}
\normalsize

Faites cuire le chou-fleur dans de l'eau salée, en le tenant un peu ferme.

Mettez dans une sauteuse le beurre et le chou-fleur, salez, poivrez au goût et
faites sauter à petit feu, lentement, pendant {\ppp20\mmm} à {\ppp25\mmm}
minutes, de façon à dorer parfaitement les bouquets sans les laisser brûler.

Servez dans un légumier.

Le chou-fleur sauté ainsi est excellent.

\section*{\centering Chou-fleur pané, sauté.}
\phantomsection
\addcontentsline{toc}{section}{ Chou-fleur pané, sauté.}
\index{Chou-fleur pané, sauté}

Pour quatre personnes prenez :

\footnotesize
\begin{longtable}{rrrp{16em}}
  1 000 & grammes & de & chou-fleur, épluché et séparé en petits bouquets,                                \\
    200 & grammes & de & beurre frais,                                                                    \\
    100 & grammes & de & pain,                                                                            \\
        &         &    & sel et poivre.                                                                   \\
\end{longtable}
\normalsize

Faites cuire le chou-fleur dans de l'eau salée en maintenant les bouquets entiers
et un peu fermes.

Coupez le pain en tranches ; séchez-le et grillez-le au four ; écrasez-le et
passez-le au tamis.

Mettez dans une sauteuse le beurre avec le pain tamisé ; laissez ce dernier
s'imbiber et se dorer ; ajoutez ensuite le chou-fleur, salez, poivrez, faites
sauter le tout pendant quelques minutes de manière à enrober de pain toutes les
parties du chou-fleur, puis servez.

La chapelure rompt la monotonie du légume et en fait valoir le goût.

\section*{\centering Purée gratinée de chou-fleur et de tomate.}
\phantomsection
\addcontentsline{toc}{section}{ Purée gratinée de chou-fleur et de tomate.}
\index{Purée gratinée de chou-fleur et de tomate}
\index{Chou-fleur en purée}

Pour six personnes prenez :

\footnotesize
\begin{longtable}{rlrp{16em}}
  1 000 & grammes  & de & chou-fleur épluché,                                                             \\
    250 & grammes  & de & purée de tomates aromatisée,                                                    \\
    125 & grammes  & de & fromage de Gruyère räpé,                                                        \\
     90 & grammes  & de & beurre,                                                                         \\
      5 & grammes  & de & sel gris,                                                                       \\
      5 & grammes  & de & sel blanc,                                                                      \\
      1 & gramme   & de & poivre,                                                                         \\
     1 & litre 1/2 & d' & eau.                                                                            \\
\end{longtable}
\normalsize

Faites bouillir l'eau salée avec le sel gris, jetez dedans le chou-fleur épluché et
lavé, laissez-le cuire pendant vingt minutes.

Retirez le chou-fleur, égouttez-le, passez-le en purée.

Faites cuire pendant cinq minutes la purée de tomates avec {\ppp60\mmm} grammes
de beurre.

Mettez dans une casserole le reste du beurre, la purée de chou-fleur,
{\ppp100\mmm} grammes de gruyère, le sel blanc, le poivre, chauffez, ajoutez la
purée de tomates, mélangez bien.

Versez le tout dans un plat allant au feu, saupoudrez avec le reste du gruyère et
faites gratiner au four pendant dix à quinze minutes.

\section*{\centering Choux de Bruxelles.}
\phantomsection
\addcontentsline{toc}{section}{ Choux de Bruxelles.}
\index{Choux de Bruxelles}

Prenez de beaux petits choux de Bruxelles bien serrés, épluchez-les, lavez-les,
blanchissez-les pendant dix minutes dans de l'eau bouillante, puis plongez-les
dans une autre casserole d'eau bouillante salée dans laquelle vous les
laisserez cuire pendant un temps variant entre {\ppp5\mmm} et {\ppp15\mmm}
minutes suivant la qualité des choux, juste assez pour qu'ils soient tendres,
mais non désagrégés. Égouttez-les.

\medskip

\index{Choux de Bruxelles (Différentes manières d'accommoder les)}
On peut apprêter les choux de Bruxelles de différentes manières :

arrosés de beurre fondu, additionnés ou non de chapelure blondie dans du
beurre, et assaisonnés avec sel, poivre et muscade, au goût ;

\index{Choux de Bruxelles sautés}
sautés au beurre à feu vif pendant six à huit minutes et saupoudrés ou non de
persil haché ;

\index{Choux de Bruxelles à l'étouffée, au beurre}
à l’étouffée, au beurre ;

\index{Choux de Bruxelles on salade}
en salade ; après les avoir fait cuire en les tenant un peu fermes, on les
laisse refroidir, puis on les assaisonne au goût avec huile, vinaigre ou jus de
citron, sel et poivre.

\section*{\centering Choux de Bruxelles à la crème.}
\phantomsection
\addcontentsline{toc}{section}{ Choux de Bruxelles à la crème.}
\index{Choux de Bruxelles à la crème}

Pour quatre personnes prenez :

\footnotesize
\begin{longtable}{rrrp{16em}}
    125 & grammes   & de & crème épaisse,                                                                 \\
     50 & grammes   & de & beurre,                                                                        \\
      7 & grammes   & de & sel blanc,                                                                     \\
        & 1 litre   & de & choux de Bruxelles,                                                            \\
        & 1/2 litre & de & lait.                                                                          \\
\end{longtable}
\normalsize

Épluchez les choux, faites-les blanchir pendant dix minutes dans de l’eau
bouillante, égouttez-les, mettez-les ensuite dans une casserole avec le lait et
le sel ; laissez mijoter à tout petit feu pendant une heure à une heure et
demie ; ajoutez le beurre et la crème ; chauffez pendant quelques instants sans
laisser bouillir et servez.

Les choux de Bruxelles adoucis par le lait qu'ils ont absorbé n'ont plus la
moindre odeur sulfhydrique. Ils sont très nourrissants et de digestion facile.

\sk

\index{Choux de Bruxelles en purée}
Comme variante, on pourra passer les choux en purée qu'on garnira avec des
croûtons dorés dans du beurre.

\section*{\centering Chou braisé au madère.}
\phantomsection
\addcontentsline{toc}{section}{ Chou braisé au madère.}
\index{Chou braisé au madère}
\index{Chou blanc braisé an madère}

Pour six personnes prenez :

\footnotesize
\begin{longtable}{rrrp{16em}}
    500 & grammes & de & saucisse de Toulouse,                                                            \\
    250 & grammes & de & lard de poitrine,                                                                \\
    200 & grammes & de & madère,                                                                          \\
    125 & grammes & de & lard fumé,                                                                       \\
    125 & grammes & de & carottes,                                                                        \\
     75 & grammes & d' & oignons,                                                                         \\
     60 & grammes & de & beurre ou de bonne graisse de rôti,                                              \\
        &         &  1 & beau chou blanc ou demi-frisé,                                                   \\
        &         &    & bouquet garni,                                                                   \\
        &         &    & épices,                                                                          \\
        &         &    & sel et poivre.                                                                   \\
\end{longtable}
\normalsize

Épluchez et lavez le chou ; faites-le blanchir pendant dix minutes dans de
l'eau bouillante ; égouttez-le.

Ébouillantez le lard de poitrine et le lard fumé pour en enlever l'excès de
sel ; essuyez-les et hachez-les.

Passez le hachis de lard dans le beurre ou la graisse de rôti pendant quelques
minutes pour le dorer légèrement.

Mettez dans une marmite chou, hachis de lard et sa cuisson, carottes, oignons,
bouquet garni, assaisonnez avec poivre et épices au goût, mouillez avec le
madère et laissez cuire doucement en marmite fermée pendant {\ppp3\mmm}
à {\ppp4\mmm} heures. Un peu avant la fin, enlevez carottes, oignons, bouquet
garni, goûtez et ajoutez du sel s'il est nécessaire. Concentrez bien le jus de
cuisson.

Faites griller la saucisse de Toulouse.

Dressez le chou sur un plat, garnissez avec la saucisse et servez.

\sk

Comme variante, on pourra remplacer le lard par des abatis de poulet revenus
dans du beurre et cuits avec carottes, oignons, bouquet garni, sel, poivre,
épices, dans du bouillon corsé de glace de volaille.

Lorsque le jus sera tombé à l’état de demi-glace, on passera abatis et jus au
tamis, on ajoutera au chou, en partie cuit avec le madère et des aromates, la
purée obtenue puis on achèvera la cuisson.

On servira comme précédemment.

\section*{\centering Chou garni.}
\phantomsection
\addcontentsline{toc}{section}{ Chou garni.}
\index{Chou garni}
\index{Chou blanc garni}

Pour six personnes prenez :

\footnotesize
\begin{longtable}{rrrp{16em}}
  1 000 & grammes & de & viande rôtie froide (filet ou faux filet de bœuf, gigot de mouton,
                         râble de lièvre, filet de chevreuil ou de sanglier, etc.),                       \\
  1 000 & grammes & d' & os de porc,                                                                      \\
    750 & grammes & de & vin blanc,                                                                       \\
    500 & grammes & de & jambon d'York, de Westphalie ou de Prague cuit au vin blanc et refroidi
                         dans sa cuisson,                                                                 \\
    250 & grammes & de & lard de poitrine peu salé,                                                       \\
    250 & grammes & de & pommes,                                                                          \\
    200 & grammes & de & bonne graisse de rôti (volaille ou faisan de préférence),                        \\
    100 & grammes & de & beurre,                                                                          \\
        &         &  3 & saucisses de Francfort,                                                          \\
        &         &  3 & saucisses de Strasbourg,                                                         \\
        &         &  1 & cervelas,                                                                        \\
        &         &  1 & très beau chou blanc de la variété dite de Brunswick,                            \\
        &         &    & vinaigre,                                                                        \\
        &         &    & sucre en poudre,                                                                 \\
        &         &    & sel et poivre.                                                                   \\
\end{longtable}
\normalsize

Épluchez le chou, lavez-le, émincez-le comme pour faire de la choucroute,
échaudez-le, égouttez-le,

Pelez les pommes, coupez-les en tranches minces ; enlevez les pépins.

Faites revenir doucement dans la graisse le lard de poitrine coupé très fin,
ajoutez ensuite le beurre, le chou, salez et poivrez, mélangez bien pendant
quelques minutes.

Foncez une marmite en porcelaine allant au feu avec les os de porc, disposez
dessus le mélange de chou et lard, mouillez avec le vin blanc, ajoutez les
pommes et laissez cuire pendant trois heures à feu doux. Mettez alors le
cervelas, les saucisses de Francfort et de Strasbourg et laissez-les pendant le
temps nécessaire à leur cuisson, c'est-à-dire environ {\ppp20\mmm}
à {\ppp30\mmm} minutes pour le cervelas, {\ppp10\mmm} à {\ppp15\mmm} minutes
pour les saucisses.

Goûtez : la préparation doit avoir un goût légèrement aigre-doux très agréable.
Si le vin et les pommes ne le lui ont pas donné, rectifiez, suivant le cas,
avec un peu de vinaigre ou avec un peu de sucre.

Au dernier moment, mettez la viande froide et le jambon coupés en tranches,
chauffez-les simplement.

Dressez le chou sur un plat, garnissez avec le jambon, la viande, les saucisses
et le cervelas ; servez.

\medskip

Ce plat diffère beaucoup de la choucroute garnie, mais il est tout aussi
intéressant.

\section*{\centering Chou à la béchamel, gratiné.}
\phantomsection
\addcontentsline{toc}{section}{ Chou à la béchamel, gratiné.}
\index{Chou à la béchamel, gratiné}

Pour quatre personnes prenez :

\footnotesize
\begin{longtable}{rrrp{16em}}
    200 & grammes & de & sauce Béchamel grasse,                                                           \\
    125 & grammes & de & beurre,                                                                          \\
    125 & grammes & de & gruyère râpé,                                                                    \\
     40 & grammes & de & mie de pain rassis tamisée,                                                      \\
        &         &  1 & chou pesant tout épluché 2 kilogrammes environ,                                  \\
        &         &    & sel et poivre.                                                                   \\
\end{longtable}
\normalsize

Faites blanchir suffisamment le chou dans de l'eau salée bouillante ;
égouttez-le.

Mettez dans une casserole {\ppp100\mmm} grammes de beurre, chauffez, ajoutez le
chou, salez et poivrez au goût ; mélangez bien pendant quelques minutes ;
couvrez la casserole et laissez mijoter doucement au four pendant deux à trois
heures environ.

Disposez dans un plat des couches de chou séparées par du gruyère et de la
béchamel ; terminez par une couche de sauce ; saupoudrez le dessus avec la mie
de pain, ajoutez le reste du beurre coupé en petits morceaux et poussez au four
doux pendant un quart d'heure pour gratiner.

\medskip

Ce plat peut être servi comme entremets de légumes dans un repas de famille.

\section*{\centering Chou au jus et au bacon, gratiné.}
\phantomsection
\addcontentsline{toc}{section}{ Chou au jus et au bacon, gratiné.}
\index{Chou au jus et au bacon, gratiné}

Pour quatre personnes prenez :

\footnotesize
\begin{longtable}{rrrp{16em}}
    300 & grammes & de & bon jus aromatisé et un peu relevé,                                              \\
    250 & grammes & de & bacon,                                                                           \\
     65 & grammes & de & gruyère râpé,                                                                    \\
     65 & grammes & de & parmesan râpé,                                                                   \\
     45 & grammes & de & beurre,                                                                          \\
     35 & grammes & de & mie de pain rassis tamisée,                                                      \\
        &         &  1 & chou pesant tout épluché 2 kilogrammes environ,                                  \\
        &         &    & sel et poivre.                                                                   \\
\end{longtable}
\normalsize

Faites blanchir convenablement le chou dans de l'eau salée bouillante, égouttez-le,

Coupez le bacon en petits cubes, faites-le revenir dans {\ppp15\mmm} grammes de
beurre, mettez-le dans une braisière avec le chou et le jus, salez, poivrez en
tenant compte de l’assaisonnement du jus ; couvrez et laissez braiser au four
doux pendant deux à trois heures. Réduisez suffisamment le jus.

Disposez dans un plat des couches de chou séparées par du gruyère et du
parmesan râpés, mouillez avec le jus réduit, saupoudrez le dessus avec la mie
de pain, ajoutez le reste du beurre coupé en petits morceaux et faites gratiner
au four pendant un quart d'heure environ.

Servez aussitôt.

\medskip

Ce chou gratiné accompagne on ne peut mieux le porc sous toutes ses formes.

\section*{\centering Chou blanc farci.}
\phantomsection
\addcontentsline{toc}{section}{ Chou blanc farci.}
\index{Chou blanc farci}

Pour six personnes prenez :

\footnotesize
\begin{longtable}{rrrp{16em}}
    250 & grammes  & de & lard gras frais,                                                                \\
    125 & grammes  & de & lard fumé,                                                                      \\
     45 & grammes  & de & fine champagne,                                                                 \\
        & 2 litres & de & bouillon,                                                                       \\
        &          &  1 & chou demi-frisé\footnote{Le chou demi-frisé est intermédiaire
                          entre le chou blanc et le chou de Milan.} de 15 centimètres de diamètre,        \\
        &          &  1 & perdrix, pouvant fournir 200 grammes de chair environ,                          \\
        &          &  1 & jaune d'œuf frais,                                                              \\
        &          &    & crépine de porc,                                                                \\
        &          &    & couenne maigre,                                                                 \\
        &          &    & quatre épices,                                                                  \\
        &          &    & curry,                                                                          \\
        &          &    & muscade,                                                                        \\
        &          &    & sel et poivre.                                                                  \\
\end{longtable}
\normalsize

\index{Farce pour chou blanc}
Désossez la perdrix, hachez ensemble perdrix, lard frais et lard fumé, mouillez
le mélange avec la fine champagne, assaisonnez avec quatre épices, curry,
muscade, sel et poivre, au goût, et liez avec le jaune d'œuf.

Faites cuire les déchets de perdrix dans le bouillon, passez et réservez.

Enlevez au chou les feuilles extérieures dures, parez-le et faites-le blanchir
entier, pendant dix minutes, dans de l'eau salée bouillante ; égouttez-le,
mettez-le ensuite sur une planche, ouvrez-le en écartant les feuilles une
à une, lavez chaque feuille soigneusement, l’une après l'autre, puis, en
commençant par le cœur, garnissez à la main les intervalles des feuilles avec
la farce que vous avez préparée. Relevez ensuite successivement les feuilles
pour reconstituer le chou dans sa forme primitive, enveloppez-le d'une crépine,
garnissez-le de couenne du côté du trognon, ficelez-le. Mettez-le dans une
casserole en le faisant reposer sur la couenne, de façon à éviter l’adhérence
au fond du récipient, mouillez avec le bouillon et laissez cuire à petit feu,
en casserole couverte, pendant quatre heures.

Pendant la cuisson, arrosez fréquemment avec le jus.

Au moment de servir, enlevez la ficelle, dressez le chou sur un plat et envoyez
à part le jus de cuisson dégraissé, dans une saucière.

Ce chou farci est un véritable poème. Excellent le jour même, il est encore
meilleur réchauffé le lendemain.

\section*{\centering Chou rouge farci.}
\phantomsection
\addcontentsline{toc}{section}{ Chou rouge farci.}
\index{Chou rouge farci}
\index{Faisan au chou}

On peut aussi bien farcir un chou rouge qu'un chou blanc, mais le chou rouge
ayant un goût plus prononcé, il vaut mieux, au lieu d'une farce à la perdrix dont
la finesse risquerait d'être masquée, employer une farce au faisan relevée par un
peu de jambon fumé et additionnée de graisse de volaille ou de panne qui en
augmentera le moelleux.

\medskip

Pour huit personnes prenez :

\footnotesize
\begin{longtable}{rrrrp{16em}}
  &     200 & grammes  & de & jambon fumé,                                                                \\
  &     150 & grammes  & de & lard gras frais,                                                            \\
  &     100 & grammes  & de & graisse de volaille ou de panne,                                            \\
  &      60 & grammes  & de & fine champagne,                                                             \\
  &      20 & grammes  & de & sel,                                                                        \\
  & \multicolumn{2}{r}{5 décigrammes}& de & poivre,                                                       \\
  & \multicolumn{2}{r}{2 décigrammes}& de & muscade,                                                      \\
  &         & 2 litres & de & bouillon,                                                                   \\
  &         &          &  1 & gros chou rouge ayant un diamètre de 20 centimètres,                        \\
  &         &          &  1 & faisan, en état de fournir 700 grammes de chair environ,                    \\
  &         &          &    & crépine de porc,                                                            \\
  &         &          &    & couenne maigre,                                                             \\
  &         &          &    & jaune d'œuf frais.                                                          \\
\end{longtable}
\normalsize

\index{Farce pour chou rouge}
Désossez le faisan, hachez ensemble faisan, jambon fumé, lard gras frais,
graisse de volaille ou panne, mouillez avec la fine champagne, assaisonnez avec
sel, poivre, muscade, liez avec le jaune d'œuf.

Tout le reste de la préparation est exactement le même que pour le chou blanc,

\section*{\centering Chou rouge confit au vinaigre.}
\phantomsection
\addcontentsline{toc}{section}{ Chou rouge confit au vinaigre.}
\index{Chou rouge confit au vinaigre}

Prenez un beau chou rouge pommé, coupez-le en tronçons, enlevez les grosses
feuilles extérieures, les fortes côtes, puis émincez-le en julienne comme pour
faire de la choucroute.

Placez-le dans un vase avec deux poignées de sel gris et laissez-le dégorger
pendant {\ppp24\mmm} heures. Tenez au frais. Sortez-le ensuite et égouttez-le.

Mettez au fond d'un bocal quelques branches de persil et d’estragon, un peu de
thym, une gousse d'ail, deux piments rouges, six clous de girofle, quarante
grains de poivre, puis le chou et, par-dessus, du persil, de l'estragon et des
tranches de citron pelé ; noyez le tout dans du vinaigre d'Orléans fort,
couvrez et laissez confire pendant dix jours, dans un endroit tiède. Au bout de
ce temps, le chou est prêt à être employé.

\medskip

Le chou rouge confit constitue un excellent condiment, en particulier avec le
bœuf bouilli. Il peut être servi tel quel où additionné d'huile d'olive et
parfumé avec du cumin.

\label{pg0791} \hypertarget{p0791}{}
\section*{\centering Choucroute au naturel.}
\phantomsection
\addcontentsline{toc}{section}{ Choucroute au naturel.}
\index{Choucroute au naturel}

Pour quatre personnes prenez :

\footnotesize
\begin{longtable}{rrrp{16em}}
  1 000 & grammes & de & choucroute.                                                                      \\
    750 & grammes & de & jarret de veau,                                                                  \\
    500 & grammes & de & vin blanc,                                                                       \\
    125 & grammes & de & beurre,                                                                          \\
    100 & grammes & de & carottes,                                                                        \\
    100 & grammes & de & navets,                                                                          \\
     20 & grammes & de & poireau,                                                                         \\
     15 & grammes & d' & oignon,                                                                          \\
      2 & grammes & de & poivre fraîchement moulu,                                                        \\
        &         &  6 & grains de genièvre.                                                              \\
\end{longtable}
\normalsize

Préparez {\ppp700\mmm} grammes de fond de veau en faisant cuire dans de l'eau,
à bouilli perdu, le jarret de veau et les légumes.

Dessalez la choucroute s'il y a lieu, lavez-la, égouttez-la, mettez-la ensuite
dans une marmite en porcelaine allant au feu, assaisonnez avec le poivre,
ajoutez le beurre et le genièvre, mouillez avec le fond de veau et le vin,
faites partir sur feu vif, puis laissez cuire plus doucement, pendant une heure
et demie, de manière que tout le liquide soit résorbé.

\medskip

Cuite ainsi, la choucroute est blanche, légèrement croustillante et elle est
excellente ; elle peut être servie seule ou comme garniture.

\medskip

Garnie avec des pommes de terre bouillies à l'eau et des saucisses de Francfort,
c'est la plus simple des choucroutes garnies.

\section*{\centering Choucroute garnie.}
\phantomsection
\addcontentsline{toc}{section}{ Choucroute garnie.}
\index{Choucroute garnie}

Il m'est rarement arrivé de manger une choucroute garnie qui m'ait complètement
satisfait ; aussi ai-je essayé de résoudre expérimentalement et avec précision
le problème qui consiste, selon moi, à conserver à la choucroute une saveur
aigrelette, de manière à ne pas en altérer le caractère, tout en réduisant
l'acidité au minimum ; j'ai voulu aussi mitiger le goût de fumé trop prononcé
qu'elle a souvent, et il m'a semblé bon de lui associer une garniture en partie
chaude et en partie froide, de manière à satisfaire tous les goûts.

Voici la formule à laquelle je me suis arrêté.

Pour six personnes ayant bon appétit prenez :

\footnotesize
\begin{longtable}{rrrrp{16em}}
  & \multicolumn{2}{r}{2 kilogrammes} & de & choucroute de Lorraine,                                      \\
  & 1 000 & grammes & de & plat de côtes de bœuf, bien persillé de graisse,                               \\
  & 1 000 & grammes & d' & os de porc,                                                                    \\
  &   500 & grammes & de & jambon de Prague, d'York ou de Westphalie,
                           cuit au vin blanc et refroidi dans la cuisson,                                 \\
  &   500 & grammes & de & poitrine de porc fumée, coupée en petits cubes,                                \\
  &   375 & grammes & de & jarret de veau,                                                                \\
  &   250 & grammes & de & bonne graisse de rôti\footnote{J'ai obtenu un résultat
                           merveilleux en employant de la graisse provenant de la
                           confection d'un faisan farci, \hyperlink{p0636}{p. \pageref{pg0636}}. 
                           Les aromes combinés du faisan, de la bécasse, du foie gras, 
                           des truffes et du porto donnaient à la préparation un charme 
                           inexprimable et ce fut un véritable triomphe.}, et en particulier
                           de graisse de faisan, d'oie ou de canard,                                      \\
  &   125 & grammes & de & couenne maigre,                                                                \\
  &     2 & grammes & de & poivre fraîchement moulu,                                                      \\
  &       & 1 litre & de & vin blanc,                                                                     \\
  &       &         & 12 & grains de genièvre,                                                            \\
  &       &         & 12 & cornichons,                                                                    \\
  &       &         &  6 & pommes de terre,                                                               \\
  &       &         &  1 & saucisses de Francfort,                                                        \\
  &       &         &  3 & saucisses de Strasbourg,                                                       \\
  &       &         &  1 & pied de veau,                                                                  \\
  &       &         &    & carotte,                                                                       \\
  &       &         &    & navet.                                                                         \\
  &       &         &    & panais,                                                                        \\
  &       &         &    & céleri,                                                                        \\
  &       &         &    & bouquet garni,                                                                 \\
  &       &         &    & sel et poivre.                                                                 \\
\end{longtable}
\normalsize

Les proportions de sel et de saumure de la choucroute du commerce étant très
variables, il est essentiel de la goûter avant de l'employer. Suivant les cas, on
pourra s'en servir telle quelle, ou bien on commencera par la laver à l'eau froide ou
à l'eau chaude. et même on la fera blanchir pendant quelques minutes dans de
l'eau bouillante. Dans les deux derniers cas. il faudra la rafraîchir ensuite à l'eau
froide et l'égoutter.

Passons à l'exécution.

Préparez un litre de fond de veau en faisant cuire pendant six heures dans de
l'eau salée et poivrée le pied et le jarret de veau, la couenne, les légumes et le
bouquet garni ; passez-le.

Garnissez le fond d'une marmite en porcelaine\footnote{Je recommande de faire
cuire la choucroute dans une marmite en porcelaine, car cette substance ne peut
lui communiquer aucun goût étranger, ni modifier sa couleur, contrairement à ce
qui aurait lieu avec une marmile en fonte.} épaisse, allant au feu, avec les os
de porc qui, tout en donnant du jus, éviteront les résultats fâcheux de coups
de feu intempestifs, mettez dessus une couche de choucroute, une partie
proportionnelle de la poitrine de porc et de la graisse de rôti, mouillez avec
du vin et du fond de veau, poivrez, ajoutez quelques grains de genièvre et
continuez ainsi à disposer des couches successives arrosées et assaisonnées
jusqu'à épuisement des matières premières, en réservant seulement un peu de
fond de veau, que vous laisserez prendre en gelée et qui servira à la fin pour
décorer les viandes froides qui accompagneront la choucroute.

Faites cuire, d'abord pendant quatre heures, à petit feu, à liquide
frissonnant, comme pour un pot-au-feu, ajoutez alors le bœuf\footnote{Je
préconise la cuisson du plat de côtes dans la choucroute, parce qu'il mitige le
goût du porc fumé qui cuit en même temps.}, dont vous aurez enlevé l'excès de
graisse, continuez la cuisson dans les mêmes conditions pendant trois autres
heures, puis retirez la viande de bœuf ; laissez-la refroidir.

Goûtez la choucroute, ajoutez du sel si c'est nécessaire, laissez cuire encore
pendant trois heures, de façon que la cuisson ait duré en tout dix heures.
Dégraissez soigneusement, retirez les os, les grains de genièvre et tenez au chaud,
Toute la poitrine de porc doit être fondue, absorbée par la choucroute.

Dix minutes avant de servir, faites cuire les saucisses dans de l'eau très
chaude, mais non bouillante, et laissez-les gonfler ; elles seront ainsi plus
moelleuses que si elles étaient cuites dans la choucroute.

Dressez la choucroute sur un plat chaud, mettez autour les pommes de terre
cuites en robe de chambre et pelées ; garnissez avec les saucisses.

Disposez sur un autre plat le bœuf froid et le jambon coupés en tranches,
décorez avec la gelée de veau réservée et les cornichons. Servez.

Le mets ainsi présenté a un aspect bon enfant, sans prétention, qui prévient en
sa faveur ; il est copieux, il a du fumet, sa saveur est délectable : il ne
provoque pas de soif immodérée et il ne laisse après lui que des souvenirs
agréables.

Comme boisson, la bière est généralement indiquée ; mais, à moins d’avoir de la
bière de tout premier ordre, j'aime autant un petit vin léger, blanc ou gris.

\section*{\centering Choucroute garnie.}
\phantomsection
\addcontentsline{toc}{section}{ Choucroute garnie.}
\index{Choucroute garnie}
\index{Choucroute garnie (autre formule}

\begin{center}
\textit{(Autre formule).}
\end{center}

Pour quatre personnes prenez :

\footnotesize
\begin{longtable}{rrrp{16em}}
  1 000 & grammes & de & choucroute,                                                                      \\
  1 000 & grammes & d' & os de porc,                                                                      \\
    750 & grammes & de & vin blanc,                                                                       \\
    250 & grammes & de & graisse de volaille ou de panne de porc fondue,                                  \\
    200 & grammes & de & jambon de Prague, d'York ou de Westphalie cuit
                         au vin blanc, refroidi dans sa cuisson et haché,                                 \\
    200 & grammes & de & langue fumée, coupée en tranches,                                                \\
      1 & gramme  & de & poivre fraîchement moulu,                                                        \\
        &         &  8 & grains de genièvre,                                                              \\
        &         &  4 & côtelettes de porc, panées,                                                      \\
        &         &    & sel.                                                                             \\
\end{longtable}
\normalsize

Dessalez et lavez la choucroute comme il est dit dans la formule précédente,
puis faites-la cuire avec les os, la graisse, le jambon, le poivre, le genièvre
et le vin pendant huit à dix heures.

Au dernier moment, faites griller les côtelettes de porc.

Dressez la choucroute sur un plat, garnissez-la avec les côtelettes, décorez
avec la langue et servez.

\sk

Comme variantes, on pourra remplacer la graisse de volaille ou de porc par de
la graisse de rôti de gibier à poil ou à plumes, le jambon, la langue et les
côtelettes par des émincés de cuissot de marcassin ou de sanglier et aussi par
du faisan rôti, par exemple.

On aura ainsi une choucroute garnie au gibier. d'un goût très fin, d'allure
distinguée, pouvant faire bonne figure dans un déjeuner d'amateurs.

\section*{\centering Choucroute garnie à la juive.}
\phantomsection
\addcontentsline{toc}{section}{ Choucroute garnie à la juive.}
\index{Choucroute garnie à la juive}
\index{Choucroute garnie, à la graisse d'oie}

Les personnes qui, par goût ou pour obéir à des exigences rituelles, ne mangent
pas de porc pourront trouver satisfaction dans cette choucroute garnie,
préparée avec un \textit{modus operandi} légèrement différent de celui des
formules précédentes ;

\medskip

Pour six personnes prenez :

\footnotesize
\begin{longtable}{rrrp{16em}}
  1 500 & grammes & de & choucroute blanche de Strasbourg,                                                \\
  1 000 & grammes & de & bœuf fumé,                                                                       \\
    500 & grammes & de & saucisses de bœuf,                                                               \\
    400 & grammes & de & vin blanc,                                                                       \\
    400 & grammes & de & bouillon,                                                                        \\
    250 & grammes & de & pommes de terre rouges, farineuses,                                              \\
     60 & grammes & de & graisse d'oie,                                                                   \\
        &         &  2 & échalotes,                                                                       \\
        &         &    & ail,                                                                             \\
        &         &    & persil,                                                                          \\
        &         &    & sel et poivre.                                                                   \\
\end{longtable}
\normalsize

Passez la choucroute à l'eau fraîche, égouttez-la. mettez-en la moitié avec
{\ppp30\mmm} grammes de graisse d'oie dans une casserole en cuivre étamé,
placez dessus la viande fumée lavée à l'eau chaude, ensuite le reste de la
choucroute, mouillez avec le vin blanc et le bouillon, couvrez et laissez cuire
à petit feu pendant six heures,

Lorsque la viande est devenue très tendre, découvrez la casserole et réduisez le
jus à point.

Épluchez les pommes de terre, lavez-les, coupez-les en quartiers, mettez-les
dans une casserole avec le reste de la graisse d'oie, les échalotes, de l'ail
et du persil hachés fin, du sel, du poivre, mouillez avec de l’eau jusqu'au
niveau de la partie supérieure des pommes de terre et faites cuire à gros
bouillons pendant une demi-heure.

Mettez les saucisses dans de l'eau salée bouillante, éloignez la casserole du
feu, mais gardez-la à une température juste au-dessous de l'ébullition pendant
vingt minutes.

Écrasez les pommes de terre au moyen d'un pilon en bois, de façon à obtenir une
purée ferme.

Dressez la choucroute sur un plat, garnissez-la avec le bœuf fumé et les
saucisses et servez, en envoyant en même temps, mais à part, la purée de pommes
de terre dans un légumier.

\section*{\centering Timbale de choucroute.}
\phantomsection
\addcontentsline{toc}{section}{ Timbale de choucroute.}
\index{Timbale de choucroute}
\index{Choucroute en timbale}
\index{Choucroute garnie, au gibier}
\index{Choucroute garnie, à la graisse de faisan}

Pour douze personnes prenez :

\footnotesize
\begin{longtable}{rrrrrp{18em}}
  & \multicolumn{3}{r}{3 kilogrammes} & de & choucroute,                                                  \\
  & \hspace{2em} & 500 & grammes & de & lard frais,                                                       \\
  & \hspace{2em} & 150 & grammes & de & beurre,                                                           \\
  & \hspace{2em} &     &         & 36 & huîtres de Cancale,                                               \\
  & \hspace{2em} &     &         &  2 & perdreaux ou 1 faisan,                                            \\
  & \hspace{2em} &     &         &  1 & foie gras d'oie,                                                  \\
  & \hspace{2em} &     &         &  1 & bouteille de champagne demi-sec,                                  \\
  & \hspace{2em} &     &         &    & croûte de timbale,                                                \\
  & \hspace{2em} &     &         &    & sel et poivre.                                                    \\
\end{longtable}
\normalsize

Lavez la choucroute pour la dessaler ; égouttez-la.

Mettez le lard avec un peu de beurre dans une casserole en porcelaine à parois
et à fond épais, laissez-le fondre. Enlevez la casserole du feu, ajoutez la
choucroute, mouillez avec les trois quarts du champagne, salez, poivrez.
Laissez en contact pendant {\ppp24\mmm} heures, puis faites cuire au four,
à petit feu, pendant une douzaine d'heures.

Faites revenir dans une braisière perdreaux ou faisan avec du beurre, mouillez
avec le reste du champagne, assaisonnez avec sel et poivre ; laissez cuire.

Désossez le gibier, escalopez-le, passez tous les débris à la presse, réservez
le jus obtenu ainsi que la cuisson du gibier. Tenez au chaud,

Ouvrez les huîtres, recueillez-les avec leur eau dans une casserole, chauffez,
donnez quelques bouillons, passez l'eau. réduisez-la. Tenez le tout au chaud.

Coupez le foie gras en tranches, passez-les rapidement dans le reste du beurre
chaud.

Mettez dans la choucroute cuisson du foie et cuisson du gibier, jus obtenu à
la presse et eau des huîtres, mélangez bien.

Garnissez une croûte de timbale avec la choucroute, les escalopes de gibier,
les tranches de foie gras et les huîtres, le tout en couches alternées ;
achevez la cuisson au four pendant une heure environ.

Servez chaud.

Cette timbale eût fait bonne figure sur la table d'un fermier général.
