\sk

\bigskip
\bigskip

\section*{\centering Le café.}
\phantomsection
\addcontentsline{toc}{section}{ Le café.}
\index{Café (Le)}

Le café, originaire de l'Orient, fut introduit en Europe au {\sc
xvii}\textsuperscript{e} siècle.

On le prépare de trois façons principales : à l’orientale, à l'europénne, à la
brésilienne.

\sk

\index{Café noir à l'orientale}
En Orient, lorsqu'on veut préparer du café, on fait bouillir de l'eau dans un
appareil en cuivre de forme bitronconique muni d'un long manche, dont les deux
bases larges forment l'ouverture et le fond, on verse dedans du café,
préalablement grillé à la couleur marron clair et moulu en poudre presque
impalpable, en proportion avec le degré de force qu'on veut donner
à l'infusion, et on ajoute plus ou moins de sucre en poudre : par exemple
{\ppp25\mmm} grammes de café et {\ppp25\mmm} grammes de sucre pour
{\ppp200\mmm} grammes d'eau. Bientôt le mélange devient mousseux et se soulève,
on retire l'appareil du feu, la mousse tombe ; on remet l'appareil sur le feu.
la mousse se reproduit ; on la fait tomber encore et on recommence l'opération
une troisième fois. On précipite ensuite le marc au fond par l'addition d'une
cuillerée à café d'eau froide, puis on verse le liquide dans les tasses.

Le café à l'orientale, bien que clarifié, n'est cependant pas limpide ; quoi qu'il en
soit, il est très agréable au goût et beaucoup moins excitant que le café à
l'européenne.

\sk

\index{Café noir à l'européenne}
La préparation du café à l'européenne demande des soins particuliers. Peu de
personnes savent bien le faire. Brillat-Savarin écrivait en {\ppp1825\mmm} que,
après avoir essayé tous les procédés employés de son temps, il s'était arrêté
à celui qui consiste à faire passer de l’eau bouillante sur du café, puis
à reprendre cette première infusion et la faire repasser un certain nombre de
fois sur le marc, en la faisant bouillir chaque fois. Il assurait qu'en opérant
ainsi, il obtenait un café aussi bon que possible ! Il existe encore des
appareils basés sur ce principe.

Il y a de cela une vingtaine d'années, dans un de mes voyages, j'ai vu apprêter
le café de la manière suivante : on commençait par carboniser les grains de
café (j'emploie à dessein le mot carboniser, car le café était littéralement
transformé en charbon), puis les grains carbonisés étaient mis tout entiers
dans de l'eau qu'on faisait bouillir jusqu'à ce qu'elle ait pris une coloration
foncée, et plus la mixture était noire meilleure on la trouvait. À la vérité,
pour tout dire, cela se passait dans un pays où les huîtres sont d'autant plus
estimées qu'elles sont plus grosses !

La préparation du café n'est pourtant pas très compliquée : il suffit de
raisonner les différents temps de l'opération : 1° choix du café ; 2°
torréfaction des grains ; 3° mouture ; 4° infusion du café moulu. Je vais les
passer en revue.

\medskip

\index{Choix du café}
1° \textit{Choix du café}. — Les variétés de café sont très nombreuses ;
beaucoup d'entre elles sont bonnes et elles ont chacune leur note propre comme
couleur, comme parfum, comme propriétés excitantes. Aussi, pour réunir le plus
possible de qualités dans l'infusion, il est bon d'employer un mélange, et
celui d'un tiers de moka, un tiers de bourbon, un tiers de martinique est
excellent.

\medskip

2° \textit{Torréfaction}. — La torréfaction, qui a pour but de provoquer la
manifestation des huiles aromatiques essentielles que renferment les grains de
café, ne présente aucune difficulté ; elle ne demande que du soin.

Lorsqu'on emploie un mélange de cafés de grosseurs différentes, comme celui que
je viens d'indiquer, il est essentiel de faire griller chaque qualité à part,
sans quoi l’une risquerait d'être brûlée quand l’autre ne serait pas encore
torréfiée à point ; l'opération doit être arrêtée lorsque les grains, colorés
en marron foncé, suintent et dégagent leur huile.

La torréfaction doit précéder immédiatement la mouture.

\medskip

3° \textit{Mouture}. — On a beaucoup discuté autrefois sur les avantages qui
résulteraient du bocardage des grains, par rapport au broyage au moulin. En
réalité, ce qui importe c'est le degré de finesse de la poudre : plus la poudre
est fine, plus nombreux sont les contacts avec l'eau dans l'opération de
l'infusion.

La mouture doit précéder immédiatement l'infusion.

\medskip

4° \textit{Infusion}. — L'infusion doit être faite dans des appareils
absolument inattaquables, de façon qu'elle ne prenne aucun goût étranger ; les
cafetières en terre vernissée, en porcelaine, en argent, etc., sont
incontestablement celles qui remplissent le mieux celle condition.

Une cafetière se compose de deux parties ; la partie supérieure, cylindrique,
munie d'un couvercle, terminée en bas par un filtre, reçoit la poudre de café ;
la partie inférieure reçoit l'infusion.

Les trous du filtre doivent être d'autant plus petits et d'autant plus nombreux
que le café est moulu plus fin et le diamètre des trous doit naturellement être
inférieur à celui des grains de poudre.

La colonne de café doit être aussi homogène que possible afin d'éviter les fissures
par lesquelles l'eau passerait trop vite, et son tassement ne doit pas être exagéré.
On arrive à ce résultat en commençant par imbiber la poudre mise dans la cafetière
avec un peu d'eau chaude, qui fait gonfler la masse et l'homogénéise.

La hauteur du café dans la cafetière n'est pas indifférente. Si elle est trop
petite, l'infusion ne sera pas suffisamment aromatisée après un seul passage
quelle que soit la quantité de café employée, et le procédé qui consiste
à repasser plusieurs fois l'infusion en la faisant bouillir à chaque passage,
comme le préconisait Brillat-Savarin, est absolument condamnable, car chaque
ébullition fait évaporer une certaine quantité d'essences volatiles. Si la
hauteur du café est trop grande, le liquide arrive refroidi dans les parties
inférieures de la colonne qui sont alors mal utilisées.

Étant donné un certain tassement de la poudre, il y a toujours une hauteur
optima qu'il convient de déterminer expérimentalement. Pour un tassement que je
qualifierai de moyen, faute d'un terme plus précis, j'ai trouvé que la hauteur
la meilleure est d'environ {\ppp8\mmm} centimètres ; en l'adoptant, j'ai
toujours obtenu une infusion très aromatisée et le marc était sensiblement
épuisé. Aussi, quel que soit le nombre des tasses à préparer, la hauteur de la
colonne doit toujours être la même, seul le diamètre de la cafetière doit
varier. C'est là un principe fondamental.

Combien de café convient-il d'employer par tasse ? Cela dépend évidemment de
la nature du café et du degré de force que l'on désire donner à l'infusion. Avec le
mélange ci-dessus indiqué, j'estime que pour avoir du bon café, très fort, il faut
{\ppp20\mmm} grammes de café torréfié par tasse,

\index{Équation de la cafetière}
L'équation qui donnera le diamètre D convenable pour le réservoir cylindrique
d'une cafetière destinée à faire un certain nombre \textit{n} de tasses est la
suivante :

\begin{center}
$ \frac{H\Delta \pi D^{2}}{4} = 20 n $
\end{center}

\normalsize

dans laquelle H représente la hauteur optima exprimée en centimètres pour une
densité Δ, ces deux éléments devant être déterminés expérimentalement.

\medskip

D'où le diamètre exprimé en centimètres

\begin{center}
$ D = \sqrt{\frac{80n}{H\Delta\pi}} $
\end{center}

Quant à la hauteur du récipient cylindrique, il suffit qu'elle soit de quelques
centimètres supérieure à H pour tenir compte du foisonnement.

Si l’on ne veut pas s'astreindre à faire fabriquer une gamme de cafetières
correspondant exactement aux données théoriques, il est bon de choisir au moins
des cafetières du commerce dont les diamètres soient voisins de ceux qui
résulteraient de l'application de la formule.

\label{pg1014} \hypertarget{p1014}{}
L'eau servant à l'infusion doit être aussi pure que possible et à une
température voisine de {\ppp100\mmm}°, de façon à dissoudre le maximum
d’aromes, mais légèrement au-dessous, car les huiles essentielles sont assez
volatiles et à {\ppp100\mmm}° il s'en évaporerait une grande partie. En
pratique, il convient d'employer de l'eau bouillante et de la verser sur le
café, petit à petit, par cuillerées à bouche ; au contact du café, sa
température baisse immédiatement de quelques degrés.

\textit{En résumé, torréfiez et moulez fin le café au moment de vous en
servir ; ayez une cafetière de dimensions adéquates au nombre de tasses que
vous voulez préparer ; tassez la poudre uniformément ; commencez par l'humecter
d'eau chaude puis versez l’eau bouillante destinée à l'infusion par très
petites quantités}.

Le récipient qui reçoit l'infusion doit être tenu au bain-marie presque
bouillant ; de la sorte, le café restera très chaud sans atteindre
{\ppp100\mmm}°, ce qui convient.

\sk

\index{Café noir à la brésilienne}

Dans la préparation du café à la brésilienne, les grains sont moulus aussi fin
que possible, puis la poudre est délayée avec un peu d'eau bouillante et la
bouillie obtenue est mise dans une chausse en coton, à mailles très serrées. On
verse ensuite, petit à petit, sur cette bouillie la quantité d'eau bouillante
en rapport avec la quantité de café et le nombre de tasses qu'on veut avoir
({\ppp100\mmm} grammes de café pour {\ppp1\mmm} litre d'eau).

Avant d'employer une chausse pour la première fois, il est bon de la faire
tremper pendant quelque temps dans de l’eau bouillante, pour lui enlever son
odeur et son goût,

\sk

\index{Café à la crème}
Pour préparer du café à la crème, on pourra employer l’une quelconque de ces
méthodes. Au lieu d'eau, on prendra du lait pour faire l'infusion et, pour
finir, on ajoutera de la crème à volonté.

\index{Café à la viennoise}
Le café à la crème apprêté de la sorte est meilleur que celui que l'on obtient
en mouillant de lait ou de crème du café noir à l'eau.
Le café à la viennoise est une crème au café garnie de crème fouettée,

\sk

Le café est stimulant et tonique ; mais ses effets varient un peu avec les
climats.

D'une façon générale, on peut dire que le café, pris même en grande quantité,
ne produit généralement aucun inconvénient dans les pays chauds, surtout
lorsqu'il est préparé à la mode orientale. Dans nos climats, au contraire, ce
n'est qu'à la condition d'en prendre à dose modérée que ses effets sont
salutaires ; à forte dose, il produit chez beaucoup de personnes une excitation
nerveuse pouvant amener de l'insomnie et déterminer même, à la longue, des
troubles circulatoires. Il est donc bon de ne pas en abuser.

\section*{\centering Le thé.}
\phantomsection
\addcontentsline{toc}{section}{ Le thé.}
\index{Thé (Le)}

Le thé, introduit en Europe dans la seconde moitié du xvii\textsuperscript{e}
siècle, ne fut guère employé en France pendant longtemps que comme tisane ;
mais depuis {\ppp1870\mmm} son usage s’est répandu. Cependant, beaucoup de
personnes ne savent pas le préparer.

On peut diviser les thés en deux grandes classes : les thés noirs et les thés
verts. Les premiers sont fermentés et moins âcres que les seconds.

\index{Choix du thé}
Il est impossible de dire quels sont les meilleurs thés ; c'est une question de
goût. Les espèces, les variétés et les mélanges de thés sont nombreux ; les
thés noirs de la Chine sont incontestablement les plus agréables et l'on
obtient un très bon mélange avec {\ppp50\mmm} pour {\ppp100\mmm} de Sou-Chong
de 1\textsuperscript{re} qualité et 50 pour 100 de Pé-Kao orange.

La quantité de thé à employer dépend naturellement de la force qu'on désire
donner à l'infusion. Beaucoup de personnes confondent la force avec la
coloration. La force du thé dépend exclusivement du pourcentage des alcaloïdes
dissous, et l'intensité de la coloration dépend de la durée de l'infusion.
Forte ou légère, l'infusion est d'autant meilleure et hygiénique que sa couleur
est plus claire. En réduisant la durée de l'infusion, on obtient une boisson
à la fois peu colorée et peu chargée en tannin. Or, le tannin a une action
défavorable sur l'estomac.

Pour faire avec le mélange indiqué du bon thé de force moyenne, {\ppp10\mmm}
grammes suffisent pour un litre. En ce qui concerne la préparation de
l'infusion, prenez de préférence une théière en porcelaine ou en argent,
chauffez-la, versez dedans la quantité de thé, nettové par un lavage rapide,
correspondant à la quantité de breuvage que vous voulez faire, versez dessus de
l'eau\footnote{J'ai déjà noté, pp. \hyperlink{p0199}{\pageref{pg0199}} et
\hyperlink{p1014}{\pageref{pg1014}}, l'avantage que l'on a à employer de
l'eau aussi pure que possible. Cet avantage est maximum quand il s'agit de
substances possédant un arome aussi subtil que celui du thé. A qualité égale de
thé, on obtiendra l'infusion la plus agréable et la meilleure avec de l'eau
distillée. \protect

Disons un mot du samovar, que certaines personnes croient indispensable pour
faire du bon thé. \protect

Cet appareil, qui consiste en une chaudière verticale tubulaire, en cuivre,
à foyer central, sert en Russie à la production de l’eau bouillante destinée
à l'infusion ; il ne présente qu'un seul avantage, dû à la position du robinet
de prise placé à une certaine hauteur au-dessus du fond, celui d'épurer les
eaux employées qui, dans le pays, sont fréquemment troubles. Dans tout autre
cas, il est absolument inutile. Il est du reste toujours mauvais de l'employer
dans les appartements autrement que sous une hotte de cheminée, car il dégage
dans l'atmosphère quantité d'oxyde de carbone.} bouillante, agitez avec une
cuiller, couvrez la théière, laissez infuser pendant une minute, au maximum, et
servez immédiatement, en passant la boisson au travers d'une passoire.

\sk

Il existe actuellement dans le commerce des théières à compartiment spécial
destiné à recevoir le thé, qui permettent de ne le laisser en contact avec
l'eau que le temps strictement nécessaire à l'infusion.

\sk

Pour la préparation du thé au lait ou à la crème, on peut prolonger un peu
l'infusion, l'âcreté du breuvage étant atténuée par le lait ou par la crème. On
peut aussi employer pour cet usage des thés autres que les thés noirs de Chine,
par exemple des thés noirs de Ceylan.

\sk

Enfin, en variant la qualité et la quantité du thé et en le faisant infuser
directement dans du lait, on obtient du lait aromatisé qu'il est facile de
doser au goût de chacun.

\sk

Comme le café, le thé, pris à doses modérées, stimule les fonctions
cérébrales ; il rend l'intelligence plus vive et plus lucide ; il active la
circulation et les sécrétions. A doses élevées, il provoque de l'agitation et
de l'insomnie.

\section*{\centering Les liqueurs.}
\phantomsection
\addcontentsline{toc}{section}{ Les liqueurs.}
\index{Liqueurs (Les)}

A la fin de tout repas bien ordonné, on a l'habitude d'offrir des liqueurs,
après le café. Les hygiénistes sévères s'élèvent contre cet usage ; mais, sil
est incontestable que l'abus des liqueurs est funeste à la santé, il me paraît
également certain qu'elles ne présentent aucun danger tout en procurant du
plaisir, à la condition de n'en pas prendre trop souvent et de n'en absorber
qu'à dose modérée. Or, je prétends que le volume d'un de à coudre de liqueur
suffit amplement si l'on sait en tirer parti, et, dans ces conditions, je ne
crois pas sans intérêt de dire un mot sur la meilleure manière de les goûter.

\index{Choix des liqueurs}
Et d'abord, quelles sont les meilleures liqueurs ?

Comme liqueurs fortes, rien ne vaut assurément les vieilles fines champagnes,
malheureusement assez rares et très coûteuses ; à leur défaut, les vieux
armagnacs authentiques, beaucoup plus abordables comme prix, sont très
recommandables. Mais, à toute eau-de-vie tripatouillée, quelque étoilée qu'elle
soit, de quelque titre pompeux qu'on la décore, je préfère de beaucoup des
alcools inférieurs, tels les alcools de fruits, les rhums, l'hydromel et même
de vulgaires alcools de grains, comme le whisky, à condition qu'ils soient
légitimes et d'âge canonique.

Pour les liqueurs douces, le premier rang appartient à mon avis aux vieilles
chartreuses qui, hélas ! tendent à disparaître. A leur place, nous devons nous
contenter d'anisettes et de curaçaos de bonnes marques.

\index{Comment je conseille de prendre les liqueurs}
Mais quelle que soit la liqueur préférée, il importe d'en obtenir le meilleur
rendement, étant donné qu'il convient d'en réduire au minimum la quantité. Pour
arriver à ce résultat, les différents sens compétents : la vue, l'odorat et le
goût doivent être mis à contribution.

La couleur d'une liqueur n'est assurément qu'une qualité secondaire et facile
à obtenir : cependant, elle dispose bien l'amateur, et, pour ce motif, j'estime
qu'il convient de servir toujours les liqueurs, comme tous les liquides
à boire, du reste, dans des verres minces et incolores.

Passons à l'odorat. Toutes les liqueurs, et en particulier les liqueurs fortes,
doivent être portées à une température convenable pour dégager leurs parfums.
Les petits verres dont on se sert habituellement sont mal conçus et il est
infiniment préférable de prendre des verres en forme de tulipe, d'une
contenance beaucoup plus grande, tels que ceux qu'emploient, à bon escient, les
dégustateurs de profession. Leur contenu est facile à chauffer convenablement
au contact de la paume de la main ; de plus, en imprimant au verre un mouvement
giratoire, on favorise le dégagement des effluves parfumés, qu'il ne reste plus
qu'à aspirer. C'est à, du reste, une épreuve à laquelle ne saurait résister
aucune liqueur frelatée ; aussi, à ce point de vue seul, l'emploi des verres en
tulipe mérite de se généraliser.

Toute bonne liqueur doit réjouir l’odorat avant de flatter le palais. Une fois
que les organes olfactifs sont bien imprégnés des aromes, mais alors seulement,
goûtez la liqueur. Voici le modus operandi que je recommande. Portez le verre
à vos lèvres, laissez tomber quelques gouttes de son contenu sur le plancher de
votre bouche, sous la langue, à la base du frein, fermez les yeux, tâchez de
vous abstraire et attendez. Vous n'attendrez pas longtemps : une salivation
abondante se produira bientôt, deux courants de salive parfumée s'élèveront en
bouillonnant des deux côtés de votre bouche et empliront vos joues ;
laissez-les aller à la rencontre l'un de l'autre ; lorsqu'ils seront réunis,
masquant complètement votre langue, soulevez-en la pointe vers le palais et
avalez lentement.

Le contenu d'un dé à coudre, pris ainsi en plusieurs fois, suffira pour procurer,
sans pouvoir nuire, tout le plaisir qu'une bonne liqueur est capable de donner.
