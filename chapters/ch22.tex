\index{Champignons}
\index{Champignons (Différentes manières d'accommoder les)}
\index{Champignons frits}

Les champignons peuvent être grillés, sautés, frits ; cuits sur le plat ; en
daube ; accommodés au beurre, à l'huile, à la crème, au jus ; à différentes
sauces dont le fond sera du vin, de la crème. de la glace de viande, du fumet
de gibier ; à la béchamel, à la poulette, à la sauce Mornay, à la sauce
financière, à la sauce suprême ; en matelote ; farcis ; au gratin ; apprêtés en
fricassées, en ragoûts, en hachis, en purées. On peut les faire entrer dans des
potages, des omelettes, des œufs brouillés, des plats de poissons, des plats de
viandes, des pâtés, des farces ; les présenter dans des croustades, des
vol-au-vent, des flans, des tartelettes ; les employer comme condiment, comme
garniture ; les servir en hors-d'œuvre ; en salade ; enfin en faire des
conserves.

Avant de se servir des champignons, il est bon d'enlever dans certaines espèces
l'épiderme du chapeau, les feuillets, les tubes ; souvent aussi il convient d'en
rejeter le pied lorsqu'il est coriace. On les lave ensuite et on les sèche dans un
linge.

Les champignons ne doivent pas être cuits longtemps, car ils perdent alors
une grande partie de leur saveur.

\medskip

\index{Champignons grillés}
Les champignons à chair épaisse, tels que les Amanites, les Tricholomes, les
Pleurotes, les Psalliotes, les Bolets, les Russules, les Lactaires, certains Pholiotes,
les Champignons de couche peuvent être grillés entiers directement sur le gril ou
en caisses beurrées, et aromatisées avec de l'ail, de l'échalote, des fines herbes.
Ils seront servis soit avec un beurre maitre d'hôtel, soit avec un beurre d'escargots
(à la bourguignonne), ou avec un beurre composé quelconque.

\medskip

\index{Champignons sur le plat}
\index{Champignons au beurre}
\index{Champignons à l'huile}
Ils pourront être cuits, entiers, dans un plat de service, avec beurre,
ciboule, ail ou fines herbes, sel, poivre, quatre épices ou muscade, au goût,
masqués ensuite avec une sauce à la crème liée avec des jaunes d'œufs, ou avec
une béchamel grasse ou maigre montée au beurre fin.

\index{Champignons à la gênoise}
Une autre façon de les présenter sur le plat est la suivante : après avoir salé
légèrement les champignons du côté creux pour leur faire rendre leur excès
d'eau et les avoir essuyés ou séchés à la bouche d’un four, les disposer dans un
plat en porcelaine foncé de feuilles de vigne et de fenouil imbibés fortement
d'huile fine ; y ajouter de l'ail, de l'oignon au goût, les assaisonner avec sel,
poivre, épices ; décorer les chapeaux des champignons avec des rondelles prises
dans les pieds et faire cuire. C'est le procédé dit « à la génoise ».

\medskip

\index{Champignons sautés}
Tous les champignons peuvent être sautés ; cependant ce mode de préparation
doit plutôt être réservé aux petits champignons car ils restent entiers. Il est
bien certain que l'on peut faire sauter de gros champignons, mais l'opération
est plus délicate. Ainsi donc les Clitocybes à odeur douce ou à odeur de
farine\footnote{Quelques échantillons seulement des Cliocybes à odeur
aromatique suffisent pour parfumer un plat.}, les Laccaria, les Hygrophores,
les Clitopiles, les Pholiotes, les Marasmes, les Verpa, les petits Champignons
de couche, les petits Cèpes pourront être sautés au beurre ou à l'huile, avec
de l'ail, du persil haché, du sel, du poivre, de la muscade (à la provençale) ;
avec de l'échalote, du persil, du sel, du poivre, des épices, puis aspergés de
jus de citron (à la bordelaise) ; avec des fines herbes, du sel, du poivre,
puis mouillés avec du fond de gibier et la cuisson montée à la crème et
aromatisée avec du jus de citron ; ou encore sautés avec des émincés de
truffes, la cuisson corsée ensuite avec de la glace de viande blonde ou de la
glace de volaille, puis montée à la crème.

Une sauce au beurre avec amandes ou pistaches hachées et jus de citron
satisfera certains amateurs.

Les champignons entiers ou coupés en morceaux pourront être sautés, seuls ou
avec des émincés de truffes, dans du beurre simple ou aromatisé, puis
additionnés de madère, de porto ou de sauternes, corsés avec de la glace de
viande ou de gibier et enfin servis dans des flans ou dans des tartelettes.

Les champignons pourront encore être sautés à la barigoule avec lard, beurre
ou huile, poivre, persil et échalotes.

\medskip

Lorsqu'on voudra faire frire des champignons, on procèdera par exemple de la
manière suivante : après les avoir blanchis pendant quelques minutes dans de
l'eau ou du lait aromatisé avec un peu de zeste de citron, on les fera frire
dans de la graisse bien chaude et on les présentera sucrés ou salés. Dans ce
dernier cas, on les servira avec un beurre maitre d'hôtel, un beurre composé où
un bon jus.

\index{Champignons en daube}
On pourra apprêter les champignons en daube avec anchois, persil, ail, poivre,
vin blanc, huile et jus de citron ; ou avec olives, bouquet garni, ail ou échalote,
sel, poivre, beurre et madère ; ou encore avec des tomates et un bon fond. Ce
mode de préparation en daube conviendra fort bien pour les Clavaires, les Hydnes,
les Entolomes, les Polypores, les Cortinaires, les Collybies, les Praxilles, la
Fistuline.

\medskip

\index{Farces pour champignons, au gras}
\index{Farces pour champignons, au maigre}
\index{Champignons farcis au gras}
\index{Champignons farcis au maigre}
Tous les grands champignons à chair épaisse, Oronges, Tricholomes, Pleurotes,
Lactaires, Russules, Volvaires, Pholiotes, Psalliotes, Bolets, Champignons de
couche et les grands champignons à chair mince, Lépiotes, Pluteus, Clitocybes,
Pézizes, ainsi que les Morilles pourront être farcis ; au maigre, avec des
farces fines de crustacés, de coquillages, de poissons, de crevettes,
d’écrevisses, de champignons, d'asperges, d'artichauts, etc., puis cuits dans
un bon fond de poisson, ou un bon fond blanc corsé avec de la glace de viande
blonde ; au gras. avec du lard, du jambon, des farces fines de volaille, de
perdreau, de cailles, de grives, d'ortolans et cuits dans un bon jus ou dans un
bon fond de veau et volaille ou de gibier.

\medskip

Pour faire des champignons au gratin, il suffira le plus souvent de saupoudrer
la surface de nombre de préparations avec de la mie de pain rassis tamisée
mélangée ou non avec du gruyère, du parmesan râpés ou les deux ensemble, de
couronner le tout avec des petits morceaux de beurre frais et de pousser au
four.

On pourra encore faire des plats gratinés de champignons émincés ou hachés
recouverts de sauce Mornay, de sauce Béchamel grasse ou maigre, ou de sauce
tomate.

\medskip

Lorsqu'on voudra préparer des champignons en fricassée, une bonne façon de les
apprêter est la suivante : les faire revenir à la casserole dans du beurre, les
saupoudrer de farine, laisser cuire sans que la farine se colore, assaisonner
avec sel, poivre, épices, ajouter persil haché, jus de viande, de volaille ou
de gibier au goût, faire la liaison de la sauce avec de la crème et des jaunes
d'œufs et, au dernier moment, la relever avec un peu de vinaigre ou de jus de
citron. Les Lycoperdons, les Bovista, les Sparassis, les Chanterelles, les
Coprins, les Helvelles, les Pézizes, les Lactaires, les Craterelles, les
Armillaires, les Paxilles, les Hydnes, les Clavaires, les Polypores pourront
être accommodés ainsi.

\medskip

Les champignons à saveur fine et délicate devront être réservés pour faire des
entremets de légumes. Les Oronges, les Lépiotes, les Tricholomes, le Lactaire
délicieux, le Pleurote de l'Eryngium, les Russules, les Chanterelles, le
Pholiote du peuplier, les Psalliotes, les Lycoperdons, les Cèpes, les
Helvelles, les Coprins, les beaux champignons de couche, les Morilles, les
Terfézia accommodés au beurre, à la crème, à la béchamel, à la poulette, à la
sauce Mornay, à la sauce suprême, à la financière, en matelote, etc.
constitueront des mets délicieux qui pourront être servis tels quels ou dans
des croustades et des vol-au-vent.

\medskip

Tous les champignons peuvent entrer dans les ragoûts de viandes ; cependant,
j'estime qu'il vaut mieux réserver à cet emploi les champignons de peu de
saveur ou les champignons séchés (cèpes, morilles, gyromitres).

\medskip

Les champignons à saveur fine ou agréable trouveront encore leur emploi dans
les omelettes, les œufs brouillés, tels les morilles, les truffes, les
craterelles, etc.

\medskip

\index{Champignons en hors-d'œuvre}
Les petits champignons de couche marinés seront servis comme hors-d'œuvre,

\medskip

\index{Champignons en ragoût}
\index{Champignons en salade}
La Fistuline et le Lactaire à lait abondant, après avoir élé blanchis, pourront
être présentés crus, en salade.

\medskip

Les Bolets remplacent avantageusement les champignons de couche dans toutes
les préparations culinaires ; de même que beaucoup de champignons, ils peuvent
être servis comme entremets de légumes ou comme garniture.

\medskip

Les Morilles parfument délicieusement lous les mets dans lesquels on les fait
entrer et elles constituent des entremets de légumes hors ligne.

\medskip

Les Truffes, comme les autres champignons, sont consommées au naturel, en
ragoûts, en salades ; elles sont très fréquemment employées comme garniture et
elles entrent dans les farces les plus fines.

L'une des meilleures façons de les préparer, façon dite « à la maréchale »,
consiste, après les avoir assaisonnées avec sel et poivre, à les barder de
lard, à les envelopper ensuite séparément dans du papier, puis à les faire
cuire dans de la cendre de bois chaude ; une heure de cuisson suffit.
Lorsqu'elles sont cuites, on les débarrasse de leur enveloppe, on les essuie,
on les dresse sur un plat garni d'une serviette que l’on rabat sur elles et on
les sert telles quelles accompagnées de beurre frais. Ces truffes au naturel
sont tout simplement exquises. À défaut de cendre de bois, les truffes bardées
de lard et assaisonnées avec sel et poivre pourront être cuites à la broche ou
à la casserole.

On peut aussi faire cuire les truffes saupoudrées de sel, de poivre et
d'aromates, soit à la vapeur d'eau, soit à celle d'un mélange de vin et de fine
champagne ; ou bien dans du beurre où encore dans du vin de Madère, de Porto,
d'Alicante ou de Champagne, en les nourrissant avec un bon jus aromatisé.

On les apprête aussi en ragoûts simples où gratinés ; on les fait sauter seules
ou avec des pommes de terre ; on les sert glacées, fourrées, à la gelée, en
croustades, etc. Les pâtés de truffes à la Monglas, les timbales truffées à la
Talleyrand, les dindes, les chapons truffés font partie de la cuisine
classique.

Les truffes grises ont leur place marquée dans les risotto. Cependant, on peut
aussi, après les avoir émincées, les faire sauter dans de l'huile et achever
leur cuisson dans du jus de volaille auquel on ajoute au dernier moment un peu
de jus de citron. Enfin, on peut les servir en salade : Rossini les adorait
ainsi.

\section*{\centering Champignons de couche farcis.}
\phantomsection
\addcontentsline{toc}{section}{ Champignons de couche farcis.}
\index{Champignons de couche farcis}

Il existe bien des manières de farcir les champignons. La plupart des formules
comportent l'usage d'oignon, d'échalote ou d'ail ; en voici plusieurs qui n'en
contiennent pas et qui, pour ce motif, pourront convenir dans bien des cas.

\medskip

Pour la garniture d'un plat destiné à six personnes prenez :

\footnotesize
\begin{longtable}{rrrp{16em}}
    140 & grammes & de & petits champignons,                                                              \\
        &         &  6 & gros champignons, pesant chacun 60 grammes en moyenne,                           \\
        &         &  1 & jaune d'œuf,                                                                     \\
        &         &    & beurre,                                                                          \\
        &         &    & jus de viande,                                                                   \\
        &         &    & jus de citron,                                                                   \\
        &         &    & mie de pain rassis tamisée,                                                      \\
        &         &    & persil haché,                                                                    \\
        &         &    & quatre épices,                                                                   \\
        &         &    & curry,                                                                           \\
        &         &    & sel et poivre.                                                                   \\
\end{longtable}
\normalsize

Épluchez tous les champignons, passez-les dans du jus de citron. Enlevez les
pieds des gros champignons, réservez les chapeaux ; hachez les petits
champignons avec les pieds des gros. Préparez une farce avec le hachis de
champignons, de la mie de pain rassis tamisée imprégnée de jus de viande, du
persil haché, du beurre, le jaune d'œuf ; assaisonnez au goût avec quatre
épices, curry, sel et poivre.

Emplissez les chapeaux des gros champignons avec la farce, placez-les ensuite,
les creux en dessus, dans un plat foncé de beurre, arrosez chaque champignon
avec du jus de citron, mettez sur chacun un petit morceau de beurre et faites
cuire au four doux.

\section*{\centering Champignons de couche farcis.}
\phantomsection
\addcontentsline{toc}{section}{ Champignons de couche farcis.}
\index{Champignons de couche farcis}
\index{Champignons de couche farcis (autre formule)}

\begin{center}
\textit{(Autre formule).}
\end{center}

Pour six personnes prenez :

\footnotesize
\begin{longtable}{rrrp{16em}}
    300 & grammes & de & foie gras de canard,                                                             \\
    200 & grammes & de & fond de veau,                                                                    \\
    150 & grammes & de & jambon de Bayonne,                                                               \\
    150 & grammes & de & beurre clarifié,                                                                 \\
        &         &  6 & gros champignons de couche, pesant chacun 6o grammes environ,                    \\
        &         &    & vin blanc,                                                                       \\
        &         &    & glace de viande,                                                                 \\
        &         &    & jus de citron,                                                                   \\
        &         &    & paprika,                                                                         \\
        &         &    & sel et poivre.                                                                   \\
\end{longtable}
\normalsize

Mettez dans le fond de veau du vin blanc et de la glace de viande ; réduisez
à bonne consistance.

Pelez les champignons, passez les chapeaux dans du jus de citron ; mettez-les
dans le beurre clarifié tenu tiède ; laissez en contact pendant une demi-heure.

Préparez une farce avec les pieds des champignons hachés et cuits dans du
beurre, le jambon haché, du paprika, du sel et du poivre au goût. Tenez au
chaud.

Faites cuire le foie gras au naturel, \hyperlink{p0592}{p. \pageref{pg0592}}.

Faites griller les chapeaux des champignons sur un gril à feu dessus pendant
cinq minutes de chaque côté, assaisonnez-les avec sel et poivre et arrosez-les
avec le beurre clarifié pendant leur cuisson.

Coupez le foie gras en tranches de la dimension des chapeaux de champignons ;
foncez chaque chapeau avec une tranche de foie gras, mettez dessus une couche
de farce, masquez-les avec le fond de veau chaud et servez.

C'est un plat agréable, dont l'onctuosité flatte le palais.

\section*{\centering Champignons de couche farcis, gratinés.}
\phantomsection
\addcontentsline{toc}{section}{ Champignons de couche farcis, gratinés.}
\index{Champignons de couche farcis, gratinés}

Pour quatre personnes prenez :

\footnotesize
\begin{longtable}{rrrp{16em}}
    500 & grammes & de & champignons de couche aussi gros que possible,                                   \\
    350 & grammes & de & très bon fond de veau,                                                           \\
    250 & grammes & de & crevettes grises,                                                                \\
     50 & grammes & de & beurre,                                                                          \\
     25 & grammes & de & gruyère, parmesan ou chester, râpé fin,                                          \\
      5 & grammes & de & farine,                                                                          \\
        &         &  1 & bottillon de pointes d’asperges,                                                 \\
        &         &    & sel et poivre.                                                                   \\
\end{longtable}
\normalsize

Pelez les champignons, séparez les pieds des chapeaux ; lavez pieds et chapeaux
dans un peu d'eau acidulée.

Hachez les pieds, mettez-les dans {\ppp50\mmm} grammes de fond de veau ;
laissez cuire.

Coupez les pointes d’asperges en petits morceaux, blanchissez-les dans de l'eau
salée.

Décortiquez les crevettes.

Avec les parures et le beurre, préparez un beurre de crevettes.

Hachez les pointes d'asperges avec les queues de crevettes, mélangez-les au
hachis de champignons, goûtez, salez et poivrez s'il est nécessaire ; cela
constituera la farce.

Faites cuire les chapeaux des champignons pendant quelques minutes dans le
reste du fond de veau ; disposez-les ensuite, les creux en l'air, dans un plat
allant au feu ; emplissez-les de farce, mettez par-dessus du fromage râpé et
les deux tiers du beurre de crevettes. Poussez au four ; arrosez pendant la
cuisson.

Dressez ensuite les champignons sur un plat de service ; tenez-les au chaud.

Maniez la farine avec le tiers restant du beurre de crevettes, mouillez avec le
reste de la cuisson des champignons ; concentrez la sauce.

Glacez les champignons avec cette sauce et servez.

\medskip

On peut servir ces champignons farcis comme hors-d'œuvre chaud, comme entremets
de légumes, comme garniture.

\section*{\centering Purée de champignons.}
\phantomsection
\addcontentsline{toc}{section}{ Purée de champignons.}
\index{Purée de champignons}

Passez des champignons crus, bien nettoyés, au travers d'un gros tamis ; mettez
la purée dans une sauteuse avec de la sauce Béchamel grasse, du sel, du poivre,
de la muscade, au goût, chauffez à feu vif, réduisez à bonne consistance,
finissez avec un peu de beurre frais et servez.

\sk

On pourra préparer de même une purée de truffes.

\section*{\centering Cèpes grillés.}
\phantomsection
\addcontentsline{toc}{section}{ Cèpes grillés.}
\index{Cèpes grillés}
\label{pg0805} \hypertarget{p0805}{}

Pour six personnes prenez :

\footnotesize
\begin{longtable}{rrrrrp{18em}}
  & \hspace{2em} & 125 & grammes & de & beurre clarifié,                                                  \\
  & \hspace{2em} &  10 & grammes & de & sel blanc,                                                        \\
  & \hspace{2em} &   4 & grammes & de & persil haché,                                                     \\
  & \multicolumn{3}{r}{4 décigrammes} & de & poivre,                                                      \\
  & \hspace{2em} &     &         &  6 & beaux cèpes bien fermes, pesant ensemble 1 kilogramme environ,    \\
  & \hspace{2em} &     &         &  6 & gousses d'ail, pesant chacune 4 grammes en moyenne.               \\
\end{longtable}
\normalsize

Nettovez les cèpes, essuyez-les, enlevez les tubes, rognez les pieds à la
hauteur des chapeaux.

Coupez chaque gousse d'ail en quatre, piquez chaque chapeau de cèpe de quatre
morceaux d'ail, puis mettez-les pendant une demi-heure dans le beurre clarifié
tenu tiède.

Disposez les cèpes sur un gril à feu dessus et chauffé au préalable ;
faites-les cuire pendant cinq minutes de chaque côté en les assaisonnant avec
le sel et le poivre et en les arrosant avec le beurre alliacé, puis retirez les
morceaux d'ail.

Dressez les chapeaux de cèpes sur un plat, saupoudrez-les de persil haché et
servez.

Le produit est parfait.

\section*{\centering Cèpes au chester.}
\phantomsection
\addcontentsline{toc}{section}{ Cèpes au chester.}
\index{Cèpes au chester}
\index{Cèpes au chester, au gruyère ou au parmesan}

Faites sauter de petits cèpes jeunes et frais dans un peu d'huile d'olive,
mettez-les ensuite dans un plat allant au feu, masquez-les avec une béchamel
grasse dans laquelle vous aurez incorporé du chester et poussez au four pour
gratiner.

\sk

On pourra apprêter de même des cèpes au gruyère, au parmesan, etc.

\section*{\centering Cèpes à la bordelaise.}
\phantomsection
\addcontentsline{toc}{section}{ Cèpes à la bordelaise.}
\index{Cèpes à la bordelaise}

Pour six personnes prenez :

\footnotesize
\begin{longtable}{rrrrp{16em}}
  &   1 000 & grammes & de & cèpes frais, de dimensions moyennes et autant que possible égaux,            \\
  &     125 & grammes & d' & huile d'olive,                                                               \\
  &      20 & grammes & d' & échalotes hachées,                                                           \\
  &      14 & grammes & de & sel blanc,                                                                   \\
  &      10 & grammes & de & persil haché,                                                                \\
  &       4 & grammes & d' & ail râpé,                                                                    \\
  & \multicolumn{2}{r}{4 décigrammes} & de & poivre,                                                      \\
  &         &         &    & jus de citron.                                                               \\
\end{longtable}
\normalsize

Nettoyez les cèpes, essuyez-les, enlevez les tubes, réservez les chapeaux.

Pelez les pieds, hachez-les, réservez-les.

Faites chauffer l'huile dans une poêle, mettez dedans les chapeaux des
cèpes du côté creux ; au bout de cinq minutes de cuisson, salez, poivrez,
puis retournez-les, continuez la cuisson pendant cinq minutes, salez, poivrez
encore, ajoutez ensuite le hachis de cèpes et laissez cuire ensemble pendant cinq
autres minutes. Mettez alors les échalotes ; an bout de deux minutes de cuisson,
ajoutez l'ail et le persil, faites cuire encore pendant une minute, salez, poivrez
de nouveau.

Dressez les cèpes sur un plat chaud, versez dessus l'huile et le hachis, aspergez
de jus de citron et servez.

\medskip

Lorsque, faute de cèpes frais, on emploie des cèpes conservés en boîte, il
faut, avant de s'en servir, les égoutter, puis les laver à l'eau bouillante ;
on opère ensuite comme avec des cèpes frais.

\medskip

Je conseille aux personnes n'ayant pas un estomac robuste les deux
modifications suivantes.

1° suppression des échalotes ;

2° remplacement des {\ppp4\mmm} grammes d'ail râpé par une gousse d'ail entière
de même poids, qu'on mettra dans l'huile en même temps que les cèpes et qu'on
retirera au bout de dix minutes.

Le reste de la formule ne change pas.

\sk

Voici une variante que je préfère. Au lieu des {\ppp125\mmm} grammes d'huile
d'olive, j'emploie un mélange de {\ppp100\mmm} grammes de beurre clarifié et
{\ppp12\mmm} grammes d'huile d'olive. Le mode opératoire reste le même ; ce ne
sont plus à la vérité des cèpes à la bordelaise que l'on obtient ainsi, mais
c'est très bon tout de même.

\section*{\centering Cèpes au gratin.}
\phantomsection
\addcontentsline{toc}{section}{ Cèpes au gratin.}
\index{Cèpes au gratin}

Pour six personnes prenez :

\footnotesize
\begin{longtable}{rrrrp{16em}}
  &     100 & grammes & de & beurre,                                                                      \\
  &     100 & grammes & de & fromage de Gruyère râpé,                                                     \\
  &      50 & grammes & de & mie de pain rassis tamisée,                                                  \\
  &      30 & grammes & d' & huile d'olive,                                                               \\
  &      30 & grammes & de & lait,                                                                        \\
  &      15 & grammes & de & chapelure,                                                                   \\
  &      12 & grammes & de & sel,                                                                         \\
  &       4 & grammes & de & persil haché,                                                                \\
  &       4 & grammes & d' & ail râpé,                                                                    \\
  &       3 & grammes & de & moutarde en poudre,                                                          \\
  & \multicolumn{2}{r}{1/2 gramme}   & de & poivre,                                                       \\
  & \multicolumn{2}{r}{1/2 gramme}   & de & muscade râpée,                                                \\
  &         &         &  5 & beaux cèpes pesant ensemble 1 kilogramme environ.                            \\
\end{longtable}
\normalsize

Nettoyez les cèpes, essuyez-les soigneusement, détachez les chapeaux ;
disposez-les dans un plat, la face brune en l'air, et mettez-les au four
moyennement chaud pendant vingt minutes pour leur faire suer leur excès d'eau.

Hachez les pieds des champignons, ajoutez l'ail, le persil, la mie de pain
trempée dans le lait, la moutarde, la muscade, l'huile, le sel, le poivre ;
mélangez le tout intimement.

Prenez un plat à gratin, beurrez-le, foncez-le avec une couche de la farce que
vous venez de préparer, ajoutez un peu de gruyère, posez dessus les chapeaux
des cèpes, et sur chacun d'eux mettez le sixième du beurre qui reste,
saupoudrez de gruyère, couvrez avec le reste de la farce et terminez par le
mélange du reste du gruyère et de la chapelure.

Poussez au four doux, pendant une demi-heure, puis faites dorer, pendant un
quart d'heure, à feu plus vif.

\section*{\centering Cèpes à la crème.}
\phantomsection
\addcontentsline{toc}{section}{ Cèpes à la crème.}
\index{Cèpes à la crème}

Pour six personnes prenez :

\footnotesize
\begin{longtable}{rrrrp{16em}}
  &   1 000 & grammes & de & tout petits cèpes, jeunes et frais,                                          \\
  &     125 & grammes & de & crème,                                                                       \\
  &      80 & grammes & de & beurre,                                                                      \\
  &      60 & grammes & de & fenouil frais en branches,                                                   \\
  &      30 & grammes & de & lait,                                                                        \\
  &      30 & grammes & de & sel gris,                                                                    \\
  &      25 & grammes & de & glace de viande,                                                             \\
  &      15 & grammes & de & farine,                                                                      \\
  &      15 & grammes & de & vinaigre doux de vin,                                                        \\
  &      10 & grammes & de & sel blanc,                                                                   \\
  & \multicolumn{2}{r}{3 décigrammes} & de & poivre.                                                      \\
\end{longtable}
\normalsize

Essuyez les cèpes sans les éplucher, parez-en les pieds, puis plonger-les
pendant quelques secondes dans deux litres d'eau bouillante dans laquelle vous
aurez mis le sel gris et le vinaigre.

Retirez-les, posez-les sur un tamis et laissez-les bien égoutter.

Mettez dans une casserole {\ppp60\mmm} grammes de beurre manié avec la farine,
délayez avec le lait, ajoutez {\ppp60\mmm} grammes de crème, un bouquet de
{\ppp50\mmm} grammes de fenouil, le sel blanc, le poivre, chauffez, donnez un
bouillon, mettez les cèpes, puis faites cuire jusqu'à ce que le jus de cuisson
soit convenablement concentré, ce qui demande {\ppp20\mmm} à {\ppp30\mmm}
minutes.

Dressez les champignons sur un plat, enlevez le bouquet, mettez dans la sauce
le reste du beurre, le reste de la crème et la glace de viande, chauffez,
versez cette sauce sur les cèpes, saupoudrez avec le reste du fenouil haché et
servez.

\sk

On peut préparer de même des champignons de couche.

\section*{\centering Cèpes à la limousine.}
\phantomsection
\addcontentsline{toc}{section}{ Cèpes à la limousine.}
\index{Cèpes à la limousine}

Pour six personnes prenez :

\footnotesize
\begin{longtable}{rrrp{16em}}
    500 & grammes & de & lard de poitrine,                                                                \\
    100 & grammes & de & carde de bette blonde ou de cardan,                                              \\
    100 & grammes & de & mie de pain rassis tamisée,                                                      \\
     10 & grammes & de & persil haché,                                                                    \\
        &         &  6 & cèpes fermes, pesant ensemble 500 grammes environ,                               \\
        &         &  6 & pommes de terre de Hollande pesant ensemble 500 grammes environ,                 \\
        &         &    & barde de lard,                                                                   \\
        &         &    & ail,                                                                             \\
        &         &    & muscade ou quatre épices,                                                        \\
        &         &    & sel et poivre.                                                                   \\
\end{longtable}
\normalsize

Lavez les cèpes, essuyez-les, enlevez-en les tubes ; séparez les chapeaux des
pieds ; hachez ces derniers.

Échaudez le lard pour en enlever l'excès de sel ; essuyez-le, hachez-le et faites-le
revenir à la poêle.

Blanchissez à l'eau bouillante la carde de bette ou de cardon ; hachez-la.

Réunissez lard de poitrine, hachis de cèpes, de bette ou de cardon, mie de pain
tamisée, persil haché ; ajoutez un peu d'ail haché, du sel et du poivre au
goût ; mélangez bien ; cela constituera la farce.

Pelez les pommes de terre ; creusez-les suffisamment pour pouvoir les farcir ;
emplissez les vides avec de la farce.

Passez au tamis la pulpe extraite des pommes de terre, pressez la pour en
extraire l'excès d'eau, mélangez-en une quantité plus ou moins grande avec le
reste de la farce, assaisonnez avec sel, poivre, muscade ou quatre épices et
garnissez avec le mélange les chapeaux des champignons.

Foncez une braisière avec une barde de lard, disposez dessus les pommes de
terre farcies et les chapeaux de cèpes, les creux en l'air, couvrez et faites braiser
au four doux pendant trois heures. Arrosez durant la cuisson.

Dressez cèpes et pommes de terre sur un plat en les alternant, masquez avec le
jus dégraissé et servez.

\section*{\centering Fricassée de cèpes au parmesan.}
\phantomsection
\addcontentsline{toc}{section}{ Fricassée de cèpes au parmesan.}
\index{Fricassée de cèpes au parmesan}

Pour six personnes prenez :

\footnotesize
\begin{longtable}{rrrrp{16em}}
  &   1 000 & grammes & de & tout petits cèpes jeunes et frais,                                           \\
  &     100 & grammes & de & beurre,                                                                      \\
  &      20 & grammes & de & glace de viande,                                                             \\
  &      20 & grammes & de & parmesan râpé,                                                               \\
  &      10 & grammes & de & sel,                                                                         \\
  & \multicolumn{2}{r}{5 décigrammes}  & de & poivre,                                                     \\
  & \multicolumn{2}{r}{5 centigrammes} & de & quatre épices.                                              \\
\end{longtable}
\normalsize

Nettoyez et essuyez les cèpes, coupez-les en tranches d'un centimètre
d'épaisseur.

Mettez dans une sauteuse beurre, cèpes, sel, poivre, quatre épices et faites
cuire à grand feu pendant une vingtaine de minutes. Quand le liquide provenant
de la cuisson des cèpes sera suffisamment réduit, ajoutez la glace de viande,
laissez-la se dissoudre, puis mettez le parmesan, goûtez et servez.

\sk

\index{Fricassée de champignons de couche au parmesan}
On peut préparer de même des champignons de couche.

\section*{\centering Chanterelles\footnote{Vulgairement gyroles.} à la crème.}
\phantomsection
\addcontentsline{toc}{section}{ Chanterelles à la crème.}
\index{Chanterelles à la crème}

Pour quatre personnes prenez :

\footnotesize
\begin{longtable}{rrrp{16em}}
  1 000 & grammes & de & chanterelles fraîchement récoltées,                                              \\
    200 & grammes & de & crème,                                                                           \\
     60 & grammes & de & beurre,                                                                          \\
        &         &    & sel et poivre.                                                                   \\
\end{longtable}
\normalsize

Épluchez les chanterelles, lavez-les, égouttez-les.

Mettez-les avec du sel et du poivre dans une sauteuse ; faites-les sauter
telles quelles, elles rendront leur eau tout en cuisant dedans ; l'opération
demande une vingtaine de minutes pour que tout le liquide ait disparu. Ajoutez
alors le beurre et la crème, goûtez, complétez l'assaisonnement s'il est
nécessaire ; laissez mijoter pendant quatre à cinq minutes, puis servez.

\section*{\centering Chanterelles au jus.}
\phantomsection
\addcontentsline{toc}{section}{ Chanterelles au jus.}
\index{Chanterelles au jus}

Pour quatre personnes prenez :

\footnotesize
\begin{longtable}{rrrp{16em}}
  1 000 & grammes & de & chanterelles fraîches,                                                           \\
    250 & grammes & de & fond de veau et volaille,                                                        \\
     60 & grammes & de & beurre,                                                                          \\
        &         &    & zeste de citron,                                                                 \\
        &         &    & estragon,                                                                        \\
        &         &    & sel et poivre.                                                                   \\
\end{longtable}
\normalsize

Épluchez les chanterelles, lavez-les, égouttez-les.

Mettez le beurre dans une sauteuse, ajoutez les champignons, du zeste de citron
et de l’estragon au goût (hachés ou mis dans un nouet), salez, poivrez en
tenant compte de l’assaisonnement du fond ; faites sauter jusqu'à ce que les
chanterelles, après avoir rendu leur eau, soient redevenues à peu près sèches.
Ajoutez alors le fond de veau et volaille et laissez cuire à petit feu pendant
une demi-heure environ. Retirez le nouet si vous avez mis le zeste de citron et
l'estragon entiers, puis servez.

\sk

\index{Chanterelles à la béchamel}
\index{Chanterelles à la Mornay}
\index{Chanterelles à la poulette}
Comme variantes, on pourra remplacer le fond de veau et volaille par une sauce
Béchamel maigre ou grasse, une sauce poulette, une sauce Mornay et, dans ce
dernier cas, on mettra le plat au four pour gratiner.

\section*{\centering Morilles au beurre.}
\phantomsection
\addcontentsline{toc}{section}{ Morilles au beurre.}
\index{Morilles au beurre}

Pour quatre personnes prenez :

\footnotesize
\begin{longtable}{rrrrrp{18em}}
  & \hspace{2em} & 500 & grammes & de & belles morilles grises,                                           \\
  & \hspace{2em} & 100 & grammes & de & beurre,                                                           \\
  & \hspace{2em} &   7 & grammes & de & sel,                                                              \\
  & \multicolumn{3}{r}{1/2 décigramme} & de & poivre,                                                     \\
  & \hspace{2em} &     &         &    & jus de citron.                                                    \\
\end{longtable}
\normalsize

Lavez soigneusement les morilles, essuyez-les et coupez-les en deux dans le
sens de la longueur ; mettez-les ensuite dans une casserole avec le beurre, le
sel, le poivre, du jus de citron, au goût, et faites-les sauter, à feu vif,
pendant un quart d'heure environ, jusqu'à évaporation de l'eau qu'elles ont
rendue.

\medskip

Les morilles ainsi préparées peuvent être servies seules ou comme garniture de
viandes,

\section*{\centering Morilles à la crème.}
\phantomsection
\addcontentsline{toc}{section}{ Morilles à la crème.}
\index{Morilles à la crème}

Pour quatre personnes prenez :

\footnotesize
\begin{longtable}{rrrrrp{18em}}
  & \hspace{2em} & 500 & grammes & de & morilles,                                                         \\
  & \hspace{2em} & 250 & grammes & de & crème,                                                            \\
  & \hspace{2em} & 100 & grammes & de & bouillon,                                                         \\
  & \hspace{2em} & 100 & grammes & de & beurre,                                                           \\
  & \hspace{2em} &  10 & grammes & de & farine,                                                           \\
  & \hspace{2em} &  10 & grammes & de & sel,                                                              \\
  & \multicolumn{3}{r}{3 décigramme} & de & poivre,                                                       \\
  & \hspace{2em} &     &         &    & le jus d'un citron.                                               \\
\end{longtable}
\normalsize

Nettoyez, lavez et séchez les morilles, coupez-les en deux dans le sens de la
longueur.

Faites fondre {\ppp70\mmm} grammes de beurre dans une casserole, mettez les
morilles, le sel, le poivre, mouillez avec le bouillon et le jus de citron ;
laissez cuire pendant une demi-heure. Au bout de ce temps, presque tout le
liquide doit avoir disparu ; ajoutez alors la crème, {\ppp10\mmm} grammes de
beurre manié avec la farine et continuez la cuisson à petit feu pendant dix
minutes.

Au moment de servir, montez la sauce avec le reste du beurre.

\medskip

\index{Croustade de morilles}
On peut servir les morilles à la crème dans une croûte de vol-au-vent, dans une
croustade de pain ou, plus simplement, avec des croûtons dorés dans du beurre,

\section*{\centering Morilles au jus.}
\phantomsection
\addcontentsline{toc}{section}{ Morilles au jus.}
\index{Morilles au jus}

Lavez les morilles avec soin pour les débarrasser de la terre qui se trouve
dans les alvéoles ; égouttez-les, essuyez-les, coupez-les en deux dans le sens
de la longueur, puis faites-les sauter à feu vif dans une casserole avec du
beurre, ajoutez du jus de viande et un peu de jus de citron, salez, poivrez et
laissez cuire doucement pendant une demi-heure en arrosant pendant la cuisson ;
enfin liez la sauce avec des jaunes d'œufs et servez.

C'est exquis.

\section*{\centering Morilles au vin.}
\phantomsection
\addcontentsline{toc}{section}{ Morilles au vin.}
\index{Morilles au vin}

Pour trois personnes prenez :

\footnotesize
\begin{longtable}{rrrp{16em}}
    500 & grammes & de & morilles,                                                                        \\
    200 & grammes & de & fond de veau et volaille,                                                        \\
    200 & grammes & de & jambon haché,                                                                    \\
    100 & grammes & de & sauternes,                                                                       \\
    100 & grammes & de & beurre,                                                                          \\
        &         &  2 & jaunes d'œufs frais,                                                             \\
        &         &    & sel et poivre.                                                                   \\
\end{longtable}
\normalsize

Préparez d’abord la sauce : mettez dans une casserole le fond de veau et
volaille, le sauternes et le jambon ; faites réduire à bonne consistance, liez ensuite
avec les jaunes d'œufs, goûtez et ajoutez sel et poivre s'il est nécessaire.

Nettoyez, lavez et séchez les morilles ; coupez-les en deux, assaisonnez-les
avec sel, poivre et faites-les sauter dans le beurre jusqu'à disparition de
l'eau qu'elles auront rendue. L'opération demande un quart d'heure environ.

Dressez-les ensuite sur un plat de service tenu au chaud au bain-marie, versez
dessus la sauce et laissez en contact pendant une quinzaine de minutes avant de servir.

\medskip

Ces morilles peuvent être servies dans une croûte de vol-au-vent.

\section*{\centering Morilles farcies, en cocotes.}
\phantomsection
\addcontentsline{toc}{section}{ Morilles farcies, en cocotes.}
\index{Morilles farcies, en cocotes}

Pour six personnes prenez :

\footnotesize
\begin{longtable}{rrrp{16em}}
    500 & grammes & de & belles morilles grises,                                                          \\
    260 & grammes & de & beurre,                                                                          \\
    200 & grammes & de & parmesan râpé,                                                                   \\
    150 & grammes & de & mie de pain rassis tamisée,                                                      \\
      5 & grammes & de & sel,                                                                             \\
      2 & grammes & de & poivre.                                                                          \\
\end{longtable}
\normalsize

Nettoyez bien les morilles, séchez-les. Détachez les pieds, hachez-les avec
quelques morilles, les plus petites, de façon à avoir en tout environ
{\ppp150\mmm} grammes de hachis.

\index{Farce pour morilles}
Mettez {\ppp200\mmm} grammes de beurre dans une casserole, laissez-le fondre,
ajoutez le hachis de morilles, la mie de pain, {\ppp150\mmm} grammes de
parmesan, le sel, le poivre, mélangez, puis garnissez les morilles avec cette
farce.

Prenez six petites cocotes en porcelaine allant au feu et munies de couvercles,
disposez, debout dans chacune, des morilles farcies, saupoudrez avec le reste du
parmesan, mettez le reste du beurre coupé en petits morceaux, lutez les couvercles
des cocotes et faites cuire au bain-marie, au four, pendant une heure.

Servez dans les cocotes.

\section*{\centering Morilles farcies.}
\phantomsection
\addcontentsline{toc}{section}{ Morilles farcies.}
\index{Morilles farcies}

Nettoyez, lavez et séchez les morilles.

Préparez un hachis avec blanc de volaille rôtie, jambon cuit, champignons de
couche, mie de pain rassis trempée dans du jus de viande, persil ;
assaisonnez-le avec sel et poivre, liez-le avec du beurre et un jaune d'œuf, et
farcissez-en les morilles par le pied.

Mettez du beurre dans une casserole, disposez dedans les morilles farcies,
mouillez avec une sauce à l'appareil Mirepoix, laissez mijoter pendant une
heure environ, puis servez.

Ces morilles farcies peuvent faire une très jolie garniture.

\section*{\centering Truffes farcies sur canapés.}
\phantomsection
\addcontentsline{toc}{section}{ Truffes farcies sur canapés.}
\index{Truffes farcies sur canapés}
\index{Canapés de truffes}
\index{Coulis de cailles}
\index{Farce paur truffes}

Pour douze personnes prenez ;

\footnotesize
\begin{longtable}{rrrp{16em}}
  750 & grammes & de & purée de bécasses préparée comme il est dit \hyperlink{p0634}{p. \pageref{pg0634}}, \\
        &         & 36 & cailles,                                                                          \\
        &         & 24 & belles truffes noires du Périgord,                                                \\
        &         &  1 & beau foie gras d'oie,                                                             \\
        &         &    & champagne sec,                                                                    \\
        &         &    & brioche non sucrée,                                                               \\
        &         &    & Mirepoix,                                                                         \\
        &         &    & beurre,                                                                           \\
        &         &    & paprika.                                                                          \\
\end{longtable}
\normalsize

Brossez, lavez les truffes ; séchez-les.

Faites rôtir les cailles à la broche en les tenant peu cuites ; passez-les à la
presse, relevez le coulis avec du mirepoix et du paprika, ajoutez-y du
champagne sec. Mettez à cuire dedans d'abord les truffes, puis le foie gras.
Tenez le foie rosé.

Enlevez une rondelle sur chaque truffe pour servir de couvercle ; creusez
l'intérieur ; emplissez les vides avec la purée de bécasses et assujettissez
les couvercles. Tenez au chaud truffes farcies et foie.

Dégraissez la cuisson, liez-la et amenez-la à l'état de demi-glace.

Préparez {\ppp24\mmm} canapés de brioche, faites-les dorer dans du beurre et
garnissez-les chacun d'une escalope de même dimension taillée dans le foie
gras.

Dressez les canapés garnis sur un plat d'argent, disposez sur chacun une truffe
que vous masquerez avec la sauce. Servez.

Une romanée ou un clos Vougeot convient avec cet entremets de légumes de
haut goût, qui pourra être le clou d'un repas fastueux, à la condition qu'il n'y
ait dans le menu aucune autre préparation contenant du foie gras ou des truffes
dont abusent souvent des amphitryons désireux de paraître.

\sk

On pourra exécuter de nombreuses variantes de truffes farcies en modifiant la
farce, le fond de cuisson et le vin.

\section*{\centering Conservation des champignons.}
\phantomsection
\addcontentsline{toc}{section}{ Conservation des champignons.}
\index{Conservation des champignons}

Les champignons qu'on veut conserver doivent être fraîchement cueillis, récoltés
par temps sec et n'ayant pas encore atteint leur complet développement.
Il existe plusieurs moyens de conserver les champignons.

\medskip

1° \textit{À l'autoclave}. — Après avoir épluché les champignons, on les jette
dans une grande bassine pleine d'eau bouillante légèrement salée et on les
laisse bouillir pendant une vingtaine de minutes. On les égoutte ensuite sur un
tamis de crin, puis on les rafraîchit dans de l’eau bouillie refroidie. On les
éponge bien ; on les met alors dans des flacons qu'on remplit entièrement avec
de l'eau salée bouillie et refroidie ({\ppp25\mmm} grammes de sel par litre
d'eau), on bouche les flacons avec de bons bouchons, stérilisés au préalable
dans de l’eau bouillante additionnée d'acide salicylique. On place les flacons
bien bouchés dans un panier en fil métallique\footnote{ À défaut de panier, on
garnira le fond de la marmite avec une grille sur laquelle on placera les
flacons qu'on séparera les uns des autres par des copeaux pour les garantir de
tout choc.}, on met le tout dans une marmite qu'on remplit d'eau jusqu'au
niveau des bouchons et on fait bouillir pendant {\ppp4\mmm} à {\ppp5\mmm}
heures en maintenant toujours le même niveau d’eau par des additions
successives d'eau bouillante. Les flacons refroidis sont mis en observation
pendant une huitaine de jours. S'il ne s'est produit, pendant ce temps, aucun
signe de fermentation, la conserve est bien faite.

On peut, à la place de flacons, employer des boîtes en fer-blanc qu'on soudera
avant de les mettre à l'autoclave. La conservation n'en sera que mieux assurée.

Les champignons ainsi conservés peuvent être employés comme des champignons
frais à condition de les passer, au sortir des flacons ou des boîtes, d'abord
dans de l’eau tiède pour les dessaler, puis dans de l'eau fraîche pour les
raffermir.

\medskip

2° \textit{Dans le sel}. — On dispose les champignons dans un vase, en couches
séparées par des lits de sel. Ce procédé n'assure la conservation que pendant
quelques mois.

\medskip

3° \textit{Dans le vinaigre}. — Les champignons peuvent être confits dans le
vinaigre à la façon des cornichons, mais il faut d'abord les blanchir à l'eau
bouillante.

Les champignons confits servent de condiment ou de hors-d'œuvre.

\medskip

4° \textit{Par dessiccation}. — Les petits champignons peuvent être séchés
entiers, les gros doivent être coupés en tranches plus ou moins épaisses après
avoir été épluchés et débarrassés des lames, des tubes et du pied. Les
champignons entiers ou en tranches sont disposés, à une certaine distance les
uns des autres, sur des claies tenues à l'ombre dans un endroit bien sec.

On peut en faire des chapelets, en ayant soin de séparer les champignons ou les
morceaux les uns des autres par des intervalles afin qu'ils ne se touchent pas.
Les chapelets sont suspendus dans un endroit à l'ombre et exposé aux courants
d'air.

Suivant la nature des champignons et l'état hygrométrique de l'air, la
dessication demande un temps variant de quelques jours à quelques semaines.

On peut encore sécher les champignons sur une aire chauffée ou dans un four
très modérément chaud.

Les champignons séchés sont gardés dans des sacs en toile ou en papier qu'on
secoue de temps en temps, ou dans des vases clos, et conservés dans un endroit
sec.

Lorsqu'on veut se servir de champignons séchés, on les fait d'abord tremper
dans de l’eau tiède pendant une heure, puis on les emploie comme s'ils étaient
frais.

Les champignons desséchés comme il est dit ci-dessus peuvent être pulvérisés au
mortier et tamisés ; la poudre obtenue est mise en flacons qu'on bouche
hermétiquement et qu'on tient à l'abri de l'humidité.

\medskip

5° \textit{Dans un corps gras}. — Dans le Midi de la France, on conserve les
champignons dans de l'huile, de la graisse ou du beurre. Après avoir blanchi
les champignons dans de l’eau bouillante et les avoir séchés dans un linge, on
les empile dans des pots en faïence ou en grès que l'on emplit ensuite d'huile
d'olive, de graisse ou de beurre fondu de façon que le niveau du corps gras
dépasse de quelques centimètres la surface des champignons. On bouche les pots
et on couvre les bouchons avec un parchemin ficelé.

On peut conserver de la même manière des champignons non blanchis, simplement
sautés dans un corps gras pour les déshydrater avant de les mettre en pots.

Les pots sont tenus dans un endroit frais et sec.

\smallskip

6° \textit{Par liquéfaction}. — En Angleterre, on fait un grand usage des
champignons liquéfiés « Ketcup ». Voici la manière d'opérer : des champignons
choisis et coupés en tranches sont placés sur un tamis de crin et saupoudrés de
sel fin ; sans les presser, on laisse s'écouler le jus que l'on recueille. Afin
d'éviter tout dépôt, on transvase ce liquide et on le met dans de petites
bouteilles qu'on n'emplit pas complètement afin de pouvoir y ajouter un peu
d'un spiritueux aromatisé quelconque destiné à empêcher la préparation de se
détériorer. On bouche les bouteilles et on les conserve au frais.

Le ketcup est employé comme condiment.

\section*{\centering Conserve de truffes.}
\phantomsection
\addcontentsline{toc}{section}{ Conserve de truffes.}
\index{Conserve de truffes}

Prenez de belles truffes bien saines, lavez-les soigneusement, laissez-les
sécher sur un linge pendant {\ppp3\mmm} à {\ppp4\mmm} heures. Ayez des flacons
de grandeur suffisante et mettez dans chacun {\ppp250\mmm} grammes de truffes,
{\ppp12\mmm} grammes d'eau-de-vie sans goût et {\ppp5\mmm} grammes de sel ;
bouchez hermétiquement les flacons, ficelez solidement les bouchons et entourez
chaque flacon d'un linge. Disposez les flacons dans une grande bassine que vous
emplirez d'eau froide jusqu'au niveau des bouchons ; calez les flacons avec des
copeaux ou chargez-les d'un poids pour les empêcher de bouger pendant
l'ébullition. Chauffez et faites bouillir pendant {\ppp2\mmm} à {\ppp3\mmm}
heures. Laissez refroidir les flacons dans la bassine, puis sortez-les,
essuyez-les, cachetez-les et conservez-les dans un lieu frais.

Les truffes préparées ainsi peuvent se conserver pendant plus de deux ans sans
s'altérer.
