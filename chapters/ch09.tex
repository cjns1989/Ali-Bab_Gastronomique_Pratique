\index{Comment je comprends l'organisation des repas d'amis}

On peut diviser les repas en deux classes : d'une part, les grands dîners de
cérémonie, les repas de corps et les banquets ; d'autre part, les repas de
famille et les repas d'amis.

Les grands dîners de cérémonie sont réglés par un protocole minutieux qui
nécessite une maison montée complète, comprenant toute une série de
spécialistes : rôtisseur, saucier, pâtissier, secondés par des gens de service
et dirigés par un maître d'hôtel expérimenté. Peu de personnes sont à même de
donner de véritables grands dîners.

En ce qui concerne les repas de corps et les banquets, l'organisation en est
réservée à des industriels spécialement outillés pour ce genre de sport qui n'a
que de très vagues rapports avec la véritable gastronomie, car les principales
préoccupations de ces entrepreneurs sont de composer un menu suggestif, de
dresser une table décorée au goût du jour et d'assurer la régularité du
service. Lorsque, pour une raison quelconque, on est obligé d'assister à une
cérémonie de ce genre, le mieux qu'on ait à faire, c'est d'y manger et surtout
d'y boire le moins possible. Restent les repas de famille et les repas d'amis.
Je ne parlerai pas des premiers : « Charbonnier est maître chez soi ». Mais dès
qu'on reçoit des amis, on assume une responsabilité, on a pour ainsi dire
charge d'âmes : aussi je crois bon de consacrer quelques mots à ce sujet, en me
plaçant au double point de vue du choix des commensaux et de la composition des
menus.

La première condition d'un repas d'amis bien organisé où chacun doit avoir son
franc-parler est une certaine communauté d'idées, chez les convives, sur les
questions fondamentales, afin de rendre impossible toute discussion aigre-douce
capable de troubler la digestion, ce qu'il est parfois malaisé d'éviter même
entre gens bien élevés aussitôt qu'on aborde certains sujets, la tolérance
étant la plus rare de toutes les vertus. Aussi est-il essentiel de ne réunir
que des gens pouvant sympathiser ; et cette considération suffirait à elle
seule pour limiter le nombre des convives si, de plus, il ne fallait tenir
compte de la difficulté pratique qu'il y a à servir convenablement, dans des
milieux moyennement outillés, beaucoup de personnes à la fois. D'ailleurs, au
delà de douze, on commence à avoir l'air d’être à table d'hôte ; l'intimité
disparaît et toute conversation générale devient peu commode. Or, une réunion
d'amis doit avoir pour but de faire vibrer harmonieusement chez tous aussi bien
les cordes de l'esprit que celles du palais.

Mais ce n'est pas assez de grouper un certain nombre d'amis sympathiques les
uns aux autres pour avoir une tablée adéquate ; il faut encore qu'il y ait
entre eux une certaine homogénéité au point de vue gastronomique car, si leurs
goûts différaient par trop, on serait amené, pour les satisfaire tous,
à composer des menus pantagruéliques qu'il convient d'éviter. Pour faire les
choses raisonnablement, il importe de choisir ses invités de façon à être
certain que, dans les quatre ou cinq plats qui doivent normalement composer un
repas prié sans prétention, il y aura de quoi satisfaire tout le monde. Lorsque
la réunion a lieu en tout petit comité, on peut aisément trouver trois plats
qui conviennent à tous : dans ce cas le menu est simplifié.

Cela posé, passons aux principes généraux qui doivent présider à la confection
des menus.

Il faudra éviter les répétitions et les analogies : par exemple, ne pas servir
dans un même repas une même viande sous deux formes, ou deux viandes
notoirement blanches, par exemple du veau et du poulet ; ni deux gibiers de
même genre ; ni une volaille avec un gibier à plumes tel qu'un faisan ou un
canard sauvage ; ne pas donner deux fois de garnitures semblables ;
entrecroiser les mets et les sauces aussi bien comme couleur que comme goût de
façon que chaque plat, tout en donnant sa note propre, fasse valoir le
précédent et prépare à déguster le suivant.

Les fromages doivent être l'objet de tous les soins, et il est bon d'en offrir
toujours au moins deux, l'un cru, l'autre cuit. Les fromages de chèvre font
remarquablement ressortir le bouquet des vins ; les fromages très aromatisés
sont en général les plus digestifs.

Pour ce qui est des vins, sauf indication spéciale pouvant résulter de
conditions et de goûts particuliers, le plus simple est d'avoir deux vins de
Bourgogne et deux vins de Bordeaux, blancs et rouges. La question des qualités
respectives des différents crus et celle du service des vins ont été traitées
au chapitre des vins. Du reste, il n'y a là rien d'absolu, et tel troisième cru
d’une excellente année est souvent supérieur à un premier cru d'une année
médiocre. Parmi les vins présentés, un vin rouge et un vin blanc au moins
doivent être d'un âge respectable. En outre, contrairement aux usages, j'estime
que tous les vins doivent être versés dans de grands verres, qui seuls
permettent de les déguster.

Dans les repas sans cérémonie, on jouit d'une assez grande latitude
relativement à l'ordre des différents plats, sous la réserve de se laisser
guider par les règles générales énoncées plus haut qui, du reste, devraient
être observées dans tous les cas. À ce sujet, je crois bon de protester contre
l'usage qui consiste à faire figurer, dans les menus des dîners classiques, les
glaces, les fromages et les fruits suivant l'ordre dans lequel je les énumère
ici ; c'est à mon sens une véritable hérésie. Et en fait, dans ces conditions,
le plus souvent ceux-là seuls mangent du fromage et des fruits qui n'ont pas
mangé de glace. Pour moi, le fromage doit clore le repas proprement dit, et
c'est en le savourant que l'on doit boire la dernière gorgée de vin rouge.
Viennent ensuite les fruits qui rafraîchissent et parfument la bouche, puis la
glace qui accentue la sensation de froid, tempérée bientôt par une coupe de
champagne. A ce moment la séance peut être levée. Si on la prolonge, on sert
alors des friandises variées, des pâtisseries, des gâteaux secs et des biscuits
qui justifient une nouvelle coupe de champagne. J'étends ce que je viens de
dire à propos des glaces, que par parenthèse l'on doit toujours servir avec des
gaufrettes, à tout ce que l’on désigne sous le nom d'entremets sucrés et, d'une
façon générale, je soutiens que toutes les douceurs doivent être présentées
à la fin du repas après les fromages et les fruits. A l’article « Café » je
parle de la préparation de ce tonique ; j'expose également, dans un chapitre
spécial, mon opinion sur le rôle des liqueurs et sur la façon de les goûter.

Mais, avant de finir, je tiens à dire un mot de l'habitude déplorable qu'on
a presque partout aujourd'hui d'attendre indéfiniment, avant de se mettre
à table, les retardataires jusqu'au dernier, sous prétexte de courtoisie. Ce
système a pour inconvénient de favoriser les gens inexacts au détriment des
autres que l'on désoblige, de faire manger ses invités à des heures
invraisemblables et de gâter les meilleurs repas. Chaque plat doit, en effet,
être servi à point, et il faut n'avoir aucune notion de cuisine pour se figurer
qu'on peut faire attendre impunément la plupart des plats chauds.

Il serait si simple d'introduire dans les invitations la formule suivante :
« On se mettra à table à telle heure précise » et de s'imposer la règle absolue
de ne jamais attendre personne. J'ajouterai même que je serais assez disposé
à ne pas laisser entrer les retardataires dans la salle à manger pendant la
dégustation d'un plat ; il y a un précédent : dans les concerts sérieux on
n'admet personne dans la salle pendant l'exécution des morceaux. Les gens
insupportables, qui sont venus au monde une heure trop tard et qui ne peuvent
jamais être à l'heure nulle part, finiraient par rester chez eux et personne ne
s'en plaindrait.

Pour terminer, je vais donner comme exemples concrets un certain nombre de
menus qui ont été appréciés par des amis et qui sont composés exclusivement de
plats figurant dans ce recueil. Les premiers ont servi pour des réunions de
quatre à six convives, les autres ont permis d'en traiter davantage, jusqu'à
douze,
