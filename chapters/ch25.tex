\section*{\centering Beignets de pommes.}
\phantomsection
\addcontentsline{toc}{section}{ Beignets de pommes.}
\index{Beignets de pommes}

Pour six personnes prenez :

\footnotesize
\begin{longtable}{rrrp{16em}}
    700 & grammes & de & pommes reinettes ordinaires,                                                     \\
    250 & grammes & d' & abricots,                                                                        \\
    200 & grammes & de & kirsch,                                                                          \\
    150 & grammes & de & farine de gruau,                                                                 \\
    100 & grammes & de & vin de Madère,                                                                   \\
    100 & grammes & de & bière légère,                                                                    \\
     45 & grammes & de & fine champagne,                                                                  \\
     45 & grammes & d' & huile d'olive,                                                                   \\
      5 & grammes & de & sel,                                                                             \\
        &         &  2 & belles pommes reinettes, pesant ensemble 500 grammes environ,                    \\
        &         &  1 & œuf frais,                                                                       \\
        &         &  1 & gousse de vanille,                                                               \\
        &         &    & sirop de sucre,                                                                  \\
        &         &    & cannelle,                                                                        \\
        &         &    & sucre.                                                                           \\
\end{longtable}
\normalsize

La veille du jour où vous voudrez faire les beignets, préparez la pâte en
mélangeant intimement la farine, l'œuf, la bière, la fine champagne, l'huile et
le sel.

Le lendemain, faites deux marmelades sucrées au goût, l'une avec les pommes
reinettes ordinaires, l'autre avec les abricots.

Pelez les deux belles pommes, coupez-les chacune en six tranches, parez-les,
faites-les pocher dans du sirop aromatisé avec le madère et la vanille, puis
mettez-les à mariner dans le kirsch. Concentrez à moitié le sirop, ajoutez-y
les marmelades de fruits, le Kirsch de la marinade, de la cannelle, mélangez
bien et réduisez le tout à une consistance un peu ferme.

Trempez les tranches de pommes dans la pâte, plongez-les dans de la friture
claire très chaude, laissez les cuire. Enlevez-les avec une écumoire,
égouttez-les bien, saupoudrez-les de sucre en poudre et faites caraméliser au
four.

Mettez la marmelade dans un compotier, disposez les beignets dessus et servez.

\section*{\centering Beignets de pommes.}
\phantomsection
\addcontentsline{toc}{section}{ Beignets de pommes.}
\index{Beignets de pommes}
\index{Beignets de pommes (autre formule)}

\begin{center}
\textit{(Autre formule).}
\end{center}

Pour six personnes prenez :

\footnotesize
\begin{longtable}{rrrp{16em}}
    750 & grammes & de & poires,                                                                          \\
    200 & grammes & de & kirsch, cognac ou rhum, au goût,                                                 \\
    130 & grammes & de & farine,                                                                          \\
    100 & grammes & de & lait chaud,                                                                      \\
     20 & grammes & de & beurre fondu,                                                                    \\
      2 & grammes & de & sel,                                                                             \\
        &         &  2 & belles pommes reinettes,                                                         \\
        &         &  2 & œufs frais,                                                                      \\
        &         &    & jus de citron,                                                                   \\
        &         &    & sucre parfumé à la vanille ou à l'orange,                                        \\
        &         &    & sucre ordinaire en poudre.                                                       \\
\end{longtable}
\normalsize

Cassez les œufs, séparez les blancs des jaunes, montez les blancs en neige.

Préparez la pâte en mélangeant intimement la farine, le lait, le beurre fondu,
les jaunes d'œufs, le sel et du sucre au goût ; travaillez-la bien, puis
incorporez-y les blancs battus.

Pelez les pommes, coupez-les en tranches minces, frottez-les avec du jus de
citron, saupoudrez-les de sucre parfumé à la vanille ou à l'orange et faites-les
mariner pendant une demi-heure dans le kirsch, le cognac ou le rhum.

Faites avec les poires une marmelade sucrée au goût.

Sortez les tranches de pommes de la marinade, essuyez-les, puis trempez-les
dans la pâte.

Ajoutez à la compote de poires le reste de la liqueur qui a servi à la marinade
des pommes ; réduisez.

Plongez les tranches de pommes enrobées de pâte dans de la friture claire très
chaude. Lorsque les beignets seront cuits à point, retirez-les, égouttez-les,
saupoudrez-les de sucre et mettez-les pendant un instant au four pour les
glacer.

Disposez la marmelade de poires dans un plat, dressez dessus les beignets et
servez.

\sk

\index{Beignets aux fruits}
\index{Beignets d'ananas}
\index{Beignets de poires}
\index{Beignets de pêches}
\index{Beignets d'oranges}
On peut préparer de même des beignets d'autres fruits ; pêches, poires, ananas,
oranges, etc.

En variant les parfums et les liqueurs ainsi que la nature des marmelades, on
obtiendra toute une série d'entremets différents très agréables.

\section*{\centering Beignets de purée de pommes.}
\phantomsection
\addcontentsline{toc}{section}{ Beignets de purée de pommes.}
\index{Beignets de purée de pommes}

Épluchez des pommes reinettes, retirez-en les pépins.

Faites fondre les pommes dans du beurre, sans addition d'eau et sans leur
laisser prendre couleur ; passez-les. Aromatisez la purée avec de la vanille,
par exemple, liez-la avec de la crème et des jaunes d'œufs ; mettez-la à la
glace.

Confectionnez avec cette purée refroidie des boules ayant la grosseur de belles
noix ; passez-les dans de la farine, puis dans de l'œuf battu, et faites-les
cuire dans de la friture chaude. Égouttez-les. Roulez-les enfin dans du sirop
de fruits chaud, ensuite dans des macarons de Nancy séchés au four et
grossièrement pilés.

Dressez les beignets sur un plat.

Servez, en envoyant en même temps une marmelade d'abricots parfumée au
kirsch, par exemple.

\sk

\index{Beignets de purée de fruits}
Il va sans dire qu'on peut préparer dans le même esprit des beignets de purée
d'autres fruits et varier à son gré les marmelades ainsi que les parfums.

Tous ces beignets sont d'une finesse remarquable.


\section*{\centering Beignets de pommes fourrées.}
\phantomsection
\addcontentsline{toc}{section}{ Beignets de pommes fourrées.}
\index{Beignets de pommes fourrées}

Prenez de belles pommes reinettes, enlevez à chacune un bouchon du côté de la
queue en la conservant, réservez-les.

Creusez l'intérieur des pommes, puis mettez-les à mariner pendant trois heures
dans du vin blanc sucré relevé par de la fine champagne et aromatisé avec de
l’eau de fleurs d'oranger additionnée de jus de citron. Égouttez-les ensuite,
emplissez-les de crèmes, de confitures ou de marmelades variées, obturez les
orifices avec les bouchons réservés que vous collerez à l'œuf.

Trempez les pommes fourrées dans de la pâte à beignets, faites-les frire dans
une friture claire et abondante, très chaude, saupoudrez-les de sucre,
glacez-les au four et servez.

C'est une délicieuse surprise.

\section*{\centering Beignets à la polonaise.}
\phantomsection
\addcontentsline{toc}{section}{ Beignets à la polonaise.}
\index{Beignets à la polonaise}

Pour six personnes prenez :

\footnotesize
\begin{longtable}{rrrp{16em}}
    200 & grammes & de & farine,                                                                          \\
    100 & grammes & de & lait chaud,                                                                      \\
     20 & grammes & de & beurre,                                                                          \\
     15 & grammes & de & sucre en poudre ordinaire,                                                       \\
     15 & grammes & de & sucre en poudre, parfumé à la vanille ou à l'orange,                             \\
     15 & grammes & d'a& mandes sèches écrasées au mortier,                                               \\
      7 & grammes & de & levain,                                                                          \\
      2 & grammes & de & sel,                                                                             \\
        &         &  1 & jaunes d'œufs frais,                                                             \\
        &         &  1 & œuf frais,                                                                       \\
        &         &    & \hangindent=1em des fruits de confitures, ou des fruits frais cuits
                         dans du sirop et débarrassés de tout excès de sirop, ou des marmelades
                         épaisses de fruits divers.                                                       \\
\end{longtable}
\normalsize

Délayez dans le lait le levain et {\ppp60\mmm} grammes de farine ; mettez le
mélange au bain-marie et lorsqu'il aura levé trois fois, ce qui demande
vingt-cinq minutes environ, ajoutez petit à petit le reste de la farine en
pétrissant la pâte pendant une demi-heure. Tenez la pâte dans un endroit chaud,
laissez-la lever et, au bout de deux heures, ajoutez-y le beurre, le sucre
ordinaire, les amandes, le sel, deux jaunes d'œufs, l'œuf entier, pétrissez
bien le tout ensemble et laissez lever encore une fois.

Abaissez alors la pâte en feuille mince.

Découpez à l'emporte-pièce dans cette feuille de pâte des ronds de {\ppp6\mmm}
centimètres de diamètre, enduisez de jaune d'œuf le bord de la moitié des ronds
de pâte, mettez au milieu des fruits de confitures, des fruits frais cuits dans
du sirop ou des marmelades fermes de fruits variés ; couvrez les ronds ainsi
garnis avec les autres ronds, fermez ces beignets comme des ravioli et
faites-les frire dans de la friture claire moyennement chaude. Égouttez les
beignets sur du papier buvard, saupoudrez-les de sucre parfumé et servez-les
soit au naturel, soit avec du jus de fruits.

\section*{\centering Beignets viennois.}
\phantomsection
\addcontentsline{toc}{section}{ Beignets viennois.}
\index{Beignets viennois}

Saupoudrez de farine une plaque de marbre, étendez dessus une abaisse de pâte
à brioche, coupez la pâte en rondelles de {\ppp8\mmm} à {\ppp9\mmm} centimètres
de diamètre.

Mettez une petite boule de marmelade de fruits très épaisse sur une rondelle
de pâte, dont vous mouillerez le pourtour avec du blanc d'œuf, couvrez avec une
autre rondelle de pâte, fermez le beignet. Répétez l'opération jusqu'à épuisement
des substances employées.

Laissez les beignets au chaud, pendant quelque temps, dans un torchon fariné,
puis faites-les cuire dans de la friture chaude pendant une minute de chaque
côté. Égouttez-les, saupoudrez-les de sucre vanillé et servez-les sur une
serviette.

\section*{\centering Beignets soufflés ou pets de nonne.}
\phantomsection
\addcontentsline{toc}{section}{ Beignets soufflés ou pets de nonne.}
\index{Beignets soufflés ou pets de nonne}

Pour huit personnes prenez :

\footnotesize
\begin{longtable}{rrrp{16em}}
    250 & grammes & de & farine,                                                                          \\
    250 & grammes & d' & eau,                                                                             \\
    125 & grammes & de & beurre,                                                                          \\
      5 & grammes & de & sel,                                                                             \\
        &         &  8 & œufs,                                                                            \\
        &         &    & zeste râpé d'orange,                                                             \\
        &         &    & sucre en poudre.                                                                 \\
\end{longtable}
\normalsize

Préparez la pâte quelques heures d'avance.

Mettez dans une casserole l’eau, le sel, le beurre coupé en petits morceaux, du
zeste d'orange au goût, faites bouillir en remuant. Éloignez la casserole du feu,
ajoutez la farine, manipulez avec une cuiller en bois et faites sécher sur feu doux
jusqu'à ce que le fond ait une consistance sableuse.

Laissez refroidir pendant quelques minutes, puis incorporez les œufs un à un ;
mélangez bien après l'addition de chaque œuf.

Prenez avec une cuiller un peu de pâte, lissez-la en boule sur la cuiller et
plongez successivement chaque boule dans de la friture claire portée à la
température qui fait crépiter une feuille de persil ; les boules gonfleront et
tourneront spontanément dans la friture. Dès qu'elles auront cessé de tourner,
la cuisson sera achevée.

Retirez les beignets, égouttez-les, dressez-les en pyramide sur un plat garni
d'une serviette, poudrez-les avec du sucre et servez-les brûlants.

\sk

\index{Beignets soufflés garnis}
Comme variante, on peut garnir les beignets de crème ou de confiture et les
glacer au four.

\section*{\centering Crêpes.}
\phantomsection
\addcontentsline{toc}{section}{ Crêpes.}
\index{Crêpes}

Les crêpes sont des entremets sucrés en forme de disques très minces qui
s'ondulent légèrement à la cuisson, d'où leur nom. Elles sont le complément
classique des repas de carnaval.

La composition de la pâte est variable ; on la parfume le plus souvent à la
fleur d'oranger ou à la vanille, mais on peut aussi l'aromatiser différemment.

\medskip

\textit{Composition et préparation de la pâte.}

\medskip

En voici trois exemples :

\medskip

A. — Pour faire une quarantaine de crêpes à la fleur d'oranger, de {\ppp22\mmm}
centimètres de diamètre environ, dimension moyenne, prenez :

\footnotesize
\begin{longtable}{rrrp{16em}}
    750 & grammes & de & lait,                                                                            \\
    500 & grammes & de & farine de froment,                                                               \\
    180 & grammes & de & sucre,                                                                           \\
     60 & grammes & de & fine champagne,                                                                  \\
     30 & grammes & d' & eau de fleurs d'oranger,                                                         \\
      5 & grammes & de & sel gris,                                                                        \\
      2 & grammes & de & levure de bière,                                                                 \\
        &         &  8 & œufs frais pesant ensemble 480 grammes environ,                                  \\
        &         &    & beurre frais.                                                                    \\
\end{longtable}
\normalsize

Préparez la pâte {\ppp24\mmm} heures d'avance en hiver, {\ppp12\mmm} heures en
été.

Cassez les œufs, séparez les blancs des jaunes ; battez les blancs en neige.

Triturez la farine avec les jaunes d'œufs, mouillez avec le lait dans lequel
vous aurez fait dissoudre le sucre, le sel et la levure, ajoutez la fine
champagne, l'eau de fleurs d'oranger, enfin incorporez les blancs d'œufs
battus. Mélangez bien : vous aurez ainsi une pâte fluide.

\medskip

B. — Pour faire la même quantité de crêpes à la vanille, prenez :

\footnotesize
\begin{longtable}{rrrp{16em}}
    840 & grammes & de & lait,                                                                            \\
    500 & grammes & de & farine de froment,                                                               \\
    180 & grammes & de & sucre,                                                                           \\
      5 & grammes & de & sel gris,                                                                        \\
      2 & grammes & de & levure de bière,                                                                 \\
        &         &  8 & œufs frais pesant ensemble 480 grammes environ,                                  \\
        &         &  1 & gousse de vanille,                                                               \\
        &         &    & beurre frais.                                                                    \\
\end{longtable}
\normalsize

Faites bouillir le lait pendant {\ppp5\mmm} minutes avec la vanille ;
retirez-la ; laissez refroidir le lait.

Préparez ensuite la pâte comme il est dit ci-dessus.

\medskip

C. — Voici enfin une troisième composition de pâte.

\medskip

Pour faire dix à douze crêpes prenez :

\footnotesize
\begin{longtable}{rrrp{16em}}
    200 & grammes & de & farine de froment,                                                               \\
    125 & grammes & d' & eau tiède,                                                                       \\
    125 & grammes & de & bière,                                                                           \\
     15 & grammes & de & rhum,                                                                            \\
     15 & grammes & de & kirsch,                                                                          \\
      5 & grammes & de & sel blanc,                                                                       \\
        &         &  5 & œufs frais,                                                                      \\
        &         &    & zeste d'orange râpé.                                                             \\
\end{longtable}
\normalsize

Mélangez d'abord farine, œufs, sel, zeste d'orange, ajoutez ensuite par petites
quantités l'eau, la bière, le rhum et le kirsch. Travaillez bien pour obtenir
une pâte lisse. Laissez-la reposer pendant {\ppp4\mmm} heures avant de vous en
servir.

\medskip

\textit{Cuisson des crêpes. }

\medskip

Au moment d'employer la pâte, mélangez-la à nouveau.

Le meilleur appareil pour cuire les crêpes consiste en une plaque de fonte
ayant {\ppp35\mmm} et {\ppp40\mmm} centimètres de diamètre et {\ppp6\mmm}
millimètres environ d'épaisseur. Cette plaque est posée sur un trépied placé
au-dessus d’un foyer quelconque, dont la flamme est régularisée par
l'interposition d'une toile métallique.

On commence par chauffer la plaque et, lorsqu'elle est suffisamment chaude, on
la graisse légèrement avec du beurre dont on essuie l'excès. On met sur cette
plaque une cuillerée à ragoût de pâte qu'on étend vivement à l’aide d'une
raclette en bois de façon à donner à la crêpe une forme circulaire et une
épaisseur minima. Dès que la crêpe est cuite d'un côté, on l'enlève au moyen
d'un instrument en bois ayant la forme d'un coupe-papier à lame très-large, on
graisse rapidement la plaque avec un peu de beurre, on remet dessus la crêpe du
côté non cuit, on la laisse dorer et on la retire définitivement.

On essuie alors la plaque pour ôter les parcelles de pâte qui pourraient y
adhérer et on recommence d'une façon identique la cuisson d'une autre crêpe.

Cette manière de procéder a l'avantage de permettre de faire des crêpes
extrêmement minces, parfaitement homogènes, également cuites et très légères,
toutes choses plus difficiles à obtenir avec l'emploi d'une poêle.

On tiendra les crêpes au chaud jusqu'au moment de servir.

Les crêpes sont servies le plus souvent simplement saupoudrées de sucre.
On les asperge ou non de jus de citron.

\sk

\index{Crêpes à la sauce}
On peut servir les crêpes avec accompagnement d'une sauce au beurre préparée
de la façon suivante.

\medskip

Pour 40 crêpes prenez :

\footnotesize
\begin{longtable}{rrrp{16em}}
    280 & grammes & de & beurre fin,                                                                      \\
    240 & grammes & de & kirsch,                                                                          \\
    200 & grammes & de & sucre en morceaux,                                                               \\
        &         &    & le zeste de 4 oranges.                                                           \\
\end{longtable}
\normalsize

Frottez les morceaux de sucre avec le zeste des oranges de façon à les bien
parfumer ; écrasez-les ensuite dans le kirsch.

Faites fondre le beurre dans une casserole sans le laisser roussir, puis ajoutez
le kirsch sucré.

Servez la sauce dans un plat chauffé et les crêpes à part.

Les crêpes trempées dans cette sauce sont délicieusement parfumées.

\sk

On peut encore servir les crêpes avec une sauce constituée par une gelée de
fruits et une liqueur, au goût ; par exemple de la gelée d’abricots et du
sherry-brandy.

\sk

\index{Crêpes fourrées}
Les crêpes sont souvent présentées enduites de miel sur une face et roulées, ou
fourrées de confitures, soit telles quelles, soit accompagnées d'une sauce. On
conçoit facilement de nombreuses variantes. En voici un exemple : garnissez des
crêpes avec un mélange de confiture de fraises et de confiture d'ananas relevé
par un peu de kirsch et de curaçao ; roulez-les ; dressez-les sur un plat et
masquez-les avec une sauce sabaillon.

\sk

On peut eucore fourrer les crêpes de petits dés de poires fondantes, de tranches
minces de pommes sautées dans du beurre ou de fins émincés d'ananas macérés
dans une liqueur.

Le nombre des combinaisons possibles est presque illimité ; on peut s'en faire
une idée d'après les quelques données ci-dessus.

\sk

Enfin, on peut modifier la composition de la pâte pour crêpes en remplaçant le
lait par un mélange de lait, crème et beurre fondu ; y incorporer des macarons
finement pilés ou des biscuits pilés imbibés de kummel, par exemple ; du zeste
d'orange ou de mandarine émincé en julienne très fine et macéré dans du curaçao
blanc ; ou encore la parfumer avec du rhum, de l'anisette, du marasquin, etc.

\index{Crêpes à divers beurres}
On accompagnera les crêpes préparées avec ces diverses pâtes d'un beurre
assorti ; beurre travaillé avec du sucre en poudre et {\ppp15\mmm} pour
{\ppp100\mmm} de fine champagne, rhum, curaçao blanc, anisette, kummel,
marasquin, etc. additionné ou non de crème anglaise, de façon à avoir une sauce
de consistance sirupeuse.

\section*{\centering Gâteau de crêpes.}
\phantomsection
\addcontentsline{toc}{section}{ Gâteau de crêpes.}
\index{Gâteau de crèpes}
\index{Crépes en gâteau}

Les crêpes peuvent être présentées sous la forme d'un gâteau et voici deux
formules qu'il est aisé de varier à volonté.

\medskip

A. — Lorsque vous aurez fait des crêpes à la fleur d'oranger, mettez-les dans
un moule les unes sur les autres, en les séparant par de la pâte d'amandes.
Masquez le tout avec de la crème au citron et faites prendre au four doux.

\medskip

B. — Si les crêpes sont à la vanille, dressez-les comme ci-dessus dans un
moule en les séparant les unes des autres par de la crème à la vanille.
Mettez au four doux.

Servez chaud.

\section*{\centering Bliny.}
\phantomsection
\addcontentsline{toc}{section}{ Bliny.}
\index{Bliny}

Les bliny sont des sortes de petites crêpes, de compositions variées, dont on
fait grand usage en Russie, surtout pendant le carême.

\medskip

En voici une formule.

\medskip

Pour six à huit personnes prenez :

\footnotesize
\begin{longtable}{rrrrp{16em}}
  &    1 000 & grammes & de & farine de froment,                                                          \\
  &      500 & grammes & de & farine de sarrasin,                                                         \\
  &      300 & grammes & de & beurre clarifié,                                                            \\
  &       30 & grammes & de & levure de bière,                                                            \\
  & \multicolumn{2}{r}{1 litre} & de & lait non écrémé,                                                   \\
  &          &         &  5 & œufs frais,                                                                 \\
  &          &         &    & sucre en poudre,                                                            \\
  &          &         &    & sel                                                                         \\
\end{longtable}
\normalsize

Mélangez intimement les deux farines.

Cassez les œufs ; séparez les jaunes des blancs.

Faites dissoudre la levure dans le lait légèrement tiédi.

Avec le lait et le mélange des farines préparez une pâte un peu plus épaisse
que celle qui sert à la confection des crêpes ordinaires ; incorporez-y
{\ppp250\mmm} grammes de beurre clarifié tiède, les jaunes d'œufs, du sucre et
du sel au goût : mélangez bien.

Battez les blancs en neige ; mêlez-les à la pâte.

Laissez la pâte monter pendant toute une nuit.

Prenez des petites poêles rondes sans queue, de {\ppp7\mmm} centimètres de
diamètre et de {\ppp1\mmm} centimètre de profondeur. Versez dans chacune
d'elles une cuillerée à ragoût de pâte et faites cuire au four à la flamme d'un
bûcher de bois en portant les poêles sous la flamme au moyen d'une pelle à long
manche, et en arrosant pendant la cuisson avec le reste du beurre fondu.

Les bliny doivent être minces, dorées régulièrement des deux côtés, légères et
croustillantes.

Dressez les bliny sur un plat en argent, et servez-les accompagnées de beurre
clarifié et de crème aigre, dans des saucières, ainsi que de caviar frais.

\sk

Comme variantes, on pourra faire des bliny avec de la farine de gruau, du gruau
de sarrasin, de la semoule, de la crème de riz, etc. On pourra les parfumer au
citron, à la vanille, et remplacer le caviar, le beurre clarifié et la crème
aigrie par des sirops, des crèmes ou des marmelades de fruits.

\section*{\centering Crème frite.}
\phantomsection
\addcontentsline{toc}{section}{ Crème frite.}
\index{Crème frite}

Pour six Personnes prenez :

\footnotesize
\begin{longtable}{rrrp{16em}}
    500 & grammes & de & lait,                                                                            \\
    250 & grammes & de & mie de pain rassis tamisée,                                                      \\
    125 & grammes & de & sucre parfumé à la vanille.                                                      \\
     70 & grammes & de & farine,                                                                          \\
        &         &  6 & jaunes d'œufs frais,                                                             \\
        &         &  2 & œufs frais,                                                                      \\
        &         &    & huile d'olive,                                                                   \\
        &         &    & sel.                                                                             \\
\end{longtable}
\normalsize

Mettez dans une casserole {\ppp450\mmm} grammes de lait avec {\ppp100\mmm}
grammes de sucre ; faites réduire de moitié.

Mélangez la farine, les jaunes d'œufs et le reste du lait, ajoutez ensuite et
par petites quantités le lait réduit en remuant de façon à éviter la formation
de grumeaux, amenez le tout à ébullition, puis maintenez au chaud sans faire
bouillir pendant une dizaine de minutes : vous obtiendrez ainsi une crème qui
devra être ferme et moelleuse.

Coulez cette crème, sur une épaisseur de {\ppp5\mmm} à {\ppp6\mmm} millimètres,
dans un plat graissé de quelques gouttes d'huile d'olive ; laissez refroidir.

Fouettez les deux œufs entiers avec quelques grammes d'huile d'olive ; ajoutez
un peu de sel.

Coupez la crème refroidie en morceaux de {\ppp3\mmm} centimètres carrés
environ, passez successivement chaque morceau d’abord dans de la mie de pain,
ensuite dans les œufs battus, puis encore dans de la mie de pain et faites-les
frire dans de la graisse presque bouillante, de façon à les bien saisir et
à les dorer.

Retirez-les. égouttez-les, saupoudrez-les de sucre parfumé et servez.

La crème frite est un entremets de ménage agréable.

\section*{\centering Crème frite.}
\phantomsection
\addcontentsline{toc}{section}{ Crème frite.}
\index{Crème frite}
\index{Crème frite (autre formule)}

\begin{center}
\textit{(Autre formule).}
\end{center}

Pour six personnes prenez :

\footnotesize
\begin{longtable}{rrrp{16em}}
    250 & grammes & de & farine,                                                                          \\
    100 & grammes & de & lait,                                                                            \\
     25 & grammes & de & beurre,                                                                          \\
     15 & grammes & d' & eau de fleurs d'oranger,                                                         \\
        &         &  3 & jaunes d'œufs frais,                                                             \\
        &         &  1 & œuf frais,                                                                       \\
        &         &    & sucre en poudre,                                                                 \\
        &         &    & mie de pain rassis tamisée.                                                      \\
\end{longtable}
\normalsize

Mettez dans une casserole le beurre et la farine, mélangez bien sur feu doux,
mouillez avec le lait bouillant par petites quantités, sucrez, ajoutez les
jaunes d'œufs, l'eau de fleurs d'oranger, mélangez encore, puis faites cuire au
four très doux pendant une heure, en remuant fréquemment.

Étalez la crème dans un large plat, laissez-la refroidir, coupez-la ensuite en
losanges, saupoudrez-les de farine.

Battez l'œuf, passez dedans les losanges farinés, roulez-les ensuite dans de la
mie de pain rassis tamisée et faites-les frire dans de la friture claire très
chaude. Quand ils seront bien blonds, égouttez-les, dressez-les sur une
serviette et, au moment de servir, saupoudrez-les de sucre.

\section*{\centering Crème de semoule aux noix.}
\phantomsection
\addcontentsline{toc}{section}{ Crème de semoule aux noix.}
\index{Crème de semoule aux noix}

Pour huit personnes prenez :

\footnotesize
\begin{longtable}{rrrp{16em}}
  1 500 & grammes & de & crème,                                                                           \\
    200 & grammes & de & noix épluchées,                                                                  \\
    200 & grammes & de & noisettes épluchées,                                                             \\
    175 & grammes & de & sucre en poudre,                                                                 \\
    150 & grammes & de & semoule,                                                                         \\
    150 & grammes & de & confiture, au choix,                                                             \\
        &         & 30 & grammes d' amandes amères mondées,                                               \\
        &         &    & vanille.                                                                         \\
\end{longtable}
\normalsize

Pilez noix, noisettes et amandes avec {\ppp25\mmm} grammes de sucre,

Mettez la crème dans une casserole en porcelaine allant au feu, chauffez
légèrement, enlevez les peaux qui se forment à la surface, au fur et à mesure
qu'elles se produisent, réservez-les.

Ajoutez à la crème dépouillée le reste du sucre et la vanille, amenez
à ébullition, jetez dedans la semoule en pluie, laissez cuire jusqu'à
épaisissement ; puis enlevez la vanille, ajoutez noix, noisettes et amandes
pilées ; mélangez bien le tout.

Disposez dans un plat des couches successives de semoule et de peaux de
crème, saupoudrez de sucre, caramélisez à la pelle rougie ou au four, garnissez le
dessus avec la confiture et servez.

\section*{\centering Crème d'amandes au riz.}
\phantomsection
\addcontentsline{toc}{section}{ Crème d'amandes au riz.}
\index{Crème d'amandes au riz}

Mondez des amandes, pilez-les au mortier en les mouillant avec un peu de lait ;
passez au tamis.

Faites cuire du riz dans du lait sucré et aromatisé avec de la vanille ou de la
cannelle, au goût, de façon à conserver les grains de riz entiers. Retirez la
vanille ou la cannelle.

Mélangez le riz avec la purée d'amandes et de la crème douce plus ou moins
sucrée, ajoutez des raisins de Smyrne et mettez à la glace.

\section*{\centering Semoule aux amandes.}
\phantomsection
\addcontentsline{toc}{section}{ Semoule aux amandes.}
\index{Semoule aux amandes}

Pour dix personnes prenez :

\footnotesize
\begin{longtable}{rrrp{16em}}
    500 & grammes & de & semoule,                                                                         \\
    500 & grammes & de & sucre,                                                                           \\
    500 & grammes & de & lait,                                                                            \\
    250 & grammes & de & beurre,                                                                          \\
    250 & grammes & d' & amandes mondées,                                                                 \\
    250 & grammes & d' & eau,                                                                             \\
     20 & grammes & de & kirsch,                                                                          \\
     20 & grammes & de & grenadine.                                                                       \\
\end{longtable}
\normalsize

Préparez un sirop avec le sucre et l'eau, remuez-le, écumez-le et tenez-le au chaud.

Faites bouillir le lait.

Mettez dans une marmite en porcelaine allant au feu le beurre et les amandes
entières ou hachées, laissez-les blondir. ajoutez ensuite la semoule, le lait, le
sirop, le kirsch et la grenadine, réduisez jusqu'à épaississement convenable, laissez
refroidir.

\section*{\centering Crème pâtissière.}
\phantomsection
\addcontentsline{toc}{section}{ Crème pâtissière.}
\index{Crème pâtissière}

Prenez :

\footnotesize
\begin{longtable}{rrrp{16em}}
  250 & grammes & de & lait,                                                                              \\
  100 & grammes & de & sucre,                                                                             \\
   25 & grammes & de & farine,                                                                            \\
    1 & gramme  & de & sel fin,                                                                           \\
    3 & jaunes  & d' & œufs frais,                                                                        \\
      &         &    & vanille.                                                                           \\
\end{longtable}
\normalsize

Faites bouillir le lait avec la vanille que vous retirerez ensuite.

Mettez dans une casserole la farine, les jaunes d'œufs, le sucre et le sel,
mélangez bien, puis ajoutez le lait très chaud, par petites quantités, en
remuant. Portez la casserole sur le feu, chauffez doucement en tournant et
amenez la crème à bonne consistance.

Versez-la dans un plat et laissez-la refroidir.

\label{pg0856} \hypertarget{p0856}{}
\section*{\centering Crème anglaise vanillée.}
\phantomsection
\addcontentsline{toc}{section}{ Crème anglaise vanillée.}
\index{Crème anglaise vanillée}

Prenez :

\footnotesize
\begin{longtable}{rrrp{16em}}
    500 & grammes & de & lait,                                                                            \\
    500 & grammes & de & sucre,                                                                           \\
        &         & 14 & jaunes d'œufs,                                                                   \\
        &         &    & vanille.                                                                         \\
\end{longtable}
\normalsize

Faites bouillir le lait avec le sucre et une gousse de vanille. Éloignez la
casserole du feu, retirez la vanille, puis ajoutez les jaunes d'œufs. Mélangez
bien et faites épaissir sur feu doux ou au bain-marie.

\sk

\index{Crème anglaise aux vins liquoreux}
\index{Crème anglaise aux liqueurs}
On pourra préparer dans le même esprit des crèmes aux liqueurs, en remplaçant
la vanille par les liqueurs choisies ; rhum, kirsch, curaçao, anisette, cassis,
chartreuse, etc. ; ou par des vins liquoreux : frontignan, lunel, xérès,
chypre, marasquin, malvoisie, etc. ; mais, dans ce dernier cas, il faudra tenir
la crème plus ferme en y mettant plus d'œufs.

\sk

\index{Crème anglaise aux fruits}
On pourra faire aussi des crèmes aux fruits : abricots, cerises, fraises,
ananas, orange, citron, etc.

\label{pg0857} \hypertarget{p0857}{}
\section*{\centering Crème au chocolat.}
\phantomsection
\addcontentsline{toc}{section}{ Crème au chocolat.}
\index{Crème au chocolat}

Pour six personnes prenez :

\footnotesize
\begin{longtable}{rrrp{16em}}
    500 & grammes & de & lait,                                                                            \\
    125 & grammes & de & chocolat,                                                                        \\
     75 & grammes & de & sucre en morceaux,                                                               \\
      1 & gramme  & de & sel,                                                                             \\
        &         &  4 & jaunes d'œufs frais,                                                             \\
        &         &    & vanille.                                                                         \\
\end{longtable}
\normalsize

Faites bouillir le lait avec le sucre, le sel et la vanille. Enlevez la vanille
lorsque le lait sera suffisamment parfumé.

Râpez le chocolat, mettez-le dans le lait et laissez cuire à tout petit feu pendant
une demi-heure environ.

Éloignez la casserole du feu, puis ajoutez les jaunes d'œufs délayés avec
quelques gouttes de lait ; remettez la casserole sur le feu et faites épaissir
au bain-marie ou à feu doux, sans laisser bouillir. Surveillez cette dernière
opération pour éviter que la crème tourne.

Versez la crème au travers d'une passoire dans un plat ou dans un compotier ;
laissez-la refroidir.

\section*{\centering Crème au chocolat.}
\phantomsection
\addcontentsline{toc}{section}{ Crème au chocolat.}
\index{Crème au chocolat}
\index{Crème au chocolat (autre formule)}

\begin{center}
\textit{(Autre formule).}
\end{center}

Pour six personnes prenez :

\setlength\tabcolsep{.1em}
\footnotesize
\begin{longtable}{rrrrrp{18em}}
  & \hspace{2em} & 150 & grammes & de & chocolat fin,                                                     \\
  & \hspace{2em} & 125 & grammes & de & très bon beurre frais,                                            \\
  & \hspace{2em} &  90 & grammes & de & lait non écrémé,                                                  \\
  & \hspace{2em} &  75 & grammes & de & sucre,                                                            \\
  & \multicolumn{3}{r}{1 décigramme} & de & sel,                                                          \\
  & \hspace{2em} &     &         &  8 & jaunes d'œufs frais.                                              \\
\end{longtable}
\normalsize

Faites fondre, dans le lait, le sucre et le sel.

Ramollissez à chaleur très douce le beurre et le chocolat, pilez-les dans un
mortier tenu au bain-marie tiède : cela facilitera l'opération. Travaillez de
façon à obtenir une pâte homogène sans grumeaux, ajoutez-y le lait, ensuite les
jaunes d'œufs, un à un, en triturant et en malaxant bien après chaque addition.

Lorsque tous les éléments seront bien fondus ensemble et la crème devenue très
moelleuse, versez-la au travers d'une passoire, dans un compotier.

Laissez-la refroidir complètement ; puis servez-la avec des petits gâteaux.

C'est un excellent entremets sucré, très facile à préparer.

\section*{\centering Crème au chocolat.}
\phantomsection
\addcontentsline{toc}{section}{ Crème au chocolat.}
\index{Crème au chocolat}

\begin{center}
\textit{(Autre formule).}
\end{center}

Pour {\ppp10\mmm} à {\ppp12\mmm} personnes prenez :

\footnotesize
\begin{longtable}{rrrp{16em}}
    200 & grammes & de & bon chocolat,                                                                    \\
    150 & grammes & de & crème épaisse (rès fraîche,                                                      \\
    100 & grammes & de & sucre,                                                                           \\
     75 & grammes & de & lait,                                                                            \\
      1 & gramme  & de & sel,                                                                             \\
        &         &  6 & œufs frais,                                                                      \\
        &         &    & vanille.                                                                         \\
\end{longtable}
\normalsize

Faites bouillir le lait avec la vanille, le sucre et le sel ; retirez la
vanille.

Cassez les œufs, séparez les blancs des jaunes, enlevez les germes ; battez trois
blancs en neige.

Ramollissez le chocolat sur feu doux, délayez-le avec le lait et laissez-le
cuire pendant une dizaine de minutes, de façon qu'il n'existe pas de grumeaux.
Éloignez la casserole du feu et, lorsque son contenu sera un peu refroidi,
ajoutez les jaunes d'œufs ; mélangez bien.

Remettez la casserole sur le feu et faites cuire lentement à feu doux ou au
bain-marie jusqu'à épaississement convenable. Tournez pendant toute l'opération
et surveillez attentivement afin d'éviter que la crème tourne.

Laissez refroidir complètement, puis ajoutez, par petites quantités, la crème
et les blancs d'œufs en neige ; mélangez bien après chaque addition ; enfin
battez le tout.

Versez la crème dans un compotier, laissez-la reposer pendant deux ou trois
heures au frais, puis servez-la avec des biscuits, des gaufrettes légères ou
des oublies.

Cette crème est fine, légère, onctueuse, extrêmement parfumée et très digestible,

\section*{\centering Crème aux abricots.}
\phantomsection
\addcontentsline{toc}{section}{ Crème aux abricots.}
\index{Crème aux abricots}

Pour six personnes prenez :

\footnotesize
\begin{longtable}{rrrp{16em}}
    500 & grammes & de & crème Chantilly,                                                                 \\
    500 & grammes & d' & abricots bien mûrs,                                                              \\
    250 & grammes & de & sucre en poudre,                                                                 \\
        &         &  3 & feuilles de gélatine blanche,                                                    \\
        &         &    & lait.                                                                            \\
\end{longtable}
\normalsize

Coupez les abricots en morceaux, passez-les au tamis.

Retirez les amandes de six noyaux, mondez-les, piles-les au mortier avec un
peu de lait.

Rafraîchissez la gélatine dans un peu d'eau, faites-la fondre dans une très petite
quantité de lait.

Mélangez délicatement la pulpe des abricots, la pâte d'amandes des noyaux, le
sucre, la gélatine et la crème Chantilly ; versez le mélange dans un moule et
faites prendre sur glace.

Démoulez au moment de servir.

\sk

Comme variantes, on peut remplacer les abricots par d'autres fruits.

\section*{\centering Crème aux fruits.}
\phantomsection
\addcontentsline{toc}{section}{ Crème aux fruits.}
\index{Crème aux fruits}

Pour six personnes prenez :

\footnotesize
\begin{longtable}{rrrp{16em}}
    250 & grammes & de & crème,                                                                           \\
     50 & grammes & de & sucre en poudre,                                                                 \\
     30 & grammes & de & kirsch,                                                                          \\
     30 & grammes & d' & orgeat,                                                                          \\
     30 & grammes & de & beurre,                                                                          \\
     15 & grammes & de & farine,                                                                          \\
        &         &  2 & pêches,                                                                          \\
        &         &  2 & poires.                                                                          \\
\end{longtable}
\normalsize

Mettez dans une casserole la crème, {\ppp25\mmm} grammes de sucre en poudre, la farine,
mélangez bien le tout ensemble et faites cuire à chaleur douce,

Dès que l'appareil devient serré, ajoutez le beurre, le kirsch et l'orgeat ;
travaillez bien le mélange.

Pelez les fruits, coupez-les en morceaux, pochez-les dans de l'eau bouillante,
puis mettez-les dans la crème, saupoudrez avec le reste du sucre, poussez un
instant au four et servez.

\sk

\index{Crème aux abricots}
On peut, cela va sans dire, varier au goût les fruits et les parfums.

\section*{\centering Crème renversée à la vanille.}
\phantomsection
\addcontentsline{toc}{section}{ Crème renversée à la vanille.}
\index{Crème renversée à la vanille}

Pour six personnes prenez :

\footnotesize
\begin{longtable}{rrrp{16em}}
    250 & grammes & de & sucre,                                                                           \\
      2 & grammes & de & sel,                                                                             \\
        & 1 litre & de & lait,                                                                            \\
        &         &  5 & œufs frais,                                                                      \\
        &         &  7 & jaunes d'œufs frais,                                                             \\
        &         &    & vanille.                                                                         \\
\end{longtable}
\normalsize

Faites bouillir le lait avec le sucre, le sel et la vanille ; lorsqu'il est bien
parfumé, retirez la vanille,

Cassez dans une terrine les œufs entiers, ajoutez les jaunes, versez dessus le lait
par petites quantités ; mélangez bien.

Passez le mélange au travers d'une mousseline ou d'une passoire fine.

Garnissez un moule ou des petits pots avec l'appareil passé et faites cuire au
four moyennement chaud ou au bain-marie. Laissez refroidir.

Au moment de servir, démoulez la crème sur un plat, ou servez-la telle que
dans les petits pots.

\sk

\index{Crème renversée, à différents parfums}
On pourra préparer de même des crèmes renversées à d'autres parfums : café,
thé, caramel, liqueurs, etc. ; mais alors il faudra tenir compte de la quantité du
liquide employé comme parfum. On diminuera d'autant le volume du lait afin
que l'équilibre des proportions des éléments soit conservé.

\section*{\centering Crème glacée.}
\phantomsection
\addcontentsline{toc}{section}{ Crème glacée.}
\index{Crème glacée}

Les crèmes glacées sont d'excellents entremets sucrés faciles à préparer.

En voici un exemple.

\medskip

Pour huit personnes prenez :

\footnotesize
\begin{longtable}{rrrp{16em}}
    500 & grammes & de & crème Chantilly\footnote{\index{Crème Chantilly}
                                    La crème Chantilly est de la crème
                                    fouettée avec du sucre vanillé. Voici comment on
                                    la prépare : pour que la crème devienne légère et
                                    ferme sous l'action du fouet, on la fait d'abord
                                    séjourner pendant {\ppp24\mmm} heures sur de la
                                    glace, puis on rejette les parties aqueuses qui
                                    se sont séparées des parties solides ; on termine
                                    en fouettant, sur glace, la crème délayée avec du
                                    sucre vanillé.},                                                      \\
    250 & grammes & de & crème épaisse,                                                                   \\
        &         &    & gelée de groseilles framboisée,                                                  \\
        &         &    & anisette.                                                                        \\
\end{longtable}
\normalsize

Battez ensemble les deux crèmes et emplissez-en un moule huilé. Fermez le
moule en interposant entre la partie supérieure et le couvercle un linge qui rendra
la fermeture étanche, et mettez-le pendant trois heures dans un mélange réfrigérant
de glace pilée et de sel marin.

Démoulez sur un plat, décorez le dessus de la crème avec de la gelée de
groseilles framboisée et entourez-en le pied avec un sirop concentré préparé
avec de la gelée de groseilles framboisée aromatisée avec de l’anisette, par
exemple, et rafraîchi.

\sk

\index{Crème glacée aux fruits}
Comme variantes, on pourra remplacer la crème non fouettée par des marmelades
ou des confitures épaisses de fraises ou d'autres fruits. On conçoit aisément
une grande quantité d'entremets analogues en variant le remplissage, le décor
et le parfum.

\index{Crème glacée aux marrons}
On pourra de même faire une crème glacée aux marrons, en remplaçant la crème
épaisse par de la purée de marrons vanillée,
\hyperlink{p0902}{p. \pageref{pg0902}}. On garnira le plat avec des marrons
glacés.

Tous ces entremets sont très moelleux. On les sert avec des gaufrettes, des
biscuits, des macarons, des crêpes bretonnes séchées au four, etc.

\section*{\centering Bavarois.}
\phantomsection
\addcontentsline{toc}{section}{ Bavarois.}
\index{Bavarois}
\index{Bavarois (Définition des)}
\index{Bavarois (Différentes manières d'apprêter les)}

Les bavarois sont des entremets sucrés, glacés, à base de crèmes aromatisées et
de crème fouettée. On fait des bavarois à différents parfums : vanille, anis,
rose, violette, fleur d'oranger, œillet, etc. ; aux liqueurs, menthe, anisette,
curaçao, kirsch, rhum, marasquin, etc. ; au punch ; aux fruits, aux purées et
aux marmelades de fruits : abricots, fraises, framboises, cerises, groseilles,
prunes, pêches, citron, orange, cédrat, ananas, noix, noisettes, amandes,
pistaches, marrons, etc. ; ou encore au café, au thé, au chocolat, aux
pralines, etc. ; enfin, on y fait entrer parfois des biscuits.

Voici deux exemples de bavarois.

\medskip

\section*{\centering Bavarois praliné.}
\phantomsection
\addcontentsline{toc}{section}{ Bavarois praliné.}
\index{Bavarois praliné}
\index{Appareil bararois}

Pour six personnes prenez :

\footnotesize
\begin{longtable}{rrrp{16em}}
    350 & grammes & de & lait,                                                                            \\
    250 & grammes & de & crème fouettée,                                                                  \\
    150 & grammes & de & sucre,                                                                           \\
    100 & grammes & d' & amandes douces mondées,                                                          \\
     40 & grammes & de & kirsch,                                                                          \\
      5 & grammes & de & gélatine pure en feuilles,                                                       \\
      1 & gramme  & de & sel,                                                                             \\
        &         &  3 & jaunes d'œufs frais,                                                             \\
        &         &    & vanille.                                                                         \\
\end{longtable}
\normalsize

Préparez d'abord le pralinage\footnote{Il est toujours préférable de préparer
soi-même le pralinage plutôt que de prendre des pralines du commerce.
Cependant, on peut remplacer les amandes douces et le sucre qui entrent dans le
pralinage par {\ppp200\mmm} grammes de pralines grises.}. Faites fondre
{\ppp100\mmm} grammes de sucre dans un poêlon avec un peu d'eau, ajoutez les
amandes, laissez-les cuire jusqu'à ce qu'elles soient dorées. Mettez les
amandes bien enrobées de sucre sur un marbre huilé, sans qu'elles se touchent ;
laissez-les refroidir. Pilez-les ensuite au mortier.

Préparez une crème anglaise \hyperlink{p0856}{p. \pageref{pg0856}} avec le lait,
les jaunes d'œufs, le sucre, le sel et de la vanille.

Trempez la gélatine dans un peu d'eau, ajoutez-la à la crème anglaise, mélangez
bien jusqu'à ce que la gélatine soit dissoute, passez la crème, laissez-la
refroidir.

Incorporez à la crème anglaise le kirsch, les pralines pilées, puis la crème
fouettée, en soulevant la masse.

Huilez un moule à manqué ou un moule à biscuit de Savoie, versez dedans
l'appareil bavarois et mettez à la glace pendant deux heures.

Démoulez le bavarois sur un plat refroidi garni d'une serviette et servez.

\section*{\centering Bavarois aux cerises.}
\phantomsection
\addcontentsline{toc}{section}{ Bavarois aux cerises.}
\index{Bavarois aux cerises}
\index{Appareil bararois}

Pour six personnes prenez :

\footnotesize
\begin{longtable}{rrrp{16em}}
    450 & grammes & de & confiture ou de gelée de cerises.                                                \\
    250 & grammes & de & crème fouettée,                                                                  \\
    250 & grammes & de & crème épaisse,                                                                   \\
     40 & grammes & de & marasquin,                                                                       \\
      5 & grammes & de & gélatine pure en feuilles,                                                       \\
      1 & gramme  & de & sel,                                                                             \\
        &         &  4 & jaunes d'œufs.                                                                   \\
\end{longtable}
\normalsize

Faites chauffer la crème épaisse ; au premier bouillon, éloignez la casserole
du feu, puis mettez dedans les jaunes d'œufs et le sel ; mélangez sur feu doux
et amenez le tout à bonne consistance ; ne faites pas bouillir. Ajoutez ensuite
la gélatine après l'avoir trempée dans un peu d'eau ; mélangez bien jusqu'à
complète dissolution. Passez la crème ; laissez-la refroidir.

Incorporez à cette crème {\ppp200\mmm} grammes de confiture ou de gelée de
cerises délayée avec le marasquin, puis la crème fouettée, en soulevant
l'appareil,

Huilez un moule. Versez dedans l'appareil bavarois aux cerises, mettez à la
glace pendant une heure et demie à deux heures.

Démoulez le bavarois sur un compotier, entourez-en le pied avec le reste de la
confiture ou de la gelée de cerises et servez.

\sk

\index{Bavarois aux fruits crus ou aux fruits cuits}
On pourra préparer dans le même esprit des bavarois à d'autres fruits,
à d'autres confitures, et avec des jus et des purées de fruits frais.

Lorsqu'on emploiera des purées ou des jus de fruits frais, il conviendra de se
servir de moules en porcelaine ou en verre, inattaquables aux acides, au lieu
de prendre des moules en métal.

\section*{\centering Charlotte\footnote{ \index{Charlottes (Définition des)}
                                         \index{Définition des charlottes.}
On désigne sous le nom de charlottes des gâteaux dont l'extérieur est formé
généralement pur des tranches de mie de pain ou des biscuits (la brioche peut
aussi servir à cet usage), et dont l'intérieur est constitué par des marmelades
de fruits ou des crèmes.} aux pommes.}

\phantomsection
\addcontentsline{toc}{section}{ Charlotte aux pommes.}
\index{Charlotte aux pommes}

Pour six personnes prenez :

\footnotesize
\begin{longtable}{rrrp{16em}}
  1 250 & grammes & de & pommes reinettes,                                                                \\
    300 & grammes & de & confiture d'abricots,                                                            \\
    200 & grammes & de & sucre,                                                                           \\
    150 & grammes & de & beurre frais,                                                                    \\
    100 & grammes & d' & eau,                                                                             \\
     75 & grammes & de & kirsch,                                                                          \\
        &         &    & pain de mie anglais,                                                             \\
        &         &    & sirop.                                                                           \\
\end{longtable}
\normalsize

Pelez les pommes, coupez-les en tranches assez épaisses, enlevez les pépins et
les parties cartilagineuses du centre des fruits.

Mettez dans une casserole plate les pommes, le sucre, l'eau et {\ppp75\mmm}
grammes de beurre ; faites cuire à petit feu en remuant souvent.

Parez le pain de mie, enlevez dessus une tranche de {\ppp6\mmm} millimètres
d'épaisseur ; mettez-la de côté. Coupez une partie du pain de mie en tranches
triangulaires ou en croissants, d'un demi-centimètre d'épaisseur, qui serviront
à couvrir le fond du moule à charlotte que vous allez employer. Détaillez le
reste du pain en tranches rectangulaires de {\ppp3\mmm} centimètres de largeur
environ, de même épaisseur que les précédentes et de la hauteur du moule.

Liquéfiez le reste du beurre.

Prenez un moule à charlotte de {\ppp10\mmm} centimètres de diamètre ; garnissez
le fond avec un disque de papier blanc résistant ; disposez dessus les tranches
triangulaires ou les croissants trempés dans du beurre et égouttés ; ayez soin
que les tranches ou les croissants se touchent bien. Dressez régulièrement et
bien droit sur tout le pourtour du moule les tranches de pain rectangulaires
beurrées d'un côté, le côté beurré appliqué contre la paroi interne du moule,
en les faisant chevaucher les unes sur les autres.

Lorsque les pommes sont cuites et le jus bien réduit, incorporez-y en fouettant
le reste du beurre, puis {\ppp200\mmm} grammes de confiture d'abricots et
enfin, hors du feu, {\ppp60\mmm} grammes de kirsch ; mélangez bien avec une
spatule en bois.

Versez l'appareil dans le moule, tassez-le bien pour qu'il n'y ait pas de
vides, couvrez avec la tranche de pain réservée et faites cuire au four doux
pendant une demi-heure environ.

Délayez, dans du sirop chaud, le reste de la confiture passée et parfumez avec le
reste du kirsch.

Démoulez la charlotte sur un plat en argent, arrosez-la avec le sirop
kirsch-abricot et servez.

\section*{\centering Charlotte au chocolat.}
\phantomsection
\addcontentsline{toc}{section}{ Charlotte au chocolat.}
\index{Charlotte au chocolat}

Pour six personnes prenez :

\footnotesize
\begin{longtable}{rrrp{16em}}
    250 & grammes & de & chocolat,                                                                        \\
    200 & grammes & de & biscuits à la cuiller\footnote{ On pourra remplacer les biscuits
                         par de la brioche qu'on détaillera comme il est indiqué pour le
                         pain dans la charlotte aux pommes.},                                             \\
    125 & grammes & de & beurre fin,                                                                      \\
    125 & grammes & de & crème fouettée vanillée,                                                         \\
     80 & grammes & de & sucre,                                                                           \\
      1 & gramme  & de & sel,                                                                             \\
        &         &  3 & œufs frais,                                                                      \\
        &         &    & crème à la vanille.                                                              \\
\end{longtable}
\normalsize

Cassez les œufs, séparez les blancs des jaunes ; battez les blancs en neige.

Faites fondre dans un peu d'eau {\ppp200\mmm} grammes de chocolat, le sel et le
sucre, ajoutez ensuite le beurre et les jaunes d'œufs ; travaillez bien le
mélange sur feu doux, mais ne faites pas bouillir. Éloignez la casserole du feu
et, lorsque son contenu sera un peu refroidi, incorporez-y les blancs d'œufs.
Laissez refroidir complètement l'appareil au chocolat.

\label{pg0865} \hypertarget{p0865}{}
Prenez un moule à charlotte, mettez au fond un disque de papier blanc,
garnissez d'abord le fond du moule avec des biscuits taillés en forme de poires
et réunis au milieu par un petit rond en biscuit, puis les parois avec des
biscuits coupés régulièrement aux deux bouts, en les faisant chevaucher.

Mélangez la crème fouettée à l'appareil au chocolat en soulevant la masse avec
une spatule, versez le mélange dans le moule, couvrez avec le reste des
biscuits et mettez à la glace pendant une heure au moins.

Démoulez la charlotte sur un plat, glacez-la avec le reste du chocolat fondu
dans un peu d'eau, et servez en envoyant en même temps de la crème à la
vanille, \hyperlink{p0868}{p. \pageref{pg0868}}.

\section*{\centering Charlotte de crème fouettée au café.}
\phantomsection
\addcontentsline{toc}{section}{ Charlotte de crème fouettée au café.}
\index{Charlotte de crème fouettée au café}

Pour six personnes prenez :

\footnotesize
\begin{longtable}{rrrp{16em}}
    300 & grammes & de & crème fouettée.                                                                  \\
    200 & grammes & de & biscuits à la cuiller,                                                           \\
     60 & grammes & de & bonbons en forme de grains de café, contenant du sirop de café,                  \\
        &         &    & essence de café.                                                                 \\
\end{longtable}
\normalsize


Chemisez un moule à charlotte avec les biscuits. comme il est dit
\hyperlink{p0865}{p. \pageref{pg0865}} ; versez dans l'intérieur la crème fouettée
parfumée avec de l'essence de café, par couches dans lesquelles vous noierez
les bonbons au café ; couvrez avec le reste des biscuits. Mettez à la glace
pendant une heure au moins.

Démoulez sur un plat et servez.

\section*{\centering Charlotte à l'ananas.}
\phantomsection
\addcontentsline{toc}{section}{ Charlotte à l'ananas.}
\index{Charlotte à l'ananas}

Pour six personnes prenez :

\footnotesize
\begin{longtable}{rrrp{16em}}
    350 & grammes & de & lait,                                                                            \\
    250 & grammes & de & crème Chantilly,                                                                 \\
    250 & grammes & de & gelée d'ananas,                                                                  \\
    200 & grammes & de & biscuits à la cuiller,                                                           \\
     90 & grammes & de & sucre,                                                                           \\
      5 & grammes & de & gélatine pure en feuilles,                                                       \\
        &         &  1 & gramme de sel,                                                                   \\
        &         &  4 & jaunes d'œufs,                                                                   \\
        &         &    & ananas,                                                                          \\
        &         &    & curaçao.                                                                         \\
\end{longtable}
\normalsize

Préparez une crème de bonne consistance avec le lait, les jaunes d'œufs, le
sucre et le sel, ajoutez-y la gélatine après l'avoir rafraîchie dans de l'eau.
Passez la crème, laissez-la refroidir.

\index{Bavarois à l'ananas}
Incorporez à cette crème la gelée d'ananas et du curaçao au goût, puis la crème
fouettée, en soulevant la masse ; vous aurez ainsi un bavarois à l'ananas.

Chemisez un moule à charlotte avec les biscuits, comme il est indiqué
\hyperlink{p0865}{p. \pageref{pg0865}}, aspergez l'intérieur avec un peu de
curaçao, versez dans le moule l'appareil bavarois, par couches successives que
vous parsèmerez de petits cubes d’ananas ; couvrez avec le reste des biscuits
et mettez à la glace pendant une heure et demie à deux heures.

Démoulez la charlotte sur un plat et servez.

\sk

\index{Charlotte à la macédoine de fruits}
On pourra préparer dans le même esprit des charlottes à toute espèce de fruits
et avec des macédoines de fruits.

Lorsqu'on emploiera des fruits frais, on épluchera d’abord les fruits, puis on
les passera au tamis. La purée obtenue sera incorporée à la crème Chantilly.

\section*{\centering Cœur de nègre.}
\phantomsection
\addcontentsline{toc}{section}{ Cœur de nègre.}
\index{Cœur de nègre}

Pour quatre personnes prenez :

\footnotesize
\begin{longtable}{rrrp{16em}}
    250 & grammes & de & crème Chantilly moulée dans un cœur,                                             \\
    125 & grammes & de & chocolat fin vanillé ou non,                                                     \\
     50 & grammes & de & beurre fin,                                                                      \\
        &         & 12 & biscuits légers, de préférence des pailles d'or,                                 \\
        &         &  2 & jaunes d'œufs frais.                                                             \\
\end{longtable}
\normalsize

Mettez le chocolat avec quelques cuillerées d'eau dans une casserole, chauffez
jusqu'à ce qu'il soit complètement fondu, ajoutez ensuite le beurre et les
jaunes d'œufs, travaillez sur feu doux puis laissez refroidir.

Dressez le cœur de crème sur un plat, masquez-le avec le chocolat et insérez à
la base les biscuits.

C'est un entremets très agréable et très léger, dont la préparation demande peu
de temps.

\section*{\centering Œufs à la neige.}
\phantomsection
\addcontentsline{toc}{section}{ Œufs à la neige.}
\index{Œufs à la neige}
\index{Crème anglaise à la vanille}

Pour six personnes prenez :

\footnotesize
\begin{longtable}{rrrp{16em}}
    125 & grammes & de & sucre,                                                                           \\
      3 & grammes & de & sel blanc,                                                                       \\
        & 1 litre & de & lait,                                                                            \\
        &         &  5 & œufs frais de grosseur moyenne,                                                  \\
        &         &  1 & gousse de vanille.                                                               \\
\end{longtable}
\normalsize

Cassez les œufs, séparez les blancs des jaunes ; battez les blancs en neige.

Faites bouillir le lait avec le sucre, le sel et la vanille, retirez la
vanille, puis, dans ce lait, faites cuire par petites portions les blancs
battus en neige que vous prendrez avec une cuiller à sauce et que vous
plongerez dans le lait. Au bout de quelques secondes, elles monteront en boules
avec le lait, retournez-les, elles monteront de nouveau. Enlevez-les alors avec
une écumoire et mettez-les sur un plat.

\label{pg0868} \hypertarget{p0868}{}
Préparez la crème.

Laissez bouillir encore le lait pendant une dizaine de minutes pour le
concentrer, puis éloignez la casserole du feu.

Battez les jaunes avec un peu de lait bouilli, versez-les dans le lait.
Remettez la casserole sur un feu doux et tournez constamment jusqu'à ce que le
tout prenne la consistance d’une crème épaisse ; surveillez bien, car c'est le
moment psychologique de l'opération : si on laisse l'appareil trop longtemps
sur le feu, la sauce tourne.

Passez la sauce sur les blancs : laissez refroidir.

La même gousse de vanille peut servir pour sept ou huit opérations : la
première fois, on ne la laisse que pendant deux ou trois bouillons, la dernière
fois, on la laissera pendant toute la durée de la préparation.

\sk

\index{Fondant au caramel}
Comme variante, on pourra glacer les blancs avec le fondant suivant :

\footnotesize
\begin{longtable}{rrrp{16em}}
    100 & grammes & de & sucre,                                                                           \\
     40 & grammes & d' & eau,                                                                             \\
     30 & grammes & de & beurre fin.                                                                      \\
\end{longtable}
\normalsize

Faites fondre le sucre dans l'eau, amenez-le à l'état de caramel blond ;
ajoutez ensuite le beurre, mélangez bien. Versez le fondant sur les œufs.
Laissez refroidir.

\section*{\centering Omelette soufflée.}
\phantomsection
\addcontentsline{toc}{section}{ Omelette soufflée.}
\index{Omelette soufflée}

Pour six personnes prenez :

\footnotesize
\begin{longtable}{rrrp{16em}}
    120 & grammes & de & sucre en poudre,                                                                 \\
     15 & grammes & d' & un parfum quelconque, kirsch, rhum, vanille, citron, etc.,                       \\
      1 & gramme  & de & sel fin,                                                                         \\
        &         &  6 & blancs d'œufs,                                                                   \\
        &         &  3 & jaunes d'œufs,                                                                   \\
        &         &    &  beurre.                                                                         \\
\end{longtable}
\normalsize

Battez longuement les jaunes d'œufs avec {\ppp100\mmm} grammes de sucre et le
parfum choisi.

Fouettez les blancs additionnés de {\ppp10\mmm} grammes de sucre, amenez-les
à être très fermes ; au milieu de l'opération ajoutez le sel qui contribuera
à les empêcher de tomber.

Prenez un quart des blancs, mêlez-le avec les jaunes, puis versez le tout dans
le reste des blancs ; achevez le mélange rapidement en coupant la pâte sans la
faire tomber.

Beurrez légèrement un plat creux allant au feu, poudrez-en la surface avec un
peu de sucre et laissez tomber dedans la pâte d'un seul coup. Relevez
l'appareil le plus possible dans le plat, faites avec un couteau à lame épaisse
une entaille de {\ppp3\mmm} centimètres de profondeur sur presque toute la
longueur de la partie médiane et quelques boutonnières sur les côtés ; enfin
mettez au four chaud, de manière qu'il y ait plus de chaleur dessous que
dessus ; surveillez la cuisson. Au bout de {\ppp5\mmm} minutes tournez le plat
bout à bout pour que l'omelette cuise uniformément des deux côtés, saupoudrez
de sucre et achevez la cuisson, qui dure en tout {\ppp10\mmm} minutes environ.

Si au bout de quelques minutes, vous vous apercevez que la chaleur du four
est trop forte, couvrez l’omelette avec un papier beurré.

Servez immédiatement au sortir du four.

\section*{\centering Omelette soufflée en surprise.}
\phantomsection
\addcontentsline{toc}{section}{ Omelette soufflée en surprise.}
\index{Omelette soufflée en surprise}

Ce ravissant entremets, paradoxal en apparence, constitué par une glace servie
dans une omelette soufflée brûlante, est un joli corollaire de la découverte
faite, en {\ppp1804\mmm}, par le physicien américain comte Thomson de Rumfort,
de la propriété du blanc d'œuf fouetté d'être mauvais conducteur de la chaleur.

La première application gastronomique de ce fait expérimental a été la glace
meringuée, préparée en enrobant une bombe glacée dans des blancs d'œufs
fouettés en neige ferme avec du sucre en poudre, que l’on faisait prendre au
four.

Plus tard, il y a une quarantaine d'années à peine, on eut l'idée de remplacer
le blanc d'œuf par l'appareil à omelette soufflée qui ne contient que deux
tiers de blanc et ainsi fut créée l'omelette soufflée en surprise.

\medskip

Voici la technique de l'opération.

Préparez d'abord un appareil à omelette soufflée, puis prenez un plat allant au
feu, foncez-le avec un quart de l'appareil ; mettez délicatement dessus une
glace aux fruits ou à un parfum quelconque, moulée à des dimensions
horizontales en rapport avec celles du plat employé et dont l'épaisseur pourra
être de {\ppp3\mmm} à {\ppp4\mmm} centimètres ; couvrez-la avec le reste de
l'appareil et lissez le dessus à la spatule. Posez ensuite le plat dans un
autre ustensile empli de glace pilée et enfournez le tout dans un four
fortement chauffé par-dessus (le four à gaz convient parfaitement pour cette
opération). Au bout de cinq minutes, l'omelette sera soufflée. Saupoudrez-la
d'un peu de sucre parfumé à la vanille, décorez-la rapidement, au moyen d'un
cornet, avec de la gelée de confitures, et servez aussitôt. Il est inutile dans
cette préparation de beurrer le plat et de faire des entailles dans l'appareil,
la soufflure étant produite ici par la chaleur d'en haut.

Les personnes qui craindraient de ne pouvoir opérer assez rapidement,
seront certaines de réussir du premier coup en augmentant l'épaisseur de
la couche isolante au moyen de biscuits à la cuiller. Voici comment il faudra
procéder.

Coupez en deux des biscuits à la cuiller de façon à réduire leur épaisseur de
moitié. Foncez un plat avec le quart de l'appareil, disposez dessus une couche
de demi-biscuits, mettez la glace par-dessus, au-dessus une autre couche de
demi-biscuits, puis le reste de l'appareil et faites cuire comme précédemment.

L'omelette soufflée en surprise préparée avec des biscuits est un peu moins
moelleuse que celle qui n'en contient pas.

\sk

On concoit facilement de nombreuses modifications dans les détails. Comme
variante, on pourra, par exemple, mettre au-dessus de la glace des fraises et de
la crème fouettée avant de couvrir avec l'appareil. Ce qui m'a paru intéressant
à préciser, c'est le principe de ces entremets curieux.

\section*{\centering Soufflés.}
\phantomsection
\addcontentsline{toc}{section}{ Soufflés.}
\index{Soufflés}

Ces entremets légers sont très variés. On fait des soufflés à la crème, au
chocolat, au café, à la vanille, aux violettes, aux pistaches ; aux fruits
frais de toutes sortes : fraises, framboises, cerises, groseilles, airelles,
abricots, pêches, prunes, poires, pommes, coings, grenades, ananas, orange,
mandarine, citron, melon, etc. ; aux fruits cuits ; confitures et marmelades ;
aux fruits confits ; aux biscuits. On les parfume à toutes liqueurs. On les
dresse généralement en timbales ou en moule spécial beurré et saupoudré de
sucre, ou encore dans des plats, des coquilles, des croustades et même dans des
savarins.

\section*{\centering Soufflé à la crème vanillée.}
\phantomsection
\addcontentsline{toc}{section}{ Soufflé à la crème vanillée.}
\index{Soufflé à la crème vanillée}

Pour six personnes prenez :

\footnotesize
\begin{longtable}{rrrp{16em}}
   175 & grammes & de & lait,                                                                             \\
   100 & grammes & de & sucre,                                                                            \\
    25 & grammes & de & beurre fin,                                                                       \\
    20 & grammes & de & fécule de pommes de terre,                                                        \\
     1 & gramme  & de & sel,                                                                              \\
       &         &  6 & blancs d'œufs frais,                                                              \\
       &         &  3 & jaunes d'œufs frais,                                                              \\
       &         &    & vanille,                                                                          \\
       &         &    & sucre glace.                                                                      \\
\end{longtable}
\normalsize

Faites bouillir le lait avec le sucre, le sel et la vanille ; retirez la
vanille, laissez un peu refroidir.

Délayez la fécule et le beurre dans le lait, chauffez en tournant ; au premier
bouillon éloignez la casserole du feu ; ajoutez les jaunes d'œufs, mélangez, puis
incorporez les blancs battus en neige ferme.

Mettez l'appareil dans un plat creux, une timbale ou un moule à soufflé et
faites cuire au four moyennement chaud pendant {\ppp15\mmm} à {\ppp20\mmm}
minutes au plus. Surveillez la cuisson, car le soufflé retombe sil est trop
cuit.

Avant de le sortir du four, poudrez-le de sucre fin pour le glacer.

Servez chaud,

\sk

En remplaçant la vanille par d'autres parfums ou par des liqueurs, on aura
toute une série de soufflés de goûts différents.

\section*{\centering Soufflé au chocolat.}
\phantomsection
\addcontentsline{toc}{section}{ Soufflé au chocolat.}
\index{Soufflé au chocolat}

Pour six personnes prenez :

\footnotesize
\begin{longtable}{rrrp{16em}}
    125 & grammes & de & chocolat,                                                                        \\
    120 & grammes & de & sucre en poudre,                                                                 \\
     30 & grammes & de & lait,                                                                            \\
      1 & gramme  & de & sel fin,                                                                         \\
        &         &  6 & blancs d'œufs,                                                                   \\
        &         &  3 & jaunes d'œufs.                                                                   \\
\end{longtable}
\normalsize

Dissolvez à petit feu le chocolat dans le lait, éloignez la casserole du feu,
ajoutez les jaunes d'œufs battus avec {\ppp100\mmm} grammes de sucre, mélangez.

Fouettez les blancs avec le reste du sucre et le sel, incorporez-les au mélange.

Mettez l'appareil dans un moule à soufflé ou un moule à charlotte et faites
cuire au four moyen pendant {\ppp15\mmm} à {\ppp20\mmm} minutes.

\section*{\centering Soufflé aux fraises.}
\phantomsection
\addcontentsline{toc}{section}{ Soufflé aux fraises.}
\index{Soufflé aux fraises}

Pour six personnes prenez :

\footnotesize
\begin{longtable}{rrr>{\raggedright\arraybackslash}p{16em}}
    175 & grammes & de & lait,                                                                            \\
    125 & grammes & de & sucre,                                                                           \\
    125 & grammes & de & \hangindent=1em purée de fraises obtenue en passant au tamis de
                         Venise un mélange d'un tiers de fraises Héricart et de deux tiers
                         de fraises des bois,                                                             \\
     20 & grammes & de & fécule de pommes de terre,                                                       \\
        &         & 12 & belles fraises Héricart,                                                         \\
        &         &  6 & blancs d'œufs frais,                                                             \\
        &         &  4 & jaunes d'œufs frais,                                                             \\
        &         &    & curaçao.                                                                         \\
\end{longtable}
\normalsize

Mettez le lait, la fécule et le sucre dans une casserole. Chauffez en
tournant ; au premier bouillon, éloignez la casserole du feu, ajoutez les
jaunes d'œufs, mélangez ; puis incorporez la purée de fraises, les blancs
d'œufs battus en neige ferme et du curaçao au goût.

Versez l'appareil dans une timbale ou dans un moule à soufflé et faites cuire au
four modérément chaud pendant une vingtaine de minutes.

Lorsque la cuisson est presque terminée, saupoudrez le soufflé d’un peu de
sucre fin et disposez dessus les douze belles fraises Héricart,

Servez chaud.

\sk

Comme variantes, on pourra faire ce soufflé à d'autres parfums et à d'autres
liqueurs : fraises, orange et kirsch, par exemple.

\sk

On préparera de même tous les soufflés aux fruits frais : abricots, pêches,
ananas, prunes, etc.

\sk

On pourra faire des soufflés aux fruits plus simples en employant de la pulpe
de fruits passée au tamis et cuite pendant une minute ou deux dans un poids égal
de sucre au cassé. On versera cet appareil sur des blancs d'œufs battus en neige
ferme et on fera cuire comme un soufflé ordinaire.

\section*{\centering Soufflé aux cerises.}
\phantomsection
\addcontentsline{toc}{section}{ Soufflé aux cerises.}
\index{Soufflé aux cerises}

Pour six personnes prenez :

\footnotesize
\begin{longtable}{rrrp{16em}}
    250 & grammes & de & cerises privées de leurs noyaux, cuites au sirop
                         très concentré et égouttées, ou 200 grammes de
                         confiture de cerises très serrée,                                                \\
    200 & grammes & de & lait,                                                                            \\
     30 & grammes & de & sucre,                                                                           \\
     25 & grammes & de & fécule de pommes de terre,                                                       \\
        &         &  6 & blancs d'œufs frais,                                                             \\
        &         &  4 & jaunes d'œufs frais,                                                             \\
        &         &    & cerises confites ou gelée de groseilles framboisée,                              \\
        &         &    & kirsch et marasquin.                                                             \\
\end{longtable}
\normalsize

Préparez, avec le lait, la fécule, le sucre, les jaunes d'œufs, les blancs battus
en neige, du kirsch et du marasquin, au goût, un appareil à soufflé comme il est
dit ci-dessus. Incorporez-y les cerises cuites au sirop ou la confiture.

Versez l'appareil dans une timbale ou dans un plat creux et faites cuire au four
moyennement chaud.

Au dernier moment, disposez sur le soufflé une bordure de cerises confites ou
décorez avec de la gelée de groseilles framboisée.

\sk

On pourra faire de même des soufflés à toutes sortes de confitures, de fruits
cuits ou de marmelades de fruits sans jus.

\section*{\centering Soufflé aux fruits confits.}
\phantomsection
\addcontentsline{toc}{section}{ Soufflé aux fruits confits.}
\index{Soufflé aux fruits confits}

Faites macérer des fruits confits, entiers s'ils sont petits, coupés en petits cubes
s'ils sont gros, dans une liqueur appropriée.

Préparez un appareil de soufflé à la crème, parfumé au goût ; incorporez-y les
fruits et faites cuire comme à l'ordinaire.

Garnissez ou accompagnez le soufflé avec des fruits frais assortis.

\section*{\centering Soufflé à l'orange.}
\phantomsection
\addcontentsline{toc}{section}{ Soufflé à l'orange.}
\index{Soufflé à l'orange}

Pour quatre personnes prenez :

\footnotesize
\begin{longtable}{rrrp{16em}}
    100 & grammes & de & sucre en poudre,                                                                 \\
     15 & grammes & de & fécule de pommes de terre,                                                       \\
     15 & grammes & de & lait,                                                                            \\
        &         &  4 & blancs d'œuf,                                                                    \\
        &         &  2 & jaunes d'œufs,                                                                   \\
        &         &  2 & oranges entières,                                                                \\
        &         &    & zeste d'une orange.                                                              \\
\end{longtable}
\normalsize

Hachez le zeste d'orange.

Battez les blancs d'œufs en neige ferme.

Coupez les oranges en deux, enlevez-en délicatement la pulpe. exprimez-en le
jus, réservez-le ; réservez les calottes d'écorce d'oranges.

Délayez la fécule dans le lait, ajoutez le jus d'orange et le sucre ; faites
bouillir. Éloignez la casserole du feu. mettez les jaunes d'œufs et le zeste
haché (l'appareil doit être ferme), puis incorporez-y les blancs battus en
neige.

Emplissez les calottes d'écorce avec cette crème, faites cuire au four,
à chaleur assez vive, et servez.

\section*{\centering Soufflés en croustades.}
\phantomsection
\addcontentsline{toc}{section}{ Soufflés en croustades.}
\index{Soufflés en croustades}

Tous les soufflés peuvent être présentés en croustades.

Garnissez un moule à croustade, de forme basse et bien beurré, avec une couche
mince de pâte à brioche ou de pâte à tarte séchée, disposez dedans l'appareil à
soufflé et faites cuire comme d'ordinaire.

Démoulez avec précaution sur un plat et servez le soufflé tel quel ou flambez-le
au moment de le servir.

\section*{\centering Soufflés en savarin.}
\phantomsection
\addcontentsline{toc}{section}{ Soufflés en savarin.}
\index{Soufflés en savarin}

Une jolie façon de présenter des soufflés consiste à disposer dans un plat un
savarin, trempé de sirop liquoreux et maintenu en état par une bande de papier,
dont on emplit l'intérieur avec un appareil à soufflé. On fait cuire comme de
coutume et, au moment de servir, on arrose de nouveau le savarin avec du sirop
liquoreux, après avoir enlevé le papier qui l'entourait,

\section*{\centering Soufflé aux langues de chat.}
\phantomsection
\addcontentsline{toc}{section}{ Soufflé aux langues de chat.}
\index{Soufflé aux langues de chat}

Pour six personnes prenez :

\footnotesize
\begin{longtable}{rrrp{16em}}
    250 & grammes & de & lait,                                                                            \\
    250 & grammes & de & sucre,                                                                           \\
    125 & grammes & de & langues de chat,                                                                 \\
     25 & grammes & de & beurre,                                                                          \\
     15 & grammes & de & farine,                                                                          \\
        &         &  4 & œufs,                                                                            \\
        &         &    & kirsch, marasquin, anisette ou une autre liqueur, au choix,                      \\
        &         &    & sucre glace.                                                                     \\
\end{longtable}
\normalsize

Cassez les œufs ; séparez les blancs des jaunes ; battez les blancs en neige ferme.

Faites bouillir le lait avec le sucre.

Mettez dans une casserole de dimensions convenables le beurre et la farine,
laissez cuire sans que la farine prenne couleur, puis délayez avec le lait.
Après quelques bouillons, éloignez la casserole du feu, ajoutez les jaunes
d'œufs, faites épaissir le mélange à feu doux, pendant quelques minutes, et
parfumez avec du kirsch, du marasquin ou de l’anisette, par exemple. Lorsque
l'appareil sera suffisamment refroidi, incorporez-y les blancs d'œufs.

Trempez les langues de chat dans la liqueur choisie.

Beurrez un plat allant au feu, poudrez légèrement avec du sucre, mettez entre
deux couches d'appareil une couche de langues de chat parfumées, décorez avec
la pointe d'un couteau, poussez au four et glacez le soufflé avec du sucre, au
fur et à mesure de sa cuisson. Une dizaine de minutes de séjour au four
suffisent en moyenne.

\section*{\centering Soufflé froid au citron.}
\phantomsection
\addcontentsline{toc}{section}{ Soufflé froid au citron.}
\index{Soufflé froid au citron}

Pour six personnes prenez :

\footnotesize
\begin{longtable}{rrrp{16em}}
    180 & grammes & de & sucre en poudre,                                                                 \\
     50 & grammes & de & gelée de groseilles,                                                             \\
        &         &  6 & œufs,                                                                            \\
        &         &    & jus de deux citrons.                                                             \\
\end{longtable}
\normalsize

Cassez les œufs, séparez les blancs des jaunes, fouettez les blancs en neige.

Battez les jaunes avec le sucre, ajoutez le jus de citron, mêlez, chauffez au
bain-marie jusqu'à épaississement, puis mélangez à la masse chaude les blancs
d'œufs battus en neige. Versez cet appareil dans un compotier, laissez
refroidir, décorez avec la gelée de groseilles et servez avec des biscuits.

\section*{\centering Pseudo-soufflé glace.}
\phantomsection
\addcontentsline{toc}{section}{ Pseudo-soufflé glace.}
\index{Pseudo-soufflé glace}

Pour douze personnes prenez :

\footnotesize
\begin{longtable}{rrrp{16em}}
        & 1 litre & de & sirop clarifié\footnote{ Dissolvez à froid 2 kilogrammes
                                      de sucre dans un litre d'eau, puis mettez la
                                      moitié d'un blanc d'œuf fouetté et faites cuire
                                      ensemble, en remuant constamment et en ajoutant
                                      de temps en temps un peu d'eau, ce qui favorisera
                                      l'ascension des impuretés. Écumez ; lorsque
                                      l'écume devient blanche et légère, passez au
                                      travers d'un linge.},                                               \\
        &         & 12 & jaunes d'œufs,                                                                   \\
        &         &  3 & macarons,                                                                        \\
        &         &  2 & œufs entiers,                                                                    \\
        &         &    & kirsch, marasquin, anisette ou liqueur quelconque, au goût.                      \\
\end{longtable}
\normalsize

Mélangez ensemble jaunes d'œufs, œufs entiers, sirop et liqueur, versez le
mélange dans une bassine chauffée, fouettez, maintenez la bassine sur le feu
jusqu'à ce que l'appareil ait acquis une consistance spongieuse, versez-le
ensuite dans un moule à soufflé garni d'une bande de papier de quelques
centimètres de hauteur. Mettez à la glacière pendant trois heures.

Pulvérisez les macarons, passez-les au tamis.

Au moment de servir, saupoudrez le soufflé avec la poudre de macarons pour lui
donner l'apparence d'un soufflé ordinaire.

\section*{\centering Praliné à la crème.}
\phantomsection
\addcontentsline{toc}{section}{ Praliné à la crème.}
\index{Praliné à la crème}

Pour six personnes prenez :

\footnotesize
\begin{longtable}{rrrp{16em}}
    500 & grammes & de & lait,                                                                            \\
    250 & grammes & de & pralines grises,                                                                 \\
    250 & grammes & de & sucre en poudre,                                                                 \\
      2 & grammes & de & sel,                                                                             \\
        &         &  6 & œufs frais,                                                                      \\
        &         &    & vanille.                                                                         \\
\end{longtable}
\normalsize

Pilez les pralines dans un mortier.

Cassez les œufs ; séparez les blancs des jaunes.

Battez les blancs en neige ferme, incorporez-y pendant l'opération {\ppp60\mmm}
grammes de sucre, puis ajoutez les pralines, mélangez le tout intimement.

Caramélisez les parois intérieures d'un moule avec un peu de sucre ; versez
dedans les blancs d'œufs pralinés ; tassez l'appareil en secouant légèrement le
moule ; faites cuire au bain-marie, dans un four doux, pendant trois quarts
d'heure. Il faut que l'appareil monte à près de la moité de sa hauteur en
cuisant.

L'entremets est à point quand il commence à se détacher du moule ; laissez-le
refroidir.

Préparez une crème avec le lait, les jaunes d'œufs, le reste du sucre, le sel
et la vanille ; laissez-la refroidir.

Démoulez le praliné sur un plat, masquez-le avec la crème refroidie et servez,

\section*{\centering Meringue ordinaire.}
\phantomsection
\addcontentsline{toc}{section}{ Meringue ordinaire.}
\index{Meringue ordinaire}
\index{Appareil à meringue ordinaire}

Prenez :

\footnotesize
\begin{longtable}{rrrp{16em}}
    200 & grammes & de & sucre en poudre,                                                                 \\
      2 & grammes & de & sel,                                                                             \\
        &         & 12 & blancs d'œufs frais.                                                             \\
\end{longtable}
\normalsize

Battez les blancs d'œufs en neige ferme ; pendant l'opération, mélangez‑y le
sucre et le sel.

Coulez à la poche sur une feuille de papier des masses de blanc battu, ovales
ou rondes, de la grosseur d'un œuf, saupoudrez-les de sucre dont vous secouerez
l'excès, mettez la feuille de papier sur une planche mouillée et faites cuire
au four doux ouvert.

Décollez les meringues du papier, déprimez-les du côté plat sur un œuf en bois,
et faites-les sécher au four, à chaleur modérée, de façon qu'elles soient
croquantes.

\index{Garniture pour meringues}
Garnissez l'intérieur des meringues soit avec de la crème Chantilly, soit avec
de la crème fouettée parfumée avec du sirop de café ou aromatisée avec une
purée de fruits tels que fraises, framboises, cerises, pistaches, etc. ; ou
encore avec une bouillie épaisse de chocolat.

Accolez les meringues deux à deux, garniture contre garniture et servez.

\section*{\centering Meringue italienne montée.}
\phantomsection
\addcontentsline{toc}{section}{ Meringue italienne montée.}
\index{Meringue italienne montée}
\index{Appareil à meringue italienne}

Prenez :

\footnotesize
\begin{longtable}{rrrp{16em}}
    500 & grammes & de & sucre en poudre,                                                                 \\
      2 & grammes & de & sel,                                                                             \\
        &         & 10 & blancs d'œufs frais.                                                             \\
\end{longtable}
\normalsize

Mettez le tout dans une bassine tenue sur feu très doux, mélangez en fouettant
jusqu'à ce que l'appareil soit devenu assez consistant pour tenir entre les
branches du fouet.

\section*{\centering Meringue italienne au sucre cuit.}
\phantomsection
\addcontentsline{toc}{section}{ Meringue italienne au sucre cuit.}
\index{Meringue italienne au sucre cuit}

Prenez :

\footnotesize
\begin{longtable}{rrrp{16em}}
    500 & grammes & de & sucre,                                                                           \\
    200 & grammes & d' & eau,                                                                             \\
      2 & grammes & de & sel,                                                                             \\
        &         & 10 & blancs d'œufs frais.                                                             \\
\end{longtable}
\normalsize

Battez les blancs en neige très ferme.

Faites cuire le sucre avec l'eau au grand boulé ; ajoutez le sel, puis versez le
sirop en filet fin continu sur les blancs, en fouettant énergiquement pendant toute
l'opération, de façon que le sucre soit complètement absorbé par les blancs.

\sk

On emploie ces deux compositions comme la meringue ordinaire.

\section*{\centering Gâteau au chocolat.}
\phantomsection
\addcontentsline{toc}{section}{ Gâteau au chocolat.}
\index{Gâteau au chocolat}
\index{Fondant au chocolat}

Pour six personnes prenez :

\footnotesize
\begin{longtable}{rrrp{16em}}
    200 & grammes & de & chocolat,                                                                        \\
    175 & grammes & de & beurre,                                                                          \\
    100 & grammes & de & sucre,                                                                           \\
     75 & grammes & d' & amandes mondées,                                                                 \\
      3 & grammes & de & sel fin,                                                                         \\
        &         &  4 & œufs frais,                                                                      \\
        &         &    & vanille en poudre.                                                               \\
\end{longtable}
\normalsize

Pilez les amandes avec le sucre.

Cassez les œufs, séparez les blancs des jaunes ; battez les blancs en neige.

Faites fondre {\ppp75\mmm} grammes de chocolat dans un peu d'eau, incorporez-y,
hors du feu, par petites quantités, {\ppp125\mmm} grammes de beurre, puis les
jaunes d'œufs, un à un, et le sel, en travaillant bien après chaque addition.
Ajoutez ensuite les amandes pilées, de la vanille, au goût, et les blancs
d'œufs ; mélangez délicatement.

Versez l'appareil dans un moule à manqué bien beurré au pinceau et saupoudrez
de farine. Faites cuire au four doux pendant {\ppp25\mmm} minutes.

Glacez le gâteau, au sortir du four, avec un fondant au chocolat préparé avec
le reste du chocolat auquel vous aurez amalgamé le reste du beurre. Laissez
refroidir.

C'est un gâteau délicat, très léger.

\section*{\centering Gâteau au chocolat.}
\phantomsection
\addcontentsline{toc}{section}{ Gâteau au chocolat.}
\index{Gâteau au chocolat}

\begin{center}
\textit{(Autre formule).}
\end{center}

Prenez poids égaux d'œufs, de sucre en poudre, de chocolat râpé et de beurre.

Cassez les œufs, séparez les blancs des jaunes, battez les blancs en neige.

Délayez dans une casserole, tenue au bain-marie, le chocolat râpé avec la
quantité d'eau chaude nécessaire pour obtenir une pâte épaisse, tournez jusqu'à
ce que le chocolat soit complètement fondu, puis incorporez à la pâte le beurre
coupé en morceaux et les jaunes d'œufs, l’un après l’autre, en remuant
toujours.

Posez la casserole sur le feu, chauffez doucement, mettez le sucre, chauffez
encore, enfin ajoutez les blancs d'œufs.

Graissez un moule au pinceau avec du beurre, versez dedans l'appareil et faites
cuire au four doux ouvert, en surveillant la cuisson.

Servez avec une crème à la vanille ou une crème pralinée.

\section*{\centering Gâteau aux amandes et au chocolat.}
\phantomsection
\addcontentsline{toc}{section}{ Gâteau aux amandes et au chocolat.}
\index{Gâteau aux amandes et au chocolat}

Pour six personnes prenez :

\footnotesize
\begin{longtable}{rrrp{16em}}
    125 & grammes & de & chocolat,                                                                        \\
    125 & grammes & d' & amandes douces,                                                                  \\
    125 & grammes & de & sucre en poudre,                                                                 \\
    125 & grammes & de & beurre,                                                                          \\
        &         &  4 & œufs frais.                                                                      \\
\end{longtable}
\normalsize

Ébouillantez les amandes, mondez-les, passez-les ensuite au moulin.

Râpez le chocolat.

Cassez les œufs, séparez les blancs des jaunes ; battez les blancs en neige.

Mettez dans une terrine les jaunes d'œufs ; battez-les, ajoutez-y d'abord le
beurre non fondu et le sucre, travaillez, mettez ensuite le chocolat et les
amandes ; travaillez encore afin d'obtenir un ensemble homogène auquel vous
incorporerez les blancs d'œufs en neige,

Beurrez un moule à charlotte, versez dedans l'appareil et faites cuire au
bain‑marie pendant deux heures. Démoulez le gâteau sur un plat creux,
laissez-le refroidir.

Au moment de servir, masquez le gâteau avec une crème froide aux amandes ou
à la vanille, ou encore avec une crème pralinée.

\sk

Pour faire la crème pralinée prenez :

\footnotesize
\begin{longtable}{rrrp{16em}}
    375 & grammes & de & lait,                                                                            \\
    150 & grammes & de & pralines grises.                                                                 \\
    100 & grammes & de & sucre,                                                                           \\
      1 & gramme  & de & sel,                                                                             \\
        &         &  6 & jaunes d'œufs frais,                                                             \\
        &         &    & rhum,                                                                            \\
        &         &    & vanille.                                                                         \\
\end{longtable}
\normalsize

Pilez les pralines au mortier, passez-les au tamis fin.

Triturez ensuite la poudre de pralines avec les jaunes d'œufs, travaillez
bien ; l'appareil doit être légèrement mousseux. Parfumez-le avec du rhum,
au goût.

Faites bouillir le lait avec le sucre, le sel et de la vanille, éloignez la casserole
du feu, retirez la vanille, puis mettez dans le lait l'appareil praliné, mélangez et
faites prendre doucement. Laissez refroidir.

\sk

On pourra servir le gâteau aux amandes et au chocolat, chaud, avec une sauce
Sambaglione.

\section*{\centering Gâteau aux noix.}
\phantomsection
\addcontentsline{toc}{section}{ Gâteau aux noix.}
\index{Gâteau aux noix}

Pour six personnes prenez :

\medskip

1° pour le gâteau :

\footnotesize
\begin{longtable}{rrrrp{16em}}
  &  300 & grammes & de & noix épluchées,                                                                 \\
  &  250 & grammes & de & sucre,                                                                          \\
  &      &         &  7 & œufs frais,                                                                     \\
  &      &         &  1 & citron ;                                                                        \\
\end{longtable}
\normalsize

2° pour la crème :

\footnotesize
\begin{longtable}{rrrrp{16em}}
  &  400 & grammes & de & lait,                                                                           \\
  &  125 & grammes & de & chocolat,                                                                       \\
  &   60 & grammes & de & sucre,                                                                          \\
  & \multicolumn{2}{r}{1/2 gramme} & de & sel,                                                            \\
  &      &         &  3 & jaunes d'œufs frais,                                                            \\
  &      &         &    & vanille ;                                                                       \\
\end{longtable}
\normalsize

3° pour le glaçage :

\footnotesize
\begin{longtable}{rrrrp{16em}}
  &  400 & grammes & de & lait,                                                                           \kill
  &      &         &    & chocolat,                                                                       \\
  &      &         &    & sucre.                                                                          \\
\end{longtable}
\normalsize

Cassez les {\ppp7\mmm} œufs, séparez les blancs des jaunes ; battez les blancs
en neige.

Écrasez les noix, travaillez-les bien avec les jaunes d'œufs et le sucre ;
incorporez-y les blancs battus en neige ; puis ajoutez plus ou moins de zeste
râpé et de jus de citron au goût ; mélangez bien.

Versez l'appareil obtenu dans un moule à génoise beurré et faites cuire à feu
doux au four pendant une heure environ.

Laissez refroidir ; démoulez.

En même temps, préparez la crème au chocolat suivant les indications données
\hyperlink{p0857}{p. \pageref{pg0857}} ; laissez-la refroidir.

Faites fondre du chocolat et du sucre dans une très petite quantité d’eau,
laissez cuire pendant un instant ; puis glacez-en le gâteau.

Servez le gâteau et la crème séparément.

\sk

On pourra préparer de même des gâteaux aux noisettes, aux amandes ou aux
marrons.

Tous ces gâteaux pourront être servis aussi avec une crème anglaise au café ; ils
seront alors glacés au café.

\section*{\centering Gâteau glacé ou crème fantoche.}
\phantomsection
\addcontentsline{toc}{section}{ Gâteau glacé ou crème fantoche.}
\index{Gâteau glacé ou crème fantoche}
\index{Crème fantoche}

Pour six personnes prenez :

\footnotesize
\begin{longtable}{rrrp{16em}}
    500 & grammes & de & crème à la vanille,                                                              \\
    250 & grammes & de & crème Chantilly,                                                                 \\
    170 & grammes & de & chocolat,                                                                        \\
     30 & grammes & de & sucre,                                                                           \\
        &         &  3 & jaunes d'œufs frais,                                                             \\
        &         &  2 & feuilles de gélatine blanche.                                                    \\
\end{longtable}
\normalsize

Faites cuire le chocolat avec le sucre dans la quantité d’eau nécessaire pour
obtenir un appareil de bonne consistance. Éloignez la casserole du feu, ajoutez
les jaunes d'œufs ; chauffez de nouveau et laissez cuire à feu doux pendant
quelques instants. Incorporez ensuite la gélatine dissoute dans un peu d'eau
fraîche et passée à la passoire fine. Laissez refroidir.

Mettez au fond d'un moule une couche de chocolat, au-dessus une couche de crème
Chantilly et ainsi de suite jusqu'à ce que le moule soit plein. Couvrez en
interposant un papier beurré entre la crème et le couvercle. Faites refroidir
pendant une heure et demie dans de la glace pilée et salée.

Au moment de servir, démoulez le gâteau et entourez-en le pied avec la crème
vanillée bien froide.

\section*{\centering Gâteau glacé aux noisettes et aux pistaches.}
\phantomsection
\addcontentsline{toc}{section}{ Gâteau glacé aux noisettes et aux pistaches.}
\index{Gâteau glacé aux noisettes et aux pistaches}

Pour douze personnes prenez :

\footnotesize
\begin{longtable}{rrrp{16em}}
    400 & grammes & de & beurre fin,                                                                      \\
    400 & grammes & de & sucre en poudre,                                                                 \\
    350 & grammes & de & sucre en morceaux,                                                               \\
    250 & grammes & de & belles noisettes mondées,                                                        \\
    250 & grammes & de & pistaches mondées,                                                               \\
      3 & grammes & de & sel fin,                                                                         \\
        & 1 litre & de & lait,                                                                            \\
        &         & 16 & jaunes d'œufs frais,                                                             \\
        &         &    &  biscuits à la cuiller,                                                          \\
        &         &    &  vanille.                                                                        \\
\end{longtable}
\normalsize

Hachez les noisettes et les pistaches, pilez-les séparément dans un mortier avec
quelques gouttes de lait.

Préparez avec le lait, les jaunes d'œufs, le sucre en morceaux, le sel et la
vanille une crème très épaisse.

Ramollissez le beurre au bain-marie, de façon à le rendre presque liquide,
ajoutez-y le sucre en poudre, les noisettes et les pistaches pilées ;
travaillez bien, puis incorporez à la masse la moitié de la crème vanillée.
Tenez le reste au frais.

Garnissez le fond et les parois d'un moule uni avec des biscuits à la cuiller,
versez dedans l'appareil, couvrez avec une couche de biscuits et faites prendre
à la glace.

Au moment de servir, démoulez le gâteau sur un plat à entremets et masquez‑le
avec la crème vanillée gardée au frais.

\sk

Comme variantes, on pourra préparer ce gâteau par couches alternées de biscuits
et d'appareil noisettes-pistaches.

\section*{\centering Gâteau glacé aux fruits.}
\phantomsection
\addcontentsline{toc}{section}{ Gâteau glacé aux fruits.}
\index{Gâteau glacé aux fruits}

Prenez des biscuits à la cuiller, coupez-les en deux dans le sens de leur
longueur. Garnissez d'une couche de demi-biscuits un moule uni à entremets posé
sur de la glace. Aspergez cette couche de biscuits avec du marasquin ou du
curaçao, puis placez dessus des cerises bien mûres, lavées, séchées et coupées
en deux, des raisins de Corinthe et de Malaga épépinés et du cédrat coupé en
petits dés ; couvrez d'une couche de crème épaisse à la vanille ou aux
pistaches, laissez prendre. Placez au-dessus une nouvelle assise de
demi-biscuits, mouillez-la aussi de marasquin ou de curaçao, semez dessus des
pistaches émincées, des grains de raisin frais, de l'ananas coupé en petits dés
et mijoté dans du sirop d’ananas, couvrez d'une couche de crème à la vanille ou
aux pistaches, laissez prendre.

Disposez ainsi quatre assises de biscuits et de garniture, finissez par une
couche de biscuits et faites glacer.

Démoulez au moment de servir.

\section*{\centering Gâteau de semoule aux dattes.}
\phantomsection
\addcontentsline{toc}{section}{ Gâteau de semoule aux dattes.}
\index{Gâteau de semoule aux dattes}

Pour six personnes prenez :

\footnotesize
\begin{longtable}{rrrp{16em}}
    400 & grammes & de & dattes très mûres.                                                               \\
    300 & grammes & de & beurre fin.                                                                      \\
    100 & grammes & de & semoule,                                                                         \\
     60 & grammes & de & sucre,                                                                           \\
      2 & grammes & de & sel,                                                                             \\
      1 & gramme  & de & cannelle de Ceylan en poudre,                                                    \\
        & 1 litre & de & lait,                                                                            \\
        &         &  1 & œuf frais,                                                                       \\
        &         &    & vanille,                                                                         \\
        &         &    & miel.                                                                            \\
\end{longtable}
\normalsize

Mettez dans une casserole le lait, le sucre, le sel et la vanille, faites
bouillir ; jetez dedans la semoule en pluie et laissez cuire pendant vingt
minutes. Éloignez la casserole du feu, retirez la vanille.

Cassez l'œuf, battez-le et, en tournant, incorporez-le à la semoule pour la
lier.

Laissez refroidir la pâte de semoule, puis faites-en deux galettes rondes de
{\ppp20\mmm} centimètres de diamètre environ.

Retirez les noyaux des dattes, passez la pulpe au tamis.

Mélangez intimement beurre et purée de dattes et étalez le mélange sur l'une
des galettes de semoule placée sur une tôle garnie d'un papier beurré ;
saupoudrez avec la cannelle, couvrez avec la seconde galette, badigeonnez le
dessus avec du miel et faites cuire à four modéré.

Servez ce gâteau tiède, de préférence, et envoyez en même temps de la crème
à la vanille dans une saucière.

Cet entremets, d'origine arabe, est très agréable.

\section*{\centering Gâteau de riz à la vanille.}
\phantomsection
\addcontentsline{toc}{section}{ Gâteau de riz à la vanille.}
\index{Gâteau de riz à la vanille}

Pour dix personnes prenez :

\footnotesize
\begin{longtable}{rrrrp{16em}}
  & 300 & grammes & de & riz de la Caroline,                                                              \\
  & 300 & grammes & de & sucre en morceaux,                                                               \\
  & 150 & grammes & de & beurre,                                                                          \\
  &  13 & grammes & de & sel,                                                                             \\
  & \multicolumn{2}{r}{2 litres} & de & lait,                                                             \\
  &     &         &  8 & jaunes d'œufs frais,                                                             \\
  &     &         &  4 & œufs frais,                                                                      \\
  &     &         &  1 & gousse de vanille,                                                               \\
  &     &         &    & mie de pain rassis tamisée.                                                      \\
\end{longtable}
\normalsize

Lavez le riz plusieurs fois à l’eau froide, passez-le ensuite pendant cinq
minutes dans deux litres d'eau bouillante, égouttez‑le et rafraîchissez‑le.
Faites bouillir le lait avec la vanille pendant cinq minutes, retirez ensuite
la vanille.

Mettez dans une casserole un litre et demi de lait vanillé ; lorsqu'il sera en
ébullition, ajoutez le riz, le sel, {\ppp200\mmm} grammes de sucre,
{\ppp100\mmm} grammes de beurre ; laissez cuire pendant une heure à feu très
doux, en évitant que le riz s'attache aux parois de la casserole. Éloignez
ensuite la casserole du feu, cassez les œufs entiers battez-les sans les faire
mousser et mélangez-les au riz à l'aide d'une cuiller en bois.

Graissez avec le reste du beurre un moule cylindrique de {\ppp7\mmm}
centimètres de hauteur et de {\ppp12\mmm} centimètres de diamètre,
saupoudrez-en l'intérieur de mie de pain, puis emplissez le moule avec
l'appareil au riz et mettez au four.

La cuisson doit durer une demi-heure.

En même temps, préparez la sauce.

Délayez dans une casserole les jaunes d'œufs avec le reste du lait vanillé,
ajoutez le reste du sucre et faites cuire à feu doux jusqu'à ce que la sauce
prenne la consistance nécessaire pour masquer une cuiller. Éloignez la
casserole du feu et tournez encore pendant quelques minutes. Passez la sauce au
chinois et servez-la dans une saucière, en même temps que le gâteau.

\sk

On peut faire le gâteau de riz à d'autres parfums : citron, orange, ananas, etc.,
en remplaçant la vanille par du zeste de citron râpé, du zeste d'orange râpé dans
du jus d'orange, de l'ananas, etc.

On peut aussi y mélanger des raisins secs épépinés,.

\section*{\centering Gâteau de riz aux fruits.}
\phantomsection
\addcontentsline{toc}{section}{ Gâteau de riz aux fruits.}
\index{Gâteau de riz aux fruits}

Pour dix personnes prenez :

\footnotesize
\begin{longtable}{rrrp{16em}}
  1 000 & grammes & de & fruits assortis, tels que : ananas, cerises, poires, pommes, par exemple,        \\
    330 & grammes & de & sucre,                                                                           \\
    300 & grammes & d' & eau,                                                                             \\
    200 & grammes & de & riz de la Caroline,                                                              \\
    100 & grammes & de & confiture de groseilles, de fraises, d'abricots, par exemple,                    \\
     30 & grammes & de & beurre.                                                                          \\
      8 & grammes & de & sel,                                                                             \\
        & 1 litre & de & lait,                                                                            \\
        &         &  4 & œufs frais,                                                                      \\
        &         &  1 & gousse de vanille.                                                               \\
\end{longtable}
\normalsize

Épluchez et épépinez les fruits ; faites-les cuire dans l'eau avec
{\ppp200\mmm} grammes de sucre, puis dorez-les au four. Concentrez le jus et
réservez-le.

Lavez le riz à plusieurs reprises dans de l'eau froide, blanchissez-le pendant
cinq minutes dans de l’eau bouillante, égouttez-le.

Prenez une casserole d'une contenance de deux litres au moins, mettez dedans le
lait, le sel et la vanille, faites bouillir pendant cinq minutes, puis retirez
la vanille, mettez le riz blanchi et laissez mijoter pendant une heure.

Triturez ensemble le beurre, les œufs et le reste du sucre, puis mélangez le
tout au riz à l'aide d'une cuiller en bois.

Versez le mélange dans un plat creux, mettez au four pendant une demi-heure,
puis renversez le gâteau dans un compotier ou dans un plat et garnissez‑le avec
les fruits, que vous glacerez avec la confiture délayée dans le jus réservé.

\section*{\centering Riz à l'impératrice.}
\phantomsection
\addcontentsline{toc}{section}{ Riz à l'impératrice.}
\index{Riz à l'impératrice}

Pour douze personnes prenez :

\footnotesize
\begin{longtable}{rrrrp{16em}}
  &   1 000 & grammes & de & crème,                                                                       \\
  &     500 & grammes & de & lait,                                                                        \\
  &     500 & grammes & de & fruits variés, tels que : abricots, ananas, cerises, prunes, etc.,           \\
  &     300 & grammes & de & riz de la Caroline,                                                          \\
  &     300 & grammes & de & sucre en morceaux,                                                           \\
  &     150 & grammes & de & sucre semoule,                                                               \\
  &      10 & grammes & de & sel,                                                                         \\
  & \multicolumn{2}{r}{1/2 litre} & de & crème fouettée,                                                  \\
  &         &         &  5 & jaunes d'œufs frais,                                                         \\
  &         &         &  1 & gousse de vanille,                                                           \\
  &         &         &    & angélique confite,                                                           \\
  &         &         &    & gelée de fruits glacée,                                                      \\
  &         &         &    & kirsch ou marasquin,                                                         \\
  &         &         &    & huile d'amandes douces.                                                      \\
\end{longtable}
\normalsize

Faites mariner les fruits dans du kirsch ou dans du marasquin.

Lavez le riz à plusieurs reprises dans de l'eau froide, laissez-le tremper
pendant deux heures, puis passez-le pendant une minute dans de l’eau
bouillante.

Mettez dans une casserole le lait, la vanille, le sucre en morceaux et le sel,
amenez à ébullition. Au bout de cinq minutes. enlevez la vanille, ajoutez le
riz et laissez mijoter pendant vingt-cinq minutes.

Faites bouillir la crème avec le sucre semoule et la gousse de vanille,
ajoutez-y, hors du feu, les jaunes d'œufs délayés avec un peu de lait, puis
continuez la cuisson au bain-marie. Retirez ensuite la vanille ; mettez la
crème fouettée ; mélangez bien cet appareil bavarois.

Incorporez au riz, par petites quantités et sans l'écraser, l'appareil
bavarois, les fruits et leur marinade, de l'angélique coupée en lamelles ;
versez le tout dans un moule légèrement graissé avec de l'huile d'amandes
douces ; mettez pendant deux heures à la glace.

Démoulez sur un plat rond garni d’une serviette et servez avec une gelée de
fruits glacée.

C'est le plus joli des entremets froids de riz.

\section*{\centering Croquettes de riz.}
\phantomsection
\addcontentsline{toc}{section}{ Croquettes de riz.}
\index{Croquettes de riz}
\index{Caisse de riz}

Faites cuire du riz dans du lait parfumé à la vanille, au citron, à l'orange ou à
l'ananas, par exemple, incorporez-y un peu de crème cuite encore chaude, laissez
refroidir le mélange, puis faites-en des croquettes cylindriques.

Battez des œufs comme pour une omelette, trempez chaque croquette d'abord dans
les œufs battus, puis dans de la mie de pain rassis tamisée ; lissez-les au
couteau et faites-les frire dans de la friture claire, chaude.

Égouttez-les, saupoudrez-les de sucre et servez-les avec du jus de fruits sucré
ou avec une sauce aux fruits préparée de la façon suivante :

Prenez des abricots, des cerises, des fraises, des framboises, des groseilles,
des pêches, des poires, des prunes Reine-Claude, des mirabelles, etc., au
choix ; faites-les cuire comme des confitures un peu fluides, passez-les et
additionnez-les au moment de servir d’une liqueur ou d'un autre parfum ; enfin
montez-les avec un peu de beurre fin.

C'est un excellent entremets de famille, très facile à préparer.

\section*{\centering Croustades de riz.}
\phantomsection
\addcontentsline{toc}{section}{ Croustades de riz.}
\index{Croustades de riz}

Faites cuire du riz sucré dans du lait ; travaillez-le ensuite à la spatule avec
des œufs.

Beurrez des petits moules ronds ou ovales, foncez-les avec de la pâte
feuilletée, saupoudrez légèrement les parois de mie de pain rassis tamisée,
puis metttez une couche de riz, au-dessus un mélange de fruits frais cuits au
sirop ou de fruits confits, enfin une autre couche de riz. Placez les moules
sur une plaque à rebord garnie d'un peu d’eau, couvrez avec un papier beurré et
faites cuire au four.

\section*{\centering Pain aux amandes, à la viennoise.}
\phantomsection
\addcontentsline{toc}{section}{ Pain aux amandes, à la viennoise.}
\index{Pain aux amandes, à la viennoise}

Pour six personnes prenez :

\medskip

1° pour le pain :

\footnotesize
\begin{longtable}{rrrp{16em}}
    160 & grammes & de & beurre,                                                                          \\
    140 & grammes & de & sucre en poudre,                                                                 \\
     70 & grammes & d'a& mandes,                                                                          \\
     50 & grammes & de & chapelure,                                                                       \\
     15 & grammes & de & fine champagne,                                                                  \\
      1 & gramme  & de & sel,                                                                             \\
        &         &  4 & œufs frais,                                                                      \\
        &         &    & farine ;                                                                         \\
\end{longtable}
\normalsize

2° pour le garnissage :

\footnotesize
\begin{longtable}{rrrp{16em}}
    100 & grammes & de & confiture d'abricats,                                                            \\
     50 & grammes & de & chocolat râpé, .                                                                 \\
     50 & grammes & de & sucre en poudre,                                                                 \\
     30 & grammes & d' & eau,                                                                             \\
        &         & 18 & amandes mondées.                                                                 \\
\end{longtable}
\normalsize

Pilez les amandes,

Mettez dans un mortier {\ppp140\mmm} grammes de beurre, le sel et le sucre,
travaillez, ajoutez les amandes, la chapelure, la fine champagne, {\ppp3\mmm}
jaunes d'œufs et un œuf entier, travaillez encore ; incorporez enfin à la masse
les trois blancs d'œufs restants, battus en neige ferme.

Graissez un moule avec le reste du beurre, saupoudrez de farine, versez l'appareil
dans le moule et faites cuire à four doux ; laissez refroidir, démoulez.

Enlevez une calotte mince à la partie supérieure du pain, creusez l'intérieur,
emplissez le vide avec la confiture d'abricots, couvrez avec la calotte enlevée.

Faites fondre le chocolat avec le sucre dans les {\ppp30\mmm} grammes d'eau,
donnez un bouillon et glacez-en le gâteau.

Décorez avec les {\ppp18\mmm} amandes entières, glacées à part au sucre cuit.

\sk

On peut préparer de même des pains aux noix, aux noisettes ou aux pistaches.

\section*{\centering Pudding aux dattes, à la crème d'amandes.}
\phantomsection
\addcontentsline{toc}{section}{ Pudding aux dattes, à la crème d'amandes.}
\index{Pudding aux dattes, à la crème d'amandes}
\index{Crème d'amandes}

Pour huit personnes prenez :

\medskip

1° pour le pudding :

\footnotesize
\begin{longtable}{rrrp{16em}}
    600 & grammes & de & dattes fraîches,                                                                 \\
    250 & grammes & de & raisins de Corinthe ou de Smyrne,                                                \\
        &         &    & biscuits à la cuiller,                                                           \\
        &         &    & sucre en poudre,                                                                 \\
        &         &    & beurre ;                                                                         \\
\end{longtable}
\normalsize

2° pour la crème :

\footnotesize
\begin{longtable}{rrrp{16em}}
    250 & grammes & d' & amandes mondées,                                                                 \\
    250 & grammes & de & sucre,                                                                           \\
    250 & grammes & de & beurre,                                                                          \\
     90 & grammes & de & crème épaisse,                                                                   \\
        &         &  6 & œufs frais,                                                                      \\
        &         &    & kirsch ou vanille\footnote{\index{Créme de pistaches}
                                                    \index{Créme pralinée}
                         On pourra préparer de même une
                         crème pralinée ou une crème de pistaches en remplaçant les
                         amandes par des pralines ou par des pistaches, le kirsch ou
                         la vanille par du rhum dans la crème pralinée et par du curaçao
                         dans la crème de pistaches.}.                                                    \\
\end{longtable}
\normalsize

Enlevez les noyaux des dattes, passez la pulpe au tamis ; épépinez les raisins.
Beurrez un moule, saupoudrez-en l'intérieur avec un peu de sucre, garnissez le
fond et les parois avec des biscuits à la cuiller, saupoudrez encore de sucre,
puis mettez une couche de purée de dattes, au-dessus une couche de raisins,
à nouveau du sucre, puis des biscuits ; continuez l'alternance des couches
jusqu'à épuisement des substances employées.

Préparez la crème.

Pilez les amandes au mortier avec un peu de blanc d'œuf, ajoutez le sucre,
pilez encore ; incorporez ensuite le beurre liquéfié juste à point, les œufs,
le kirsch ou la vanille, travaillez bien en tournant et en ramenant la pâte au
milieu du mortier ; la préparation doit mousser ; enfin mettez la crème.

Versez une partie de cet appareil dans le moule jusqu'à ce qu'il soit à peu
près plein ; faites prendre au bain-marie. Faites cuire le reste de l'appareil
sur feu doux jusqu'à obtention d'une bonne consistance. Surveillez l'opération.

Démoulez le pudding sur un plat, masquez-le avec le reste de la crème
d'amandes et servez chaud.

\sk

\index{Crème aux noisettes}
Comme variante, on pourra servir ce pudding avec de la crème aux noisettes.

\sk

Pour faire de la crème aux noisettes prenez :

\footnotesize
\begin{longtable}{rrrp{16em}}
    350 & grammes & de & lait,                                                                            \\
    250 & grammes & de & noisettes grillées,                                                              \\
    150 & grammes & de & sucre,                                                                           \\
     60 & grammes & de & beurre,                                                                          \\
      2 & grammes & de & sel,                                                                             \\
        &         &  4 & œufs frais.                                                                      \\
\end{longtable}
\normalsize

Pilez les noisettes, passez-les au tamis.

Cassez les œufs dans une casserole, battez-les au fouet, sur un feu doux, avec
le sucre. Aussitôt qu'ils sont montés, ajoutez les noisettes, le beurre ramolli
à la chaleur, le lait bouilli ; mélangez bien ; chauffez doucement jusqu'à
bonne consistance.

\section*{\centering Pudding aux noix, à la crème.}
\phantomsection
\addcontentsline{toc}{section}{ Pudding aux noix, à la crème.}
\index{Pudding aux noix, à la crème}

Pour huit personnes prenez :

\medskip

1° pour le pudding :

\footnotesize
\begin{longtable}{rrrp{16em}}
    500 & grammes & de & noix fraîches, débarrassées de leurs enveloppes,                                 \\
    250 & grammes & de & sucre en poudre,                                                                 \\
    250 & grammes & de & mie de pain rassis tamisée,                                                      \\
        &         &  8 & œufs frais,                                                                      \\
        &         &    & huile de noix ;                                                                  \\
\end{longtable}
\normalsize

\index{Crème vanillée}
2° pour la crème :

\footnotesize
\begin{longtable}{rrrp{16em}}
    500 & grammes & de & crème,                                                                           \\
     30 & grammes & de & sucre en poudre,                                                                 \\
      1 & gramme  & de & sel,                                                                             \\
        &         &  6 & jaunes d'œufs,                                                                   \\
        &         &    & vanille.                                                                         \\
\end{longtable}
\normalsize

Écrasez les noix, au mortier.

Cassez les œufs, séparez les jaunes des blancs.

Battez les jaunes avec le sucre et, petit à petit, ajoutez-y les noix
écrasées ; cette opération dure environ une demi-heure.

Battez les blancs et, par petites quantités, incorporez-les avec la mie de pain
à la masse précédente.

Graissez un moule avec de l'huile de noix, emplissez-le avec le mélange et
faites cuire pendant une bonne heure au bain-marie très chaud.

Préparez la crème.

Faites cuire {\ppp450\mmm} grammes de crème avec le sucre, le sel et la
vanille.

Délayez les jaunes d'œufs avec le reste de la crème froide, et incorporez-les
à la crème cuite. Maintenez le tout sur feu doux, en mélangeant constamment,
jusqu'à consistance convenable.

Démoulez le pudding lorsqu'il sera cuit et servez-le avec la crème.

\section*{\centering Pudding aux amandes.}
\phantomsection
\addcontentsline{toc}{section}{ Pudding aux amandes.}
\index{Pudding aux amandes}

Pour six personnes prenez :

\medskip

1° pour le pudding :

\footnotesize
\begin{longtable}{rrrp{16em}}
    100 & grammes & d' & amandes gnillées,                                                                \\
     90 & grammes & de & biscuits de Reims roses,                                                         \\
     75 & grammes & de & pralines grises,                                                                 \\
     65 & grammes & de & beurre,                                                                          \\
     50 & grammes & de & sucre en poudre parfumé à la vanille,                                            \\
        &         &  4 & œufs frais,                                                                      \\
        &         &    & biscuits à la cuiller,                                                           \\
        &         &    & kirsch ou marasquin, au goût ;                                                   \\
\end{longtable}
\normalsize

2° pour la crème anglaise :

\footnotesize
\begin{longtable}{rrrp{16em}}
    500 & grammes & de & lait,                                                                            \\
    150 & grammes & de & sucre,                                                                           \\
      5 & grammes & de & sel,                                                                             \\
        &         &  6 & jaunes d'œufs,                                                                   \\
        &         &    & vanille.                                                                         \\
\end{longtable}
\normalsize

Avec les éléments du 2\textsuperscript{e} paragraphe faites une crème anglaise
suivant les indications données \hyperlink{p0856}{p. \pageref{pg0856}}.

Préparez le pudding. Cassez les œufs, séparez les blancs des jaunes ; battez
blancs et jaunes séparément.

Pilez au mortier biscuits de Reims, amandes et pralines, ajoutez le sucre, le
beurre, triturez bien ; mettez ensuite les jaunes d'œufs, un peu de kirsch ou
de marasquin ; mélangez ; puis incorporez {\ppp50\mmm} grammes de crème
anglaise et, par petites quantités, les blancs d'œufs battus. Travaillez bien
l'appareil après chaque addition.

Beurrez un moule, chemisez-le avec des biscuits à la cuiller que vous
aspergerez de kirsch ou de marasquin, puis versez l'appareil dans le moule.
Recouvrez avec des biscuits à la cuiller, et faites cuire au bain-marie pendant
une heure un quart environ.

Démoulez le pudding sur un plat à entremets ; mettez-le à rafraîchir,
masquez-le ensuite avec le reste de la crème anglaise et servez.

\section*{\centering Pudding aux pralines.}
\phantomsection
\addcontentsline{toc}{section}{ Pudding aux pralines.}
\index{Pudding aux pralines}

Pour six personnes prenez :

\footnotesize
\begin{longtable}{rrrp{16em}}
    500 & grammes & de & lait,                                                                            \\
    350 & grammes & de & sucre en morceaux,                                                               \\
    250 & grammes & de & pralines grises,                                                                 \\
     50 & grammes & de & sucre semoule,                                                                   \\
     50 & grammes & d' & eau,                                                                             \\
      2 & grammes & de & sel,                                                                             \\
        &         &  6 & œufs frais,                                                                      \\
        &         &    & vanille.                                                                         \\
\end{longtable}
\normalsize

Cassez les œufs, séparez les blancs des jaunes, battez les blancs en neige.

Pilez grossièrement les pralines au mortier, ajoutez-les ainsi que le sucre
semoule aux blancs battus, sans les faire tomber.

Faites un caramel blond avec {\ppp200\mmm} grammes de sucre en morceaux et
l'eau. Enduisez d'une partie de ce caramel un moule uni, versez‑y l'appareil
praliné et faites cuire au bain-marie bouillant pendant deux heures.

Quand le pudding est cuit, renversez le moule sur un plat à entremets, mais
ne démoulez pas.

\index{Crème au caramel}
Préparez une crème au caramel.

Faites bouillir le lait avec le reste du sucre, le sel et la vanille, éloignez
la casserole du feu, retirez la vanille, ajoutez les jaunes d'œufs, le reste du
caramel et continuez la cuisson jusqu'à consistance convenable ; laissez
refroidir.

Au moment de servir, enlevez le moule et masquez le pudding avec la crème.

\section*{\centering Pudding mousseline.}
\phantomsection
\addcontentsline{toc}{section}{ Pudding mousseline.}
\index{Pudding mousseline}

Pour huit personnes prenez :

\footnotesize
\begin{longtable}{rrrp{16em}}
    250 & grammes & de & sucre en poudre,                                                                 \\
    125 & grammes & de & beurre,                                                                          \\
     90 & grammes & de & fécule,                                                                          \\
      1 & gramme  & de & sel,                                                                             \\
        &         &  6 & œufs frais,                                                                      \\
        &         &  1 & citron,                                                                          \\
        &         &    & lait,                                                                            \\
        &         &    & sirop d'abricots.                                                                \\
\end{longtable}
\normalsize

Cassez les œufs, séparez les jaunes des blancs.

Mettez dans une casserole {\ppp200\mmm} grammes de sucre et le beurre, mélez,
ajoutez la fécule, le jus du citron et le zeste haché très fin.

Délayez ce mélange avec du lait, ajoutez le sel, les jaunes d'œufs et faites
cuire de manière à obtenir une pâte de la consistance d'une bouillie. Si c'est
nécessaire, ajoutez encore un peu de lait, puis laissez refroidir.

Battez les blancs d'œufs en neige, mélangez-les à la pâte.

Beurrez un moule, saupoudrez légèrement de sucre, versez dedans la pâte, mais
ne l’emplissez pas trop ; puis placez-le dans un vase contenant de l'eau et
mettez le tout au four à température modérée.

Faites cuire pendant une bonne heure, puis démoulez. Laissez refroidir.

Servez le pudding avec du sirop d'abricots.

\sk

On peut aussi servir ce pudding avec une crème à la vanille.

\section*{\centering Macarons.}
\phantomsection
\addcontentsline{toc}{section}{ Macarons.}
\index{Macarons}

Les macarons, petite pâtisserie d'origine italienne, ont pour éléments
fondamentaux les amandes\footnote{On pourra remplacer les amandes par des
pistaches ou par des noisettes et même par des mélanges de ces différents
fruits. On aura ainsi des macarons exquis tout à fait originaux,}, le sucre,
les blancs d'œufs.

Ils sont parfumés le plus souvent avec de la vanille, mais on peut faire des
macarons à d'autres parfums, aux liqueurs et même aux fruits.

\section*{\centering Macarons moelleux.}
\phantomsection
\addcontentsline{toc}{section}{ Macarons moelleux.}
\index{Macarons moelleux}

Prenez :

\footnotesize
\begin{longtable}{rrrp{16em}}
    750 & grammes & de & sucre,                                                                           \\
    600 & grammes & d' & amandes fraîches mondées,                                                        \\
    100 & grammes & de & crème épaisse,                                                                   \\
        &         &  8 & blancs d'œufs,                                                                   \\
        &         &    & parfum (vanille, rhum, kirsch, etc., au choix).                                  \\
\end{longtable}
\normalsize

Pilez les amandes au mortier avec {\ppp5\mmm} blancs d'œufs, ajoutez-y le sucre
et le parfum choisi, travaillez convenablement la pâte, puis incorporez-y la
crème et ensuite le reste des blancs battus en neige ferme. La pâte doit être
légèrement molle, mais elle ne doit pas s'étaler.

Dressez les macarons à l'aide d'une poche sur du papier blanc poudré de sucre,
posez le tout sur une plaque en tôle et faites cuire au four un peu chaud.

\section*{\centering Macarons secs.}
\phantomsection
\addcontentsline{toc}{section}{ Macarons secs.}
\index{Macarons secs}

Prenez :

\footnotesize
\begin{longtable}{rrrp{16em}}
    750 & grammes & de & sucre,                                                                           \\
    300 & grammes & d' & amandes mondées,                                                                 \\
     25 & grammes & de & rhum ou de kirsch, au goût,                                                      \\
        &         &  5 & blancs d'œufs.                                                                   \\
\end{longtable}
\normalsize

Pilez les amandes avec {\ppp2\mmm} blancs d'œufs, ajoutez le sucre par petites
quantités en continuant à piler, puis les autres blancs d'œufs et la liqueur
choisie. Travaillez bien ; la pâte doit être ferme.

Dressez les macarons à la main ou à la seringue sur une plaque beurrée et
farinée, et faites cuire au four doux un peu ouvert. Surveillez la cuisson afin
d'éviter un coup de feu.

\section*{\centering Mousse aux pommes.}
\phantomsection
\addcontentsline{toc}{section}{ Mousse aux pommes.}
\index{Mousse aux pommes}

Pour six personnes prenez :

\footnotesize
\begin{longtable}{rrrrp{16em}}
  & 125 & grammes & de & sucre en poudre,                                                                 \\
  & & \multicolumn{2}{r}{7 à 8} & pommes, suivant leur grosseur,                                          \\
  &     &         &  4 & feuilles de gélatine blanche,                                                    \\
  &     &         &    & jus de citron,                                                                   \\
  &     &         &    & zeste de citron.                                                                 \\
\end{longtable}
\normalsize

Pelez les pommes, épépinez-les et faites-les cuire avec un peu d'eau et
d'écorce de citron, c'est-à-dire faites une marmelade serrée sans sucre ;
laissez-la refroidir.

Dissolvez la gélatine dans un peu d’eau.

Mélangez à la marmelade la gélatine dissoute, le sucre, le jus d'un citron et
un peu de zeste râpé, battez jusqu'à ce que l'appareil mousse comme des blancs
d'œufs, puis mettez-le dans un moule que vous entourerez de glace.

\sk

Comme variantes, on pourra garnir le moule de couches alternées de mousse aux
pommes et de gelée de groseilles, ou de gelée de coings, ou encore de gelée
d'ananas, de confiture d'abricots, etc.

\section*{\centering Mousse au kirsch, garnie de fruits.}
\phantomsection
\addcontentsline{toc}{section}{ Mousse au kirsch, garnie de fruits.}
\index{Mousse au kirsch, garnie de fruits}

Pour huit personnes prenez :

\footnotesize
\begin{longtable}{rrrp{16em}}
  1 500 & grammes & de & fruits tels que fraises, framboises, cerises, pêches,
                         abricots, prunes, ananas, etc.,                                                  \\
    150 & grammes & de & sucre en poudre,                                                                 \\
    120 & grammes & de & kirsch,                                                                          \\
        &         &  8 & jaunes d'œufs frais,                                                             \\
        &         &  2 & blancs d'œufs frais.                                                             \\
\end{longtable}
\normalsize

Épluchez les fruits ; enlevez les noyaux. Tenez les fruits au frais dans un
plat.

Mettez les jaunes d'œufs, le sucre et le kirsch dans une terrine ; mélangez, puis
ajoutez les blancs ; fouettez jusqu'à obtention d'une crème ayant une consistance
mousseuse.

Versez cette crème sur les fruits et servez.

\sk

On pourra préparer de même des mousses à d’autres liqueurs.

\section*{\centering Pommes au beurre.}
\phantomsection
\addcontentsline{toc}{section}{ Pommes au beurre.}
\index{Pommes au beurre}

Prenez de belles pommes reinettes, pelez-les et évidez-les en forme de cône de
façon à enlever tous les pépins sans creuser au delà.

Taillez dans de la mie de pain boulot des canapés du diamètre des pommes et
ayant deux centimètres d'épaisseur ; faites-les dorer de tous les côtés dans du
beurre.

Disposez les canapés dans un plat foncé de beurre et allant au feu, placez une
pomme sur chacun d'eux ; emplissez le creux des pommes avec du beurre.

Mettez au four et, lorsque les pommes auront absorbé le beurre, remettez-en,
laissez-le imbiber de nouveau les pommes, puis saupoudrez pommes et canapés de
sucre en poudre parfumé à la vanille. Glacez au four.

Garnissez enfin le creux des pommes avec des confitures de fraises, de
framboises, de groseilles, etc., au choix, et servez chaud.

\section*{\centering Pommes à la crème ou au sirop.}
\phantomsection
\addcontentsline{toc}{section}{ Pommes à la crème ou au sirop.}
\index{Pommes à la crème ou au sirop}

Pour six personnes prenez :

\footnotesize
\begin{longtable}{rrrp{16em}}
  1 500 & grammes & de & pommes reinettes,                                                                \\
    500 & grammes & de & sucre,                                                                           \\
        &         &  1 & citron,                                                                          \\
        &         &    & crème à la vanille ou sirop de framboises, au goût,                              \\
        &         &    & beurre.                                                                          \\
\end{longtable}
\normalsize

Épluchez les pommes, coupez-les en quartiers, retirez‑en les pépins et les
parties dures.

Préparez un sirop avec le sucre et de l'eau, ajoutez‑y le jus du citron et le
zeste râpé, puis faites cuire dedans les pommes, à feu vif, pendant vingt
à vingt-cinq minutes, jusqu'à ce qu'il ne reste plus de jus.

Beurrez un moule, versez dedans l'appareil de pommes, laissez refroidir, puis
démoulez et servez avec une crème à la vanille ou, mieux encore, avec du sirop
de framboises.

\sk

Comme variante, on peut conserver les pommes entières après les avoir épluchées
et vidées. On les fait cuire, comme ci-dessus, dans du sirop, on les dresse sur
un compotier et on les sert masquées avec de la crème à la vanille ou du sirop
de framboises. Le goût est le même, la forme seule de l’entremets change.

\sk

On peut apprêter de même des poires.

\section*{\centering Pommes à la gelée de cerises.}
\phantomsection
\addcontentsline{toc}{section}{ Pommes à la gelée de cerises.}
\index{Pommes à la gelée de cerises}

Pour huit personnes prenez :

\footnotesize
\begin{longtable}{rrrp{16em}}
    500 & grammes & de & marmelade épaisse de pommes,                                                     \\
    200 & grammes & de & gelée de cerises,                                                                \\
    125 & grammes & de & sucre,                                                                           \\
    100 & grammes & d' & eau,                                                                             \\
        &         &  8 & belles pommes reinettes,                                                         \\
        &         & 24 & cerises confites.                                                                \\
\end{longtable}
\normalsize

\index{Cuisson du sucre}
\index{Définition de la cuisson du sucre}
Faites un sirop au lissé\footnote{Les quantités de sucre et d'eau employées
pour les sirops sont généralement dans la proportion de {\ppp1\mmm} kilogramme
de sucre pour {\ppp500\mmm} grammes d'eau. \protect

Les différentes phases de la cuisson du sucre portent des noms divers, dont
voici les principaux, par ordre croissant de température, avec leur définition.
\protect

\textit{Sucre à la nappe} : sirop concentré, marque {\ppp32\mmm}° à l'aréomètre
de Beaumé.
\protect

\textit{Au lissé} : un échantillon de sucre au lissé, pris entre deux doigts,
donne, quand on les écarte, un fil fin ; si le fil se rompt immédiatement,
petit lissé, {\ppp34\mmm}° ; si le fil s'allonge davantage sans se rompre,
grand lissé, {\ppp35\mmm}°.
\protect

\textit{Au perlé} : la surface du bain se couvre de perles ; lorsque les perles
sont petites et peu nombreuses, c'est le petit perlé, {\ppp36\mmm}° ; lorsque leur
nombre augmente et qu'elles deviennent plus grosses c'est le grand perlé,
{\ppp37\mmm}°.
\protect

\textit{Au soufflé} : après avoir trempé une écumoire dans un bain de sucre au
soufflé et avoir soufflé dessus, on voit se produire des globules ; si ces
globules restent attachés à l'écumoire, le sucre est au petit soufflé,
{\ppp38\mmm}° ; s'ils se détachent en flocons, il est au grand soufflé ou à la
plume, {\ppp40\mmm}°.
\protect

\textit{À la glu} : en prenant entre les doigts mouillés du sucre à la glu, on
le sent adhérer aux doigts, comme un corps gluant, {\ppp41\mmm}°.
\protect

\textit{Au boulé} : apris avoir pris entre les doigts mouillés du sucre à cette
phase de cuisson, on peut le rouler en boule ; si cette boule est très molle,
le sucre est au petit boulé, {\ppp43\mmm}° ; si elle est plus consistante, il
est au grand boulé, {\ppp44\mmm}°.
\protect

\label{pg0898} \hypertarget{p0898}{}
\textit{Au cassé} : en trempant d'abord le doigt dans de l'eau froide, puis
dans du sucre au cassé et ensuite dans de l'eau froide, le sucre adhérant au
doigt se détache et se brise ; s’il reste collant sous la dent, il est dit au
petit cassé ; si, au contraire, il se casse net, il est dit au grand cassé.
\protect

Si l'on continue à chauffer au-dessus de la température du grand cassé, le
sucre ne tarde pas à se colorer ; lorsqu'il jaunit il se transforme en
\textit{sucre d'orge} ; lorsqu'il brunit, il devient du \textit{caramel}.}
avec le sucre et l'eau.

Pelez les pommmes, creusez-les avec un vide-pommes, puis faites-les cuire dans
le sirop.

Foncez un plat avec la marmelade, dressez dessus les pommes, dont vous emplirez
les creux avec la gelée de cerises ; garnissez avec les cerises confites et
servez.

\sk

On pourra faire de nombreuses variantes en employant d'autres marmelades,
d'autres gelées et d'autres fruits confits.

\section*{\centering Turban de poires, garni de crème Chantilly, à la crème anglaise.}
\phantomsection
\addcontentsline{toc}{section}{ Turban de poires, garni de crème Chantilly, à la crème anglaise.}
\index{Turban de poires, garni de crème Chantilly, à la crème anglaise}

Pour six personnes prenez :

\footnotesize
\begin{longtable}{rrrp{16em}}
    350 & grammes & de & lait,                                                                            \\
    250 & grammes & de & crème Chantilly,                                                                 \\
    210 & grammes & de & sucre,                                                                           \\
      3 & grammes & de & sel,                                                                             \\
        &         &  7 & jaunes d'œufs frais,                                                             \\
        &         &  6 & poires,                                                                          \\
        &         &    & kirsch ou marasquin,                                                             \\
        &         &    & vanille.                                                                         \\
\end{longtable}
\normalsize

Pelez les poires, coupez-les en quartiers et faites-les cuire avec {\ppp60\mmm}
grammes de sucre dans un peu d'eau ; égouttez-les, laissez-les refroidir, puis
dressez-les en turban sur un plat.

Garnissez le centre du turban avec la crème Chantilly.

Préparez la crème anglaise.

Faites bouillir le lait avec de la vanille. Triturez ensemble les jaunes
d'œufs, le reste du sucre et le sel, ajoutez le lait petit à petit, mélangez
bien, puis mettez le tout sur feu doux, tournez jusqu'au moment où la crème
devient bien épaisse ; surveillez pour qu'elle ne tourne pas. Versez-la au
travers d'une passoire dans un autre récipient, remuez-la pour l'aérer, ajoutez
plus ou moins de kirsch ou de marasquin, au goût. Laissez refroidir.

Masquez les poires avec la crème anglaise.

Servez glacé.

\section*{\centering Coings à la compote de pommes.}
\phantomsection
\addcontentsline{toc}{section}{ Coings à la compote de pommes.}
\index{Coings à la compote de pommes}

Pour six personnes prenez :

\footnotesize
\begin{longtable}{rrrp{16em}}
    750 & grammes & de & coings,                                                                          \\
    750 & grammes & de & pommes reinettes grises,                                                         \\
    250 & grammes & de & sucre,                                                                           \\
        &         &    & cidre non mousseux,                                                              \\
        &         &    & eau.                                                                             \\
\end{longtable}
\normalsize

Pelez les coings et les pommes ; coupez-les en tranches.

Préparez une compote serrée avec les pommes, du cidre et {\ppp100\mmm} grammes
de sucre. L'opération dure une vingtaine de minutes sur feu vif ; tout le cidre
doit être évaporé.

Faites cuire les coings dans de l'eau pendant une heure ; puis ajoutez le reste
du sucre et continuez la cuisson à feu doux pendant autant de temps encore, de
façon à obtenir un jus concentré.

Mettez au fond d'un compotier la compote de pommes ; dressez dessus les
tranches de coings, en les faisant chevaucher les unes sur les autres ; laissez
refroidir incomplètement le jus des coings, puis masquez-en les tranches de
coings.

Refroidissez à la glacière.

\section*{\centering Pruneaux au vin.}
\phantomsection
\addcontentsline{toc}{section}{ Pruneaux au vin.}
\index{Pruneaux au vin}

Pour six personnes prenez :

\footnotesize
\begin{longtable}{rrrp{16em}}
    750 & grammes & d' & un vin sucré liquoreux, tel que le muscat de Frontignan,                         \\
    500 & grammes & de & beaux pruneaux d'Agen,                                                           \\
        &         &    & cannelle de Ceylan.                                                              \\
\end{longtable}
\normalsize

Faites bouillir dans une casserole en porcelaine le vin avec la cannelle, de
façon à le bien parfumer, enlevez la cannelle, puis plongez les pruneaux dans
le vin bouillant, retirez la casserole du feu, laissez les pruneaux refroidir
et absorber le liquide.

Au bout de trois jours, les pruneaux sont gonflés, saturés de vin, moelleux et
délicieusement aromatisés,

\sk

Comme variante, on pourra préparer ces pruneaux en ajoutant à la cuisson du jus
et du zeste de citron. On les servira froids ou glacés, avec de la crème
Chantilly.

\section*{\centering Pêches flambées.}
\phantomsection
\addcontentsline{toc}{section}{ Pêches flambées.}
\index{Pêches flambées}

Pour six personnes prenez :

\footnotesize
\begin{longtable}{rrrp{16em}}
    200 & grammes & de & bonne eau-de-vie aromatisée telle que, par exemple,
                         l'eau-de-vie de vin de Châteauneuf-du-Pape,                                      \\
     60 & grammes & de & sucre en poudre,                                                                 \\
        &         &  6 & belles pêches.                                                                   \\
\end{longtable}
\normalsize

Pelez les pêches à la serviette, mettez-les dans une casserole en bi-métal
munie d'un couvercle et pochez-les dans un sirop léger.

Achevez la préparation sur le dressoir de la salle à manger ; placez la
casserole avec son contenu sur un réchaud à alcool ; versez dedans
{\ppp150\mmm} grammes d'eau-de-vie de Châteauneuf-du-Pape, couvrez, chauffez
pendant quelques instants, puis faites flamber.

Pendant le flambage, saupoudrez avec le sucre qui se caramélisera au contact de
la flamme.

Ajoutez ensuite le reste de l’eau-de-vie, flambez encore et servez les pêches en feu.

\sk

\index{Fruits flambés}
On pourra préparer de même d'autres fruits flambés, par exemple des poires
cuites au porto blanc ou au xérès et flambées à la fine champagne.

\section*{\centering Poires glacées.}
\phantomsection
\addcontentsline{toc}{section}{ Poires glacées.}
\index{Poires glacées}

Pour douze personnes prenez :

\footnotesize
\begin{longtable}{rrrrp{16em}}
  & 500 & grammes & de & fraises,                                                                         \\
  & 500 & grammes & de & framboises,                                                                      \\
  & 200 & grammes & de & sucre parfumé à la vanille,                                                      \\
  & 100 & grammes & de & porto rouge,                                                                     \\
  &     & 1 litre & de & vin blanc,                                                                       \\
  & \multicolumn{2}{r}{1/2 litre} & de & crème épaisse,                                                   \\
  &     &         & 12 & belles poires bien mûres, à chair non grenue.                                    \\
\end{longtable}
\normalsize

Pelez les poires en respectant les queues ; faites-les cuire dans le vin
blanc ; laissez-les refroidir dans la cuisson.

Écrasez les fraises et les framboises, recueillez le jus, passez-le et
faites-le rafraîchir sur glace.

Mettez dans un vase placé sur glace la crème et le sucre, battez, ajoutez
ensuite le porto, battez encore. Laissez glacer.

Foncez un plat avec la crème au porto glacée.

Égouttez les poires sur une serviette, disposez-les ensuite sur la crème, puis
versez délicatement sur chaque poire une cuillerée à bouche de jus rafraîchi de
fraises et de framboises, garnissez le pied des fruits avec le reste du jus et
servez.

\section*{\centering Pêches glacées.}
\phantomsection
\addcontentsline{toc}{section}{ Pêches glacées.}
\index{Pêches glacées}

Pour six personnes prenez :

\footnotesize
\begin{longtable}{rrrp{16em}}
    500 & grammes & de & fraises des bois,                                                                \\
    250 & grammes & de & sucre semoule,                                                                   \\
    100 & grammes & de & crème Chantilly,                                                                 \\
     30 & grammes & de & fine champagne,                                                                  \\
        &         &  6 & belles pêches.                                                                   \\
\end{longtable}
\normalsize

Échaudez les pêches à l'eau bouillante, pelez-les ; cuisez-les dans du sirop
pendant quelques minutes, laissez-les refroidir, puis faites-les glacer.

Passez les fraises au tamis de crin, ajoutez-y le sucre, la crème Chantilly et
la fine champagne, mélangez ; mettez à rafraîchir sur glace.

Au moment de servir, masquez les pêches glacées avec le jus rafraîchi.

\section*{\centering Marrons au sucre.}
\phantomsection
\addcontentsline{toc}{section}{ Marrons au sucre.}
\index{Marrons au sucre}

Prenez de beaux marrons, faites-les griller doucement, épluchez-les et
laissez-les refroidir.

Préparez du sucre dit « au cassé ». Pendant la cuisson, déglacez le sirop
à plusieurs reprises, opération qui consiste à enlever la glace qui se dépose
sur les parois de la bassine dans laquelle cuit le sucre ; ce déglaçage
s'exécute facilement à la main en opérant vivement et en ayant soin chaque fois
de plonger la main, au préalable, dans de l’eau froide.

Le sirop étant au point, le sucrage des marrons doit être fait très vite, de la
manière suivante : on pique successivement chaque marron au bout d'une longue
aiguille, on le plonge brusquement dans le sirop et on le met à refroidir sur un
marbre huilé.

Les marrons au sucre ainsi préparés sont excellents ; ils diffèrent des marrons
glacés ordinaires du commerce.

\label{pg0902} \hypertarget{p0902}{}
\section*{\centering Purée de marrons vanillée.}
\phantomsection
\addcontentsline{toc}{section}{ Purée de marrons vanillée.}
\index{Purée de marrons vanillée}

Prenez :

\footnotesize
\begin{longtable}{rrrp{16em}}
  1 000 & grammes & de & purée de marrons obtenue en passant au tamis des marrons
                         triés cuits dans de l'eau bouillante non salée et épluchés
                         pendant qu'ils sont chauds,                                                      \\
  1 000 & grammes & de & sucre en poudre,                                                                 \\
    200 & grammes & d' & eau,                                                                             \\
        &         &    & vanille.                                                                         \\
\end{longtable}
\normalsize

Mettez dans une bassine l'eau et le sucre ; laissez cuire pendant quelques
instants ; ajoutez ensuite la purée de marrons, la vanille, et continuez la
cuisson à feu doux, en remuant constamment avec une cuiller en bois. Au bout
d'une demi-heure environ l'eau doit être évaporée.

Retirez la vanille, laissez refroidir un peu l'appareil et versez-le dans des pots,
ébouillantés au préalable.

\section*{\centering Croûte aux marrons.}
\phantomsection
\addcontentsline{toc}{section}{ Croûte aux marrons.}
\index{Croûte aux marrons}

Pour dix personnes prenez :

\footnotesize
\begin{longtable}{rrrp{16em}}
    125 & grammes & de & marrons glacés concassés,                                                        \\
     60 & grammes & de & vin de Malaga,                                                                   \\
     50 & grammes & de & raisins de Malaga épépinés,                                                      \\
     50 & grammes & d' & angélique confite, coupée en petits morceaux,                                    \\
     15 & grammes & de & fine champagne,                                                                  \\
        &         &  2 & macarons,                                                                        \\
        &         &  1 & brioche de 500 grammes ou un pain rond,                                          \\
        &         &  1 & jaune d'œuf.                                                                     \\
\end{longtable}
\normalsize

Enlevez sur le dessus de la brioche ou du pain une couche mince destinée
à faire couvercle, puis évidez l'intérieur. Séchez cette croûte au four.

Mettez dans un vase les marrons, les raisins, l'angélique, le malaga, mélangez
bien, emplissez la croûte avec ce mélange, fermez avec le couvercle, dorez
l'extérieur au jaune d'œuf battu avec la fine champagne, saupoudrez le dessus
de macarons écrasés, puis faites prendre couleur au four.

\sk

\index{Croûte aux fruits confits}
On peut préparer de même une croûte aux fruits confits, en remplaçant le
mélange ci-dessus par un mélange de fruits confits, de fine champagne et de
kirsch.

\section*{\centering Abricotine aux framboises.}
\phantomsection
\addcontentsline{toc}{section}{ Abricotine aux framboises.}
\index{Abricotine aux framboises}

Pour six personnes prenez :

\footnotesize
\begin{longtable}{rrrp{16em}}
  1 000 & grammes & d' & abricots,                                                                        \\
    500 & grammes & de & framboises,                                                                      \\
     75 & grammes & de & crème fouettée,                                                                  \\
        &         &  3 & feuilles de gélatine blanche,                                                    \\
        &         &    & sucre, au goût.                                                                  \\
\end{longtable}
\normalsize

Écrasez et passez au tamis les abricots crus, sucrez-les fortement.

Lavez la gélatine dans de l’eau, faites-la fondre, passez-la au travers d'une
passoire très fine sur la purée d’abricots, ajoutez la crème fouettée ;
mélangez.

Versez l'appareil dans un moule que vous fermerez hermétiquement en interposant
un papier sous le couvercle ; mettez-le dans de la glace pilée et salée.

Passez les framboises au tamis, sucrez la purée obtenue, faites-la rafraîchir.

Démoulez l'abricotine, entourez-la avec la purée de framboises rafraîchie ;
servez aussitôt.

\section*{\centering Fraises au jus glacé.}
\phantomsection
\addcontentsline{toc}{section}{ Fraises au jus glacé.}
\index{Fraises au jus glacé}

A. — Prenez de belles fraises Héricart, épluchez-les, dressez les plus belles en
pyramide sur un compotier. Passez au tamis les moins belles, ajoutez à la purée
obtenue même poids de sucre et quelques gouttes de jus de citron. Tenez sur
glace pendant une heure.

Au moment de servir, masquez les fraises avec la purée glacée.

\medskip

B. — Prenez des fraises ananas, épluchez-les, dressez les plus belles sur un
compotier. Passez les autres au tamis avec des fraises des quatre saisons en
quantité suffisante. Ajoutez à la purée obtenue même poids de sucre en poudre
et un peu de jus d'orange. Tenez à la glace pendant une heure.

Masquez les fraises avec la purée glacée au moment de servir.

\section*{\centering Framboises au jus.}
\phantomsection
\addcontentsline{toc}{section}{ Framboises au jus.}
\index{Framboises au jus}
\index{Framboises au jus glacé}

Prenez de belles framboises bien saines, épluchez-les ; réservez les plus grosses,
passez les autres au tamis.

Faites cuire la purée de framboises avec même poids de sucre. Lorsqu'elle est
presque froide, ajoutez-y les framboises réservées et mettez le tout à la glace
pendant une heure.

Dressez au moment de servir.

\sk

\index{Fruits au jus glacé}
On peut apprêter d'une façon analogue toutes sortes de fruits.

\section*{\centering Oranges au sucre.}
\phantomsection
\addcontentsline{toc}{section}{ Oranges au sucre.}
\index{Oranges au sucre}

Prenez de bonnes oranges, lourdes et bien mûres. Coupez-les en quartiers,
pelez-les, enlevez les pépins. Mettez les fruits dans un vase avec du sucre en
poudre. Couvrez et laissez en contact pendant une heure à une heure et demie
en les sautant de temps en temps.

Dressez les quartiers d'oranges dans un compotier et arrosez-les simplement
avec le jus que les fruits ont rendu.

\section*{\centering Oranges à l’eau-de-vie.}
\phantomsection
\addcontentsline{toc}{section}{ Oranges à l’eau-de-vie.}
\index{Oranges à l’eau-de-vie}

Pelez de belles oranges, coupez-les en rouelles régulières ; enlevez les
pépins. Dressez les rouelles d'oranges en turban sur un compotier,
saupoudrez-les abondamment de sucre, ajoutez quelques fines languettes de zeste
et de la bonne eau-de-vie. Laissez en contact et arrosez fréquemment avec le
jus.

\section*{\centering Macédoine de fruits.}
\phantomsection
\addcontentsline{toc}{section}{ Macédoine de fruits.}
\index{Macédoine de fruits}

Prenez des fruits frais de saison : fraises, framboises, groseilles, cerises,
pêches, abricots, prunes, raisins, poires, pommes, oranges, mandarines, melons,
etc. ; pelez-les, enlevez les noyaux et les pépins. Laissez entiers les grains
de raisins, les groseilles, les fraises, les framboises, les cerises ; coupez
les autres fruits en tranches, en cubes ou en quartiers.

Faites mariner une partie de ces fruits dans des liqueurs appropriées :
anisette, curaçao, cassis, kirsch, rhum, fine champagne, marasquin, kummel,
etc. ; gardez l’autre partie au naturel.

Prenez aussi des fruits confits et des marrons glacés ; coupez-les en morceaux.

Préparez une sauce consistante avec des jaunes d'œufs, du sucre et de la crème
épaisse, parfumez-la avec du jus de citron et incorporez-y des noix, des noisettes,
des amandes ou des pistaches hachées.

Garnissez des coupes en cristal avec les différents éléments de la macédoine en
les assortissant autant que possible ; masquez avec la sauce et servez très
frais.

\section*{\centering Fruits rafraîchis.}
\phantomsection
\addcontentsline{toc}{section}{ Fruits rafraîchis.}
\index{Fruits rafraîchis}

Les fruits rafraîchis, d’une préparation facile, constituent un excellent
entremets qui plaît à tout le monde.

Le nombre des combinaisons possibles de fruits et de liqueurs étant
considérable, on peut faire des fruits rafraîchis de bien des sortes.

\medskip

En voici trois exemples.

\medskip

1° Pour six personnes prenez :

\footnotesize
\begin{longtable}{rrrp{16em}}
  1 000 & grammes & de & grosses fraises du D\textsuperscript{r} Morère,                                  \\
  1 000 & grammes & de & framboises,                                                                      \\
    500 & grammes & de & groseilles rouges,                                                               \\
    150 & grammes & de & sucre en poudre, parfumé à la vanille,                                           \\
    100 & grammes & de & kirsch,                                                                          \\
     50 & grammes & de & marasquin de Zara.                                                               \\
\end{longtable}
\normalsize

Écrasez les framboises et les groseilles, passez le jus dans un légumier en
métal muni d'un couvercle ; ajoutez-y le sucre, le kirsch et le marasquin ;
mélangez. Mettez les fraises sur le mélange liquide, couvrez et faites
rafraîchir pendant deux heures dans de la glace pilée mélangée de sel.

Servez avec des cuillers à sorbets.

\medskip

2° Prenez des oranges, exprimez-en le jus, mettez dans ce jus des cerises sans
noyaux et des tranches d'ananas ; faites glacer comme ci-dessus.

Cette combinaison ne manque pas de charme.

\medskip

3° Prenez de belles fraises Héricart, des citrons et des oranges.

Faites un sirop de sucre, aromatisez-le avec le jus des citrons et des oranges
relevé par un peu d'essence de zeste des deux fruits, ajoutez les fraises et
mettez à rafraîchir sur glace.

Dans cette formule très simple et peu connue, l'arome des fraises est exalté
par le jus acidulé des citrons et des oranges.

\section*{\centering Jus de fruits rafraîchis.}
\phantomsection
\addcontentsline{toc}{section}{ Jus de fruits rafraîchis.}
\index{Jus de fruits rafraîchis}

On peut faire des jus de fruits avec des fruits crus ou avec des fruits
blanchis à l'eau.

Pour préparer ces jus, prenez toute espèce de fruits et faites-en tels mélanges
que vous suggérera votre goût. Passez-les au tamis ; clarifiez le jus en le
filtrant, puis mettez-le dans une bassine avec du sucre en quantité
suffisante ; chauffez et amenez à bonne consistance.

Faites rafraîchir à la glacière.

Les jus de fruits rafraîchis sont servis avec des biscuits.

\sk

On peut corser les jus de fruits par l'addition d'un peu de vin liquoreux.

\section*{\centering Fruits exotiques au champagne.}
\phantomsection
\addcontentsline{toc}{section}{ Fruits exotiques au champagne.}
\index{Fruits exotiques au champagne}

Prenez : ananas, anones ou pommes de cannelle, barbadines ou pommes-liane,
mammées ou abricots de Saint-Domingue, jambosa ou pommes-rose, monbins ou
pommes de Cythère, corossols, kakis du Japon, mangues, physalis, goyaves,
sapotilles, bananes, grenades, dattes, etc.

Épluchez les fruits, enlevez les noyaux ; coupez la pulpe en morceaux,
mettez-la dans un vase, ajoutez du jus d'orange relevé par du jus de citron des
Antilles, du sucre au goût.

Laissez en contact pendant plusieurs heures, puis ajoutez du champagne doux
et faites rafraîchir à la glacière.

\section*{\centering Melon.}
\phantomsection
\addcontentsline{toc}{section}{ Melon.}
\index{Melon}

Le melon peut être rangé parmi les hors-d'œuvre, parmi les entremets ou
parmi les fruits, suivant les cas.

Si on le sert au naturel, glacé ou non, avec du sel et du poivre, il me semble
logique de le donner comme hors-d'œuvre ; mais dès qu'on le sucre, j'estime
qu'il a plutôt sa place avec les entremets sucrés ou avec les fruits.

Sa digestibihité est extrêmement augmentée quand on lui associe certains vins
ou certaines liqueurs : champagne, madère, xérès, porto blanc, cognac, kirsch,
etc. On peut alors le préparer sous trois formes : entier, en tranches ou en
purée, et le servir rafraîchi, glacé ou confit.

\sk

Lorsque vous voulez présenter le melon entier, entaillez-le du côté de la queue,
de façon à en détacher un bouchon, enlevez les graines par l’orifice ainsi produit,
puis introduisez dans l'intérieur du sucre semoule parfumé à la vanille, un verre
à bordeaux de vin, ou un peu moins de liqueur et mettez à la glacière, au moins
pendant quatre heures, en déplaçant de temps en temps le melon, de façon à faire
pénétrer le liquide de tous les côtés.

\sk

Lorsque vous voulez le servir en tranches, faites-le rafraîchir ou glacez-le.

Pour rafraîchir le melon, comme l'on fait rafraîchir les fruits, découpez-le en
tranches, en ne conservant que la partie comestible, passez le jus que cette
opération a pu donner, réservez-le. Roulez les tranches dans du sucre semoule,
disposez-les dans un légumier en argent, laissez-les mariner pendant trois
heures avec du bon cognac ou du kirsch ; au bout de ce temps, la plus grande
partie du liquide sera absorbée. Mouillez alors avec le jus réservé du melon
et achevez le mouillement avec un verre à bordeaux de vin de Porto blanc, de
Madère ou de Champagne. Couvrez le légumier et mettez-le pendant quatre heures
dans un seau de glace additionnée de sel.

\medskip

On peut servir les tranches séparées ou reconstituer le melon ; c'est une
question d'esthétique.

\sk

Pour glacer le melon, découpez-le en tranches comme ci-dessus, disposez-les
dans un plat posé sur de la glace et arrosez-les avec du vin sucré, une liqueur
ou du champagne. Quand on emploie le champagne, on y ajoute volontiers un peu
de jus de citron.

\sk

Pour préparer le melon en purée, passez la chair au tamis, additionnez-la au
goût de sucre, de vin et de liqueurs : vous aurez ainsi une purée parfumée que
vous ferez rafraîchir et que vous servirez dans des coupes.

\sk

Je ne ferai que mentionner le melon confit, entier ou en tranches, sa
préparation ne présentant rien de particulier.

\sk

Enfin, on peut présenter le melon sous la forme d'une glace ; voici les
proportions des différents éléments qu'il convient d'employer.

Pour trois quarts de litre de purée de melon, prenez trois quarts de litre de
sirop vanillé à {\ppp30\mmm}°, le jus de deux oranges, celui de deux citrons,
un peu de zeste d'orange, quelques gouttes d’eau de fleurs d'oranger et deux
cuillerées à bouche de kirsch. Mélangez ces différents éléments, mettez-les
dans un moule et faites glacer.

\medskip

Toutes ces préparations de melon sont très appréciées.

\section*{\centering Chausson aux poires.}
\phantomsection
\addcontentsline{toc}{section}{ Chausson aux poires.}
\index{Chaussons aux poires}

Pour douze personnes prenez :

\footnotesize
\begin{longtable}{rrrp{16em}}
  1 000 & grammes & de & poires à cuire,                                                                  \\
    500 & grammes & de & farine,                                                                          \\
    400 & grammes & de & beurre,                                                                          \\
     60 & grammes & d' & eau chaude,                                                                      \\
     10 & grammes & de & sel,                                                                             \\
        &         &  1 & œuf frais,                                                                       \\
        &         &    & cannelle de Ceylan en poudre,                                                    \\
        &         &    & sucre en poudre.
\end{longtable}
\normalsize

Pelez les poires, coupez-les en quartiers, saupoudrez-les de cannelle et de
sucre au goût.

Préparez une pâte bien homogène en pétrissant ensemble la farine, le beurre,
le sel et l'eau.

Abaissez la pâte au rouleau, donnez-lui la forme d'un disque ; garnissez la
moitié du disque avec les fruits, fermez en chausson, décorez le dessus, dorez
à l'œuf, puis faites cuire au four pendant une heure environ.

Avec la même quantité d'éléments, on peut faire douze petits chaussons ; dans
ce cas, la cuisson durera un peu moins longtemps.

\sk

\index{Chaussons aux pommes}
On préparera de même des chaussons aux pommes.

\section*{\centering Croquante, sauce Sambaglione.}
\phantomsection
\addcontentsline{toc}{section}{ Croquante, sauce Sambaglione.}
\index{Croquante, sauce Sambaglione}

La croquante est un gâteau délicieux, relalivement peu connu, qui accompagne
très bien une crème ou une glace ; elle peut également être servie avec une
sauce Sambaglione.

\medskip

Pour six personnes prenez :

\medskip

1° pour la croquante :

\footnotesize
\begin{longtable}{rrrp{16em}}
    180 & grammes & de & beurre,                                                                          \\
    150 & grammes & de & sucre semoule,                                                                   \\
    125 & grammes & de & farine de maïs tamisée,                                                          \\
        &         &  1 & œuf frais,                                                                       \\
        &         &    & parfum au goût ;                                                                 \\
\end{longtable}
\normalsize

2° pour la sauce :

\footnotesize
\begin{longtable}{rrrp{16em}}
    200 & grammes & de & vin blanc de Graves ou de Marsala,                                               \\
     80 & grammes & de & sucre en poudre,                                                                 \\
        &         &  6 & jaunes d'œufs frais,                                                             \\
        &         &    & zeste d'un citron.                                                               \\
\end{longtable}
\normalsize

Mettez dans une terrine la farine, le sucre semoule et l'œuf entier, triturez ;
ajoutez le beurre, fondu au bain-marie au préalable ; battez pendant un quart
d'heure ; la pâte doit être sans grumeaux.

Beurrez un moule à croquante\footnote{Un moule à croquante se compose de deux
parties : 1° d'une plaque de tôle circulaire à rebord, formant moule, munie
latéralement d'un anneau ; 2° d'une autre plaque de tôle circulaire formant
couvercle et munie également d'un anneau.
\protect

Les dimensions d'un moule courant, correspondant aux proportions de matières
premières indiquéés ci-dessus, sont les suivantes : diamètre du moule,
{\ppp26\mmm} centimètres ; diamètre du couvercle, {\ppp28\mmm} centimètres ;
hauteur du rebord, {\ppp15\mmm} millimètres.}, saupoudrez-le légèrement de
farine, versez dedans la pâte, couvrez avec un papier beurré et mettez au four
moyen.

Quand la croquante est à peu près cuite, ce qu'on reconnait à la teinte café au
lait claire qu'elle a prise, mettez le couvercle, retournez le moule sur le
couvercle qui fera fond ; achevez la cuisson.

Posez le gâteau sur un plat et coupez-le avant qu'il soit refroidi, tout en lui
conservant sa forme.

\sk

Préparez la sauce Sambaglione ou sabaillon de la façon suivante.

Mélangez dans une casserole les jaunes d'œufs et le sucre en poudre, mettez
sur feu très doux pour commencer, ajoutez le vin par petites quantités, battez
vivement et constamment en augmentant un peu le feu, mais sans faire bouillir ;
cela produit une mousse abondante.

La cuisson sera à point lorsque le tout aura épaissi. Ajoutez alors le zeste
de citron râpé, mélangez bien.

Servez la sauce à part, en même temps que la croquante ;

\sk

La sauce sabaillon est un excellent accompagnement pour nombre de pâtisseries
et pour la plupart des puddings.

\sk

On prépare des sabaillons avec toutes sortes de vins fins.

On fait aussi des sabaillons avec du vin blanc ordinaire, mais alors on les
parfume avec une liqueur ou un spiritueux ({\ppp35\mmm} grammes de liqueur ou
de spiritueux pour {\ppp250\mmm} grammes de sucre, {\ppp250\mmm} grammes de vin
et {\ppp6\mmm} jaunes d'œufs) ; ou avec un autre parfum, au goût.

\sk

On sert fréquemment les sabaillons dans des coupes, comme entremets sucrés,
avec des biscuits.

\section*{\centering Rubans croquants.}
\phantomsection
\addcontentsline{toc}{section}{ Rubans croquants.}
\index{Rubans croquants}

Pour dix personnes prenez :

\footnotesize
\begin{longtable}{rrrp{16em}}
    500 & grammes & de & farine,                                                                          \\
    400 & grammes & de & sucre en poudre,                                                                 \\
    200 & grammes & de & beurre frais,                                                                    \\
    100 & grammes & de & sucre parfumé à la vanille, à l'orange ou au citron,                             \\
     30 & grammes & de & crème fraîche,                                                                   \\
     30 & grammes & de & rhum, kirsch ou cognac,                                                          \\
        &         &  7 & jaunes d'œufs frais,                                                             \\
        &         &  4 & œufs frais.                                                                      \\
\end{longtable}
\normalsize

Préparez une pâte bien lisse avec tous les éléments, moins le sucre parfumé,
travaillez-la bien et faites-en une abaisse aussi mince que possible. Découpez
cette abaisse en rubans de {\ppp1\mmm} centimètre et demi à {\ppp2\mmm}
centimètres de largeur, nouez-les et faites-les frire aussitôt dans de la
graisse de porc bien claire et chaude, en évitant de les brûler. Lorsque les
rubans sont cuits, retirez-les, égouttez-les, séchez-les rapidement sur du
papier buvard, saupoudrez-les de sucre parfumé, pendant qu'ils sont encore
chauds et servez.

\medskip

Les rubans croquants sont excellents avec le thé.

\section*{\centering Madeleines fourrées.}
\phantomsection
\addcontentsline{toc}{section}{ Madeleines fourrées.}
\index{Madeleines fourrées}

\sk

Nombreuses sont les façons de préparer les madeleines : à chaud, à froid ; la
pâte simplement mêlée, ou fouettée.

En voici une formule :

\footnotesize
\begin{longtable}{rrrp{16em}}
    250 & grammes & de & farine tamisée\footnote{On fait parfois des madeleines
                         avec de la crème de riz.},                                                       \\
    250 & grammes & de & sucre en poudre,                                                                 \\
    250 & grammes & de & beurre fin,                                                                      \\
      2 & grammes & de & sel blanc,                                                                       \\
        &         & 10 & œufs frais.                                                                      \\
        &         &    & eau de fleurs d'oranger ou essence de citron,                                    \\
        &         &    & purée de fraises,                                                                \\
        &         &    & sirop de fraises.                                                                \\
\end{longtable}
\normalsize

Coupez le beurre en morceaux et ramollissez-le à la chaleur jusqu'à ce qu'il soit
devenu liquide et crémeux.

Mettez dans une terrine la farine, le sucre, le sel et les œufs, mélez bien
avec une spatule sans travailler la pâte, ajoutez ensuite le beurre fondu et de
l’eau de fleurs d'oranger ou de l'essence de citron ; mélangez à nouveau.

Beurrez au pinceau avec du beurre clarifié des moules à madeleines,
emplissez-les seulement aux deux tiers de la hauteur avec un même poids de pâte
afin d'obtenir des madeleines bien égales et faites cuire au four chaud.

Démoulez les madeleines, mettez-les à refroidir sur un clayon ou un tamis,
puis parez-en le pied pour qu'elles reposent bien droit.

Pour les fourrer, enlevez sur le dessus de chaque madeleine un rond de
{\ppp2\mmm} centimètres de diamètre, évidez l'intérieur, emplissez-le avec de
la purée de fraises, bouchez l'orifice avec le rond et glacez le dessus avec du
sirop épais de fraises.

\sk

Il va sans dire qu'on peut préparer de même des madeleines fourrées avec
d'autres fruits, tels que cerises, ananas, abricots, pistaches, etc. ; les
parfums de la pâte et les sirops devront être modifiés en conséquence : essence
d'orange, essence de noyau, vanille ; sirop de cerises, d'ananas, d’abricots,
etc.

\section*{\centering Biscuit de Savoie.}
\phantomsection
\addcontentsline{toc}{section}{ Biscuit de Savoie.}
\index{Biscuit de Savoie}

Pour six à huit personnes prenez :

\footnotesize
\begin{longtable}{rrrp{16em}}
    375 & grammes & de & sucre en poudre,                                                                 \\
     80 & grammes & de & farine tamisée,                                                                  \\
     80 & grammes & de & fécule,                                                                          \\
      2 & grammes & de & sel,                                                                             \\
        &         & 14 & œufs frais,                                                                      \\
        &         &    & vanille en poudre.                                                               \\
\end{longtable}
\normalsize

Cassez les œufs, séparez les blancs des jaunes ; battez les blancs en neige
ferme.

Mélangez les jaunes d'œufs avec le sucre, le sel et la vanille, battez vivement
ce mélange au fouet jusqu'à ce qu'il soit léger et mousseux ; ajoutez ensuite
la farine et la fécule, par petites quantités, en les faisant tomber en pluie,
puis incorporez à la masse les blancs d'œufs en neige.

Beurrez avec soin, au pinceau, un moule à biscuit avec du beurre frais fondu,
versez dedans l'appareil et faites cuire au four doux pendant une demi-heure
environ.

\section*{\centering Biscuit mousseline.}
\phantomsection
\addcontentsline{toc}{section}{ Biscuit mousseline.}
\index{Biscuit mousseline}

Pour six personnes prenez :

\footnotesize
\begin{longtable}{rrrp{16em}}
    320 & grammes & de & sucre en poudre,                                                                 \\
    160 & grammes & de & fécule,                                                                          \\
      1 & gramme  & de & sel,                                                                             \\
        &         &  8 & œufs frais,                                                                      \\
        &         &    & vanille en poudre,                                                               \\
        &         &    & beurre frais,                                                                    \\
        &         &    & sucre glace.                                                                     \\
\end{longtable}
\normalsize

Cassez les œufs, séparez les blancs des jaunes ; battez les blancs en neige.

Travaillez les jaunes avec le sucre et le sel, le mélange doit mousser ;
ajoutez ensuite la fécule, la vanille ; mélangez bien, puis incorporez
doucement les blancs d'œufs en neige.

Beurrez soigneusement, au pinceau, avec du beurre frais liquéfié, un moule
à biscuit, saupoudrez de sucre glace ; emplissez le moule aux trois quarts avec
l'appareil et faites cuire au four très doux pendant {\ppp25\mmm} minutes.

\sk

\index{Biscuit mousseline fourré}
Comme variante, on pourra fourrer ce biscuit de crème pâtissière.

\section*{\centering Manqué.}
\phantomsection
\addcontentsline{toc}{section}{ Manqué.}
\index{Manqué}

Le premier « manqué » fut un gâteau de Savoie, dont la pâte, non réussie,
avait été améliorée par une addition de beurre et d'amandes pilées avec du sucre.
Il obtint un immense suceës. Mais il fallait le baptiser. Quel nom lui donner ?
On ne chercha pas longtemps : son origine l'indiquait sans conteste.

Les manqués sont entrés dans le répertoire de la pâtisserie ; on les prépare de
différentes manières ; on y ajoute des noisettes, des pistaches, des fruits,
des liqueurs ; on les garnit souvent de gelées et de marmelades de fruits ; on
les décore avec des fruits confits ; enfin, on les glace parfois au chocolat,
au café, avec des sirops, etc.

\medskip

Voici deux exemples de manqué.

\section*{\centering Manqué praliné.}
\phantomsection
\addcontentsline{toc}{section}{ Manqué praliné.}
\index{Manqué praliné}

Pour dix personnes prenez :

\footnotesize
\begin{longtable}{rrrp{16em}}
    400 & grammes & de & sucre en poudre,                                                                 \\
    300 & grammes & de & farine tamisée,                                                                  \\
    140 & grammes & de & beurre fin,                                                                      \\
    125 & grammes & d' & amandes ou d'un mélange en parties égales d'amandes et d'avelines,               \\
     60 & grammes & de & pralin\footnote{Le pralin s'obtient en mélangeant, dans {\ppp400\mmm}
                         grammes de sucre au grand boulé, {\ppp250\mmm} grammes d'amandes hachées
                         ou effilées qu'on laisse cuire un instant.},                                     \\
      5 & grammes & de & sel,                                                                             \\
        &         & 12 & œufs frais,                                                                      \\
        &         &    & fruits confits,                                                                  \\
        &         &    & vanille en poudre,                                                               \\
        &         &    & sucre glace,                                                                     \\
        &         &    & beurre frais.                                                                    \\
\end{longtable}
\normalsize

Pilez les amandes avec un œuf.

Cassez les autres œufs ; séparez les blancs des jaunes ; enlevez les germes.
Battez les blancs en neige ferme. Travaillez les jaunes d'œufs avec le sucre en
poudre, le sel et la vanille ; l'appareil doit être mousseux. Ajoutez ensuite
la farine, légèrement séchée au four, en la faisant tomber en pluie, puis le
beurre que vous aurez fait fondre et les amandes pilées. Mélangez soigneusement
de façon à avoir une belle pâte, sans grumeaux, à laquelle vous incorporerez
doucement les blancs d'œufs battus.

Beurrez copieusement, au pinceau. avec du beurre frais fondu, un moule à manqué
ou à génoise, saupoudrez de sucre glace, versez dedans l'appareil et faites
cuire au four moyen, un peu ouvert.

Lorsque le gâteau est cuit, masquez-le avec le pralin et décorez-le, à volonté,
avec des fruits confits.

Laissez refroidir.

\section*{\centering Manqué glacé au café.}
\phantomsection
\addcontentsline{toc}{section}{ Manqué glacé au café.}
\index{Manqué glacé au café}

Pour huit personnes prenez :

\footnotesize
\begin{longtable}{rrrp{16em}}
    375 & grammes & de & sucre en poudre,                                                                 \\
    125 & grammes & de & farine tamisée,                                                                  \\
    125 & grammes & de & fécule,                                                                          \\
    125 & grammes & de & beurre fin,                                                                      \\
     90 & grammes & de & noisettes ou de pistaches,                                                       \\
     20 & grammes & de & rhum ou de fine champagne,                                                       \\
      3 & grammes & de & sel,                                                                             \\
        &         &  9 & œufs frais,                                                                      \\
        &         &    & sirop à 32°,                                                                     \\
        &         &    & reste de citron.                                                                 \\
        &         &    & sucre glace,                                                                     \\
        &         &    & essence de café.                                                                 \\
\end{longtable}
\normalsize

Cassez les œufs, séparez les blancs des jaunes ; enlevez les germes. Montez
sept blancs en neige ferme.

Pilez les noisettes ou les pistaches avec un peu de blanc d'œuf.

Mettez dans une terrine les jaunes d'œufs, le sucre et le sel ; triturez bien de
façon à obtenir un ensemble mousseux ; ajoutez ensuite, par petites quantités, en
pluie, la farine et la fécule, puis le beurre que vous aurez fait fondre, du zeste de
citron râpé, au goût, le rhum ou la fine champagne et les noisettes ou les pistaches.
Mélangez bien ; enfin, incorporez doucement les blancs d'œufs en neige.

Prenez un moule à manqué, beurrez-le comme il faut au pinceau, saupoudrez
de sucre glace, puis emplissez-le aux trois quarts avec l'appareil.

Faites cuire au four doux pendant une demi-heure environ.

Au sortir du four, glacez le gâteau avec du sirop à {\ppp32\mmm}° additionné
d'essence de café.

Laissez refroidir.

\label{pg0915} \hypertarget{p0915}{}
\section*{\centering Choux à la crème.}
\phantomsection
\addcontentsline{toc}{section}{ Choux à la crème.}
\index{Choux à la crème}

Pour six à huit personnes prenez :

\footnotesize
\begin{longtable}{rrrp{16em}}
    250 & grammes & de & farine,                                                                          \\
    250 & grammes & d’ & eau,                                                                             \\
    200 & grammes & de & beurre,                                                                          \\
     15 & grammes & de & sucre,                                                                           \\
      5 & grammes & de & sel,                                                                             \\
        &         &  8 & œufs frais.                                                                      \\
\end{longtable}
\normalsize

Mettez dans une casserole l'eau, le sucre. le sel et {\ppp150\mmm} grammes de
beurre, chauffez, retirez la casserole du feu après le premier bouillon.

Délayez la farine avec le contenu de la casserole, mélangez et travaillez bien
la pâte ; elle doit être sans grumeaux.

Remettez la casserole sur le feu pendant cinq minutes pour sécher la pâte ;
surveillez et évitez qu'elle s'attache au fond.

Quand la pâte est suffisamment séchée, ajoutez les œufs un à un et en même
temps le reste du beurre partagé en six morceaux ; travaillez bien après chaque
addition. La pâte sera à point lorsque, après en avoir pris un peu dans une
cuiller que l'on retourne, son poids la fait tomber sans quelle s'étale.

Mettez sur une plaque en tôle des boules de pâte de {\ppp5\mmm} à {\ppp6\mmm}
centimètres de diamètre, dorez-les à l'œuf et faites-les cuire au four
moyennement chaud, un peu ouvert. Fendez les choux sur le côté et, à l’aide
d'un cornet, fourrez-les de crème Saint-Honoré, à la vanille, au chocolat ou au
café.

On peut, au lieu de fendre les choux sur le côté, enlever sur le dessus un
couvercle de {\ppp3\mmm} centimètres de diamètre et garnir l'intérieur des
choux avec de la crème à la vanille, au citron, aux fruits ; dans ce dernier
cas, on ne dore pas les choux, on les glace avec du sucre au cassé.

\sk

\index{Éclairs}
On peut préparer de même des éclairs ; la composition de la pâte est la même
que celle des choux.

On dresse sur une plaque en tôle des bâtonnets de pâte de {\ppp10\mmm}
centimètres de longueur environ, on ne les dore pas et on les fait cuire au
four assez chaud.

Lorsqu'ils sont cuits, on les garnit par le côté, comme les choux, avec des
crèmes au café, au chocolat, à la vanille, aux fruits ; on glace le dessus au
café, au chocolat, ou avec un sirop rappelant le parfum et la garniture de
l'intérieur.

\sk

On,prépare d'une façon très analogue des profiteroles pour entremets, mais la
pâte contient moins de beurre ({\ppp100\mmm} grammes au lieu de {\ppp200\mmm}
grammes). On donne aux profiteroles {\ppp4\mmm} à {\ppp5\mmm} centimètres de
diamètre et on les sèche à l'étuve. On les fourre généralement de crème
pâtissière à la vanille, au chocolat, au café, et on les sert avec une crème
assortie.

\section*{\centering Saint-Honoré.}
\phantomsection
\addcontentsline{toc}{section}{ Saint-Honoré.}
\index{Saint-Honoré}

Pour cinq à six personnes prenez :

\medskip

1° pour la croûte :

\footnotesize
\begin{longtable}{rrrrp{16em}}
  & 225 & grammes & de & lait,                                                                            \kill
  &     &         &    & pâte à foncer,  \hyperlink{p0320}{p. \pageref{pg0320}},                          \\
  &     &         &    & pâte à choux,   \hyperlink{p0915}{p. \pageref{pg0915}},                          \\
  &     &         &    & sucre au cassé, \hyperlink{p0898}{p. \pageref{pg0898}},                          \\
  &     &         &    & sucre glace,                                                                     \\
  &     &         &    & œuf frais ;                                                                      \\
\end{longtable}
\normalsize

2° pour la crème :

\footnotesize
\begin{longtable}{rrrrp{16em}}
  & 225 & grammes & de & lait,                                                                            \\
  & 125 & grammes & de & sucre en poudre,                                                                 \\
  &  25 & grammes & de & farine,                                                                          \\
  & \multicolumn{2}{r}{1/2 gramme} & de & sel,                                                            \\
  &     &         &  3 & œufs frais,                                                                      \\
  &     &         &    & vanille.                                                                         \\
\end{longtable}
\normalsize

Mettez sur une plaque en tôle une abaisse de pâte à foncer de {\ppp22\mmm}
centimètres de diamètre et de {\ppp2\mmm} à {\ppp3\mmm} millimètres
d'épaisseur, disposez sur le pourtour une couronne de pâte à choux que vous
dorerez à l'œuf ; faites cuire au four chaud. Saupoudrez le dessus avec de la
glace de sucre, c’est-à-dire avec du sucre en poudre très fine, et glacez au
four. Laissez refroidir.

Préparez des boules de pâte à choux de {\ppp2\mmm} centimètres de diamètre,
faites-les cuire au four doux ; laissez-les refroidir, puis trempez-les dans du
sucre au cassé et posez-les sur le four de la couronne.

Apprêtez la crème.

Cassez les œufs, séparez les blancs des jaunes. Battez les blancs en neige
ferme ; pendant l'opération, incorporez-y le sucre en poudre.

\index{Crème Saint-Honoré, à la vanille}
Délayez dans le lait la farine et les jaunes d'œufs ; mélangez bien, puis
versez le tout, au travers d'une passoire, dans une casserole ; ajoutez le sel
et la vanille ; chauffez. Au premier bouillon, éloignez la casserole du feu,
puis ajoutez doucement les blancs d'œufs battus en neige. Remettez la casserole
sur le feu, laissez cuire un instant, mais pas trop pour ne pas enlever la
légèreté à la crème ; retirez la vanille. Laissez refroidir.

Garnissez, à la cuiller, l'intérieur du gâteau avec la crème.

\sk

\index{Crème Saint-Honoré au café}
\index{Crème Saint-Honoré au chocolat}
On fera de même des Saint-Honoré à la crème au chocolat ou au café,

\sk

On fait aussi des Saint-Honoré dont la couronne est garnie de boules de pâte
à choux alternant avec des quartiers d'orange, le tout glacé avec du sucre au
cassé : dans ce cas on parfume la crème avec de l'orange et du curaçao.

\sk

\index{Crème Saint-Honoré aux fruits}
On peut encore faire des Saint-Honoré aux fruits : fraises, ananas, abricots,
etc.

Voici comment on devra opérer.

Préparez une croûte comme il est dit plus haut, mais remplacez les boules de
choux par de belles et grosses fraises si vous faites un gâteau aux fraises,
par des mirabelles si le gâteau est aux abricots, par des cubes d’ananas dans
la préparation à l'ananas. Glacez ces fruits au cassé.

Apprêtez une crème aux fraises, aux abricots ou à l'ananas, de la façon
suivante :

Passez au tamis des fraises, des abricots ou de l'ananas, ajoutez à la purée du
sucre en poudre en quantité suffisante, un peu de gélatine blanche rafraîchie et
dissoute, pour donner de la fermeté, puis mélangez cet appareil à de la crème
Chantilly. Faites prendre sur glace.

Au dernier moment, garnissez le gâteau avec la crème et servez sur un plat
recouvert d'une serviette.

\section*{\centering Pain de Gênes.}
\phantomsection
\addcontentsline{toc}{section}{ Pain de Gênes.}
\index{Pain de Gênes}

Prenez :

\footnotesize
\begin{longtable}{rrrp{16em}}
    675 & grammes & de & sucre,                                                                           \\
    500 & grammes & d' & amandes mondées fraîches ou sèches, au choix,                                    \\
    250 & grammes & de & beurre,                                                                          \\
    125 & grammes & de & farine fine,                                                                     \\
      2 & grammes & de & sel,                                                                             \\
        &         & 10 & œufs frais,                                                                      \\
        &         &    & vanille ou kirsch.                                                               \\
\end{longtable}
\normalsize

Pilez les amandes au mortier avec les œufs, ajoutez‑y le sucre, le sel,
travaillez bien ; l'appareil doit mousser. Mettez ensuite la farine, mélangez
avec une spatule, enfin incorporez intimement à la pâte le beurre rendu
liquide. Parfumez l'appareil avec de la vanille ou du kirsch, au goût.

Prenez un moule cannelé, beurrez-le surtout sur les côtés, mettez au fond un
disque de papier fin, puis l'appareil et faites cuire au four un peu chaud.

\sk

On peut faire le pain de Gênes plus léger en ajoutant à la pâte {\ppp3\mmm} ou
{\ppp4\mmm} blancs d'œufs battus en neige. Mais il faut alors le faire cuire
à la température plus douce du four moyennement chaud, autrement le gâteau
retomberait.

\section*{\centering Génoise\footnote{La pâte à génoise est employée parfois
pour faire le fond de certains gâteaux. Dans ce cas, on la couche sur des
plaques d'office bien beurrées.}.}
\phantomsection
\addcontentsline{toc}{section}{ Génoise.}
\index{Génoise}

Pour huit personnes prenez :

\footnotesize
\begin{longtable}{rrrp{16em}}
    250 & grammes & de & farine de gruau tamisée,                                                         \\
    250 & grammes & de & sucre,                                                                           \\
    225 & grammes & de & beurre fin,                                                                      \\
    100 & grammes & d' & amandes douces,                                                                  \\
      6 & grammes & d' & amandes amères,                                                                  \\
      3 & grammes & de & sel,                                                                             \\
        &         &  8 & œufs frais,                                                                      \\
        &         &    & vanille en poudre.                                                               \\
\end{longtable}
\normalsize

Échaudez, mondez les amandes ; pilez-les avec un œuf.

Mettez dans une bassine tenue au bain-marie les œufs qui restent, le sucre, le
sel, la vanille ; battez fortement de façon à faire mousser la pâte ; puis,
lorsqu'elle est devenue légère, retirez la bassine du bain-marie et continuez
à battre à froid jusqu'à ce que la pâte ait augmenté d'environ un tiers de son
volume. Ajoutez alors la farine, en pluie, mélangez convenablement ; enfin
incorporez le beurre fondu tiède. Travaillez bien.

Beurrez, au pinceau, un moule à manqué ou à biscuit, coulez dedans la pâte de
génoise et faites cuire au four moyen.

\sk

On pourra préparer une pâte à génoise sans amandes, qu'on fourrera avec
différentes crèmes ou avec des fruits : marmelades ou confitures. Dans ces
derniers cas, on glacera le gâteau avec du sirop parfumé avec une liqueur
assortie aux fruits.

\section*{\centering Moka.}
\phantomsection
\addcontentsline{toc}{section}{ Moka.}
\index{Moka}

Pour huit personnes prenez :

\medskip

1° pour la pâte :

\footnotesize
\begin{longtable}{rrrp{16em}}
    250 & grammes & de & farine de gruau tamisée,                                                         \\
    250 & grammes & de & sucre en poudre,                                                                 \\
    225 & grammes & de & beurre fin,                                                                      \\
    100 & grammes & d’ & amandes douces,                                                                  \\
      6 & grammes & d' & amandes amères,                                                                  \\
      3 & grammes & de & sel,                                                                             \\
        &         &  8 & œufs frais,                                                                      \\
        &         &    & vanille en poudre ;                                                              \\
\end{longtable}
\normalsize

2° pour la crème :

\footnotesize
\begin{longtable}{rrrp{16em}}
    300 & grammes & de & beurre fin,                                                                      \\
    250 & grammes & de & sirop à 32°,                                                                     \\
      1 & gramme  & de & sel,                                                                             \\
        &         &  8 & jaunes d'œufs frais,                                                             \\
        &         &    & essence de café.                                                                 \\
\end{longtable}
\normalsize

Avec les éléments du premier paragraphe, faites une génoise comme il est dit
ci-dessus.

\index{Crème moka}
Préparez la crème.

Mettez les jaunes d'œufs avec le sel dans une bassine, versez dessus le sirop
un peu refroidi, par petites quantités, en remuant vivement, ajoutez de
l'essence de café au goût. Portez la bassine sur feu doux et faites épaissir le
mélange en remuant toujours ; puis passez l'appareil à la passoire fine.

Ramollissez le beurre en crème, versez dessus, en fouettant, l'appareil passé,
Continuez à fouetter la crème jusqu'à ce qu'elle soit luisante et qu'elle ait
du corps.

Coupez le gâteau horizontalement en trois ou quatre tranches que vous garnirez
avec de la crème : glacez-le, décorez-le avec le reste de la crème. Enfin
saupoudrez le dessus avec un peu de sucre cristallisé.

\sk

On pourra préparer dans le même esprit un gâteau garni de crème au chocolat.

\section*{\centering Tôt-fait suprême.}
\phantomsection
\addcontentsline{toc}{section}{ Tôt-fait suprême.}
\index{Tôt-fait suprême}

Pour huit personnes prenez :

\medskip

1° pour le gâteau :

\footnotesize
\begin{longtable}{rrrp{16em}}
    250 & grammes & de & farine tamisée,                                                                  \\
    200 & grammes & de & beurre tiède,                                                                    \\
    200 & grammes & de & sucre en poudre,                                                                 \\
      2 & grammes & de & sel,                                                                             \\
        &         &  6 & jaunes d'œufs frais,                                                             \\
        &         &  3 & œufs frais,                                                                      \\
        &         &    & vanille en poudre,                                                               \\
        &         &    & amandes hachées ;                                                                \\
\end{longtable}
\normalsize

\medskip

2° pour la crème :

\footnotesize
\begin{longtable}{rrrp{16em}}
    200 & grammes & de & lait,                                                                            \\
    175 & grammes & de & beurre fin,                                                                      \\
    150 & grammes & de & sucre,                                                                           \\
     60 & grammes & de & pralin, moitié amandes, moitié noisettes,                                        \\
      1 & gramme  & de & sel,                                                                             \\
        &         &  4 & jaunes d'œufs frais,                                                             \\
        &         &    & vanille,                                                                         \\
        &         &    & sirop au café.                                                                   \\
\end{longtable}
\normalsize

Mélangez ensemble les éléments du premier paragraphe moins les amandes,
battez bien le mélange, puis mettez-le dans un moule à manqué ou à génoise
beurré et fariné, semez dessus quelques amandes hachées, poudrez légèrement de
sucre et faire cuire au four moyen ouvert.

Laissez refroidir le gâteau.

Préparez la crème suprême.

\index{Crème suprême}
Triturez les jaunes d'œufs avec le sucre ; le mélange doit mousser.

Faites bouillir le lait avec le sel et de la vanille. Éloignez la casserole du
feu, retirez la vanille, ajoutez les jaunes d'œufs, puis remettez la casserole
sur feu doux. Lorsque l'appareil a bien épaissi et qu'il est près de bouillir,
enlevez la casserole du feu et fouettez jusqu'à ce que ce soit presque froid.
Versez le mélange, toujours en fouettant, sur le beurre que vous aurez amené
à être crémeux par la chaleur ; continuez à fouetter et, lorsque la crème est
bien lisse et qu'elle a du corps, incorporez-y le pralin.

Fourrez le gâteau avec cette crème, moins une partie qui vous servira pour la
décoration.

Glacez le tôt-fait avec du sirop au café et décorez-le avec le reste de la crème.

\section*{\centering Tartes.}
\phantomsection
\addcontentsline{toc}{section}{ Tartes.}
\index{Tartes}
\index{Définition des tartes}

On désigne sous le nom de tartes des entremets sucrés constitués par une croûte
basse en pâtisserie, faite au moule ou sans moule, relevée sur les côtés qui sont
tenus un peu plus épais que le fond et dont le bord est légèrement crénelé, qu'on
garnit de fruits, de confitures ou de crèmes.

La croûte est faite indifféremment avec de la pâte à foncer ou de la pâte
brisée, mais on peut aussi employer pour sa confection une pâte fine analogue
à celle qui sert pour la couverture des tartes à l'anglaise\footnote{Les tartes
à l'anglaise sont des préparations de fruits cuits dans un plat couvert d'une
abaisse de pâte.}.

On cuit généralement ensemble croûte et remplissage, exception faite pour les
tartes contenant pêche, raisin, fraise ou framboise, dans lesquelles les fruits sont
cuits au sirop, puis disposés dans les croûtes cuites à part.

Lorsque les tartes sont cuites, on recouvre la surface des fruits avec une
couche de sirop, de gelée ou de confiture à l’abricot, à la groseille, à la
framboise, à la cerise, assortie avec le remplissage.

\section*{\centering Tarte aux pommes.}
\phantomsection
\addcontentsline{toc}{section}{ Tarte aux pommes.}
\index{Tarte aux pommes}

Pour six personnes prenez :

\footnotesize
\begin{longtable}{rrrp{16em}}
  1 000 & grammes & de & pommes reinettes grises,                                                         \\
    300 & grammes & de & farine,                                                                          \\
    300 & grammes & de & beurre,                                                                          \\
     25 & grammes & d' & eau,                                                                             \\
      3 & grammes & de & sel,                                                                             \\
        &         &    & gelée d'abricots,                                                                \\
        &         &    & kirsch,                                                                          \\
        &         &    & sucre en poudre.                                                                 \\
\end{longtable}
\normalsize

Préparez une pâte homogène avec la farine, le beurre, l'eau et le sel,
laissez-la reposer pendant une douzaine d'heures. Faites-en une abaisse carrée
de {\ppp8\mmm} à {\ppp9\mmm} millimètres d'épaisseur que vous mettrez sur une
plaque en tôle ; garnissez le pourtour de l'abaisse d’un petit rebord en pâte,
crénelez-le au sommet. Piquez le fond en différents endroits avec la pointe
d'un couteau afin d'éviter que la pâte se boursoufle à la cuisson, saupoudrez
de sucre, puis disposez dessus, par rangées, des tranches de pommes pelées et
épépinées qui chevaucheront les unes sur les autres.

Faites cuire au four chaud pendant une vingtaine de minutes. Le four doit être
très chaud au début de l'opération pour que la pâte soit bien saisie ; on
abaissera ensuite un peu sa température.

Liquéfiez la gelée d'abricots, aromatisez-la avec du kirsch, glacez-en les pommes
et servez.

C'est, à mon avis, l’une des meilleures tartes aux pommes que l'on puisse faire.

\sk

Comme variante, on pourra faire une tarte mi-partie pommes, mi-partie coings,
de la façon suivante.

On fera cuire : les pommes en purée avec un peu de cidre, les coings coupés en
tranches dans un sirop de sucre et la croûte à part.

On garnira le fond de la croûte avec la purée de pommes sur laquelle on
disposera les tranches de coings en les faisant chevaucher, on glacera le
dessus avec le sirop de cuisson des coings mélangés avec de la confiture
d'abricots passée.

Cette tarte sera servie froide.

\sk

Les tartes aux abricots seront apprêtées de la même manière. On les parsèmera
d'amandes d'abricots et on les glacera à la gelée d’abricots parfumée avec un
peu de curaçao ou de kummel.

\section*{\centering Tarte aux fraises.}
\phantomsection
\addcontentsline{toc}{section}{ Tarte aux fraises.}
\index{Tarte aux fraises}

Pour six personnes prenez :

\footnotesize
\begin{longtable}{rrrp{16em}}
  1 000 & grammes & de & fraises Héricart,                                                                \\
    300 & grammes & de & farine,                                                                          \\
    300 & grammes & de & beurre,                                                                          \\
     25 & grammes & d' & eau,                                                                             \\
      3 & grammes & de & sel,                                                                             \\
        &         &    & gelée de groseilles framboisée.                                                  \\
\end{longtable}
\normalsize

Préparez la pâte avec la farine, le beurre, l'eau et le sel ; laissez-la
reposer pendant {\ppp12\mmm} heures, puis faites-en une abaisse de {\ppp8\mmm}
à {\ppp9\mmm} nullimètres d'épaisseur, que vous mettrez sur une plaque en tôle.
Entourez l'abaisse d'un petit rebord en pâte ; piquez le fond en différents
endroits avec la pointe d'un couteau. Emplissez la croûte avec un corps inerte,
cailloux ou haricots, et faites cuire au four chaud. Lorsque la croûte est
cuite, laissez-la refroidir.

Mettez les fraises épluchées dans du sirop à la plume ; lorsque l'ébullition
commence, enlevez-les, laissez-les refroidir.

Foncez la croûte avec une couche de gelée de groseilles framboisée, disposez
dessus les fraises refroidies, glacez-les avec de la gelée de groseilles
framboisée et servez.

\sk

On préparera de même la tarte aux cerises, la gelée de groseilles pourra être
parfumée avec du marasquin.

\sk

Les tartes aux prunes : Reine-Claude, mirabelle, quetch, seront apprêtées d'une
façon identique : le glaçage seul différera. Il sera fait à la gelée d'abricots au
kirsch ou à la gelée de cerises au marasquin.

\section*{\centering Tarte aux dattes.}
\phantomsection
\addcontentsline{toc}{section}{ Tarte aux dattes.}
\index{Tarte aux dattes}

Pour six personnes prenez :

\footnotesize
\begin{longtable}{rrrp{16em}}
    800 & grammes & de & dattes bien en chair,                                                            \\
    500 & grammes & de & beurre,                                                                          \\
    300 & grammes & de & farine,                                                                          \\
    125 & grammes & de & noisettes mondées,                                                               \\
     25 & grammes & d' & eau,                                                                             \\
      3 & grammes & de & sel,                                                                             \\
        &         &    & sucre en poudre.                                                                 \\
\end{longtable}
\normalsize

Préparez une pâte homogène avec la farine, {\ppp300\mmm} grammes de beurre,
l'eau, le sel. Laissez-la reposer pendant {\ppp12\mmm} heures.

Hachez grossièrement les noisettes et grillez-les au four.

Retirez les noyaux des dattes, passez la pulpe au tamis en vous aidant d'un
pilon en bois, vous obtiendrez ainsi environ {\ppp600\mmm} grammes de purée de
dattes que vous triturerez au mortier avec le reste du beurre et les deux tiers
des noisettes, de façon à obtenir un mélange lisse.

Faites avec la pâte une abaisse de {\ppp8\mmm} à {\ppp9\mmm} millimètres
d'épaisseur que vous placerez sur une plaque en tôle, entourez-la d'un petit
rebord en pâte, piquer le fond avec la pointe d'un couteau en plusieurs
endroits, saupoudrez de sucre, emplissez l'intérieur d'un corps inerte,
cailloux ou haricots, et faites cuire au four chaud.

Laissez la croûte refroidir légèrement au sortir du four, garnissez-la avec la
pâte de dattes, parsemez le dessus avec le reste des noisettes grillées et
servez.

Cette tarte originale est exquise.

\section*{\centering Tarte à la crème.}
\phantomsection
\addcontentsline{toc}{section}{ Tarte à la crème.}
\index{Tarte à la crème}

Pour six personnes prenez :

\footnotesize
\begin{longtable}{rrrp{16em}}
    400 & grammes & de & lait,                                                                            \\
     75 & grammes & de & sucre,                                                                           \\
     15 & grammes & de & kirsch, de curaçao ou de marasquin,                                              \\
      1 & gramme  & de & sel,                                                                             \\
        &         &  3 & œufs frais,                                                                      \\
        &         &  1 & gousse de vanille,                                                               \\
        &         &    & cerises glacées,                                                                 \\
        &         &    & abricots glacés,                                                                 \\
        &         &    & chinois dorés glacés,                                                            \\
        &         &    & angélique glacée,                                                                \\
        &         &    & pâte à foncer ou pâte à tarte.                                                   \\
\end{longtable}
\normalsize

Faites bouillir le lait pendant cinq minutes avec le sucre, le sel et la vanille ;
retirez la vanille.

Battez les œufs, délayez-les dans le lait, ajoutez le kirsch, le curaçao ou le
marasquin, mélangez bien.

Garnissez un moule à tarte de {\ppp20\mmm} centimètres de diamètre avec la pâte
choisie, emplissez l'intérieur d'un corps inerte pour empêcher toute
déformation et mettez au four.

Lorsque la pâte est suffisamment cuite, enlevez le corps inerte, versez dans la
croûte, au travers d'une passoire, la crème parfumée ci-dessus, parsemez-la de
fruits glacés coupés en morceaux ; faites prendre au four moyennement chaud.

Servez froid.

\section*{\centering Tartelettes.}
\phantomsection
\addcontentsline{toc}{section}{ Tartelettes.}
\index{Tartelettes}
\index{Définition des tartelettes}
\index{Croûte pour tartelettes}

Les tartelettes sont de petites tartes. On les fait aux fruits et à la crème.
Leur croûte est constituée le plus souvent par de la pâte feuilletée, de la
pâte à gâteau de plomb ou de la pâte aux amandes, mais on peut aussi faire des
tartelettes avec les pâtes employées pour les tartes.

\sk

Pour faire de la pâte à gâteau de plomb, prenez :

\footnotesize
\begin{longtable}{rrrp{16em}}
    150 & grammes & de & farine,                                                                          \\
    100 & grammes & de & beurre,                                                                          \\
     30 & grammes & de & crème double,                                                                    \\
      3 & grammes & de & sel,                                                                             \\
      3 & grammes & de & sucre,                                                                           \\
        &         &  1 & œuf frais.                                                                       \\
\end{longtable}
\normalsize

Mélanger ensemble tous les éléments moins la crème, pétrissez bien et fraisez
trois fois en humectant chaque fois avec la crème. La pâte doit être assez molle :
si elle était trop ferme, on ajouterait encore un peu de crème.

Laissez reposer la pâte pendant une heure avant de vous en servir.

\sk

Pour faire de la pâte aux amandes prenez :

\footnotesize
\begin{longtable}{rrrp{16em}}
    150 & grammes & de & farine,                                                                          \\
    100 & grammes & d' & amandes mondées,                                                                 \\
    100 & grammes & de & sucre en poudre,                                                                 \\
     50 & grammes & de & beurre,                                                                          \\
      9 & grammes & de & sel,                                                                             \\
        &         &  2 & œufs frais,                                                                      \\
        &         &    & zeste de citron.                                                                 \\
\end{longtable}
\normalsize

Pilez les amandes au mortier. Triturez ensemble la farine, les amandes pilées,
le sucre, le beurre, le sel, ajoutez ensuite les œufs, pétrissez bien : la pâte
doit être ferme. Laissez-la reposer pendant une heure.

\sk

Pour faire des tartelettes, prenez une des pâtes indiquées, au choix, faites-en
une abaisse mince avec laquelle vous garnirez des moules à tartelettes ; piquez
les fonds avec la pointe d'un couteau en plusieurs endroits, puis emplissez les
moules ainsi garnis avec un corps sans saveur, petits cailloux de rivière bien
lavés ou haricots secs, afin d'éviter la déformation des croûtes et faites
cuire au four assez chaud.

Emplissez les croûtes avec des fruits cuits au sirop, couvrez-les avec un
papier beurré pour mitiger l'action de la chaleur sur les fruits, remettez-les
pendant un instant au four.

Laissez refroidir les tartelettes, puis glacez le bord des croûtes avec un peu de
sirop.

Passez le sirop de cuisson des fruits, faites-en une réduction à {\ppp32\mmm}°,
versez-en un peu sur les tartelettes et servez.

\section*{\centering Flans.}
\phantomsection
\addcontentsline{toc}{section}{ Flans.}
\index{Flans}
\index{Définition des flans}
\index{Flans (Définition des)}

Les flans sont des entremets qu'on prépare dans des croûtes semblables à celles
des tartes, ou sans croûte. Apprêtés en croûte et aux fruits, ils sont
recouverts d'un grillage en pâte ; apprêtés sans croûte, ils sont moulés dans
des tourtières.

On fait des flans aux fruits, à différentes crèmes, au fromage blanc, au riz.

\section*{\centering Flan de crème pralinée.}
\phantomsection
\addcontentsline{toc}{section}{ Flan de crème pralinée.}
\index{Flan de crème pralinée}

Pour six personnes prenez :

\footnotesize
\begin{longtable}{rrrp{16em}}
    750 & grammes & de & lait,                                                                            \\
    150 & grammes & de & farine,                                                                          \\
    150 & grammes & de & sucre,                                                                           \\
      2 & grammes & de & sel,                                                                             \\
        &         & 24 & pralines grises,                                                                 \\
        &         &  5 & œufs frais,                                                                      \\
        &         &  4 & jaunes d'œufs frais,                                                             \\
        &         &    & vanille.                                                                         \\
\end{longtable}
\normalsize

Chauffez le lait avec la vanille, ne le faites pas bouillir.

Mélangez la farine avec le sucre et le sel, ajoutez‑y les œufs entiers et les
jaunes d'œufs, travaillez bien le mélange, délayez-le ensuite avec le lait
chaud en évitant les grumeaux, puis incorporez les pralines, pilées au mortier
et passées au tamis fin.

Enduisez fortement de beurre une tourtière, versez dedans la pâte en une
couche assez épaisse et faites cuire au four doux.

Un flan bien réussi doit être moelleux et souple.

\section*{\centering Flan à la marmelade de pommes.}
\phantomsection
\addcontentsline{toc}{section}{ Flan à la marmelade de pommes.}
\index{Flan à la marmelade de pommes}

Préparez une pâte, au choix, faites-en une abaisse mince, d'une épaisseur de
{\ppp3\mmm} millimètres environ ; foncez-en un moule à flan que vous aurez mis
sur un plafond. Beurrez l'intérieur au pinceau avec du beurre fondu, garnissez
avec de la marmelade de pommes reinettes, sucrée et aromatisée avec un peu de
zeste de citron.

Faites avec les déchets de la pâte une abaisse de même épaisseur que la
précédente et de la grandeur du flan, saupoudrez-la de farine, coupez dedans
des languettes de {\ppp4\mmm} à {\ppp5\mmm} millimètres de largeur avec
lesquelles vous formerez un grillage sur la marmelade ; soudez-en les
extrémités sur le bord du flan.

Faites cuire au four chaud, puis glacez le dessus au sirop.

\sk

\index{Flans messins}
Dans les flans messins, au lieu de garnir la croûte avec de la marmelade, on
dispose sur la pâte des tranches de pommes reinettes, coupées en croissants
d'un centimètre d'épaisseur, qu'on range en rosace, en les faisant chevaucher
les unes sur les autres, depuis le bord du flan jusqu’au milieu sur lequel on
pose un rond tourné ou non. Lorsque le flan est cuit, on le masque avec une
couche de marmelade d'abricots passée et étendue avec du sirop, ou avec une
couche de gelée de cerises ou de groseilles délayée avec du sirop ; on couvre
avec une grille de pâte dorée à l'œuf, cuite à part, et glacée au sucre ou au
sirop.

\sk

\index{Flans (Différentes manières d'apprêter les)}
On peut encore faire des flans mi-partie marmelade, mi-partie tranches de
pommes. On met de la marmelade au fond de la croûte, on couvre cette marmelade
avec des croissants de pommes chevauchant les uns sur les autres comme dans les
flans messins.

\sk

\index{Flans polonais}
Dans les flans polonais, on garnit la croûte aux trois quarts avec des tranches
de pommes sur lesquelles on dispose, en rosace, des croissants de pâte feuilletée
qu'on dore à l'œuf et qu'on glace lorsque la cuisson est achevée.

\section*{\centering Tourte aux fruits.}
\phantomsection
\addcontentsline{toc}{section}{ Tourte aux fruits.}
\index{Tourte aux fruits}

Faites une abaisse en pâte feuilletée, découpez dedans un disque du diamètre
que vous voulez donner à la tourte, posez-le sur un plafond beurré puis, avec
la même pâte, préparez une bande de {\ppp3\mmm} centimètres de largeur sur
{\ppp2\mmm} centimètres d'épaisseur, de longueur suffisante pour faire le tour
de la tourte, et un couvercle.

Mouillez le pourtour du disque et collez la bande dessus ; saupoudrez de sucre
fin l'intérieur de la tourte, l'extérieur de la bande de pâte et le dessus du
couvercle.

Disposez dans la tourte des fruits privés de leurs noyaux, entiers ou coupés en
deux suivant leur grosseur, en laissant un vide d'un centimètre entre la bande
et les fruits ; couvrez. Faites cuire au four chaud, glacez au sirop et laissez
refroidir. Suivant le fruit employé, vous pourrez finir la tourte soit avec des
abricots que vous garnirez avec les amandes des noyaux et que vous glacerez au
sirop d'abricots, soit avec des cerises ou des prunes que vous glacerez au
sirop de cerises ou au sirop de mirabelles. Dans l'un ou dans l'autre cas,
saupoudrez d'un peu de sucre cristallisé, laissez refroidir, mettez ensuite le
couvercle et servez.

\sk

On fait aussi des tourtes aux marmelades ou aux gelées de fruits : on les
couvre soit avec un couvercle fait d'une abaisse de pâte, décorée à volonté et
glacée au sirop, soit avec un grillage de pâte comme pour les flans.

\sk

Enfin on fait des tourtes à la crème.

\section*{\centering Millefeuille.}
\phantomsection
\addcontentsline{toc}{section}{ Millefeuille.}
\index{Millefeuille}

Pour 10 à 12 personnes prenez :

\footnotesize
\begin{longtable}{rrrp{16em}}
    600 & grammes & de & farine fine tamisée,                                                             \\
    600 & grammes & de & beurre fin,                                                                      \\
    325 & grammes & d' & eau,                                                                             \\
     12 & grammes & de & sel,                                                                             \\
        &         &  3 & jaunes d'œufs frais,                                                             \\
        &         &    &  marmelade ou confiture d'abricots,                                              \\
        &         &    &  marmelade de pommes et d'abricots,                                              \\
        &         &    &  gelée de groseilles framboisée ou de cerises,                                   \\
        &         &    &  sirop ou pâte à meringue,                                                       \\
        &         &    &  fruits cuits au sirop ou petits gâteaux en pâte feuilletée,                     \\
        &         &    &  fruits confits.                                                                 \\
\end{longtable}
\normalsize

Préparez un feuilletage de {\ppp8\mmm} à {\ppp10\mmm} tours avec la farine, le
beurre, l'eau, les jaunes d'œufs et le sel ; abaissez-le au rouleau, à une
épaisseur de 1 centimètre et demi environ. Divisez-le en {\ppp16\mmm} ou
{\ppp18\mmm} parties, dont vous formerez des disques de {\ppp18\mmm}
à {\ppp20\mmm} centimètres de diamètre ; faites-en deux plus larges que les
autres, destinés à former le dessus et le dessous du gâteau. Laissez-les
reposer pendant deux heures au moins.

Mouillez les abaisses avec du blanc d'œuf délayé avec un peu d'eau ; posez-les
sur des plaques beurrées, saupoudrez-les de sucre, piquez-les avec une
fourchette pour empêcher qu'elles se boursouflent à la cuisson et mettez-les au
four moyennement chaud. La cuisson demande une demi-heure environ ; les
galettes doivent être d'une belle couleur blonde. Laissez-les refroidir.

Montez le gâteau en prenant comme fond l'une des deux plus grandes galettes sur
laquelle vous mettrez une couche de marmelade ou de confiture d'abricots,
placez au-dessus une autre galette plus petite que vous couvrirez de gelée de
groseilles framboisée ou de gelée de cerises, puis continuez à superposer les
galettes que vous garnirez de même manière en observant la même alternance.
Finissez par la seconde grande galette.

Garnissez le tour du gâteau, entre les deux grandes galettes, avec de la
marmelade très épaisse de pommes et d'abricots ; lissez, puis glacez le gâteau
avec de la marmelade ou de la confiture d'abricots chaude, étendue légèrement
avec du sirop parfumé au kirsch. Ou bien, au lieu de glacer le gâteau,
masquez-le avec de la pâte à meringue ; dans ce dernier cas, saupoudrez de
sucre et passez le gâteau, pendant un moment, au four pour solidifier la
meringue.

On peut aussi, au goût, glacer le millefeuille avec du sirop au rhum.

Décorez soit avec des fruits cuits au sirop, soit avec des ronds, des losanges ou
des croissants de petites dimensions, en pâte feuilletée, creusés au milieu et dans
le creux desquels vous mettrez des fruits confits : par exemple, des cerises.

\section*{\centering Gâteau à l'orange.}
\phantomsection
\addcontentsline{toc}{section}{ Gâteau à l'orange.}
\index{Gâteau à l'orange}

Pour huit personnes prenez :

\footnotesize
\begin{longtable}{rrrp{16em}}
    250 & grammes & de & farine,                                                                          \\
    150 & grammes & de & beurre,                                                                          \\
    150 & grammes & de & sucre semoule,                                                                   \\
      2 & grammes & de & bicarbonate de soude,                                                            \\
        &         &  4 & œufs frais.                                                                      \\
        &         &    & écorces d'oranges confites,                                                      \\
        &         &    & confiture d'oranges.                                                             \\
        &         &    & curaçao blanc.                                                                   \\
\end{longtable}
\normalsize

Mettez dans une terrine la farine, le beurre ramolli au préalable et le sucre ;
travaillez bien ; ajoutez ensuite les œufs un à un en travaillant la pâte après chaque
addition, et enfin le bicarbonate de soude. Travaillez encore l'appareil.

Emplissez à moitié avec cet appareil un moule beurré, parsemez l'intérieur de
tranches d'écorces d'oranges confites et faites cuire au four chaud. On
reconnait que la cuisson est à point lorsqu'une paille enfoncée dans le gâteau
en sort sèche.

Servez avec accompagnement d'une sauce faite avec de la confiture d'oranges
passée délayée dans du sirop et relevée par un peu de curaçao.

Le gâteau à l'orange est aussi bon froid que chaud.

\sk

On pourra préparer dans le même esprit des gâteaux à d'autres fruits qu'on
garnira avec des fruits confits assortis, par exemple un gâteau aux cerises,
garni de cerises confites, servi avec une sauce à la confiture de cerises.

\section*{\centering Gâteau fourré aux fruits.}
\phantomsection
\addcontentsline{toc}{section}{ Gâteau fourré aux fruits.}
\index{Gâteau fourré aux fruits}

La veille du jour où vous voudrez confectionner ce gâteau, préparez un levain
avec un peu de farine, un œuf, une cuillerée d’eau et un peu de levure de
bière. Le lendemain, mélangez ce levain à :

\footnotesize
\begin{longtable}{rrrp{16em}}
    250 & grammes & de & farine,                                                                          \\
    180 & grammes & de & beurre,                                                                          \\
     20 & grammes & de & sucre en poudre,                                                                 \\
     10 & grammes & de & sel blanc,                                                                       \\
        &         &  4 & œufs frais.                                                                      \\
\end{longtable}
\normalsize

Incorporez les œufs l'un après l'autre en travaillant la pâte d'un large
mouvement rotatif vertical, de façon à la bien aérer.

Lorsque la pâte est devenue parfaitement homogène, mettez-la dans une terrine
couverte ; laissez-la monter. Quand elle aura suffisamment monté, au bout d'une
heure environ, retournez-la sens dessus dessous dans la terrine et laissez-la monter
une deuxième fois.

Beurrez un moule droit, large, versez dedans la pâte ; secouez pour la tasser
et laissez-la monter pour la troisième fois.

Entourez le haut du moule d'une collerette de papier que vous fixerez au moyen
d'une ficelle, mettez au four chaud et laissez cuire pendant une demi-heure
environ.

Démoulez le gâteau pendant qu'il est chaud ; laissez-le refroidir.
Décalottez-le ensuite, creusez l'intérieur en laissant des bords suffisamment
épais, emplissez le vide avec des fruits frais cuits dans du sirop, saucez avec
du sirop aromatisé, couvrez avec la partie du gâteau enlevée et amincie, puis
décorez le dessus avec des fruits confits et de la gelée de confitures.

\sk

Lorsqu'on n'a pas de fruits frais, en hiver par exemple, on peut exécuter ce
gâteau avec des fruits conservés.

Voici une formule simple de remplissage préparé avec des pêches en boîtes ou
en flacons.

Décantez le sirop qui accompagne les pêches, mêlez-le avec des confitures
d'abricots ou de prunes, chauffez, liez avec un peu de fécule, ajoutez ensuite
les pêches, parfumez avec un peu de kirsch ou d'essence de noyau et versez le
mélange pas trop chaud dans le gâteau préparé comme ci-dessus et légèrement
chauffé.

\section*{\centering Gâteau de Compiègne à l'ananas.}
\phantomsection
\addcontentsline{toc}{section}{ Gâteau de Compiègne à l'ananas.}
\index{Gâteau de Compiègne à l'ananas}

Pour douze personnes prenez :

\footnotesize
\begin{longtable}{rrrp{16em}}
    500 & grammes & de & farine,                                                                          \\
    500 & grammes & de & beurre,                                                                          \\
    125 & grammes & de & levure de bière,                                                                 \\
    100 & grammes & de & sucre en poudre,                                                                 \\
     10 & grammes & de & sel,                                                                             \\
        &         & 15 & jaunes d'œufs frais,                                                             \\
        &         &  3 & œufs frais,                                                                      \\
        &         &    & ananas,                                                                          \\
        &         &    & fruits confits,                                                                  \\
        &         &    & sirop d'abricots,                                                                \\
        &         &    & sirop de sucre parfumé au kirsch et au marasquin.                                \\
\end{longtable}
\normalsize

Confectionnez un gâteau de Compiègne avec les matières désignées ci-dessus,
moins l'ananas, les fruits confits et les sirops, en opérant de la façon suivante.

Faites d'abord un levain avec {\ppp125\mmm} grammes de farine et la levure.

Préparez une pâte bien lisse avec le reste de la farine, les œufs entiers, le
sel et le sucre, laissez-la reposer ; ajoutez ensuite les jaunes d'œufs, le
beurre et le levain ; travaillez bien.

Mettez la pâte dans un moule à biscuit de Savoie, beurré, faites cuire au four
moyennement chaud ; puis démoulez et laissez refroidir.

Coupez le gâteau en tranches, saupoudrez-les de sucre, glacez-les au four, puis
dressez-les en turban sur un plat, en intercalant entre elles des tranches
d'ananas.

Coupez de l'ananas en gros dés, mettez-les dans du sirop d'abricots, chauffez ;
garnissez-en le centre du turban et décorez le dessus avec des fruits confits.

Au dernier moment, arrosez le gâteau avec du sirop chaud parfumé au kirsch et
au marasquin ; servez aussitôt.

\section*{\centering Gâteau aux pistaches.}
\phantomsection
\addcontentsline{toc}{section}{ Gâteau aux pistaches.}
\index{Gâteau aux pistaches}

Pour six personnes prenez :

\medskip

1° pour le gâteau :

\footnotesize
\begin{longtable}{rrrp{16em}}
    130 & grammes & de & sucre cristallisé                                                                \\
    120 & grammes & de & farine de riz,                                                                   \\
    100 & grammes & de & beurre fin,                                                                      \\
     30 & grammes & de & marasquin,                                                                       \\
      2 & grammes & de & sel,                                                                             \\
        &         &  4 & œufs frais,                                                                      \\
        &         &    & beurre fondu,                                                                    \\
        &         &    & farine ;                                                                         \\
\end{longtable}
\normalsize

2° pour le garnissage :

\footnotesize
\begin{longtable}{rrrp{16em}}
    150 & grammes & de & sucre cristallisé,                                                               \\
    150 & grammes & de & pistaches,                                                                       \\
    125 & grammes & de & beurre fin,                                                                      \\
    100 & grammes & de & kirsch,                                                                          \\
        &         &  3 & blancs d'œufs frais.                                                             \\
\end{longtable}
\normalsize

Faites fondre {\ppp100\mmm} grammes de beurre ; clarifiez-le.

Mettez les œufs avec le sucre et le sel dans un vase légèrement chauffé,
fouettez pendant un quart d'heure environ, de manière à bien aérer la masse ;
incorporez ensuite la farine de riz, mélangez rapidement pour que l'ensemble
reste léger, ajoutez le beurre clarifié tiède, le marasquin ; mélangez encore.

Graissez un moule cylindrique de {\ppp4\mmm} à {\ppp5\mmm} centimètres de
hauteur avec du beurre fondu, saupoudrez de farine et emplissez-le aux trois
quarts avec l'appareil.

Quarante-cinq minutes de cuisson au four moyennement chaud suffisent.

Laissez refroidir le gâteau ; démoulez-le.

Pendant la cuisson, préparez la garniture.

Pilez {\ppp80\mmm} grammes de pistaches mondées avec le kirsch.

Battez les blancs d'œufs en neige.

Faites cuire le sucre au soufflé, laissez-le un peu refroidir, puis
incorporez‑y les {\ppp125\mmm} grammes de beurre par petites quantités, les
pistaches pilées avec le kirsch et mélangez le tout aux blancs d'œufs battus en
neige, en évitant de mouiller le fouet.

Coupez le gâteau en trois tranches horizontales, étendez sur toutes les faces
une couche de l'appareil aux pistaches, puis reconstituez le gâteau, en
séparant les tranches les unes des autres par un semis de pistaches hachées.

Décorez le dessus avec le reste de l'appareil et quelques pistaches, grillées ou non.

\sk

Comme variante, on pourra garnir le gâteau avec la pâte de pistaches suivante :

\footnotesize
\begin{longtable}{rrrp{16em}}
    250 & grammes & de & pistaches mondées,                                                               \\
    200 & grammes & de & sucre en poudre,                                                                 \\
    125 & grammes & d' & amandes douces sèches, mondées,                                                  \\
        &         &  2 & œufs frais,                                                                      \\
        &         &    & vanille,                                                                         \\
        &         &    & vert d'épinards.                                                                 \\
\end{longtable}
\normalsize

Cassez les œufs, séparez les blancs des jaunes.

Pilez amandes et pistaches au mortier avec de la vanille ; pendant l'opération
ajoutez par petites quantités le sucre, les jaunes d'œufs, {\ppp1\mmm} ou
{\ppp2\mmm} blancs suivant que vous voulez avoir une pâte ferme ou une pâte
molle. Colorez avec du vert d'épinards ; triturez jusqu'à obtention d'une pâte
homogène et suffisamment consistante.

Si les blancs sont battus en neige, ne les incorporez que tout à fait à la fin et
sans les faire tomber.

\sk

On pourra préparer dans le même esprit des gâteaux aux amandes, aux noix ou
aux noisettes.

\section*{\centering Gâteau feuilleté aux amandes.}
\phantomsection
\addcontentsline{toc}{section}{ Gâteau feuilleté aux amandes.}
\index{Gâteau feuilleté aux amandes}

Pour six personnes prenez :

\footnotesize
\begin{longtable}{rrrp{16em}}
    500 & grammes & de & pâte feuilletée,                                                                 \\
    300 & grammes & d' & amandes douces,                                                                  \\
    250 & grammes & de & sucre en poudre,                                                                 \\
    250 & grammes & de & beurre,                                                                          \\
     20 & grammes & d' & eau de fleurs d'oranger,                                                         \\
     15 & grammes & d' & amandes amères,                                                                  \\
      2 & grammes & de & sel blanc,                                                                       \\
        &         &  4 & œufs frais,                                                                      \\
        &         &  1 & jaune d'œuf.                                                                     \\
\end{longtable}
\normalsize

 Mettez les amandes douces et les amandes amères pendant cinq minutes dans de
 l'eau bouillante, rafraîchissez-les, mondez-les, séchez-les au four doux.
 Pilez-les ensuite au mortier en ajoutant un œuf par petites quantités pendant
 l'opération, afin que la pâte ne tourne pas en huile.

Quand la pâte est devenue bien fine, travaillez-la avec le sucre, le beurre, le
sel, l'eau de fleurs d'oranger et les trois autres œufs que vous mettrez l'un
après l'autre ; malaxez bien après l'addition de chaque œuf jusqu'à obtention
d'une pâte bien liée, puis sortez-la du mortier et mettez-la dans une terrine.

Faites avec la pâte feuilletée deux abaisses semblables de {\ppp1\mmm}
centimètre d'épaisseur, posez une abaisse sur une plaque de tôle, étalez dessus
la pâte d'amandes, en réservant tout autour un bord non fourré de {\ppp2\mmm}
centimètres, mouillez avec un peu d’eau cette partie réservée, couvrez avec
l'autre abaisse, soudez ensemble les bords des deux abaisses, dorez le dessus
du gâteau avec du jaune d'œuf étendu d'eau, et faites cuire au four à feu
modéré pendant {\ppp50\mmm} minutes. Glacez le dessus avec du sucre en poudre.

Retirez le gâteau du feu, posez-le sur un tamis ; laissez-le refroidir.

\sk

Au lieu d'un grand gâteau, on peut faire, cela va sans dire, autant de petits
gâteaux qu'il y a de convives.

\sk

On pourra préparer dans le même esprit une pâte feuilletée fourrée avec des
marmelades de fruits. Au lieu de pâte d'amandes, on mettra sur l’abaisse des
marmelades d'abricots, de pêches, de pommes, etc., ou des confitures de
cerises, de prunes, de fraises, etc.

\section*{\centering Gâteau suave.}
\phantomsection
\addcontentsline{toc}{section}{ Gâteau suave.}
\index{Gâteau suave}

Le gâteau suave comprend une pâte et un remplissage.

\medskip

Pour douze personnes prenez :

\medskip

1° pour la pâte :

\footnotesize
\begin{longtable}{rrrp{16em}}
    500 & grammes & de & farine de gruau tamisée,                                                         \\
    250 & grammes & d' & amandes douces mondées et hachées fin,                                           \\
    250 & grammes & de & beurre,                                                                          \\
    250 & grammes & de & sucre en poudre,                                                                 \\
     10 & grammes & de & sel fin,                                                                         \\
        &         &  6 & jaunes d'œufs frais,                                                             \\
        &         &  2 & blancs d'œufs frais ;                                                            \\
\end{longtable}
\normalsize

2° pour le remplissage :

\footnotesize
\begin{longtable}{rrrp{16em}}
    500 & grammes & de & farine de gruau tamisée,                                                         \kill
        &         &    & \hangindent=1em amandes douces ou pistaches, ou mi-partie d'amandes
                         douces et mi-partie de pistaches, ou encore une pâte
                         d'amandes fondante\footnote{Pour préparer de la pâte
                         d'amandes fondante, pilez au mortier des amandes avec
                         une liqueur fine, incorporez-y par petites quantités
                         et en triturant fortement du sucre cuit au cassé, le
                         tout dans les proportions suivantes :
                         \protect\endgraf
                         \begin{tabular}{rrrclr}
                         \hspace{2.5em} &           &         &    &                              & \\
                         \hspace{2.5em} & 350 à 500 & grammes & de & sucre,                       & \\
                         \hspace{2.5em} &       250 & grammes & d’ & amandes,                     & \\
                         \hspace{2.5em} &   30 à 35 & grammes & de & la liqueur choisie
                                                                  (kirsch, curaçao, anisette,     & \\
                         \hspace{2.5em} &           &         &    & marasquin, par exemple).     & \\
                         \end{tabular}},                                                                  \\
        &         &    & sucre,                                                                           \\
        &         &    & vanille en poudre,                                                               \\
        &         &    & \hangindent=1em confitures variées de fraises, framboises, cerises,
                                         groseilles, ananas, etc.,                                        \\
        &         &    & jus de citron ;                                                                  \\
\end{longtable}
\normalsize

3° pour la décoration :

\footnotesize
\begin{longtable}{rrrp{16em}}
    125 & grammes & de & sucre en poudre,                                                                 \\
        &         &  1 & blanc d'œuf frais,                                                               \\
        &         &    & gelées de fruits,                                                                \\
        &         &    & fruits confits,                                                                  \\
        &         &    & vanille en poudre.                                                               \\
\end{longtable}
\normalsize

Travaillez ensemble tous les éléments du premier paragraphe excepté les blancs
d'œufs, de façon à obtenir une pâte homogène dont vous ferez {\ppp5\mmm} ou
{\ppp6\mmm} disques très minces\footnote{On peut, cela va sans dire, faire plus
de disques, si l'on veut un remplissage plus varié.}. Passez les disques dans
les blancs d'œufs battus, placez-les sur une tôle beurrée et faites-les cuire
des deux côtés au four moyennement chaud. Laissez-les refroidir.

Préparez une pâte d'amandes ou de pistaches en triturant dans un mortier des
amandes ou des pistaches avec du sucre, de la vanille et du jus de citron ; ou
bien préparez une pâte d'amandes fondante, comme il est dit dans la note.

Disposez les disques les uns sur les autres, en les séparant par des couches
plus épaisses de pâte d'amandes ou de pistaches, ou de pâte d'amandes fondante
et de confitures variées, qui constituent le remplissage.

Glacez le gâteau avec une glace royale bien homogène, lisse et blanche obtenue
en fouettant le blanc d'œuf avec le sucre en poudre, passé au tamis de soie, et
un peu de vanille en poudre, puis en travaillant le mélange à la spatule ;
décorez le dessus avec des gelées de fruits découpées et des fruits confits.
Mettez au four pendant quelques minutes seulement pour sécher le glaçage sans
le colorer.

Laissez refroidir.

Les personnes les plus réfractaires aux charmes de la pâtisserie sont obligées
de déclarer ce gâteau suave.

\section*{\centering Gâteau d'amandes au miel.}
\phantomsection
\addcontentsline{toc}{section}{ Gâteau d'amandes au miel.}
\index{Gâteau d'amandes au miel}

Pour douze personnes prenez :

\footnotesize
\begin{longtable}{rrrp{16em}}
    500 & grammes & de & miel,                                                                            \\
    500 & grammes & de & farine,                                                                          \\
    375 & grammes & de & sucre en poudre,                                                                 \\
    250 & grammes & d' & amandes douces,                                                                  \\
    100 & grammes & d' & eau,                                                                             \\
     65 & grammes & d' & écorce d'orange confite,                                                         \\
     65 & grammes & d' & écorce de citron confite,                                                        \\
     30 & grammes & de & kirsch,                                                                          \\
     20 & grammes & de & cannelle en poudre,                                                              \\
      8 & grammes & de & levure,                                                                          \\
      6 & grammes & de & sel fin,                                                                         \\
      5 & grammes & de & girofle en poudre,                                                               \\
      5 & grammes & de & muscade râpée,                                                                   \\
        &         &  1 & citron.                                                                          \\
\end{longtable}
\normalsize

Coupez en lanières fines ou plus simplement hachez grossièrement les amandes
ainsi que les écorces d'orange et de citron.

Exprimez le jus du citron ; râpez le zeste.

Faites bouillir le miel dans une casserole, versez-le ensuite dans une terrine,
ajoutez les amandes, les écorces d'orange et de citron, le kirsch,
{\ppp250\mmm} grammes de sucre, la cannelle, le girofle, la muscade, le sel, le
jus de citron et le zeste râpé ; mêlez avec une cuiller en bois, puis ajoutez
la levure dissoute dans un peu d'eau, la farine, mélangez bien ; vous
obtiendrez une pâte que vous pourrez laisser reposer un ou deux jours ou
employer de suite. Dans les deux cas, la préparation réussit aussi bien.

Abaissez la pâte à {\ppp1\mmm} centimètre d'épaisseur.

Saupoudrez de farine une tôle carrée, posez la pâte sur cette tôle et faites
cuire au four, à feu vif.

Au sortir du four, coupez le gâteau en morceaux de {\ppp3\mmm} centimètres sur
{\ppp4\mmm} sans les séparer, en vous servant d'une règle pour que les morceaux
soient réguliers.

Mettez le reste du sucre dans le reste de l'eau, faites cuire au soufflé ;
puis, avec un pinceau, glacez d'un seul coup la surface de chaque morceau avec
ce sirop.

Si le glaçage se trouvait manqué, on pourrait le recommencer aussitôt la
première couche refroidie.

Lorsque le gâteau est bien froid, séparez les morceaux aux endroits marqués par
les coupures, brossez-les pour enlever la farine qui pourrait y adhérer encore,
et dressez-les sur un compotier,

\section*{\centering Galette des rois.}
\phantomsection
\addcontentsline{toc}{section}{ Galette des rois.}
\index{Galette des rois}

Pour six personnes prenez :

\footnotesize
\begin{longtable}{rrrp{16em}}
    400 & grammes & de & farine,                                                                          \\
    350 & grammes & de & beurre,                                                                          \\
    200 & grammes & d' & eau,                                                                             \\
      8 & grammes & de & sel,                                                                             \\
        &         &    & jaune d'œuf.                                                                     \\
\end{longtable}
\normalsize

Préparez une pâte feuilletée ; abaissez-la au rouleau à une épaisseur de
{\ppp1\mmm} centimètre {\ppp1\mmm}/{\ppp2\mmm} environ ; donnez-lui une forme
ronde ou carrée, et insérez dedans la fève traditionnelle.

Faites des entailles parallèles dans deux directions croisées sur la face
supérieure, dorez-la au jaune d'œuf délayé avec un peu d'eau ; mettez au four
moyen.

La durée moyenne de la cuisson est d’une demi-heure environ. On doit obtenir un
gâteau bien doré, également feuilleté sur toute son épaisseur et sans aucune
interposition de parties pâteuses.

Afin d'éviter que le feuilleté s'affaisse, il est important de ne pas ouvrir le
four avant la fin de la cuisson.

\section*{\centering Galette fourrée.}
\phantomsection
\addcontentsline{toc}{section}{ Galette fourrée.}
\index{Galette fourrée}

Préparez une abaisse de pâte feuilletée de {\ppp1\mmm} centimètre
{\ppp1\mmm}/{\ppp2\mmm} d'épaisseur ; coupez-la transversalement en deux
parties égales.

Abaissez séparément chaque moitié à {\ppp3\mmm}/{\ppp4\mmm} de centimètre
d'épaisseur.

Étalez sur l'une de la confiture de fruits : abricots, fraises, etc., couvrez
avec la seconde abaisse, soudez les bords à la pince, décorez le dessus à l'œuf
et faites cuire au four comme la galette des rois.

\section*{\centering Palmiers.}
\phantomsection
\addcontentsline{toc}{section}{ Palmiers.}
\index{Palmiers}

Pour faire des palmiers, préparez une pâte feuilletée comme pour la galette des
rois, mais saupoudrez-la de sucre après chacun des deux derniers tours.

Abaissez-la à {\ppp3\mmm}/{\ppp4\mmm} de centimètre d'épaisseur, pliez-la en
quatre dans le sens de la longueur en mettant les deux extrémités en dedans.

Coupez ensuite l'abaisse en travers, en tranches de {\ppp3\mmm}/{\ppp4\mmm} de
centimètre d'épaisseur : disposez ces tranches sur une plaque sans qu'elles se
touchent, saupoudrez-les de sucre fin et faites-les cuire au four chaud.

Les palmiers sont excellents avec le thé.

\section*{\centering Brioche.}
\phantomsection
\addcontentsline{toc}{section}{ Brioche.}
\index{Brioche}

La brioche est un excellent gâteau à pâte levée ; sa composition a beaucoup
d'analogie avec celle du baba. On donne généralement à la brioche l'une des
trois formes suivantes : boule à tête, couronne, cylindre ; cette dernière
forme est réservée tout particulièrement pour la brioche « mousseline ».

La composition de la pâte à brioche est assez variable, et nombreuses sont les
façons de la préparer.

Voici deux formules de brioche, la première extrêmement fine ; la deuxième plus
ordinaire et pouvant convenir parfaitement pour faire des croustades.

\bigskip
\centering\sc
\footnotesize
première formule
\bigskip

\justifying
\normalfont
\normalsize

Pour six personnes prenez :

\footnotesize
\begin{longtable}{rrrp{16em}}
    650 & grammes & de & beurre fin,                                                                      \\
    500 & grammes & de & farine de gruau tamisée,                                                         \\
    100 & grammes & d' & eau tiède,                                                                       \\
     25 & grammes & de & sucre,                                                                           \\
     15 & grammes & de & levure de bière, en hiver ; 10 grammes, en été,                                  \\
     10 & grammes & de & sel,                                                                             \\
        &         &  8 & œufs frais\footnote{Avec les quantités indiquées
                         on peut faire soit une grosse brioche à tête ou en
                         couronne, soit deux brioches mousseline de plus d'un
                         litre de volume, soit encore dix petites brioches à tête.}.                      \\
\end{longtable}
\normalsize

\medskip
\centering\sc
\footnotesize
deuxième formule
\bigskip

\justifying
\normalfont
\normalsize

Pour six personnes prenez :

\footnotesize
\begin{longtable}{rrrp{16em}}
    500 & grammes & de & farine tamisée,                                                                  \\
    375 & grammes & de & beurre,                                                                          \\
    100 & grammes & d' & eau tiède,                                                                       \\
     15 & grammes & de & levure de bière, en hiver ; 10 grammes en été,                                   \\
     10 & grammes & de & sucre,                                                                           \\
     10 & grammes & de & sel,                                                                             \\
        &         &  6 & œufs frais.                                                                      \\
\end{longtable}
\normalsize

\begin{center}
\textit{Préparation de la pâte.}
\end{center}

Préparez un levain pas trop dur avec {\ppp125\mmm} grammes de farine et la
levure délayée avec un peu d'eau tiède ; travaillez-le, roulez-le en boule,
faites-y quelques entailles éparses au couteau ou entaillez-le en croix ;
laissez-le lever à une température douce.

Disposez en fontaine, sur une planche ou sur un tour, le reste de la farine ;
mettez le sel et le sucre, fondus dans le reste de l'eau tiède, {\ppp3\mmm}
œufs ; travaillez bien la pâte en la coupant et en la soulevant de toutes
manières de façon à l'aérer et à lui faire prendre du corps ; ajoutez ensuite
les autres œufs, un à un, en travaillant après chaque addition. Lorsque la pâte
est bien lisse et bien homogène, incorporez-y d'abord le beurre, non ramolli,
puis le levain. Mélangez bien en coupant la pâte. La préparation de la pâte
fine demande de grands soins.

\index{Définition de la locution : rompre la pâte}
Mettez la pâte dans une terrine et faites-la lever à température douce ;
rompez-la deux ou trois fois, c'est-à-dire abaissez-la et pliez-la en quatre ;
laissez-la reposer dans un endroit frais après chaque opération.

Au moment de vous en servir, moulez-la dans la forme que vous voudrez donner
à la brioche.

\sk

Pour faire la brioche à tête, séparez la pâte en deux parties inégales et
roulez-les en boules, l'une très grosse formant le corps de la brioche, l'autre
plus petite destinée à former la tête. Pratiquez dans le haut de la grosse
boule une encoche dans laquelle vous insérerez la plus petite boule ; dorez
à l'œuf ; donnez quelques coups de couteau dans la brioche pour empêcher
qu'elle éclate ; posez-la sur une plaque de tôle garnie d'un papier beurré et
faites cuire au four chaud. Évitez les coups de feu. La cuisson doit durer au
moins une heure. On reconnaît qu'elle est à point lorsque, après avoir enfoncé
dans le gâteau une paille ou une aiguille à brider, on l'en retire sèche.

\sk

Si vous faites de petites brioches, divisez la pâte en vingt boules, dont dix
quatre fois plus grosses que les autres ; finissez les petites brioches comme
il est dit pour la grosse brioche à tête.

\sk

\index{Brioche en couronne}
Pour faire la brioche en couronne, moulez la pâte en boule, appuyez au centre
avec les doigts pour former une cavité que vous élargirez progressivement, en
plusieurs fois, en laissant la pâte reposer un peu chaque fois. Lorsque la
couronne est bien formée, dorez-la à l'œuf, ciselez-la et mettez-la au four
moyennement chaud. On reconnait que la cuisson est à point par le procédé de la
paille.

\sk

\index{Brioche mousseline}
La brioche mousseline diffère essentiellement de la brioche ordinaire par le
degré plus élevé de fermentation de la pâte. On la fait avec tête ou sans tête.

Pour obtenir une brioche mousseline, prenez un moule à timbale ou à charlotte,
beurrez-le, garnissez-le d'un papier beurré dépassant le moule d'un tiers
environ de sa hauteur ; emplissez le moule à moitié avec la pâte bien rompue ;
laissez-la lever à froid jusqu'à ce qu'elle emplisse les deux tiers du moule.
Dorez le dessus à l'œuf, cisaillez le haut de la brioche et mettez-la au four
doux. La cuisson doit durer environ une heure et demie ; on reconnaît qu'elle
est à point par le procédé de la paille.

\sk

\index{Brioches diverses}
Il existe d'assez nombreuses variétés de brioches obtenues en incorporant aux
éléments de la pâte des raisins, du cédrat, de la crème fouettée, vanillée ou
non, de l'eau-de-vie, des pistaches, etc., etc.

\index{Fougaces}
Dans le Midi, on prépare des brioches sucrées contenant jusqu'à {\ppp200\mmm}
grammes de sucre pour {\ppp500\mmm} grammes de farine. Ces brioches sont
connues sous le nom de fougaces dans la région de Montpellier.

\sk

La brioche se prête à une infinité de combinaisons soit avec des fruits cuits
ou des fruits contits, des confitures ou des marmelades, soit avec des vins,
des liqueurs ou des sirops, qui donnent des entremets très agréables, faciles
à préparer.

\section*{\centering Baba à la polonaise.}
\phantomsection
\addcontentsline{toc}{section}{ Baba à la polonaise.}
\index{Baba à la polonaise}

Le baba à la polonaise, le père de tous les babas, est un excellent gâteau sec,
On peut facilement le conserver pendant toute une semaine et, comme sa
préparation est un peu longue, on le fait souvent de grandes dimensions. J'ai
entendu parler de babas monstres, dont la confection exigeait {\ppp120\mmm}
œufs. De semblables pièces, assez difficiles à réussir, nécessitent des fours
spéciaux ; aussi est il plus commode de faire à la fois plusieurs gâteaux de
dimensions moindres.

\medskip

Avant d'entrer dans les détails de l'opération, je tiens à faire quelques
recommandations générales.

Le four doit être très chaud pour que le baba monte et cuise rapidement. Une
façon pratique d'essayer la température du four consiste à y jeter une poignée
de farine. Si elle crépite et brûle, le four est trop chaud ; si elle rougit
seulement, la température est convenable.

Lorsque la levure est bonne, toute l'opération dure huit heures ; elle peut
durer douze heures si la pâte monte lentement.

La température la meilleure pour travailler la pâte et la faire monter est de
{\ppp25\mmm}° à {\ppp31\mmm}° C.

Tous les préparatifs doivent être faits la veille.

Les jaunes d'œufs seront séparés des blancs et les germes, qui alourdiraient le
gâteau, devront être éliminés,

Les jaunes seront battus dans une terrine plongée dans de l'eau chaude.

La légèreté du baba dépend du travail de la pâte qui, après l'addition de
chaque ingrédient, doit être travaillée pendant une heure.

Pour voir si la pâte est de bonne consistance, un moyen pratique consiste à en
prendre une poignée et à la serrer : si elle prend, c'est quelle est
suffisamment épaisse ; sinon, il faut ajouter de la farine et retravailler la
pâte juste assez pour que la farine y soit mélangée intimement.

Il importe de bien faire attention de ne pas heurter le moule en le mettant au
four, autrement le soufflé pourrait tomber. Il est bon également de prendre des
précautions en défournant.

Enfin, il est essentiel de laisser complètement refroidir le baba avant de le
démouler.

Ces précautions générales indiquées, j'arrive à la technique de l'opération.

\medskip

Pour faire quatre babas pouvant servir chacun pour une dizaine de personnes
prenez :

\footnotesize
\begin{longtable}{rrrp{16em}}
  2 250 & grammes & de & farine séchée et tamisée,                                                        \\
    800 & grammes & de & beurre frais,                                                                    \\
    600 & grammes & de & sucre,                                                                           \\
    250 & grammes & de & raisins de Smyrne épépinés.                                                      \\
    125 & grammes & de & cédrat haché,                                                                    \\
    125 & grammes & de & levure de bière,                                                                 \\
    100 & grammes & d' & amandes douces,                                                                  \\
     40 & grammes & de & sel,                                                                             \\
     25 & grammes & d' & amandes amères,                                                                  \\
        & 1 litre & de & lait,                                                                            \\
        &         & 40 & œufs frais,                                                                      \\
        &         &    & mie de pain rassis tamisée,                                                      \\
        &         &    & zeste râpé d'un citron.                                                          \\
\end{longtable}
\normalsize

Délayez la levure dans {\ppp200\mmm} grammes de lait, amalgamez-y la moitié de
la farine et la moitié du sel.

Faites tiédir le reste du lait, mettez dedans le sucre que vous laisserez
fondre, puis ajoutez-le au levain que vous venez de faire. Travaillez bien ;
couvrez la pâte et tenez-la au chaud pour qu'elle lève.

Faites fondre le beurre, écumez-le, filtrez-le afin d'avoir un beurre très clair.
Tenez-le au chaud.

Cassez les œufs, séparez les blancs des jaunes, enlevez les germes ; battez les
jaunes.

Triturez le reste de la farine avec le reste du sel et les jaunes d'œufs,
mélangez cette pâte à la pâte qui aura levé ; travaillez et battez pendant une
heure. Mettez ensuite le beurre, moins 40 grammes ; travaillez encore la pâte
pendant une demi-heure ; enfin incorporez-y les amandes, les raisins, le cédrat
et le zeste de citron.

Tenez la pâte au chaud pour qu'elle lève.

Graissez avec le reste du beurre quatre hauts moules cylindriques ;
enduisez-les de mie de pain rassis tamisée ; emplissez-les, à un peu moins de
la moitié de la hauteur, avec l'appareil ; tenez-les au chaud pour que la pâte
monte encore.

Lorsque les moules seront presque pleins, enfournez-les, avec précaution, sans
les heurter, dans un four chaud. Laissez cuire pendant une heure.

Enfoncez une paille dans les babas avant de les retirer du four ; si vous la
sortez sèche, les babas sont cuits ; sinon, il faut encore les laisser au four.

Défournez les babas, laissez-les refroidir, puis démoulez-les.

\section*{\centering Baba fourré glacé.}
\phantomsection
\addcontentsline{toc}{section}{ Baba fourré glacé.}
\index{Baba fourré glacé}

On peut évidemment fourrer et glacer un baba préparé d'après la formule
précédente ; cependant, il vaut mieux préparer un peu différemment le baba qui
doit être glacé.

Pour faire quatre babas, pouvant servir chacun pour une dizaine de personnes
prenez :

\medskip

1° pour le gâteau :

\footnotesize
\begin{longtable}{rrrrp{16em}}
  &   2 250 & grammes & de & farine séchée et tamisée,                                                    \\
  &   1 200 & grammes & de & beurre,                                                                      \\
  &     800 & grammes & de & sucre,                                                                       \\
  &     150 & grammes & de & levure de bière,                                                             \\
  &     100 & grammes & de & amandes douces,                                                              \\
  &      75 & grammes & de & rhum,                                                                        \\
  &      45 & grammes & de & sel,                                                                         \\
  & \multicolumn{2}{r}{1 litre 1/2} & de & lait,                                                          \\
  &         &         & 60 & œufs,                                                                        \\
  &         &         &    & vanille,                                                                     \\
  &         &         &    & zerte de citron râpé ;                                                       \\
\end{longtable}
\normalsize

2° pour le remplissage :

\footnotesize
\begin{longtable}{rrrrp{16em}}
  &   2 250 & grammes & de & farine séchée et tamisée,                                                    \kill
  &         &         &    & crème,                                                                       \\
  &         &         &    & sucre parfumé à la vanille,                                                  \\
  &         &         &    & ananas,                                                                      \\
  &         &         &    & confitures variées,                                                          \\
  &         &         &    & marasquin ;                                                                  \\
\end{longtable}
\normalsize

3° pour le glaçage :

\footnotesize
\begin{longtable}{rrrrp{16em}}
  &   2 250 & grammes & de & farine séchée et tamisée,                                                    \kill
  &     500 & grammes & de & chocolat,                                                                    \\
  &     500 & grammes & de & sucre parfumé à la vanille,                                                  \\
  &     500 & grammes & de & beurre fin,                                                                  \\
  &     150 & grammes & d' & eau ;                                                                        \\
\end{longtable}
\normalsize

4° pour la décoration :

\footnotesize
\begin{longtable}{rrrrp{16em}}
  &   2 250 & grammes & de & farine séchée et tamisée,                                                    \kill
  &         &         &    & gelées de fruits variés,                                                     \\
  &         &         &    & fruits confits.                                                              \\
\end{longtable}
\normalsize

Faites bouillir le lait avec un peu de sel et de la vanille ; retirez la
vanille quand le lait sera bien parfumé.

Préparez un levain en délayant la moitié de la farine avec un litre de lait
chaud ; travaillez bien afin d'éviter les grumeaux, puis ajoutez la levure
dissoute dans un peu de lait tiède, mélangez, couvrez et mettez dans un endroit
chaud pour que la pâte lève.

Séparez les jaunes d'œufs des blancs, enlevez soigneusement les germes, battez
les jaunes.

Faites fondre le beurre, clarifiez-le.

Malaxez le reste de la farine avec les jaunes d'œufs battus, mouillez avec le
reste du lait dans lequel vous aurez fait fondre le sucre, ajoutez le levain.
le reste du sel, les amandes coupées fin, du zeste de citron râpé, le rhum et
travaillez pendant une demi-heure au moins. Enfin, mêlez le beurre à la
préparation et continuez à travailler la pâte jusqu'à ce qu'elle soit légère.

Mettez-la dans un vase, au chaud, et, quand elle aura levé, emplissez-en
à moitié des moules beurrés.

Tenez toujours au chaud et quand la pâte aura bien monté, effectuez la cuisson
au four. Laissez refroidir les babas dans les moules ; démoulez-les ensuite.

Le lendemain, enlevez sur le sommet des babas une calotte de {\ppp2\mmm}
centimètres de hauteur ; réservez.

Videz\footnote{Cete opération se fait très commodément au moyen d'un petit
outil analogue à celui qu'emploient les sondeurs pour tirer les carottes
témoins des terrains qu'ils ont traversés.} l'intérieur des babas, puis
emplissez les vides avec de la crème battue, sucrée, parfumée à la vanille, et
dans laquelle vous aurez mis des morceaux d'ananas et différentes confitures,
un peu épaisses, aspergées de marasquin.

Couvrez les babas avec les calottes.

Mettez dans une casserole l'eau, le chocolat, le sucre et la vanille, laissez
cuire, puis ajoutez le beurre et mélangez sur feu doux jusqu'à obtention d'une
masse épaisse, mais suffisamment fluide. Glacez-en les gâteaux.

Décorez les babas avec des gelées de fruits et des fruits confits.

\section*{\centering Savarin.}
\phantomsection
\addcontentsline{toc}{section}{ Savarin.}
\index{Savarin}

Le savarin, dédié par Julien à la mémoire de Brillat-Savarin. est une variété de
baba.

\medskip

Pour douze personnes prenez :

\footnotesize
\begin{longtable}{rrrp{16em}}
    500 & grammes & de & farine tamisée,                                                                  \\
    350 & grammes & de & beurre,                                                                          \\
    100 & grammes & de & crème,                                                                           \\
     20 & grammes & de & sucre en poudre,                                                                 \\
     15 & grammes & de & levure,                                                                          \\
     10 & grammes & de & sel blanc,                                                                       \\
        &         &  8 & œufs.                                                                            \\
\end{longtable}
\normalsize

Préparez un levain avec {\ppp125\mmm} grammes de farine tiédie et la levure
dissoute dans la crème tiède, mettez-le dans un vase, couvrez avec le reste de
la farine et tenez dans un endroit à température douce. Laissez monter le
levain au double de son volume.

Fouettez les œufs au bain-marie de manière à en faire une mousse tiède.

Clarifiez le beurre.

Faites une pâte avec la farine, le levain, la moitié des œufs, la moitié du
beurre, le sucre et le sel ; travaillez-la pendant une dizaine de minutes, puis
incorporez‑y le reste des œufs et le reste du beurre. Travaillez encore de
façon à obtenir une pâte ayant la consistance d’une pommade.

Laissez-la monter un peu, puis garnissez-en, un peu plus qu'à moitié, un moule
à savarin beurré. Tenez au chaud, la pâte montera encore.

Lorsque le moule est presque plein, mettez-le au four ; {\ppp45\mmm} minutes de
cuisson suffisent en moyenne. Le criterium d’une cuisson parfaite consiste
à retirer lisse et brillante une aiguille à brider que l'on a enfoncée dans le
gâteau.

Sortez le savarin du four, démoulez-le et imbibez-le, avec un sirop parfumé,
par exemple, avec du sirop kirsch-anisette ou du sirop kirsch-rhum
à {\ppp30\mmm}°.

La finesse de l’entremets dépend beaucoup de la qualité des liqueurs employées
pour aromatiser le sirop. Avec du vieux rhum, pur tafia vieilli en fût, c'est
parfait.

Il est bon d'accompagner le savarin d’un vin liquoreux, tel qu'un vieux vin de
Frontignan.

\sk

Comme variantes, on peut introduire dans la pâte du safran, de la cannelle, des
amandes, de l'angélique, des raisins, de l'orange, du cédrat ou d'autres fruits
confits.

\sk

Le gorenflot est une sorte de savarin qu'on fait dans un moule hexagonal et
qu'on arrose, lorsqu'il est cuit, avec du sirop à {\ppp28\mmm}° parfumé avec de
l'anisette et de l'orgeat, relevé par un peu d'absinthe.

\sk

On trempe aussi parfois les gorenflots avec un sirop fait au lait
d'amandes\footnote{Pour faire du lait d'amandes, prenez :                                                 \\
\begin{tabular}{rrrl}
\hspace{12em} 500 & grammes & d' & eau chaude,                                                            \\
\hspace{12em} 125 & grammes & d' & amandes douces,                                                        \\
\hspace{12em}  80 & grammes & de & sucre ou de miel,                                                      \\
\hspace{12em}   8 & grammes & d' & eau de fleurs d'oranger.                                               \\
\end{tabular}                                                                                             \\
\protect\endgraf
Pilez les amandes avec un peu d'eau, passez-les au travers d'un linge, ajoutez
le reste de l'eau et le sucre ou le miel ; faites bouillir, réduisez à moitié,
laissez refroidir, puis parfumez avec l'eau de fleurs d'oranger.}.

\section*{\centering Pain perdu.}
\phantomsection
\addcontentsline{toc}{section}{ Pain perdu.}
\index{Pain perdu}

On prépare le pain perdu de plusieurs manières. À la campagne, on emploie à
cet usage du gros pain rassis.

Voici deux façons recommandables de l'apprêter.

\medskip

1° Coupez du pain rassis en tranches, trempez-les dans de l'œuf battu,
légèrement salé et aromatisé, au goût, avec de l’eau de fleurs d'oranger, de la
vanille, du rhum, du kirsch ou du zeste de citron râpé, puis faites-les frire
dans du beurre ou même dans de la graisse ; servez-les chaudes, saupoudrées de
sucre semoule.

\medskip

2° Coupez, comme précédemment, du pain rassis en tranches. Mettez dans une
casserole du lait légèrement salé et aromatisé avec de l'eau de fleurs
d'oranger, de la vanille, du rhum, du kirsch ou du zeste de citron râpé, faites
bouillir, puis laissez refroidir un peu. Trempez les tranches de pain dans ce
lait parfumé, laissez-les s'en imprégner le plus possible sans trop les
ramollir, passez-les ensuite dans de l'œuf battu et faites-les frire dans du
beurre. Saupoudrez-les de sucre semoule au sortir de la friture et servez-les
avec un bon jus de fruits que vous obtiendrez facilement en toute saison en
faisant fondre au bain-marie de la gelée de confitures.

\sk

On peut apprêter en pain perdu des brioches et des croissants rassis.

\section*{\centering Pain perdu meringué.}
\phantomsection
\addcontentsline{toc}{section}{ Pain perdu meringué.}
\index{Pain perdu meringué}

Prenez des tranches de pain rassis sans croûte ou, de préférence, des brioches
et des croissants rassis, dont vous aurez râpé l'extérieur, coupez-les par le milieu
dans le sens de l'épaisseur, creusez-en l'intérieur et fourrez-le avec de la crème
au chocolat.

Reconstituez les tranches de pain, les brioches ou les croissants, mettez-les
dans un plat foncé de beurre et allant au feu, versez dessus une crème à la
vanille. Au dernier moment, masquez-les avec un appareil à meringue, poussez
au four pour solidifier la meringue, puis servez.

\section*{\centering Plum-cake.}
\phantomsection
\addcontentsline{toc}{section}{ Plum-cake.}
\index{Plum-cake}

Pour six personnes prenez :

\footnotesize
\begin{longtable}{rrrp{16em}}
    250 & grammes & de & farine tamisée,                                                                  \\
    250 & grammes & de & sucre en poudre,                                                                 \\
    250 & grammes & de & beurre fin,                                                                      \\
    190 & grammes & de & rhum,                                                                            \\
    125 & grammes & de & raisins de Corinthe,                                                             \\
    100 & grammes & de & raisins de Malaga,                                                               \\
     90 & grammes & de & fruits confits hachés,                                                           \\
      6 & grammes & de & sel,                                                                             \\
        &         &  8 & œufs frais,                                                                      \\
        &         &    & zeste de citron,                                                                 \\
        &         &    & sirop au rhum,                                                                   \\
        &         &    & vanille.                                                                         \\
\end{longtable}
\normalsize

Épluchez, lavez les raisins, séchez-les dans un linge. Épépinez-les et
faites-les macérer dans le rhum vanillé ou non, au goût.

Mettez le beurre dans une bassine tenue au bain-marie ; lorsqu'il est devenu
crémeux, sortez la bassine du bain-marie, ajoutez le sucre, en remuant, puis
les œufs, un à un, en tournant sans cesse, et enfin la farine. Mélangez et
travaillez bien la pâte en la soulevant pour l'aérer.

Incorporez ensuite le rhum, les raisins, les fruits confits et du zeste de
citron râpé.

Prenez un moule à charlotte en fer blanc, beurrez-le légèrement, chemisez-le de
papier blanc dépassant le moule de {\ppp3\mmm} centimètres. Emplissez-le avec
la préparation, dont vous parsèmerez le dessus avec quelques raisins de Corinthe
que vous aurez réservés, et faites cuire au four moyen un peu ouvert.

Glacez le plum-cake, au sortir du four, avec du sirop au rhum vanillé ou non.
Dentelez le papier.

\sk

On obtiendra un plum-cake plus léger en faisant la préparation avec les blancs
d'œufs montés en neige,

\section*{\centering Plum-pudding.}
\phantomsection
\addcontentsline{toc}{section}{ Plum-pudding.}
\index{Plum-pudding}

Le plum-pudding est l'entremets national anglais.

On y fait entrer des fruits variés, au choix, et on l'aromatise de différentes
façons : au rhum, au cognac, au madère, etc.

\medskip

Voici une formule donnant les proportions des éléments d'un plum-pudding pour
{\ppp18\mmm} à {\ppp24\mmm} personnes :

\footnotesize
\begin{longtable}{rrrrp{16em}}
  &     500 & grammes & de & graisse de rognon de bœuf épluchée et hachée fin,                            \\
  &     250 & grammes & de & farine,                                                                      \\
  &     250 & grammes & de & mie de pain tamisée,                                                         \\
  &     200 & grammes & de & raisins de Malaga épépinés,                                                  \\
  &     200 & grammes & de & raisins de Corinthe épluchés et lavés,                                       \\
  &     200 & grammes & de & cassonade blonde,                                                            \\
  &     100 & grammes & de & rhum,                                                                        \\
  &     100 & grammes & de & lait,                                                                        \\
  &      75 & grammes & d' & écorce d'orange confite, coupée en petits dés,                               \\
  &      75 & grammes & de & cédrat confit, coupé en petits dés,                                          \\
  &      75 & grammes & d' & amandes hachées,                                                             \\
  &      75 & grammes & de & cerises sèches sans noyaux,                                                  \\
  &      10 & grammes & de & sel,                                                                         \\
  &       3 & grammes & de & cannelle de Ceylan en poudre,                                                \\
  &       3 & grammes & de & muscade en poudre,                                                           \\
  &       2 & grammes & de & gingembre pulvérisé,                                                         \\
  & \multicolumn{2}{r}{6 décigrammes} & de & girofle en poudre,                                           \\
  &         &         &  6 & œufs frais.                                                                  \\
  &         &         &  1 & pomme reinette épluchée et coupée en petits dés,                             \\
  &         &         &    & zeste d'un citron haché très fin,                                            \\
  &         &         &    & beurre.                                                                      \\
\end{longtable}
\normalsize

Mettez dans une terrine : graisse, farine, mie de pain, cassonade, écorce
d'orange, cédrat, pomme, amandes, raisins de Malaga et de Corinthe, cerises
sèches, zeste de citron, lait, rhum, œufs, sel, cannelle, muscade, gingembre,
girofle ; mélangez bien.

Beurrez un moule à pudding, emplissez-le avec l'appareil, mettez le couvercle,
attachez-le.

À défaut de moule, mettez l'appareil dans une serviette beurrée et farinée, que
vous lierez avec une ficelle.

Emplissez d'eau aux deux tiers une marmite au fond de laquelle vous aurez mis
un plat ; chauffez ; lorsque l’eau bout, mettez le plum-pudding dans la marmite
sur le plat et continuez à faire bouillir pendant cinq heures en remplaçant, au
fur et à mesure, l'eau qui s'évapore par de l’eau bouillante.

Démoulez le pudding sur un plat chaud et servez-le soit avec une sauce
sabaillon, soit simplement saupoudré de sucre et arrosé avec {\ppp200\mmm}
grammes de vieux rhum que vous allumerez au dernier moment.

\section*{\centering Plum-pudding.}
\phantomsection
\addcontentsline{toc}{section}{ Plum-pudding.}
\index{Plum-pudding}

\begin{center}
\textit{(Autre formule).}
\end{center}

Prenez :

\footnotesize
\begin{longtable}{rrrrp{16em}}
  &     500 & grammes & de & graisse de rognon de bœuf épluchée et hachée fin,                            \kill
  &         &         &    & prunes Reine-Claude,                                                         \\
  &         &         &    & raisins de Smyrne,                                                           \\
  &         &         &    & mie de pain rassis tamisée,                                                  \\
  &         &         &    & saindoux,                                                                    \\
  &         &         &    & madère,                                                                      \\
  &         &         &    & pale-ale,                                                                    \\
  &         &         &    & porter,                                                                      \\
  &         &         &    & coriandre,                                                                   \\
  &         &         &    & cannelle de Ceylan ;                                                         \\
\end{longtable}
\normalsize

le tout en proportions variables suivant le goût.

Enlevez les noyaux des prunes, épépinez les raisins.

Mélangez ensemble tous les éléments de la formule, moins le madère, de façon
à obtenir une pâte consistante ; enveloppez-la dans un linge fariné et
faites-la cuire pendant six heures dans du vin de Madère.

Lorsque le vin est absorbé, sortez le pudding du linge, mettez-le sur un plat
et servez-le tel quel ou avec une sauce sabaillon, une crème à la vanille, ou
encore une sauce au jus de citron obtenue en faisant cuire dans un sirop des
citrons pelés vifs, avec un peu de zeste blanchi pour donner plus de parfum.

\section*{\centering Pain d'épice.}
\phantomsection
\addcontentsline{toc}{section}{ Pain d'épice.}
\index{Pain d'épice}

Le pain d'épice était connu dès la plus haute antiquité ; son invention a dû
probablement suivre de près celle du pain.

De l'Orient, où il était en faveur, il passa en Europe. Il figurait chez les Grecs
au dessert, et les Romains en faisaient des offrandes aux dieux.

Au moyen âge, il était servi, en France, aux dîners de la Cour, et le plus renommé
état celui de Reims.

Sous Louis XIV, on faisait entrer dans sa composition de la farine de seigle,
du miel, de la mélasse, de l'anis, du girofle et de la cannelle.

Les formules actuelles de pain d'épice sont nombreuses. En voici une qui m'a
donné un très bon résultat.

\footnotesize
\begin{longtable}{rrrp{16em}}
    500 & grammes & de & farine,                                                                          \\
    300 & grammes & d' & eau,                                                                             \\
    250 & grammes & de & miel,                                                                            \\
    250 & grammes & de & sucre,                                                                           \\
     50 & grammes & de & rhum,                                                                            \\
     12 & grammes & de & bicarbonate de soude,                                                            \\
      5 & grammes & d' & anis vert en poudre,                                                             \\
      2 & grammes & de & cannelle de Ceylan en poudre,                                                    \\
      1 & gramme  & de & sel,                                                                             \\
        &         &    & beurre.                                                                          \\
\end{longtable}
\normalsize

Faites dissoudre dans l'eau bouillante le sucre, le miel, le bicarbonate de
soude et le sel, ajoutez le rhum, l'anis et la cannelle.

Délayez la farine avec ce mélange, de façon à obtenir une pâte homogène sans
grumeaux.

Beurrez copieusement un moule, mettez dedans la pâte, poussez d'abord au four
très chaud, puis continuez la cuisson à feu modéré. Une heure de cuisson suffit.

On servira le pain d'épice avec du beurre frais.

\sk

Si l'on veut obtenir un pain d'épice moins ferme, il suffit de diminuer, dans
la formule, la proportion de la farine et d'augmenter celle de l’eau.

\sk

Comme variantes, on peut introduire dans la composition du pain d'épice des
amandes, des pistaches grillées ou non, de la coriandre, de la badiane, du
girofle, du gingembre, des fruits confits, de l'écorce confite d'orange ou de
citron, de l’angélique confite, des raisins secs, des pruneaux, privés de leurs
noyaux, de l'angélique fraîche, du cédrat mariné dans du rhum, du kirsch, ou du
marasquin ; des confitures, des gelées de fruits, des liqueurs. etc.

\sk
