\section*{\centering Omelette.}
\addcontentsline{toc}{section}{ Omelette.}
\index{Omelette}

Il peut paraître banal de donner une recette d'omelette, car tout le monde croit
savoir la faire. Cependant, en réalité, il ne manque pas de gens qui n'ont jamais
mangé une omelette vraiment bonne.

Sans doute, la préparation n'en est pas très compliquée, mais il est absolument
nécessaire de prendre certaines précautions qu'on néglige souvent. Les
praticiens qui la réussissent à coup sûr opèrent d'une certaine façon plus ou
moins différente mais toujours la même et ils ont consciemment ou
inconsciemment des points de repère qui les guident.

Je me propose simplement de préciser les points délicats, de façon à permettre
à chacun de faire sûrement une omelette savoureuse. Je ne m'attarderai pas
à décrire une par une la préparation des différentes variétés d'omelettes qui
sont innombrables ; je m'attacherai uniquement à la confection de l'omelette la
plus simple, l'omelette au naturel. Celui qui sait la faire sait les faire
toutes, sauf peut-être l'omelette soufflée, qui diffère du reste complètement
de l'omelette proprement dite, et dont je dirai un mot en parlant des entremets
sucrés.

Il convient d'abord de ne pas oublier que dans l'omelette, aussi bien que dans
toutes les préparations à base d'œufs, le produit se ressent considérablement
de la qualité des œufs ; aussi, faut-il n'employer que des œufs absolument
frais et, parmi eux, je donne la préférence aux œufs à coquille plus ou moins
colorée, ceux dont la coquille est d'une blancheur étincelante étant, à mon
avis, plus fades. Il est presque inutile de dire que l'on doit employer du
beurre de premier choix, que la poêle servant à la confection de l’omelette
doit être d'une propreté irréprochable et de dimensions en rapport avec le
nombre des œufs, l'omelette ne devant jamais être trop épaisse.

Il est incontestablement difficile de bien préparer une omelette de plus de
douze œufs ; il est donc préférable de s'en tenir à des omelettes plus petites.

Pour préciser, je vais décrire la préparation d'une omelette de quatre œufs dont
chacun pèserait {\ppp70\mmm} grammes en moyenne, soit {\ppp280\mmm} grammes ensemble.

Dans ces conditions, il convient d'employer une poêle ayant un fond de
{\ppp15\mmm} à {\ppp17\mmm} centimètres de diamètre.

\medskip

Pour deux personnes prenez :

\medskip

\setlength\tabcolsep{.2em}
\footnotesize
\begin{longtable}{rrrrp{16em}}
    & 50  & grammes & de & beurre,                                                                        \\
    & 30  & grammes & de & lait,                                                                          \\
    & \multicolumn{2}{r}{4 à 5 grammes} & de & sel,                                                       \\
    & 1/2 & gramme  & de & poivre,                                                                        \\
    &     &         &  4 & œufs pesant ensemble 280 grammes environ.                                      \\
\end{longtable}
\normalsize

Cassez les œufs dans un bol, ajoutez le lait, le sel et le poivre ; battez le
tout pendant une minute environ. Faites fondre le beurre dans la poêle sur un
feu vif ; lorsqu'il sera à la température voulue, ce qui demande environ deux
minutes, se reconnaît pratiquement à ce que la mousse qui surnage le beurre
fondu a presque disparu et correspond à une coloration noisette, versez dedans
les œufs. Au bout d'une demi-minute environ, la partie inférieure de l'omelette
sera prise. Soulevez-la alors rapidement sur tout le pourtour avec une
fourchette, inclinez la poêle successivement dans tous les sens en faisant
couler sous la couche prise un peu du liquide qui surnage, laissez prendre
encore, répétez l'opération une deuxième fois et achevez la cuisson de
l’ensemble. La cuisson complète dure à peu près deux minutes. L'omelette est
à point lorsqu'on voit se dégager de la fumée sur le pourtour de la poêle. A ce
moment précis, pliez l'omelette, faites-la glisser sur un plat chaud et servez.

En suivant pas à pas toutes ces indications, vous aurez une omelette joliment
dorée à l'extérieur, juteuse à l'intérieur, moelleuse à souhait.

Le rôle du lait, que l'on pourrait du reste remplacer par de l'eau dans le
mélange, est double : il sert à répartir convenablement l’assaisonnement et il
empêche une prise trop rapide des œufs.

\sk

Les omelettes dans lesquelles on fait entrer des substances chaudes, bouillies,
frites ou grillées, telles que les omelettes au lard, aux champignons, aux
harengs, etc., sont cuites d'une façon identique. Les éléments qu'on veut
y incorporer sont ajoutés seulement au moment de verser les œufs dans la poêle.

Les œufs destinés à la confection d'omelettes aromatisées par des substances
crues ou cuites et refroidies, telles que les omelettes aux fines 
herbes\footnote{Pour faire une omelette aux fines herbes pour deux personnes, prenez :
    \setlength\tabcolsep{.2em}
    \begin{tabular}{r r r l}
                  65 & grammes  & de & bon beurre frais,                                                 \\
                  45 & grammes  & d' & eau,                                                              \\
               4 à 5 & grammes  & de & sel,                                                              \\
                   3 & grammes  & de & persil haché (les feuilles, sans côtes ni tiges),                 \\
               2 à 3 & grammes  & de & cerfeuil haché (les feuilles, sans côtes ni tiges),               \\
               2 à 3 & grammes  & de & civette hachée (les parties les plus tendres),                    \\
                   2 & grammes  & d' & estragon haché (les feuilles seulement),                          \\
                 1/2 & gramme   & de & poivre fraîchement moulu,                                         \\
                     &          &  4 & œufs frais, pesant ensemble environ 280 grammes,                  \\
    \end{tabular}
    \medskip
    \protect\endgraf
    Cassez les œufs, ajoutez-y les fines herbes, le sel, le poivre, l'eau,
    battez vigoureusement, laissez en contact pendant une demi-heure environ,
    Battez encore un peu les œufs au dernier moment et faites l'omelette comme
    précédemment. },
    
aux hachis de viandes, aux queues de crevettes ou d'écrevisses, aux morilles,
aux truffes, etc., doivent être battues un quart d'heure avant la cuisson et
les substances aromatisantes doivent rester en contact avec les œufs pendant ce
temps pour les bien pénétrer de leur parfum ; on battra de nouveau l'ensemble
au fouet ou à la fourchette au moment de commencer la cuisson.

\section*{\centering Œufs brouillés.}
\addcontentsline{toc}{section}{ Œufs brouillés.}
\index{Œufs brouillés}

Les œufs brouillés peuvent, comme l'omelette, être préparés soit au naturel,
soit accompagnés de divers ingrédients. Voici une formule d'œufs brouillés au
naturel.

Pour cinq personnes prenez :

\medskip

\footnotesize
\begin{longtable}{rrrp{16em}}
  125  & grammes & de & beurre,                                                                          \\
  125  & grammes & de & crème,                                                                           \\
       &         & 10 & œufs frais,                                                                      \\
       &         &    & sel, poivre.                                                                     \\
\end{longtable}
\normalsize

Cassez les œufs, passez-les ensemble au travers d’une mousseline pour éliminer
les germes et les membranes qui dépareraient le plat, mais ne les battez pas.
Prenez une petite casserole aussi haute que large et de dimensions telles que
ce qui cuira dedans atteigne la hauteur de 5 à 6 centimètres, beurrez-la,
mettez dedans les œufs et faites cuire à petit feu ou au bain-marie, en
tournant constamment. Dès que le mélange commence à prendre, ajoutez la crème,
le reste du beurre coupé en petits morceaux, salez et poivrez au goût. Achevez
la cuisson dans les mêmes conditions et servez vivement sur assiettes chaudes.

\sk

Il existe un très grand nombre pe variétés d'œufs brouillés, la quantité des
ingrédients qui peuvent entrer dans leur composition étant considérable. Je n'en
mentionnerai que deux : les œufs brouillés aux crevettes et les œufs brouillés au
boudin.

Pour les œufs brouillés aux crevettes, les proportions des éléments sont les
suivantes.

\medskip

Pour cinq personnes prenez :

\medskip

\footnotesize
\begin{longtable}{rrrp{16em}}
  200  & grammes & de & crevettes vivantes\footnote{D'une façon générale, le liquide
                                               qui convient le mieux pour la cuisson
                                               des crustacés, des mollusques et des
                                               poissons de mer, venant d'être pêchés
                                               et devant être consommés au naturel,
                                               est l'eau de mer.},                                        \\
  125  & grammes & de & beurre,                                                                           \\
  125  & grammes & de & crème,                                                                            \\
       &         &  8 & œufs frais,                                                                       \\
       &         &    & sel et poivre.                                                                    \\
\end{longtable}
\normalsize

\index{Cuisson des crustacés, mollusques et poissons}
\index{Cuisson des fruits de mer au naturel}
\index{Cuisson des poissons de mer au naturel}
Court-bouillonnez les crevettes dans de l'eau de mer ou à défaut, dans de l'eau
salée ; quand elles seront cuites, épluchez-les, mettez de côté les queues,
écrasez au mortier et passez au tamis les parures : vous obtiendrez ainsi un
jus relevé.

On conduit la cuisson tout à fait comme précédemment ; on ajoute aux œufs
les queues de crevettes et le jus des parures au moment où l'on met la crème.

Les œufs brouillés aux crevettes ainsi préparés constituent un plat réellement
fin, qui plaît à tout le monde.

\sk

Comme originalité, je citerai les œufs brouillés au boudin qu'on prépare de
même, en remplaçant simplement les crevettes par l'intérieur d'un boudin
fraîchement grillé.

\section*{\centering Œufs en cocote.}
\addcontentsline{toc}{section}{ Œufs en cocote.}
\index{Œufs en cocote}

La cuisson des œufs dans de petites cocotes en porcelaine présente la plus
grande analogie avec celle des œufs à la coque et elle offre l'avantage de se
prêter à de nombreuses combinaisons culinaires.

Pour préparer l'œuf en cocote au naturel, digestible par excellence, véritable
plat pour dyspeptiques, on commence par mettre dans les cocotes, chauffées au
préalable au bain-marie, une cuillerée d’eau ou de bouillon, puis on casse un
œuf dans chaque cocote et on fait la cuisson au bain-marie, pendant quatre
à six minutes, suivant qu'on aime les œufs plus ou moins cuits. On sert
immédiatement et chaque convive assaisonne ses œufs, suivant son goût, au
moment de les manger.

On peut également apprêter des œufs en cocote en remplaçant l'eau ou le
bouillon par du beurre ou par de la crème : on assaisonne alors pendant la
cuisson.

\sk

Une bonne manière de préparer des œufs en cocote à la crème consiste à faire
cuire des œufs comme précédemment, mais pendant deux minutes seulement, dans
des cocotes enduites de beurre ou de crème, puis à ajouter dans chaque cocote
15 à 20 grammes de crème assaisonnée au goût, et à achever la cuisson en
mettant le bain-marie au four.

\sk

On préparera de même des œufs au jus de viande, à la pulpe de viande, au fond
de volaille, au fond de veau ordinaire, au fond de veau à l’estragon, à la
tomate, etc.

\sk

On pourra aussi foncer des cocotes de différentes compositions telles que :
farce de quenelles de volaille, purée de foie gras, fond de veau tomaté ou non,
béchamel, purée de jambon au gras, purée de gibier à la crème, etc. et masquer
les œufs avec une julienne de légumes, un salpicon aux morilles ou aux truffes,
de la purée Soubise, de la purée d'oseille, d'épinards ou de tomates, des
pointes d’asperges à la crème, des queues de crevettes, etc., en mariant
convenablement les deux substances qui enveloppent l'œuf : d'où un très grand
nombre de combinaisons, dont il suffit d'indiquer le principe.

\section*{\centering Œufs pochés en aspic, au gras.}
\addcontentsline{toc}{section}{ Œufs pochés en aspic, au gras.}
\index{Œufs pochés en aspic, au gras}
\index{Aspic d'œufs pochés, au gras}

Faites bouillir de l'eau dans laquelle vous aurez mis, par litre, {\ppp15\mmm}
grammes de sel gris et le jus d'un citron ; cassez les œufs un à un dans l’eau
bouillante en les faisant tomber successivement dans le liquide à l'endroit où
il bouillonne, éloignez la casserole du feu et laissez-les pocher pendant trois
minutes. Retirez-les ensuite avec une écumoire, passez-les dans de l’eau
fraîche pour leur enlever tout goût acidulé et parez-les.

Prenez des moules à dariole, versez dans chacun un peu de gelée de volaille
fondue ou, mieux encore, de la gelée de veau, volaille et gibier, comme celle
dont la préparation est indiquée \hyperlink{p0418}{p. \pageref{pg0418}},
laissez prendre, mettez dessus un œuf poché, saupoudrez d'un hachis de jambon
et de truffes, par exemple, recouvrez de gelée, faites prendre sur glace, puis
démoulez et servez sur des feuilles de cœur de laitue.

C'est très frais.

\section*{\centering Œufs poches garnis en aspic, au maigre.}
\addcontentsline{toc}{section}{ Œufs poches garnis en aspic, au maigre.}
\index{Œufs pochés garnis en aspic, au maigre}
\index{Aspic d'œufs pochés garnis, au maigre}

Préparez une gelée d'aspic maigre.

On prépare une gelée rigoureusement maigre en faisant cuire exclusivement, avec
des légumes, dans de l'eau salée additionnée ou non de vin, des poissons ou des
déchets de poissons en quantité suffisante pour obtenir, par la concentration
du liquide, un produit de consistance convenable.

Mais, lorsque le poisson est relativement rare, on se contente souvent d'une
gelée mi-grasse, mi-maigre obtenue avec un mélange de poissons et de pied ou
de jarret de veau.

Dans les deux cas, la cuisson doit être clarifiée au blanc d'œuf et passée à la
serviette.

Faites pocher des œufs, comme dans la formule précédente.

Foncez des moules à dariole avec de la gelée, laissez-la prendre ; mettez dessus
un œuf poché, des huîtres blanchies dans leur eau, des moules cuites au
naturel, des queues d'écrevisses ou de crevettes court-bouillonnées, du corail
d'oursins, du caviar, des escalopes de filets de poissons court-bouillonnés,
des pointes d'asperges à la béchamel, des tranches de truffes cuites dans du
madère, etc., au choix. Couvrez avec de la gelée. Tenez sur glace.

Démoulez et servez avec une salade de légumes.

\section*{\centering Œufs pochés gratinés sur canapés.}
\addcontentsline{toc}{section}{ Œufs pochés gratinés sur canapés.}
\index{Œufs pochés gratinés sur canapés}
\index{Canapés d'œufs pochés, au gras}
\index{Canapés d'œufs pochés, au maigre}

On peut préparer ce plat au maigre ou au gras.

\medskip

\textit{Au maigre}. — Pour quatre personnes prenez :

\medskip

\footnotesize
\begin{longtable}{rrrrp{16em}}   
  &   125 & grammes & de & parmesan râpé,                                                                 \\
  &   125 & grammes & de & beurre,                                                                        \\
  &   100 & grammes & de & crème,                                                                         \\
  &    60 & grammes & de & farine,                                                                        \\
  &    60 & grammes & de & purée de queues de crevettes ou d'écrevisses,                                  \\
  &    25 & grammes & de & gruyère râpé,                                                                  \\
  &    15 & grammes & de & chapelure,                                                                     \\
  &     5 & grammes & de & sel,                                                                           \\
  & \multicolumn{2}{r}{2 décigrammes} & de & poivre fraîchement moulu,                                    \\
  &       & 1 litre & de & lait,                                                                          \\
  &       &         &  8 & œufs frais,                                                                    \\
  &       &         &  1 & oignon moyen,                                                                  \\
  &       &         &  1 & carotte moyenne,                                                               \\
  &       &         &  1 & racine de céleri,                                                              \\
  &       &         &  1 & bouquet garni,                                                                 \\
  &       &         &    & pain anglais,                                                                  \\
  &       &         &    & muscade,                                                                       \\
  &       &         &    & sel et poivre.                                                                 \\
\end{longtable}
\normalsize

\label{pg0269} \hypertarget{p0269}{}
Préparez une béchamel maigre, c'est-à-dire faites dorer légèrement, dans
75 grammes de beurre, l'oignon, la carotte et le céleri émincés, mettez ensuite
la farine, tournez pendant cimq minutes sans laisser roussir, mouillez avec le
lait, ajoutez le bouquet, du sel et du poivre, donnez un bouillon, puis laissez
simplement mijoter jusqu'à cuisson complète. Passez alors la sauce, mélangez-y
la crème et amenez le tout à la consistance voulue pour masquer une cuiller.
Incorporez à cette béchamel le parmesan râpé ; vous aurez ainsi une mornay.

Préparez des canapés carrés de {\ppp6\mmm} centimètres de côté : soit huit petites tranches
minces de pain anglais, dorées dans {\ppp25\mmm} grammes de beurre et garnies de purée
de queues de crevettes ou d'écrevisses ; soit huit petits sandwichs de mêmes
dimensions, garnis des mêmes purées.

Faites pocher légèrement les œufs.

Beurrez un plat avec le reste du beurre, foncez-le d'une couche de mornay,
disposez sur la sauce les canapés préparés, mettez sur chaque canapé un œuf
poché peu cuit, assaisonnez avec sel, poivre et muscade, recouvrez de mornay,
saupoudrez de chapelure et de gruyère râpé mélangés, poussez au four pour
gratiner, puis servez.

\medskip

\textit{Au gras}. — Remplacez dans la formule précédente la béchamel maigre par
de la béchamel grasse, \hyperlink{p0566}{p. \pageref{pg0566}}, et la purée de
crevettes ou d'écrevisses par de la purée de jambon ou de la purée de foie gras
truffé.

\section*{\centering Œufs sur le plat en surprise.}
\addcontentsline{toc}{section}{ Œufs sur le plat en surprise.}
\index{Œufs pochés sur le plat en surprise}

Pour six personnes prenez :

\medskip

\footnotesize
\begin{longtable}{rrrp{16em}}
  125  & grammes & de & champignons de couche,                                                            \\
   50  & grammes & de & truffes,                                                                          \\
   50  & grammes & de & beurre,                                                                           \\
   25  & grammes & de & glace de viande,                                                                  \\
       &         &  6 & œufs,                                                                             \\
       &         &  2 & foies de poulardes,                                                               \\
       &         &    & bouillon,                                                                         \\
       &         &    & chapelure,                                                                        \\
       &         &    & jus de citron,                                                                    \\
       &         &    & persil,                                                                           \\
       &         &    & sel et poivre.                                                                    \\
\end{longtable}
\normalsize


Pelez les truffes et les champignons ; passez ces derniers au jus de citron.

Emincez les foies, les truffes et les champignons ; faites-les sauter à la
poêle dans le beurre clair.

Versez cette préparation dans un plat allant au feu, cassez dessus les œufs,
ajoutez la glace de viande dissoute dans du bouillon, salez, poivrez, saupoudrez
d'un peu de chapelure et de persil haché, mettez pendant quelques minutes au
four et servez.

\section*{\centering Œufs sur le plat à la crème.}
\addcontentsline{toc}{section}{ Œufs sur le plat à la crème.}
\index{Œufs pochés sur le plat à la crème}

La crème se marie très bien avec les œufs. Jai déjà signalé son usage dans les
œufs brouillés, les œufs en cocote, les œufs gratinés. Je ne passerai pas en
revue toutes les préparations dans lesquelles elle entre ; je me bornerai
à dire simplement quelques mots sur son emploi dans les œufs sur le plat.

Pour préparer des œufs sur le plat à la crème, faites cuire à la façon
ordinaire des œufs dans un plat en porcelaine allant au feu, salez, poivrez au
goût, saupoudrez légèrement d'un peu de civette hachée, noyez le tout dans de
la crème légèrement acidulée par un filet de vinaigre, couvrez, chauffez à feu
doux pendant une dizaine de minutes et servez dans le plat même.

\section*{\centering Œufs sur le plat gratinés.}
\addcontentsline{toc}{section}{ Œufs sur le plat gratinés.}
\index{Œufs pochés sur le plat gratinés}

Foncez de beurre un plat allant au feu, saupoudrez largement de parmesan,
ajoutez de la crème aigre délayée dans du fond de veau, mettez au four. Cinq
minutes avant de servir, cassez dans le plat des œufs en les laissant entiers,
salez, poivrez, recouvrez de parmesan et remettez au four simplement le temps
nécessaire pour que les blancs d'œufs soient bien pris.

\section*{\centering Œufs durs à la tripe.}
\addcontentsline{toc}{section}{ Œufs durs à la tripe.}
\index{Œufs durs à la tripe}

Commencez par faire cuire des œufs durs et pour cela faites bouillir de l’eau
dans une casserole, mettez-y les œufs placés au préalable dans un panier en
fils de fer, pour la commodité de la manœuvre, laissez bouillir pendant dix
minutes ; enlevez alors le panier de l’eau chaude et plongez-le dans de l'eau
courante froide.

Lorsque les œufs seront bien refroidis, sortez-les des coquilles, coupez-les en
deux, enlevez les jaunes et émincez les blancs en languettes.

Préparez une sauce Béchamel, \hyperlink{p0269}{p. \pageref{pg0269}}, en forçant
la proportion des oignons\footnote{Les amateurs d'oignon arrivent à mettre dans
la sauce, passée ou non, un volume d'oignons qui peut atteindre celui des
œufs.}, mettez dedans les demi jaunes et les blancs émincés, décorez avec du
persil haché et servez.

\sk

Comme variantes, on peut passer les jaunes au tamis et les incorporer à la
sauce, ou bien, laissant la sauce Béchamel telle quelle, s'en servir pour
décorer le plat.

On peut remplacer la béchamel par une mornay\footnote{La sauce Mornay est une
béchamel au fromage.}, saupoudrer le plat d'un mélange de mie de pain rassis
tamisé et de parmesan râpé et faire gratiner au four. On aura ainsi des œufs
à la tripe gratinés.

\section*{\centering Œufs durs gratinés.}
\addcontentsline{toc}{section}{ Œufs durs gratinés.}
\index{Œufs durs gratinés}

Pour six personnes prenez :

\medskip

\footnotesize
\begin{longtable}{rrrp{16em}}
  125 & grammes & de & sauce Béchamel grasse, préparée comme il est dit 
                       \hyperlink{p0566}{p. \pageref{pg0566}},                                            \\
   60 & grammes & de & beurre,                                                                            \\
   60 & grammes & de & parmesan râpé,                                                                     \\
      &         &  8 & œufs.                                                                              \\
\end{longtable}
\normalsize

Faites durcir les œufs, laissez-les refroidir, puis coupez-les en deux ;
retirez les jaunes ; passez-les, à l'aide d’un pilon, au travers d’un tamis ;
émincez les blancs.

Beurrez un plat allant au feu, disposez au fond les blancs d'œufs émincés,
masquez-les avec la béchamel, couvrez avec les jaunes d'œufs tamisés,
saupoudrez de fromage, mettez par-dessus le reste du beurre coupé en petits
morceaux ; faites gratiner au four pendant dix minutes et servez.

\section*{\centering Œufs durs aux pommes de terre.}
\addcontentsline{toc}{section}{ Œufs durs aux pommes de terre.}
\index{Œufs durs aux pommes de terre}

Pour quatre personnes prenez :

\medskip

\footnotesize
\begin{longtable}{rrrp{16em}}
1 000 & grammes & de & pommes de terre,                                                                   \\
  125 & grammes & de & beurre,                                                                            \\
   15 & grammes & de & vinaigre de vin,                                                                   \\
      &         &  4 & œufs,                                                                              \\
      &         &  3 & échalotes,                                                                         \\
      &         &    & persil,                                                                            \\
      &         &    & sel et poivre.                                                                     \\
\end{longtable}
\normalsize

Faites cuire les pommes de terre dans de l'eau salée, pelez-les, coupez-les en
tranches.

En même temps, faites durcir les œufs, retirez-les des coquilles et coupez-les
en deux.

Disposez les moitiés d'œufs et les tranches de pommes de terre dans un plat,
maintenez le tout au chaud. Mettez dans une poêle le beurre et les échalotes
hachées ; laissez dorer ; salez, poivrez, ajoutez le vinaigre, chauffez, puis
versez au travers d'une passoire sur les œufs et les pommes de terre. Parez de
persil haché et servez.

Ce plat, que j'ai eu l'occasion de goûter à la campagne, m'a paru excellent
après une longue marche.

\section*{\centering Œufs farcis\footnote{
\index{Définition des farces}
\index{Farces (Définition des)}
 Au sens culinaire, les farces sont des hachis de
substances alimentaires dûment assaisonnés et aromatisés, plus ou moins liés. On
les emploie généralement pour garnir des œufs durs, des poissons, des viandes,
de la volaille, du gibier, des pâtés, des légumes. Leur rôle est de donner
à l'ensemble de la préparation une note nouvelle s’harmonisant avec celle de
l'élément principal et le faisant valoir. Elles servent quelquefois aussi au
remplissage de bouchées.}, à la crème.}

\addcontentsline{toc}{section}{ Œufs farcis à la crème.}
\index{Œufs farcis à la crème}

Pour cinq personnes prenez :

\medskip

\footnotesize
\begin{longtable}{rrrp{16em}}
  250 & grammes & de & bonne crème, susceptible de cuire sans tourner,                                    \\
   6o & grammes & de & beurre,                                                                            \\
      &         & 10 & œufs frais,                                                                        \\
      &         &  1 & échalote,                                                                          \\
      &         &    & persil haché,                                                                      \\
      &         &    & sel et poivre.                                                                     \\
\end{longtable}
\normalsize

Faites durcir\footnote{Afin d'éviter toute odeur sulfureuse, faites-les durcir
la veille du jour où vous préparerez le plat.} les œufs, enlevez-les des
coquilles, coupez-les en deux dans le sens de leur grand axe et sortez les
jaunes que vous passerez au travers d'une passoire, à l’aide d'un pilon.

Mettez dans une poêle le beurre et l'échalote hachée fin, faites blondir,
ajoutez les jaunes d'œufs passés et le persil haché, salez, poivrez, travaillez
le tout jusqu'à ce que vous ayez obtenu une pâte parfaitement homogène ;
emplissez-en le creux des blancs.

Disposez les blancs farcis sur un plat allant au feu, le côté plat de l'œuf sur le
fond ; chauffez la crème, ajoutez encore un peu de sel et de poivre, versez-la sur
les œufs, puis mettez au four pendant quelques minutes et servez.

\section*{\centering Œufs au fromage.}
\addcontentsline{toc}{section}{ Œufs au fromage.}
\index{Œufs au fromage}

Foncez de beurre un plat allant au feu ; garnissez-en le fond de tranches de
pain anglais, coupées aussi minces que possible, saupoudrez d'une légère couche
de fromage de Gruyère râpé, un peu salé ; cassez un œuf sur chaque tranche de
pain, remettez un peu de fromage et faites cuire rapidement sur un feu vif sans
autre assaisonnement.

On peut également interposer entre le pain et le fromage des petites lames de
jambon fumé ; on peut aussi, dans l’une ou l’autre façon, couvrir le tout de
crème, puis effectuer la cuisson au lour.

Toutes ces préparations, qui sortent un peu de l'ordinaire, sont généralement
bien accueillies.

\section*{\centering Symphonie d'œufs.}
\addcontentsline{toc}{section}{ Symphonie d'œufs.}
\index{Symphonie d'œufs}

Pour six personnes prenez douze œufs.

Faites-en cuire deux durs ; laissez-les refroidir et hachez-les.

Faites pocher six œufs et tenez-les au chaud.

Préparez dans une large poêle, avec les quatre œufs qui restent, une omelette
mince, simple, aromatisée ou non, ou contenant des substances chaudes, au goût.

Dès que l'omelette est cuite, poudrez-en l'intérieur avec le hachis d'œufs
durs, puis disposez dessus les œufs pochés, pliez l'omelette, faites-la glisser
sur un plat et servez,

L'omelette doit être servie de manière que chaque convive ait dans sa part un
œuf poché.

On peut envoyer en même temps une saucière de sauce tomate, de béchamel ou de
crème plus ou moins acidulée par du jus de citron, au goût.

Une manière plus élégante de présenter le plat consiste à faire autant de petites
omelettes qu'il y a de convives et à mettre un œuf poché dans chaque omelette.

\section*{\centering Crème au fromage.}
\addcontentsline{toc}{section}{ Crème au fromage.}
\index{Crème au fromage}

Pour trois personnes prenez :

\medskip

\footnotesize
\begin{longtable}{rrrp{16em}}
   70 & grammes & de & fromage de Gruyère râpé,                                                           \\
   60 & grammes & de & crème épaisse,                                                                     \\
   10 & grammes & de & beurre,                                                                            \\
      &         &  6 & œufs frais,                                                                        \\
      &         &    & sel et poivre.                                                                     \\
\end{longtable}
\normalsize

Battez ensemble les œufs, la crème et {\ppp60\mmm} grammes de fromage ; assaisonnez au
goût.

Beurrez six moules à dariole, saupoudrez le beurre avec le reste du fromage,
mettez au-dessus le mélange que vous venez de préparer, couvrez les moules pour
empêcher l'appareil de se soulever à la cuisson et faites pocher dans de l'eau
bouillante pendant vingt minutes.

Démoulez et servez avec une sauce hollandaise tomatée.

\section*{\centering Croquettes de crème au jambon et à la langue.}
\addcontentsline{toc}{section}{ Croquettes de crème au jambon et à la langue.}
\index{Croquettes de crème au jambon et à la langue}

Pour six personnes prenez :

\medskip

\footnotesize
\begin{longtable}{rlrp{16em}}
  500 & grammes & de & lait,                                                                              \\
  100 & grammes & de & jambon d'York ou de Bayonne,                                                       \\
  100 & grammes & de & langue à l'écarlate,                                                               \\
  100 & grammes & de & champignons de couche épluchés,                                                    \\
   80 & grammes & de & carottes épluchées,                                                                \\
   50 & grammes & de & beurre,                                                                            \\
   30 & grammes & d' & oignons épluchés,                                                                  \\
   30 & grammes & de & poireaux (le blanc seulement),                                                     \\
   25 & grammes & de & farine,                                                                            \\
   20 & grammes & de & truffe épluchée,                                                                   \\
    5 & grammes & de & sel blanc,                                                                         \\
  1/2 & gramme  & de & poivre,                                                                            \\
      & 1 litre & d' & eau,                                                                               \\
      &         &  6 & œufs frais,                                                                        \\
      &         &  1 & abatis de poulet,                                                                  \\
      &         &  1 & bouquet garni, composé de 10 grammes de persil,
                       1 branche de thym et 1 feuille de laurier,                                         \\
      &         &    & mie de pain rassis tamisée,                                                        \\
      &         &    & muscade,                                                                           \\
      &         &    & persil,                                                                            \\
      &         &    & citron.                                                                            \\
\end{longtable}
\normalsize

Faites cuire à petit feu, pendant trois heures, abatis, champignons, carottes,
oignons, poireaux, bouquet garni dans l'eau de façon à obtenir {\ppp125\mmm}
à {\ppp150\mmm} grammes de jus concentré. Laissez refroidir, dégraissez,
passez.

Faites bouillir le lait.

Hachez jambon, langue et truffe.

Battez ensemble {\ppp3\mmm} œufs entiers et {\ppp2\mmm} jaunes.

Mettez dans une casserole le beurre avec la farine, maniez sans laisser
roussir, mouillez avec le lait et le jus concentré, laissez cuire pendant
quelques minutes, puis ajoutez les œufs battus, le hachis de viandes et de
truffe, le sel, le poivre, un peu de muscade, mélangez bien, remettez le tout
sur le feu pour un instant, enfin faites prendre comme une crème, au
bain-marie.

Laissez refroidir.

Battez l'œuf entier et les deux blancs qui restent.

Découpez la crème en morceaux, roulez-les en croquettes, passez-les dans l'œuf
battu, puis dans la mie de pain rassis tamisée, de façon à les bien enrober et
laites-les frire comme des beignets dans de la graisse claire, bien chaude.

Dressez les croquettes frites sur un plat chaud et décorez, avant de servir, avec
du persil fnit et des tranches de citron.

\section*{\centering Fondue au fromage.}
\addcontentsline{toc}{section}{ Fondue au fromage.}
\index{Fondue au fromage}

Pour six personnes prenez :

\medskip

\footnotesize
\begin{longtable}{rrrp{16em}}
  240 & grammes & de & fromage de Gruyère râpé,                                                           \\
  240 & grammes & de & beurre frais,                                                                      \\
  150 & grammes & de & truffes noires du Périgord,                                                        \\
  125 & grammes & de & jus bien dégraissé de viande de boucherie
                       ou de volaille, de préférence de dindon rôti,                                      \\
      &         & 12 & œufs,                                                                              \\
      &         &    & le jus d'un demi-citron,                                                           \\
      &         &    & sel et poivre.                                                                     \\
\end{longtable}
\normalsize

Lavez. nettoyez et pelez les truffes, coupez-les en petits dés, faites-les
sauter pendant cinq minutes dans {\ppp80\mmm} grammes de beurre en les remuant
sans cesse, salez et poivrez au goût, puis retirez-les et mettez-les de côté
sur une assiette.

Cassez les œufs, séparez les blancs des jaunes, passez-les séparément au
travers d'un linge ou d'un chinois afin de retenir les membranes et les germes
qui coupent leur homogénéité, et battez blancs et jaunes isolément.

Lorsque les blancs seront montés en mousse, incorporez-y les jaunes par petites
quantités, salez, ajoutez {\ppp120\mmm} grammes de beurre coupé en lames, le
fromage râpé et les truffes : vous obtiendtes ainsi une préparation dont vous
vous servirez immédiatement.

Mettez {\ppp80\mmm} grammes de jus dans un plat en porcelaine un peu profond et
allant au feu. Lorsqu'il sera bouillant, versez la préparation dedans et
fouettez le tout ; quand le mélange commencera à devenir onctueux, retirez le
plat du feu, continuez à fouetter jusqu'à ce que vous ayez obtenu une crème
lisse, ajoutez le reste du jus et le reste du beurre coupé en petits morceaux,
poivrez.

Remettez le tout sur feu doux, achevez la cuisson très lentement, comme pour
une crème, aromatisez avec le jus de citron, puis servez sur assiettes chaudes.

La fondue doit être absolument homogène, sans le moindre grumeau. Préparée
comme je viens de le dire, elle me paraît ne pas être éloignée de la
perfection.

\section*{\centering Soufflés.}
\addcontentsline{toc}{section}{ Soufflés.}
\index{Soufflés}
\index{Définition des soufflés chauds}
\index{Soufflés chauds (Définition des)}

Les soufflés sont des mets ou des entremets préparés avec des appareils à base
d'œufs, dont les blancs sont battus en neige, ce qui les fait souffler à la
cuisson.

On les sert dans les ustensiles où ils ont cuit : plats creux, timbales, petites
cocottes, petites caisses, etc.

On peut introduire dans les soufflés différentes substances alimentaires : de
la farine, de la fécule, du fromage râpé, du poisson ou des crustacés, de la
viande de boucherie ou de porc, des issues, de la volaille, du gibier à poil ou
à plumes, des légumes divers cuits et passés en purée ; des substances sucrées,
fruits, liqueurs, etc.

On trouvera dans ce livre un certain nombre de formules concrètes de soufflés
(Voir \hyperlink{p9001}{table alphabétique}).

\sk

Voici quelques indications générales sur un certain nombre de combinaisons
eupeptiques de soufflés, de garnitures et de sauces qui pourront donner des
idées aux chercheurs : soufflé aux crustacés, en turban de riz au maigre, sauce
hollandaise aux œufs de homard ; soufflé de poissons de mer, en turban de
pommes de terre et de poissons de rivière, sauce au fumet de poisson ; soufflé
au jambon, en turban de macaroni, sauce demi-glace tomatée relevée par du
paprika ; soufflé de poulet, en turban de riz à la moelle, sauce
Nantua\footnote{La sauce Nantua est une béchamel montée à la crème, finie avec
du beurre d'écrevisses et garnie de queues d'écrevisses.} ; soufflé de canard,
aux olives farcies de foie gras, sauce madère ; 
\index{Soufflé de poissons de mer}
\index{Soufflé au jambon}
\index{Soufflé de poulet}
\index{Foie gras d'oie en soufflé}
\index{Soufflé de canard aux olives}
\index{Soufflé de gibier}
soufflé de foie gras avec une
purée de morilles à la crème, sauce au fumet de venaison ; soufflé de gibier
à poil, en turban de pommes Champs-Élysées aux truffes, sauce grand
veneur\footnote{La sauce grand veneur est une sauce poivrade au fumet de
venaison, additionnée de sang de gibier à poil dilué dans de la marinade (100
grammes de sang de gibier par litre de sauce), mis au dernier moment. On
chauffe sans laisser bouillir, puis on passe la sauce à l'étamine.} ; 
\index{Bécasses en soufflé}
soufflé de gibier à plumes, tel le soufflé de bécasse, à la purée de truffes
garnie de croûtons frits recouverts avec les intérieurs flambés à la fine
champagne, sauce demi-glace au fumet de bécasse.

\section*{\centering Soufflé au fromage.}
\addcontentsline{toc}{section}{ Soufflé au fromage.}
\index{Soufflés au fromage}

Pour six personnes prenez :

\medskip

\footnotesize
\begin{longtable}{rrrp{16em}}
  250 & grammes & de & crème,                                                                             \\
  125 & grammes & de & gruyère râpé,                                                                      \\
   30 & grammes & de & beurre,                                                                            \\
   25 & grammes & de & parmesan râpé,                                                                     \\
   25 & grammes & de & fécule,                                                                            \\
      &         &  6 & œufs frais.                                                                        \\
      &         &    & muscade,                                                                           \\
      &         &    & sel,                                                                               \\
      &         &    & poivre fraîchement moulu.                                                          \\
\end{longtable}
\normalsize

Cassez les œufs ; séparez les blancs des jaunes.

Mettez dans une casserole la crème, le beurre et la fécule ; chauffez en
mélangeant constamment jusqu'à l'obtention d'une pâte lisse et homogène ;
assaisonnez au goût avec muscade, sel et poivre. Éloignez la casserole du feu ;
ajoutez les jaunes d'œufs, le gruyère et le parmesan ; mélangez encore.

Battez quatre blancs en neige ferme, incorporez-les doucement à l'appareil en
évitant de faire tomber la neige ; versez le tout dans un plat creux légèrement
beurré et poussez au four pas trop chaud. Vingt-cinq à trente minutes de cuisson
suffiront.

Servez aussitôt.

\section*{\centering Œufs pochés en soufflé.}
\addcontentsline{toc}{section}{ Œufs pochés en soufflé.}
\index{Œufs pochés en soufflé}

Pour six personnes prenez les mêmes substances que ci-dessus, plus six œufs
frais.

Préparez l'appareil à soufflé au fromage, comme précédemment,

Faites pocher six œufs.

Beurrez un plat creux, mettez dedans une partie de l'appareil, au-dessus les
œufs pochés, couvrez avec le reste de l'appareil et poussez au four comme
précédemment.

Ce plat est une agréable surprise.

\sk
