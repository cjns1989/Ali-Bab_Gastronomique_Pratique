\sk

\begin{center}
\textit{GLACES}
\end{center}
\index{Glaces}
\index{Définition des glaces d'entremets}

\bigskip

Les glaces sont des crèmes ou des sirops aromatisés congelés. On fait des glaces
aux fruits, aux liqueurs, aux essences, aux amandes, aux pistaches, au pralin,
au thé, au café, au chocolat, à la vanille, etc.

Le nombre des combinaisons pour glaces est considérable ; des volumes ont été
publiés sur cette matière\footnote{Je renvoie aux ouvrages spéciaux les
lecteurs qui désirent approfondir la question.}.

L'art du glacier réside dans la façon de varier la forme, la garniture, le parfum,
la couleur, la consistance, le groupement des différents éléments qui constituent
les édifices que son imagination enfante ; il consiste aussi dans la manière de les
décorer afin de flatter, tout à la fois, l'œil et le palais.

Les glaces sont la clôture des diners. Elles doivent toujours être préparées
d'avance. Leur préparation comporte deux parties : la composition, le glaçage.

\sk

\index{Compositions pour crèmes glacées}
\index{Compositions pour glaces}
Les compositions pour crèmes glacées sont obtenues en mélangeant et en
travaillant convenablement du sucre et des jaunes d'œufs, dans une casserole,
jusqu'à ce que la préparation soit bien homogène (en terme consacré : qu'elle
fasse le ruban) ; on la délaye doucement ensuite avec du lait bouillant et on
la fait épaissir sur feu doux, sans laisser bouillir, jusqu'à ce qu'elle masque
bien la cuiller. On passe alors la crème et on la laisse refroidir en la
remuant fréquemment.

Les proportions de sucre et de jaunes d'œufs pour une même quantité de lait
varient suivant qu'on veut obtenir des glaces maigres ou des glaces grasses. Si
l'on emploie peu de jaunes d'œufs et de sucre, on a des glaces maigres et
fermes ; on a, au contraire, des glaces grasses et moelleuses en augmentant la
quantité de ces deux éléments.

\medskip

Une bonne composition moyenne pour crèmes glacées est la suivante :

\footnotesize
\begin{longtable}{rrrrp{16em}}
  \hspace{4em} & nnn à ooo & grammes & de & sucre,                                                        \kill
  & 350 & grammes & de & sucre,                                                                           \\
  &     & 1 litre & de & lait,                                                                            \\
  &     &         & 10 & jaunes d'œufs.                                                                   \\
\end{longtable}
\normalsize

Les quantités extrêmes pour les crèmes glacées sont :

\footnotesize
\begin{longtable}{rrrrp{16em}}
  \hspace{4em} & nnn à ooo & grammes & de & sucre,                                                        \kill
  \multicolumn{2}{r}{200 à 500} & grammes & de & sucre,                                                   \\
  & & 1 litre & de & lait,                                                                                \\
  &  & \multicolumn{2}{r}{7 à 16} & jaunes d'œufs.                                                        \\
\end{longtable}
\normalsize

En remplaçant tout ou partie du lait par de la crème fraîche ou même en
incorporant à la composition glacée de la crème Chantilly, on obtiendra des
glaces encore plus onctueuses.

\medskip

On parfume les crèmes glacées par simple addition ou par infusion directe des
parfums dans le lait.

Les crèmes glacées aux liqueurs sont obtenues en incorporant à la composition
une certaine quantité de la liqueur choisie, quantité qui ne doit pas être
inférieure à {\ppp100\mmm} grammes par litre de composition.

\sk

Les compositions pour sirops glacés sont faites avec du sirop de sucre à froid
à {\ppp32\mmm}° auquel on mélange la purée, le jus de fruits ou la liqueur.
Elles pèsent de {\ppp18\mmm} à {\ppp30\mmm}° au pèse-sirop. Le jus de citron
fait toujours partie de la combinaison, en plus ou moins grande quantité
suivant le plus ou moins d'acidité des fruits ou suivant la liqueur employée.
On fait entrer du jus d'orange et du jus de citron dans les compositions de
glaces aux fruits rouges ; cela exalte le parfum des fruits. On additionne,
parfois aussi, de crème fraîche les compositions aux fruits, lorsqu'elles sont
prises.

\sk

Le glaçage peut être fait de deux façons.

\medskip

1° Les compositions sont mises dans des moules et glacées directement ; c'est
le procédé employé pour les glaces légères.

2° Les compositions sont d'abord congelées à la sorbétière\footnote{La
sorbétière est un appareil comprenant une turbine et un ou deux malaxeurs
tournant en sens inverse. La composition à glacer est versée dans l'appareil :
la rotation projette les particules liquides sur les parois, ce qui facilite
leur congélation ; le ou les malaxeurs détachent au fur et à mesure la glace
qui se forme et s'homogénéise.} , mises ensuite en moule et glacées
définitivement. Les crèmes glacées et les sirops glacés sont obtenus de cette
façon. Voici comment on procède : la sorbétière, vide, est mise dans un seau en
bois contenant un mélange bien tassé de glace concassée, sel marin et salpêtre,
dans les proportions de {\ppp10\mmm} kilogrammes de glace, {\ppp1 500\mmm}
grammes de sel et {\ppp250\mmm} grammes de salpêtre, qui l'entoure jusqu'aux
deux tiers de la hauteur seulement, afin d'éviter qu'il entre accidentellement,
pendant le travail, de la glace salée dans la préparation. C'est ce qu'on
appelle le \index{Définition du mot sanglage} \textit{sanglage}. Lorsque
l'appareil est bien refroidi, c'est-à-dire au bout d'un quart d'heure environ,
on verse la composition dans la sorbétière qu'on met ensuite en mouvement, soit
en la faisant tourner sur elle-même alternativement de droite et de gauche si
on travaille à la main, soit à la manivelle si la sorbétière est mécanique. La
composition doit être lisse, homogène, fine et moelleuse lorsqu'elle a été bien
travaillée.

On peut dresser les glaces sortant de la sorbétière simplement en monticule sur
une serviette couvrant un plat, ou dans des coupes. On peut aussi les mouler.

Lorsqu'on moule les glaces, on emploie pour cet usage des moules spéciaux
fermant hermétiquement. On verse dedans la préparation qu'on tasse bien pour
chasser l'air qui pourrait former des cavités dans la glace et lorsque les
moules sont bien pleins, on lute leur fermeture avec du beurre afin d'éviter
qu'il entre dedans de l'eau salée pendant le sanglage. Ces précautions prises,
on met les moules dans le mélange réfrigérant, indiqué pour la sorbétière, qui
doit les recouvrir complètement et où on les laisse pendant une heure pour les
glaces ordinaires et deux heures au moins pour les glaces légères.

Pour servir, on lave d'abord le moule dans de l’eau froide au sortir de la glace,
puis on le passe un instant dans de l'eau tiède, on démoule sur un plat couvert
d'une serviette pliée et on sert aussitôt.

\medskip

Voici deux exemples de glaces, l’un de crème glacée, l'autre de sirop glacé.

\section*{\centering Glace à la vanille.}
\phantomsection
\addcontentsline{toc}{section}{ Glace à la vanille.}
\index{Glace à la vanille}

La glace à la vanille est le type des crèmes glacées.

\medskip

Pour douze personnes prenez :

\footnotesize
\begin{longtable}{rrrp{16em}}
  1 000 & grammes & de & crème,                                                                           \\
    450 & grammes & de & sucre en poudre,                                                                 \\
        &         & 12 & jaunes d'œufs,                                                                   \\
        &         &  1 & gousse de vanille.                                                               \\
\end{longtable}
\normalsize

Chauffez la crème, au bain-marie, avec la vanille, de façon à la bien parfumer.

Mélangez, en travaillant convenablement, le sucre et les jaunes d'œufs ;
lorsque le mélange est bien lisse et qu'il fait le ruban, ajoutez par petites
quantités la crème chaude. Faites épaissir en remuant, sur feu doux, puis
refroidissez complètement la préparation. Versez-la ensuite dans une
sorbétière, sanglez et laissez à la glace pendant deux heures ; ou moulez-la
après l'avoir fait prendre à la sorbétière. Sanglez ensuite.

\section*{\centering Glace aux fraises ou aux framboises.}
\phantomsection
\addcontentsline{toc}{section}{ Glace aux fraises ou aux framboises.}
\index{Glace aux fraises ou aux framboises}

La glace aux fraises ou aux framboises est le type des sirops glacés.

\medskip

Pour douze personnes prenez :

\footnotesize
\begin{longtable}{rrrp{16em}}
    750 & grammes & de & purée de fraises ou de framboises,                                               \\
    750 & grammes & de & sirop de sucre à froid à 32°,                                                    \\
        &         &  3 & oranges,                                                                         \\
        &         &  2 & citrons.                                                                         \\
\end{longtable}
\normalsize

Pressez les oranges et les citrons, ajoutez le jus obtenu à la purée de fraises
ou de framboises et mélangez le tout avec le sirop de sucre. Versez la
composition dans une sorbétière et sanglez pendant deux heures. Démoulez et
servez comme à l'ordinaire.

\sk

On sert aussi des glaces panachées, de compositions et de couleurs différentes.

Après les avoir fait prendre isolément dans des sorbétières, on les dispose par
couches horizontales ou concentriques dans un moule qu'on sangle pendant deux
heures. On démoule et on sert.

\sk

On peut enfin présenter ces glaces panachées dressées côte à côte dans des
coupes en cristal.

\sk

On garnit aussi des coupes avec des glaces additionnées de fruits frais ou
parfumés avec une liqueur, entiers ou coupés en dés ; de fruits rafraîchis ; de
fruits mi-sucre ; de compote ou de purée de fruits ; de crème fraîche ou de
crème Chantilly.

\bigskip

\begin{center}
\textit{FROMAGES GLACÉS}
\end{center}
\index{Fromages glacés}
\index{Définition des fromages glacés}

\bigskip

On nomme fromages glacés les crèmes glacées, moulées dans des moules à côtes.
Ils sont généralement faits de deux glaces, de couleurs et de parfums
différents, dressées verticalement dans le moule.

\bigskip

\begin{center}
\textit{BISCUITS GLACÉS}
\end{center}
\index{Biscuits glacés}
\index{Biscuits glacés (Définition des)}
\index{Définition des biscuits glacés}
\index{Compositions pour biscuits glacés}
\index{Compote d'abricots}
\index{Coupes glacées}
\index{Biscuits glacés (Formules de)}

\bigskip

Les biscuits glacés sont le résultat de la congélation d'une crème anglaise
aromatisée, faite avec {\ppp12\mmm} jaunes d'œufs, {\ppp500\mmm} grammes de
sucre et {\ppp1\mmm} litre de lait, passée et refroidie en la vannant, puis
rendue mousseuse avec le fouet ; ou de la même composition additionnée de crème
fouettée.

On fait aussi les biscuits glacés en mélangeant sur feu doux du sucre et des
jaunes d'œufs dans les mêmes proportions que ci-dessus jusqu'à ce que le
mélange soit ferme et fasse le ruban. On fouette ensuite le mélange jusqu'à ce
qu'il soit complètement froid, puis on y ajoute {\ppp250\mmm} grammes de
meringue italienne et {\ppp800\mmm} grammes de crème fouettée,

On moule les biscuits glacés dans des moules rectangulaires spéciaux munis de
deux couvercles, l'un faisant le fond, l'autre le dessus. Le milieu du moule et
chacun des couvercles reçoit une composition différente. On fait glacer, puis on
démoule ces sortes de briquettes présentant trois teintes ou trois couches de glaces
différentes. On coupe ces briquettes en tranches perpendiculairement à leur grand
axe. On place ces morceaux dans des caisses en papier et on les tient dans un
rafraîchissoir garni de glace jusqu'au moment de servir.

On fait des biscuits glacés à tous parfums.

Exemples :

\medskip

1° Biscuit kirsch, marron, pistache.

Mouler le milieu en composition aux marrons, l'un des couvercles en composition
aux pistaches, l’autre en composition au kirsch.

2° Biscuit fraise, praliné, abricot.

Mouler le milieu en praliné, l'un des couvercles en composition aux fraises,
l’autre en composition aux abricots.

\bigskip

\begin{center}
\textit{MOUSSES GLACÉES}
\end{center}
\index{Mousses glacées}
\index{Définition des mousses glacées}
\index{Compositions pour mousses glacées}

\bigskip

Les mousses glacées sont des glaces qui ont beaucoup d'analogie avec les
biscuits glacés. Elles renferment toujours de la crème fouettée. Les compositions
pour mousses sont faites à base de crème anglaise ou à base de sirop de sucre.

On fait des mousses aux fruits, aux liqueurs, aux noix fraîches, au café, au
chocolat, à la vanille, etc.

Lorsqu'on prépare des mousses à la crème anglaise, la composition comporte
généralement :

\footnotesize
\begin{longtable}{rrrp{16em}}
    500 & grammes   & de & sucre,                                                                         \\
    400 & grammes   & de & crème fouettée,                                                                \\
        & 1/2 litre & de & lait,                                                                          \\
        &           & 16 & jaunes d'œufs frais.                                                           \\
\end{longtable}
\normalsize

Quand la composition est refroidie. on y ajoute le parfum.

On peut faire à la crème anglaise des mousses aux fruits ; cependant, il vaut
mieux les préparer au sirop de sucre à {\ppp35\mmm}°.

\medskip

Voici quelques formules de mousses.

\medskip

\section*{\centering Mousse à la vanille glacée.}
\phantomsection
\addcontentsline{toc}{section}{ Mousse à la vanille glacée.}
\index{Mousse à la vanille glacée}

Pour douze personnes prenez :

\footnotesize
\begin{longtable}{rrrrp{16em}}
  & 500 & grammes & de & sucre,                                                                           \\
  & 400 & grammes & de & crème fouettée,                                                                  \\
  & \multicolumn{2}{r}{1/2 litre} & de & lait,                                                            \\
  &     &         & 16 & jaunes d'œufs frais,                                                             \\
  &     &         &    & vanille.                                                                         \\
\end{longtable}
\normalsize

Préparez une crème anglaise avec le sucre, le lait, les jaunes d'œufs et la
vanille : refroidissez-la en la vannant, puis incorporez-la à la crème
fouettée.

Emplissez avec la composition {\ppp12\mmm} petites caisses en papier blanc et
mettez à la glacière pendant quatre heures.

\section*{\centering Mousse à la chartreuse glacée.}
\phantomsection
\addcontentsline{toc}{section}{ Mousse à la chartreuse glacée.}
\index{Mousse à la chartreuse glacée}

Pour douze personnes prenez :

\footnotesize
\begin{longtable}{rrrrp{16em}}
  & 500 & grammes & de & sucre,                                                                           \\
  & 400 & grammes & de & crème fouettée,                                                                  \\
  & 100 & grammes & de & chartreuse verte,                                                                \\
  & \multicolumn{2}{r}{1/2 litre} & de & lait,                                                            \\
  &     &         & 16 & jaunes d'œufs frais.                                                             \\
\end{longtable}
\normalsize

Préparez la crème anglaise, refroidissez-la comme il est dit ci-dessus, puis
incorporez-la ainsi que la chartreuse à la crème fouettée.

Mettez la composition en petits moules foncés de papier blanc et sanglez pendant
trois heures.

\section*{\centering Mousse au café glacée.}
\phantomsection
\addcontentsline{toc}{section}{ Mousse au café glacée.}
\index{Mousse au café glacée}

Pour douze personnes prenez :

\footnotesize
\begin{longtable}{rrrp{16em}}
    500 & grammes & de & sucre,                                                                           \\
    400 & grammes & de & crème fouettée,                                                                  \\
    250 & grammes & d' & essence de café,                                                                 \\
    150 & grammes & de & beurre,                                                                          \\
        &         & 16 & jaunes d'œufs frais.                                                             \\
\end{longtable}
\normalsize

Triturez et mélangez bien sur feu doux sucre et jaunes d'œufs ; lorsque la
préparation fait le ruban, ajoutez l'essence de café, chauffez encore pendant
quelques instants, passez ensuite à l’étamine. Montez alors la composition avec
le beurre, refroidissez-la en la remuant, puis incorporez-la à la crème
fouettée.

Emplissez avec cette mousse {\ppp12\mmm} petites caisses et mettez-les à glacer
pendant quatre heures.

\section*{\centering Mousse au chocolat glacée.}
\phantomsection
\addcontentsline{toc}{section}{ Mousse au chocolat glacée.}
\index{Mousse au chocolat glacée}

Pour douze personnes prenez :

\footnotesize
\begin{longtable}{rrrp{16em}}
    800 & grammes & de & crème fouettée,                                                                  \\
    350 & grammes & de & chocolat,                                                                        \\
    250 & grammes & de & sucre en poudre,                                                                 \\
    200 & grammes & d' & eau.                                                                             \\
\end{longtable}
\normalsize

Mettez dans une casserole le chocolat et l'eau, laissez fondre à petit feu, puis
ajoutez le sucre et réduisez le mélange pendant un quart d'heure environ, à feu
doux, en remuant constamment avec une cuiller en bois.

Lorsque cette préparation est refroidie, incorporez-la à la crème fouettée de
manière que la crème reste ferme.

Emplissez des caisses ou des moules avec cette mousse et faites-la glacer
pendant trois à quatre heures,

\section*{\centering Mousse aux fraises glacée.}
\phantomsection
\addcontentsline{toc}{section}{ Mousse aux fraises glacée.}
\index{Mousse aux fraises glacée}

Pour douze personnes prenez :

\footnotesize
\begin{longtable}{rrrrp{16em}}
  & 800 & grammes & de & crème fouettée,                                                                  \\
  & \multicolumn{2}{r}{1/2 litre} & de & sirop de sucre à froid, à 35°,                                   \\
  & \multicolumn{2}{r}{1/2 litre} & de & purée de fraises.                                                \\
\end{longtable}
\normalsize

Mélangez ensemble sirop de sucre et purée de fraises, puis incorporez le
mélange à la crème fouettée.

Emplissez {\ppp12\mmm} petits moules foncés de papier blanc avec la composition
et sanglez fortement pendant trois heures.

\bigskip

\begin{center}
\textit{BOMBES GLACÉES}
\end{center}
\addcontentsline{toc}{section}{ Bombes glacées.}
\index{Bombes glacées}
\index{Bombes glacées (Définition des)}
\index{Bombes glacées (Formules de)}
\index{Compositions pour bombes glacées}
\index{Définition des bombes glacées}

\bigskip

Les bombes glacées sont des glaces composées, constituées par une enveloppe
généralement mince de glace simple et un remplissage plus où moins moelleux de
composition de bombe.

Anciennement, on moulait les bombes en moules sphériques, d'où leur nom ;
aujourd'hui, on les fait dans des moules tronconiques unis à sommet arrondi.

Les compositions pour bombes sont faites avec {\ppp32\mmm} jaunes d'œufs pour
un litre de sirop à {\ppp28\mmm}°.

On monte le mélange au fouet sur feu doux ; lorsqu'il est bien homogène, on le
retire du feu, on le passe, on le met sur glace et on continue à le fouetter
jusqu'à complet refroidissement. On le parfume ensuite, puis on y ajoute
{\ppp1\mmm} {\ppp200\mmm} grammes de crème fouettée très ferme.

\medskip

Pour mouler les bombes, on commence par chemiser le fond et les parois du moule
d'une enveloppe assez mince faite d'une composition de glace ordinaire choisie,
prise à la sorbétière ; puis l'intérieur est rempli avec une composition de
bombe parfumée au goût, ou avec une mousse. On couvre avec un rond de papier
blanc, on ferme hermétiquement le moule et on le sangle fortement pendant deux
heures au moins.

Au moment de servir, on démoule la bombe sur un bloc de glace ou sur une
serviette pliée posée dans un plat.

\medskip

On fait des bombes à tous parfums de crèmes et de sirops glacés ; on les garnit
de marmelades ou de fruits coupés en dés ou émincés ; enfin on les décore.

\medskip

Voici quelques exemples de bombes glacées.

\medskip

1° Bombe chemisée de glace aux cerises, intérieur composition de bombe au
marasquin ou au kirsch ;

2° Bombe chemisée de glace à la vanille, intérieur composition de bombe à la
fraise ;

3° Bombe chemisée de glace aux pêches, intérieur crème Chantilly additionnée de
groseilles de Bar ou de framboises ;

4° Bombe chemisée de glace aux amandes ou aux noisettes, intérieur composition
de bombe au chocolat ;

5° Bombe chemisée de glace au café, intérieur composition de bombe
à l'eau-de-vie de Châteauneuf-du-Pape ;

6° Bombe chemisée de glace au citron ou à l'orange, intérieur composition de
mousse aux fraises ;

7° Bombe chemisée de glace aux abricots, intérieur composition de bombe
pralinée. Après démoulage, décorer avec des cerises confites ;

8 Bombe chemisée de glace aux framboises, intérieur composition de bombe au
champagne. Après démoulage, décorer avec de la meringue italienne et des dés
d'ananas ;

9” Bombe chemisée de glace au curaçao où au kummel, intérieur composition
de bombe aux prunes Reine-Claude ou aux mirabelles ;

10° Bombe chemisée de glace aux pistaches, intérieur composition de bombe à
la chartreuse ;

11° Bombe chemisée de glace à la grenadine, intérieur composition de bombe
aux poires ;

12° Bombe chemisée de glace pralinée, intérieur composition de mousse à
l'anisette.


\bigskip

\begin{center}
\textit{PARFAITS}
\end{center}
\addcontentsline{toc}{section}{ Parfaits.}
\index{Parfaits}
\index{Compositions pour parfaits}
\index{Définition des parfaits}

\bigskip

Les parfaits sont des glaces, non chemisées, à un seul parfum.

Autrefois, le nom de parfait était réservé exclusivement à la mousse au café
glacée en moule. Aujourd'hui, on fait des parfaits à tous les parfums : parfaits
pralinés ; parfaits à la vanille, au chocolat, aux liqueurs, aux spiritueux, etc.

Les compositions pour parfaits sont préparées avec {\ppp32\mmm} jaunes d'œufs
pour un litre de sirop froid à {\ppp28\mmm}°, qu'on fait prendre sur feu deux
comme une crème anglaise, On passe le mélange à l'étamine et on le refroidit en
le fouettant sur glace. Lorsqu'il est complètement froid, on y ajoute le parfum
choisi et {\ppp400\mmm} grammes de crème fouettée bien ferme.

On moule les compositions dans des moules à parfait et on sangle fortement
pendant deux à trois heures.

\bigskip

\begin{center}
\textit{SOUFFLÉS GLACÉS}
\end{center}
\addcontentsline{toc}{section}{ Soufflés glacés.}
\index{Soufflés glacés}
\index{Définition des soufflés glacés}
\index{Compositions pour soufflés glacés}

\bigskip


Les soufflés glacés sont des glaces préparées de manière à donner l'apparence
de soufflés.

Les compositions pour soufflés glacés sont de deux sortes. Pour les soufflés
à parfums tels que vanille, pistache, praliné, chocolat, café, liqueurs, etc.,
on emploie la mousse à la crème ; pour les soufilés aux fruits, on prend une
pâte à meringue obtenue en mélangeant {\ppp10\mmm} blancs d'œufs frais battus
en neige avec {\ppp500\mmm} grammes de sucre cuit au soufflé. Lorsque le
mélange est refroidi, on y ajoute un demi-litre de purée de fruits et un parfum
assorti, puis 400 grammes de crème fouettée bien ferme.

\section*{\centering Soufflé aux mirabelles, glacé.}
\phantomsection
\addcontentsline{toc}{section}{ Soufflé aux mirabelles, glacé.}
\index{Soufflé aux mirabelles, glacé}

Pour douze personnes prenez :

\footnotesize
\begin{longtable}{rrrrp{16em}}
  & 500 & grammes   & de & sucre,                                                                         \\
  & 400 & grammes   & de & crème fouettée,                                                                \\
  &  60 & grammes   & de & kirsch,                                                                        \\
  &  50 & grammes   & de & curaçao,                                                                       \\
  & \multicolumn{2}{r}{1/2 litre} & de & purée de mirabelles,                                             \\
  &     &           & 10 & blancs d'œufs frais.                                                           \\
\end{longtable}
\normalsize

Préparez la meringue. Lorsqu'elle est froide, ajoutez-y la purée de mirabelles,
le kirsch, le curaçao et enfin la crème fouettée.

Prenez un moule à soufflé ordinaire, entourez-le d'une bande de papier blanc
dépassant de plusieurs centimètres le bord du moule, versez dedans la
préparation qui doit déborder un peu le moule, et mettez à la glacière
fortement sanglée.

Au moment de servir, enlevez délicatement la bande de papier blanc ; la
composition doit donner l'illusion d'avoir monté comme un soufflé dans le
moule.

Dressez le soufflé sur une serviette ou sur un bloc de glace.

\medskip

Au lieu de mouler la préparation dans une grande timbale à soufflé, on peut la
mouler dans autant de petites caisses qu'il y a de convives.

\medskip

Les combinaisons de soufflés glacés sont très nombreuses.

\bigskip

\begin{center}
\textit{PUDDINGS GLACÉS}
\end{center}
\addcontentsline{toc}{section}{ Puddings glacés.}
\index{Puddings glacés}
\index{Définition des puddings glacés}

\bigskip

Les puddings glacés sont plutôt des entremets que des glaces. Les appareils
bavarois entrent souvent dans leur composition ; mais on fait aussi des
puddings glacés à d'autres préparations.

\medskip

\index{Compositions pour puddings glacés}
En voici deux formules :

\medskip

1° Chemisez une bombe de glace pralinée, emplissez le moule avec une
composition de bombe à la chartreuse et une composition de bombe aux poires
disposées par couches alternées, séparées par des gaufrettes très fines ou des
oublies, finissez le remplissage avec de la glace pralinée, fermez
hermétiquement le moule et sanglez-le pendant deux à trois heures.

Démoulez et servez en envoyant en même temps mais à part du sirop à l'abricot,
glacé.

2° Foncez un moule à charlotte avec des biscuits à la cuiller émincés, ou avec
des tranches minces de brioche mousseline, en les serrant bien ; aspergez-les
fortement de curaçao, emplissez le moule avec des couches alternées de
quartiers de mandarines épépinés et pelés à vif, de rondelles de bananes
trempées dans du curaçao et de croissants fins de pommes pochés au sirop
vanillé, séparées par un semis de macarons écrasés. Noyez le tout avec une
composition de bombe au kirsch. Mettez à la glace pendant trois heures.

Démoulez et servez. Envoyez à part un sirop à la mandarine, glacé.
