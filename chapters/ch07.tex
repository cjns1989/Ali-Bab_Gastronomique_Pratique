\index{Champignons comestibles de France (Les)}
Mon but n'est pas de faire ici un Traité de Mycologie. Je pourrais me borner
à donner simplement quelques recettes se rapportant aux champignons les plus
connus, mais il m'a semblé qu'une étude sommaire des principaux champignons
comestibles de France ne serait pas déplacée dans ce livre, étant donné la
multiplicité de leurs espèces et l'importance de leur rôle en Gastronomie. Il
suffit, en effet, pour s'en rendre compte, de supposer un instant qu'ils
n'existent pas : combien de savoureux entremets de légumes, combien d'exquises
garnitures nous manqueraient ; quels seraient la saveur et le parfum d'un grand
nombre de sauces et de ragoûts !

De tout temps les champignons ont concouru à l'alimentation des hommes.

La truffe blanche était connue du temps de Moïse.

Les Grecs appréciaient les champignons. En reconnaissance de la création faite
par Chérips d'un nouveau ragoût de truffes, ils décernèrent à ses enfants le titre
de Bourgeois d'Athènes.

Les Romains étaient très friands de cryptogames et ils considéraient l'oronge
comme un manger des dieux. Horace les a chantés et Apicius leur a consacré tout
un ouvrage.

En France, les champignons ont toujours été très recherchés. Plusieurs espèces,
telles que les champignons de couche, les morilles, les cèpes, les oronges et
les truffes viennent en tête parmi les meilleures.

Malheureusement, tous les champignons ne sont pas comestibles ; un certain
nombre d’entre eux sont même franchement toxiques. Cette toxicité était connue
anciennement : Pline en parle et Locuste, la grande empoisonneuse, a su en
tirer parti pour faire périr Claude et Britannicus.

\medskip

Les champignons sont des végétaux cryptogames dépourvus de chlorophylle. Ils se
reproduisent au moyen de semences extrêmement petites, appelées
\textit{spores}, qui forment une poussière abondante se disséminant dans
l'atmosphère. Les spores, en germant, donnent naissance à des filaments
ramifiés très fins et très nombreux ; c'est le mycelium, appelé « blanc de
champignon » dans les champignons de couche. Dans les espèces connues du public
sous le nom de champignons, le mycelium produit des fruits volumineux qui
paraissent être toute la plante mais qui, en réalité, n'en sont qu'une partie
puisqu'il naît de nouveaux fruits après qu'on à cueilli les champignons. Une
seule spore donne naissance à un mycelium très étendu qui envahit le sol sous
la forme d'une grande moisissure circulaire donnant des fruits sur toute sa
périphérie. Ainsi se produisent les « ronds de sorcières ».

Le mycelium existe toujours bien que, parfois, il soit diflicile ou même
impossible à constater dans de rares espèces : il sert à nourrir la plante aux
dépens du corps sur lequel il se fixe.

Les spores sont de petites vésicules de couleur variable ; elles sont contenues
dans une sorte de membrane plus ou moins épaisse appelée \textit{hymenium}. Cet
hymenium est disposé à la surface d'un réceptacle ou inclus à l'intérieur d’un
conceptacle ; il est constitué par des cellules fertiles nommées les unes
\textit{asques}, les autres \textit{basides} entremêlées à des cellules
stériles : d'où la classification en \textit{Ascomycètes} et
\textit{Basidiomycètes}. Lorsque les appareils sporifères se développent
à l'intérieur d'un conceptacle, le champignon prend une forme plus ou moins
sphérique (ex. : la truffe) : lorsqu'ils sont disposés sur un réceptacle, ils
amènent chez le champignon une grande variété de formes.

Au point de vue gastronomique, il faut retenir dans la classe des
Basidiomycètes le groupe des \textit{Hyménomycètes}, le plus important de tous
et celui des \textit{Gastéromycètes} ; dans la classe des Ascomycètes, le
groupe des \textit{Discomycètes} et celui des \textit{Périsporiacées}.

L'hymenium, chez les Hyménomycètes, est limité à une partie de la surface des
champignons qui prend des caractères spéciaux ; il est toujours tourné vers la
terre. Suivant les espèces, il peut être formé par des lames rayonnantes, des
tubes, des aiguillons, des dents, des verrues, etc. La partie du champignon qui
la porte s'appelle le chapeau, lequel repose sur un pied central ou
excentrique ; parfois le pied manque. Une exception existe dans ce groupe pour
les clavaires qui forment des petits buissons de tiges ramifiées.

Chez les Discomycètes, au contraire, l'hymenium est tourné en haut et disposé
en forme de disque, d'où leur nom. Ce disque peut former une cupule plus ou
moins régulière (Pézizes) ou prendre l'aspect d'une lame façonnée en cône et
creusée d'alvéoles (Morilles) ou contournée plus ou moins irrégulièrement
(Helvelles).

Il existe encore chez certains champignons des organes importants à connaître :
ce sont la \textit{volve}, l'\textit{anneau,} la \textit{cortine}, le
\textit{voile}.

La volve (volva) est une membrane qui s'insère à la base du pied et qui
enveloppe entièrement le champignon dans son jeune âge : on dit alors quil est
à l'état d'œuf. Quand le champignon grandit, la volve se déchire ou se
fragmente et il se produit une véritable éclosion. Les débris de la volve
persistent chez le champignon à l'état adulte ; c'est pourquoi il faut savoir
la reconnaitre.

L'anneau est une collerette circulaire qui s'insère sur le pied. Lorsque le
champignon est jeune, l'anneau est fixé au bord du chapeau : il enferme alors
les lames avec leurs spores dans une chambre close.

La cortine (cortina) est une membrane très délicate formée par des filaments
très fins, une sorte de filet qui se fixe à la partie supérieure du pied et
recouvre le chapeau dans le jeune âge. Lorsque le chapeau s'épanouit, la
cortine se déchire et ne laisse que quelques filaments suspendus au bord du
chapeau, ou un anneau filamenteux sur le pied.

Le voile est un tissu mince, fugace, qui enveloppe entièrement le champignon.
Il est moins bien caractérisé que la volve.

\medskip

Les champignons peuvent être divisés en comestibles, vénéneux, suspects et
indifférents.

Ils croissent presque exclusivement à la surface du sol, exception faite pour
ceux qui vivent sur le tronc ou sur les branches d'arbres ou d'arbrisseaux et
pour les espèces souterraines.

Ils apparaissent en toutes saisons : quelques-uns naissent au printemps ;
d'autres, plus rares, se montrent en hiver : la fin de l'été et l'automne sont
les saisons de prédilection du plus grand nombre des espèces.

Leurs formes sont très diverses : leurs couleurs sont multiples et variées,
variables même avec l’âge sur le même individu. La couleur peut être uniforme
dans toute la plante, ou bien le chapeau et le pied peuvent présenter des
colorations différentes.

Quelques champignons sont phosphorescents.

La plupart des champignons comestibles ont, à l'état frais comme à l’état sec,
peu ou pas de saveur. Cependant, chez quelques-uns, il existe une saveur agréable
de châtaigne, de noisette, etc. ; d'autres, au contraire, ont une saveur âcre,
poivrée, brûlante.

Leur odeur est fréquemment nulle, Certains ont simplement ce qu'on appelle
une odeur douce de champignon, mais quelques espèces ont une odeur agréable
de farine fraîche, de violette, d'anis, de vanille, de lavande, de flouve, de
fruits, etc.

Les champignons vénéneux ou indifférents, lorsqu'ils ne sont pas inodores, ont
parfois une odeur désagréable et même fétide ; mais ils peuvent aussi, comme
les bons champignons, avoir un parfum agréable.

Parmi les champignons vénéneux, les uns sont mortels\footnote{Les espèces de
champignons qui causent des empoisonnements mortels sont peu nombreuses. Elles
appartiennent aux genres Amanita, Volvaria, Entoloma.} ; d'autres déterminent
une intoxication plus ou moins grave ; enfin certains provoquent seulement une
violente indigestion.

Aussi, en dehors des champignons classiques : champignons de couche, cèpes,
morilles, chanterelles, clavaires, hydnes, truffes, etc., achetés sur les
marchés des grandes villes où il existe un contrôle sérieux, faut-il être très
circonspect dans le choix d'autres champignons et ne se fier pour leur
détermination qu'à des personnes qualifiées.

Les champignons s'altèrent très vite et leur décomposition peut les rendre
dangereux ; par exemple les helvelles qui, lorsqu'elles sont trop avancées,
peuvent causer des accidents graves.

En France, sur {\ppp1\mmm} {\ppp600\mmm} champignons dits supérieurs, il existe
{\ppp100\mmm} champignons réellement intéressants : {\ppp80\mmm}, parce qu'ils
sont d'excellents comestibles ; {\ppp20\mmm}, parce qu'ils sont très dangereux.

\newpage                        
\index{Champignons comestibles de France (Classification des)}
\index{Classification des champignons comestibles de France}

% \newgeometry{top=.2cm, bottom=.2cm, right=0cm, left=0cm}
\enlargethispage{20mm}

% \ifbool{K}{K variable contains "true"... this is the K version}{False}
\begin{center}
\setstretch{.85}
\scriptsize
\textsc{TABLEAU DES PRINCIPAUX CHAMPIGNONS COMESTIBLES DE FRANCE} \\
\rule{2cm}{0.4pt} \\ 
\medskip
BASIDIOMYCÈTES \\ 
\end{center}

\begin{center}
\scriptsize
\begin{spacing}{0}
% \ifbool{K}{\small}
% \ifbool{K}{K version - leave tabcolsep alone}{\setlength{\tabcolsep}{.2em} N version. tabcolsep changed to .2em}
% \begin{tabular}{ m{9em} m{.5em} m{7em} m{.5em} m{7em} m{.5em} m{7em}} 
\begin{tabular}{ p{9em} p{.5em} p{8em} p{.5em} p{8em} p{.5em} p{7em}} 
% \arraystretch{0}
                                          & \ff⎧ &                                                    & \ff⎧ &                                           & \ff⎧ & Amanita.                             \\ 
                                          & \ff⎪ &                                                    & \ff⎪ &                                           & \ff⎪ & Lepiota.                             \\ 
                                          & \ff⎪ &                                                    & \ff⎪ &                                           & \ff⎪ & Armilliaria.                         \\  
                                          & \ff⎪ &                                                    & \ff⎪ &                                           & \ff⎪ & Tricholoma.                          \\  
                                          & \ff⎪ &                                                    & \ff⎪ &                                           & \ff⎪ & Clitocybe.                           \\  
                                          & \ff⎪ &                                                    & \ff⎪ &                                           & \ff⎪ & Collybia.                            \\  
                                          & \ff⎪ &                                                    & \ff⎪ & Spores blanches.\dotfill                  & \ff⎨ & Pleurotus.                           \\  
                                          & \ff⎪ &                                                    & \ff⎪ &                                           & \ff⎪ & Hygrophorus.                         \\  
                                          & \ff⎪ &                                                    & \ff⎪ &                                           & \ff⎪ & Cantharellus.                        \\  
                                          & \ff⎪ &                                                    & \ff⎪ &                                           & \ff⎪ & Lactarius.                           \\  
                                          & \ff⎪ & \textit{Agaricinées.}\dotfill                      & \ff⎨ &                                           & \ff⎪ & Russula.                             \\ 
%                                         & \ff⎪ &                                                    & \ff⎪ &                                           & \ff⎩ & Marasmius.                           \\[.5em] 
                                          & \ff⎪ &                                                    & \ff⎪ &                                           & \ff⎩ & Marasmius.                           \\
                                          & \ff⎪ &                                                    & \ff⎪ &                                           & \ff⎧ & Volvaria.                            \\ 
                                          & \ff⎪ &                                                    & \ff⎪ &                                           & \ff⎪ & Pluteus.                             \\ 
\textsc{Hyménomycètes.}                   & \ff⎨ &                                                    & \ff⎪ & Spores roses.\dotfill                     & \ff⎨ & Entoloma.                            \\ 
%                                         & \ff⎪ &                                                    & \ff⎪ &                                           & \ff⎩ & Clitopilus.                          \\[.5em] 
                                          & \ff⎪ &                                                    & \ff⎪ &                                           & \ff⎩ & Clitopilus.                          \\
                                          & \ff⎪ &                                                    & \ff⎪ &                                           & \ff⎧ & Pholiota.                            \\ 
                                          & \ff⎪ &                                                    & \ff⎪ & Spores ocracées.\dotfill                  & \ff⎨ & Cortinarius.                         \\
%                                         & \ff⎪ &                                                    & \ff⎪ &                                           & \ff⎩ & Paxillus.                            \\[.5em]
                                          & \ff⎪ &                                                    & \ff⎪ &                                           & \ff⎩ & Paxillus.                            \\
%                                         & \ff⎪ &                                                    & \ff⎪ & \multicolumn{2}{l}{Spores pourpres.\dotfill}     & Psalliota.                           \\[.5em]
                                          & \ff⎪ &                                                    & \ff⎪ & \multicolumn{2}{l}{Spores pourpres.\dotfill}     & Psalliota.                           \\
%                                         & \ff⎪ &                                                    & \ff⎩ & \multicolumn{2}{l}{Spores noires.\dotfill}       & Coprinus.                            \\[.5em]
                                          & \ff⎪ &                                                    & \ff⎩ & \multicolumn{2}{l}{Spores noires.\dotfill}       & Coprinus.                            \\
                                          & \ff⎪ &                                                    &      &                                           & \ff⎧ & Polyporus.                           \\ 
                                          & \ff⎪ & \multicolumn{3}{l}{\textit{Polyporées.}\dotfill}                                                      & \ff⎨ & Boletus.                             \\ 
%                                         & \ff⎪ &                                                    &      &                                           & \ff⎩ & Fistulina.                           \\[.5em] 
                                          & \ff⎪ &                                                    &      &                                           & \ff⎩ & Fistulina.                           \\
%                                         & \ff⎪ & \multicolumn{4}{l}{\textit{Hydnées.}\dotfill}                                                                & Hydnum.                              \\[.5em] 
                                          & \ff⎪ & \multicolumn{4}{l}{\textit{Hydnées.}\dotfill}                                                                & Hydnum.                              \\
                                          & \ff⎪ & \multicolumn{3}{l}{\textit{Clavariées.}\dotfill}                                                      & \ff⎰ & Sparassis.                           \\
%                                         & \ff⎪ &                                                    &      &                                           & \ff⎱ & Clavaria.                            \\[.5em] 
                                          & \ff⎪ &                                                    &      &                                           & \ff⎱ & Clavaria.                            \\
                                          & \ff⎩ & \multicolumn{4}{l}{\textit{Théléphorées.}\dotfill}                                                           & Craterellus.                         \\[.8em] 
\textsc{Gastéromycètes.}                  &      & \multicolumn{3}{l}{\textit{Lycoperdinées.}\dotfill}                                                          & \makecell[l]{\ff⎰\\\ff⎱} & \makecell[l]{Lycoperdon.\\Bovista.} \\ 
\end{tabular}
\end{spacing}
\end{center}

\begin{center}
\scriptsize
ASCOMYCÈTES \\ 
\end{center}

\begin{center}
\begin{spacing}{0}
\scriptsize
% \ifbool{K}{\small}
% \ifbool{K}{}{\setlength{\tabcolsep}{.2em}}
% \begin{tabular}{ m{9em} m{.5em} m{7em} m{.5em} m{7em} m{.5em} m{7em}} 
\begin{tabular}{ p{9em} p{.5em} p{8em} p{.5em} p{8em} p{.5em} p{7em}} 
                                          &      &                                                    &      &                                           & \ff⎧ & Peziza.                              \\
                                          &      &                                                    &      &                                           & \ff⎪ & Morchella.                           \\
\multicolumn{5}{l}{\textsc{Discomycètes.}\dotfill}                                                                                                       & \ff⎨ & Helvella.                            \\
                                          &      &                                                    &      &                                           & \ff⎪ & Gyromitra.                           \\
                                          &      &                                                    &      &                                           & \ff⎩ & Verpa.                               \\[.5em]
                                          &      &                                                    &      &                                           & \ff⎧ & Tuber.                               \\
\multicolumn{5}{l}{\textsc{Périsporiacées.}\dotfill}                                                                                                     & \ff⎨ & Chœromyces.                          \\
                                          &      &                                                    &      &                                           & \ff⎩ & Terfezia.                            \\
\end{tabular}
\end{spacing}
\end{center}

\restoregeometry
\normalsize
\setstretch{0.9}
\setlength\tabcolsep{0pt} 
\newpage
\section*{\centering Hyménomycètes.}

Les Hyménomycètes comprennent tous les champignons dont les Organes de
fructification sont placés à l'extérieur.

\subsection*{\centering \textit{Agaricinées}.}

Les Agaricinées sont caractérisées par la présence de lames ou feuillets
rayonnants qui adhèrent à la face inférieure du chapeau et portent les spores.
Les spores sont blanches, roses, ocracées, pourpres ou noires ; leur coloration
est à peu près constante dans la même espèce à la maturité, mais elle n'est pas
toujours semblable à celle de l'hymenium qui les contient.

\subsubsection*{\centering \small\sc Agaricinées à spores blanches.}

\paragraph{Amanites.} 

Les Amanites (Amanita) sont des champignons munis d'une volve, enfouie sous
terre le plus souvent, et d'un anneau fixé au tiers supérieur du pied. Ce genre
renferme quelques champignons comestibles excellents, à chair épaisse et ferme,
d'odeur et de saveur agréables.

On appelle \textit{Oronges} les espèces comestibles les plus remarquables et
aussi, quelquefois, des espèces vénéneuses.

Les principales amanites comestibles sont :

l'\textit{Oronge vraie} (A. Cæsarea), ou \textit{Oronge des Césars}, célèbre
depuis l'époque romaine ; beau champignon orangé plus ou moins foncé, pourvu
d'un anneau de même couleur en forme de collerette retombante et d'une volve
blanche en étui à la base du pied ;

l'\textit{Oronge blanche} (A. ovoïdea), ou \textit{Coucoumelle blanche} ;
champignon tout blanc ;

l'\textit{Oronge vineuse} (A. rubescens), ou \textit{Golmotte} ; excellent
champignon rougeâtre ou lie de vin, à chair rouge vineuse ou rougissant
lentement à l'air ;

l'\textit{Amanite en étui} (A. vaginata), ou \textit{Coucoumelle grise},
Grisette ; champignon grisâtre ou roussâtre.

\paragraph{Lépiotes.}

Les Lépiotes (Lepiota) sont des champignons dépourvus de volve, mais munis
d'un anneau souvent mobile ; la surface du chapeau est écailleuse ou pelucheuse.
Ce sont des champignons assez recherchés, à chair blanche ou un peu rousse.
Leur odeur et leur saveur, peu prononcées, sont agréables.

Les principales lépiotes comestibles sont :

la \textit{Lépiote élevée} (L. procera), ou \textit{Grande Coulemelle},
\textit{Grisotte}, appelée encore \textit{Champignon à la bague}. Ce champignon
est un des plus grands qui existent ; il est remarquable par son port élégant
et sa haute et forte structure. C'est un bon champignon, très apprécié ;

la \textit{Lépiote pudique} (L. pudica), ou \textit{Toute blanche} : champignon
très savoureux à odeur agréable de faux mousseron ; ressemble extérieurement
à une pratelle ;

la \textit{Lépiote raboteuse} (L. rhacodes), ou \textit{Coulemelle bâtarde},
dont la chair rougit à l'air ;

la \textit{Lépiote excoriée} (L. excoriata), ou \textit{Petite Coulemelle}.

\paragraph{Armillaires.}

Les Armillaires (Armillaria), de même que les lépiotes, ont un anneau à la
partie supérieure du pied. Ce sont des champignons très nuisibles pour les
arbres.

Deux espèces sont comestibles :

l'\textit{Armillaire de miel} (A. mellea), espèce très commune dans les bois,
mais peu prisée ; comestible seulement lorsque les sujets sont jeunes ;

l'\textit{Armillaire robuste} (A. robusta) qui fournit de nombreuses variétés
dont l’une, à odeur de radis et à saveur piquante, est très recherchée dans les
environs de Nice.

\paragraph{Tricholomes.}

Les Tricholomes (Tricholoma) n'ont ni volve, ni anneau ; ce sont des
champignons à pied épais et robuste. Beaucoup sont comestibles. Leurs
différentes colorations les ont fait classer en tricholomes blancs, bleus,
bruns ou gris et jaunes. Voici les principaux :

le \textit{Tricholome de la Saint-Georges} (T. Georgii, ou \textit{Mousseron
vrai}, l'un des meilleurs parmi les tricholomes ; blanc ou légèrement
jaunâtre ; odeur agréable de farine ;

le \textit{Tricholome colombe} (T. columbetta) ; champignon satiné, difforme ;
odeur agréable ;

le \textit{Tricholome nu} (T. nudum), vulgairement \textit{Petit pied bleu}.
Champignon entièrement violet quand il est jeune. Chair bleue. Comestible
délicat ;

le \textit{Tricholome améthyste} (T. amethystinum) ;

le \textit{Tricholome masqué} (T. personatum) ;

le \textit{Tricholome russule} (T. russula) ; odeur de farine ; saveur agréable ;

le \textit{richolome ruiné} (T. pessundatum) ; odeur de farine ; saveur douce ;

le \textit{Tricholome équestre} (T. equestre), vulgairement nommé
\textit{Chevalier}. Champignon trapu, entièrement jaune. Comestible assez
recherché.

\paragraph{Clitocybes.}

Les Clitocybes (Clitocybe) sont des champignons assez grands, à chapeau mince
creusé en forme d'entonnoir. Quelques espèces sont odorantes et comestibles,
par exemple :

le \textit{Clitocybe nébuleux} (C. nebularis), à odeur de farine ;

le \textit{Clitocybe d'hiver} (C. brumalis), à odeur douce ;

le \textit{Clitocybe vert} (C. viridis), à odeur d'anis ;

le \textit{Clitocybe à odeur douce} (C. suaveolens), à odeur d’anis.

A côté des clitocybes, parmi lesquels il est souvent rangé, il faut mentionner
le \textit{Laccaria vernissé}, appelé quelquefois \textit{Clitocybe vernissé}
(L. laccata), petit champignon mince et fluet, véritable protée par l'aspect
changeant de sa forme et par la diversité de ses couleurs. C’est un bon
comestible, un peu difficile à recueillir à cause de sa petitesse. Sa chair,
violacée, a une odeur agréable.

\paragraph{Collybies.}

Les Collybies (Collybia) sont des champignons à chapeau un peu bosselé et
mamelonné au centre, à pied fibreux, cartilagineux, qu'on peut plier sans le
casser. Ils doivent être récoltés jeunes et sans taches, sans quoi ils sont
coriaces, indigestes, même purgatifs dans certains cas.

Les plus communs sont :

le \textit{Collybie à long pied} (C. longipes) ; chair blanche à odeur de noisette ;

le \textit{Collybie à pied en fuseau} (C. fusipes), sans odeur, mais de saveur agréable ;

le \textit{Collybie à pied velouté} (C. velutipes), chair douce, saveur agréable.

\paragraph{Pleurotes.}

Les Pleurotes (Pleurotus) sont des champignons charnus, rarement terrestres,
parasites sur les herbes, les arbrisseaux et les arbres vivants ou morts. Leur
chapeau est le plus souvent semi-orbiculaire en forme d'éventail ou de
coquille, avec pied latéral court et épais, ou sans pied. Leur chair est
blanche et ferme ; leur odeur et leur saveur sont agréables.

Quelques pleurotes sont de bons comestibles :

le \textit{Pleurote de l'Orme} (P. ulmarius), vulgairement \textit{Oreille d'Orme} ;

le \textit{Pleurote de l'Eryngium} (P. Eryngii), vulgairement \textit{Oreille de
Chardon}. Champignon extrêmement fin et très recherché.

le \textit{Pleurote en forme d'huître} (P. ostreatus), vulgairement \textit{Nouret} ;

le \textit{Pleurote du chéne} (P. dryinus) ;

le \textit{Pleurote du Pommier} (P. Pometi).

\paragraph{Hygrophores.}

Les Hygrophores (Hygrophorus) sont des champignons peu recherchés à cause de la
mollesse de leur chair. Cependant, ils sont agréables à manger ; ils ont une
saveur douce et une odeur rappelant parfois celle des mousserons.

Les espèces comestibles les plus connues sont :

l'\textit{Hygrophore virginal} (H. virgineus) ;

l'\textit{Hygrophore pudibond} (H. pudorinus) ;

l'\textit{Hygrophore des prés} (H. pratensis) ;

l'\textit{Hygrophore rougissant} (H. erubescens).

\paragraph{Chanterelles.}

Les Chanterelles (Cantharellus) ont un chapeau irrégulier creusé en entonnoir,
formé par l'épanouissement du pied, à bord mince, festonné, ondulé ou enroulé ;
la face inférieure est sillonnée de plis et de veines très saillantes. Ces
champignons ont une odeur fine, délicate, rappelant un peu celle du citron.
Leur saveur est âpre et un peu poivrée. Cette âpreté disparait à la cuisson.

Une seule espèce est comestible :

la \textit{Chanterelle comestible} (C. cibarius), vulgairement
\textit{Gyrole} ; beau champignon jaune d’or ou orangé vif, quelquefois blanc
crème.

\paragraph{Lactaires.}

Les Lactaires (Lactarius) sont des champignons robustes, caractérisés par le
suc laiteux blanc, quelquefois jaune ou rouge qu'ils contiennent et qui
s'écoule lorsqu'on les brise. Leur chapeau grand, ferme, charnu, dont le bord
est roulé en dessous, est plus ou moins déprimé au milieu. Cette famille
renferme beaucoup d'espèces dont quelques-unes sont comestibles. On peut citer
:

le \textit{Lactaire délicieux} (L. deliciosus), appelé vulgairement
\textit{Vache rouge}, à chair âcre, dont l’âcreté disparaît à la cuisson ; lait
orangé ;

le \textit{Lactaire à lait abondant} (L. lactifluus), vulgairement
\textit{Vache}, \textit{Vachotte}, qui contient un lait blanc, abondant et
doux ;

le \textit{Lactaire doux} (L. subdulcis) ;

le \textit{Lactaire très doux} (L. mitissimus).

\paragraph{Russules.}

Les Russules (Russula) sont des champignons fermes, mais cassants, à port
particulier, dont le chapeau présente les couleurs les plus variées, souvent
très vives.

Les russules sont très nombreuses ; les espèces comestibles ont une saveur
douce. Les meilleures sont :

la \textit{Russule sans lait} (R. delica), ou \textit{Éléphantine},
vulgairement \textit{Prévat}, \textit{Pigeon blanc}. Elle ressemble plus à un
lactaire qu'à une russule ; elle a une saveur aromatique un peu résineuse ;

la \textit{Russule verdoyante} (R. virescens), vulgairement \textit{Palomet}.
Champignon verdâtre, impossible à confondre avec aucun autre champignon.
Comestible agréable, surtout lorsqu'il est jeune ;

la \textit{Russule bleu-jaunâtre} (R. cyanoxantha), vulgairement
\textit{Charbonnier} ; saveur douce ;

la \textit{Russule à feuillets inégaux} (R. hétérophylla), vulgairement
\textit{Bisotte} ;

la \textit{Russule jolie} (R. lepida) ; saveur de noisette,

\paragraph{Marasmes.}

Les Marasmes (Marasmius) sont de petits champignons peu charnus, généralement
coriaces, de couleur blanc crème ou café au lait.

Une seule espèce comestible à citer :

le \textit{Marasme d'Oréade} (M. Oreades), vulgairement \textit{Faux
Mousseron}, \textit{Nymphe des Montagnes}. Chair blanche, douce, d'odeur
agréable augmentant par la dessiccation. Ce champignon se dessèche facilement
sans pourrir et se conserve très bien.

\subsubsection*{\centering \small\sc Agaricinées à spores roses.}

\paragraph{Volvaires.}

Les Volvaires (Volvaria) sont des champignons ayant une volve, mais pas
d'anneau. Ils se rapprochent beaucoup des Amanites par leur aspect général.

Une seule espèce est comestible :

la \textit{Volvaire soyeuse} (V. bombycina), ayant une volve très grande,
jaune. Peu d'odeur ; saveur agréable.

\paragraph{Pluteus.}

Les Pluteus n'ont ni volve, ni anneau.
Une espèce est comestible :
le \textit{Pluteus couleur de cerf} (P. cervinus).

\paragraph{Entolomes}.

Les Entolomes (Entoloma) sont des champignons charnus, à port de tricholomes.

La seule espèce comestible est :

l'\textit{Entolome en bouclier} (E. clypeatum), champignon soyeux, vendu sur
certains marchés sous le nom de \textit{Mousseron gris}.

\paragraph{Clitopiles}.

Les Clitopiles (Clitopilus) sont des champignons dont l'aspect rappelle un peu
celui des Chanterelles. Leur chapeau, charnu, se continue avec le pied.

Ils sont représentés par une espèce comestible :

le \textit{Clitopile petite prune} (C. prunulus), vulgairement
\textit{Mousseron d'Automne}, \textit{Oreille de Meunier}. Très bon champignon
à chair blanche et ferme, à odeur de farine fraîche.

\subsubsection*{\centering \small\sc Agaricinées à spores ocracées.}

\paragraph{Pholiotes}.

Les Pholiotes (Pholiota) sont des champignons épais, charnus, ayant un
véritable anneau membraneux à la partie supérieure du pied. Ils ressemblent aux
armillaires.

Deux espèces méritent d'être mentionnées :

le \textit{Pholiote du Peuplier} (P. ægerita), champignon couleur abricot,
recouvert d'un léger voile soyeux blanc jaunâtre. Célèbre et très estimé dans
le Midi, où il est connu sous le nom de \textit{Pivoulade} :

le \textit{Pholiote précoce} (P. precox).

\paragraph{Cortinaires}.

Les Cortinaires (Cortinarius) sont caractérisés par la présence d'une cortine et
par la couleur de leurs feuillets. Ils sont extrêmement nombreux et variés. Peu
sont comestibles. Ils doivent être récoltés jeunes.

Le plus intéressant est :

le \textit{Cortinaire violet cendré} (G. violaceocinereus).

\paragraph{Paxilles.}

Les Paxilles (Paxillus) sont de gros champignons à chapeau convexe d'abord,
puis concave au centre, dont les bords sont fortement enroulés dans le jeune
âge ; à feuillets se séparant très facilement des chapeaux.

Une seule espèce mérite l'attention au point de vue culinaire :

le \textit{Paxille à bord enroulé} (P. involutus), d'odeur et de saveur
agréables. Comestible seulement quand il est jeune.

\subsubsection*{\centering \small\sc Agaricinées à spores pourpres.}

\paragraph{Psalliotes}.

Les Psalliotes ou Pratelles (Psalliota), parmi lesquels prend place le
\textit{Champignon de couche}, qui est la variété cultivée de la Pratelle
champêtre, sont des champignons délicats. Leur chapeau, presque globuleux dans
le jeune âge, est généralement de couleur blanche ou blanchâtre ; les lames,
libres, sont d'abord blanches, puis roses, enfin brun pourpre à la maturité. Le
pied, plein, ferme, qui se détache facilement du chapeau, est muni vers le
milieu de sa hauteur d'un anneau simple ou double. Le parfum des psalliotes est
d'une grande finesse ; leur saveur est exquise.

Plusieurs espèces de psalliotes sont intéressantes ; les plus répandues sont :

le \textit{Psalliote des champs} (P. campestris), vulgairement
\textit{Champignon rose des champs} ;

le \textit{Psalliote des jachères} (P. arvensis), vulgairement \textit{Boule de
neige} ;

le \textit{Psalliote des prés} (P. pratensis).

Le psalliote ou pratelle des champs présente de nombreuses variétés, toutes
comestibles. Il doit être récollé avec grande attention, car on peut le
confondre facilement avec des champignons mortels ; l'\textit{Amanite
phalloïde} et l'\textit{Amanite citrine}.

\subsubsection*{\centering \small\sc Agaricinées à spores noires.}

\paragraph{Coprins}.

Les Coprins (Coprinus) sont des champignons éphémères, remarquables par
l'élégance grêle de leur port. Ils sont constitués par un chapeau presque
membraneux, assez grand, cylindrique ou ovoïde dans le jeune âge, dont la
surface est parsemée de larges écailles filamenteuses ou de peluches soyeuses
formées par l'épiderme et disposées assez régulièrement. Le pied, brillant, est
muni d'un anneau mince, mobile. Les coprins sont blancs.

Une seule espèce est comestible :

le \textit{Coprin à chevelure} (C. comatus). Très fin, mais bon seulement
lorsqu'il est jeune, très frais et encore tout blanc, car il se désagrège
rapidement et se liquéfie.

\subsection*{\centering \textit{Polyporée}.}

Les Polyporées comprennent les champignons dont l'hymenium est formé par
des tubes rangés verticalement sous le chapeau.

\paragraph{Polypores.}

Les Polypores (Polyporus) comestibles sont des champignons charnus qui poussent
sur le bois vivant ou mort. Ils sont remarquables par l'élégance des formes que
présentent certaines espèces. La face inférieure de leurs chapeaux est tapissée
de tubes faisant corps avec la chair du chapeau, soudés entre eux et dont les
orifices affectent plus ou moins la forme des alvéoles d'une ruche.

Les polypores doivent être récoltés jeunes. À ce moment, leur chair est blanche
et assez tendre ; leur odeur est celle de bon champignon et leur saveur est
agréable. Plus vieux, ils sont durs et leur odeur devient forte.

Les principaux polypores comestibles sont :

le \textit{Polypore feuillé} (P. frondosus), vulgairement \textit{Poule des
bois} ;

le \textit{Polypore en râpe} (P. scobinaceus), vulgairement \textit{Pied de
mouton noir} ;

le \textit{Polypore des brebis} (P. ovinus) : odeur agréable.

\paragraph{Bolets}.

Les Bolets ou Cèpes (Boletus) sont des champignons de couleurs variées, épais,
charnus, à chapeau globuleux dans le jeune âge, plus ou moins étalé au fur et
à mesure du développement du champignon. Leur caractéristique est la présence
de tubes à la face inférieure du chapeau, dont on peut les séparer facilement.
Le pied central, plein, ferme, charnu, est généralement renflé au milieu ou
à son extrémité inférieure ; sa couleur diffère le plus souvent de celle du
chapeau. La chair des bolets est blanche ou jaune, ferme et moelleuse. La
saveur et l'odeur, agréables, peuvent servir quelquefois à distinguer certaines
espèces. Ce sont des champignons excellents et très recherchés.

Parmi les bolets comestibles, on peut citer :

le \textit{Bolet comestible} (B. edulis), appelé vulgairement \textit{Cèpe de
Bordeaux}, \textit{Champignon polonais} ;

le \textit{Bolet bronzé} (B. æreus), vulgairement \textit{Cèpe noir},
\textit{Tête de nègre} ;

le \textit{Bolet d'été} (B. æstivalis), vulgairement \textit{Cèpe d'été} ;

Ces trois espèces sont exquises.

le \textit{Bolet bai-brun} (B. badius) ; chair douce ;

le \textit{Bolet raboteux} (B. scaber), vulgairement \textit{Roussile}. Doit
être récolté jeune ;

le \textit{Bolet des bœufs} (B. bovinus) ; chair douce.

\paragraph{Fistulines.}

Les Fistulines (Fistulina) sont des champignons sans forme bien déterminée,
présentant une masse volumineuse, épaisse, charnue, entière ou échancrée, d'un
diamètre de {\ppp20\mmm} centimètres environ. Ils ont l'aspect d'une langue ou
d'un foie supporté par un pied latéral, gros et court. Leur couleur est rouge
sang ou rouge brun. Leur chair est rougeâtre, succulente, marbrée de blanc.
Leur saveur est aigrelette.

Ce genre de champignons est représenté par une seule espèce :

la \textit{Fistuline foie} (F. hepatica), vulgairement \textit{Langue de bœuf},
\textit{Foie de bœuf.}

La fistuline doit être consommée jeune ; vieille, elle est de digestion difficile.

\subsection*{\centering \textit{Hydnées.}} 

Les Hydnées sont caractérisées par la présence d'aiguillons, de dents, de
soies, de tubercules formés par l'hymenium, à la face inférieure du chapeau.

\paragraph{Hydnes.}

Les Hydnes (Hydnum) sont des champignons à chapeau de forme variable,
mais ayant fréquemment celle d'un parasol plus ou moins irrégulier, dont le
dessous est recouvert d'aiguillons. Les hydnes sont de couleurs variées. Leur
odeur est nulle ou peu prononcée. Leur saveur, un peu acerbe, est amère et
poivrée. Cette amertume et cette âcreté disparaissent à la cuisson.

Plusieurs hydnes sont comestibles. Les plus communs sont :

l'\textit{Hydne bosselé} (H. repandum), vulgairement \textit{Pied de mouton
blanc} ;

l'\textit{Hydne à écailles imbriquées} (H. imbricatum).

\subsection*{\centering \textit{Clavariées.}} 

Les Clavariées sont des champignons d'assez grande taille, en forme de petits
arbres, de colonnes, de massues. Ils sont le plus souvent ramifiés à la manière
des coraux ou des choux-fleurs.

\paragraph{Sparassis.}

Les Sparassis (Sparassis) sont des champignons fragiles, très rameux, à rameaux
aplatis, ayant quelque peu l'apparence d'une éponge.

Deux espèces, très délicates, malheureusement assez rares, existent en France :

le \textit{Sparassis crépu} (S. crispa) ;

le \textit{Sparassis lamelleux} (S. laminosa).

\paragraph{Clavaires.}

Les Clavaires (Clavaria) sont des champignons charnus, fragiles, généralement
très ramifiés, à rameaux arrondis, de teintes variées ; ils forment des touffes
d’un poids assez considérable.

Beaucoup de clavaires sont comestibles ; les plus estimées sont :

la \textit{Clavaire jaune} (C. flava), vulgairement \textit{Barbe de Chéne},
\textit{Mainotte}  ; saveur agréable ;

la \textit{Clavaire belle} (C. formosa) ;

la \textit{Clavaire à pointes pourpres} (C. acroporphyrea).

\subsection*{\centering \textit{Téléphorées}.}

Les Téléphorées comprennent des champignons particuliers ayant une forme
irrégulière de corne d'abondance ou de coupe supportée par un pied.

\paragraph{Craterelles.}

Les Craterelles (Craterellus) sont des champignons minces ayant la forme d'une
trompette ou d'un entonnoir à bord retourné et festonné, dont le creux se
continue en tube dans le pied. Par leur aspect, ils rappellent les
chanterelles, mais ils en diffèrent par la couleur qui est ici gris cendré.

Trois espèces de craterelles sont comestibles : la plus répandue est :

la \textit{Craterelle corne d’abondance} (C. cornucopioides), vulgairement
nommée \textit{Corne d'abondance}, \textit{Trompette des morts}. Bon comestible
ayant un peu la saveur de la truffe ; se conserve très bien par simple
dessiccation.

\section*{\centering Gastéromycètes.}

Les Gastéromycètes sont caractérisés par la formation des spores dans une
enveloppe entièrement close qui ne leur donne issue que par sa déchirure ou par
sa destruction complète. Ils diffèrent en cela des Hyménomycètes, dont la
fructification est externe.

\subsection*{\centering \textit{Lycoperdinées}.}

Les Lycoperdinées comprennent des champignons ayant une enveloppe persistante
et un hymenium d'abord charnu, se réduisant ensuite en une sorte de poussière.

\paragraph{Lycoperdons.}

Les Lycoperdons ou Vesses de Loup (Lycoperdon) sont des champignons creux, en
forme de toupie arrondie ou allongés en poire. Ils ont une enveloppe mince qui
se détruit complètement ou s'ouvre seulement par un pore à sa partie supérieure
pour donner issue aux spores. Les lycoperdons sont blancs ou de couleur
claire ;  leur chair est blanche et tendre dans le jeune âge ; plus tard, elle
verdit et enfin se convertit en poussière. Certains Iycoperdons atteignent
parfois de grandes dimensions.

Les lycoperdons sont comestibles lorsqu'ils sont jeunes. Les plus communs
sont :

le \textit{Lycoperdon en forme de matras} (L. excipuliforme), pouvant atteindre
jusqu'à {\ppp20\mmm} centimètres de hauteur ;

le \textit{Lycoperdon d'hiver} (L. hiemale) :

le \textit{Lycoperdon hérissé de pierreries} (L. gemmatum) :

le \textit{Lycoperdon en forme de poire} (L. piriforme).

\paragraph{Bovista.}

Les Bovista (Bovista) sont des champignons ayant une grande analogie avec
les lycoperdons dont ils ne diffèrent guère extérieurement que par l'absence du
pied et par la présence d'un voile qui les enveloppe.

Ce sont des champignons peu communs, dont le moins rare est :

le \textit{Bovista gigantesque} (B. gigantea), vulgairement
\textit{Vesse-loup}. Très grosse espèce, pouvant atteindre {\ppp20\mmm}
à {\ppp30\mmm} centimètres de hauteur et peser {\ppp4\mmm} à {\ppp5\mmm}
kilogrammes. Comestible lorsqu'il est jeune, alors que sa chair est blanche et
ferme.

\section*{\centering Discomycètes.}

Les Discomycètes comprennent tous les champignons dont les fructifications ont
une forme de disque ou de coupe.

\paragraph{Pézizes.}

Les Pézizes (Peziza) sont des champignons fragiles, ayant généralement la forme
d'une coupe plus ou moins régulière, avec ou sans pied, à face supérieure
concave, lisse ou parfois veinée, à face inférieure lisse ou pelucheuse. Les
pézizes ont des couleurs très vives : presque toute la gamme du spectre y est
représentée, mais le rouge domine et la coloration est différente pour les deux
faces. Toutes les pézizes peuvent être consommées sans inconvénient ; ce sont
de bons comestibles ayant une consistance un peu cireuse et poussant en toutes
saisons.

Les principales espèces comestibles sont ;

la \textit{Pézise veinée} (P. venosa) ;

la \textit{Pézize en coupe} (P. acetabulum) ;

la \textit{Pézize vésiculeuse} (P. vesiculosa) ;

la \textit{Pézize blanc noir} (P. leucomelas) ;

la \textit{Pézize oreille d'âne} (P. onotica).

\paragraph{Morilles.}

Les Morilles (Morchella) sont des champignons constitués par un chapeau creux
plus ou moins régulièrement conique, quelquefois arrondi, dont la surface
externe présente des anfractuosités profondes séparées les unes des autres par
des côtes élevées, et un pied creux, dont la cavité se continue sans ligne de
démarcation avec celle du chapeau. Ce chapeau est jaunâtre, roux, brun ou
gris ; sa couleur est toujours plus foncée au fond des alvéoles que sur les
côtes ; le pied est blanc. Leur parfum, qui ne se retrouve dans aucune autre
espèce de champignons, est extrêmement agréable et leur saveur est exquise.
C'est un comestible très recherché des amateurs, dont plus d'un le préfère à la
truffe. Les morilles apparaissent au printemps, à la fin de mars, en avril, au
commencement de mai. Elles sont abondantes dans les années à printemps doux et
après la chute de pluie tiède.

Trois espèces de morilles existent en France :

la \textit{Morille comestible} (M. esculenta), la plus estimée de toutes ;

la \textit{Morille conique} (M. conica) :

la \textit{Morille à pied ridé} (M. rimosipes), appelée encore \textit{Morille
semi-libre}, vulgairement \textit{Morillon}. Espèce moins fine.

\paragraph{Helvelles.}

Les Helvelles (Helvella) sont des champignons fragiles, demi-transparents, de
formes bizarres (croissant, mître, cornette). Ils sont constitués par un
chapeau lisse formé de plusieurs lobes contournés, pliés, ondulés et comme
chiffonnés, dressés ou rabattus ; le pied est blanc ou roussâtre, souvent
cannelé et lacuneux. Leur parfum est moins délicat que celui des morilles, mais
leur saveur est très fine.

Les helvelles comestibles les plus communes sont :

l'\textit{Helvelle crépue} (H. crispa), vulgairement \textit{Oreille de Chat} ;

l'\textit{Helvelle lacuneuse} (H. lacunosa), vulgairement \textit{Oreille de
Judas} ;

l'\textit{Helvelle élastique} (H. elastica).

Les helvelles ne doivent être consommées que lorsqu'elles sont très fraîches,
car elles deviennent facilement toxiques.

\paragraph{Gyromitres.}

Les Gyromitres (Gyromitra) sont des champignons à tête arrondie, irrégulière,
formée de côtes et creusée de sillons plus ou moins contournés qui rappellent
assez bien les circonvolutions d'un cerveau.

Les gyromitres sont représentées par une espèce :

le \textit{Gyromitre comestible} (G. esculenta), à odeur et à saveur très
agréables. On le trouve à Paris à l'état sec.

\paragraph{Verpa.}

Les Verpa (Verpa) sont des champignons dont le chapeau, plus ou moins conique,
a la forme d'un dé recouvrant le pied à sa partie supérieure.

Ils sont représentés par l'espèce coméstible suivante :

le \textit{Verpa en forme de dé} (V. digitaliformis).

\section*{\centering Périsporiacées.}

Les Périsporiacées sont earactérisées par des fructifications membraneuses ou
charnues dont les asques ne communiquent pas avec le milieu intérieur.

\paragraph{Truffes.}

Les Truffes (Tuber) sont des champignons souterrains, charnus et indéhiscents,
à mycelium invisible. Leur surface est habituellement couverte de rugosités ;
quelquefois elle est lisse. Ces rugosités sont formées par des verrues
prismatiques, polyédriques ou coniques, plus ou moins larges et saillantes, qui
donnent aux truffes leur aspect grenu, chagriné. Ces champignons sont
caractérisés par la présence, dans leur intérieur, de veines rameuses
anastomosées de façon à former un réseau donnant à leur chair sectionnée une
apparence marbrée.

La chair des truffes est blanche dans le jeune âge et presque sans parfum, puis
elle devient brune, noire ou rousse par suite de la formation et de la maturité des
organes reproducteurs. C'est à cette dernière période de leur développement que
les truffes ont acquis tout leur parfum, qui est variable avec les espèces.

Les truffes eroissent le plus communément dans les terrains
calcaires\footnote{On a signalé, il ya longtemps déjà, la concomitance des
truffes et du phosphate de chaux dans les départements du Lot, de la Meuse et
des Ardennes, J'ai moi-même constaté dans le Gard un autre exemple de
concomitance végéto-minérale. En faisant des recherches de minerai de zinc,
j'ai observé que les amas calaminaires étaient fréquemment jalonnés par des
gisements truffiers et l'analyse de ces truffes y a décelé des traces de zinc.
Peut-être ne faut-il voir là qu'une simple coïncidence due à la sympathie de la
truffe pour le calcaire qui constitue la roche encaissante des amas de
calamine, et il serait sans doute excessif de tirer de cette remarque sans
prétention de savantes conclusions sur la genèse de la truffe et sur celle du
zinc. Il n'en est pas moins vrai que le fait est curieux ; c'est pourquoi je le
mentionne.} et principalement dans les bois de chênes et de châtaigniers, ou
dans leur voisinage, à une profondeur variant entre {\ppp10\mmm} et
{\ppp15\mmm} centimètres ; mais on les trouve aussi dans les terrains légers et
friables, sous les noisetiers, les charmes, les hêtres, les pins, les
genévriers, etc. On les rencontre souvent par groupes de {\ppp10\mmm}
à {\ppp20\mmm} individus ; d'autres fois elles sont isolées ou réunies par
trois ou quatre seulement.

Les truffes mettent presque un an à se développer. Ce n'est qu'en novembre et
en décembre qu'elles ont acquis le maximum de leurs qualités. Elles existent
dans presque toutes les contrées du monde, mais les meilleures sont
incontestablement celles qui poussent en France et en tête desquelles prend
place la Truffe du Périgord, la « reine des truffes ».

Toutes les truffes sont comestibles ou du moins inoffensives.

On peut grouper les truffes en deux classes : a) les truffes à surface
verruqueuse, chagrinée ; b) les truffes à surface lisse ou légèrement
papilleuse.

\begin{center}
A. — \textit{Truffes à surface verruqueuse, chagrinée} :
\end{center}

la \textit{Truffe à spores noires} (T. melanosporum). C'est la Truffe noire du
Périgord, espèce la plus appréciée, dont l'odeur et la saveur exquises font les
délices des gourmands. Cette truffe, parasite du chêne pubescent (Quercus
pubescens), est récoltée non seulement dans le Périgord, mais aussi dans le
Quercy, le Dauphiné, la Provence, l'Angoumois et le Poitou. Les centres de
production les plus renommés sont, par ordre alphabétique, Cahors, Martel,
Nérac, Romans, Sarlat, Sorges, Soumane-en-Vaucluse, Thénon ;

la \textit{Truffe noire} (T. cibarium}). Cette espèce, très commune en France
et très recherchée, présente parfois d'autres nuances qui ont permis d'établir
les variétés rougeâtre, violette, blonde et même blanche qui auraient, au dire
des amateurs, une saveur et un parfum différents plus ou moins appréciés. La
truffe noire est récoltée dans le Languedoc, le Périgord, le Dauphiné et
l'Angoumois ;

la \textit{Truffe d'hiver} (T. brumale). Cette espèce, un peu moins parfumée
que les précédentes, existe surtout en Bourgogne, en Champagne et en Lorraine,
où elle est désignée sous le nom de \textit{Truffe violette} ;

la \textit{Truffe d'été} (T. æstivum). Espèce moins estimée. Pousse dans le
Midi et l'Est de la France ;

la \textit{Truffe des montagnes} (T. montanum). Se rapproche comme goût de la
truffe du Périgord ;

la \textit{Truffe à veines plissées} (T. mesentericum). Truffe de second ordre
à odeur forte. Pousse dans le Midi, le Centre et l'Est de la France et aux
environs de Paris ;

la \textit{Truffe crochue} (T. incinatum). Truffe de second ordre, connue sous
le nom de \textit{Truffe rouge}, la plus répandue en France après la truffe
noire. On la trouve dans la chaîne calcaire qui va de la Côte-d'Or à l'Aube en
passant par la Haute-Marne.

\begin{center}
B. — \textit{Truffes à surface lisse ou légèrement papilleuse :}
\end{center}

Dans ce groupe prennent place les truffes grises, jaunâtres, jaunes ou rousses
du Poitou et de la Provence, la truffe jaune de Bourgogne, de Lorraine et de la
Franche-Comté, peu comestibles. Mais la plus importante des truffes à surface
lisse est incontestablement :

la Truffe grise\footnote{Très commune et très recherchée en Italie, où elle est
connue sous le nom de \textit{Truffe du Piémont}.} (T. magnatum), dont l'odeur
et la saveur tiennent à la fois de l'ail, de l'oignon et du fromage. Cette
truffe, appelée encore \textit{Truffe à l'ail}, assez rare en France, où on ne
la trouve guère qu'en Provence et aux environs d'Avignon, a ses détracteurs
malgré sa réputation assez grande. En réalité, elle demande une certaine
accoutumance, mais elle ne manque pas de charme quand on y est fait\footnote{On
donne à tort le nom de truffes à certains autres champignons rappelant plus ou
moins celles-ci par leur forme et la coloration de leur chair. Elles
appartiennent à une autre famille, celle des Hyménogastrées ; par exemple :

le \textit{Mélanogaster panaché} (M. variegatus), appelé aussi \textit{Truffe
musquée}.}.

\medskip

On a reproché à la truffe bien des méfaits : on l'a accusée d'être
indigeste\footnote{Les truffes paraissent indigestes surlout lorsqu'elles
accompagnent des crustacés, des foïes gras, des gibiers avancés, mais alors ce
sont ces derniers éléments qui sont les véritables coupables.}, de provoquer
l'insomnie, la goutte. Que sais-je encore ? En réalité, les truffes de bonne
qualité sont nutritives et elles sont au moins aussi digestibles que la viande,
à condition de les mâcher. Brillat-Savarin a très finement rendu hommage
à leurs vertus spéciales en disant qu'elles pouvaient, en certaines
circonstances, rendre les femmes plus tendres et les hommes plus aimables ;
cela seul devrait suffire à les réhabiliter auprès de leurs rares contempteurs.

\paragraph{Chæromyces.}

A côté des truffes proprement dites, on peut placer une espèce souterraine
intéressante mais assez rare, poussant surtout dans les sables du Nord de
l'Afrique :

le \textit{Chœromyce à méandres} (C. meandriformis), appelé aussi \textit{Truffe
blanche}, à chair blanche ou roux clair sillonnée de veines nombreuses jaune
d’ocre, dont la saveur rappelle celle de la truffe noire.

\paragraph{Terfezia.}

Les Terfezia (Terfezia), nommés vulgairement \textit{Terfez}, ou
\textit{Terfex}, sont des champignons souterrains, charnus, indéhiscents,
à chair blanchâtre, brunissant ensuite, et à mycelium apparent. Ils sont plus
ou moins piriformes ; ils ont peu de parfum, mais une saveur douce comparable
à celle du mousseron. Contrairement aux truffes, on les trouve à la surface ou
près de la surface des terrains arénacés plus ou moins calcaires et dans le
voisinage de petits arbres appelés \textit{Cistes}.

Ils sont représentés en France par une seule espèce :

le \textit{Terfezia du Lion} (T. Leonis), rare en France où on ne le trouve que
dans les départements des Alpes-Maritimes, du Var et en Corse : abondant en
Algérie, surtout après les pluies.

Ce sont probablement les seules truffes connues des Anciens, celles dont
parlent Dioscoride, Théophraste et Pline. Elles venaient de Carthage, de Lesbos
et des déserts de Libye.

Les terfezia portent les noms suivants : \textit{Terfas} en Afrique,
\textit{Kamès} en Asie, \textit{Turma en Espagne}, \textit{Tartufo bianco} en
Sicile.

Les terfezia, dont la variété la plus grosse est le \textit{Tirmania Africana},
occupent sur le globe une surface autrement considérable que les truffes noires
et ils constituent pour de nombreuses populations une véritable ressource
alimentaire. Simplement séchés, ils se conservent pendant des années. On n'en
voit guère encore sur nos marchés, mais il est probable qu'ils finiront par
y être transportés en plus grande quantité.
