\index{Considérations générales sur les principaux procédés de cuisson des aliments
Leur classification et quelques définitions}
 
\sk

Les différents procédés de cuisson des aliments peuvent être divisés en deux
grandes classes : Les procédés par voie sèche et les procédés par voie humide.

1° \textit{Procédés par voie sèche. —} Comme type de cette classe, je citerai
le procédé de cuisson des viandes employé par les anciens,
\hyperlink{p0013}{p. \pageref{pg0013}} et celui de la cuisson des pommes de terre
sous la cendre.

L'inconvénient de la cuisson absolument à sec est de brûler inévitablement plus
ou moins la partie superficielle des aliments : aussi l'idée de graisser ou de
barder les substances à cuire vint-elle bientôt à l'esprit.

Dans la cuisine moderne, les procédés ressortissant à cette classe comprennent
\textit{le rôtissage, le grillage et la cuisson en croûte.}

\index{Définition des rôtis}
\index{Rôtis, (Définition des)}
On prépare les rôtis à la broche ou dans un vase quelconque non clos, et la
cuisson s'opère à l'air libre ou au four.

Le combustible de choix pour les rôtis à la broche, à l'air libre ou au four,
est le bois : pour le petit gibier, rien ne vaut le sarment de vigne.

\index{Définition des grillades}
\index{Grillades, (Définition des)}
Les grillades, comme leur nom l'indique, sont préparées sur des grils ; c'est au
charbon de bois ou à la braise que l'opération réussit le mieux.

Il est d’une importance capitale d'éviter la fumée produite par les graisses
tombant dans le feu qui impressionne désagréablement la grillade, et, pour
concentrer à l'intérieur le jus des viandes, il est essentiel de commencer par
les saisir, sans les brûler, à une température aussi élevée que possible, soit
vers {\ppp120\mmm}° ; on achève ensuite la cuisson à une température
inférieure, vers {\ppp70\mmm}° environ.

La cuisson en croûte est faite à l'air chaud, au four.

2° \textit{Procédés par voie humide. —} Les procédés par voie humide
comprennent l'ébullition, la cuisson lente directement dans un liquide ou au
bain-marie, le pochage, la confection des différents ragoûts et celle des
fritures.

\index{Bouillis, (Définition des)}
\index{Bouillons, (Définition des)}
\index{Définition des bouillis}
\index{Définition des bouillons}
\textit{a}) On dit qu'une substance est \textit{bouillie} lorsque, après avoir
été mise dans un liquide à la température ambiante, elle cuit dans ce liquide
porté progressivement à la température d'ébullition qu'on maintient jusqu'à la
fin de l'opération. La viande prend alors le nom de \textit{bouilli}.

\index{Consommés (Définition des)}
\index{Définition des consommés}
Quelle que soit la substance bouillie, le liquide de cuisson se transforme en
\textit{bouillon}. Un bouillon concentré prend le nom de consommé.

\index{Courts-bouillons (Définition des)}
\index{Définition des court-bouillons}
Les courts-bouillons\footnote{On désigne ainsi des mélanges liquides à base de
vin et d'eau plus ou moins vinaigrée, assaisonnés et aromatisés avec sel,
poivre en grains, oignons, carottes, persil, thym, laurier, etc, qu'on fait
bouillir avant de s'en servir. Les substances cuites dedans ultérieurement
(poissons, crustacés, mollusques) n'y séjournent généralement que peu de temps,
pendant de « courts bouillons » ; d’où le nom donné à ces préparations.} sont
généralement réservés à la cuisson des poissons, des crustacés et des
mollusques.

\index{Blanchir}
\index{Blanchir (Définition du mot)}
\index{Définition du mot blanchir au sens culinaire}

\textit{Blanchir} une substance alimentaire, c'est la faire cuire
incomplètement dans un liquide. Les substances à blanchir sont d'abord lavées
ou mise à dégorger dans de l'eau froide, puis plongées dans un liquide clair
dont la température est progressivement élevée jusqu'à {\ppp100\mmm}° ; enfin leur
cuisson est achevée dans des jus. Cette opération s'applique surtout aux
légumes, elle a pour but de leur enlever leur âcreté et leur amertume : on
l'emploie aussi pour les issues d'animaux de boucherie et de porc, ainsi que
pour les foies de volaille, les crêtes et les rognons de coq.

\index{Pocher, définition}
\index{Définition du mot pocher}
\textit{b}) Étymologiquement, le mot \textit{pocher} signifie mettre en poche.
Au sens culinaire, il est employé surtout pour exprimer l'action qui consiste
à faire cuire des œufs dans un liquide bouillant : l'albumine du blanc, en se
coagulant, enveloppe le jaune d'une véritable poche. Les œufs à la coque sont
des œufs pochés en coquilles.

Mais, d'une façon générale, toutes les fois qu'on met un corps contenant de
l'albumine dans un liquide suffisamment chaud, cette albumine se coagule à la
surface ; elle protège de cette façon l'intérieur de la substance et la
préparation prend un goût particulier qui mérite une désignation spéciale.
Lorsqu'une farce, un poisson, des filets de volaille sont cuits dans un liquide
clair chaud, on dit bien qu'ils sont pochés, mais on n’'emploie guère ce terme
quand il s'agit de viande de boucherie. C'est ainsi que l'on désigne à tort,
sous le nom de gigot bouilli, le gigot dit aussi « à l'anglaise », qui est cuit
dans de l’eau ou dans du bouillon bouillant. Il me paraît indispensable de
différencier nettement les deux modes de cuisson, suivant que la substance
à cuire est mise dans un liquide froid, qu'on porte ultérieurement
à l'ébullition, comme dans la préparation du pot-au-feu, ou qu'elle est mise
directement dans un liquide à une température suffisante pour coaguler aussitôt
l'albumine à la surface, comme dans la préparation du gigot à l'anglaise et
dans celle du filet de bœuf poché aux tomates,
\hyperlink{p0464}{p. \pageref{pg0464}}. Dans le premier cas, il est légitime de
dire que la substance est \textit{bouillie} ; dans le second, il me semble
absolument logique de la qualifier de \textit{pochée}.

\index{Définition des ragoûts}
\index{Ragoûts (Définition des)}
\textit{c}) Les \textit{ragoûts} sont des mets appétissants dont les éléments
solides sont cuits plus ou moins dans une sauce.

Ils peuvent être divisés en six groupes : les ragoûts proprement dits, les
sautés, les salmis, les braisés, les fricassées et les salpicons.

\index{Civets (Définition des)}
\index{Définition des civets}
\index{Définition des gibelottes}
\index{Gibelottes (Définition des)}
\index{Définition des matelotes} 
\index{Matelotes (Définition des)}

Dans les \textit{ragoûts proprement dits}, dont le type est le ragoût de
mouton, on commence par faire revenir la viande, c'est-à-dire qu'on la fait
chauffer dans un vase découvert avec un corps gras, de façon à lui faire
prendre couleur avant de la mettre à cuire dans la sauce, en vase
incomplètement clos. Parmi les ragoûts proprement dits on range la
\textit{gibelotte}, qui est un ragoût de lapin,
\hyperlink{p0648}{p. \pageref{pg0648}} ; les \textit{civets}, qui sont des ragoûts
avec sauce au vin liée au sang, dont le plus connu est le civet de lièvre, mais
qu'on peut faire aussi avec d'autres viandes (voir le civet
de canard, \hyperlink{p0588}{p. \pageref{pg0588}} et même avec du poisson (voir le
civet de lamproie, \hyperlink{p0341}{p. \pageref{pg0341}} ; et la
\textit{matelote blanche}, qui est un ragoût de poisson, dont la sauce au vin
est liée au beurre manié avec de la farine.

\index{Définition des sautés}
\index{Sautés (Définition des)}
Les \textit{sautés} de viande peuvent être considérés comme des ragoûts,
à sauce plus ou moins courte, de viandes tendres coupées en petits morceaux ou
naturellement de petites dimensions, nécessitant moins de cuisson que les
autres. On prépare en sautés les grenouilles, les rognons, les filets mignons,
le poulet, le veau, etc. On commence par les faire revenir, en les faisant
sauter avec un corps gras, dans une casserole basse, ouverte, munie d'un long
manche et portant le nom de sauteuse ; puis on achève leur cuisson dans un
mouillement plus ou moins abondant en continuant à faire sauter pour empêcher
que les substances s'attachent au fond de l'appareil.

Dans les sautés de légumes, la cuisson s'achève sans mouillement.

Comme exemples de légumes sautés, on peut citer les pommes de terre,
\hyperlink{p0717}{p. \pageref{pg0717}}, et le riz,
\hyperlink{p0710}{p. \pageref{pg0710}}.

\index{Définition des salmis} Les \textit{salmis} sont des ragoûts dans
lesquels la viande n'est pas seulement revenue, mais à moitié cuite au moins
avant d'être mise à mijoter dans une sauce ; exemples : le
salmis de bécasses, \hyperlink{p0630}{p. \pageref{pg0630}}, et le
salmis de homard, \hyperlink{p0284}{p. \pageref{pg0284}}.

\index{Daubes (Définition des)}
\index{Définition des daubes}
\index{Définition des braisés}
\index{Braisés (Définition des)}
Les \textit{viandes braisées, ou en daube}, car les deux expressions sont
synonymes, sont des ragoûts cuits en vase hermétiquement clos. La préparation
est faite dans des appareils appelés braisières ou daubières ; le nom de
braisière a été donné à ces ustensiles parce qu'à l’origine on garnissait de
braise leur couvercle à rebord. La cuisson des braisés est conduite à petit feu
sous l’action combinée de la chaleur du combustible et de la vapeur des jus
emprisonnés dans l'appareil. Aujourd'hui, on fait généralement les braisés au
four.

\index{Définition du fricandeau} 
\index{Fricandeau (Définition du)} 

Parmi les plats de viandes braisées, on peut mentionner la daube de faux filet,
\hyperlink{p0464-2}{p. \pageref{pg0464-2}}, le fricandeau, viande de veau piquée,
braisée dans son jus, \hyperlink{p0499}{p. \pageref{pg0499}}, et comme plats de
légumes cuits en vase clos, les pommes de terre à la vapeur, les pommes de
terre à l’étuvée ou à l'étouffée, le couscous,
\hyperlink{p0704}{p. \pageref{pg0704}}, la potée fermière,
\hyperlink{p0772}{p. \pageref{pg0772}}.

\index{Définition des matelotes}
\index{Définition des blanquettes}
\index{Blanquettes (Définition des)}
\index{Définition des fricassées}
\index{Fricassées (Définition des)}
Les \textit{fricassées} sont des ragoûts dans lesquels la viande n'est pas
revenue avant d'être mise dans la sauce. Exemples : les \textit{blanquettes},
qui sont des fricassées de viandes blanches (poulet, veau, agneau, chevreau),
avec sauce à la crème, et les \textit{matelotes} ordinaires, qui sont des
fricassées de poisson avec sauce au vin.

\index{Définition des salpicons}
\index{Salpicons (Définition des)}
Les \textit{salpicons} sont des ragoûts fins qui servent surtout de garnitures.
Ils sont composés de viandes délicates et de légumes de choix, truffes,
champignons, cuits à part, puis coupés en petits cubes, enfin liés avec une
sauce savoureuse.

Par extension, on donne le nom de salpicons à des mélanges de légumes ou de
fruits cuits ou confits, coupés en dés et liés avec des sauces ou des jus
appropriés.

\textit{d}) Faire \textit{frire} une substance, c'est la faire cuire dans un
bain constitué par un ou plusieurs corps gras.

\index{Friturier (Quelques considérations sur L'Art du)} 

Le friturier doit savoir que tous les corps gras ne s'altèrent pas à la même
température ; le beurre ordinaire se décompose à {\ppp120\mmm}° ; clarifié, il
peut aller jusqu'à {\ppp135\mmm}° ; les graisses animales, dont les meilleures
comme goût sont la panne de porc fondue, la graisse de rognon de veau ou de
bœuf, les graisses de volaille et notamment la graisse d'oie, se décomposent
entre {\ppp180\mmm}° et {\ppp210\mmm}° ; enfin viennent les graisses végétales,
dont certaines, telle l'huile d'olive bien pure, résistent à {\ppp300\mmm}°.

Pour mettre au point un bain de friture, il est essentiel de commencer par le
clarifier, c'est-à-dire lui enlever ses impuretés, puis on le passe au travers
d'un linge ; cette dernière opération devra être répétée chaque fois que le
bain aura servi. Un bon bain de friture doit toujours être clair ; il doit de
plus être abondant, de manière que l'introduction du corps à frire n'en abaisse
pas sensiblement la température.

\index{Définition des fritures}
\index{Fritures (Définition des)}
On peut diviser les fritures en deux groupes très distincts :

1° la friture au beurre, dans laquelle la cuisson est menée d'une façon
relativement lente, à une température forcément inférieure à {\ppp120\mmm}° si
l'on opère avec du beurre ordinaire, ou à {\ppp135\mmm}° si l’on emploie du
beurre clarifié.

2° la friture à la graisse animale ou végétale, dans laquelle la substance est
saisie, puis cuite, à une température toujours supérieure à {\ppp140\mmm}° et
dépassant parfois {\ppp200\mmm}°.

À mon avis, seul le mode de coction du deuxième groupe constitue la véritable
friture, et il serait préférable de lui réserver exclusivement ce nom, la
cuisson au beurre présentant des caractères différents. Je dirai donc que
\textit{faire frire une substance, c'est la faire cuire à une température
toujours supérieure à {\ppp140\mmm}°, dans un bain de graisse animale ou végétale}.

Au point de vue de leur température, les bains de friture proprement dits sont
souvent désignés par l’une des expressions suivantes : friture moyenne, friture
chaude, friture très chaude, qui correspondent respectivement à {\ppp140\mmm}°,
{\ppp160\mmm}° et {\ppp180\mmm}°.

Les substances panées ou enrobées de pâte doivent être plongées directement
dans de la friture chaude, de façon que l'enveloppe se prenne aussitôt sans se
désagréger.

Les poissons d’une certaine dimension, soles, merlans, etc., qui demandent une
cuisson relativement prolongée, doivent être mis dans une friture moyenne.

Les petits poissons, pour devenir croustillants, doivent être jetés dans un
bain très chaud qui les saisisse vivement. L'huile d'olive leur convient
d'autant mieux que, la cuisson durant très peu de temps, le bain ne leur
communique aucun goût étranger.

Les légumes à frire sont, le plus souvent, plongés au début dans une friture
à température moyenne pour les déshydrater. Le type de la friture de graisse
animale pour légumes est un mélange de panne de porc fondue et de graisse de
rognon de bœuf ou de veau. La cuisson est parfois faite en deux temps et même
en trois, comme il est expliqué à l'article « Pommes de terre soufflées »,
\hyperlink{p0715}{p. \pageref{pg0715}}. On peut aussi employer la graisse d'oie ou
l'huile, si l'on en aime le goût.

On considère généralement que le meilleur bain de friture pour la pâtisserie,
les beignets, etc., est un mélange de {\ppp50\mmm} pour {\ppp100\mmm} de
graisse d'oie et {\ppp50\mmm} pour {\ppp100\mmm} de graisse de rognon de veau ;
ou encore du beurre fondu clarifié ; j'aime mieux la panne de porc fondue.

Lorsqu'on n'a pas commencé une opération de friture dans un bain très chaud, il
est bon de pousser plus ou moins la température à la fin.

\index{Définition des poêlés} 
\index{Poêlés (Définition des)} 

Comme exemples de cuisson dans du beurre fondu, je citerai la formule de la
\hyperlink{p0332}{p. \pageref{pg0332}}, dite « à la meunière », applicable à tous
les poissons fins, de dimensions moyennes, et aussi les préparations dites
« poêlées » de viandes et de volaille, qui différent des rôtis à la poêle par
la quantité de beurre qui les imbibe et en modifie le goût.

Le chauffage au gaz est le procédé le plus commode pour toutes les préparations
culinaires par voie humide, parce qu'il permet de régler la température au gré
de l'opérateur et qu'il ne communique aucun goût aux substances en cuisson, ce
qui se produit quand on l’emploie pour les préparations par voie sèche,
notamment pour les grillades.

\sk

\index{Aspics (Définition des)}
\index{Définition des aspics}
\index{Définition des gratins}
\index{Gratins (Définition des)}
Les \textit{gratins} et les \textit{aspics} ne sont pas des plats cuits d’une
façon spéciale : après avoir été préparés suivant l'une des méthodes indiquées
plus haut, ils subissent simplement un finissage.

On désigne sous le nom de gratins des mets cuits saupoudrés de fromage râpé,
de mie de pain ou de chapelure mélangées ou non avec du fromage râpé, et dont
on fait dorer la garniture au four ou à la pelle rougie au feu.

On désigne sous le nom d’aspics des substances alimentaires cuites, enrobées
dans de la gelée et servies froides.

\medskip

Groupons dans un tableau synoptique les différents procédés dont nous venons de
parler.

\sk

\newgeometry{top=0mm, bottom=0mm, right=0mm, left=0mm}
\enlargethispage{35mm}
\begin{sidewaystable}
\begin{center}
% \scriptsize
\ifbool{K}{\tiny}{\scriptsize}
\setstretch{1.1}
\setlength\tabcolsep{.2em}
\begin{longtable}{|p{2mm}|p{2mm}|p{2mm}|p{2mm}|p{2mm}|p{2mm}|p{2mm}|p{2mm}|p{2mm}|p{2mm}|p{2mm}|p{2mm}|p{2mm}|p{2mm}|p{2mm}|p{2mm}|p{2mm}|p{2mm}|p{2mm}|p{2mm}|p{2mm}|p{2mm}|p{2mm}|p{2mm}|p{2mm}|p{2mm}|p{2mm}|p{2mm}|p{2mm}|p{2mm}|}
\hline
M &M &M &M &M &M &M &M &M &M &M &M &M &M &M &M &M &M &M &M &M &M &M &M &M &M &M &M &M &M                                                                                                             \kill
%\hline
\multicolumn{30}{|c|}{I. — PROCÉDÉS PAR VOIE SÈCHE}                                                                                                                                                \\[.2em]
\hline
\multicolumn{10}{|c|}{\makecell{\textit{a}) RÔTISSAGE\\\textsc{Rôtis.}}}     
  & \multicolumn{10}{c|}{\makecell{\textit{b}) GRILLAGE\\\textsc{Grillades.}}} 
    &  \multicolumn{10}{c|}{\makecell{\textit{c}) CUISSON EN CROÛTE\\\textsc{Pâtés.}}}                                                                                                              \\
\hline
\multicolumn{30}{|c|}{II. — PROCÉDÉS PAR VOIE HUMIDE}                                                                                                                                              \\[.2em]
\hline
\multicolumn{3}{|c|}{\makecell{\textit{a})\\ÉBULLITION\\}}  
  & \multicolumn{3}{c|}{\makecell{\textit{b})\\POCHAGE\\}}     
    & \multicolumn{18}{c|}{\makecell{\textit{c})\\CUISSON DES DIVERS RAGOÛTS\\comprenant :\\}} 
      & \multicolumn{6}{c|}{\makecell{\textit{d})\\FRITURES\\}}                                                                                                                                     \\[.2em]
\hline
\multicolumn{3}{|c|}{\makecell[tl]{ comprenant la\\confection des\\potages et des\\soupes, la pré-\\paration des\\viandes et des\\légumes bouil-\\lis ; les courts-\\bouillons ;\\blanchir : cuis-\\son incom-\\plète dans\\un liquide.}} 
    & \multicolumn{3}{c|}{\makecell[tl]{ des sub-\\stances albu-\\mineuses :\\œufs pochés,\\gigot poché,\\filet poché,\\etc, etc.}} 
      & \multicolumn{3}{c|}{\makecell[tl]{ 1° \textit{les ragoûts}\\\textit{proprement dits}\\tels que :\\  ragoûts de\\mouton,gibe-\\lotte, civets,\\matelote blan-\\che, etc.}}
        & \multicolumn{3}{c|}{\makecell[tl]{ 2° \textit{les sautés}\\de viandes ten-\\dres et de lé-\\gumes.}}
          & \multicolumn{3}{c|}{\makecell[tl]{ 3° \textit{les salmis}\\de gibier, de\\crustacés, etc.}}
           & \multicolumn{3}{c|}{\makecell[tl]{ 4° \textit{les braisés}\\viandes braisées\\ou en daube ;\\légumes à\\l'étouffée ou à\\l'étuvée.}}
            & \multicolumn{3}{c|}{\makecell[tl]{ 5° \textit{les fricas-}\\\textit{sées} de viandes\\blanches : blan-\\quette de veau ;\\matelote, etc.}}
             & \multicolumn{3}{c|}{\makecell[tl]{ 6° \textit{les salpi-}\\\textit{cons} : ragoûts\\de viandes dé-\\licates et de\\légumes de\\choix cuits à\\part et liés\\avec des sau-\\ces savoureu-\\ses.}}
              & \multicolumn{3}{c|}{\makecell[tl]{ 1° \textit{au beurre}\\à basse tempé-\\rature :\\ poêlés, pois-\\sons « à la meu-\\nière ».}}
                & \multicolumn{3}{c|}{\makecell[tl]{ 2° \textit{à la graisse}\\\textit{animale ou vé-}\\\textit{gétale} à tempé-\\rature élevée\\(véritable fri-\\ture), etc. :\\ poissons, lé-\\gumes, pâtes,\\etc. etc.}} \\
\hline
\multicolumn{30}{|c|}{\makecell{PROCÉDÉS DE FINISSAGE\\\textsc{Gratins, Aspics, etc.}}}                                                                                                            \\
\hline
\end{longtable}
\end{center}
\end{sidewaystable}
\restoregeometry

\setstretch{.9}
Après avoir précisé ces quelques notions fondamentales, il est bon de dire un
mot des fonds de cuisson, glaces de viandes, essences, fumets et appareils, des
sauces et des potages, pour achever cette introduction à l'art culinaire.
\index{Classification des divers procédés de cuisson}
