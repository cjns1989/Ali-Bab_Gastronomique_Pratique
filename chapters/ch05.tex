\index{Classification des sauces}
Les sauces sont des combinaisons alimentaires liquides, liées ou non liées, qui
servent d'accompagnement à certains mets.

\index{Définition des sauces}
\index{Sauces (Définition des)}
Les sauces liées, de beaucoup les plus importantes, se composent toutes d'un
fond plus ou moins succulent, assaisonné, aromatisé, et d’une liaison. Le
nombre des fonds de sauce est considérable, celui des aromates est très grand
et il y a beaucoup de façons de lier une sauce : aussi, dans ces conditions, il
est facile de comprendre que, le nombre des combinaisons possibles étant pour
ainsi dire infini, il y a là une véritable mine pour le chercheur.

Toutes les sauces liées peuvent être rangées dans l’une des classes suivantes :

1° les sauces à base d'huile, dont le type est la mayonnaise.

2° les sauces à base de lait ou de crème, dont le type est la sauce Béchamel
maigre.

3° les sauces à base de beurre, dont le type est la sauce dite hollandaise.

4° les sauces à base de consommé de substances animales, les plus nombreuses.

5° les sauces à base de bouillon de légumes, employées surtout dans la cuisine
végétarienne.

6° les sauces mixtes, procédant à la fois de deux ou de plusieurs des classes
précédentes.

Le type des sauces non liées est la vinaigrette.

Les sauces à base de consommé de substances animales peuvent être divisées en
plusieurs groupes, suivant la nature du consommé employé : consommé de viande
de boucherie, de volaille, de gibier, de crustacés, de mollusques, de poissons.

On peut encore les partager, au point de vue de leur couleur, en deux groupes :
les sauces brunes et les sauces blondes.

Le type des sauces brunes est la sauce dite espagnole\footnote{La sauce
espagnole et la sauce allemande tirent leur désignation de leur couleur.},
à base de fond brun, liée au roux. Les sauces civet, liées au sang, font partie
de ce groupe.

Les sauces blondes peuvent être subdivisées en ; veloutés gras ou maigres,
à base de fond blanc, liés à la fécule ou à la farine cuite à blanc et
soigneusement dépouillés ; en sauces dites allemandes, à base de velouté, dont
la liaison est parachevée avec des jaunes d'œufs ; et en \textit{sauces
suprêmes}, à base de velouté, liées à la fois à la fécule ou à la farine cuite
à blanc et aux jaunes d'œufs, puis montées à la crème et mises au point avec du
beurre fin. Ces dernières, essentiellement savoureuses, sont le triomphe de la
cuisine raffinée.

L'art du saucier consiste à marier les éléments dont il dispose, à les fondre
de manière à obtenir un tout qui s’harmonise parfaitement avec la base
fondamentale du plat, qu'il est destiné à faire valoir, mais dont il ne doit
être qu'un accompagnement.

Dans les grandes maisons, la fonction du saucier est spécialisée : son travail
consiste à préparer chaque jour pour ses créations les fonds et les appareils
dont il se servira, le moment venu, concurremment avec les essences et les
aromates dont son laboratoire doit être pourvu. De même qu'un peintre, à la
recherche d'un ton, amalgame sur sa palette différentes couleurs pour arriver
à son but, l'artiste saucier, puisant successivement dans différentes
préparations élémentaires, tenues au chaud, au bain-marie, amalgame leurs
différentes saveurs pour arriver au résultat rêvé.

Mais il est rare qu'un modeste amateur puisse travailler ainsi. C’est pourquoi,
dans mes formules. au lieu de parler d'éléments composés dont on ne dispose que
rarement, j'ai jugé préférable de donner pour chaque sauce les proportions des
éléments simples qu'on trouve partout, et d'indiquer ensuite le détail des
opérations à exécuter. C'est là, je crois, la solution la plus pratique du
problème compliqué de la préparation des sauces.
