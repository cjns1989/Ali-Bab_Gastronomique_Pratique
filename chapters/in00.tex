\begin{itshape}
L'art culinaire a pour objet d'accommoder les aliments de manière à les rendre
appétissants et digestibles. Grâce à lui, nous pouvons éprouver plusieurs fois par
jour, aussi longtemps que nous sommes bien portants, des sensations agréables et,
si nous tombons malades, il facilite notre alimentation et nous permet ainsi de lutter
contre le mal en nous aidant à soutenir nos forces.

Cela suffit à montrer son importance, qui n'est cependant pas appréciée à sa
valeur, en raison des préjugés qui font généralement considérer les connaissances
humaines comme appartenant à un ordre d'autant plus élevé qu'elles sont moins
utiles. Pour ma part, j'avoue être arrivé à l'âge de vingt-cinq ans sans en avoir eu
la moindre idée et c'est seulement pendant mes premiers grands voyages, alors que,
jeune ingénieur, j'allais chercher ou étudier des gisements miniers dans des pays
perdus, que je pris goût à la cuisine. Réduits d'ordinaire, mes compagnons et moi,
aux produits de la chasse et de la pêche que les indigènes nous faisaient cuire le
plus souvent simplement grillés ou bouillis, sans autre apprêt, nous n'avions guère,
comme moyen de varier un peu nos menus, que la ressource des conserves dont on
se fatigue vite. Nous devions lutter contre l'inappétence et l'anémie qui en résulte ;
aussi la moindre innovation dans la préparation des mets était-elle accueillie avec
enthousiasme, pour peu qu'elle nous procurât une sensation gustative tranchant sur
la monotonie habituelle.

Les principes fondamentaux de l'art culinaire sont très simples. Il va sans dire
que pour faire de la bonne cuisine il est essentiel d'avoir de bonnes matières 
premières, que pour obtenir des sauces savoureuses il faut employer comme mouillement 
des jus aromatisés, du vin, des alcools de qualité convenable, enfin il n'est pas
besoin d'être très expert pour savoir que, sauf dans les cas où l'on veut saisir une
viande, il convient, le plus souvent, de faire cuire les aliments lentement, à petit
feu, en les faisant mijoter.

En appliquant ces principes, on fera toujours de la cuisine mangeable ; mais
pour faire de la cuisine fine la plupart des personnes ont besoin de guide
précis, sans quoi elles s'exposent à ne jamais arriver à préparer un plat
vraiment bien, ou, si elles y parviennent un jour par hasard, elles risquent de
ne plus le réussir le lendemain. Beaucoup d'entre elles se figurent alors qu'il
y a dans l'opération un tour de main professionnel qui leur échappe et cela les
décourage. Or, il est bien évident que pour refaire exactement ce que l'on
a fait une fois il suffit d'opérer toujours de la même manière.

Les cuisiniers habiles voient le moment précis où la cuisson est à point, ils
ont l'instinct des proportions de condiments qu'il convient d'employer et n'ont
pas besoin de se servir de balances ; mais les personnes qui n'ont pas une très
grande expérience doivent pour réussir suivre à la lettre les indications
données dans des recettes détaillées avec précision et peser tous les éléments.
Ce n'est que lorsqu'on est arrivé à être de force à se rendre compte à priori
de la valeur d’une préparation culinaire en lisant simplement sa formule, comme
un musicien expert juge un opéra sans l'avoir entendu à la seule lecture de la
partition, que l'on peut songer à voler de ses propres ailes

Un mot encore sur la genèse des plats nouveaux. Un plat est réellement nouveau
lorsque les éléments qui le composent sont associés pour la première fois, ou tout au
moins lorsque leur combinaison est faite dans des proportions inédites donnant une
saveur différente de celle des préparations connues. C'est l'étude des associations
végétales et végéto-animales qui constitue la base des créations gastronomiques. Qui
ne connaît les affinités respectives du pigeon pour les petits pois, de la perdrix pour
le chou, du gigot de mouton pour les haricots, etc., ainsi que les mets classiques
auxquels elles ont donné naissance ?

Malheureusement la découverte de ces groupements harmonieux s'est toujours
faite jusqu'à présent pour ainsi dire exclusivement par hasard et il n'y a, à vrai dire,
aucune direction à indiquer, aucune règle à formuler pour la création des plats. Seul
le goût peut amener l'artiste à des combinaisons nouvelles. Quel service rendrait à
l'Humanité le savant qui parviendrait à formuler d'une manière générale les lois des
associations eupeptiques ! Leur connaissance nous ferait probablement entrevoir des
plats que nous ne soupçonnons pas aujourd'hui, de même que les vides de la Table de
Mendéléef nous ont permis d'affirmer l'existence de corps que nous ignorions jusqu'alors.

Comme je l'ai dit plus haut, j'ai pris goût à la cuisine dans mes voyages, j'en
apprécie aujourd'hui toute la valeur et je m'en occupe volontiers à l'occasion. J'ai
modifié certains plats connus, j'en ai créé de nouveaux et, si j'en juge par les
compliments de mes amis, dont certains se piquent d'être de fins gourmets, ainsi que
par l'empressement flatteur avec lequel leurs femmes demandent mes recettes, je puis
croire que j'ai souvent réussi.

Aujourd'hui, je réunis en volume un certain nombre de ces recettes que je me suis
efforcé de rédiger d'une façon claire et précise ; j'entre, en les décrivant, dans tous
les détails nécessaires, j'indique les proportions pondérales des éléments toutes les
fois que je le crois utile, je m'attache, en un mot, à donner de véritables formules
scientifiques qui permettent de reproduire exactement les préparations décrites, sans
qu'il faille, pour les réussir, avoir de grandes connaissances culinaires ; il suffit
d'opérer soigneusement. Je mentionne dans ce recueil un certain nombre de plats très
simples, d'autres sont plus ou moins compliqués, quelques-uns sont coûteux, mais
tous peuvent être exécutés par une seule personne : c'est, en somme, de la bonne cuisine 
à la portée de tout le monde.

Comme application plus ou moins directe de l'art culinaire à la thérapeutique, j'ai
fait suivre mes formules de l'exposé d'une méthode de traitement de l'obésité des 
gourmands, sur laquelle je me permets d'attirer l'attention du lecteur. Je sais bien que je
ne satisferai pas les nombreux obèses à la recherche de la pilule miraculeuse qui les
fasse maigrir, tout en leur permettant de s'empiffrer : je prévois aussi qu'un certain
nombre de personnes diront qu'il n'y a rien de nouveau dans mon régime et qu'il est
bien certain qu'on doit maigrir en mangeant peu. Il y a cependant quelque chose de
particulier dans ma méthode. En effet, contrairement à la plupart des régimes 
alimentaires préconisés et qu'il est presque impossible d'observer longtemps, le modus
vivendi que je recommande ne fatigue pas, on peut le suivre indéfiniment sans
dégoût et j'ai obtenu en l'appliquant des résultats remarquables, tant au point de vue
de la diminution du poids, qu'à celui de l'amélioration de la santé générale, sans
faire usage d'aucun médicament.

Tous mes amis connaissent l'ancien obèse sujet principal de mon expérimentation :
ils sont prêts à témoigner de la réalité de la cure, comme ils sont prêts à attester les
qualités de ma cuisine. C'est sous leurs auspices que je présente ce livre au public.
\end{itshape}
