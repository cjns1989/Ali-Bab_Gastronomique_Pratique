\sk

Le service des repas comprend les ustensiles de cuisine, les meubles, la
vaisselle, les couverts, l'argenterie et le linge de table.

\sk

\textit{Temps préhistoriques. —} A l'origine, le service des repas était réduit
à sa plus simple expression. L'homme préhistorique mangeait le plus souvent
debout, à la hâte, traqué par ses semblables et par les animaux sauvages
toujours prêts à lui arracher sa nourriture. Lorsqu'il se croyait dans une
sécurité relative, il mangeait accroupi ou assis par terre. Mais ses repas
étaient toujours irréguliers et généralement précaires. Aussi, quand le hasard
mettait à sa disposition une nourriture abondante, il se gavait gloutonnement
pour plusieurs jours et, repu, il s'étendait pour digérer\footnote{Certaines
peuplades sauvages agissent encore ainsi et les raffinés, pleins à éclater, se
font piétiner le ventre pour activer la digestion.}, comme un boa.

Les contemporains de l'âge de pierre faisaient cuire les viandes sur des
pierres chauffées, et ils se servaient de haches en silex pour les tronçonner.
Ils buvaient l'eau des sources ou des cours d'eau à même, la bouche dans le
liquide, ou dans le creux de leurs mains. Ils employaient aussi parfois des
feuilles\footnote{À Madagascar, les indigènes utilisent les feuilles de
ravenala, non seulement pour couvrir les habitations, mais aussi pour servir de
plats, d'assiettes, de verres, de nappes et de serviettes.} ou des coquillages.
Plus tard, à la fin de l'âge de pierre, apparurent quelques poteries grossières
et des récipients en os\footnote{L'usage des crânes humains comme ustensile
à boire se retrouve de nos jours chez des peuplades sauvages, généralement
cannibales. La plus grande satisfaction des anthropophages intellectuels
consiste à boire dans le crâne d'un ennemi dont la chair fait les frais du
festin.

Les Esquimaux emploient encore actuellement des os de cétacés comme ustensiles
de table, et, simple détail de ménage, ils font nettoyer la vaisselle par leurs
chiens qui la lèchent consciencieusement.} ou en corne.

L'âge du bronze, la période lacustre virent naître les meules pour moudre le
grain, les instruments en bronze et en cuivre.

Ensuite vinrent les instruments en fer.

\sk

\textit{Antiquité. — Égypte. —} En Égypte, le matériel culinaire comprenait des
vases en terre et en métal, des poêles, des moules à pâtisserie, des fours, et
le service de la table se composait de plats, de coupes, de couteaux, de
cuillers.

Les gens du peuple consommaient leurs repas accroupis autour de petites tables
en bois, rondes ou ovales, basses, portant des plats en terre dans lesquels
étaient contenus les mets qu'ils prenaient avec les doigts.

Chez les gens aisés, le service était autre. Dans la pièce servant aux repas,
des tables en bois, en pierre ou en métal, ornées d'incrustations, de ciselures
ou de peintures étaient dressées au milieu. On disposait dessus les mets dans
des plats en terre ou en métal, les boissons fermentées dans des amphores et
l'eau\footnote{On commençait par laisser déposer l'eau du Nil destinée à la
boisson avant de la servir. Dès la \textsc{XVII}\textsuperscript{e} Dynastie,
on la décantait ensuite an moyen de siphons.} dans des vases en terre poreuse,
analogues aux alcarazas. Des tabourets à pieds droits ou croisés, des chaises
ou des fauteuils, destinés aux convives, étaient placés autour de la salle. La
maîtresse de la maison veillait aux apprêts et présidait aux repas.

Des serviteurs portaient à chaque invité, dans des corbeilles, les différents
mets du menu qui leur étaient servis sur des petits plats au moyen de cuillers
et de crocs en métal.

Avant de commencer le repas, les convives se lavaient les mains dans de l'eau
parfumée, et ils renouvelaient ces soins de propreté après chaque service.

Les boissons étaient versées dans des coupes en bronze et chaque personne,
après avoir vidé sa coupe, s'essuyait les lèvres avec un linge.

Des musiciens et des chanteurs se faisaient entendre pendant la durée du repas.

\sk

\textit{Hébreux. —} Chez les Hébreux nomades, le mobilier de table comprenait
presque exclusivement des tapis et des coussins, des ustensiles en cuir, des
outres pour renfermer les liquides, matériel très pratique, facilement
transportable et incassable.

Lorsqu'ils furent en Égypte, ils remplacèrent les vases en cuir par des
poteries, vernies à l'intérieur et, au début, sans aucun ornement.

Au x siècle avant J.-C., les habitations des Hébreux étaient pourvues d'une
cuisine munie d'un fourneau. Ils faisaient usage de cuillers et de fourchettes
en bois ou en bronze, mais seulement pour servir les mets. Les repas, pris en
commun, les convives assis autour d'une table, étaient généralement précédés
d'une prière ; le principal avait lieu à midi.

Sous la domination des Perses, ils acquirent de leurs vainqueurs des goûts de
luxe et de mollesse. Ils contractèrent l'habitude de se parfumer et de prendre
leurs repas allongés sur des lits plus ou moins somptueux. Ceux des riches
étaient en ivoire sculpté. Les ustensiles de cuisine et de table gagnèrent en
richesse et, fréquemment, furent en argent ciselé et même en or. L'usage de
verres de couleur décorés de fruits en relief devint à la mode.

Les femmes, surtout celles de la classe aisée, étaient servies généralement
à part.

Des danses égayaient les festins et des orchestres, où la flûte et le tambour
se mêlaient au luth et à la harpe\footnote{On attribue l'invention des
instruments à corde à Jubal, fille de Laurech.}, charmaient les mélomanes.

Dans les cérémonies religieuses les repas jouaient un rôle considérable : c'est
ainsi que le luxe des repas funéraires atteignit un tel excès que l'on vit des
gens se ruiner en donnant aux funérailles des leurs des festins fastueux
auxquels tout le peuple était convié\footnote{Il y à loin de ces agapes aux
modestes collations que, dans les enterrements populaires, les parents du
défunt offrent de nos jours, à la porte du cimetière, aux amis fidèles qui sont
venus l'accompagner à sa dernière demeure, chez le marchand de vin,
à l'enseigne : « On est mieux ici qu'en face »}. 

\textit{Assyriens. —} Le mobilier, dans la classe pauvre, était des plus
rudimentaires : il était constitué par des cubes de pierre, des nattes et des
escabeaux primitifs.

Chez les Assyriens de condition moyenne, il existait des chaises et des
tabourets, et le matériel de cuisine se composait d'un four à pain et d'un
foyer en plein air sur lequel était disposée une marmite unique servant
à toutes les préparations culinaires.

Les riches possédaient des meubles en bois d'essences choisies, ou en ivoire,
sculptés et décorés : comme ustensiles, des marmites, des plats, des cuillers
à pot et des coupes en métaux précieux. Pendant les repas les invités étaient
assis autour d'une table ; seul, le chef de la famille était couché ; les
femmes mangeaient à part. Des eunuques servaient.

Chez les rois assyriens, les lits, excessivement riches, étaient couverts de
broderies, les pieds, les dossiers et les appuie-bras des fauteuils étaient
incrustés d'ivoire, de métaux précieux et de pierreries. Les reines prenaient
leurs repas assises sur des fauteuils.

D'après Diodore de Sicile, ce fut le roi Assurbanipal qui porta au plus haut
degré le luxe de la table.

\sk 

\textit{Hindous. —} Les accessoires de la table portaient la marque d'une
grande recherche dans la classe riche hindoue, mais les gens du peuple
n'avaient que des ustensiles primitifs dont ils se servent encore, du reste, de
nos jours.

\sk

\textit{Perses. —} Chez les Perses, le luxe de la table était très grand. Les
ustensiles de cuisine et de salle à manger étaient en métaux précieux.

A la Cour, la pièce principale du palais, enguirlandée de fleurs et embaumée de
parfums, qui servait à la fois de salon et de salle à manger, était pavée de
marbre, de porphyre et d'albâtre. Elle était garnie de tapis et ornée de
tentures retenues à des colonnes de marbre par des anneaux d'argent. Le lit du
roi, la vaisselle, les cratères, les coupes, les vases, les plats, etc.,
étaient en argent ou en or.

Le roi mangeait seul et couché. Ce n'était qu'exceptionnellement qu'il
admettait auprès de lui la reine et ses fils.

Des eunuques assuraient le service.

Lorsque le roi avait des invités, les convives étaient groupés dans une salle
voisine de la pièce principale, qui lui était réservée, et séparée par un
rideau disposé de manière que le roi voyait ses invités sans être vu par eux.
Cependant, certains jours de grands festins, des familiers, jamais plus de
douze à la fois, étaient admis dans la pièce qu'il occupait, mais le monarque
seul était couché sur un lit que les convives entouraient, accroupis sur des
tapis. Le roi buvait un vin spécial que ses échansons étaient tenus de goûter
chaque fois qu'ils lui en versaient ; il y en avait d'autres pour les invités.

Le personnel des cuisines était nombreux. Sous le règne de Darius, il
comprenait {\ppp277\mmm} cuisiniers, {\ppp89\mmm} préposés à la préparation des
breuvages. Au service de la table, en plus des échansons et des eunuques,
étaient attachés {\ppp40\mmm} parfumeurs et {\ppp329\mmm} courtisanes
musiciennes et danseuses.

\sk

\textit{Grecs. —} En Grèce, comme chez tous les peuples de l'Antiquité, au
début, il n'existait pas de salle à manger proprement dite. La pièce principale
de la maison, qui était en même temps pièce de réception et cuisine, servait au
repas. Cette disposition avait pour inconvénient majeur d'emplir la pièce de
fumée et d’une odeur de graillon, car le plus souvent les aliments étaient
préparés sur l'autel domestique. Plusieurs siècles s'écoulèrent avant que les
Grecs comprissent la nécessité de faire la cuisine dans une pièce spéciale
munie d'une cheminée, et ce ne fut qu'après la conquête de la Macédoine qu'ils
commencèrent à réserver dans chaque habitation une pièce destinée à servir de
salle à manger. Cette pièce, généralement carrée, était de dimensions telles
que quatre tables, entourées chacune de trois lits groupés en fer à cheval,
pouvaient y tenir à l'aise.

À l’origine, les Grecs se tenaient assis à table : l'usage des lits comme
sièges pendant le repas leur vint de Lybie. Il fut d’abord le privilège des
seigneurs, puis celui des hommes qu'on voulait honorer ; mais, à partir du
siècle de Périclès, l'emploi en devint général.

Le mobilier de la salle à manger se composait de tables, de lits à une ou deux
personnes, de sièges, de coussins et, en plus, de coffres, de buffets sur
lesquels étaient dressées des pièces d’argenterie et d'orfèvrerie souvent très
riches. La salle était ornée de tentures et de tapis.

Le matériel de table comprenait des plats, des huiliers, des corbeilles à pain,
des paniers pour vases contenant des liquides, des cuillers, mais seulement
pour manger la bouillie, des coupes, des verres à boire si nombreux qu'Athénée
en décrit une centaine de variétés de toutes formes et de toutes dimensions.

Celui des cuisines était constitué par des chaudières, des marmites, des
casseroles analogues aux nôtres, des réchauds, des écumoires, des poêles
à frire et des plats pour cuire les œufs, des amphores, des bouilloires, des
bols en poterie ou en métaux, des cuillers, des fourchettes à deux dents, des
couteaux et des crocs.

Les Grecs prenaient leurs aliments avec les doigts ; ceux qui craignaient de se
salir ou de se brûler étaient gantés ; les autres, après chaque plat,
s'essuyaient les doigts avec de la mie de pain qu'ils roulaient et jetaient
ensuite aux chiens.

Dans les repas de cérémonie, avant de se mettre à table, les convives faisaient de
nombreuses ablutions, ils se parfumaient ensuite, mettaient des robes blanches,
chaussaient des sandales et s'ornaient de fleurs.

Le personnel des cuisines fut d'abord recruté dans la basse classe de la
population et parmi les esclaves ; puis, vers le \textsc{v}\textsuperscript{e}
siècle avant notre ère, des écoles dans lesquelles la durée des études était de
deux ans, formèrent des cuisiniers professionnels. Des femmes furent
spécialisées dans la préparation des douceurs, et des maîtres d'hôtel eurent la
haute direction des services de la cuisine et de la table. Des concours
culinaires, stimulant le zèle des artistes, s'organisèrent. Le premier lauréat
du premier concours culinaire d'Athènes fut un nommé Chiroménès qui avait
présenté au jury des truffes assaisonnées d'épices, bardées de lard et cuites
au vin, à l'étouffée, dans une enveloppe de pâte. Son nom mérite de passer à la
postérité,

Les villes de l'antique Grèce réputées pour la somptuosité, parfois excessive,
de la table sont : Athènes, Corinthe, Sybaris et Syracuse,

\sk

\textit{Étrusques. —} Les Étrusques, d'origine probablement phénicienne, qui
habitaient le nord de l'antique Italie, jouissaient d'une civilisation
avancée. Ils aimaient la bonne chère, les beaux meubles, le faste de la table,
les somptueux habits à l'orientale, les bijoux et les ornements précieux. Ils
excellaient à travailler les métaux ; ils fabriquaient des vases en argile
noire, représentant des bustes « Canope » et d'autres de style différent
« Bucchero » simplement gravés ou portant des ornements en relief ; des
amphores, de la vaisselle décorée magnifiquement.

Leurs maisons les plus simples, carrées, étaient construites en bois avec un
auvent de chaque côté : les plus opulentes faisaient pressentir les maisons
romaines avec atrium.

Les meubles étaient luxueux : les plus riches étaient revêtus de figurines et
d'appliques métalliques estampées.

On peut se faire une idée de l'art étrusque dans les musées de l'ancienne
Étrurie et, en particulier, dans celui de Volterra. Des sculptures du temps
font voir des scènes de festins dans des cadres rehaussés de mobiliers
richement ornés et de pièces de vaisselle de grande valeur. Des fresques,
trouvées dans les nécropoles, montrent les convives habillés luxueusement et
chargés de bijoux, mangeant et buvant, charmés par des musiciens et égayés par
des danseuses à peine vêtues.

Les femmes étaient admises aux repas et aux réjouissances.

\sk

\textit{Romains. —} Les premiers Romains vivaient très simplement dans des
cabanes couvertes de roseaux ou de paille, composées d'une seule pièce : aussi,
mangeaient-ils généralement sur le seuil de leur demeure.

Vers le \textsc{vii}\textsuperscript{e} siècle avant notre ère, plus civilisés,
ils édifièrent dans Rome des maisons avec cour et jardin, dans lesquelles la
pièce principale « atrium » servait à la fois de salon, de salle à manger et de
cuisine. Un trou carré ou rectangulaire, percé au plafond, était destiné
à évacuer la fumée et à laisser passer l'eau de pluie qui était recueillie dans
une cavité de même forme, ménagée dans le sol de la salle, exactement
au-dessous du trou.

Puis, après la conquête de l'Italie et les premières expéditions en Afrique,
les maisons à Rome prirent de l'extension et eurent de véritables salles
à manger « triclinium », pavées de mosaïques et dont les parois en stuc,
décorées de fresques, étaient garnies de tentures, de colonnes, de vasques, de
plantes grimpantes : des cuisines et des magasins pour différents usages de la
table.

Les torches, avec lesquelles, au début, on obtenait l'éclairage, furent remplacées
par des flambeaux de fibres végétales enduites de poix ou de cire qu'on fixait
dans des candélabres, des appliques ou des torchères, et, après la conquête de la
Grèce, par des lampes à huile.

Pas plus que les autres peuples de l'Antiquité, les Romains ne connurent les
assiettes. Ils se servaient avec les mains ; les cuillers n'étaient employées
que pour manger la bouillie, les œufs, les escargots, les crèmes et les
confitures. Après le repas ou même après chaque service, ils se lavaient les
mains ; les gourmands se léchaient les doigts. On essuyait, on brossait, on
lavait les tables après le repas ; dans les maisons riches, on changeait de
table après chaque service.

Au temps de César vint l'usage de couvrir les tables d'étoffes de lin, de laine
ou de soie : ce furent les premières nappes. De la nappe à la serviette il
n'y avait qu'un pas ; il fut vite franchi.

Le matériel de cuisine et de table comprenait ceux des Égyptiens, des Grecs et
des Étrusques, et, en plus, des marmites et des seaux en bronze, des
chauffe-plats et des trépieds, des plateaux pour le service, des cuillers
variées, des couteaux, des fourchettes à deux dents réservées pour la cuisine
et le service, des moules à pâtisserie, des tables en pierre pour découper,
munies de cavités servant de mortiers, enfin des pierres d'évier avec
écoulement d'eau.

Les verres à boire en verre blanc datent du temps de Néron.

Après la chute de Carthage, un luxe inouï envahit Rome. Certains lits de salle
à manger atteignaient des prix fabuleux, jusqu'à {\ppp800\mmm} {\ppp000\mmm}
francs de notre monnaie, et Livius Drusus (pour ne citer que celui-là)
possédait {\ppp5\mmm} {\ppp000\mmm} kilogrammes d'argenterie et d'orfèvrerie.

Chez Lucullus, douze salles richement ornées, portant chacune le nom d'une
divinité : Apollon, Mars, Cérès, etc. et décorées de leurs statues en marbre de
Paros, servaient à la fois de salons et de salles à manger. Lorsque le maître
de céans ordonnait qu'un repas fût servi dans l’une de ces salles, cela
correspondait à une dépense prévue ; c'est ainsi qu'un festin commandé pour le
salon d'Apollon devait coûter {\ppp50\mmm} {\ppp000\mmm} drachmes, soit plus de
{\ppp34\mmm} {\ppp000\mmm} francs.

Dans le palais d'or de Néron, le plafond de la salle à manger était constitué
par des panneaux mobiles en ivoire permettant un mouvement circulaire. Il
imitait les révolutions du ciel et représentait les saisons de l’année par un
changement à chaque service. De ce plafond, des essences parfumées et des
fleurs pleuvaient sur les convives.

Les heures des repas étaient sensiblement les mêmes que les nôtres ; le dîner
se prolongeait souvent tard dans la nuit et tournait en orgie. Chaque invité
avait le droit d'amener des amis ; des parasites s'invitaient d'eux-mêmes et
payaient leur écot avec leur esprit. Leur espèce n'est pas perdue.

Le personnel, très nombreux, était bien dressé et spécialisé. Les chefs de
cuisine gagnaient jusqu'à {\ppp20\mmm} {\ppp000\mmm} francs, sans compter les
cadeaux qui doublaient ou triplaient cette somme. Les plus recherchés étaient
membres de l’Académie culinaire qui fut fondée à Rome sous Adrien, et dont le
siège était au Palatin.

\sk

\textit{Gaulois, Gallo-Romains, Francs. —} Les habitants de la Gaule
indépendante vivaient dans des cavernes ou dans des huttes faites de troncs
d'arbres et de terre. Ils prenaient leurs repas dans les habitations ou en
plein air. Ils cuisaient leur pain sous la cendre. Ils mangeaient assis et se
servaient, pour découper leurs aliments, d'un couteau dont la gaine était
pendue à leur côté. Ils buvaient dans des cornes ou dans des crânes, montés
sur une garniture d'argent ou d'or, d'animaux qu'ils avaient tués, surtout
l'urus. Ils couchaient sur des lits d'herbes ou de peaux de bêtes. Les premiers
ustensiles de cuisine paraissent avoir été des meules ou des mortiers pour
écraser le grain, puis vinrent des vases en argile cuite, de forme élégante et
décorés de dessins géométriques en blanc ou en couleur ; des corbeilles faites
avec des lamelles de bois très fines et très minces, des cribles en crin. Ils
n'avaient ni buffets, ni armoires. Chez les riches, il existait seulement des
tables en bois grossièrement travaillées, creusées parfois d'excavations
circulaires de petites dimensions qui tenaient lieu de plats et d'assiettes.

Plus tard, les Gaulois acquirent des connaissances dans l'art des mines et
devinrent très habiles en métallurgie. Ils firent des plats en bronze, des
vases et des ustensiles en divers métaux qu'ils recouvraient d'un placage ou
qu'ils étamaient, et ils travaillèrent l'or et l'argent.

Ils furent les créateurs des tonneaux en bois à douves, cerclés de fer, pour le
transport des vins.

Un demi-siècle avant J.-C., les Gaulois riches possédaient de solides maisons
en pierres, spacieuses et garnies d'un mobilier sinon luxueux, du moins assez
confortable.

Après la victoire des Romains, les Gaulois des campagnes continuèrent à vivre
de la vie simple des habitants de la Gaule indépendante ; ceux des villes
adoptèrent la façon de vivre de leurs vainqueurs. Ils eurent des villas
somptueusement décorées, semblables à celles de Rome et de Pompéi, pourvues de
salles à manger élégantes et artistiques, magnifiquement servies.

Les Francs, comme tous les peuples barbares, ne connaissaient ni habitation
confortable, ni salle à manger, ni matériel de table ; mais, après leur
établissement en Gaule et au contact des Romains, leur façon de vivre changea.
Les riches seigneurs eurent des maisons construites généralement en bois et
bâties sur des collines ou à flanc de coteau. Elles avaient des allures de
château féodal ou de forteresse, étant entourées de fossés, de palissades et
parfois flanquées de tours, avec les dépendances d'un domaine bien ordonné
pouvant répondre a tous les besoins du maître. Autour de la maison seigneuriale
s'élevaient des cabanes et des huttes nombreuses servant d'habitation aux
colons, artisans et affranchis employés par le chef franc pour cultiver ses
terres, fabriquer tout ce qui lui était nécessaire et le défendre en temps de
guerre.

Les salles à manger, peu garnies de meubles, étaient spacieuses : les murs
étaient ou simplement blanchis à la chaux et, alors, souvent garnis de
tapisseries, ou ornés de peintures aux tons éclatants. Des portières et des
rideaux complétaient la décoration.

Le mobilier était simple, peu important et mal façonné ; il se composait de
tables montées sur tréteaux, de banquettes et de bancs à dossier, en bois, de
coffres servant de buffets et d’armoires. Mais le matériel de table était très
riche : vaisselle d’or ou d'argent, ustensiles en métaux précieux, pièces
d'orfèvrerie.

Les Francs faisaient usage de cuillers, de couteaux, de vases à boire en bois,
en terre cuite, en verre, en marbre, en métaux précieux.

Les cuisines occupaient généralement un bâtiment séparé. Elles étaient vastes
et pouvaient servir de réfectoire pour les domestiques et les esclaves.

Chaque castel possédait un moulin à bras et un four.

Chez les rois francs, la salle à manger était divisée en trois parties par deux
rangées de colonnes. Une partie était réservée pour la famille royale, une autre
pour les invités et la troisième pour les officiers de la maison du roi.

Le mobilier n'était ni plus important, ni plus luxueux que celui des seigneurs,
mais des ustensiles d'or et d'argent, des grands bassins de mêmes métaux
enrichis de pierreries, de la vaisselle précieuse, des plats, des coupes en
verre multicolore, en marbre, en céramique constituaient le matériel de la
table.

Au \textsc{v}\textsuperscript{e} siècle, les seigneurs francs, suivant la mode
romaine, adoptèrent l'usage des lits groupés autour des tables.

Dans les festins, les murs étaient tapissés de feuillage, le sol jonché de
fleurs ou recouvert de riches tapis, la table elle-même offrait l'aspect d'un
parterre fleuri. Les différents mets étaient présentés dans des ustensiles de
substances variées : plats d'or ou d'argent, de verre, de marbre, poteries
noires, corbeilles peintes, vases incrustés de gemmes précieuses.

Le repas avait lieu au son des flûtes et des hautbois. Des mimes, des baladins,
des chanteurs, des danseuses divertissaient les convives.


Le festin terminé, les invités prenaient part à des jeux de hasard. Chez les
seigneurs et les chefs francs, des combats singuliers avaient lieu.

\sk

\textit{Moyen âge. —} Les demeures, en France, au moyen âge, furent plus
luxueuses et plus confortables, surtout après les Croisades, et elles
s’embellirent de tous les apports nouveaux que l'industrie, très florissante,
put fournir.

Les manoirs, plus grandioses, eurent une pièce de réception appelée la « salle
des hôtes », dans laquelle se donnaient les repas de cérémonie et les festins ;
mais il n'existait pas de salle à manger proprement dite.

Les repas ordinaires étaient servis, en hiver, dans les chambres, les
antichambres et même les cuisines ; en été, ils avaient lieu dans le jardin,
sous un bosquet ou dans un salon de verdure.

Le sol des pièces, dans les habitations, était la terre battue, recouverte de
paille ou de foin, plus tard de nattes de paille. Chez quelques seigneurs
riches, des tapis épais de laine remplaçaient la paille ou le foin. Les murs
étaient peints à l'huile ou revêtus d'un enduit à la colle.

L'éclairage se faisait avec des torches, des flambeaux de cire et des petites
lampes plates, en cuivre, alimentées avec de l'huile, posés à terre ou sur la
table et supportés par des lustres ou des couronnes.

Le mobilier était constitué par des bancs à quatre pieds, sortes de banquettes,
des bancs-coffres à dossier avec accotoirs, garnis de housses ou de coussins,
des fauteuils, des chaires richement ouvragées, des tables fixes ou des tables
démontables, des escabeaux, des écrins, espèces de coffres où l'on enfermait
les objets précieux, des dressoirs, des vaisseliers ou buffets de bois fins
à un ou plusieurs gradins\footnote{Le nombre des gradins était proportionné
à la situation nobiliaire du possesseur : {\ppp5\mmm} pour une reine ou un
prince souverain ; {\ppp4\mmm} pour un prince ; {\ppp3\mmm} pour un comte ;
etc. Les gens non titrés avaient des vaisseliers sans gradin.}, artistement
sculptés, destinés à recevoir les plus belles pièces d'argenterie ou
d'orfèvrerie. Le dressoir tenait une très grande place dans les cérémonies, les
processions, et il faisait partie du matériel de toutes les réceptions. Le luxe
des dressoirs était poussé au plus haut degré chez les seigneurs et les
prélats. À la Cour, chez certains rois, il existait des buffets d'or et
d'argent, des dressoirs d'une très grande richesse, garnis de tapisseries et de
drap d’or supportant des pièces rares d'orfèvrerie et de la vaisselle
magnifique en or.

Les cuisines occupaient une place importante dans les habitations du moyen âge.
Elles étaient généralement situées dans des bâtiments spéciaux. Au
\textsc{xiii}\textsuperscript{e} siècle, elles étaient rondes le plus souvent
et munies d'une cheminée avec prise d'air au plafond pour évacuer la fumée et
les odeurs. Au \textsc{xiv}\textsuperscript{e} siècle, on donna aux cuisines
une forme carrée avec deux ou plusieurs cheminées et prise d'air centrale. Dans
les châteaux et les palais, les cuisines étaient spacieuses : elles avaient de
nombreuses annexes pour tous les services en dépendant. Chez les bourgeois, les
cuisines étaient petites, non distinctes des habitations et prises, souvent,
sur une chambre, qui servait alors tout à la fois de salle à manger, de chambre
à coucher et de cuisine.

Le mobilier des cuisines était composé d'un vaisselier ou dressoir de cuisine,
somptueux chez les princes, les riches seigneurs et les hauts prélats, d'un
coffre à conserver la viande salée, d'une armoire à épices, d'une huche pour
pétrir et serrer le pain, de tables sur tréteaux ou sur pieds, de bancs, de
banquettes et d'escabeaux.

Dans les cheminées, immenses, profondes et assez hautes pour qu'une personne
adulte pût s'y tenir debout, pendait la crémaillère avec son allonge ou
crémaillon en fer forgé souvent travaillé avec art (celle du roi Charles
V était en argent), à laquelle on suspendait la grosse marmite de fer de vingt
litres environ. En avant du foyer, se dressaient deux landiers ou chenets de
fonte, sur la queue desquels on disposait d'énormes bûches. Les tiges
verticales des landiers, hautes d'un mètre au moins, étaient munies de crampons
qui servaient à accrocher l'écumoire, la large louche, la longue fourchette
à deux dents servant à fouiller dans les marmites et les pots, les pincettes,
les tisonniers, les pelles à feu et plusieurs broches\footnote{L'origine des
broches remonte au \textsc{xv}\textsuperscript{e} siècle ; on leur adjoignit
bientôt des lèchefrites. On employa comme tourne-broches d'abord des gamins et
plus tard des chiens,} ; elles étaient terminées presque toujours en forme de
corbeilles dans lesquelles on plaçait un récipient quelconque destiné
à réchauffer ou à cuire certains mets. Il existait aussi des réchauds qu'on
pouvait rapprocher ou éloigner du foyer.

Le déploiement du luxe des accessoires de la table fut inouï au moyen âge : il
atteignit son apogée du \textsc{xii}\textsuperscript{e} au
\textsc{xiv}\textsuperscript{e} siècle. Si les pauvres n'avaient que des
écuelles et des plats en terre et en bois ; si, dans la classe inférieure, il
n'existait que des plats, des vases et des récipients émaillés grossièrement ;
par contre, les bourgeois possédaient de la vaisselle en étain comptant de
véritables objets d'art, de la vaisselle en faïence d'Italie\footnote{Fabriquée
à Faenza, bourg d'Italie.}, et, sur les tables des seigneurs et des rois, des
trésors ruisselaient. Ce n'étaient que pièces d'orfèvrerie d'apparat,
vaisselle, plats, aiguières, flacons, drageoirs pour les épices, saucières,
fourchettes\footnote{Originaires de Byzance, semble-t'il.}, très rares, pour
manger les fruits, salières de formes et de dimensions diverses, pintes,
coupes, hanaps, les plus riches de toutes les coupes, nefs merveilleuses,
précurseurs des surtouts, en or, en argent ou en vermeil, et riches ustensiles
en cristal\footnote{On conçoit quelles fortunes fabuleuses il fallait avoir
pour posséder de pareils services, l'argent et l'or ayant environ une valeur
{\ppp25\mmm} fois plus grande que de nos jours.}.

Des lois somptuaires, dont la première remonte à Philippe le Bel, s'efforcèrent
de lutter contre ce luxe effréné, mais ce fut en vain. Juvénal des Ursins, dans
une harangue prononcée aux États de Tours ({\ppp1\mmm} {\ppp468\mmm}), dit
« qu'il n'y a presque plus personne en France qui ne veuille manger en
vaisselle d'argent ».

Le matériel de la table comprenait encore des moutardiers en étain ou en
argent, des pots à confitures en verre, fort beaux, des chopes à couvercle, des
râpes à épices, des pintes en étain et en cristal, des tasses, des gobelets
avec trépieds, de nombreux vases à boissons de toutes tailles et de toutes
formes, en métaux précieux, en étain, en cuivre, en cristal de roche, en verre,
en marbre, en terre cuite, en bois de toutes essences : des boîtes à épices en
faïence, des grands et des petits couteaux à trancher et à désosser, des
couteaux à pain, à huîtres ; des verres en verre de Venise, des coupes en œufs
d'autruche et en noix de coco, des rafraîchissoirs parfois très artistement
travaillés. Pour la cuisine, des brocs en bois cerclés de fer, des fioles, des
cruches, des pots en grès, des fourchettes à trois dents, des entonnoirs en
cuir et en fer-blanc.

Au \textsc{xiv}\textsuperscript{e}siècle, on commença à assortir les pièces de
vaisselle, les unes aux autres, pour constituer des services.

Le linge de table, dont l'emploi était tombé en désuétude après la chute de
Rome, redevint à la mode. Les gens aisés avaient des nappes simples ou brodées
et des longières, sortes de nappes étroites posées sur le pourtour des tables,
servant aux convives pour s'essuyer les doigts. A la fin du
\textsc{xv}\textsuperscript{e} siècle, il y eut des napperons et des
serviettes.

Cependant, malgré tout ce luxe, il n'y avait pas d’assiettes
individuelles\footnote{L'usage des assiettes individuelles date de la fin du
\textsc{xv}\textsuperscript{e} siècle et, pendant longtemps, il n'y eut qu'une
seule assiette par convive pour toute la durée du repas.} ; les soupes, les
sauces et tous les mets liquides étaient présentés dans des écuelles. Dans les
intérieurs pauvres, chacun plongeait à tour de rôle sa cuiller dans l'écuelle
contenant la soupe ou d'autres mets liquides et l'on buvait à la ronde dans le
même vase. Dans les maisons bien tenues, il y avait une écuelle à soupe ainsi
qu'une écuelle à vin pour deux convives, et le tact de l'amphitryon consistait
à grouper ses invités par couples\footnote{Les convives étaient généralement
assis. Cependant, l'habitude de manger couché ne disparut pas complètement
après la décadence de Rome. Au \textsc{xii}\textsuperscript{e} siècle, certains
festins galants avaient encore lieu à la mode romaine.} sympathiques. Des
serviteurs tenaient des linges devant les dames pendant qu'elles buvaient afin
d'éviter qu'elles tachassent leurs robes.

Les viandes et les mets solides peu humides ou secs étaient servis sur des
plaques de métal ou de bois, ordinairement rondes, appelées tranchoirs ou
tailloirs, garnies au préalable d'une ou de plusieurs tranches de pain
bis\footnote{On faisait aussi des salières en pain bis.} rassis destiné à boire
le jus sortant de la pièce de viande fraîchement coupée et à éviter que la
nappe fût tachée. Ces tranches de pain, qui portaient le nom de pain-tranchoir,
n'étaient pas consommées à table ; elles étaient données aux valets ou aux
pauvres. Il y avait autant de tranchoirs que de convives.

Les plats étaient apportés couverts par des cloches (elles étaient en argent
chez les riches) et les gobelets posés sur la table étaient munis de
couvercles, usage introduit par la crainte des empoisonnements. C'est de cette
coutume qu'est venue l'expression : « mettre le couvert ».

La richesse se manifestait encore par le nombre des serviteurs. Les bourgeois
se contentaient le plus souvent d'un domestique ou d'une servante à tout
faire : mais, lorsqu'ils donnaient des repas priés, ils louaient d’autres
valets, par ostentation, pour faire croire qu'ils avaient une maison montée.
Chez les riches seigneurs et chez les rois, le personnel de bouche était
nombreux. Philippe le Bel, le promoteur des premières lois somptuaires, avait
{\ppp54\mmm} personnes pour le service de sa cuisine et de sa table ; Charles
V en avait {\ppp158\mmm}.

Des banquets, des festins somptueux, dans lesquels le luxe rivalisait avec
l'excentricité (réminiscence de ceux de l'époque romaine), étaient donnés
à tout propos. Les grands tenaient table ouverte. Les salles, les tables
étaient enguirlandées de fleurs et des couronnes fleuries ornaient les vases et
les coupes à boire.

On annonçait le repas par des sonneries de cor.

Avant de se mettre à table, les convives se lavaient les mains avec de l'eau
parfumée que des pages, des écuyers ou des damoiselles leur présentaient dans
de riches bassins. Chez les souverains, cette fonction était réservée aux
chambellans.

Les mets les plus recherchés et les plus variés étaient servis à profusion.
Pendant le repas, des trouvères et des troubadours récitaient des contes, des
poèmes, des fables, des romans ; des ménétriers se faisaient entendre ; des
baladins, des jongleurs, des bouffons montraient des animaux et faisaient des
tours ; des concerts vocaux et instrumentaux, des pantomimes, des ballets
avaient lieu ; des scènes galantes, des épisodes guerriers, des sujets
mythologiques, des vues pittoresques étaient représentés. Ce furent les
premiers « entremets »\footnote{Le mot entremets désignait, à l'origine, tout
ce qui était présenté entre les plats de viande ou de poisson et, après avoir
été appliqué aux divertissements, il servit à désigner les légumes présentés
seuls et les douceurs telles que les sorbets qui séparaient les services.
Aujourd'hui, le terme d'entremets est surtout réservé aux douceurs terminant
les repas.}. Des fontaines automatiques à compartiments distribuaient les vins.

A la Cour, le roi faisait largesse au peuple.

Parmi les festins de l'époque, les plus célèbres sont : le festin « du
faisan », donné par Philippe le Bon, duc de Bourgogne, pour fêter l'expédition
organisée contre Mahomet II ({\ppp1\mmm} {\ppp453\mmm}) ; le festin donné par
Gaston III, comte de Foix, à l'occasion du mariage de la fille de Charles VII
({\ppp1\mmm} {\ppp458\mmm}).

\sk

\textit{Renaissance. —} A cette époque, pas plus qu'au moyen âge, il n'existait
de salles à manger. On servait les repas dans la salle des hôtes, dans les
chambres à coucher ou dans la cuisine, lorsqu'il faisait froid ; dans le jardin,
pendant la belle saison. Le sol des pièces, dallé ou carrelé, était couvert
d'herbes odoriférantes et de fleurs, en été ; de nattes, en hiver.

Au \textsc{xvi}\textsuperscript{e} siècle, le chauffage s'effectuait au bois
brûlant sans arrêt dans les hautes et profondes cheminées ; mais bientôt des
poêles en terre vernissée, venant d'Allemagne, apportèrent leur supplément de
chaleur.

La Renaissance marqua nombre d'objets de son empreinte. Les meubles de table,
dont on peut voir de beaux spécimens dans nos musées, étaient représentés par
des buffets, des armoires, des tables, des chaises, des stalles magnifiquement
sculptés ; on voyait aussi des tables extensibles, dites tables-tirantes, avec
tablettes aux extrémités. Mais ces meubles étaient disséminés, dans toutes les
pièces, dans les couloirs et même dans les cuisines. En changeant de style, les
dressoirs étaient devenus des crédences et les bancs avaient fait place aux
tabourets, très riches, parfois même en argent massif.

Pendant la Renaissance, le linge de table fut très en faveur. Il était de toile
damassée, très ouvragée ; il y avait même des tabliers damassés. À tous les
degrés de la société, chez les riches comme chez les pauvres, dans
l'aristocratie comme dans la bourgeoisie, les tables étaient couvertes de
nappes et chaque convive avait une serviette. Dans la haute société, on
changeait de serviettes plusieurs fois pendant le repas. On parfumait le linge
et on prit l'habitude de plier les nappes et les serviettes de nombreuses
façons décoratives.

Les tables étaient somptueusement dressées et surchargées de pièces superbes
d'orfèvrerie et de vaisselle précieuse.

Le matériel comprenait encore des vinaigriers en verre, en cristal, en étain et
en argent, des canettes en étain munies d'un couvercle et d'une anse, des
buires très élégantes, des vidrecomes, des pichets à anse, en faïence, en grès,
en étain, en argent, des petits paniers filigranés d'or et d'argent pour les
fruits, les gâteaux, etc. ; des paniers en vannerie fine et en vannerie dorée
garnis de dentelles et de rubans, des bonbonnières ou drageoirs de poche, des
couteaux de poche, des cure-dents en orfèvrerie, en os et en bois odoriférant,
des casse-noix, des pots à vin et à bière en grès, en étain, en argent, des
coquetiers en étain et en métal précieux, des assiettes en faïence, des cruches
à bec, des vases et des coupes avec couvercles, montés ou non à charnière.

Les fontaines de table en métaux précieux ont presque complètement disparu du
service de la table, sauf à la Cour.

Un fait important du \textsc{xvi}\textsuperscript{e} siècle fut l'apparition de
la fourchette individuelle, mise à la mode par Henri III ; mais son usage ne se
généralisa pas. Des cuillers à long manche virent le jour, nécessitées par le
port chez les seigneurs et les dames des hautes et larges collerettes empesées
appelées fraises.

Les cuisines, très spacieuses des demeures seigneuriales et des palais,
participèrent au faste luxueux de la table. En dehors des pièces d'argenterie
qu'elles contenaient, il y avait des ustensiles nombreux pour tous les usages,
en bois, en ivoire, en étain, en fer, en fonte, en cuivre, en bronze : des
soufflets articulés, des poissonnières, des bassines, des tourtières, des
poêles à frire, des tinettes en bois\footnote{On en faisait de petites en or
ou en argent pour les crédences.}, des fontaines de cuisine, certaines très
belles, en cuivre.

A la fin du \textsc{xvi}\textsuperscript{e} siècle, les
rôtissoires\footnote{Elles étaient en fer, quelquefois aussi en argent.} avec
coquille commencèrent à paraître et les glacières pour tenir au frais les mets
et les boissons entrèrent en usage.

Malgré de nouvelles lois somptuaires et les édits enjoignant de vendre la
vaisselle précieuse pour remédier à la pénurie des finances, le luxe le plus
grand régna pendant la Renaissance, personne ne voulant renoncer à cette marque
extérieure de fortune. Sous François I\textsuperscript{er}, il atteignit son
point culminant. A toutes les richesses existantes vinrent s'ajouter les belles
pièces céramiques de Bernard Palissy et les merveilles d'orfèvrerie ciselées de
Benvenuto Cellini.

Le service dans la classe riche était assuré par un personnel nombreux. Chez le
roi François I\textsuperscript{er}, il était représenté par un état-major de
{\ppp95\mmm} serviteurs gentilshommes, la plupart ayant un nom illustre, auquel
était adjointe une armée de subalternes.

Les repas comprenaient plusieurs services abondamment pourvus. Dans ceux de
grande cérémonie, une magnificence inouïe était déployée. Le goût des plaisirs
raffinés, apporté d'Italie par Catherine de Médicis, avait gagné toutes les
classes, aristocratie, bourgeoisie et même une partie du peuple ; mais les
pauvres mangeaient toujours dans des écuelles de terre et de bois. Dans les
festins de la Cour de France, la richesse se faisait encore remarquer par
l'abondance des viandes de toutes sortes et par les « entremets-spectacles »
somptueux qui y étaient représentés. Les banquets maigres n'étaient pas moins
luxueux.

Le règne de Henri IV mit fin à ces fêtes dispendieuses.

\sk

\textit{Temps modernes. —} Au \textsc{xviii}\textsuperscript{e} siècle, il
y eut dans beaucoup d'habitations, surtout dans les maisons bourgeoises et les
hôtels, une pièce spéciale affectée aux repas ; mais dans la plupart des
châteaux princiers et des maisons seigneuriales on continua à servir les repas
dans une pièce quelconque. Louis XIV et Louis XV se faisaient servir dans leur
cabinet ; on ne dressait la table dans leur antichambre ou dans celle de la
reine que pendant la saison chaude.

Le sol des habitations était couvert de nattes, en paille colorée et tressée,
extrêmement élégantes et très décoratives ; puis vinrent les tapis de cuir
gaufré, en été, de laine, en hiver. Au \textsc{xvii}\textsuperscript{e} siècle,
les murs des maisons riches étaient revêtus de nattes multicolores ou de cuir
peint et doré ; plus tard, il y eut des tapisseries de soie, des verdures de
Flandre et des tapisseries de haute lice. Sous Louis XIV, la décoration murale
changea ; ce furent des ornements appliqués directement sur les murs :
moulures, corniches, trumeaux, pâtisseries décoratives, etc. Sous Louis XV, on
vit des lambris de menuiserie blanche et dorée, des petits panneaux de
tapisseries, des cartouches, des médaillons artistiques, peints à l'huile, et
des glaces partout.

Dans les intérieurs bourgeois, les salles à manger, lorsqu'elles existaient,
étaient meublées très simplement, les murs recouverts de nattes. Quand la mode
des nattes passa, ils furent peints à l'huile ou blanchis à la chaux ; plus
tard, on les recouvrit de papiers peints.

Le chauffage était assuré au moyen de feux de bois brûlant dans les cheminées,
de dimensions plus restreintes qu'au siècle précédent et ressemblant beaucoup au
type des cheminées actuelles, et par des poêles en terre vernissée plus variés de
formes et plus artistiques que les premiers connus.

L'éclairage était fait à l'aide de chandeliers et de flambeaux de table,
d'appliques, de girandoles, de lustres, et aussi au moyen de flambeaux et de
torches, en cuivre, en bronze, en cristal, en argent, en vermeil, tenus par des
domestiques. Les seigneurs conservèrent longtemps cette dernière façon de
s'éclairer dans les grands dîners ; cela leur permettait de faire voir qu'ils
avaient une domesticité nombreuse et satisfaisait leur orgueil et leur amour de
l’ostentation.

Le peuple se servait de chandelles et de l'antique lampe à huile,

Le mobilier était représenté par des tables rondes ou ovales\footnote{Il y en
eut même en fer à cheval.} sculptées, des tables à rallonges, des tables avec
abattants à charnières dites « tables à l'anglaise », en acajou. Elles étaient
recouvertes de tapis de parade ou d'étoffes surchargées de broderies, qu'on
enlevait pour dresser la table.

Il y avait encore des tables et des meubles en marqueterie de bois, d’autres
recouvertes de tablettes de marbre portées par des pieds sculptés et dorés ;
des tables à transformation dites « tables machinées » : des tables volantes
sortant toutes dressées de dessous le plancher ; des buffets mouvants et des
meubles à métamorphoses, vrais chefs-d'œuvre d'ingéniosité, très en vogue sous
Louis XV, surtout dans les petits soupers, et permettant de se passer de
valets\footnote{Le marquis de Bouillac eut le premier l'idée de prendre ses
repas hors de la présence de ses domestiques, qu'il sonnait simplement
lorsqu'il en avait besoin. On appela ce mode de service le « repas à la
clochette ».}. Il existait aussi des tables à thé : des consoles de formes
variées sur lesquelles on plaçait la vaisselle de rechange ; des petites
tables, munies de {\ppp2\mmm}, {\ppp3\mmm}, {\ppp4\mmm} tablettes, placées
à côté de la grande table et faisant office de servantes ; des fauteuils
massifs à dossier et à bras, de lourdes chaises à dossier, des tabourets, des
sièges pliants appelés « perroquets\footnote{A la Cour, ils furent pendant
longtemps des sièges aristocratiques.} ».

Au \textsc{xviii}\textsuperscript{e} siècle seulement vinrent les sièges
confortables disposés autour de la table.

Le dressoir devint le « buffet de la salle à manger » (il y en avait deux et
même trois dans les intérieurs riches), le « vaisselier » dans les intérieurs
rustiques.

Le linge ouvré, nappes, napperons et serviettes, très recherché, était l'objet
de soins minutieux.

Le luxe de la vaisselle précieuse et des pièces d'orfèvrerie atteignit souvent
des proportions exagérées et même invraisemblables\footnote{La marquise de
Pompadour laissa après sa mort pour {\ppp687\mmm} {\ppp000\mmm} francs de
vaisselle d'or et d'argent.}. On l’augmentait sans cesse suivant la mode et le
goût du jour. On s'offrait comme cadeaux des pièces de vaisselle précieuse. La
possession d'un nombre important de ces ustensiles en or ou en argent
constituait une preuve de richesse, de noblesse et de
distinction\footnote{Nombre de familles nobles ruinées avaient gardé leur
vaisselle d'argent.}. Louis XIII, Louis XIV et Louis XV édictèrent des lois
somptuaires contre ce luxe excessif et scandaleux, surtout alors que le peuple
mourait de faim. A plusieurs reprises, ils envoyèrent à la fonte toutes les
richesses de la Cour : trône et meubles d'argent, vaisselle précieuse, qui
faisaient l'ornement des palais, pour être convertis en monnaie et remédier aux
crises financières extrêmement graves. Les familles royales et quelques nobles
suivirent l'exemple, mais la plupart préférèrent enterrer leur vaisselle en
attendant des jours meilleurs. Ce qui n'avait pas été livré de bon gré était
saisi.

Le luxe se reporta sur la faïence, la porcelaine et les cristaux : faïences de
Rouen, de Nevers, de Strasbourg, de Lunéville ; porcelaines de Chine et de Saxe,
de Vincennes, de Bourg-la-Reine, de Sceaux, de Sèvres, de Chantilly ; verreries
de Venise et cristaux de Bohême.

Sous Louis XVI, les crises économiques s'apaisèrent pendant un certain temps.
Avec la prospérité, l'argenterie et le luxe reparurent sur les tables et dans
le mobilier. Tout le monde voulait avoir de la vaisselle plate et de la
vaisselle montée, en argent. La porcelaine fut admise à la Cour en même temps
que la vaisselle d'or et d'argent. La faïence fut pour les gens peu aisés.

Le matériel de table était très varié. Il comprenait, en plus de la vaisselle
somptueuse, des assiettes creuses, plates et à desserts des plats, des
saucières, des cruches, des jattes, des bassins, des salières, des aiguières,
des pots à sucre et à crème, des plats à oille\footnote{Oille ou olla, ragoût
de viandes diverses et de légumes, d'origine espagnole.}, des tasses, des
sucriers, des soucoupes, des théières, des cafetières, des chocolatières, des
soupières, des légumiers, des porte-huiliers, des porte-moutardiers, des
plateaux, des surtouts, des vases et des objets de toutes sortes, fort beaux,
en porcelaine ; des porte-assiettes ronds en argent, en étain ou en osier, des
plateaux-cabarets pour le thé, le café, le chocolat, en bois laqué de Chine, en
métal, en faïence, en lapis-lazuli, en cristal de roche, etc. ; des plateaux,
des boîtes à thé, des chocolatières en métal, surtout en argent ; des pinces
à sucre, des tire-moelle, des cuillers à café, à sucre, à moutarde, à olives,
des cuillers à ragoût, des louches\footnote{La louche fut inventée par le duc
de Montausier qui passait pour un raffiné original, d'une délicatesse
excessive.}, des flacons et des coupes de verre ou de cristal, des bouteilles,
des aiguières de faïence casquées avec anse en crosse, des aiguières de cristal
de roche taillées, gravées et décorées, des caves à liqueurs, des verres
à boire en verre de Venise et en cristal de Bohême taillé et gravé. Les seaux
à rafraîchir en faïence et en porcelaine remplacèrent les bassins d'argent et
de vermeil ; les fontaines de marbre, de bronze, de plomb moulé fixées aux murs
des salles à manger chassèrent les riches fontaines des siècles précédents.

Dans la bourgeoisie, on se servait pour la table de bouteilles, de cruches et
de pots : les carafes, les aiguières, les buires étaient l'apanage des nobles
maisons.

Dans les châteaux et les manoirs riches, il existait près de la salle à manger
une petite pièce munie d'un fourneau potager qui servait à tenir au chaud ou
à réchauffer les plats quand la cuisine était très éloignée.

Tous les intérieurs aisés possédaient une cuisine. Dans la bourgeoisie, elle
était généralement reléguée dans un recoin éloigné et sombre de l'habitation ;
aussi, était-elle souvent mal tenue et même malpropre. Les cuisines
aristocratiques, au contraire, étaient spacieuses, bien éclairées, élégantes,
propres et commodes. Les seigneurs et les gens opulents mirent leur orgueil
à les décorer de batteries de cuisine et de nombreux meubles de luxe dont ils
étaient très fiers et qu'ils montraient à leurs visiteurs.

Comme matériel, il y avait, au \textsc{xvii}\textsuperscript{e} siècle, un
fourneau ou potager percé de plusieurs trous pour cuire les potages et les
ragoûts, des rôtissoires, des tourne-broches automatiques, des tamis, des
soufflets, parfois très ouvragés, en bois, en métal, en faïence avec réservoir
à air en peau, des tire-bouchons, des tonnelets en grès, des passoires, des
marabouts ou coquemars en fer battu, des turbotières, des tourtières d'argent,
des terrines, des moules à glace en étain et en argent, des fontaines
filtrantes, des cuisines portatives.

Au \textsc{xviii}\textsuperscript{e} siècle, l'aspect des cuisines des maisons
riches changea. Le gros matériel encombrant fut supprimé, le petit mobilier
s'accrut et la batterie de cuisine s'enrichit de nouvelles pièces ; le cuivre
et le fer-blanc occupèrent les premières places dans la collection des
ustensiles, puis l'argent, proscrit par les lois somptuaires, reparut et servit
à fabriquer de la batterie de cuisine.

Jusqu'à la fin du \textsc{xvi}\textsuperscript{e} siècle, on se servit des
doigts pour porter à la bouche les viandes et certains légumes. Au
\textsc{xvii}\textsuperscript{e} siècle, il existait encore une écuelle pour
deux convives. Au \textsc{xviii}\textsuperscript{e} siècle, chaque convive eut
son couvert individuel : assiette personnelle, serviette, cuiller, fourchette,
couteau, verre à boire. Les verres n'étaient pas mis sur la table ; ils
restaient sur le buffet ou sur une servante, rangés dans le même ordre que les
convives, et un serviteur les passait en faisant attention pour qu'il n'y eût
pas d'échanges. Quand le convive avait bu, le domestique rinçait le verre et le
remettait en place sur la servante.

Vers le milieu du \textsc{xviii}\textsuperscript{e}siècle, les vases contenant
les boissons prirent place sur la table, les couteaux furent épointés et
l'usage de la fourchette se généralisa dans les classes moyennes. A cette même
époque, on commença à découper les viandes à l'office au lieu de les découper
sur la table, comme cela avait lieu auparavant et, pendant les repas, on
changea d’assiette au moins deux fois, à chaque plat chez les nobles.

Dans les grands dîners et dans les fêtes, au \textsc{xvii}\textsuperscript{e}
siècle, on dressait sur la table des pyramides de desserts et de porcelaines
étagées, souvent de proportions gigantesques, garnies de fruits frais ou secs,
de confitures, de compotes, de sucreries, de gâteaux, etc. ; il y eut aussi des
desserts montés : vases et supports en cristal ou en métal précieux, puis des
surtouts en argent et en vermeil, appelés « milieux de table ». Au
\textsc{xviii}\textsuperscript{e} siècle vinrent les surtouts de porcelaine ou
de faïence représentant des fleurs, des arbustes, des personnages en
miniature : bergers et musiciens ; ensuite les surtouts en métal précieux
à fond de glace et les surtouts en cuivre doré pour les gens moins riches.

Les tables étaient ornées de fleurs naturelles ; plus tard, on imagina de
fabriquer pour leur décoration des fleurs artificielles. Dans les banquets
et les festins de la Cour et des grands seigneurs, il y avait en toutes saisons,
même l'hiver, profusion de fleurs placées dans des corbeilles, des caisses ou
des vases en cristal, en cuivre doré ou en argent ; des guirlandes et des
festons de fleurs enrubannées couraient autour de la table ; il existait encore
des pyramides très hautes de fleurs, des buissons de fleurs réunis par des
arcades de fruits et des guirlandes de fleurs. Il y eut aussi des sujets en
pâte au sucre et à l'amidon coloré et des décorations givrées sur des plantes
vertes et sur les glaces des surtouts qui, fondant à la chaleur, donnaient aux
convives étonnés l'illusion du dégel des rivières ou de l'éclosion de feuilles
et de fleurs.

L'appel aux repas était fait par son de cloche. Les invités ne se lavaient plus
les mains dans la salle des repas. A la Cour et chez les nobles, ces soins de
propreté avaient lieu dans un réduit ou dans une antichambre : chez les
bourgeois, dans la cuvette d'une fontaine accrochée au mur de la salle
à manger. Louis XIV se faisait présenter simplement, au commencement et à la
fin des repas, une serviette mouillée placée entre deux assiettes d'or ;
l'aristocratie suivit son exemple et on servit aux maîtres, dans les châteaux,
des serviettes mouillées entre assiettes d'argent ou de vermeil.

On récitait à table entre convives des poésies, des morceaux littéraires, des
proverbes ; on se lançait des épigrammes : cela s'appelait des « vaudevilles ». Il
y avait aussi des spectacles lumineux, des comédies, de la musique et des danses.

L'étiquette s'atténua : seuls les grands festins conservèrent leur cérémonial
pompeux.

Le personnel de cuisine et de table était généralement nombreux. A la Cour,
sous Louis XIV, il comprenait {\ppp500\mmm} personnes sous l'autorité du
Grand-Maître de la Maison du roi, lequel était parfois un prince du sang. Tous
les officiers de la Maison de bouche étaient nobles et de familles illustres.

Parmi les festins des temps modernes dans lesquels une somptuosité insensée
était étalée, il faut citer ceux que Louis XIV donna aux fêtes de Versailles,
en {\ppp1\mmm} {\ppp664\mmm}, fêtes connues sous le nom de Plaisirs de l'Île
enchantée. Elles durèrent sept jours et furent remarquables par la profusion
des mets, le nombre des pièces de vaisselle précieuse, la quantité de fleurs
rares et l'étincellement des girandoles et des lustres innombrables augmenté
encore par la lumière des torchères portées par {\ppp200\mmm} valets de pied.
Puis, les festins de {\ppp1\mmm} {\ppp668\mmm} donnés à l'occasion de la paix
d'Aix-la-Chapelle ; ceux du prince de Condé en {\ppp1\mmm {\ppp688\mmm}, et
enfin le festin de Reims, donné dans le palais archiépiscopal, en l'honneur du
sacre de Louis XV. La table du roi, surmontée d'un dais en velours violet orné
de fleurs de lys, était dressée sur une estrade élevée de quatre marches. Le
Grand-Panetier, le Grand-Échanson et le Grand-Écuyer tranchant étaient vêtus de
velours noir et de drap d'or. Le Prince de Rohan remplissait les fonctions de
Grand-Maître.

Le Roi entra dans la salle du festin précédé des hérauts et des maîtres de
cérémonie, accompagné des gentilshommes servants et escorté par les flûtes, les
hautbois et les trompettes de sa maison.

Les différents services du repas furent apportés et servis par les officiers de
bouche, au son de fanfares.

\sk

\textit{Temps contemporains. —} La Révolution n'apporta au service de la table
que la fourchette à quatre dents. Au \textsc{xix}\textsuperscript{e} siècle,
les salles à manger et les cuisines subirent d'importantes transformations. Les
misérables petites cuisines sombres et sans air ne se rencontrent plus que dans
les intérieurs pauvres. Celles des maisons bourgeoises et des hôtels
particuliers sont généralement spacieuses, claires et aérées. Les murs sont
peints à l'huile et, jusqu'à une certaine hauteur, ils ont un revêtement en
carreaux de faïence décorée ; le sol est carrelé ou dallé de carreaux de
céramique. L'éclairage est fait au gaz ou à l'électricité. Le fourneau potager
à vécu ; un fourneau en fonte, à compartiments, avec garnitures en cuivre et en
acier, appelé « cuisinière », chauffé au charbon de terre, au coke, au gaz,
à l'électricité, le remplace. Dans les maisons importantes, ces fourneaux sont
munis de grillades, rôtissoires, fours, étuves, chaudières, réchauds,
bains-marie, chauffe-plats, etc., permettant la préparation des repas les plus
compliqués. De larges hottes avec trappes et ventilateurs assurent l'évacuation
rapide des fumées et des odeurs.

Aux parois des cuisines sont fixées des tringles en cuivre auxquelles sont
accrochés des casseroles, des sauteuses, des couvercles, des passoires, des
écumoires, des cuillers à pot et autres ustensiles de cuisine en cuivre étamé
ou en bi-métal, et des étagères sur lesquelles prennent place des marmites, des
fait-tout, des bassines, des moules de toutes formes et de toutes tailles, des
plats, des théières, des cafetières, des boîtes à épices, des tamis, des
accessoires de batterie de cuisine en bois, en fonte, en fer battu, en nickel,
en tôle émaillée, en cuivre étamé, en terre cuite, en faïence, en porcelaine,
en grès, en verre, etc., etc., difficiles ou impossibles à accrocher. Comme
ameublement, il y a des armoires renfermant le linge et la vaisselle de
cuisine, un dressoir, des tables diverses, dont certaines à dessus de marbre
servent à différentes manipulations, avec tiroirs contenant cuillers,
fourchettes, couteaux et autres instruments nécessaires à la cuisine, des
chaises, des glacières, et tous articles de ménage. Il existe souvent, attenant
aux cuisines, une autre pièce pourvue d'un évier, où se fait le nettoyage des
ustensiles ayant servi à la confection des repas et où se trouvent des
récipients destinés à recevoir les déchets et les résidus, ou mieux encore une
trappe permettant de les évacuer directement dans le sous-sol.

Dans quelques fermes modèles, les cuisines possèdent encore la haute et
profonde cheminée avec marmite à crémaillère, landiers et, concurremment, une
cuisinière moderne avec marmites et autres commodités. De belles batteries de
cuisine et des ustensiles divers pour la préparation des aliments et les
besoins de toutes sortes voisinent avec une grande horloge monumentale parfois
fort belle, des armoires, des tables, des chaises, des fontaines filtrantes, un
évier, etc.

Certaines cuisines de grands hôtels particuliers et de restaurants sont de
véritables merveilles d'agencement et de confort.

Il existe maintenant partout des salles à manger, même dans les plus petits
logements ; leur ameublement et leur décoration varient naturellement avec les
ressources et la richesse des occupants,

Dans les maisons aisées, les salles à manger sont le plus souvent de grandes
dimensions, hautes de plafond, claires et bien aérées. Le sol est un parquet
ciré garni d'un grand et beau tapis de laine ou recouvert entièrement de
moquette clouée. En été, le tapis est souvent remplacé par une tresse. Les murs
sont tapissés de papiers peints artistiques, de cuirs gaufrés rehaussés d'or,
de tissus de prix : ils sont ornés de tapisseries, d'œuvres d'art, d'appliques
à lumières, de faïences anciennes, etc. Du plafond pend une suspension ; au
début de l'époque, elle était modeste et contenait une lampe à huile, plus tard
au pétrole ; de nos jours c'est souvent une pièce d'art importante en fer
forgé, en cuivre, en bronze doré, parfois scintillante de cristaux, au gaz ou
à l'électricité.

La salle à manger contemporaine est, en général, très meublée. On y voit
buffet, dressoir, panetière, servante, argentier, table, fauteuils et chaises
confortables, le tout au goût du jour ; quelquefois aussi des consoles et
d'autres petits meubles sur lesquels sont exposés des potiches, des céramiques,
des grès, des étains, des cristaux, des objets d'art. Près de la salle
à manger, se trouve une pièce destinée au service : c'est l' « office ».

Dans les logements modestes, la salle à manger tient souvent lieu de salon.

Le matériel de la table est extrêmement varié. Il se compose de pièces
d'argenterie ou de ruolz, de services en faïence ou en porcelaine, de
verreries, de cristaux, dont le détail serait fastidieux, tout le monde pouvant
en voir des échantillons dans les nombreux magasins spéciaux. On les modifie,
on les perfectionne, on en crée de nouveaux suivant le goût et la mode.

Les temps contemporains marquent la disparition presque complète de la
vaisselle d'argent : on ne trouve plus guère de vaisselle plate que dans de
rares familles nobles anciennes et chez quelques nouveaux riches,

Notre époque a vu naître successivement le maillechort, le métal blanc
alfénide, le ruolz, l'aluminium, le nickel.

La richesse de la table des siècles précédents n'est plus. Sous
Napoléon I\textsuperscript{er}, à part dans quelques grands dîners, les repas
manquent de somptuosité. Sous Louis XVIII, la belle vaisselle reparaît. Sous
Louis-Philippe, les repas de la Cour sont bourgeois. Le règne de Napoléon III
fait renaître les belles fêtes et les splendides festins. La table est riche et
élégante ; de grands candélabres d'argent, un surtout de même métal et de
jolies coupes garnies de fleurs concourent à son ornementation. La vaisselle
plate est en argent guilloché ; les jours de gala, elle est en vermeil et en
porcelaine de Sèvres ancien ; mais beaucoup d'autres pièces sont en ruolz. Le
chef du service de bouche porte l'habit noir, les huissiers sont en habit
marron à la française ; un superbe nègre vêtu magnifiquement à l'orientale est
exclusivement aux ordres de l'Impératrice.

Le régime démocratique, en nivelant les classes, a supprimé tout le bel
apparat. Les festins ne sont plus que des réunions publiques sans véritable
grandeur et sans relief : festins commémoratifs après la Révolution ; festins
réformistes qui préparèrent la chute des Bourbons ; festins fraternels de
{\ppp1\mmm} {\ppp848\mmm} ; banquets fédératifs des Sociétés mutuelles de
France ; banquets politiques, parmi lesquels il faut citer les banquets des
maires de {\ppp1\mmm} {\ppp889\mmm} au Palais de l'Industrie et de
{\ppp1\mmm} {\ppp900\mmm} au Jardin des Tuileries. Tous ces banquets sont par
rapport aux grandes réceptions et aux festins princiers de l’ancien régime ce
que les bals de l'Hôtel-de-Ville, chansonnés par Mac-Nab, sont aux anciens bals
de la Cour.

A la fin du \textsc{xix}\textsuperscript{e} siècle et dans cette première
partie du \textsc{xx}\textsuperscript{e}, le linge de table est plus que jamais
en faveur : linge damassé, brodé, ouvré, à entre-deux, à jours, à motifs
incrustés, en dentelles.

Le service comporte de nombreuses façons. Dans les grands dîners, la table,
garnie d'un tapis de laine, est recouverte d'une nappe sur laquelle est posé un
magnifique chemin de table et des dessous de carafes assortis. Une guirlande de
fleurs rares, à parfum discret, relie de petits sujets en biscuit de Sèvres, de
Saxe, ou de jolis cristaux minuscules garnis de bouquets de fleurs et de
feuillage, ou un semis de fleurs fines dessinant des arabesques orne la table.
Un service de riche porcelaine est dressé ; il est accompagné de verres
à boire, de toutes dimensions et de toutes formes, en cristal taillé
à facettes. Des carafes ou des buires, en cristal taillé, avec montures en
argent, pour les vins, des carafes ou des brocs, en cristal taillé ou torse et
argent pour l'eau ajoutent leur cachet à l'ensemble. Au milieu de la table
prennent place des surtouts bas, des corbeilles en argent garnies de fruits,
des porcelaines portant des desserts, des friandises, des gâteaux. Les
cuillers, les fourchettes, les couteaux, les porte-menus, les porte-couteaux,
les serviettes pliées artistement complètent la décoration. Un petit bouquet
pouvant être mis au corsage ou une orchidée marque la place des dames. Des
plantes vertes sont disposées de-ci de-là et des lumières à profusion éclairent
la salle. Le service est fait par des domestiques en habit noir. Dans certains
intérieurs opulents, les domestiques sont en culotte de satin noir ou de panne
de couleur avec bas de soie et escarpins vernis à boucle d'argent.

L'aspect d'une salle à manger bien décorée et savamment dressée, étincelante
d'argenterie et de cristaux, resplendissante de lumière, est véritablement
féerique. Parfois, il y a abus de décor et, en présence d'un menu banal servi
dans un cadre splendide, on est tenté de s'écrier : « Trop de fleurs ».

La grande somptuosité des dîners et des festins d'autrefois a subi une éclipse.
Cependant, l'extrême recherche et le goût qui président actuellement
à l'organisation et à la décoration des cuisines et des salles à manger sont
poussés à un tel degré de magnificence qu'on peut affirmer qu'à ce point de
vue, du moins, nous ne sommes pas en décadence.
