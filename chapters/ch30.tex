\sk

\bigskip

On sert fréquemment, dans les réunions en dehors des repas, des boissons
froides ou chaudes, rafraîchissantes ou réconfortantes.

Les boissons rafraîchissantes, orangeades. citronnades, limonades, eaux de
fruits doivent être peu sucrées et ne pas marquer plus de {\ppp9\mmm}° au
pèse-sirop.

Les boissons réconfortantes sont souvent plus ou moins alcoolisées. Beaucoup de
boissons étrangères le sont avec excès. On trouvera ici un certain nombre de
formules de ces boissons dans lesquelles les proportions d'alcool ont été
diminuées pour répondre au goût français. Les amateurs de breuvages très corsés
pourront augmenter à leur convenance les doses des vins, des liqueurs et des
spiritueux.

\section*{\centering Orangeade.}
\phantomsection
\addcontentsline{toc}{section}{ Orangeade.}
\index{Orangeade}

Pour douze personnes prenez :

\footnotesize
\begin{longtable}{rrrrp{16em}}
  & 500 & grammes & de & sucre en morceaux,                                                               \\
  & \multicolumn{2}{r}{3 litres} & d' & eau filtrée,                                                      \\
  &     &         &  8 & belles oranges mûres,                                                            \\
  &     &         &  2 & citrons.                                                                         \\
\end{longtable}
\normalsize

Râpez les zestes des oranges et des citrons, mettez-les dans l'eau, ajoutez le
sucre, laissez-le fondre. Passez les oranges et les citrons au presse-citron,
mélangez le jus obtenu à l'eau sucrée aromatisée ; laissez en contact pendant
une heure, puis passez à l'étamine. Tenez au frais.

\section*{\centering Citronnade.}
\phantomsection
\addcontentsline{toc}{section}{ Citronnade.}
\index{Citronnade}

Pour douze personnes prenez :

\footnotesize
\begin{longtable}{rrrrp{16em}}
  & 600 & grammes & de & sucre en morceaux,                                                               \\
  & \multicolumn{2}{r}{3 litres} & d' & eau filtrée,                                                      \\
  &     &         &  6 & citrons.                                                                         \\
\end{longtable}
\normalsize

La préparation est la même que celle de l'orangeade.

\sk

On peut gazéifier la citronnade en remplaçant une partie de l’eau filtrée par
de l'eau de Seltz qu'on ajoute au moment de servir. On met souvent alors dans
chaque verre une rondelle de citron.

\section*{\centering Limonade fraîche.}
\phantomsection
\addcontentsline{toc}{section}{ Limonade fraîche.}
\index{Limonade fraîche}

Pour douze personnes prenez :

\footnotesize
\begin{longtable}{rrrrp{16em}}
  & 500 & grammes & de & sucre en morceaux,                                                               \\
  & \multicolumn{2}{r}{3 litres} & d' & eau filtrée,                                                      \\
  &     &         &  8 & belles oranges mûres,                                                            \\
  &     &         &  9 & limons.                                                                          \\
\end{longtable}
\normalsize

Même préparation que dans les formules ci-dessus.

\sk

Dans la limonade gazeuse, l'eau filtrée est remplacée par de l'eau gazeuse qu'on
ajoute au moment de servir, à moins que toute la préparation ne soit conservée en
siphons ou en bouteilles.

\sk

Le lemon-squash est une limonade ou une citronnade servie glacée avec du
soda-water.

\section*{\centering Limonade à la framboise.}
\phantomsection
\addcontentsline{toc}{section}{ Limonade à la framboise.}
\index{Limonade à la framboise}

Pour douze personnes prenez :

\footnotesize
\begin{longtable}{rrrrp{16em}}
  & 500 & grammes & de & jus frais de framboises bien mûres,                                              \\
  & 350 & grammes & de & sucre en morceaux,                                                               \\
  & \multicolumn{2}{r}{2 litres} & d' & eau filtrée,                                                      \\
  &     &         &  6 & limons ou 6 citrons.                                                             \\
\end{longtable}
\normalsize

Faites dissoudre le sucre dans l'eau, ajoutez le jus de framboises et celui des
limons ou des citrons, un peu de zeste ; laissez en contact pendant {\ppp2\mmm}
à {\ppp3\mmm} heures. Passez la préparation au chinois fin ou à l'étamine.

Tenez au frais jusqu'au moment de servir.

\sk

On peut préparer de même des limonades à d'autres jus de fruits : groseilles,
fraises, cerises, mûres, myrtilles, etc.

\sk

On fait aussi des limonades cuites, soit en préparant simplement un sirop de
limons, soit en ajoutant à du sirop bouillant le jus des fruits choisis.

\section*{\centering Limonade blanche.}
\phantomsection
\addcontentsline{toc}{section}{ Limonade blanche.}
\index{Limonade blanche}

Pour huit personnes prenez :

\footnotesize
\begin{longtable}{rrrrp{16em}}
  & 250 & grammes & de & sucre en morceaux,                                                               \\
  & 200 & grammes & de & sauternes,                                                                       \\
  & \multicolumn{2}{r}{1 litre} & de & lait,                                                              \\
  &     &         &  3 & limons ou 3 citrons.                                                             \\
\end{longtable}
\normalsize

Faites bouillir le lait.

Frottez le sucre sur le zeste des limons ou des citrons pour le parfumer ;
mettez-le dans le lait avec le jus des fruits passés au presse-citron ; laissez
refroidir. Ajoutez au dernier moment le sauternes, mélangez bien et servez
froid.

\section*{\centering Eau de groseilles framboisée.}
\phantomsection
\addcontentsline{toc}{section}{ Eau de groseilles framboisée.}
\index{Eau de groseilles framboisée}

Pour douze personnes prenez :

\footnotesize
\begin{longtable}{rrrrp{16em}}
  & 750 & grammes & de & grammes de groseilles rouges et blanches, mûres à point,                         \\
  & 600 & grammes & de & sucre en morceaux,                                                               \\
  & 250 & grammes & de & framboises bien mûres,                                                           \\
  & \multicolumn{2}{r}{2 litres} & d' & eau filtrée.                                                      \\
\end{longtable}
\normalsize

Écrasez les fruits, mouillezavec l'eau, mélangez bien ; laissez en contact
pendant une demi-heure, puis passez au chinois fin ou à la double mousseline.
Ajoutez le sucre au jus passé, laissez-le fondre en le remuant de temps en
temps.

Tenez au frais jusqu'au moment de servir.

\sk

On peut préparer dans le même esprit des eaux d'autres fruits.

\section*{\centering Eau de cerises.}
\phantomsection
\addcontentsline{toc}{section}{ Eau de cerises.}
\index{Eau de cerises}

On peut préparer l'eau de cerises à chaud ou à froid.

\medskip

Pour douze personnes prenez :

\footnotesize
\begin{longtable}{rrrrp{16em}}
  & 1 200 & grammes & de & cerises mûres,                                                                 \\
  &   500 & grammes & de & sucre en morceaux,                                                             \\
  & \multicolumn{2}{r}{2 litres} & d' & eau filtrée.                                                      \\
\end{longtable}
\normalsize

\textit{Préparation à chaud}. — Enlevez les queues et les noyaux des cerises ;
passez la pulpe au tamis ; versez dessus l’eau bouillante et laissez infuser
pendant {\ppp2\mmm} à {\ppp3\mmm} heures. Filtrez à la chausse, ajoutez le
sucre, laissez-le fondre ; remuez. Tenez au frais.

\medskip

\textit{Préparation à froid}. — Débarrassez les cerises des queues et des
noyaux ; écrasez ces derniers.

Pressez la pulpe au tamis, mettez-la dans une terrine avec les noyaux écrasés ;
laissez macérer le tout pendant 2 heures au moins. Mouillez ensuite avec l'eau.
Filtrez le jus ou passez-le à la double mousseline ; ajoutez le sucre ; remuez
de temps en temps pour activer la dissolution. Tenez au frais.

\medskip

Au moment de servir, parfumez avec un peu de kirsch, au goût.

\sk

\index{Eaux d'autres fruits}
\index{Eaux d'abricots}
On peut préparer dans le même esprit de l'eau d'abricots.

\section*{\centering Eau d'ananas.}
\phantomsection
\addcontentsline{toc}{section}{ Eau d'ananas.}
\index{Eau d'ananas}

Pour douze personnes prenez :

\footnotesize
\begin{longtable}{rrrrp{16em}}
  & 1 000 & grammes & de & ananas frais, pelé à vif ; ou, à défaut, de l'ananas de conserve,              \\
  &    60 & grammes & de & kirsch,                                                                        \\
  & \multicolumn{2}{r}{1 litre} & de & sirop de sucre à 20°,                                              \\
  &       &         &    & eau de Seltz.                                                                  \\
\end{longtable}
\normalsize

Écrasez ou hachez l'ananas, jetez-le dans le sirop bouillant ; éloignez le
poêlon du feu et laissez infuser pendant {\ppp2\mmm} heures. Passez à la
chausse ou à l'étamine et mettez à rafraîchir.

Au dernier moment, ajoutez le kirsch. Servez avec de l’eau de Seltz.

\section*{\centering Eau de melon.}
\phantomsection
\addcontentsline{toc}{section}{ Eau de melon.}
\index{Eau de melon}

Pour douze personnes prenez :

\footnotesize
\begin{longtable}{rrrrp{16em}}
  & 1 000 & grammes & de & pulpe d'un melon mûr à point,                                                  \\
  &    50 & grammes & d' & eau de fleurs d'oranger,                                                       \\
  & \multicolumn{2}{r}{1 litre} & de & sirop de sucre à 20°,                                              \\
  &       &         &    & eau de Seltz.                                                                  \\
\end{longtable}
\normalsize

Même préparation que pour l'eau d’ananas.

\section*{\centering Orgeat.}
\phantomsection
\addcontentsline{toc}{section}{ Orgeat.}
\index{Orgeat}

Prenez du sirop d'orgeat, \hyperlink{p0989}{p. \pageref{pg0989}}, versez-en plus
ou moins, au goût, dans des verres et remplissez avec de l'eau très fraîche.
Remuez.

On peut mélanger à la préparation une certaine quantité, au goût, de sirop de
grenades, \hyperlink{p0988}{p. \pageref{pg0988}}. C'est exquis.

\section*{\centering Vin blanc au jus de fraises.}
\phantomsection
\addcontentsline{toc}{section}{ Vin blanc au jus de fraises.}
\index{Vin blanc au jus de fraises}

Pour douze personnes prenez :

\footnotesize
\begin{longtable}{rrrrp{16em}}
  &   500 & grammes & de & jus de fraises Héricart,                                                       \\
  &   300 & grammes & de & sucre en morceaux,                                                             \\
  & \multicolumn{2}{r}{2 bouteilles} & de & vin blanc de Graves,                                          \\
  &       &         &    & eau gazeuse.                                                                   \\
\end{longtable}
\normalsize

Mélangez vin blanc, sucre et jus de fraises ; laissez en contact au frais
pendant deux heures environ.

Servez avec de l'eau gazeuse très fraîche.

\sk

On peut préparer dans le même esprit d'autres vins blancs pas trop secs à
d'autres jus de fruits.

\medskip

Toutes ces boissons sont très agréables.

\section*{\centering Vin du Rhin à l'orange et au citron.}
\phantomsection
\addcontentsline{toc}{section}{ Vin du Rhin à l'orange et au citron.}
\index{Vin du Rhin à l'orange et au citron}

Pour douze personnes prenez :

\footnotesize
\begin{longtable}{rrrrp{16em}}
  & 125 & grammes & de & gelée de groseilles ou d'ananas,                                                 \\
  & \multicolumn{2}{r}{2 bouteilles} & de & vin du Rhin,                                                  \\
  &     &         &  2 & oranges,                                                                         \\
  &     &         &  1 & citron,                                                                          \\
  &     &         &    & sucre.                                                                           \\
\end{longtable}
\normalsize

Parfumez du sucre, au goût, en le frottant avec les zestes des oranges et du
citron. Passez les fruits au presse-citron, mélangez le jus avec le vin,
ajoutez la gelée et le sucre ; laissez fondre ; remuez pour homogénéiser le
tout.

Tenez au frais.

\section*{\centering Café glacé.}
\phantomsection
\addcontentsline{toc}{section}{ Café glacé.}
\index{Café glacé}

Pour douze personnes prenez :

\footnotesize
\begin{longtable}{rrrrp{16em}}
  & 750 & grammes & de & sucre en morceaux,                                                               \\
  & 500 & grammes & de & crème double bien fraîche,                                                       \\
  & \multicolumn{2}{r}{1 litre} & de & lait,                                                              \\
  & \multicolumn{2}{r}{3/4 à 1 litre} & de & café concentré,                                              \\
  &     &         &    & vanille.                                                                         \\
\end{longtable}
\normalsize

Faites bouillir le lait avec de la vanille et le sucre ; laissez-le refroidir,
puis ajoutez le café et la crème.

Glacez à la sorbétière en tenant l'appareil presque liquide.

Servez dans des tasses froides.

\section*{\centering Cocktails.}
\phantomsection
\addcontentsline{toc}{section}{ Cocktails.}
\index{Cocktails}
\index{Cocktails (Définition des)}
\index{Définition des cocktails}

Les cocktails sont des boissons alcooliques glacées, d'origine américaine. On
les sert généralement comme apéritifs.

La caractéristique de leur composition est là présence du bitter angustura.

On peut imaginer autant de cocktails différents qu'il y a de combinaisons
possibles de liqueurs alcooliques.

Le mode de préparation est toujours le même. On met de la glace pilée dans un
shaker\footnote{Le shaker (littéralement secoueur) est un appareil composé de
deux gobelets réunis par la grande base.} jusqu'à moitié de sa hauteur, puis
les différents éléments du cocktail ; on secoue pour glacer le liquide ; on
passe dans des gobelets, des verres à bordeaux ou des verres pointus et l'on
sert avec des pailles.

\section*{\centering Cocktail à l’absinthe.}
\phantomsection
\addcontentsline{toc}{section}{ Cocktail à l’absinthe.}
\index{Cocktail à l’absinthe}

Pour chaque personne prenez :

\footnotesize
\begin{longtable}{rrrrp{16em}}
  & 20 & grammes & d' & absinthe,                                                                         \\
  & 20 & grammes & de & sirop de sucre,                                                                   \\
  &  2 & grammes & d' & anisette,                                                                         \\
  & \multicolumn{2}{r}{6 gouttes} & de & bitter angustura,                                                \\
  &    &         &    & glace pilée,                                                                      \\
  &    &         &    & zeste d'un citron.                                                                \\
\end{longtable}
\normalsize

Mettez dans un shaker, secouez pour glacer, passez dans un verre à cocktail,
ajoutez le jus du zeste d'un citron et finissez d'emplir avec de l'eau fraîche
ou de l'eau de Seltz.

Servez avec des petites pailles.

\section*{\centering Cocktail au vermouth.}
\phantomsection
\addcontentsline{toc}{section}{ Cocktail au vermouth.}
\index{Cocktail au vermouth}

Pour chaque personne prenez :

\footnotesize
\begin{longtable}{rrrrp{16em}}
  & 75 & grammes & de & vermouth de Turin,                                                                \\
  & 10 & grammes & de & sirop de sucre,                                                                   \\
  &  6 & grammes & de & curaçao rouge,                                                                    \\
  & \multicolumn{2}{r}{6 gouttes} & de & marasquin de Zara,                                               \\
  & \multicolumn{2}{r}{4 gouttes} & de & bitter angustura,                                                \\
  &    &         &    & glace pilée,                                                                      \\
  &    &         &    & zeste d'un citron.                                                                \\
\end{longtable}
\normalsize

Faites la préparation dans un shaker, passez dans un verre à cocktail ou dans
un verre à bordeaux, ajoutez le jus du zeste d'un citron et servez avec des
petites pailles.

\section*{\centering Cocktail au gin.}
\phantomsection
\addcontentsline{toc}{section}{ Cocktail au gin.}
\index{Cocktail au gin}

Pour chaque personne prenez :

\footnotesize
\begin{longtable}{rrrrp{16em}}
  & 30 & grammes & de & gin,                                                                              \\
  & 10 & grammes & de & curaçao,                                                                          \\
  & 10 & grammes & de & sirop de sucre,                                                                   \\
  & \multicolumn{2}{r}{6 gouttes} & de & bitter angustura,                                                \\
  &    &         &    & glace pilée,                                                                      \\
  &    &         &    & zeste d'un citron.                                                                \\
\end{longtable}
\normalsize

Préparez dans un shaker, servez comme précédemment.

\section*{\centering Cocktail au cognac.}
\phantomsection
\addcontentsline{toc}{section}{ Cocktail au cognac.}
\index{Cocktail au cognac}

Pour chaque personne prenez :

\footnotesize
\begin{longtable}{rrrrp{16em}}
  & 30 & grammes & de & cognac,                                                                           \\
  &  5 & grammes & de & sirop d'ananas ou d'orgeat,                                                       \\
  &  5 & grammes & de & sirop de fraises,                                                                 \\
  & \multicolumn{2}{r}{6 gouttes} & de & marasquin,                                                       \\
  & \multicolumn{2}{r}{4 gouttes} & de & bitter angustura,                                                \\
  &    &         &    & glace pilée,                                                                      \\
  &    &         &    & zeste d'un citron.                                                                \\
\end{longtable}
\normalsize

Préparez et servez comme ci-dessus.

\section*{\centering Cocktail au champagne.}
\phantomsection
\addcontentsline{toc}{section}{ Cocktail au champagne.}
\index{Cocktail au champagne}

Pour chaque personne prenez :

\footnotesize
\begin{longtable}{rrrrp{16em}}
  &  6 & grammes & de & curaçao,                                                                          \\
  &  5 & grammes & de & sirop de sucre,                                                                   \\
  & \multicolumn{2}{r}{6 gouttes} & de & bitter angustura,                                                \\
  &    &         &    & glace pilée,                                                                      \\
  &    &         &    & zeste d'un citron,                                                                \\
  &    &         &    & champagne sec.                                                                    \\
\end{longtable}
\normalsize

Dans la préparation des cocktails au champagne, on ne se sert pas du shaker.

Mettez dans un grand gobelet en cristal moitié de la hauteur de glace pilée, le
sirop de sucre, le curaçao, le bitter et le jus du zeste d'un citron ;
emplissez avec du champagne, mélangez et servez avec de grandes pailles.

\section*{\centering Cobblers.}
\phantomsection
\addcontentsline{toc}{section}{ Cobblers.}
\index{Cobblers}
\index{Définition des cobblers}
\index{Cobblers (Définition des)}

Les cobblers sont des boissons alcooliques, glacées, d'origine américaine,
différentes des cocktails. Elles sont le plus souvent garnies de fruits et ne
contiennent jamais d'angustura.

On les prépare dans de grands gobelets et on les sert avec des grandes pailles.

\section*{\centering Cobbler au champagne.}
\phantomsection
\addcontentsline{toc}{section}{ Cobbler au champagne.}
\index{Cobbler au champagne}

Pour chaque personne prenez :

\footnotesize
\begin{longtable}{rrrrp{16em}}
  & 10 & grammes & de & sucre en poudre,                                                                  \\
  & \multicolumn{2}{r}{1 tranche} & d' & orange,                                                          \\
  &    &         &    & zeste d'un citron,                                                                \\
  &    &         &    & glace pilée,                                                                      \\
  &    &         &    & champagne sec.                                                                    \\
  &    &         &    & fruits de saison.                                                                 \\
\end{longtable}
\normalsize

Mettez de la glace pilée dans un grand gobelet en cristal jusqu'à moitié de sa
hauteur, ajoutez le sucre, la tranche d'orange, le jus du zeste de citron,
emplissez avec du champagne sec, garnissez avec des cubes d'ananas ou d'autres
fruits et servez avec de grandes pailles.

\section*{\centering Cobbler au sauternes.}
\phantomsection
\addcontentsline{toc}{section}{ Cobbler au sauternes.}
\index{Cobbler au sauternes}

Pour chaque personne prenez :

\footnotesize
\begin{longtable}{rrrp{16em}}
    125 & grammes & de & sauternes,                                                                       \\
     35 & grammes & de & sirop de grenadine,                                                              \\
      5 & grammes & de & sucre en poudre,                                                                 \\
        &         &    & glace pilée,                                                                     \\
        &         &    & cerises ou fruits de saison.                                                     \\
\end{longtable}
\normalsize

Emplissez à moitié de glace pilée un grand gobelet en cristal, ajoutez le sirop
de grenadine, le sucre en poudre, le sauternes, mélangez avec une cuiller
à soda, garnissez avec des cerises ou des fruits de saison et servez avec de
grandes pailles.

\section*{\centering Sherry-cobbler.}
\phantomsection
\addcontentsline{toc}{section}{ Sherry-cobbler.}
\index{Sherry-cobbler}

Pour chaque personne prenez :

\footnotesize
\begin{longtable}{rrrp{16em}}
    150 & grammes & de & xérès blond,                                                                     \\
     25 & grammes & de & curaçao,                                                                         \\
     20 & grammes & de & cognac,                                                                          \\
     12 & grammes & de & porto rouge,                                                                     \\
     10 & grammes & de & sucre en poudre,                                                                 \\
        &         &  2 & tranches d'orange épépinées,                                                     \\
        &         &    & glace pilée.                                                                     \\
\end{longtable}
\normalsize

Emplissez aux trois quarts un grand gobelet en cristal de glace pilée, ajoutez
xérès, curacao, cognac, sucre en poudre ; versez le tout dans un shaker,
secouez pour glacer, puis remettez le mélange dans le gobelet. Garnissez avec
les tranches d'orange sur lesquelles vous verserez le porto, sans mélanger.
Servez avec de longues pailles.

\section*{\centering Porto-cobbler.}
\phantomsection
\addcontentsline{toc}{section}{ Porto-cobbler.}
\index{Porto-cobbler}

Pour chaque personne prenez :

\footnotesize
\begin{longtable}{rrrp{16em}}
    125 & grammes & de & porto rouge,                                                                     \\
     30 & grammes & de & sirop de groseilles,                                                             \\
        &         &  4 & belles fraises Héricart ou 8 fraises des quatre saisons,                         \\
        &         &    & glace pilée.                                                                     \\
\end{longtable}
\normalsize

Mettez le porto et le sirop de groseilles dans un grand gobelet en cristal ;
emplissez-le de glace pilée, remuez bien ; garnissez avec les fraises et servez
avec de grandes pailles.

\section*{\centering Soyer.}
\phantomsection
\addcontentsline{toc}{section}{ Soyer.}
\index{Soyer}

Le soyer est une sorte de cobbler.

\medskip

Pour chaque personne prenez :

\footnotesize
\begin{longtable}{rrrp{16em}}
     60 & grammes & de & sirop de grenadine,                                                              \\
     15 & grammes & de & fine champagne,                                                                  \\
        &         &  1 & tranche d'orange et 1 tranche de citron ou 8 à 10 grains de raisin,              \\
        &         &    & champagne sec,                                                                   \\
        &         &    & glace pilée.                                                                     \\
\end{longtable}
\normalsize

Mettez de la glace pilée dans un grand gobelet de cristal jusqu'aux trois quarts
de la hauteur ; ajoutez la fine champagne, le sirop de grenadine ; emplissez avec
du champagne sec ; mélangez bien avec une cuiller à soda. Garnissez le breuvage
avec la tranche de citron et la tranche d'orange, ou avec des grains de raisin bien
mûr, et servez avec de longues pailles.

\section*{\centering Bishop.}
\phantomsection
\addcontentsline{toc}{section}{ Bishop.}
\index{Bishop}

Le bishop est une autre variété de cobbler.

\medskip

Pour chaque personne prenez :

\footnotesize
\begin{longtable}{rrrp{16em}}
     20 & grammes & de & jus d'orange.                                                                    \\
     12 & grammes & de & sucre en poudre,                                                                 \\
      5 & grammes & de & vieux rhum,                                                                      \\
      4 & grammes & de & jus de citron,                                                                   \\
        &         &    & vin vieux de Bourgogne ou de Bordeaux,                                           \\
        &         &    & eau de Seltz,                                                                    \\
        &         &    & glace pilée,                                                                     \\
        &         &    & fruits de saison.                                                                \\
\end{longtable}
\normalsize

Mettez dans un grand verre trois quarts de glace pilée, les jus d'orange et
de citron, le sucre en poudre, le rhum et un peu d’eau de Seltz, mélangez,
emplissez le verre avec du vin. Ornez avec des fruits de saison et servez avec
de longues pailles.

\section*{\centering Egg nogg.}
\phantomsection
\addcontentsline{toc}{section}{ Egg nogg.}
\index{Egg nogg}

Les egg noggs sont des boissons américaines glacées servies souvent dans les
goirées.

\medskip

Pour chaque personne prenez :

\footnotesize
\begin{longtable}{rrrp{16em}}
     25 & grammes & de & bon cognac,                                                                      \\
     12 & grammes & de & vieux rhum,                                                                      \\
     10 & grammes & de & sucre en poudre,                                                                 \\
        &         &  1 & jaune d'œuf frais,                                                               \\
        &         &    & lait non écrémé,                                                                 \\
        &         &    & glace pilée,                                                                     \\
        &         &    & muscade ou cannelle.                                                             \\
\end{longtable}
\normalsize

Mettez dans un grand gobelet un tiers de sa hauteur de glace pilée, ajoutez le
sucre, le cognac, le rhum, le jaune d'œuf, emplissez avec du lait. Versez le
tout dans un shaker, remuez vigoureusement, passez ensuite dans un grand verre,
saupoudrez d'un peu de muscade râpée ou de cannelle en poudre ou les deux
ensemble et servez avec de longues pailles.

\sk

Comme variante, on peut ajouter à la formule ci-dessus {\ppp40\mmm} grammes de
madère.

\sk

On prépare des egg noggs au xérès, au madère, au whisky, etc.

\section*{\centering Egg lemonade.}
\phantomsection
\addcontentsline{toc}{section}{ Egg lemonade.}
\index{Egg lemonade}

Pour chaque personne prenez :

\footnotesize
\begin{longtable}{rrrp{16em}}
     12 & grammes &  de & sucre en poudre,                                                                \\
        &         &   1 & œuf frais,                                                                      \\
        &         & 1/2 & citron,                                                                         \\
        &         &     & glace pilée.                                                                    \\
\end{longtable}
\normalsize

Délayez dans un peu d'eau le sucre et l'œuf ; mettez le mélange dans un shaker
avec le jus de citron, trois quarts de glace pilée et suffisamment d'eau,
secouez vivement ; passez dans un gobelet en cristal et servez avec des
pailles.

\section*{\centering Boisson américaine à l'œuf et au porto.}
\phantomsection
\addcontentsline{toc}{section}{ Boisson américaine à l'œuf et au porto.}
\index{Boisson américaine à l'œuf et au porto}

Pour chaque personnes prenez :

\footnotesize
\begin{longtable}{rrrp{16em}}
     80 & grammes & de & porto rouge,                                                                     \\
     20 & grammes & de & fine champagne,                                                                  \\
     12 & grammes & de & sucre en poudre,                                                                 \\
        &         &  1 & œuf frais,                                                                       \\
        &         &    & glace pilée,                                                                     \\
        &         &    & muscade et cannelle.                                                             \\
\end{longtable}
\normalsize

Même préparation que celle de l'egg nogg.

\section*{\centering Cups.}
\phantomsection
\addcontentsline{toc}{section}{ Cups.}
\index{Cups}
\index{Cups (Définition des)}

Les cups sont des boissons froides au vin, très appréciées en Angleterre. Elles
sont servies surtout dans les garden party.

Le plus souvent, on les garnit de fruits.

\section*{\centering Claret-cup.}
\phantomsection
\addcontentsline{toc}{section}{ Claret-cup.}
\index{Claret-cup}

Pour douze personnes prenez :

\footnotesize
\begin{longtable}{rrrrp{16em}}
  & 250 & grammes & de & soda-water ou d'eau d'Apollinaris,                                               \\
  & 125 & grammes & de & cerises mûres,                                                                   \\
  & 100 & grammes & de & marasquin,                                                                       \\
  & 100 & grammes & de & porto,                                                                           \\
  &  60 & grammes & de & sucre,                                                                           \\
  & \multicolumn{2}{r}{1 bouteille} & de & vin de Bourgogne ou de Bordeaux,                               \\
  &     &         &  1 & orange,                                                                          \\
  &     &         &  1 & citron.                                                                          \\
\end{longtable}
\normalsize

Mélangez tous les éléments moins les fruits ; tenez à la glace. Au dernier
moment, ajoutez les cerises, l'orange et le citron coupés en rondelles et
épépinés.

Servez dans des verres mousseline.

\section*{\centering Champagne-cup.}
\phantomsection
\addcontentsline{toc}{section}{ Champagne-cup.}
\index{Champagne-cup}

Pour douze personnes prenez :

\footnotesize
\begin{longtable}{rrrrp{16em}}
  & 250 & grammes & de & vin de Tokay,                                                                    \\
  &  80 & grammes & de & sirop d'ananas,                                                                  \\
  &  60 & grammes & de & curaçao de Hollande,                                                             \\
  &  50 & grammes & de & chartreuse verte,                                                                \\
  & \multicolumn{2}{r}{1 bouteille} & de & champagne sec,                                                 \\
  &     &         &  2 & oranges,                                                                         \\
  &     &         &  1 & citron,                                                                          \\
  &     &         &    & eau de Seltz ou d’Apollinaris, au goût,                                          \\
  &     &         &    & ananas coupé en tranches,                                                        \\
  &     &         &    & fraises des bois,                                                                \\
  &     &         &    & sucre en poudre.                                                                 \\
\end{longtable}
\normalsize

Mettez dans une terrine vin de Tokay, curaçao, chartreuse, sirop d'ananas, eau
de Seltz ou d'Apollinaris, oranges et citron coupés en rouelles, couvrez la
terrine et laissez en contact pendant deux heures.

Passez le mélange dans un autre vase, ajoutez le champagne, les rouelles
d'ananas et les fraises ; faites rafraîchir dans de la glace pilée. Adoucissez
avec un peu de sucre, au goût.

Servez dans des verres mousseline.

\section*{\centering Bavaroises.}
\phantomsection
\addcontentsline{toc}{section}{ Bavaroises.}
\index{Bavaroises}
\index{Bavaroises (Définition des)}
\index{Définition des bavaroises}


Les bavaroises sont des boissons chaudes, légèrement alcoolisées que l'on
prépare au thé, au chocolat, au café ou au lait. Ce sont d'excellents breuvages
pour soirées d'hiver.

La bavaroise au thé, la première en date, a été créée au
{\sc xvii}\textsuperscript{e} siècle à la cour du roi de Bavière. En voici la
formule.

\medskip

Pour douze personnes prenez :

\footnotesize
\begin{longtable}{rrrp{16em}}
    500 & grammes & d' & infusion chaude de thé noir,                                                     \\
    500 & grammes & de & lait bouillant,                                                                  \\
    250 & grammes & de & sucre en poudre,                                                                 \\
    200 & grammes & de & kirsch ou de rhum,                                                               \\
    100 & grammes & de & sirop de capillaire,                                                             \\
        &         &  8 & jaunes d'œufs frais.                                                             \\
\end{longtable}
\normalsize

Travaillez dans une casserole les jaunes d'œufs avec le sucre, ajoutez le thé,
le lait et le sirop de capillaire en fouettant pour rendre le mélange
mousseux ; aromatisez à la fin avec le kirsch ou le rhum et servez.

\sk

\index{Bavaroise au chocolat}
Pour faire une bavaroise au chocolat, on remplacera dans la formule précédente
l'infusion de thé et le lait par du chocolat au lait obtenu en faisant fondre
{\ppp200\mmm} grammes de chocolat dans un litre de lait.

\sk

\index{Bavaroise au café}
\index{Bavaroise au thé}
En remplaçant l'infusion de thé par une infusion de café plus ou moins forte,
on aura une bavaroise au café.

\sk

\index{Bavaroise au lait}
Enfin, on obtient la bavaroise au lait en remplaçant l'infusion de thé par du
lait aromatisé avec de la vanille, du citron, de l'orange, etc.

\section*{\centering Vin de Bourgogne chaud, à l'orange.}
\phantomsection
\addcontentsline{toc}{section}{ Vin de Bourgogne chaud, à l'orange.}
\index{Vin de Bourgogne chaud, à l'orange}

Pour douze personnes prenez :

\footnotesize
\begin{longtable}{rrrrp{16em}}
  & \multicolumn{2}{r}{2 bouteilles} & de & vin de Bourgogne,                                             \\
  &     &         &  3 & oranges,                                                                         \\
  &     &         &  2 & zestes d'oranges,                                                                \\
  &     &         &    & sucre au goût.                                                                   \\
\end{longtable}
\normalsize

Faites dissoudre le sucre dans un peu d'eau bouillante, ajoutez les zestes
d'oranges, laissez infuser pendant un quart d'heure. Passez, ajoutez le vin
chaud et servez avec une tranche d'orange épépinée dans chaque verre.

\section*{\centering Vin de Bordeaux chaud, au citron et à la cannelle.}
\phantomsection
\addcontentsline{toc}{section}{ Vin de Bordeaux chaud, au citron et à la cannelle.}
\index{Vin de Bordeaux chaud, au citron et à la cannelle}

Pour douze personnes prenez :

\footnotesize
\begin{longtable}{rrrrp{16em}}
  & 400 & grammes & de & sucre,                                                                           \\
  & \multicolumn{2}{r}{2 bouteilles} & de & vin de Bordeaux,                                              \\
  &     &         &  3 & citrons,                                                                         \\
  &     &         &  2 & zestes de citrons,                                                               \\
  &     &         &  2 & clous de girofle,                                                                \\
  &     &         &  1 & petit morceau de cannelle,                                                       \\
  &     &         &  1 & petit morceau de macis.                                                          \\
\end{longtable}
\normalsize

Chauffez le tout ensemble, moins les trois citrons, jusqu'à ce que le vin se
couvre de mousse ; passez au chinois et servez avec une tranche de citron
épépiné dans chaque verre.

\section*{\centering Punchs.}
\phantomsection
\addcontentsline{toc}{section}{ Punchs.}
\index{Punchs}
\index{Définition des punchs chauds}

Le punch est un breuvage d'origine persane. Son nom dérive d'un mot persan
signifiant cing et rappelant le nombre des éléments qui le constituaient
primitivement : alcool, thé, sucre, citron, cannelle. Aujourd'hui, on désigne
sous ce nom des boissons alcoolisées ne renfermant que quelques-uns des
éléments primitifs.

Voici quelques formules de punchs actuellement en usage en France et
à l'étranger.

\section*{\centering Punch au rhum ou au kirsch.}
\phantomsection
\addcontentsline{toc}{section}{ Punch au rhum ou au kirsch.}
\index{Punch au rhum ou au kirsch}

Pour douze personnes prenez :

\footnotesize
\begin{longtable}{rrrrp{16em}}
  & 750 & grammes & de & rhum ou de kirsch,                                                               \\
  & 500 & grammes & de & sucre,                                                                           \\
  & \multicolumn{2}{r}{1 litre} & d' & infusion de thé noir, très chaude.                                 \\
\end{longtable}
\normalsize

Réunisez le tout, faites flamber, versez dans des verres ; mettez dans chacun
une tranche de citron épépiné, si c'est un punch au rhum, et servez chaud.

\section*{\centering Punch au lait.}
\phantomsection
\addcontentsline{toc}{section}{ Punch au lait.}
\index{Punch au lait}

Pour douze personnes prenez :

\footnotesize
\begin{longtable}{rrrrp{16em}}
  & 500 & grammes & de & sucre,                                                                           \\
  & 250 & grammes & de & rhum.                                                                            \\
  & 250 & grammes & de & cognac,                                                                          \\
  & \multicolumn{2}{r}{1 litre} & de & lait,                                                              \\
  &     &         &    & vanille où cannelle.                                                             \\
\end{longtable}
\normalsize

Faites bouillir le lait avec le sucre, de la vanille ou de la cannelle.

Flambez le rhum et le cognac réunis. Versez-les ensuite dans le lait et servez
chaud.

\section*{\centering Punch marquise.}
\phantomsection
\addcontentsline{toc}{section}{ Punch marquise.}
\index{Punch marquise}

Pour douze personnes prenez :

\footnotesize
\begin{longtable}{rrrrp{16em}}
  & 250 & grammes & de & cognac,                                                                          \\
  & 250 & grammes & de & sucre,                                                                           \\
  & \multicolumn{2}{r}{1 litre} & de & vin de Sauternes,                                                  \\
  &     &         &  3 & citrons,                                                                         \\
  &     &         &  1 & clou de girofle.                                                                 \\
\end{longtable}
\normalsize

Mettez dans une casserole le vin, le sucre, du zeste de citron et le clou de
girofle. Chauffez jusqu à ce que le vin se couvre d'une fine mousse blanche ;
passez-le.

Chauffez le cognac, mettez-le dans un bol à punch, ajoutez le vin et faites
flamber le tout jusqu'à extinction.

Servez chaud avec une tranche de citron dans chaque verre.

\section*{\centering Punch truffé.}
\phantomsection
\addcontentsline{toc}{section}{ Punch truffé.}
\index{Punch truffé}

Pour six à huit personnes prenez :

\footnotesize
\begin{longtable}{rrrrp{16em}}
  & 350 & grammes &  de & fine champagne,                                                                 \\
  & 350 & grammes &  de & vieux rhum,                                                                     \\
  & 250 & grammes &  de & sucre,                                                                          \\
  & 120 & grammes &  de & vin de Malaga,                                                                  \\
  &     &         &   1 & belle truffe noire du Périgord,                                                 \\
  &     &         &   1 & citron,                                                                         \\
  &     &         & 1/4 & noix muscade.                                                                   \\
\end{longtable}
\normalsize

Mettez dans un bol à punch la fine champagne, la muscade et le sucre, faites
flamber, mélangez bien. Lorsque le sucre sera dissous, ajoutez le rhum et le
jus du citron ; activez la flamme. En même temps, faites cuire la truffe dans
le malaga, retirez-la, puis ajoutez le malaga au mélange rhum et fine
champagne.

Coupez la truffe en tranches minces, mettez une tranche dans chaque verre de
punch et servez chaud.

\section*{\centering Punch à l'anglaise.}
\phantomsection
\addcontentsline{toc}{section}{ Punch à l'anglaise.}
\index{Punch à l'anglaise}

Pour douze personnes prenez :

\footnotesize
\begin{longtable}{rrrrp{16em}}
  & 1 000 & grammes & d' & infusion de thé noir, très chaude,                                             \\
  &   300 & grammes & de & sirop de sucre à 30°,                                                          \\
  &   150 & grammes & de & rhum de la Jamaïque,                                                           \\
  &   150 & grammes & d' & eau-de-vie à 80°,                                                              \\
  &       &         &  4 & clous de girofle,                                                              \\
  &       &         &  2 & oranges,                                                                       \\
  &       &         &    & zeste de citron, au goût.                                                      \\
\end{longtable}
\normalsize

Mettez dans un poêlon le rhum, l'eau-de-vie, le sirop, les clous de girofle, du
zeste de citron ; chauffez ; ajoutez ensuite l'infusion de thé ; faites
flamber. Servez en mettant dans chaque verre une tranche d'orange épépinée.

\section*{\centering Grogs.}
\phantomsection
\addcontentsline{toc}{section}{ Grogs.}
\index{Grogs}
\index{Définition des grogs}

Les grogs sont des boissons alcoolisées, sucrées, aromatisées avec du citron.

On les sert chaudes ou froides.

\section*{\centering Grog ordinaire.}
\phantomsection
\addcontentsline{toc}{section}{ Grog ordinaire.}
\index{Grog ordinaire}

Pour douze personnes prenez :

\footnotesize
\begin{longtable}{rrrrp{16em}}
  & 750 & grammes & de & fine champagne, cognac, rhum, kirsch, whisky ou gin, au goût,                    \\
  & \multicolumn{2}{r}{2 litres} & d' & eau chaude ou froide,                                             \\
  &     &         &  3 & citrons,                                                                         \\
  &     &         &    & sirop de sucre à 30°.                                                            \\
\end{longtable}
\normalsize

Versez dans chaque verre un douzième du spiritueux choisi, ajoutez du sirop de
sucre au goût, de l'eau bouillante ou de l’eau froide, une tranche de citron
sans pépins et servez.

\section*{\centering Grog américain.}
\phantomsection
\addcontentsline{toc}{section}{ Grog américain.}
\index{Grog américain}

Pour douze personnes prenez :

\footnotesize
\begin{longtable}{rrrrp{16em}}
  & 300 & grammes & de & sirop de sucre à 30°,                                                            \\
  &   5 & grammes & de & macis,                                                                           \\
  & \multicolumn{2}{r}{1 litre} & d' & eau-de-vie à 80°,                                                  \\
  &     &         &  3 & clous de girofle,                                                                \\
  &     &         &  3 & zestes d'oranges bigarrades,                                                     \\
  &     &         &    & cannelle.                                                                        \\
\end{longtable}
\normalsize


Mettez le macis, les clous de girofle, les zestes d'oranges et de la cannelle,
au goût, dans l'eau-de-vie ; laissez en contact pendant quinze jours.

Passez au travers d'un chinois fin ou d'une double mousseline, ajoutez le sirop
de sucre, mélangez. Réservez.

Pour servir, étendez la préparation avec un litre d’eau bouillante et servez
très chaud.

\section*{\centering Grog américain.}
\phantomsection
\addcontentsline{toc}{section}{ Grog américain.}
\index{Grog américain}

\begin{center}
\textit{(Autre formule).}
\end{center}

\medskip

Pour douze personnes prenez :

\footnotesize
\begin{longtable}{rrrp{16em}}
    250 & grammes & de & rhum vieux,                                                                      \\
    250 & grammes & de & bon cognac,                                                                      \\
    250 & grammes & de & sirop de sucre,                                                                  \\
    250 & grammes & d' & infusion concentrée de thé noir,                                                 \\
     90 & grammes & de & curaçao rouge,                                                                   \\
        &         &  3 & citrons.                                                                         \\
\end{longtable}
\normalsize

Mélangez bien le tout, moins les citrons.

Au moment de servir, chauffez le mélange, étendez-le avec un même volume d'eau
bouillante. Versez dans des verres, garnissez chaque verre avec une tranche de
citron épépiné et servez très chaud.

\section*{\centering Grog indien.}
\phantomsection
\addcontentsline{toc}{section}{ Grog indien.}
\index{Grog indien}

Le grog indien est une variante du grog américain, aromatisée avec de la vanille
et relevée avec du piment.

\sk
