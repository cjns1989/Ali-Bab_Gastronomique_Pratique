\setlength{\epigraphwidth}{.5\textwidth}
\setlength{\epigraphrule}{0\textwidth}
\renewcommand{\epigraphsize}{\footnotesize}
\epigraph{Fromage, poésie, bouquet de nos repas !         \\
          Que sentirait la vie, si l'on ne l'avait pas ?}
          {(V. Meusy, Chansons du Chat Noir.)}

Les fromages sont des aliments albuminoïdes très digestibles\footnote{Associés
au pain, ils constituent un aliment presque complet.} et aussi des peptogènes
stimulant la digestion\footnote{Ils favorisent notamment l'assimilation des
graisses et des hydrates de carbone.} ; c'est à ce dernier titre qu'ils sont
servis à la fin des repas.

Je crois devoir en dire quelques mots.

L'art du fromager remonte à plus de deux mille ans. Purement empirique
à l'origine, il tend de plus en plus à s'orienter dans la voie scientifique, et
des savants très distingués y ont consacré des mémoires
importants\footnote{L'un des plus récents est : « Technique fromagère théorique
et pratique », par P. Mazé, chef de service à l'Institut Pasteur
(\textit{Annales de l'Institut Pasteur}, {\ppp1910\mmm}).}.

Aujourd'hui, un bon fromager doit savoir choisir\footnote{La qualité des
fromages est fonction de la qualité des laits employés à leur préparation et,
toutes choses égales d'ailleurs, on obtient les meilleurs avec le lait
provenant d'animaux élevés sur de bons pâturages. Pendant longtemps on n'en
employa pas d'autre en France et l'antique réputation des fromages français
tenait au moins en grande partie à cette cause ; mais depuis quelque temps, au
risque de tuer la poule aux œufs d'or, les producteurs cherchent avant tout
à augmenter le rendement des bêtes laitières et ils y arrivent, au détriment de
la qualité, en nourrissant leurs animaux avec des tourteaux et de la pulpe de
betteraves.} et mélanger, le cas échéant, les laits qu'il emploie ; déterminer
dans chaque cas particulier les proportions de présure, les conditions
physiques et chimiques du milieu (température, acidité) pour assurer une
coagulation convenable ; égoutter le caillé qui contient presque toute la
caséine du lait et la majeure partie du beurre, en vue d'obtenir un produit
ferme, compact, homogène, élastique, ne renfermant que la quantité de sérum
nécessaire à la bonne marche des fermentations ; favoriser au séchoir ou à la
cave la production de certaines espèces microbiennes et entraver le
développement de certaines autres de façon à empêcher les fromages en
préparation de tomber malades, car, dans l’état actuel de la science, les
maladies des fromages sont incurables.

\index{Classification des fromages}
Tous les fromages sont à base de lait caillé ; ils peuvent être groupés en
quatre grandes classes :

1° les fromages de caillé frais ;

2° les fromages de caillé fermenté à basse température et sans pression
extérieure, à pâte molle ;

3° les fromages de caillé fermenté à température plus ou moins élevée, à pâte
dure, comprimés ;

4° les fromages persillés.

\medskip

\textit{A}. — Dans la classe des fromages de caillé frais se trouvent : le
fromage blanc, simple caillé de lait de vache égoutté ; le fromage à la crème,
le petit-suisse, caillés de lait de vache battus avec de la crème, bons surtout
de mai à septembre ; le gournay, le bondon de Neufchâtel-en-Bray, caillés
moulés de lait de vache et de brebis ou de lait de brebis seulement,
additionnés de crème et ensemencés d'une moisissure qui est le plus souvent le
\textit{penicilium candidum}, bons de mai à octobre ; le fromage demi-sel,
fromage à la crème légèrement salé, et le vacherin des Alpes, caillé malaxé
avec de la crème, qu'on conserve pendant quelques jours pour lui communiquer
une saveur aigrelette.

\medskip

\textit{B}. — Dans la classe des fromages de caillé fermenté à pâte molle, il
faut citer : le brie, l'un des fromages français les plus renommés, le
camembert, le coulommiers, le port-salut, le livarot, le maroilles, le géromé,
le munster, la tome, le mont-dore, le Saint-Marcellin, le banon et le
chabichou.

Le brie, bon durant cinq mois de l’année, de décembre à avril, doit être
moelleux, mais non coulant, et sa pâte couleur jaune pâte ; le fromage
d'Olivet, près d'Orléans, grande réputation régionale, appartient à la famille
du brie. Le camembert, de fabrication analogue à celle du brie, n'en diffère
guère que par la présence dans sa pâte du ferment lactique \textit{oïdium
camemberti}, qui lui donne probablement son goût spécial ; le fromage de
monsieur Fromage, fabriqué par lui-même, est une variété de camembert de haut
goût. Le coulommiers\footnote{Le coulommiers double crème, fromage de luxe,
frais, est préparé à basse température avec du lait fraîchement trait, dont le
caillé est malaxé avec de la crème.} est un fromage de printemps et d'été,
à pâte très homogène, gras, onctueux au toucher ; sa croûte est d’un blanc
grisâtre. Le port-salut est un fromage d'été, salé, à pâte jaune, lisse. Le
livarot et le maroilles, fromages d'automne et d'hiver, n'ont qu'un nombre
d'amateurs limité à cause de leur odeur plus ou moins ammoniacale. Le géromé
est un fromage salé mélangé de cumin, bon d'octobre à mars. Le munster est un
fromage d'hiver de haut goût que l'on consomme le plus souvent saupoudré de
cumin ; la tome de la vallée d'Herens est aromatique et grasse. Enfin le
mont-dore, le Saint-Marcellin, le banon et le chabichou sont d'excellents
fromages de chèvre ; le premier est trempé dans du vin blanc avant d'être mis
en cave.

\medskip

\textit{C}. — Dans la classe des fromages de caillé fermenté à pâte dure se
placent : le gruyère, type des fromages cuits à une température de
{\ppp50\mmm}° à {\ppp60\mmm}, dont l'emmenthal est une qualité supérieure, le
cantal, le hollande, le parmesan, le chester et le stilton.

Tous les fromages de cette classe sont des fromages de conserve.

Un bon gruyère ou un bon emmenthal doit avoir une pâte consistante, jaune
clair, et des yeux grands, réguliers, pas trop nombreux ; le vieil emmenthal
arrosé en cave de vin blanc, pendant des années, est un produit incomparable.
Le fromage du Cantal est un fromage de lait de vache salé. Le hollande
s'obtient en traitant à {\ppp36\mmm}° du lait de vache non écrémé ; sa pâte
doit être jaune rougeâtre, tendre, homogène, onctueuse ; il est excellent
frais, en été. Le parmesan, également fromage de lait de vache, est surtout un
fromage d’assaisonnement. Lorsqu'il est de bonne qualité et à point comme âge,
il est légèrement humide et de couleur jaune d'or. Le chester, à pâte safranée,
dont la préparation ressemble beaucoup à celle du hollande, est très apprécié
en Angleterre ; mais le roi des fromages anglais est le stilton. Lorsqu'il est
vieux et pochardé de madère d'âge vénérable, c'est une pure merveille.

\medskip

\textit{D}. — Le type des fromages persillés est le roquefort préparé avec des
laits de chèvre et de brebis. Le persillage s'obtient en cultivant sur de la
mie de pain une moisissure, le \textit{penicilium glaucum}, qui produit des
marbrures de couleur verte. La maturation du roquefort se fait dans des caves
à la température constante de {\ppp9\mmm}° à {\ppp10\mmm}°.

Parmi les autres fromages persillés, on peut citer le gorgonzola, fromage bleu
safrané, très estimé en Italie.

\sk

Les contempteurs du fromage sont rares ; cependant il en est qui, par une
idiosyncrasie invraisemblable, ne peuvent tolérer la moindre quantité d'aliment
ayant été en contact avec le fromage le plus anodin, et qu'un simple macaroni
au gratin met en fuite.

\medskip

Mais en revanche les amateurs sont légion ; il se trouve même des fanatiques
qui préféreraient du pain sec et du fromage au meilleur dîner dans lequel on
n'en servirait pas, car il faut bien reconnaitre que le fromage clôture
admirablement tout repas : il le complète s'il a été insuffisant, il le
couronne s'il a été bon.

\sk

\index{Comment je conseille aux obèses de manger le fromage}
Les personnes qui craignent d'apporter à leur ration alimentaire un supplément
qui n'est pas négligeable, mangeront le fromage de la façon suivante : elles
prendront seulement gros comme une noisette d’un bon fromage bien à point et
l'écraseront avec la langue contre la voûte palatine jusqu'à absorption
complète ; elles en auront ainsi tout le goût et le parfum, et leur salive
portera discrètement dans leur estomac, sans le charger, ses bienfaisants
effets.
