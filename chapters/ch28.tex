\sk

\bigskip
\bigskip

\begin{center}
\textit{SORBETS}
\end{center}
\index{Sorbets}
\index{Définition des sorbets}
\index{Compositions pour sorbets}

\bigskip

Les sorbets, très en faveur chez les Orientaux, notamment chez les Persans,
sont des boissons présentées sous la forme de glaces légères, mousseuses, peu
congelées. Ils doivent peser {\ppp15\mmm} à {\ppp17\mmm}° au pèse-sirop. On les
sert en France, dans les repas de cérémonie, entre le premier et le second
service, avant le rôti.

À l'origine, les sorbets étaient préparés sans blanc d'œuf, ce qui les
distinguait des punchs. Aujourd'hui, on y fait entrer de la pâte à meringue,
meringue italienne de préférence, ou de la crème Chantilly.

On fait des sorbets aux jus de fruits, rarement aux purées, et aux sirops de
fruits : ananas, abricots, cerises, fraises, framboises, groseilles, pêches.
oranges, citrons, etc. ; aux liqueurs : anisette, chartreuse, curaçao, cognac,
crème de menthe, marasquin, kirsch, rhum, etc. ; aux vins liquoreux : xérès,
madère, malaga, marsala, muscat, porto, samos, zucco, etc. ; aux vins blancs
secs : champagne, chablis, constance, sauternes, vin du Rhin, etc. ; à d'autres
parfums : vanille, ambre, rose, violette, etc.

Les sorbets doivent toujours contenir du jus de citron et souvent du jus d'orange,
en quantité variable suivant le degré d'acidité des fruits ou la nature des vins
employés.

\section*{\centering Sorbet à l’abricot.}
\phantomsection
\addcontentsline{toc}{section}{ Sorbet à l’abricot.}
\index{Sorbet à l’abricot}

Pour douze personnes prenez :

\footnotesize
\begin{longtable}{rrrp{16em}}
    600 & grammes & de & jus d'abricots bien mûrs, pressés et passés au tamis fin,                        \\
     60 & grammes & de & kirsch,                                                                          \\
     50 & grammes & de & curaçao,                                                                         \\
        &         &  2 & citrons,                                                                         \\
        &         &  2 & blancs de meringue italienne,                                                    \\
        &         &    & sirop de sucre à froid, à 22°.                                                   \\
\end{longtable}
\normalsize

Pressez les citrons, recueillez-en le jus, ajoutez-le au jus d'abricots, puis
incorporez au mélange du sirop de sucre en telle quantité que la composition
pèse {\ppp15\mmm}° au saccharimètre. Mettez cette composition dans une
sorbétière tenue dans de la glace pilée additionnée de sel gris et de
salpêtre ; actionnez l'instrument sans travailler autrement. Lorsque la masse
est devenue assez ferme, mélangez-y la meringue italienne, le kirsch et le
curaçao.

Prenez la composition glacée avec une cuiller, dressez-la en pyramide dans des
verres à sorbet ou, à défaut, dans des verres à vins fins.

\section*{\centering Sorbet aux cerises.}
\phantomsection
\addcontentsline{toc}{section}{ Sorbet aux cerises.}
\index{Sorbet aux cerises}

Pour douze personnes prenez :

\footnotesize
\begin{longtable}{rrrp{16em}}
    500 & grammes & de & jus de cerises pressées et passées au tamis,                                     \\
    200 & grammes & de & marasquin de Zara,                                                               \\
     35 & grammes & de & liqueur de noyaux,                                                               \\
        &         &  2 & blancs de meringue italienne,                                                    \\
        &         &  1 & petit citron,                                                                    \\
        &         &    & sirop de sucre à froid, à 22°.                                                   \\
\end{longtable}
\normalsize

Mélangez le jus de cerises, le jus de citron et du sirop de sucre de manière
à avoir une composition pesant {\ppp15\mmm}° au pèse-sirop ; glacez-la à la
sorbétière, puis incorporez‑y la meringue italienne, les deux tiers du
marasquin et la liqueur de noyaux,

Dressez en pyramide la composition glacée dans des verres à sorbet et, avant
de servir, versez dans chaque verre un peu du reste du marasquin.

\sk

Tous les sorbets à base de fruits frais pourront être préparés dans le même esprit.

\medskip

Lorsqu'on n'a pas de fruits frais, on peut faire des sorbets avec des sirops de fruits.

\section*{\centering Sorbet au kirsch.}
\phantomsection
\addcontentsline{toc}{section}{ Sorbet au kirsch.}
\index{Sorbet au kirsch}

Pour douze personnes prenez :

\footnotesize
\begin{longtable}{rrrp{16em}}
    120 & grammes & de & kirsch de la Forêt-Noire.                                                        \\
        & 1 litre & de & sirop de sucre à 19°,                                                            \\
        &         &  2 & citrons,                                                                         \\
        &         &  2 & oranges,                                                                         \\
        &         &  2 & blancs de meringue italienne.                                                    \\
\end{longtable}
\normalsize

Pressez citrons et oranges, ajoutez le jus au sirop de sucre, faites glacer
à la sorbétière. Au dernier moment, incorporez à la composition la meringue
italienne et le kirsch.

Servez dans des verres à sorbet.

\sk

Tous les sorbets aux liqueurs pourront être préparés d'une façon analogue.

\section*{\centering Sorbet au vin de Samos.}
\phantomsection
\addcontentsline{toc}{section}{ Sorbet au vin de Samos.}
\index{Sorbet au vin de Samos}

Pour douze personnes prenez :

\footnotesize
\begin{longtable}{rrrp{16em}}
        & 1 bouteille  & de & samos,                                                                      \\
        &              &  2 & citrons,                                                                    \\
        &              &  2 & oranges,                                                                    \\
        &              &  2 & blancs de meringue italienne.                                               \\
        &              &    & sirop de sucre à froid à 20°.                                               \\
\end{longtable}
\normalsize

Pressez citrons et oranges, mélangez le jus obtenu avec le vin de Samos et du
sirop de sucre de façon à amener la composition à {\ppp15\mmm}° ou
{\ppp16\mmm}° au pèse-sirop. Faites prendre à la sorbétière, puis incorporez
à la masse molle la meringue italienne.

Dressez dans des verres à sorbet et, au moment de servir, arrosez le tour de
chaque pyramide avec un peu de samos.

\sk

Tous les sorbets aux vins liquoreux, aux vins blancs secs où mousseux seront
apprêtés pareillement, en prenant soin toutefois d'augmenter ou de diminuer la
dose du jus de citron en raison de la nature des vins employés.

\bigskip

\begin{center}
\textit{PUNCHS GLACÉS}
\end{center}
\addcontentsline{toc}{section}{ Punchs glacés.}
\index{Punchs glacés}
\index{Définition des punchs glacés}

\bigskip

Les punchs glacés sont des sorbets dans la composition desquels entrent au
moins plusieurs des cinq éléments constitutifs du punch primitif ; le thé, le
rhum, le sucre, le citron et la cannelle. Ce sont des boissons mousseuses très
alcoolisées,

Voici un exemple de punch glacé.

\smallskip

Pour douze personnes prenez :

\footnotesize
\begin{longtable}{rrrp{16em}}
    450 & grammes & de & champagne sec,                                                                   \\
    200 & grammes & de & rhum (tafia pur vieil en fût),                                                   \\
    200 & grammes & de & sucre,                                                                           \\
    100 & grammes & d' & infusion de thé noir de Chine,                                                   \\
        &         &  2 & blancs de meringue italienne,                                                    \\
        &         &  2 & citrons,                                                                         \\
        &         &  1 & orange,                                                                          \\
        &         &    & sirop de sucre à froid, à 22°.                                                   \\
\end{longtable}
\normalsize

Faites cuire le sucre au boulé.

Pressez citrons et orange, versez le jus dans le sirop, ajoutez plus ou moins de
zeste des deux fruits ; laissez en contact pendant deux heures. Retirez les zestes.

Mettez, dans le sirop parfumé, le vin de Champagne, l'infusion de thé, du sirop
de sucre à froid en telle proportion que la composition arrive à peser
{\ppp17\mmm}° à {\ppp18\mmm}° au pèse-sirop. Faites glacer à la sorbétière en
tenant la composition un peu ferme, puis incorporez-y la meringue italienne ;
mélangez bien. Au dernier moment, ajoutez le rhum par très petites quantités en
travaillant bien à la houlette.

Dressez comme les sorbets.

\bigskip

\begin{center}
\textit{GRANITÉS}
\end{center}
\addcontentsline{toc}{section}{ Granités.}
\index{Granités}
\index{Définition des granités}

\bigskip

Les granités sont des sorbets qui ont pour base des sirops au jus de fruits ne
pesant pas plus de {\ppp14\mmm}°. Ils ne contiennent pas de meringue. On ne les
travaille pas en les glaçant et ils forment, une fois pris, une masse plus ou
moins granuleuse.

\bigskip

\begin{center}
\textit{MARQUISES}
\end{center}
\addcontentsline{toc}{section}{ Marquises.}
\index{Marquises}
\index{Définition des marquises}

\bigskip

Les marquises sont des sortes de granités dont la composition est faite surtout
avec des fraises, de l'ananas et du kirsch.

Au moment de servir, on ajoute à ces sorbets de la crème Chantilly très ferme.

\bigskip

\begin{center}
\textit{SPOOMS}
\end{center}
\addcontentsline{toc}{section}{ Spooms.}
\index{Spooms}
\index{Définition des spooms}

\bigskip

Les spooms sont des sorbets aromatisés le plus souvent avec des vins liquoreux
et dans lesquels la proportion de meringue italienne est double de celle
contenue dans les autres sorbets, ce qui les rend plus légers et plus mousseux.

On les dresse dans des verres comme les autres sorbets.

\sk

On fait aussi des spooms au jus de fruits.
