\section*{\centering Poussins grillés, sauce diable.}
\phantomsection
\addcontentsline{toc}{section}{ Poussins grillés, sauce diable.}
\index{Poussins grillés, sauce diable}

Pour quatre personnes prenez :

\footnotesize
\begin{longtable}{rrrrrp{18em}}
  & \hspace{2em}  & 100 & grammes & de & fond de veau,                                                    \\
  & & 100 & grammes & de & vin blanc,                                                                     \\
  & &  50 & grammes & de & beurre,                                                                        \\
  & &  30 & grammes & de & vinaigre,                                                                      \\
  & &  20 & grammes & d' & échalotes hachées,                                                             \\
  & &  15 & grammes & de & glace de viande,                                                               \\
  & &  10 & grammes & de & farine,                                                                        \\
  & &   5 & grammes & de & sel blanc,                                                                     \\
  & &   3 & grammes & de & persil haché,                                                                  \\
  & \multicolumn{3}{r}{2 décigrammes} & de & poivre fraîchement moulu\footnote{Les proportions indiquées
                                             pour le poivre et pour le cayenne correspondent à un
                                             assaisonnement modéré et donnent simplement une idée
                                             de l'ordre de grandeur des quantités.},                      \\
  & \multicolumn{3}{r}{1 décigramme}  & de & cayenne,                                                     \\
  & &     &         &  2 & poussins,                                                                      \\
  & &     &         &    & mie de pain rassis tamisée.                                                    \\
\end{longtable}
\normalsize

Ouvrez les poussins sur le dos, séparez-les en deux, aplatissez chaque moitié
avec le plat d'un couperet ; assaisonnez-les avec la moitié du sel, du poivre
et du cayenne et faites-les cuire sur le gril pendant cinq minutes de chaque
côté.

Roulez ensuite les quatre demi-poussins dans la mie de pain, remettez-les sur
le gril et laissez-les cuire encore pendant cinq minutes de chaque côté,

Pendant toute la durée de la cuisson, arrosez avec {\ppp30\mmm} grammes de beurre fondu.

Servez avec une sauce diable que vous aurez préparée de la façon suivante :
réduisez à glace le vin et le vinaigre avec les échalotes ; faites un roux avec
le reste du beurre et la farine, mouillez avec le fond de veau, dans lequel
vous aurez fait dissoudre la glace de viande, ajoutez la réduction, mettez le
reste du sel, du poivre et du cayenne, goûtez, passez à l'étamine, saupoudrez
de persil haché et servez.

\sk

On peut apprêter de même des poulets de grain, mais la cuisson durera une
demi-heure au lieu de vingt minutes.

Un beau poulet de grain peut suffire pour quatre personnes.

\section*{\centering Friture de poussins.}
\phantomsection
\addcontentsline{toc}{section}{ Friture de poussins.}
\index{Friture de poussins}

Pour quatre personnes prenez :

\medskip

\footnotesize
\begin{longtable}{rrrp{16em}}
    200 & grammes & de & vin blanc,                                                                       \\
     20 & grammes & de & persil,                                                                          \\
     14 & grammes & de & sel blanc,                                                                       \\
     10 & grammes & d' & échalote ciselée,                                                                \\
      5 & grammes & de & poivre fraîchement moulu,                                                        \\
      2 & grammes & de & paprika,                                                                         \\
        &         &  2 & poussins, pesant chacun 300 grammes en moyenne,                                  \\
        &         &  2 & œufs,                                                                            \\
        &         &    & farine,                                                                          \\
        &         &    & chapelure.                                                                       \\
\end{longtable}
\normalsize

Coupez les poussins en quatre, mettez les morceaux dans le vin blanc avec le
poivre, l'échalote, {\ppp10\mmm} grammes de sel et {\ppp10\mmm} grammes de
persil ; laissez-les mariner pendant une heure, puis retirez-les et
égouttez-les.

Cassez les œufs, assaisonnez-les avec le reste du sel et le paprika ; battez
comme pour une omelette.

Passez les quartiers de poussins dans de la farine, puis dans les œufs battus,
enfin dans de la chapelure ; plongez-les ensuite dans de la graisse fondue, sur
le point de fumer, et laissez-les cuire pendant un quart d'heure. Égouttez-les.

Faites frire dans la friture le reste du persil.

Dressez en pyramide les quartiers de poussins, décorez avec le persil frit et
servez.

Envoyez en même temps une saucière de sauce tomate et des citrons.

\sk

\index{Friture de pigeons}
On peut apprêter de même des pigeons.

\section*{\centering Poulet en cocote.}
\phantomsection
\addcontentsline{toc}{section}{ Poulet en cocote.}
\index{Poulet en cocote}

Pour quatre personnes prenez :

\medskip

\footnotesize
\begin{longtable}{rrrrrp{18em}}
  & \hspace{2em}  & 125 & grammes & de & lard de poitrine, peu salé et non fumé,                          \\
  & & 100 & grammes & de & beurre,                                                                        \\
  & &  70 & grammes & d' & oignons,                                                                       \\
  & &  60 & grammes & de & madère,                                                                        \\
  & &  50 & grammes & de & glace de viande,                                                               \\
  & &  30 & grammes & de & vin blanc,                                                                     \\
  & &   5 & grammes & de & sel,                                                                           \\
  & \multicolumn{3}{r}{1 décigramme} & de & poivre,                                                       \\
  & \multicolumn{3}{r}{1 décigramme} & de & quatre épices,                                                \\
  & &     &         &  3 & artichauts,                                                                    \\
  & &     &         &  1 & poulet tendre et gras, pesant 1 kilogramme environ,                            \\
  & &     &         &  1 & petite botte de pointes d'asperges.                                            \\
\end{longtable}
\normalsize

Mettez dans l'intérieur du poulet {\ppp10\mmm} grammes d’oignon, {\ppp10\mmm}
grammes de beurre, le sel, le poivre et les quatre épices : bridez-le, puis
faites-le dorer de lous les côtés dans {\ppp25\mmm} grammes de beurre.

Coupez le lard en petites lames et faites-le revenir dans le beurre qui a servi
à dorer le poulet.

Prenez une cocote en porcelaine à parois et à fond épais, mettez dedans le
reste du beurre, le poulet, le lard et le reste des oignons : couvrez en
interposant un papier beurré entre le bord supérieur de la cocote et le
couvercle, et faites cuire à l'étuvée, au four doux, pendant une demi-heure.

En même temps, faites blanchir les pointes d'asperges et préparez les fonds
d'artichauts.

Enlevez quelques feuilles extérieures, pelez ensuite les artichauts, en
laissant adhérentes aux fonds les parties charnues des autres feuilles, puis
coupez les fonds en quatre et enlevez-en le foin.

Retirez pour un instant le poulet, le lard et les oignons de la cocote,
débridez le poulet, passez le jus.

Remettez dans la cocote le poulet, le lard, le jus, ajoutez les fonds
d'artichauts, les pointes d'asperges, la glace de viande, mouillez avec le
madère et le vin blanc, goûtez pour l'assaisonnement, laissez mijoter pendant
une demi-heure, puis servez.

Ce poulet est parfumé et fondant.

\sk

Comme variantes, on peut ajouter d'autres légumes : pommes de terre, raves,
cerfeuil bulbeux, etc., mais je préfère la formule ci-dessus, à condition
d'avoir des pointes d'asperges et des artichauts frais.

\sk

Enfin, comme variante de luxe, je noterai le poulet en cocote aux morilles et
aux truffes.

\sk

\label{pg0548} \hypertarget{p0548}{}
\index{Chicken pie}
Le « chicken-pie » est une variante anglaise du poulet en cocote.

La cocote est fermée par une abaisse en pâte feuilletée.

Les Anglais font entrer dans la préparation du chicken-pie du poulet, du veau,
du lard fumé (bacon) et des œufs durs coupés en deux, le tout assaisonné avec
sel, poivre, fines herbes passées au beurre, et mouillé avec du consommé de
volaille.

On peut encore y ajouter des champignons, des olives et un peu de madère.

\section*{\centering Poulet sauté aux cèpes frais.}
\phantomsection
\addcontentsline{toc}{section}{ Poulet sauté aux cèpes frais.}
\index{Poulet sauté aux cèpes frais}

Pour trois personnes prenez :

\medskip

\footnotesize
\begin{longtable}{rrrp{16em}}
    100 & grammes & de & beurre,                                                                          \\
        &         &  6 & beaux cépes frais,                                                               \\
        &         &  1 & beau poulet,                                                                     \\
        &         &    & persil,                                                                          \\
        &         &    & jus de citron,                                                                   \\
        &         &    & sel et poivre.                                                                   \\
\end{longtable}
\normalsize

Nettoyez les cèpes, pelez-les, coupez-les morceaux.

Découpez le poulet comme pour une fricassée, faites-le revenir dans {\ppp30\mmm} grammes
de beurre.

Mettez dans une sauteuse {\ppp60\mmm} grammes de beurre, le poulet revenu, les cèpes,
assaisonnez avec sel. poivre et faites cuire, d'abord à feu assez vif, puis
à feu doux.

La cuisson entière dure environ une demi-heure.

Au moment de servir, ajoutez du jus de citron et du persil haché manié avec le
reste du beurre.

Ce plat est absolument remarquable avec des cèpes fraîchement cueillis.

\sk

On peut préparer d'une façon analogue un poulet sauté aux morilles ou aux
champignons de couche.

\section*{\centering Poulet sauté aux cèpes de conserve.}
\phantomsection
\addcontentsline{toc}{section}{ Poulet sauté aux cèpes de conserve.}
\index{Poulet sauté aux cèpes de conserve}

Pour quatre personnes prenez :

\medskip

\footnotesize
\begin{longtable}{rrrp{16em}}
    100 & grammes & de & vin blanc,                                                                       \\
     90 & grammes & de & beurre,                                                                          \\
     60 & grammes & d' & huile d'olive,                                                                   \\
     20 & grammes & d' & échalotes,                                                                       \\
     10 & grammes & de & farine,                                                                          \\
      5 & grammes & d' & ail,                                                                             \\
        &         &  1 & beau poulet entier,                                                              \\
        &         &  1 & boîte de cèpes,                                                                  \\
        &         &    & légumes de pot-au-feu,                                                           \\
        &         &    & persil,                                                                          \\
        &         &    & jus de citron,                                                                   \\
        &         &    & sel et poivre.                                                                   \\
\end{longtable}
\normalsize

Découpez le poulet ; réservez l'abatis.

Préparez avec l'abatis, des légumes, de l'eau, du sel et du poivre un bouillon
à bouilli perdu ; concentrez-le de manière à obtenir quelques cuillerées de
jus.

Faites un roux avec {\ppp30\mmm} grammes de beurre et la farine ; mouillez avec le vin et
le jus de poulet, ajoutez la moitié des échalotes et de l'ail hachés ;
chauffez ; goûtez et complétez avec sel et poivre, s'il y a lieu,
l'assaisonnement de la sauce qui doit être courte.

Mettez dans une sauteuse {\ppp30\mmm} grammes d'huile ; lorsqu'elle sera très chaude,
ajoutez le poulet et faites-le bien rissoler, en le sautant, de façon qu'il
soit cuit aux trois quarts. Retirez les morceaux de poulet, égouttez-les et
achevez leur cuisson lentement, dans la sauce précédemment préparée, sans
laisser bouillir.

En même temps, apprêtez les cèpes : sortez-les de la boîte, passez-les à l'eau
tiède afin d'enlever la couche mucilagineuse qui les enveloppe, puis séchez-les
dans un linge.

Coupez-les en morceaux et faites-les sauter dans le reste de l'huile jusqu'à ce
qu'ils soient raffermis et bien dorés. Enlevez-les, égouttez-les ; remplacez
l'huile par {\ppp30\mmm} grammes de beurre, remettez les cèpes, ajoutez le reste des
échalotes et de l'ail hachés ; faites sauter le tout ensemble pendant quelques
instants.

Dressez les morceaux de poulet sur un plat ; montez la sauce au fouet avec 15
grammes de beurre et masquez-en le poulet.

Mettez les cèpes dans un légumier, arrosez-les avec un peu de jus de citron et
le reste du beurre, manié avec un peu de persil haché, que vous aurez fait
fondre.

Servez en même temps le poulet et les cèpes.

L'huile, qui supporte une température plus élevée que le beurre sans se
décomposer, permet de faire rissoler très bien les éléments de la préparation
el de raffermir les cèpes qui, à l'état de conserve, sont plutôt mous ; le
beurre, dans lequel la cuisson s'achève, enlève tout goût d'huile, et
l'addition finale de beurre frais, tant dans la sauce que dans les cèpes, donne
de la finesse au mets.

\section*{\centering Poulet sauté au vin blanc.}
\phantomsection
\addcontentsline{toc}{section}{ Poulet sauté au vin blanc.}
\index{Poulet sauté au vin blanc}

Pour trois personnes prenez :

\footnotesize
% \begin{longtable}{| r | r | r | r | r | p{18em}}
\begin{longtable}{rrrrrp{18em}}
  & \hspace{2em}    & 200 & grammes & de & vin blanc sec,                                                 \\
  & &  125 & grammes & de & champignons de couche,                                                        \\
  & &  125 & grammes & de & tomates à grosses côtes, ce qui correspond à une grosse tomate,               \\
  & &   60 & grammes & de & beurre,                                                                       \\
  & &   30 & grammes & de & glace de viande dissoute dans du bouillon, ou 60 grammes de jus de viande,    \\
  & &   20 & grammes & de & fine champagne,                                                               \\
  & &   10 & grammes & de & sel,                                                                          \\
  & &    2 & grammes & de & persil,                                                                       \\
  & \multicolumn{3}{r}{1 gramme 1/2}  & d' & ail,                                                         \\
  & \multicolumn{3}{r}{2 décigrammes} & de & poivre fraîchement moulu,                                    \\
  & \multicolumn{3}{r}{2 décigrammes} & de & poivre de Cayenne,                                           \\
  & &     &         &  1 & jeune poulette tendre pesant 800 grammes environ sans l'abatis,                \\
  & &     &         &    & jus de citron.                                                                 \\
\end{longtable}
\normalsize

Découpez la poulette comme pour une fricassée.

Mettez le beurre dans une casserole, amenez-le à couleur noisette, puis faites
dorer dedans à feu vif les morceaux de poulet, ajoutez ensuite les champignons
épluchés et passés au jus de citron, la tomate épluchée, débarrassée de ses
pépins et coupée en petits dés ; sautez le tout ensemble pendant cinq minutes,
mouillez avec le vin, la fine champagne et le jus de viande, assaisonnez avec
sel, poivre, cayenne et continuez la cuisson pendant un quart d'heure.

Dressez les morceaux de poulet sur un plat tenu au chaud, dégraissez la sauce,
incorporez-y le persil et l'ail hachés, réduisez-la à consistance suffisamment
épaisse et masquez-en les morceaux de poulet.

Préparé ainsi, le poulet sauté est très eupeptique.

On le sert sans accompagnement de légumes.

\section*{\centering Poulet sauté à la crème et à l’armagnac.}
\phantomsection
\addcontentsline{toc}{section}{ Poulet sauté à la crème et à l’armagnac.}
\index{Poulet sauté à la crème et à l’armagnac}

Pour quatre personnes prenez :

\medskip

\footnotesize
\begin{longtable}{rrrp{16em}}
    300 & grammes & de & crème,                                                                           \\
     40 & grammes & de & beurre,                                                                          \\
     40 & grammes & d' & armagnac,                                                                        \\
     25 & grammes & d' & échalotes,                                                                       \\
     25 & grammes & d' & oignon,                                                                          \\
     20 & grammes & d' & huile d'olive,                                                                   \\
     10 & grammes & de & farine,                                                                          \\
        &         &  1 & beau poulet avec son abatis,                                                     \\
        &         &    & légumes de pot-au-feu,                                                           \\
        &         &    & jus de citron,                                                                   \\
        &         &    & sel et poivre.                                                                   \\
\end{longtable}
\normalsize

Nettoyez, videz, flambez et découpez le poulet.

Préparez un bouillon à bouilli perdu avec l'abatis, des légumes, de l'eau, du sel
et du poivre ; concentrez-le ; passez-le.

Chauffez dans une sauteuse le beurre et l'huile ; lorsque le mélange sera bien
chaud, mettez dedans les morceaux de poulet, passés au préalable dans la
farine, les échalotes, l'oignon ciselé et faites revenir le tout à feu vif
pendant une dizaine de minutes ; diminuez le feu et faites sauter ensuite
pendant une vingtaine de minutes. Retirez les morceaux de poulet ; tenez-les au
chaud.

Déglacez la sauteuse avec le bouillon de poulet concentré, la crème et
l'armagnac ; amenez la sauce à bonne consistance ; goûtez, ajoutez du jus de
citron et, s'il y a lieu, du sel et du poivre. Passez la sauce ; remettez
dedans les morceaux de poulet et achevez la cuisson de l'ensemble sur le coin
du fourneau sans laisser bouillir.

Servez en envoyant en même temps, par exemple, du riz aux cèpes, une purée
de champignons ou une purée de truffes.

\sk

Comme variantes, on peut remplacer l'armagnac par de la fine champagne et
ajouter dans la sauce des champignons passés au jus de citron et cuits dans du
beurre, ou des truffes cuites dans du madère. Dans ce dernier cas, on ajoutera à
la sauce un peu de la cuisson des truffes, en évitant que le madère domine.

Le poulet sauté à la crème et à l'armagnac ou à la fine champagne est digne
des meilleures tables,

\section*{\centering Poulet sauté à l'estragon.}
\phantomsection
\addcontentsline{toc}{section}{ Poulet sauté à l'estragon.}
\index{Poulet sauté à l'estragon}

Découpez un jeune poulet, dorez-le dans {\ppp60\mmm} grammes de beurre ; salez,
poivrez ; ajoutez ensuite un bouquet de {\ppp30\mmm} grammes d' estragon et
quelques cuillerées de bon fond de veau et volaille ; couvrez. Laissez cuire,
en casserole couverte, pendant une vingtaine de minutes, en faisant sauter.

Dressez le poulet sur un plat, déglacez la cuisson avec du velouté de veau et
volaille, passez, ajoutez un peu d'estragon haché, versez la sauce sur le
poulet et servez.

\section*{\centering Poulet sauté à l’indienne.}
\phantomsection
\addcontentsline{toc}{section}{ Poulet sauté à l’indienne.}
\index{Poulet sauté à l’indienne}

Pour trois personnes prenez :

\medskip

\footnotesize
\begin{longtable}{rrrrrp{18em}}
  & \hspace{2em} &  250 & grammes & de & tomates,                                                         \\
  &              &  100 & grammes & de & cèpes secs,                                                      \\
  &              &  100 & grammes & de & beurre,                                                          \\
  &              &   50 & grammes & de & glace de viande,                                                 \\
  &              &   35 & grammes & d' & oignons,                                                         \\
  &              &   15 & grammes & d' & huile d'olive,                                                   \\
  &              &    5 & grammes & de & sel,                                                             \\
  &              &    4 & grammes & d' & échalote,                                                        \\
  &              &    2 & grammes & d' & ail,                                                             \\
  &              &    2 & grammes & de & poudre de curry,                                                 \\
  &              &    2 & grammes & de & poudre de safran,                                                \\
  &              &    1 & gramme  & de & poudre de gingembre,                                             \\
  & \multicolumn{3}{r}{2 décigrammes}  & de & poivre fraîchement moulu,                                   \\
  & \multicolumn{3}{r}{1  décigramme}  & de & poivre de Cayenne,                                          \\
  &              &      &         &  1 & jeune poulette tendre, pesant environ 800 grammes sans l'abatis, \\
  &              &      &         &    & bouillon.                                                        \\
\end{longtable}
\normalsize

Faites tremper les cèpes dans de l'eau pendant plusieurs heures.

Découpez la poulette.

Chauffez l'huile avec le beurre dans une casserole, amenez-les à la teinte
noisette, mettez dedans les morceaux de poulet, laissez cuire à petit feu
jusqu'à couleur d'or, puis ajoutez les oignons, l'échalote et l'ail hachés, les
tomates épluchées, débarrassées de leurs pépins et coupées en petits dés.
Faites sauter pendant cinq minutes, mouillez avec du bouillon dans lequel vous
aurez fait dissoudre la glace de viande et assaisonnez avec sel, poivre,
cayenne, gingembre, safran et curry.

Lavez les cèpes à plusieurs eaux, essuyez-les, coupez-les en morceaux et
ajoutez-les au ragoût. Achevez la cuisson à grand feu, en casserole couverte.

Préparez du riz sec, \hyperlink{p0707}{p. \pageref{pg0707}}, mettez-le sur un plat,
creusez-le un peu au milieu et disposez dans ce creux les morceaux de poulet.

Servez, en envoyant la sauce à part, dans une saucière.

Le poulet sauté à l'indienne est certainement un plat plutôt relevé ;
cependant, grâce à l'association de tous les condiments en proportions
convenables, il n'y à pas de note criarde dans l’ensemble, qui satisfait
généralement les estomacs même les plus délicats.

\section*{\centering Poulet sauté des Rajahs.}
\phantomsection
\addcontentsline{toc}{section}{ Poulet sauté des Rajahs.}
\index{Poulet sauté des Rajahs}

Pour quatre personnes prenez :

\footnotesize
\begin{longtable}{rrrrrp{18em}}
  & \hspace{2em} & 400 & grammes & de & riz,                                                              \\
  & \hspace{2em} & 400 & grammes & de & bouillon,                                                         \\
  & \hspace{2em} & 250 & grammes & de & crevettes crises,                                                 \\
  & \hspace{2em} & 125 & grammes & de & beurre,                                                           \\
  & \hspace{2em} &  10 & grammes & de & curry,                                                            \\
  & \hspace{2em} &   5 & grammes & de & farine,                                                           \\
  & \hspace{2em} &   5 & grammes & de & sel\footnote{Les proportions de sel, poivre et curry
                                            dépendent de l'assaisonnement du bouillon et de celui
                                            des crevettes ; elles dépendent aussi de la noix de
                                            coco qui peut être plus ou moins grosse, plus ou moins
                                            sucrée. Celles qui sont indiquées sont des proportions
                                            moyennes ; il sera prudent de goûter plusieurs fois
                                            pendant la préparation et de ne mettre l'assaisonnement
                                            que par petites quantités.}                                   \\
  & \multicolumn{3}{r}{1/2 décigramme} & de & poivre,                                                     \\
  &              &     &         &  1 & jeune poulette tendre, pesant 800 grammes sans l'abatis,          \\
  &              &     &         &  1 & noix de coco mûre, mais fraîche, c'est-à-dire ayant encore
                                        son eau et pouvant fournir environ 250 grammes de pulpe,          \\
  &              &     &         &    & achards, chou palmiste confit, petits poissons au piment,
                                        confiture de goyaves\footnote{Tous éléments qu'on trouve
                                        chez les marchands de comestibles exotiques.}.                    \\
\end{longtable}
\normalsize

Ouvrez la noix, extrayez-en la pulpe, râpez-la, faites-la bouillir pendant un quart
d'heure dans le bouillon, puis passez le tout au presse-purée ; réservez le jus.

Faites cuire les crevettes comme il est dit
\hyperlink{p0287}{p. \pageref{pg0287}}, décortiquez les queues, réservez-les ;
préparez un beurre de crevettes avec {\ppp65\mmm} grammes de beurre et les
parures,

Apprêtez le riz sec, comme il est dit \hyperlink{p0707}{p. \pageref{pg0707}} ;
tenez-le au chaud,

Découpez la poulette et faites sauter les morceaux dans le reste du beurre.

En même temps, préparez la sauce.

Mettez dans une casserole le beurre de crevettes et la farine, chauffez sans
laisser roussir, mouillez avec le jus réservé, assaisonnez avec sel, poivre et
curry, en quantité suffisante pour donner de la chaleur à la sauce, tout en lui
conservant la saveur de la noix de coco qui doit être sa caractéristique, et
continuez la cuisson, à petit feu, de manière à amener la sauce à une
consistance convenable ; cette opération demande une demi-heure environ. Un
quart d'heure avant la fin, ajoutez les queues de crevettes.

Servez, en envoyant en même temps, mais à part, le poulet, le riz et la sauce,
ainsi que des raviers d'achards, de chou palmiste confit, de petits poissons au
piment, de confiture de goyaves, etc., il y en aura pour tous les goûts.

\section*{\centering Poulet sauté au citron.}
\phantomsection
\addcontentsline{toc}{section}{ Poulet sauté au citron.}
\index{Poulet sauté au citron}

Pour trois personnes prenez :

\medskip

\footnotesize
\begin{longtable}{rrrp{16em}}
    200 & grammes & de & bouillon,                                                                        \\
     60 & grammes & de & beurre,                                                                          \\
     50 & grammes & de & glace de viande,                                                                 \\
     10 & grammes & de & farine,                                                                          \\
        &         &  2 & citrons,                                                                         \\
        &         &  1 & jeune poulet pesant 800 grammes environ sans l'abatis,                           \\
        &         &  1 & bouquet garni, comprenant 5 grammes de persil, une
                         brindille de thym et un morceau de feuille de laurier,                           \\
        &         &    & caramel,                                                                         \\
        &         &    & sel et poivre.                                                                   \\
\end{longtable}
\normalsize

Découpez le poulet, faites-le revenir à la casserole dans le beurre ; quand il
aura pris couleur, retirez-le ; mettez la farine, faites un roux, laissez-le
déposer, égouttez le beurre, mouillez avec le bouillon dans lequel vous aurez
fait dissoudre la glace de viande, mélangez bien, puis remettez le poulet,
ajoutez le bouquet, trois tranches de citron épépinées et coupées chacune en
cinq segments, du sel et du poivre au goût.

Laissez cuire pendant quelques instants en casserole couverte, donnez un peu
de couleur avec du caramel et achevez la cuisson. Une demi-heure de cuisson en
tout suffit largement.

Dressez les morceaux de poulet sur un plat, enlevez le bouquet, goûtez la
sauce, complétez son assaisonnement, soit avec un peu de jus de citron, soit
avec un peu de caramel, au goût, de façon qu'elle soit légèrement aigrelette,
puis masquez-en le poulet, en laissant dans la sauce les morceaux de citron.

Décorez le plat avec le deuxième citron coupé en tranches et servez.

\section*{\centering Poulet sauté au parmesan.}
\phantomsection
\addcontentsline{toc}{section}{ Poulet sauté au parmesan.}
\index{Poulet sauté au parmesan}

Pour trois personnes prenez :

\medskip

\footnotesize
\begin{longtable}{rrrp{16em}}
    150 & grammes & de & crème,                                                                           \\
    120 & grammes & de & beurre,                                                                          \\
     60 & grammes & de & parmesan râpé,                                                                   \\
     15 & grammes & de & farine,                                                                          \\
      7 & grammes & de & sel,                                                                             \\
        &         &  3 & jaunes d'œufs frais,                                                             \\
        &         &  1 & jeune poulette tendre,                                                           \\
        &         &    & chapelure.                                                                       \\
\end{longtable}
\normalsize

Découpez la poulette, faites sauter les morceaux pendant vingt minutes dans 20
grammes de beurre ; saupoudrez de {\ppp2\mmm} grammes de sel.

Faites cuire la farine dans le reste du beurre sans la laisser roussir, ajoutez
la crème, {\ppp5\mmm} grammes de parmesan, le reste du sel et travaillez bien le tout sur
un feu doux ; achevez ensuite la liaison avec les jaunes d'œufs.

Prenez un plat allant au feu, garnissez-en le fond avec {\ppp25\mmm} grammes de parmesan.
disposez dessus les morceaux de poulet sauté, masquez-les avec la sauce, mettez
au four pendant cinq minutes, couvrez avec le reste du parmesan et de la
chapelure, remettez un instant au four pour dorer et servez.

\section*{\centering Poulet sauté à l’espagnole.}
\phantomsection
\addcontentsline{toc}{section}{ Poulet sauté à l’espagnole.}
\index{Poulet sauté à l’espagnole}

Ce plat est l'un des meilleurs de la cuisine espagnole. J'en donne ici une
formule qui diffère de la recette originelle, notamment par le mouillement, qui
est fait en Espagne avec de l'eau, ce qui entraîne l'emploi d'une proportion
plus grande de graisse et d'huile, et rend le mets beaucoup moins fin.

\medskip

Pour six personnes prenez :

\footnotesize
\begin{longtable}{rrrrrp{18em}}
    & \hspace{2em}  & 1 000 & grammes &  de & petits pois en cosses,                                      \\
    & \hspace{2em}  &   250 & grammes &  de & jambon fumé, maigre,                                        \\
    & \hspace{2em}  &   250 & grammes &  de & riz,                                                        \\
    & \hspace{2em}  &   250 & grammes &  de & tomates,                                                    \\
    & \hspace{2em}  &    50 & grammes &  de & saindoux,                                                   \\
    & \hspace{2em}  &    15 & grammes &  d' & huile d'olive,                                              \\
    & \hspace{2em}  &     5 & grammes &  de & sel,                                                        \\
 & \multicolumn{3}{r}{2 centigrammes} &  de & cayenne\footnote{Le poulet sauté à l'espagnole doit être
                                              un peu relevé. Les proportions de sel et de cayenne sont
                                              seulement données à titre de renseignement.},               \\
    &               &       & 1 litre &  de & fond de veau,                                               \\
    &               &       &         &   2 & piments verts d'Espagne,                                    \\
    &               &       &         &   1 & poulet,                                                     \\
    &               &       &         &   1 & bel oignon d'Espagne,                                       \\
    &               &       &         &   1 & gros fond d'artichaut,                                      \\
    &               &       &         & 1/2 & gousse d'ail.                                               \\
\end{longtable}
\normalsize

Mettez dans une poêle le saindoux, l'huile, le poulet découpé, le jambon coupé
en petits morceaux, assaisonnez légèrement et faites sauter pendant une
demi-heure ; enlevez poulet et jambon, réservez-les.

En même temps, faites cuire, aux trois quarts, les petits pois écossés dans le
fond de veau.

Dans la graisse qui a servi pour le poulet faites revenir l'oignon, l'ail et
les piments hachés, ajoutez le riz blanchi dans de l'eau bouillante et
rafraîchi, mouillez avec la cuisson des petits pois, mettez les tomates pelées,
débarrassées de leurs pépins et hachées, le fond d'artichaut coupé en morceaux,
les petits pois, le poulet et le jambon réservés, assaisonnez et laissez cuire
le tout ensemble pendant une demi-heure. Dégraissez, goûtez pour
l’assaisonnement, dressez sur un plat et servez.

\sk

Dans les pays de langue espagnole de l'Amérique, on a plus on moins modifié le
plat primitif par des additions successives, et l'on est arrivé ainsi à un
mélange méritant une désignation spéciale, auquel on pourrait donner le nom de
poulet sauté hispano-américain, bien que le rôle du poulet y soit un peu
effacé.

Pour fixer les idées, je donnerai les proportions des éléments de ce plat.

\medskip

Pour dix personnes prenez :

\footnotesize
\begin{longtable}{rrrp{16em}}
        &         &    & les matières premières figurant dans la formule du poulet sauté
                         à l'espagnole, plus :                                                            \\
    250 & grammes & de & porc frais,                                                                      \\
    250 & grammes & de & filets de poissons, tels que sole, merlan, colin, etc.                           \\
    125 & grammes & de & crevettes,                                                                       \\
        &         & 40 & moules,                                                                          \\
        &         & 10 & petites saucisses,                                                               \\
        &         &  2 & pommes de terre,                                                                 \\
        &         &  2 & œufs durs,                                                                       \\
        &         &    & safran.                                                                          \\
\end{longtable}
\normalsize

Coupez le porc en morceaux, faites-le sauter ainsi que les saucisses avec le
poulet et le jambon, mais retirez les saucisses au bout de quatre à cinq
minutes et tenez-les au chaud.

Faites cuire les moules au naturel, les crevettes suivant le rite, les pommes
de terre à la vapeur et faites frire les filets de poissons. Tenez au chaud ces
différents éléments.

Décortiquez les crevettes, réservez les queues.

Toute la préparation se fera comme précédemment : on mettra du safran, au goût,
en même temps que le riz.

Dix minutes avant la fin de la cuisson, ajoutez les pommes de terre coupées en
morceaux, les filets de poissons frits, les trois quarts des moules et des
queues de crevettes, mélangez bien.

Au moment de servir, décorez le plat avec les œufs durs coupés en tranches,
les saucisses, le reste des moules et des queues de crevettes.

Il est d'usage de prendre de l'anisette, comme digestif, après ce plat.

\section*{\centering Poulet à la juive.}
\phantomsection
\addcontentsline{toc}{section}{ Poulet à la juive.}
\index{Poulet à la juive}

\centering\textit{(Mode alsacienne.)}

\bigskip

\justifying
Pour quatre personnes prenez :

\medskip

\footnotesize
\begin{longtable}{rrrp{16em}}
     50 & grammes & de & graisse de poulet, crue, ou de graisse de rôti de poulet,
                         ou, à défaut, de graisse de rognon de veau,                                      \\
     40 & grammes & de & beurre,                                                                          \\
     30 & grammes & d’ & échalotes hachées,                                                               \\
     15 & grammes & de & farine,                                                                          \\
      5 & grammes & d' & ail,                                                                             \\
      5 & grammes & de & persil hache,                                                                    \\
        &         &  1 & poulet,                                                                          \\
        &         &    & légumes de pot-au-feu,                                                           \\
        &         &    & sel et poivre.                                                                   \\
\end{longtable}
\normalsize

Découpez le poulet comme pour une fricassée ; réservez l'abatis.

Préparez {\ppp400\mmm} grammes de jus en faisant cuire, à bouilli perdu,
l'abatis et des légumes dans de l'eau salée.

Faites fondre ou faites chauffer la graisse dans une casserole, mettez dedans
les morceaux de poulet, laissez-les blondir légèrement, ajoutez d'abord les
échalotes et l'ail, tournez pendant quelques minutes, puis la farine, tournez
encore sans laisser roussir ; mouillez avec le jus de poulet, salez, poivrez et
achevez la cuisson à feu moyen, en casserole fermée, pendant une heure à une
heure et demie, suivant la grosseur du poulet.

Dégraissez la sauce, montez-la avec le beurre ; goûtez pour l'assaisonnement,
qui doit être un peu relevé en poivre, saupoudrez avec le persil et servez.

\section*{\centering Poulet à l'étoile.}
\phantomsection
\addcontentsline{toc}{section}{ Poulet à l'étoile.}
\index{Poulet à l'étoile}

Faites braiser un poulet de choix dans une cuisson composée de bouillon blanc
et de vin de Porto blanc un peu liquoreux. Assaisonnez avec du poivre de
Cayenne et du poivre indien\footnote{Le poivre indien est le fruit pulvérisé du
\textit{Xylopia œthiopica}, de la famille des Anonacées ; son odeur tient à la
fois de celles du gingembre et du curcuma ; sa saveur est piquante et
légèrement musquée.}, de façon à donner de la chaleur à la sauce, sans lui ôter
son velouté et sa douceur apparente.

Incorporez de la crème à la cuisson réduite.

Ce poulet peut être servi avec des pointes d’asperges cuites à l'eau,
à l'anglaise, et présentées à part sur une serviette.

On peut également panacher les asperges de truffes, mais c'est là une
complication inutile : l'asperge, dans sa simplicité, donne au plat un
caractère plus hautement aristocratique ; la truffe sent la finance.

Cette recette. qui rappelle celle du poulet Archiduc, est due au génie de Dot,
ancien cuisinier en chef de l'escadre de la Méditerranée, le seul rival
possible du grand Frédéric.

Elle est d'autant plus remarquable que les restes du poulet, découpés et servis
froids. après avoir été masqués avec la sauce liée, constituent un chaud-froid
de premier ordre. D'ailleurs, le plat froid diffère beaucoup comme goût du plat
chaud, à tel point que des amateurs se sont trouvés surpris d'apprendre que
c'était le même poulet qu'ils avaient dégusté sous les deux formes.

\section*{\centering Poulet gascon.}
\phantomsection
\addcontentsline{toc}{section}{ Poulet gascon.}
\index{Poulet gascon}

Pour quatre personnes prenez :

\medskip

\footnotesize
\begin{longtable}{rrrp{16em}}
    200 & grammes & de & bouillon,                                                                        \\
    125 & grammes & de & lard maigre,                                                                     \\
     50 & grammes & de & glace de viande,                                                                 \\
     30 & grammes & de & beurre ou de bonne graisse,                                                      \\
     15 & grammes & de & farine,                                                                          \\
        &         &  4 & gousses d'ail,                                                                   \\
        &         &  2 & jaunes d'œufs frais,                                                             \\
        &         &  1 & poulet,                                                                          \\
        &         &  1 & barde de lard ou du lard à piquer,                                               \\
        &         &    & jus de citron ou verjus,                                                         \\
        &         &    & sel et poivre.                                                                   \\
\end{longtable}
\normalsize

Piquez de lard ou bardez le poulet, à volonté, et frottez-le avec du jus de citron.

Foncez une casserole avec le lard maigre, mettez dessus le poulet, couvrez avec
une feuille de papier huilé et faites cuire au four, à petit feu. Pendant la
cuisson, mouillez avec le bouillon dans lequel vous aurez fait dissoudre la
glace de viande et arrosez souvent le poulet.

Dix minutes avant de servir, faites blondir la farine dans le beurre ou dans la
graisse, ajoutez l'ail écrasé et passé au tamis, délayez avec le jus de cuisson
du poulet, complétez l’assaisonnement avec sel et poivre, s'il y a lieu, donnez
quelques bouillons, liez avec les jaunes d'œufs et ajoutez un peu de citron ou
de verjus.

Dressez le poulet sur un plat, versez dessus la sauce et servez.

\section*{\centering Poulet au blanc.}
\phantomsection
\addcontentsline{toc}{section}{ Poulet au blanc.}
\index{Poulet au blanc}

Pour six personnes prenez :

\medskip

\footnotesize
\begin{longtable}{rrrp{16em}}
    300 & grammes & de & carottes,                                                                        \\
    250 & grammes & de & champignons de couche,                                                           \\
    150 & grammes & de & beurre,                                                                          \\
    100 & grammes & de & crème épaisse,                                                                   \\
     50 & grammes & d' & oignons.                                                                         \\
     30 & grammes & de & navet,                                                                           \\
     25 & grammes & de & farine,                                                                          \\
     20 & grammes & de & sel,                                                                             \\
      2 & grammes & de & poivre,                                                                          \\
        &         &  3 & jaunes d'œufs frais,                                                             \\
        &         &  2 & clous de girofle,                                                                \\
        &         &  1 & beau poulet tendre avec son abatis.                                              \\
        &         &  1 & bouquet garni,                                                                   \\
        &         &    & jus de citron,                                                                   \\
        &         &    & muscade.                                                                         \\
\end{longtable}
\normalsize

Enlevez l'abatis du poulet, nettoyez-le, mettez-le dans une casserole,
contenant la quantité d'eau suffisante pour baigner le poulet à moitié, avec
carottes, oignons, navet coupés en morceaux, bouquet garni, sel, poivre, clous
de girofle, un peu de muscade ; laissez cuire pendant une heure, puis ajoutez
le poulet paré, que vous couvrirez d'un papier beurré pour éviter qu'il
jaunisse. Continuez la cuisson doucement en casserole incomplètement fermée
pendant une heure. Passez le jus.

Faites cuire les champignons avec {\ppp50\mmm} grammes de beurre, le jus de la moitié
d'un citron et un peu de sel.

Prenez une casserole de trois quarts de litre de contenance environ, dans
laquelle vous mettrez le reste du beurre et la farine. Faites cuire sans
laisser prendre couleur, mouillez avec le jus du poulet, ajoutez les
champignons, la crème, les jaunes d'œufs, chauffez en tournant sans laisser
bouillir, goûtez et relevez, s'il y a lieu, avec poivre ou muscade, ou bien les
deux.

Découpez le poulet, dressez les morceaux sur un plat, masquez-les avec la
sauce et servez.

\section*{\centering Poulet gratiné.}
\phantomsection
\addcontentsline{toc}{section}{ Poulet gratiné.}
\index{Poulet gratiné}

Découpez un petit poulet, faites-le revenir légèrement, puis laissez-le cuire
doucement.

Délayez un peu de farine avec de la crème, salez, laissez cuire sans faire
bouillir, puis ajoutez, hors du feu, deux œufs entiers et deux jaunes d'œufs,
mélangez.

Disposez les morceaux de poulet dans un plat allant au feu, masquez-les avec la
sauce, saupoudrez de parmesan râpé et de mie de pain rassis tamisée ; passez au
four pour gratiner.

\section*{\centering Ragoût de poulet truffé au champagne.}
\phantomsection
\addcontentsline{toc}{section}{ Ragoût de poulet truffé au champagne.}
\index{Ragoût de poulet truffé au champagne}

Pour quatre personnes prenez :

\medskip

\footnotesize
\begin{longtable}{rrrp{16em}}
    200 & grammes & de & champagne très sec,                                                              \\
    200 & grammes & de & truffes noires du Périgord,                                                      \\
    250 & grammes & de & carottes\footnote{Cette quantité de carottes convient lorsqu'on emploie
                                           du vin très sec ; elle devra être diminuée d'autant
                                           plus que le vin employé sera plus doux.},                      \\
    150 & grammes & de & lard de poitrine,                                                                \\
    100 & grammes & de & beurre,                                                                          \\
     80 & grammes & de & fine champagne,                                                                  \\
     60 & grammes & d' & oignons,                                                                         \\
     50 & grammes & de & fond de veau,                                                                    \\
        &         &  1 & jeune poulet,                                                                    \\
        &         &  1 & bouquet garni,                                                                   \\
        &         &    & farine,                                                                          \\
        &         &    & graisse,                                                                         \\
        &         &    & sel et poivre.                                                                   \\
\end{longtable}
\normalsize

Nettoyez les truffes, brossez-les, lavez-les, essuyez-les.

Découpez le poulet comme pour une fricassée ; mettez de côté l'abatis et la
partie de la carcasse située sous les ailes.

Passez les morceaux de poulet dans de la farine.

Faites revenir, à feu vif, dans {\ppp50\mmm} grammes de beurre, les carottes, les
oignons, l'abatis et la carcasse coupée en morceaux, pendant une dizaine de
minutes. Réservez le beurre de cuisson.

Faites dorer à part, dans un peu de graisse, le lard coupé en petits cubes ;
puis mettez le mélange précédemment revenu et le bouquet ; mouillez avec le
champagne et le fond de veau, couvrez, laissez cuire pendant une demi-heure.
Retirez le lard, passez à la presse les déchets de poulet, ajoutez le jus
obtenu à la sauce ; passez-la, dégraissez-la, concentrez-la.

Faites revenir vivement les morceaux de poulet dans le beurre de cuisson
réservé, flambez-le avec la fine champagne, salez légèrement, poivrez au goût,
mouillez avec la sauce concentrée, ajoutez le lard, les truffes et achevez la
cuisson à feu vif et en casserole couverte pendant un quart d'heure environ.

Dégraissez la sauce, puis montez-la avec le reste du beurre frais coupé en
petits morceaux ; goûtez et complétez l'assaisonnement, s'il y a lieu.

Servez le ragoût tel que ou accompagné d'un légumier de riz pilaf.

\sk

En remplaçant les truffes par des champignons de saison et le champagne par un
vin blanc sec quelconque, on aura des ragoûts différents, tel le ragoût de poulet
aux morilles et au chablis qui ne le cède guère au ragoût ci-dessus, dont le luxe
est presque insolent.

\section*{\centering Soufflé de poulet au riz.}
\phantomsection
\addcontentsline{toc}{section}{ Soufflé de poulet au riz.}
\index{Soufflé de poulet au riz}

Pour six personnes prenez :

\medskip

\footnotesize
\begin{longtable}{rrrp{16em}}
    250 & grammes & de & riz,                                                                             \\
    150 & grammes & de & beurre,                                                                          \\
    125 & grammes & de & jambon salé, non fumé, ou de lard de poitrine,                                   \\
     75 & grammes & de & crème épaisse,                                                                   \\
     50 & grammes & de & gruyère râpé,                                                                    \\
        &         &  2 & œufs frais,                                                                      \\
        &         &  1 & poulet tendre,                                                                   \\
        &         &  1 & grosse échalote hachée,                                                          \\
        &         &  1 & oignon moyen haché,                                                              \\
        &         &  1 & bouquet garni,                                                                   \\
        &         &    & légumes de pot-au-feu,                                                           \\
        &         &    & truffes cuites dans du madère, à volonté,                                        \\
        &         &    & sel, poivre, épices.                                                             \\
\end{longtable}
\normalsize

Cassez les œufs, séparez les blancs des jaunes ; battez les blancs en neige.

Enlevez {\ppp650\mmm} grammes de chair blanche du poulet sans peau ni nerfs.

Préparez un bouillon à bouilli perdu avec l'abatis, les restes du poulet, des
légumes de pot-au-feu, de l'eau, du sel, du poivre et un peu d'épices ;
dégraissez:le, passez-le, concentrez-le de façon à obtenir
{\ppp1\mmm} {\ppp200\mmm} grammes de liquide.

Lavez le riz à l'eau froide, laissez-le tremper, puis séchez-le.

Faites revenir, sans prendre couleur, le lard ou le jambon haché, l’échalote,
l'oignon, le bouquet garni dans {\ppp30\mmm} grammes de beurre, mouillez avec un peu de
bouillon de poulet ; laissez cuire ; passez ensuite la cuisson.

Mettez dans une casserole {\ppp100\mmm} grammes de beurre, chauffez, ajoutez le riz,
arrosez avec la cuisson passée ; ne couvrez pas. Dès que le bouillon sera
absorbé, agitez la casserole pour détacher les grains de riz qui auraient pu
adhérer au fond, mouillez de nouveau avec du bouillon et continuez ainsi
jusqu'à ce que tout le bouillon soit absorbé. Goûtez, complétez
l'assaisonnement s'il est nécessaire, ajoutez le fromage, mélangez, puis versez
le tout dans une moule annulaire. Tenez au chaud au bain-marie.

Pilez au mortier la chair de poulet, assaisonnez avec sel, poivre et épices, au
goût, ajoutez des truffes, pilez encore. Passez le tout au tamis, incorporez-y
la crème par petites quantités, les jaunes d'œufs et les blancs battus en
neige, de façon à avoir un appareil moelleux et léger.

Beurrez un moule à charlotte avec le reste du beurre, mettez dedans l'appareil
et faites cuire au four, au bain-marie, pendant une vingtaine de minutes
environ.

Démoulez le turban de riz sur un plat, décorez-le avec des émincés de truffes,
disposez au centre le soufflé de poulet et servez, en envoyant en même temps
une sauce Godard ou une sauce Chivry.

\sk

On prépare la sauce Godard de la façon suivante.

Pour six personnes prenez :

\medskip

\footnotesize
\begin{longtable}{rrrp{16em}}
    450 & grammes & de & sauce espagnole grasse,                                                          \\
    150 & grammes & de & champagne sec ou de chablis,                                                     \\
    125 & grammes & de & champignons,                                                                     \\
    125 & grammes & de & jambon salé, non fumé,                                                           \\
    125 & grammes & de & sous-noix de veau,                                                               \\
     50 & grammes & de & beurre,                                                                          \\
     50 & grammes & de & madère,                                                                          \\
     20 & grammes & de & carotte,                                                                         \\
     20 & grammes & d' & oignon,                                                                          \\
        &         &  1 & petit bouquet garni.                                                             \\
\end{longtable}
\normalsize

Faites revenir dans le beurre le jambon et le veau coupés en petits morceaux,
ajoutez ensuite carotte, oignon et bouquet garni ; laissez pincer légèrement,
puis mettez les champignons épluchés et émincés, mouillez avec le vin. Réduisez
de moitié ; passez la sauce en pressant. Remettez la sauce sur le feu, ajoutez
la sauce espagnole, faites réduire encore de moitié, dépouillez pendant la
cuisson. Aromatisez alors avec le madère, laissez cuire encore pendant une
dizaine de minutes. Enfin, passez la sauce à l'étamine.

\sk
\newpage

\index{Beurre Chivry}
\index{Sauce Chivry}
Pour préparer la sauce Chivry, prenez pour six personnes :

\medskip

\footnotesize
% \interfootnotelinepenalty=10000
\begin{longtable}{rrrp{16em}}
    250 & grammes & de & velouté de volaille\footnote{Le velouté de volaille s'obtient de la
                    façon suivante.
                    \protect\endgraf
                    \smallskip
                    Pour faire un litre de velouté, prenez :                                              \\
                    \smallskip
                    \begin{tabular}{rrrrl}
                                          &  1 100 & grammes & de & fond blanc de volaille                \\
                                          &     70 & grammes & de & farine,                               \\
                                          &     50 & grammes & de & beurre.                               \\
                    \end{tabular}
                    \smallskip
                    \protect\endgraf
                    Faites revenir doucement la farine dans le
                    beurre, de façon à obtenir un roux de teinte
                    ivoire. Mouillez avec le fond de volaille,
                    amenez à ébullition, puis laissez cuire à petit
                    feu pendant une heure et demie environ, dépouillez
                    fréquemment la sauce pendant la cuisson, passez-la
                    ensuite à l'étamine et refroidissez-la en la remuant.
                    \sks
                    \index{Fond de volaille}
                    \hspace{1.9em}On prépare le fond de volaille ainsi qu'il suit.                        \\
                    \protect\endgraf
                    Pour faire trois litres de fond, prenez :                                             \\
                    \begin{tabular}{rrrrl}
                                         & 2 500 & grammes & de & jarret de veau,                         \\
                                         &   500 & grammes & d' & os de veau,                             \\
                                         &   250 & grammes & de & carottes,                               \\
                                         &   110 & grammes & d' & oignons,                                \\
                                         &    65 & grammes & de & poireaux,                               \\
                                         &    30 & grammes & de & céleri,                                 \\
                                         &    20 & grammes & de & sel,                                    \\
                                         &     3 & litres 1& /3 & d' eau,                                 \\
                                         &       &         &  1 & vieille poule avec son abatis,          \\
                                         &       &         &  1 & abatis de poule supplémentaire,         \\
                                         &       &         &  1 & clou de girofle,                        \\
                                         &       &         &  1 & bouquet garni, composé de 20 grammes    \\
                                         &       &         &    & de persil, quelques brindilles de thym  \\
                                         &       &         &    & et un peu de laurier.                   \\
                    \end{tabular}
                    \smallskip
                    \protect\endgraf
                    Faites cuire le tout doucement, comme pour un pot-au-feu,
                    pendant 3 à 4 heures, écumez, dégraissez et passez au
                    chinois.}                                                    \\
     50 & grammes & de & vin blanc sec,                                                                   \\
     50 & grammes & d' & un mélange en parties égales de ciboulette, pimprenelle, persil,
                        cerfeuil et estragon frais,                                                       \\
     40 & grammes & de & beurre,                                                                          \\
      7 & grammes & d' & échalote.                                                                        \\
\end{longtable}
\normalsize

Faites blanchir pendant quelques minutes l'échalote et {\ppp35\mmm} grammes du mélange de
plantes aromatiques, rafraîchissez-les, essuyez-les.

Préparez un beurre Chivry en broyant au mortier ces plantes et l'échalote avec
le beurre ; passez le tout à l'étamine. Faites bouillir le vin, mettez dedans
le reste des herbes aromatiques, couvrez, laissez infuser hors du feu pendant
une dizaine de minutes ; passez au travers d’un linge. Versez cette infusion
dans le velouté de volaille bouillant et montez la sauce avec le beurre Chivry.

\section*{\centering Bouchées de poulet farcies, sauce suprême.}
\phantomsection
\addcontentsline{toc}{section}{ Bouchées de poulet farcies, sauce suprême.}
\index{Bouchées de poulet farcies, sauce suprême}

Pour dix personnes prenez :

\smallskip

\index{Enveloppe pour bouchées de poulet}
\index{Bouchées (Enveloppes pour)}
1° pour l'enveloppe :

\smallskip

\footnotesize
\begin{longtable}{rrrp{16em}}
    500 & grammes & de & fond de volaille concentré,                                                      \\
    300 & grammes & de & blanc de poulet,                                                                 \\
    200 & grammes & de & farine,                                                                          \\
    175 & grammes & de & graisse de volaille,                                                             \\
    125 & grammes & de & graisse de rognon de veau,                                                       \\
        &         &  4 & œufs entiers,                                                                    \\
        &         &  2 & blancs d'œufs,                                                                   \\
        &         &  1 & jaune d'œuf,                                                                     \\
        &         &    & muscade,                                                                         \\
        &         &    & sel et poivre ;                                                                  \\
\end{longtable}
\normalsize

\smallskip

\index{Farce pour bouchées de poulet}
\index{Farce pour bouchées de volaille}
2° pour la farce :

\smallskip

\footnotesize
\begin{longtable}{rrrp{16em}}
    300 & grammes & d' & un mélange de foies de volaille cuits au beurre,
                         de rognons de coq blanchis et de champignons grillés ;                           \\
\end{longtable}
\normalsize

3° pour la sauce :

\medskip

\footnotesize
\begin{longtable}{rrrp{16em}}
    150 & grammes & de & gelée de veau et volaille,                                                       \\
     60 & grammes & de & crème épaisse,                                                                   \\
     40 & grammes & de & vin de Sauternes,                                                                \\
     25 & grammes & de & beurre,                                                                          \\
     15 & grammes & de & farine,                                                                          \\
        &         &  1 & jaune d'œuf frais,                                                               \\
        &         &    & jus de citron,                                                                   \\
        &         &    & sel et poivre.                                                                   \\
\end{longtable}
\normalsize

Triturez ensemble la farine, les œufs entiers, ajoutez en tournant le fond de
volaille chaud, par petites quantités ; faites prendre sur feu doux en
travaillant l'appareil pendant une demi-heure environ, de façon à l’amener
à bonne consistance.

Pilez séparément blanc de poulet, graisse de volaille et graisse de veau ;
réunissez-les, assaisonnez avec sel, poivre et muscade au goût ; pilez encore ;
mélangez, puis incorporez l'appareil ci-dessus et ajoutez les deux blancs
d'œufs battus en neige.

Passez le tout au tamis ; travaillez bien la pâte pour la rendre lisse et
homogène ; puis abaissez-la à une épaisseur de {\ppp5\mmm} à {\ppp6\mmm} millimètres et taillez
dedans vingt morceaux carrés.

Mettez sur chaque morceau de pâte un vingtième de la farce ; fermez les
bouchées.

Préparez la sauce.

Mettez dans une casserole le beurre et la farine ; tournez sans laisser prendre
couleur, délayez avec la gelée de veau et volaille, mouillez avec le sauternes,
faites cuire doucement pendant un quart d'heure, puis ajoutez la crème,
mélangez bien, ne laissez plus bouillir ; enfin, achevez la liaison avec le
jaune d'œuf. Goûtez et complétez l'assaisonnement avec sel, poivre et jus de
citron.

Faites pocher les bouchées dans de l'eau salée bouillante ; égouttez-les,
dorez-les au jaune d'œuf et passez-les au four.

Dressez les bouchées sur un plat garni d'une serviette et servez en envoyant en
même temps la sauce dans une saucière.

\sk

On conçoit facilement qu'en changeant la composition de la pâte, de la farce et
de la sauce, on aura de nombreuses variantes de bouchées de volaille, très
intéressantes.

\section*{\centering Poularde rôtie, sauce ivoire.}
\phantomsection
\addcontentsline{toc}{section}{ Poularde rôtie, sauce ivoire.}
\index{Poularde rôtie, sauce ivoire}

La poularde rôtie, sauce ivoire, est une variété de poulet au blanc.

Faites rôtir une poularde comme d'ordinaire, puis préparez une sauce suprême et
incorporez-y le jus de cuisson de la poularde : cette addition suffira le plus
souvent pour lui donner une couleur ivoire ; dans le cas contraire, on ajoutera
un peu de caramel pour obtenir la teinte voulue.

Au moment de servir, découpez la poularde, dressez les morceaux sur un plat et
masquez-les avec la sauce.

\section*{\centering Poularde demi-deuil.}
\phantomsection
\addcontentsline{toc}{section}{ Poularde demi-deuil.}
\index{Poularde demi-deuil}

Pour six personnes prenez :

\medskip

\footnotesize
\begin{longtable}{p{3em}p{2em}p{4em}p{1em}p{12em}}
     1° &     &         &  1 & poularde,                                                                  \\
        &     &         &    & fond de veau,                                                              \\
        &     &         &    & truffes à volonté,                                                         \\
        &     &         &    & madère,                                                                    \\
        &     &         &    & bouquet garni ;                                                            \\
\end{longtable}
\normalsize

\medskip

\footnotesize
\setlength\tabcolsep{.15em}
\begin{longtable}{p{3em}p{2em}p{4em}p{1em}p{12em}}
     2° & 750 & grammes & de & sous-noix de veau,                                                         \\
        & 250 & grammes & de & jambon,                                                                    \\
        & 200 & grammes & de & beurre,                                                                    \\
        & 150 & grammes & de & crème épaisse,                                                             \\
        & 125 & grammes & de & champignons,                                                               \\
        &  60 & grammes & de & farine,                                                                    \\
        &     & 1 litre & de & consommé,                                                                  \\
        &     &         &  3 & carottes moyennes,                                                         \\
        &     &         &  2 & oignons moyens,                                                            \\
        &     &         &  1 & bouquet garni,                                                             \\
        &     &         &    & céleri,                                                                    \\
        &     &         &    & sel et poivre.                                                             \\
\end{longtable}
\normalsize

Faites cuire : d’une part, les truffes dans du madère ; d'autre part, la
poularde pendant une heure dans le fond de veau avec le bouquet ; dégraissez et
concentrez la cuisson.

\label{pg0566} \hypertarget{p0566}{}
En même temps, préparez une bonne sauce Béchamel grasse de la façon suivante :
faites revenir dans le beurre, pendant dix minutes, le veau et le jambon coupés
en morceaux gros comme des noix, les oignons, les carottes et du céleri
émincés, ajoutez la farine, tournez pendant cinq minutes, mouillez avec le
consommé, mettez les champignons émincés, le bouquet, du sel, du poivre, donnez
un bouillon, puis continuez la cuisson à petit feu, en laissant mijoter pendant
deux heures. Écumez, dégraissez, passez la sauce, mélangez-la avec la crème
épaisse, chauffez sans faire bouillir, amenez le tout à consistance voulue pour
masquer une cuiller.

Découpez la volaille, dressez les morceaux sur un plat, masquez-les avec la
sauce Béchamel dans laquelle vous aurez mis le jus de cuisson de la poularde et
une truffe hachée fin, décorez en mettant sur chaque morceau de poularde
quelques rondelles de truffe et servez.

\section*{\centering Poularde farcie de langue, braisée.}
\phantomsection
\addcontentsline{toc}{section}{ Poularde farcie de langue, braisée.}
\index{Poularde farcie de langue, braisée}
Pour douze personnes prenez :

\medskip

\footnotesize
\begin{longtable}{rrrp{16em}}
    250 & grammes & de & crème,                                                                           \\
    250 & grammes & de & graisse de volaille,                                                             \\
    100 & grammes & de & carottes,                                                                        \\
    100 & grammes & d’ & oignons,                                                                         \\
        &         &  1 & litre de bouillon,                                                               \\
        &         &  1 & belle poularde,                                                                  \\
        &         &  1 & poulet,                                                                          \\
        &         &  1 & langue de bœuf fumée,                                                            \\
        &         &  1 & barde de lard,                                                                   \\
        &         &  1 & bouquet garni (persil, thym, laurier),                                           \\
        &         &    & truffes à volonté,                                                               \\
        &         &    & sel et poivre.                                                                   \\
\end{longtable}
\normalsize

Mettez la langue à dessaler pendant {\ppp24\mmm} heures, puis faites-la cuire
dans un court-bouillon au vin ; laissez-la refroidir dans sa cuisson. Enlevez
la peau de la langue.

Désossez la poitrine de la poularde sans toucher aux ailes, en opérant par le
cou, de façon à n'abîmer ni la peau, ni les filets.

\index{Darioles de poulet à la crème et aux truffes}
Préparez une farce avec la chair du poulet, {\ppp150\mmm} grammes de crème et
des truffes, salez et poivrez ; mettez-en les trois quarts dans la poularde ;
garnissez avec le reste des moules à dariole\footnote{\index{Définition des darioles}
                                                      \index{Darioles (Définition des)}
Les darioles sont des petites préparations moulées, chaudes ou froides, qui
servent de garniture.} beurrés.

Enlevez sur la langue dépouillée un quart de son volume, du côté de la pointe,
qui servira à décorer le plat, puis introduisez le reste dans la poularde, par
le cou, au milieu de la farce.

Faites cuire les déchets de poularde et de poulet dans le bouillon, passez le jus ;
réservez-le.

Bridez la poularde, bardez-la de lard, mettez-la dans une braisière avec la
graisse de volaille, les carottes, les oignons, le bouquet garni, mouillez avec
le jus réservé ; laissez mijoter pendant une heure et demie.

Dix minutes avant de servir, passez la cuisson, dégraissez-la, liez-la avec le reste
de la crème et ajoutez des truffes hachées.

En même temps, faites cuire les darioles au bain-marie.

Dressez la poularde sur un plat, disposez autour les darioles démoulées comme
garniture, décorez avec des truffes et des tranches de langue, taillées dans le
morceau réservé, de façon que les convives aient immédiatement sous les yeux
tous les éléments constitutifs du plat. Envoyez la sauce dans une saucière.

Cette poularde se recommande d'elle-même aux amateurs.

\sk

Comme variante, on pourra servir cette poularde avec une sauce suprême dans
laquelle on aura incorporé du parmesan.

\section*{\centering Poularde farcie de pieds de mouton, braisée.}
\phantomsection
\addcontentsline{toc}{section}{ Poularde farcie de pieds de mouton, braisée.}
\index{Poularde farcie de pieds de mouton, braisée}

Pour quatre personnes prenez :

\medskip

\footnotesize
\begin{longtable}{rrrp{16em}}
    200 & grammes & de & champignons de couche,                                                           \\
    200 & grammes & de & carottes,                                                                        \\
     80 & grammes & de & beurre,                                                                          \\
     75 & grammes & d' & oignons,                                                                         \\
        &         &  8 & pieds de mouton,                                                                 \\
        &         &  1 & petite poularde bien en chair,                                                   \\
        &         &  1 & bouquet garni,                                                                   \\
        &         &    & jus de citron,                                                                   \\
        &         &    & sel et poivre.                                                                   \\
\end{longtable}
\normalsize

Nettoyez la poularde, videz-la, réservez l'abatis.

Faites cuire les pieds de mouton comme il est dit
\hyperlink{p0445}{p. \pageref{pg0445}} ; désossez-les ; coupez-les en languettes.

Épluchez les champignons, passez-les dans du jus de citron ; émincez-les.

\index{Farce aux pieds de mouton}
Faites sauter dans {\ppp40\mmm} grammes de beurre pieds de mouton et champignons ; salez
el poivrez.

Farcissez la poularde avec le mélange pieds de mouton et champignons sautés,
puis faites-la dorer, en cocote, dans le reste du beurre ; ajoutez ensuite
l'abatis, les carottes, les oignons et le bouquet ; lutez la cocote avec de la
pâte à pain et laissez cuire au four doux pendant une heure et demie.

Retirez l'abatis, les légumes et le bouquet ; dégraissez et passez la sauce.
Dressez la poularde sur un plat, masquez-la avec la sauce et servez.

Les pieds de mouton donnent à cette préparation un moelleux particulier très
agréable.

\sk

Comme variantes, on pourra remplacer les champignons de couche par des cèpes,
des morilles on des truffes.

\section*{\centering Poularde farcie de morilles, braisée.}
\phantomsection
\addcontentsline{toc}{section}{ Poularde farcie de morilles, braisée.}
\index{Poularde farcie de morilles, braisée}

Pour huit personnes prenez :

\medskip

\footnotesize
\begin{longtable}{rrrp{16em}}
  1 500 & grammes & de & fond de veau,                                                                    \\
    500 & grammes & de & morilles grises,                                                                 \\
    300 & grammes & de & riz,                                                                             \\
    200 & grammes & de & pointes d'asperges,                                                              \\
    100 & grammes & de & crêtes et de rognons de coq,                                                     \\
     80 & grammes & de & beurre,                                                                          \\
     25 & grammes & de & farine,                                                                          \\
        &         &  2 & jaunes d'œufs frais,                                                             \\
        &         &  2 & foies de volaille,                                                               \\
        &         &  1 & belle poularde,                                                                  \\
        &         &    & parmesan,                                                                        \\
        &         &    & sel et poivre.                                                                   \\
\end{longtable}
\normalsize

Troussez la poularde et faites-la braiser dans le fond de veau.

En même temps, blanchissez séparément les pointes d'asperges, les crêtes et les
rognons de coq ; cuisez les morilles dans du jus avec les foies de volaille et
apprêtez avec le riz un risotto, \hyperlink{p0712-2}{p. \pageref{pg0712-2}}, auquel
vous incorporerez les crêtes, les rognons et les foies de volaille.

Tenez le tout au chaud.

Préparez alors la sauce : faites cuire la farine dans {\ppp60\mmm} grammes de beurre, sans
lui laisser prendre couleur, mouillez avec la cuisson de la poularde, dépouillez à
petit feu, concentrez, achevez la liaison avec les jaunes d'œufs, assaisonnez au
goût, enfin mettez le reste du beurre coupé en petits morceaux et laissez-le fondre.

Découpez dans l'estomac de la poularde un rectangle aussi long et aussi large
que possible et emplissez la volaille avec les morilles par l'ouverture ainsi
pratiquée.

Emincez l'estomac en aiguillettes, remettez le tout en place de façon
à reconstituer la poularde, masquez avec de la sauce, saupoudrez de parmesan
râpé et faites gratiner au four.

Dressez le risotto en socle sur un plat, entourez-le d'une bordure de pointes
d’asperges, disposez la poularde sur le socle et envoyez en même temps le reste
de la sauce dans une saucière.

On peut, cela va sans dire, décorer le plat avec des truffes, mais c'est
absolument inutile, à mon avis.

Cette poularde, simplement garnie de morilles, repose des poulardes farcies de
mélanges compliqués.

\section*{\centering Poularde farcie de nouilles, braisée.}
\phantomsection
\addcontentsline{toc}{section}{ Poularde farcie de nouilles, braisée.}
\index{Poularde farcie de nouilles, braisée}

Pour six personnes prenez :

\medskip

\footnotesize
\begin{longtable}{rrrp{16em}}
    750 & grammes & de & fond de veau et de volaille,                                                     \\
    300 & grammes & de & sauce Mornay,                                                                    \\
    200 & grammes & de & nouilles,                                                                        \\
    125 & grammes & de & rognons de coq,                                                                  \\
    125 & grammes & de & chapeaux de petits champignons,                                                  \\
    125 & grammes & de & crème épaisse,                                                                   \\
    125 & grammes & de & fromage (gruyère ou parmesan râpé ou mélange des deux),                          \\
        &         &  1 & poularde,                                                                        \\
        &         &    & queues d'écrevisses ou escalopes de homard,                                      \\
        &         &    & beurre,                                                                          \\
        &         &    & sel et poivre.                                                                   \\
\end{longtable}
\normalsize


Faites pocher à moitié les nouilles dans {\ppp500\mmm} grammes de fond, ajoutez les
rognons de coq blanchis et les chapeaux de champignons passés au beurre, le ou
les fromages et la crème, salez et poivrez au goût ; mélangez et farcissez la
poularde avec ce mélange.

Faites dorer la poularde dans un peu de beurre ; égouttez-la ; mouillez avec le
reste du fond ; laissez cuire à petit feu, en arrosant pendant la cuisson.

Au dernier moment, dressez la poularde sur un plat allant au feu, masquez-la
avec la sauce Mornay et glacez au four.

Décorez le plat avec des queues d'écrevisses ou des escalopes de homard chaudes
et servez.

\section*{\centering Poularde braisée au vin.}
\phantomsection
\addcontentsline{toc}{section}{ Poularde braisée au vin.}
\index{Poularde braisée au vin}

Pour huit personnes prenez :

\medskip

\footnotesize
\begin{longtable}{rrp{16em}}
  1 litre & de & gelée de veau et volaille, \hyperlink{p0418}{p. \pageref{pg0418}},                       \\
        1 bouteille & de & vin blanc sec (bourgogne, champagne ou vin du Rhin),                           \\
                    &  4 & jaunes d'œufs frais,                                                           \\
                    &  1 & belle poularde,                                                                \\
                    &    & graisse de porc,                                                               \\
                    &    & carottes,                                                                      \\
                    &    & bouquet garni.                                                                 \\
\end{longtable}
\normalsize

Faites dorer la poularde dans la graisse, retirez-la, mettez-la dans une
braisière, mouillez avec le vin, ajoutez la gelée, des carottes, un bouquet
garni et laissez cuire à petit feu pendant deux heures. Enlevez alors la
poularde, tenez-la au chaud, passez la cuisson, concentrez-la et montez-la avec
les jaunes d'œufs.

Dressez la poularde sur un plat, masquez-la avec la sauce et garnissez le
pourtour avec des cèpes grillés \hyperlink{p0805}{p. \pageref{pg0805}} et bien
égouttés, de manière qu'ils ne soient qu'imperceptiblement parfumés à l'ail.

\section*{\centering Poularde braisée à l'estragon.}
\phantomsection
\addcontentsline{toc}{section}{ Poularde braisée à l'estragon.}
\index{Poularde braisée à l'estragon}

Troussez une belle poularde.

Préparez un bon fond avec l'abatis de la poularde, un abatis supplémentaire, du
jarret et du pied de veau, des légumes, de l’eau, du sel, du poivre et des
aromates. Dégraissez-le, passez-le.

Dorez légèrement la poularde dans {\ppp60\mmm} grammes de beurre, puis achevez sa cuisson
dans le fond préparé auquel vous ajouterez un fort bouquet d’estragon. Laissez
cuire à petit feu, pendant {\ppp1\mmm} heure et demie à {\ppp2\mmm} heures, suivant la grosseur de
l'animal.

Réduisez le fond de cuisson, liez-le, passez-le, puis additionnez-le de
feuilles d'estragon frais hachées,

Découpez la poularde, dressez-la sur un plat en la reconstituant, décorez-la
avec des feuilles d'estragon frais légèrement blanchies, masquez-la avec un peu
de sauce et servez en envoyant en même temps le reste de la sauce dans une
saucière.

\section*{\centering Poularde truffée pochée.}
\phantomsection
\addcontentsline{toc}{section}{ Poularde truffée pochée.}
\index{Poularde truffée pochée}

Pour six personnes prenez :

\medskip

\footnotesize
\begin{longtable}{rrrp{20em}}
        &         &  1 & \hangindent=1em poularde pesant environ 1 500 grammes, sans l'abatis,            \\
        &         &  2 & \hangindent=1em vessies de porc de dimensions suffisantes pour contenir
                         la poularde, mais l'une plus grande que l'autre,                                 \\
        &         &    & truffes à volonté,                                                               \\
        &         &    & beurre,                                                                          \\
        &         &    & madère,                                                                          \\
        &         &    & sel et poivre.                                                                   \\
\end{longtable}
\normalsize

Nettoyez soigneusement les deux vessies avec une brosse et de l’eau
savonneuse ; rincez-les bien à l'eau claire.

Brossez les truffes, lavez-les, séchez-les dans un linge ; mettez-les dans du
madère avec un peu de sel et de poivre ; laissez en contact pendant une heure,
puis faites-les cuire dans ce même madère.

Mettez dans l'intérieur de la poularde les truffes, {\ppp30\mmm} grammes environ de leur
cuisson, du sel et du poivre ; bridez-la, enduisez-la extérieurement d’un peu
de beurre et introduisez-la dans la vessie la plus petite. Ficelez
hermétiquement, puis insérez le tout dans la seconde vessie que vous fermerez
hermétiquement aussi\footnote{Il peut arriver que, par suite de la cuisson
prolongée, la vessie extérieure, ramollie, crève ; le rôle de la seconde vessie
est d'empêcher le contact de la poularde avec l'eau.}.

Faites bouillir de l'eau dans une grande marmite, plongez dedans la poularde
enveloppée de ses vessies, laissez-la cuire pendant deux heures ; puis
retirez-la et débarrassez-la de ses enveloppes.

Découpez la poularde, dressez les morceaux sur un plat, décorez avec les
truffes, arrosez avec le jus de cuisson et servez.

La poularde reste absolument blanche ; elle est fondante et elle conserve
intégralement toute sa saveur,

Ce procédé de pochage est très recommandable pour toute sorte de viandes,
volailles et gibiers auxquels on veut conserver leur arome propre.

\section*{\centering Poule au riz.}
\phantomsection
\addcontentsline{toc}{section}{ Poule au riz.}
\index{Poule au riz}

Pour six personnes prenez :

\medskip

\footnotesize
\begin{longtable}{rrrp{16em}}
    375 & grammes & de & riz,                                                                             \\
    125 & grammes & de & fromage de Gruyère râpé,                                                         \\
        & 1 litre & de & bouillon,                                                                        \\
        &         &  1 & jeune poule tendre\footnote{Avec une vieille poule la cuisson
                                            demanderait plus de temps, 3 à 4 heures et
                                            même davantage et le plat serait certainement
                                            moins bon.},                                                  \\
        &         &  1 & bouquet garni (persil, thym et laurier),                                         \\
        &         &    & du beurre, en proportion inverse de la quantité
                         de graisse dont est pourvue la poule,                                            \\
        &         &    & sel et poivre.                                                                   \\
\end{longtable}
\normalsize

Mettez la poule dans une casserole avec le bouquet, assaisonnez au goût,
mouillez a moitié hauteur avec le bouillon, couvrez d'un papier beurré et
laissez cuire à petit feu pendant une heure et demie ; retirez le bouquet.

Faites cuire le riz pendant dix minutes dans de l’eau salée, égouttez-le,
passez-le à l'eau froide, puis achevez sa cuisson dans le bouillon de poule
qu'il absorbera.

Amenez le riz, comme dans la préparation du riz au gras,
\hyperlink{p0708}{p. \pageref{pg0708}}, à être en grains moelleux, entiers, non
crevés, incorporez-y alors le gruyère râpé et, au besoin, un peu de beurre que
vous laisserez fondre.

Dressez sur un plat, le riz au fond, la poule dessus et servez.

\sk

Comme variantes, on peut mettre avec le riz, en place du fromage, {\ppp125\mmm}
grammes de champignons de couche ou de morilles, ou {\ppp65\mmm} grammes de
cèpes secs détrempés, coupés en morceaux et sautés au beurre.

Enfin, pendant la saison des petits pois, on pourra également remplacer le
fromage par des petits pois que l’on fera cuire avec la poule.

\section*{\centering Coq au vin.}
\phantomsection
\addcontentsline{toc}{section}{ Coq au vin.}
\index{Coq au vin}

Ce plat remonte au \textsc{xvi}\textsuperscript{e} siècle. Il était connu
à cette époque sous le nom claironnant de « Coq au vin », et on le préparait
rapidement, en présence des convives, devant un grand feu de bois très vif,
dans les vieilles « Hostelleries » de France. Mais, comme il peut être fait
tout aussi bien avec une jeune poule qu'avec un jeune coq et qu'en définitive
c'est un ragoût, il vaudrait mieux le désigner sous le nom de « Ragoût de
poulet au vin ».

Pour quatre personnes prenez :

\medskip

\footnotesize
\begin{longtable}{rrrp{16em}}
    125 & grammes & de & lard de poitrine coupé en dés,                                                   \\
    125 & grammes & de & champignons frais de saison,                                                     \\
    100 & grammes & de & vin (de préférence vin blanc un peu sec),                                        \\
     65 & grammes & de & beurre,                                                                          \\
     50 & grammes & de & fine champagne ou d'armagnac,                                                    \\
      5 & grammes & d' & ail haché fin,                                                                   \\
        &         &  6 & petits oignons,                                                                  \\
        &         &  4 & petites échalotes,                                                               \\
        &         &  1 & jeune poulet tendre et gras,                                                     \\
        &         &  1 & bouquet de persil, thym et laurier,                                              \\
        &         &    & carotte,                                                                         \\
        &         &    & farine,                                                                          \\
        &         &    & sel et poivre.                                                                   \\
\end{longtable}
\normalsize

Coupez le poulet en morceaux, comme pour une fricassée.

Faites dorer, dans {\ppp40\mmm} grammes de beurre, le lard, les oignons, les échalotes et
quelques rondelles de carotte ; retirez-les ensuite et remplacez-les par les
morceaux de poulet que vous assaisonnnerez avec sel, poivre et ail et que vous
ferez revenir à feu très vif ; puis remettez lard, carotte, oignons et
échalotes revenus ; flambez à la fine champagne ou à l'armagnac, saupoudrez de
farine, tournez pendant quelques instants ; ajoutez ensuite les champignons
épluchés, mouillez avec le vin, couvrez et faites cuire à feu très vif, pendant
un quart d'heure environ.

Au dernier moment, mettez les morceaux de poulet et le lard dans un plat tenu
au chaud, dégraissez la sauce, passez-la, montez-la avec le reste du beurre,
versez-la sur le poulet et servez.

\section*{\centering Tartines de rognons de coq, en gelée.}
\phantomsection
\addcontentsline{toc}{section}{ Tartines de rognons de coq, en gelée.}
\index{Tartines de rognons de coq, en gelée}

Pour six personnes prenez :

\medskip

\footnotesize
\begin{longtable}{rrrp{16em}}
    250 & grammes & de & rognons de coq,                                                                  \\
    150 & grammes & de & crème fouettée,                                                                  \\
    125 & grammes & de & fond de veau aromatisé et concentré,                                             \\
    100 & grammes & de & foie gras d'oie, cuit,                                                           \\
        &         & 12 & petites tranches de pain de mie anglais,                                         \\
        &         &    & truffes à volonte,                                                               \\
        &         &    & madère,                                                                          \\
        &         &    & beurre,                                                                          \\
        &         &    & sel, poivre, paprika, quatre épices.                                             \\
\end{longtable}
\normalsize

Faites cuire : les truffes dans du madère, les rognons de coq dans le fond de
veau chaud mais non bouillant ; dès que les rognons seront fermes, retirez-les,
tenez-les au chaud.

Faites dorer les tranches de pain de mie dans du beurre ; laissez-les refroidir.

Passez en purée les rognons et le foie gras, ajoutez la crème fouettée,
travaillez le tout sur glace, de manière à obtenir un ensemble parfaitement
homogène ; assaisonnez le mélange au goût. Étalez-le sur les tranches de pain,
décorez avec des lames de truffe, masquez avec le fond de veau tiède, laissez
refroidir et prendre en gelée.

Dressez les tartines sur un plat garni d'une serviette et servez comme
hors-d'œuvre.

\section*{\centering Rognons de coq et foies de volaille en aspic.}
\phantomsection
\addcontentsline{toc}{section}{ Rognons de coq et foies de volaille en aspic.}
\index{Rognons de coq et foies de volaille en aspic}
\index{Aspic de rognons de coq et de foies de volaille}

Préparez une bonne gelée d'aspic avec jarret et pied de veau, abatis de
volaille et légumes ; clarifiez-la bien.

Mettez dans un moule une couche de cette gelée ; au-dessus une couche d'un
mélange de rognons de coq, foies de volaille et ris de veau cuits au préalable
dans un bon jus, le tout additionné d'un peu de langue à l'écarlate ; recouvrez
de sauce chaud-froid blanche, serrée et un peu relevée. Continuez ces
alternances jusqu'à ce que le moule soit plein. Faites prendre sur glace.

Démoulez au moment de servir.

C'est une excellente entrée de déjeuner.

\sk

\index{Aspic de rognons de coq et de foie gras}
Il va sans dire qu'on peut remplacer les foies de volaille par des émincés de
foie gras.

\sk

\index{Darioles de rognons de coq et de foie gras}
\index{Darioles de rognons de coq et de foies de volaille}
\index{Darioles de rognons et foies de gibier}
On peut préparer de même des petits aspics dans des moules à dariole et les
disposer comme garniture autour d'un chaud-froid de volaille.

\sk

\index{Aspic de rognons et foies de gibier}
Les rognons et les foies de gibier pourront être apprêtés de même ; mais il
conviendra alors d'aromatiser la gelée et la sauce, qui sera ici une sauce
chaud-froid brune, avec du fumet de gibier.

\section*{\centering Salpicon de foies de volaille, de crètes de coq, de rognons de coq et de truffes.}
\phantomsection
\addcontentsline{toc}{section}{ Salpicon de foies de volaille, de crètes de coq, de rognons de coq et de truffes.}
\index{Salpicon de foies de volaille, de crètes de coq, de rognons de coq et de truffes}

Pour six personnes prenez :

\medskip

\footnotesize
\begin{longtable}{rrrp{16em}}
    400 & grammes & de & foies de volaille,                                                               \\
    250 & grammés & de & consommé,                                                                        \\
    150 & grammes & de & champignons de couche,                                                           \\
    125 & grammes & de & madère,                                                                          \\
    125 & grammes & de & crêtes de coq,                                                                   \\
    100 & grammes & de & rognons de coq,                                                                  \\
    100 & grammes & de & beurre,                                                                          \\
     75 & grammes & de & truffes noires du Périgord, au moins,                                            \\
     50 & grammes & de & jus de viande,                                                                   \\
     20 & grammes & de & farine,                                                                          \\
        &         &    & jus de citron,                                                                   \\
        &         &    & sel et poivre.                                                                   \\
\end{longtable}
\normalsize

Blanchissez, dans le consommé bouillant, les rognons et les crêtes pendant
trois minutes, les foies de volaille pendant cinq minutes ; retirez-les,
tenez-les au chaud ; réservez le consommé.

Faites un roux avec le beurre et la farine, mouillez avec le consommé réservé
et le madère ; laissez cuire doucement pendant trois quarts d'heure. Pendant la
cuisson, dépouillez la sauce, assaisonnez avec sel et poivre, après avoir
goûté, car tout dépend de l’assaisonnement du consommé employé, et acidulez
légèrement, au goût, avec du jus de citron.

Un quart d'heure avant la fin, mettez dans la sauce les truffes pelées et
coupées en tranches, les champignons épluchés et passés dans du jus de citron ;
puis, au dernier moment, ajoutez jus de viande, foies, rognons, crêtes,
mélangez bien, chauffez et servez sur un plat de riz au gras, préparé comme il
est dit \hyperlink{p0708}{p. \pageref{pg0708}}.

\section*{\centering Croquettes de volaille.}
\phantomsection
\addcontentsline{toc}{section}{ Croquettes de volaille.}
\index{Croquettes de volaille}

On prépare les croquettes de volaille surtout avec du blanc de poulet rôti.

\medskip

Pour quatre à six personnes prenez :

\footnotesize
\begin{longtable}{rrrp{16em}}
    250 & grammes & de & lait,                                                                            \\
    125 & grammes & de & champignons de couche,                                                           \\
     50 & grammes & de & beurre,                                                                          \\
     20 & grammes & de & farine,                                                                          \\
        &         &  2 & jaunes d'œufs frais,                                                             \\
        &         &  1 & œuf frais entier,                                                                \\
        &         &    & blanc d'un beau poulet rôti,                                                     \\
        &         &    & huile d'olive,                                                                   \\
        &         &    & mie de pain rassis tamisée,                                                      \\
        &         &    & persil,                                                                          \\
        &         &    & sel et poivre.                                                                   \\
\end{longtable}
\normalsize

Hachez le blanc de poulet ; pelez et hachez les champignons.

Faites un roux avec le beurre et la farine, mettez dedans le poulet, les
champignons et un gramme de persil haché, mouillez avec le lait, salez, poivrez
et laissez cuire jusqu'à ce que le lait soit réduit à point ; éloignez la
casserole du feu, liez avec les jaunes d'œufs, puis laissez refroidir
complètement.

Préparez des boulettes avec l'appareil ainsi obtenu.

Pour les bien réussir, enduisez-vous les mains de farine et faites attention
qu'il ne pénètre pas de farine dans l'intérieur des boulettes pendant leur
confection, autrement elles se désagrégeraient à la cuisson.

Battez l'œuf entier avec un peu d'huile d'olive, passez les boulettes dans le
mélange, roulez-les dans de la mie de pain rassis tamisée, puis plongez-les dans
de la friture bouillante.

Servez les croquettes sur un plat recouvert d'une serviette et décoré avec un
cordon de persil frit.

\section*{\centering Barquettes de volaille ou de gibier au foie gras.}
\phantomsection
\addcontentsline{toc}{section}{ Barquettes de volaille ou de gibier au foie gras.}
\index{Barquettes de volaille ou de gibier au foie gras}
\index{Barquettes de gibier au fois gras}
\index{Barquettes de volaille au foie gras}

Voici deux formules de pâte pour barquettes :

\footnotesize
\begin{longtable}{p{4em}rp{4em}rp{12em}}
\setlength\tabcolsep{.1em}
\hspace{.65em}1° & 300 & grammes & de & farine,                                                           \\
        & 150 & grammes & de & beurre frais,                                                              \\
        &   8 & grammes & de & sel,                                                                       \\
        &   5 & grammes & d' & eau,                                                                       \\
        &     &         &  1 & œuf frais,                                                                 \\
        &     &         &  1 & jaune d'œuf frais ;                                                        \\
\end{longtable}
\normalsize

\footnotesize
\begin{longtable}{p{4em}rp{4em}rp{12em}}
\setlength\tabcolsep{.1em}
\hspace{.65em}2° & 300 & grammes & de & farine,                                                           \\
        & 300 & grammes & de & beurre frais,                                                              \\
        &  25 & grammes & d' & eau,                                                                       \\
        &   8 & grammes & de & sel.                                                                       \\
\end{longtable}
\normalsize

Dans les deux cas, mélangez les différents éléments ; roulez la pâte en boule ;
laissez-la reposer pendant quelques heures.

Préparez une purée de volaille ou de gibier de la façon suivante.

Mettez à mariner pendant quelques heures du foie gras et des truffes dans un
peu de madère ; assaisonnez avec sel, poivre, épices.

Faites cuire, d'une part, les truffes dans le madère ; d'autre part, le foie
gras dans du fond de veau auquel vous ajouterez la cuisson des truffes ;
conduisez cette dernière opération à tout petit feu de manière que le foie
reste rose à l'intérieur.

Prenez des foies de volaille ou de gibier ; donnez-leur de la fermeté en les
passant dans de la graisse de volaille, de manière à les cuire tout en les
laissant roses.

Passez au tamis de crin de la chair de volaille ou de gibier que vous aurez
fait rôtir à la broche, les foies de volaille ou de gibier et le foie gras,
ajoutez les truffes hachées, du jus de cuisson du foie gras, assaisonnez avec
sel, poivre, cayenne, au goût.

Faites une abaisse mince avec la pâte ; chemisez avec cette abaisse des petits
moules ovales en forme de petites barques, emplissez-les de gravier lavé et
faites-les cuire au four.

Lorsque les croûtes sont cuites, laissez-les refroidir ; remplacez le gravier
par la purée préparée, masquez avec une bonne gelée, laissez prendre.

Ces barquettes constituent d'excellentes entrées froides.

\section*{\centering Galantines.}
\phantomsection
\addcontentsline{toc}{section}{ Galantines.}
\index{Galantines}
\index{Définition des galantines}
\index{Galantines (Définition des)}
\index{Galantine de volaille}

On désignait autrefois sous ce nom des mets composés de viandes hachées
enveloppées dans un poisson, un jeune animal tel qu'agneau ou cochon de lait,
une volaille, un gibier. Aujourd'hui, on ne prépare plus guère que des
galantines de volaille ou de gibier.

Le mode opératoire pour faire une galantine diffère un peu suivant qu'on
emploie une grosse ou une petite pièce.

\sk

Si vous prenez une grosse pièce, poularde, dinde, etc., désossez-la d'abord,
parez-la, enlevez-en les nerfs. Mettez de côté les os et les déchets. Retirez
ensuite de la chair aux cuisses et aux filets ; émincez-la. Avec les émincés
des cuisses tapissez les parties de l'animal dépourvues de chair. Mettez les
émincés de filets à mariner dans du cognac et du madère, et aussi tous les
éléments suivants ou quelques-uns d’entre eux seulement, au choix : émincés de
langue à l'écarlate, de jambon frais maigre, de lard gras, grenadins de veau,
petits cubes de foie gras et de truffes, etc.; assaisonnez avec sel, poivre et
épices.

\index{Farce pour galantines}
Préparez une farce fine avec du veau, du porc frais, les déchets de volaille
réservés, les pelures de truffes ; assaisonnez-la et ajoutez-y des pistaches
mondées.

Garnissez l'intérieur de la volaille soit avec des couches alternées d'éléments
marinés et de farce liée avec des œufs, soit avec un mélange des éléments
marinés, de la marinade et de la farce lié avec des œufs.

Roulez l'ensemble en forme de cylindre allongé ; enveloppez la galantine, ainsi
constituée, d'abord dans une fine barde de lard, puis dans un linge que vous
coudrez et faites-la cuire dans du fond de veau et volaille, doucement et sans
arrêt, à petits bouillons. La cuisson doit durer une demi-heure environ par
kilogramme de galantine.

Laissez refroidir incomplètement la galantine dans sa cuisson ; sortez-la
ensuite, resserrez-la dans le linge qui l'enveloppe et que vous recoudrez,
ficelez-la et faites-la refroidir sous pression légère.

Dégraissez le jus de cuisson ; clarifiez-le.

Déballez la galantine, masquez-la avec le jus de cuisson clarifié, de façon
à obtenir une épaisseur de gelée de un centimètre au moins, puis dressez-la sur
un plat. Décorez avec un hachis et des découpages de gelée.

\sk

\index{Galantine de gibier}
Si vous prenez du petit gibier, plumez-le, flambez-le, désossez-le, parez-le ;
réservez les os et les déchets.

Faites une farce avec une seconde pièce semblable ou avec un gibier d'une autre
sorte, du lard gras, de la langue à l'écarlate, des truffes et des pistaches,
de la fine champagne ; assaisonnez-la, liez-la avec des œufs et garnissez-en
l'animal désossé.

Roulez, bardez, enveloppez la galantine comme précédemment et faites-la cuire
dans du fond de veau relevé par le fumet du ou des gibiers choisis.

Laissez refroidir, pressez, déballez, dressez comme il est dit plus haut et
servez avec une sauce chaud-froid préparée avec la cuisson.

\section*{\centering Pintade farcie, rôtie.}
\phantomsection
\addcontentsline{toc}{section}{ Pintade farcie, rôtie.}
\index{Pintade farcie, rôtie}

La pintade, originaire d'Afrique où elle vivait en liberté, est aujourd'hui un
oiseau de basse-cour qui a conservé de sa vie primitive un petit fumet sauvage.
Sa chair, plus parfumée que celle du poulet, rappelle celle du faisan ; elle
est assez fine, mais un peu sèche. On remédie à ce défaut en farcissant la pintade avec une
substance plutôt grasse et en la bardant de lard avant de la faire cuire.

\index{Farce aux pieds de cochon}
\index{Farce pour pintade}
Je recommande comme farce deux pieds de cochon truffés par pintade.

L'animal farci et bardé est rôti à la broche comme un poulet. La durée moyenne
de la cuisson est une demi-heure.

Le meilleur accompagnement de ce rôti est une salade verte.

\section*{\centering Caneton rôti, sauce à l'orange.}
\phantomsection
\addcontentsline{toc}{section}{ Caneton rôti, sauce à l'orange.}
\index{Caneton rôti, sauce à l'orange}

Pour quatre personnes prenez :

\footnotesize
\begin{longtable}{rrrrrp{18em}}
  & \hspace{2em} & 100 & grammes & de & \hangindent=1em fond de veau, \hyperlink{p0426}{p. \pageref{pg0426}}, 
                                        aux éléments duquel on aura ajouté l'abatis du caneton,           \\
  & \hspace{2em} &  60 & grammes & de & curaçao blanc de Hollande,                                        \\
  & \hspace{2em} &  30 & grammes & de & beurre,                                                           \\
  & \hspace{2em} &  20 & grammes & de & glace de viande,                                                  \\
  & \hspace{2em} &  15 & grammes & de & farine,                                                           \\
  & \hspace{2em} &   5 & grammes & de & sel,                                                              \\
  & \multicolumn{3}{r}{1 décigramme} & de & poivre,                                                       \\
  & \hspace{2em} &     &         &  2 & oranges,                                                          \\
  & \hspace{2em} &     &         &  1 & caneton.                                                          \\
\end{longtable}
\normalsize

Pelez une orange, coupez la pulpe en morceaux et mettez-la dans le caneton ;
émincez le zeste et réservez-le.

Faites rôtir le caneton à la broche pendant une demi-heure environ,
assaisonnez-le avec le sel et le poivre.

En même temps, préparez la sauce.

Faites blanchir, pendant dix minutes, dans de l'eau bouillante, le zeste
d'orange émincé, égouttez-le, mettez-le dans un mortier avec le foie cru du
caneton, pilez, mouillez avec le curaçao.

Chauffez le fond de veau, ajoutez-y la glace de viande, le beurre manié avec la
farine, le foie pilé avec le zeste et aromatisé avec le curaçao, le jus
dégraissé de la lèchefrite, donnez un bouillon, puis passez le tout au tamis.

Enlevez les morceaux d'orange de l'intérieur du caneton, dressez-le sur un
plat, décorez avec la seconde orange coupée en tranches minces et servez.
Envoyez en même temps la sauce dans une saucière.

\sk

On peut préparer de même un pintadeau.

\section*{\centering Caneton aux navets nouveaux.}
\phantomsection
\addcontentsline{toc}{section}{ Caneton aux navets nouveaux.}
\index{Caneton aux navets nouveaux}

Pour six personnes prenez :

\medskip

\footnotesize
\begin{longtable}{rrrp{16em}}
    200 & grammes & de & beurre,                                                                          \\
     30 & grammes & d’ & oignons,                                                                         \\
     15 & grammes & de & farine,                                                                          \\
     10 & grammes & de & sucre en poudre,                                                                 \\
        &         & 30 & navets nouveaux,                                                                 \\
        &         &  1 & beau caneton,                                                                    \\
        &         &  1 & carotte,                                                                         \\
        &         &  1 & poireau,                                                                         \\
        &         &  1 & petit bouquet garni,                                                             \\
        &         &    & sel et poivre.                                                                   \\
\end{longtable}
\normalsize

Videz et flambez le caneton.

Préparez avec l'abatis, la carotte, le poireau, le bouquet, de l'eau et du sel
un demi-litre de bouillon concentré ; passez-le.

Faites dorer le caneton dans {\ppp60\mmm} grammes de beurre, salez, poivrez, ajoutez les
oignons et laissez cuire à petit feu pendant trois quarts d'heure à une heure.

Épluchez les navets, essuyez-les et faites-les cuire directement, à l’étouffée,
dans une casserole, avec {\ppp110\mmm} grammes de beurre. La cuisson doit durer une
demi-heure environ. Dix minutes avant la fin de l'opération, saupoudrez-les
avec le sucre et donnez-leur de la couleur.

Faites un roux avec le reste du beurre et la farine, mouillez avec le bouillon
d'abatis, ajoutez le caneton avec sa cuisson passée, les navets, goûtez,
complétez l’assaisonnement si c'est nécessaire, et laissez mijoter encore
pendant une dizaine de minutes. Dégraissez.

Dressez le caneton sur un plat, disposez les navets autour, masquez avec la
sauce et servez.

\section*{\centering Caneton poché aux légumes.}
\phantomsection
\addcontentsline{toc}{section}{ Caneton poché aux légumes.}
\index{Caneton poché aux légumes}

Pour six personnes prenez :

\medskip

\footnotesize
\begin{longtable}{rrrp{16em}}
        &         &  1 & beau caneton,                                                                    \\
        &         &  1 & foie de canard,                                                                  \\
        &         &    & crépine de veau,                                                                 \\
        &         &    & glace de viande,                                                                 \\
        &         &    & fine champagne,                                                                  \\
        &         &    & truffes,                                                                         \\
        &         &    & julienne de légumes,                                                             \\
        &         &    & bouillon,                                                                        \\
        &         &    & beurre,                                                                          \\
        &         &    & farine,                                                                          \\
        &         &    & échalotes,                                                                       \\
        &         &    & oignons,                                                                         \\
        &         &    & sel et poivre.                                                                   \\
\end{longtable}
\normalsize

Videz, flambez, troussez le caneton.

Faites blanchir la julienne de légumes dans du bouillon, réservez-la ; puis,
dans le même bouillon, faites cuire l'abatis du caneton, moins le foie, de
façon à obtenir un bon jus.

\index{Farce pour caneton}
Préparez une farce avec le foie du caneton, des oignons, des échalotes, du sel,
du poivre et de la fine champagne ; mettez-la dans le caneton.

Étalez sur la crépine la julienne de légumes et des émincés de truffes crues,
enveloppez-en le caneton, ficelez et faites pocher dans du bouillon pendant
trente-cinq minutes.

En même temps, préparez la sauce : faites un roux avec beurre et farine,
mouillez avec le jus, corsez avec un peu de glace de viande, concentrez la
cuisson, ajoutez le foie cru de canard passé au tamis, chauffez, montez la
sauce au beurre, goûtez et complétez l'assaisonnement s'il est nécessaire.

Retirez la crépine, dressez le caneton sur un plat, décorez avec les légumes et
les truffes. Servez, en envoyant en même temps la sauce dans une saucière.

\section*{\centering Canard rôti, sauce rouennaise.}
\phantomsection
\addcontentsline{toc}{section}{ Canard rôti, sauce rouennaise.}
\index{Canard rôti, sauce rouennaise}

Pour quatre personnes prenez :

\medskip

\footnotesize
\begin{longtable}{rrrp{16em}}
    200 & grammes & de & sauce bordelaise,                                                                \\
        &         &  1 & canard rouennais avec son abatis,                                                \\
        &         &  1 & foie de canard supplémentaire,                                                   \\
        &         &    & cayenne.                                                                         \\
\end{longtable}
\normalsize

Plumez, videz, troussez le canard et faites-le rôtir à la broche.

Passez au tamis le foie du canard rouennais et le foie supplémentaire.

Mettez dans une casserole la sauce bordelaise, chauffez et faites cuire dedans,
sans bouillir, la purée de foies de canards, goûtez, relevez avec un peu de
cayenne, au goût ; passez à l'étamine en pressant.

Servez le canard sur un plat et envoyez la sauce dans une saucière.

\section*{\centering Canard aux olives.}
\phantomsection
\addcontentsline{toc}{section}{ Canard aux olives.}
\index{Canard aux olives}
\index{Caneton rôti, aux olives}

Pour six personnes prenez :

\medskip

\footnotesize
\begin{longtable}{rrrp{16em}}
    500 & grammes & d’ & olives,                                                                          \\
    250 & grammes & de & madère,                                                                          \\
    100 & grammes & de & beurre,                                                                          \\
     20 & grammes & de & farine,                                                                          \\
        &         &  1 & beau canard nantais,                                                             \\
        &         &  1 & abatis de canard supplémentaire,                                                 \\
        &         &  1 & carotte,                                                                         \\
        &         &  1 & oignon,                                                                          \\
        &         &  1 & bouquet garni,                                                                   \\
        &         &    & sel et poivre.                                                                   \\
\end{longtable}
\normalsize

Troussez le canard.

Mettez à dessaler dans de l'eau les olives privées de leurs noyaux.

Préparez un bouillon à bouilli perdu avec les abatis, la carotte, l'oignon, le
bouquet garni, de l'eau, du sel et du poivre ; dégraissez-le, passez-le.

Faites revenir le canard à la casserole dans {\ppp40\mmm} grammes de beurre.

Faites un roux avec le reste du beurre et la farine, mouillez avec le bouillon
d'abatis et le madère, mettez le canard revenu et laissez braiser à petit feu
pendant une heure et demie.

Une demi-heure avant la fin de la cuisson, ajoutez les olives, goûtez et
complétez l'assaisonnement s'il y a lieu.

Dressez le canard sur un plat, les olives autour, masquez avec la sauce
dégraissée et passée ; puis servez.

Si l'on prend un caneton, il vaut mieux le faire rôtir ; la préparation de la
sauce et des olives reste la même.

Le canard, les olives et le madère vont très bien ensemble.

\section*{\centering Canard froid, en gelée.}
\phantomsection
\addcontentsline{toc}{section}{ Canard froid, en gelée.}
\index{Canard froid, en gelée}

Pour six personnes prenez :

\medskip

\footnotesize
\begin{longtable}{rrrp{16em}}
    200 & grammes & de & jambon,                                                                          \\
    200 & grammes & de & fond de veau,                                                                    \\
    200 & grammes & de & madère,                                                                          \\
     60 & grammes & de & beurre,                                                                          \\
     50 & grammes & de & glace de viande,                                                                 \\
        &         &  1 & canard,                                                                          \\
        &         &    & sel, poivre, quatre épices.                                                      \\
\end{longtable}
\normalsize

Coupez le jambon en petits morceaux.

Faites revenir dans le beurre le canard et le jambon ; puis mouillez avec le
fond de veau et le madère, ajoutez la glace de viande, du sel, du poivre et un
peu de quatre épices ; couvrez avec un papier beurré ; laissez cuire à tout
petit feu pendant deux heures.

Passez le jus de cuisson, dégraissez-le, concentrez-le, goûtez pour
l'assaisonnement, masquez le canard avec la sauce et laissez prendre en gelée.

\section*{\centering Canard rouennais\footnote{Le canard rouennais est le canard
                              sauvage à collet vert, domestiqué.} à la presse.}
\phantomsection
\addcontentsline{toc}{section}{ Canard rouennais à la presse.}
\index{Canard rouennais à la presse}

Cette recette, qui paraîtra un peu compliquée à priori, est en réalité facile
à exécuter ; elle ne demande que du soin. Pour la réussir, il est nécessaire et
suffisant de n'omettre aucun détail.

\medskip

Préparez d'abord le mélange suivant de plantes aromatiques :

\medskip

\label{pg0584} \hypertarget{p0584}{}
\footnotesize
\begin{longtable}{p{0.5em}p{16em}rc}
& Serpolet   \dotfill                                             &  32,50 & pour 100                     \\
& Thym       \dotfill                                             &  20,00 & —                            \\
& Sarriette  \dotfill                                             &  10,00 & —                            \\
& Basilic    \dotfill                                             &  10,00 & —                            \\
& Origan\footnote{Origanum vulgare.}
             \dotfill                                             &   5,00 & —                            \\
& Marjolaine\footnote{Origanum majorana ou majorana hortensis.}
             \dotfill                                             &   5,00 & —                            \\
& Romarin    \dotfill                                             &   5,00 & —                            \\
& Sauge      \dotfill                                             &   5,00 & —                            \\
& Hysope     \dotfill                                             &   5,00 & —                            \\
& Laurier    \dotfill                                             &   5,00 & —                            \\
&                                                                 &  \multicolumn{2}{r}{\hrulefill}       \\
&                                                                 & 100,00 & pour 100.                    \\
\end{longtable}
\normalsize

Ce mélange, constitué de façon qu'aucun parfum ne domine, chaque plante donnant
sa note particulière, produit un ensemble harmonieusement aromatique qui se
marie délicieusement avec la chair du canard. Il doit être aussi homogène que
possible.

Il est facile d'en préparer d'avance une certaine quantité en passant d'abord
séparément les divers éléments dans un moulin du type des petits moulins
à poivre du commerce, puis en mélangeant les poudres obtenues dans les
proportions indiquées et en broyant à nouveau l’ensemble au moulin pour
parfaire l'homogénéité.

\medskip

\index{Assaisonnement pour canard}
Préparez ensuite de même, au moulin, l’assaisonnement suivant :

\medskip

\label{pg0585} \hypertarget{p0585}{}
\footnotesize
\begin{longtable}{p{0.5em}p{16em}l}
& Sel blanc                       \dotfill                        &  5\textsuperscript{gr},00             \\
& Poivre ordinaire                \dotfill                        &  1\textsuperscript{gr},00             \\
& Poivre mignonette               \dotfill                        &  1\textsuperscript{gr},00             \\
& Mélange des plantes ci-dessus   \dotfill                        &  0\textsuperscript{gr},90             \\
& Quatre épices                   \dotfill                        &  0\textsuperscript{gr},50             \\
& Curry                           \dotfill                        &  0\textsuperscript{gr},50             \\
& Poivre de cayenne               \dotfill                        &  0\textsuperscript{gr},25             \\
& Muscade                         \dotfill                        &  0\textsuperscript{gr},25             \\
& Clous de girofle                \dotfill                        &  0\textsuperscript{gr},10             \\
&                                                                 &  \hrulefill                           \\
&                                                                 &  9\textsuperscript{gr},50             \\
\end{longtable}
\normalsize

Pour quatre personnes prenez alors :

\medskip

\footnotesize
\begin{longtable}{rrrp{16em}}
    200 & grammes                    & de & porto rouge,                                                  \\
     30 & grammes                    & de & fine champagne,                                               \\
        & 9\textsuperscript{gr},50   & d' & assaisonnement ci-dessus,                                     \\
        & 2\textsuperscript{gr},25   & de & poivre,                                                       \\
        & 1\textsuperscript{gr},50   & de & sel blanc,                                                    \\
        &                            &  2 & foies de canards supplémentaires,                             \\
        &                            &  1 & abatis de canard supplémentaire,                              \\
        &                            &  1 & beau canard rouennais vivant\footnote{Les accidents
                                                  qui ont été attribués au canard rouennais n'ont
                                                  jamais tenu qu'au manque de fraîcheur de l'animal
                                                  employé. Je ne connais personne qui ait éprouvé
                                                  le moindre inconvénient à la suite de l'ingestion
                                                  d'un canard étouffé au moment de le faire cuire.},      \\
        &                            &  1 & carotte,                                                      \\
        &                            &  1 & navet,                                                        \\
        &                            &  1 & poireau (le blanc seulement),                                 \\
        &                            &  1 & citron.                                                       \\
\end{longtable}
\normalsize

Étouffez le canard, plumez-le, flambez-le, réservez le foie, enlevez l'abatis.

Préparez {\ppp200\mmm} grammes de bouillon de canard concentré en faisant bouillir,
pendant quatre heures, les deux abatis de canards, la carotte, le navet, le
poireau dans un litre d'eau salée.

Pilez le foie du canard avec les deux foies de canards supplémentaires, ajoutez
le jus de la moitié du citron, le zeste finement émincé du quart du citron, 50
centigrammes d’assaisonnement, {\ppp50\mmm} centigrammes de sel blanc, {\ppp25\mmm} centigrammes de
poivre fraîchement moulu.

Mettez dans l'intérieur du canard le porto et {\ppp7\mmm} grammes d'assaisonnement,
laissez-le mariner pendant trois à quatre heures, puis faites-le rôtir à la
broche\footnote{Il est inutile d'arroser avec du beurre ; la graisse du canard
suffit pour la cuisson.}, tel quel, de manière que les aiguillettes restent un
peu saignantes et les pattes insuffisamment cuites ; un quart d'heure devant un
bon feu suffit pour assurer le résultat.

Découpez le canard, mettez de côté les pattes, flambez-les à la fine champagne.

Concentrez et dégraissez le jus de cuisson de la lèchefrite ; ajoutez-le au
bouillon de canard.

Incorporez aux foies pilés ce qui reste de la fine champagne qui a servi
à flamber les pattes, vous obtiendrez ainsi une purée aromatisée à souhait.

Arrivons maintenant à l'opération de la presse, qu'il convient d'exécuter devant
les convives, sur la table de la salle à manger.

Foncez un plat en nickel ou en argent avec la purée de foies, disposez dessus
les aiguillettes et les ailes ; mettez le plat sur un réchaud, devant la
presse ; mouillez avec le bouillon de canard, chauffez en agitant le plat pour
bien mélanger.

Introduisez la carcasse de l'animal dans la presse, serrez à fond, recueillez
dans le plat le jus qui sort, mélangez encore en chauffant toujours. Il faut
une dizaine de minutes pour amener la sauce à bonne consistance.

Pour terminer, moulinez sur les aiguillettes et sur les ailes {\ppp1\mmm} gramme de poivre
ordinaire et servez sur assiettes chaudes.

Cette partie du canard fournit un excellent salmis.

Pendant qu'on le savoure, ciselez les pattes, assaisonnez-les avec les
2 grammes d'assaisonnement qui restent, achevez leur cuisson sur le gril,
relevez-les avec {\ppp1\mmm} gramme de sel et {\ppp1\mmm} gramme de poivre et server-les, telles
quelles, avec une salade verte quelconque.

Cette partie du canard constitue un excellent rôti, d'un goût absolument
différent de celui obtenu avec l'autre partie.

Le canard à la presse est un très joli plat, qui suffit à lui seul pour rendre
intéressant un déjeuner d'amateurs.

\section*{\centering Canard rouennais farci, en cocote.}
\phantomsection
\addcontentsline{toc}{section}{ Canard rouennais farci, en cocote.}
\index{Canard rouennais farci, en cocote}
\index{Canard rouennais grillé}

Pour douze personnes prenez :

\footnotesize
\begin{longtable}{rrrrrp{18em}}
  & \hspace{2em} & 400 & grammes & de & porto rouge,                                                      \\
  & \hspace{2em} &  90 & grammes & de & fine champagne,                                                   \\
  & \hspace{2em} &  13 & grammes & de & sel blanc,                                                        \\
  & \hspace{2em} &   7 & grammes & de & poivre ordinaire fraîchement moulu,                               \\
  & \hspace{2em} &   2 & grammes & de & poivre mignonnette,                                               \\
  & \hspace{2em} &   2 & grammes & du & mélange de plantes aromatiques indiqué \hyperlink{p0584}{p. \pageref{pg0584}}, \\
  & \hspace{2em} &   1 & gramme  & de & quatre épices,                                                    \\
  & \multicolumn{3}{r}{50 centigrammes} & de & poivre de Cayenne,                                         \\
  & \multicolumn{3}{r}{50 centigrammes} & de & muscade,                                                   \\
  & \multicolumn{3}{r}{20 centigrammes} & de & clous de girofle,                                          \\
  & \hspace{2em} &     & 1 litre & de & fond de veau,                                                     \\
  & \hspace{2em} &     &         &  2 & canards rouennais,                                                \\
  & \hspace{2em} &     &         &  1 & foie gras de canard,                                              \\
  & \hspace{2em} &     &         &  1 & barde de lard,                                                    \\
  & \hspace{2em} &     &         &  1 & citron.                                                           \\
\end{longtable}
\normalsize

Plumez, flambez les canards ; enlevez les abatis, nettoyez-les ; réservez les
foies.

Désossez les deux canards, mais faites en sorte que, dans l'un, la peau
à laquelle toute la chair doit rester adhérente ne soit pas abîmée ;
réservez-le.

Faites cuire, à bouilli perdu, les abatis dans le fond de veau ; passez la
cuisson.

Passez à la presse les déchets des canards ; recueillez-en le jus.

\index{Farce pour canard rouennais}
Hachez la chair du canard non réservé, le foie gras et les deux autres foies,
mettez les deux tiers des condiments, des aromates, de la fine champagne et du
porto, ajoutez le jus de la presse et le jus du citron ; triturez bien de façon
à obtenir une farce homogène ; laissez-la reposer pendant {\ppp3\mmm}
à {\ppp4\mmm} heures.

Assaisonnez le canard réservé, dont la chair adhère à la peau, avec le reste du
sel, du poivre et des aromates et faites-le mariner pendant {\ppp3\mmm}
à {\ppp4\mmm} heures dans le reste du porto.

Étalez ensuite le canard sur la barde de lard, garnissez-le avec la farce,
roulez-le et ficelez-le. Mettez le canard ainsi apprêté dans une cocote en
porcelaine allant au feu, arrosez avec le reste de la fine champagne, mouillez
avec une partie du fond de veau et canard et faites cuire au four, au
bain-marie, en remplaçant le liquide au fur et à mesure de son évaporation. La
cuisson doit durer environ une heure et demie après que le bain-marie est entré
en ébullition. Dégraissez soigneusement et tenez au frais jusqu'au moment de
servir.

Ce canard farci, sans mélange vulgaire de chair à saucisses, est très prisé par
les amateurs.

\section*{\centering Salmis de canard rouennais.}
\phantomsection
\addcontentsline{toc}{section}{ Salmis de canard rouennais.}
\index{Salmis de canard rouennais}
\index{Canard rouennais en salmis}

Pour quatre personnes prenez :

\footnotesize
\begin{longtable}{rrrrrp{18em}}
  & \hspace{2em} & 200 & grammes & de & champagne,                                                        \\
  & \hspace{2em} &  30 & grammes & de & fine champagne,                                                   \\
  & \hspace{2em} &   7 & grammes & d' & assaisonnement, \hyperlink{p0585}{p. \pageref{pg0585}},           \\
  & \hspace{2em} &   1 & gramme  & de & poivre fraîchement moulu,                                         \\
  & \multicolumn{3}{r}{50 centigrammes} & de & paprika,                                                   \\
  & \hspace{2em} &     &         &  1 & beau canard rouennais,                                            \\
  & \hspace{2em} &     &         &  1 & foie gras de canard,                                              \\
  & \hspace{2em} &     &         &  1 & abatis de canard supplémentaire,                                  \\
  & \hspace{2em} &     &         &  1 & carotte,                                                          \\
  & \hspace{2em} &     &         &  1 & navet,                                                            \\
  & \hspace{2em} &     &         &  1 & poireau (le blanc seulement),                                     \\
  & \hspace{2em} &     &         &    & sel.                                                              \\
\end{longtable}
\normalsize

Troussez le canard ; mettez de côté l'abatis.

Préparez {\ppp200\mmm} grammes de fond de canard avec les abatis, la carotte, le navet, le
poireau, de l'eau et du sel ; passez-le.

Pilez le foie gras de canard avec {\ppp50\mmm} centigrammes d'assaisonnement, 50
centigrammes de sel blanc et {\ppp30\mmm} centigrammes de paprika.

Mettez dans l'intérieur du canard la farce ci-dessus, le reste de
l’assaisonnement, le reste du paprika, le champagne ; laissez en contact
pendant {\ppp2\mmm} à {\ppp3\mmm} heures ; puis faites-le rôtir incomplètement à la broche.
Découpez-le ; réservez les pattes, les ailes et les aiguillettes ; passez le
reste à la presse, recueillez le jus.

Dégraissez le jus de cuisson de la lèchefrite, concentrez-le.

Mettez dans une casserole pattes, ailes et aiguillettes ; flambez-les avec la
fine champagne, mouillez avec le fond de canard, la cuisson concentrée et le
jus de la presse ; assaisonnez avec le poivre. Achevez la cuisson.

Au dernier moment, goûtez et ajoutez sel et paprika, s'il est nécessaire,

Dressez le salmis sur un plat et servez en envoyant en même temps des pommes
de terre Chip et des tranches d'oranges pelées, épépinées et macérées dans du
curaçao.

C'est un plat qui sort de la banalité.

\sk

Comme variante, on pourra remplacer le champagne par du madère et servir le
salmis avec une garniture de morilles farcies, au jus.

\section*{\centering Civet de canard.}
\phantomsection
\addcontentsline{toc}{section}{ Civet de canard.}
\index{Civet de canard}
\index{Canard an civet}

\label{pg0588} \hypertarget{p0588}{}

Pour huit personnes prenez :

\footnotesize
\begin{longtable}{rrrrrp{16em}}
  & \hspace{2em} & 250 & grammes & de & bouillon,                                                         \\
  & \hspace{2em} & 250 & grammes & de & poitrine de porc,                                                 \\
  & \hspace{2em} & 250 & grammes & de & champignons de couche,                                            \\
  & \hspace{2em} & 100 & grammes & de & sang de lapin,                                                    \\
  & \hspace{2em} &  60 & grammes & de & beurre,                                                           \\
  & \hspace{2em} &  50 & grammes & de & fine champagne,                                                   \\
  & \hspace{2em} &  20 & grammes & de & farine,                                                           \\
  & \hspace{2em} &   5 & grammes & de & \hangindent=1em sel gris\footnote{Quantité approximative,
                                        dépendant du bouillon et de la poitrine de porc employés,
                                        qui peuvent être plus ou moins salés.},                           \\
  & \hspace{2em} &   1 & gramme  & d' & \hangindent=1em un mélange de serpolet, origan, sarriette et
                                        hysope en poudre, en parties égales,                              \\
  & \multicolumn{3}{r}{50 centigrammes} & de & poivre fraîchement moulu,                                  \\
  & \multicolumn{3}{r}{ 2 centigrammes} & de & quatre épices,                                             \\
  & \hspace{2em} &     & 1 litre & de & bon vin rouge,                                                    \\
  & \hspace{2em} &     &         & 20 & petits oignons,                                                   \\
  & \hspace{2em} &     &         &  4 & échalotes,                                                        \\
  & \hspace{2em} &     &         &  3 & foies de canard,                                                  \\
  & \hspace{2em} &     &         &  1 & canard rouennais fraîchement étouffé,                             \\
  & \hspace{2em} &     &         &  1 & bouquet garni.                                                    \\
\end{longtable}
\normalsize

Découpez le canard ; recueillez le sang.

Coupez la poitrine de porc en morceaux.

Pilez le foie du canard et les trois foies supplémentaires avec le sang du
canard et le sang de lapin.

Faites revenir les morceaux de canard et de poitrine de porc dans une casserole
avec le beurre. Lorsque la viande est dorée, retirez-la, remplacez-la par les
oignons auxquels vous ferez prendre couleur, retirez-les aussi, puis faites
roussir la farine, remettez ensuite le canard, le porc, flambez avec la fine
champagne, mouillez avec le bouillon et le vin, ajoutez les oignons, le
bouquet, les échalotes, les aromates, les épices, le sel, le poivre et laissez
cuire à petit feu pendant deux heures au moins. Passez alors la sauce,
dégraissez-la à fond, mettez les champignons pelés et continuez la cuisson
pendant une demi-heure. Liez la sauce avec les foies pilés, chauffez pendant
quelques instants sans laisser bouillir, dressez et servez.

Ce plat est absolument remarquable. En été, quand le gibier fait défaut, il est
une véritable consolation pour l'amateur. Délicieux le jour même, il est
peut-être encore meilleur réchauffé le lendemain au bain-marie.

\sk

\index{Civet de poulet}
\index{Civet d'oie sauvage}
\index{Civet de pigeons}
On peut préparer d'une façon analogue des civets d'oie sauvage, de pigeons
ou de poulet ; ce dernier porte aussi le nom de « poulet au sang ».

\section*{\centering Croustade de canard au foie gras.}
\phantomsection
\addcontentsline{toc}{section}{ Croustade de canard au foie gras.}
\index{Croustade de canard au foie gras}
\index{Canard en croustades}

Pour douze personnes prenez :

\medskip

1° pour la croûte :

\medskip

\footnotesize
\begin{longtable}{rrrp{16em}}
    600 & grammes & de & farine,                                                                          \\
    200 & grammes & de & beurre,                                                                          \\
    100 & grammes & d' & eau,                                                                             \\
     20 & grammes & de & sel,                                                                             \\
     12 & grammes & d' & huile d'olive,                                                                   \\
        &         &  4 & jaunes d'œufs frais ;                                                            \\
\end{longtable}
\normalsize

\index{Garniture pour croustade de canard}
2° pour la garniture :

\medskip

\footnotesize
\begin{longtable}{rrrp{16em}}
    750 & grammes & de & fond de veau,                                                                    \\
    150 & grammes & de & madère,                                                                          \\
     30 & grammes & de & cognac,                                                                          \\
        &         &  2 & canards rouennais moyens, mais bien en chair,                                    \\
        &         &  1 & beau foie gras d'oie pesant 1 kilogramme environ,                                \\
        &         &    & truffes à volonté,                                                               \\
        &         &    & épices fines,                                                                    \\
        &         &    & sel, poivre, cayenne ;                                                           \\
\end{longtable}
\normalsize

3° pour la gelée :

\medskip

\footnotesize
\begin{longtable}{rrrp{16em}}
    750 & grammes & de & gîte de bœuf,                                                                    \\
    750 & grammes & de & jarret de veau,                                                                  \\
        &         &  1 & beau pied de veau,                                                               \\
        &         &    & légumes,                                                                         \\
        &         &    & sel et poivre.                                                                   \\
\end{longtable}
\normalsize

Parez les canards, enlevez les foies, réservez les abatis.

Avec les éléments du premier paragraphe, préparez la pâte de la croustade, qui
doit être bien lisse ; laissez-la reposer pendant quelques heures, puis, avec
les deux tiers de cette pâte, faites une abaisse rectangulaire de {\ppp5\mmm} millimètres
d'épaisseur environ : ce sera le fond de la croustade. Avec le tiers restant,
faites une autre abaisse de {\ppp2\mmm} millimètres d'épaisseur, taillez dedans quatre
bandes de {\ppp4\mmm} centimètres de largeur, dont deux auront la longueur du grand côté
du fond et les deux autres celle du petit côté. Collez ces quatre bandes sur
les bords du fond en les mouillant légèrement avec un peu d’eau ; emplissez la
croustade de cailloux lavés et cuisez-la au four.

Faites mariner, pendant quelques heures, le foie gras et les truffes dans le
madère ; assaisonnez avec sel, poivre et épices ; puis mettez le tout dans une
casserole avec le fond de veau et le cognac ; amenez à ébullition et laissez
pocher ensuite pendant une demi-heure au bain-marie, au four, de manière que le
foie gras reste rose à l'intérieur.

Escalopez un cinquième environ du foie gras en petites tranches minces ;
réservez le reste ; concentrez un peu la cuisson.

Hachez quelques truffes ; coupez les autres en rondelles.

Faites rôtir les deux canards de manière que les filets restent un peu
saignants ; enlevez la peau ; escalopez les filets.

Dans un peu de graisse chaude de la cuisson des canards, donnez de la fermeté
aux foies des canards, tout en les conservant roses.

Désossez le reste des canards, passez-en la chair au tamis de crin avec les
foies de canards et le foie gras réservé ; ajoutez le hachis de truffes et plus
ou moins de la cuisson concentrée du foie gras et des truffes ; achevez
l'assaisonnement de l'ensemble avec sel, poivre et cayenne, au goût.

Garnissez le fond de la croustade avec cette farce, placez dessus les escalopes
de canard, en interposant entre elles des escalopes de foie gras et des
rondelles de truffe.

En même temps, préparez la gelée en faisant cuire, à bouilli perdu, dans de
l’eau le gîte de bœuf, le jarret et le pied de veau, les abatis et les déchets
des canards, des légumes ; assaisonnez au goût ; concentrez le jus obtenu ;
ajoutez le reste de la cuisson du foie gras et des truffes ; passez, clarifiez
et laissez refroidir un peu.

Masquez le dessus de la croustade avec cette gelée encore liquide, puis mettez
à la glacière pour faire prendre.

Servez avec une salade verte.

Cette croustade, très appétissante, remplace avantageusement les pâtés de foie
gras ordinaires.

\section*{\centering Soufflé de canard, à la purée de marrons.}
\phantomsection
\addcontentsline{toc}{section}{ Soufflé de canard, à la purée de marrons.}
\index{Soufflé de canard, à la purée de marrons}

Pour six personnes prenez :

\medskip

\footnotesize
\begin{longtable}{rrrp{16em}}
    200 & grammes & de & crème épaisse,                                                                   \\
    150 & grammes & de & madère,                                                                          \\
    125 & grammes & de & beurre,                                                                          \\
     20 & grammes & de & farine,                                                                          \\
        & 1 litre & de & marrons,                                                                         \\
        &         & 12 & olives dessalées et privées de leurs noyaux,                                     \\
        &         &  2 & œufs frais,                                                                      \\
        &         &  1 & canard,                                                                          \\
        &         &    & foie gras cuit au naturel,                                                       \\
        &         &    & légumes de pot-au-feu,                                                           \\
        &         &    & lait ou eau,                                                                     \\
        &         &    & sel, poivre, épices.                                                             \\
\end{longtable}
\normalsize

Cassez les œufs, séparez les blancs des jaunes ; battez les blancs en neige.

Enlevez {\ppp650\mmm} grammes de chair tendre du canard sans peau ni nerfs,
réservez-la.

Préparez un bouillon à bouilli perdu avec l'abatis et les restes du canard, des
légumes de pot-au-feu, de l’eau, du sel, du poivre ; passez le bouillon,
dégraissez-le, concentrez-le au volume d’un demi-litre environ.

Faites un roux avec {\ppp30\mmm} grammes de beurre et la farine, mouillez
avec le bouillon de canard et le madère, ajoutez les olives ; laissez cuire,
puis passez le tout au tamis. Remettez la sauce sur le feu, dépouillez-la,
amenez-la à être d'une consistance sirupeuse.

En même temps, préparez la purée de marrons. Échaudez les marrons, épluchez-les
et faites-les cuire, avec {\ppp10\mmm} grammes de sel, pendant une demi-heure
environ, dans de l’eau ou dans du lait, au goût, en quantité juste suffisante
pour obtenir une purée serrée ; passez-les au tamis, incorporez-y {\ppp125\mmm}
grammes de crème et {\ppp75\mmm} grammes de beurre. Versez la purée dans un
moule annulaire que vous tiendrez au chaud, au bain-marie.

Pilez au mortier la chair de canard réservée, assaisonnez-la avec sel, poivre
et épices, au goût ; passez-la au tamis, puis ajoutez-y par petites quantités
le reste de la crème, les deux jaunes d'œufs et les deux blancs battus en
neige, de manière à avoir un appareil moelleux et léger.

Beurrez avec le reste du beurre un moule à charlotte, mettez dedans l'appareil
et faites cuire au four, au bain-marie, pendant une vingtaine de minutes.

Démoulez le turban de marrons sur un plat, décorez-le avec des médaillons de
foie gras, disposez au milieu le soufflé de canard et servez, en envoyant en
même temps la sauce dans une saucière.

\section*{\centering Foie gras de canard au naturel.}
\phantomsection
\addcontentsline{toc}{section}{ Foie gras de canard au naturel.}
\label{pg0592} \hypertarget{p0592}{}
\index{Foie gras de canard au naturel}
\index{Canard (Foie gras de) au naturel}

Prenez un beau foie gras de canard\footnote{ Les meilleurs foies gras de canard
proviennent du canard malard, résultat du croisement du canard ordinaire et du
canard d'Inde.}, comme il y en a à l'entrée de l'hiver, au moment des premiers
froids.

Après l'avoir légèrement salé, mettez-le dans une sauteuse en cuivre étamé ou
en bi-métal, enduite au préalable de quelques gouttes d'huile d'olive fine que
vous aurez fait chauffer un peu.

Placez alors la sauteuse sur un feu vif et secouez-la constamment de façon à
éviter que le foie s'attache au fond.

Au bout de douze à quinze minutes, le foie aura pris une couleur légèrement
dorée ; il sera à point.

Servez aussitôt sur assiettes chaudes.

On mange le foie gras de canard au naturel soit tel que, soit arrosé d'un peu
de jus de citron ; c'est ainsi que je le préfère.

Préparé de la sorte, le foie gras de canard est un mets digne des dieux. Il se
présente sous la forme d'une gelée rose, parfumée, appétissante ; il charme à la
fois la vue, l'odorat, le goût, et il force la reconnaissance des estomacs les plus
ingrats par l’aimable digestion qu'il procure.

\section*{\centering Dinde truffée rôtie.}
\phantomsection
\addcontentsline{toc}{section}{ Dinde truffée rôtie.}
\index{Dinde truffée rôtie}

La dinde truffée figure fréquemment sur les menus classiques. Voici une
excellente façon de la préparer.

\medskip

Pour douze personnes prenez :

\footnotesize
\begin{longtable}{rrrp{20em}}
        &         &  1 & \hangindent=1em bonne dinde tendre, bien en chair, sans l'abatis,                \\
        &         &  1 & beau foie gras d'oie,                                                            \\
        &         &  1 & fine barde de lard,                                                              \\
        &         &    & \hangindent=1em truffes noires du Périgord à volonté (au moins 1 500 grammes),   \\
        &         &    & vin de Madère,                                                                   \\
        &         &    & sel et poivre.                                                                   \\
\end{longtable}
\normalsize

Brossez soigneusement les truffes, pelez-les, faites-les cuire dans du madère ;
réservez les pelures et le madère de cuisson.

Coupez une partie des truffes en rondelles minces, insérez-les sous la peau de
l'animal, mettez dans l'intérieur du sel, du poivre et le reste des truffes ;
laissez la dinde se parfumer pendant trois jours.

Pétrissez le foie gras avec les pelures de truffes ; mettez-le à mariner
pendant {\ppp24\mmm} heures dans le madère de cuisson réservé.

Retirez les truffes de l'intérieur de la dinde, mélangez-les avec le foie gras
mariné, farcissez la bête avec le tout, bardez-la et faites-la rôtir à la
broche.

Servez la dinde sur un plat et le jus de cuisson, dégraissé, dans une saucière.

\sk

\index{Poularde truffée rôtie}
\index{Perdreau truffé rôti}
On prépare de même la poularde truffée et le perdreau truffé.

\section*{\centering Dinde farcie rôtie, à la purée de reinettes.}
\phantomsection
\addcontentsline{toc}{section}{ Dinde farcie rôtie, à la purée de reinettes.}
\index{Dinde farcie rôtie, à la purée de reinettes}

L'originalité du plat consiste dans sa garniture.

\smallskip

Pour douze personnes prenez :

\smallskip

\footnotesize
\begin{longtable}{rrrrp{18em}}
 &  \multicolumn{2}{r}{2 kilogrammes} & de & pommes reinettes,                                            \\
 &  500 & grammes & de & truffes,                                                                         \\
 &  150 & grammes & de & beurre,                                                                          \\
 &      & 1 litre & de & marrons grillés,                                                                 \\
 &      &         &  1 & dinde tendre, bien en chair,                                                     \\
 &      &         &  1 & beau foie gras d'oie,                                                            \\
 &      &         &  1 & barde de lard,                                                                   \\
 &      &         &    & madère,                                                                          \\
 &      &         &    & sel et poivre.                                                                   \\
\end{longtable}
\normalsize

Brossez, pelez les truffes, faites-les cuire dans du madère ; réservez les
pelures et le vin de cuisson.

Insérez sous la peau de la dinde des rondelles de truffe minces ; mettez le
reste dans l'intérieur de l'animal.

Pétrissez le foie gras avec les pelures des truffes ; mettez-le à mariner
pendant {\ppp24\mmm} heures dans le madère de cuisson des truffes.

\index{Farce pour perdreau}
\index{Farce pour dinde}
Amalgamez ensemble foie gras, madère de la marinade, le reste des truffes, les
marrons, du sel, du poivre et farcissez-en la dinde. Bardez-la et faites-la
rôtir à la broche.

En même temps, préparez une purée non sucrée avec les pommes reinettes.
Pelez-les, coupez-les en morceaux, mettez-les dans une casserole avec 50
grammes de beurre, couvrez et laissez fondre. Passez au tamis, remettez sur le
feu et fouettez en ajoutant le reste du beurre par petits morceaux.

Servez la dinde sur un plat, le jus de cuisson bien dégraissé dans une saucière
et la purée dans un légumier.

Certaines personnes préfèrent la purée froide à la purée chaude.

La dinde farcie rôtie, à la purée de reinettes est un excellent plat qui convient
pour les repas sans grande cérémonie.

\section*{\centering Oie aux marrons.}
\phantomsection
\addcontentsline{toc}{section}{ Oie aux marrons.}
\index{Oie aux marrons}
\index{Farce pour oie}

L'oie\footnote{« L'oie est un animal stupide ; pour un, c'est un peu trop ;
pour deux, ce n'est vraiment pas assez », à dit un philosophe pessimiste qui ne
devait pas être dyspeptique. Mon oie aux marrons échappe à cette critique
sévère ; servie comme plat de résistance, elle est capable de satisfaire quatre
amateurs sérieux et, si le menu comprend plusieurs plats, elle peut suffire
pour huit et même pour dix convives.}, réservée pour les repas intimes, est
farcie le plus souvent avec des marrons englobés ou non dans un hachis de veau
et de porc. Je trouve préférable d'enrober les marrons dans du foie gras pétri
avec du madère.

La cuisson se fait à la broche, sans barder l'oie.

Comme garniture, je recommande la purée de reinettes qui fait aussi bon
ménage avec la chair de l'oie qu'avec celle de la dinde.

\section*{\centering Foie gras d'oie au naturel.}
\phantomsection
\addcontentsline{toc}{section}{ Foie gras d'oie au naturel.}
\index{Foie gras d'oie au naturel}

Prenez un beau foie gras d'oie\footnote{Les meilleurs foies gras d'oie
proviennent de l'espèce des grandes oies grises à dessous blanc qui sont
élevées et viennent particulièrement bien en France sur les terrains calcaires
de la Gascogne, de la Guyenne, du Languedoc, et en Alsace.}, de {\ppp700\mmm} à 900
grammes, blanc rosé et très ferme ; nettoyez-le comme il faut, enlevez le fiel,
la partie verte sur laquelle il repose et l'arrivée du sang ; puis faites-le
dégorger pendant deux heures dans de l'eau froide additionnée de sel gris.
Retirez-le ensuite, essuyez-le délicatement, mettez-le dans un plat en
porcelaine allant au feu et aux trois quarts plein de graisse d'oie fondue (500
grammes environ), couvrez d'un papier beurré et faites cuire au four doux.
Arrosez toutes les deux ou trois minutes. Après sept minutes de cuisson, salez
de tous les côtés avec du sel blanc. La cuisson dure en moyenne un quart
d'heure. Laissez refroidir le foie dans sa cuisson, puis mettez-le sur un autre
plat, faites fondre légèrement une partie de la graisse dans laquelle il a cuit,
versez-la au travers d'une passoire autour du foie et laissez refroidir
à nouveau.

Cuit à point, le foie doit être rosé à l'intérieur. Il peut se conserver plusieurs
jours.

\index{Foie gras d'oie en gelée}
On pourrait remplacer la graisse par une gelée, mais la gelée a l'inconvénient
de ne pas se conserver. Les foies en gelée devront être servis le jour même de
leur préparation.

\section*{\centering Foie gras d'oie en brioche.}
\phantomsection
\addcontentsline{toc}{section}{ Foie gras d'oie en brioche.}
\index{Foie gras d'oie en brioche}

Prenez un beau foie gras d'oie de Strasbourg, assaisonnez-le avec sel et
épices ; mettez-le à mariner pendant {\ppp24\mmm} heures dans du madère.

Préparez une pâte comme il est dit \hyperlink{p0371}{p. \pageref{pg0371}} ; et
confectionnez une brioche.

Enlevez le foie de la marinade ; faites-le cuire dans de la graisse d'oie
fraîche, au bain-marie, au four, pendant un quart d'heure.

En même temps, faites cuire au four la brioche ; à moitié cuisson, coupez-la
à mi-hauteur, évidez un peu les deux morceaux et insérez dedans le foie tout
chaud, bien égoutté, en le comprimant pour éviter les vides. Reconstituez la
brioche, soudez le joint\footnote{Quand le joint est bien fait, il est
impossible de voir par où le foie à été introduit et la genèse du plat est une
énigme pour ceux qui ne sont pas initiés au « modus operandi ».} avec de la
pâte, dorez à l'œuf et achevez la cuisson au four.

Démoulez ; laissez refroidir.

Pour servir, découpez de manière que chaque morceau comporte une tranche
de foie enrobé dans de la brioche.

\section*{\centering Foie gras d’oie truffé en aspic.}
\phantomsection
\addcontentsline{toc}{section}{ Foie gras d’oie truffé en aspic.}
\index{Foie gras d’oie truffé en aspic}
\label{pg0596} \hypertarget{p0596}{}

Préparez un fond de veau bien clarifié avec pied, jarret, couenne, légumes,
bouquet garni, eau, sel et poivre.

Prenez un beau foie gras d’oie, nettoyez-le soigneusement, frottez-en la
surface d'un peu de quatre épices, au goût, piquez-le ensuite avec des morceaux
de truffe également frottés de quatre épices, et enveloppez-le d'une toilette
de porc.

Mettez-le, ainsi apprêté, dans une marmite en porcelaine allant au feu,
mouillez avec une bouteille de vin de Champagne demi-sec de qualité moyenne et
faites cuire à feu doux, pendant {\ppp20\mmm} ou {\ppp25\mmm} minutes, pas davantage, en arrosant
constamment.

Lorsque le foie est cuit, retirez-le de son enveloppe, mettez-le sur un plat,
mélangez jus de cuisson et fond de veau, réduisez, dégraissez, clarifiez,
versez sur le foie et laissez prendre en gelée.

On peut également préparer l'aspic dans un moule, mais le démoulage présente
parfois des difficultés ; de plus, l'emploi du moule, qui pousse à comprimer le
foie dans le but de donner à l’aspic une forme plus régulière, a l'inconvénient
de lui enlever un peu de son moelleux, qui est précisément l'une des
caractéristiques les plus remarquables du plat préparé sans moule, comme je
l'indique.

\section*{\centering Foie gras d’oie truffé en cocote.}
\phantomsection
\addcontentsline{toc}{section}{ Foie gras d’oie truffé en cocote.}
\index{Foie gras d’oie truffé en cocote}

Pour six personnes prenez :

\medskip

\footnotesize
\begin{longtable}{rrrp{16em}}
    150 & grammes & de & légumes de pot-au-feu : carotte, navet, panais, etc.,                            \\
    100 & grammes & de & porto blanc,                                                                     \\
    100 & grammes & de & chablis,                                                                         \\
     50 & grammes & de & beurre,                                                                          \\
     30 & grammes & de & cognac,                                                                          \\
     30 & grammes & de & vieille fine champagne,                                                          \\
     30 & grammes & de & glace de viande fine,                                                            \\
        & 1 litre & de & fond de veau,                                                                    \\
        &         &  2 & abatis de poulets,                                                               \\
        &         &  1 & beau foie gras d'oie,                                                            \\
        &         &  1 & fine barde de lard,                                                              \\
        &         &    & madère,                                                                          \\
        &         &    & truffes.                                                                         \\
        &         &    & quatre épices,                                                                   \\
        &         &    & sel et poivre.                                                                   \\
\end{longtable}
\normalsize

L'avant-veille du jour où vous voudrez servir ce mets, ouvrez le foie,
dénervez-le, insérez dedans des truffes à volonté, salez, poivrez légèrement et
mettez le tout à mariner pendant douze heures dans du madère.

Le lendemain, faites revenir dans le beurre les légumes et les abatis ; laissez
pincer un peu les légumes pour donner du goût, mouillez avec le fond de veau,
le porto, le chablis et le cognac, ajoutez la glace de viande, un peu de quatre
épices et chauffez à tout petit feu, en dépouillant fréquemment le jus pendant
la cuisson ; réduisez-le au volume d'un litre ; passez-le.

Retirez le foie de la marinade, enveloppez-le dans la barde ; mettez-le dans
une cocote en porcelaine, mouillez d’abord avec la fine champagne, ensuite avec
le jus passé, puis placez la cocote dans un large bain-marie froid. Portez au
four doux, amenez lentement à ébullition, ce qui demande une demi-heure
environ, et continuez la cuisson pendant une demi-heure encore.

Retirez alors la cocote du bain-marie, placez sur le liquide une petite
planchette surmontée d'un poids, de façon à la faire plonger de quelques
millimètres : la graisse montera à la surface au-dessus de la planchette, ce
qui facilitera le dégraissage.

Laissez refroidir et tenez au frais jusqu'au moment de servir.

Le foie gras en cocote, triomphe du grand et modeste artiste Charles Blau, est
une pure merveille : c'est une crème fondante enrobée dans une gelée veloutée,
aromatisée d'une façon exquise.

\section*{\centering Foie gras d'oie truffé, en crépinettes.}
\phantomsection
\addcontentsline{toc}{section}{ Foie gras d'oie truffé, en crépinettes.}
\index{Foie gras d'oie truffé, en crépinettes}
\index{Crépinettes de foie gras d'oie}

Pour six personnes prenez :

\medskip

\footnotesize
\begin{longtable}{rrrp{16em}}
    500 & grammes & de & foie gras d'oie paré et coupé en douze petites tranches semblables,              \\
    150 & grammes & de & ris de veau blanchi,                                                             \\
    125 & grammes & de & champignons de couche,                                                           \\
    100 & grammes & de & moelle de bœuf blanchie,                                                         \\
    100 & grammes & de & jambon non fumé, cuit,                                                           \\
     15 & grammes & de & beurre,                                                                          \\
     10 & grammes & de & ciboule,                                                                         \\
     10 & grammes & de & sel blanc,                                                                       \\
      2 & grammes & de & poivre,                                                                          \\
      2 & grammes & de & persil,                                                                          \\
    1/2 & gramme  & de & quatre épices,                                                                   \\
        &         &  2 & jaunes d'œufs frais,                                                             \\
        &         &    & truffes à volonté, cuites dans du madère,                                        \\
        &         &    & crépine de porc,                                                                 \\
        &         &    & jus de citron,                                                                   \\
        &         &    & jus d'orange.                                                                    \\
\end{longtable}
\normalsize

Pelez les champignons, passez-les dans du jus de citron et faites-les cuire
dans le beurre.

Hachez ensemble ris de veau, jambon, moelle, truffes, champignons, ciboule,
persil, assaisonnez avec le sel, le poivre, les quatre épices et liez le hachis
avec les jaunes d'œufs.

Mettez une couche de cette farce sur chaque côté des tranches de foie gras et
enveloppez isolément chaque tranche, ainsi apprêtée, dans de la crépine de
porc.

Faites griller les crépinettes comme des saucisses, enlevez l'excès de graisse
avec du papier buvard et servez sur assiettes chaudes, en envoyant en même
temps, dans une saucière, du jus d'orange relevé par du jus de citron.

\section*{\centering Ballotine\footnote{\index{Ballotines (Définitions des)}
                                        \index{Définitions des ballotines}
                                        \index{Agneau en ballotine}
                                        \index{Ballotines d'agneau}
                                        \index{Ballotines de gibier}
                                        \index{Ballotines de porc}
                                        \index{Ballotines de viande de boucherie}
                                        \index{Ballotines de volaille}
Les ballotines sont des préparations farcies, généralement servies froides,
auxquelles on donne la forme de petits ballots. Elles peuvent être constituées
par une seule pièce, telles les ballotines d'agneau ou de toute autre viande de
boucherie et de porc, ou par la réunion de plusieurs petites galantines, telles
les ballotines de volaille et de gibier. On les enveloppe de sauce chaud-froid
ou de gelée ; on les décore et on les garnit à volonté.} de foie gras d'oie
truffé.}
\phantomsection
\addcontentsline{toc}{section}{ Ballotine de foie gras d'oie truffé.}
\index{Ballotine de foie gras d'oie truffé}

Pour douze personnes prenez :

\medskip

1° pour le corps de la ballotine :

\footnotesize
\begin{longtable}{rrrrp{18em}}
  & 500 & grammes & de & truffes noires du Périgord, pelées,                                              \\
  & 500 & grammes & de & filet de porc frais, haché fin,                                                  \\
  & 300 & grammes & de & madère,                                                                          \\
  &  15 & grammes & de & fine champagne,                                                                  \\
  &   5 & grammes & d' & épices (de préférence le mélange Cieux),                                         \\
  &     &         &  1 & beau foie gras d'oie,                                                            \\
  &     &         &    & bardes fines de lard ;                                                           \\
\end{longtable}
\normalsize


2° pour la gelée :

\footnotesize
\begin{longtable}{rrrrp{18em}}
  2 &  \multicolumn{2}{r}{kilogrammes} & de & jarret de veau haché fin,                                   \\
  & 250 & grammes  & de & porto blanc,                                                                    \\
  & 100 & grammes  & de & glace de viande fine,                                                           \\
  & 100 & grammes  & de & beurre,                                                                         \\
  &  35 & grammes  & de & fine champagne,                                                                 \\
  &     & 4 litres & de & très bon consommé.                                                              \\
  &     &          &  2 & gros oignons,                                                                   \\
  &     &          &  2 & belles carottes,                                                                \\
  &     &          &  1 & bouquet garni.                                                                  \\
\end{longtable}
\normalsize

\textit{Préparation de la gelée}. — La veille du jour où vous voudrez présenter
cette ballotine, mettez dans une marmite le beurre, les oignons et les carottes
coupés en rondelles minces ; laissez-les dorer et s'attacher un peu au fond de
la marmite, ajoutez ensuite le jarret de veau que vous ferez revenir et le
bouquet garni ; mouillez avec le porto, la fine champagne et le consommé dans
lequel vous aurez fait dissoudre la glace de viande ; laissez cuire pendant
quatre heures. Passez ensuite la cuisson et concentrez-la au volume d'un litre
et demi environ. Éloignez alors la marmite sur le coin du fourneau ; laissez
mijoter, dépouillez de quart d'heure en quart d'heure jusqu'à obtention d'un
litre environ de jus partaitement dépouillé. Filtrez-le doucement au travers
d’un linge fin.

Pendant ce temps, ouvrez le foie, dénervez-le, insérez dedans les truffes,
saupoudrez avec la moitié des épices ; mettez le reste dans le hachis de filet
de porc.

Disposez dans un plat le foie et le hachis de porc ainsi assaisonnés, versez
dessus le madère ; laissez en contact pendant toute la nuit.

\medskip

\textit{Préparation de la ballotine}. — Le lendemain, retirez le foie et le
hachis de porc du madère, enrobez le foie dans le hachis, entourez le tout avec
des bardes de lard coupées en bandes ; ficelez la ballotine. Mettez-la dans une
cocote en porcelaine allant au feu, arrosez-la avec la fine champagne, mouillez
avec les trois quarts de la gelée que vous aurez fait fondre et le madère de la
marinade. Placez la cocote dans un grand bain-marie froid, poussez au four et
laissez cuire jusqu'au moment où une aiguille à brider, enfoncée de part en
part dans le foie, soit, à sa sortie, à une température telle que son contact
avec la langue ne puisse être supporté.

\medskip

\textit{Finissage}. — Retirez la cocote du four, laissez-la refroidir pendant
une demi-heure environ. Sortez la ballotine de la cocote, enveloppez-la en la
serrant dans un linge que vous coudrez, puis posez sur la ballotine une
planchette surmontée d'un poids d'un kilogramme et continuez à faire refroidir
sous pression pendant une heure. Dégraissez le jus, clarifiez-le.

\medskip

\textit{Dressage}. — Dressez la ballotine sur un plat, après l'avoir
débarrassée des ficelles, masquez-la et décorez-la avec la gelée. Tenez la
ballotine au frais jusqu'au moment de servir.

Cette ballotine est incomparable : il est difficile de concevoir quelque chose
de plus parfait.

\sk

On peut préparer aussi une excellente ballotine avec du bon foie gras et de
bonnes truffes de conserve.

Voici comment il faudra procéder pour commencer. Plongez la boîte de foie
pendant un instant dans de l’eau chaude, coupez le couvercle aussi près que
possible du bord et retournez-la sur un plat de façon à retirer le foie entier
de la boîte ; dégraissez-le complètement. Sortez les truffes de leur boîte et
raffermissez-les en les aspergeant de madère et de fine champagne.

Tout le reste de la préparation est le même, et le point de cuisson sera
déterminé, comme précédemment, par le criterium de l'aiguille : il sera
naturellement atteint plus vite que lorsqu'on emploie du foie frais, puisque le
foie de conserve a déjà subi une cuisson lors de sa mise en boîte.

\sk

Une jolie façon de présenter celle ballotine est de l'accompagner d'une salade
de chou-palmiste assaisonnée avec une mayonnaise au citron.

Comme boisson, un bourgogne de derrière les fagots, tel un vieux musigny,
s'impose.

\section*{\centering Pâté de conserve de foie gras d’oie truffé.}
\phantomsection
\addcontentsline{toc}{section}{ Pâté de conserve de foie gras d’oie truffé.}
\index{Pâté de conserve de foie gras d’oie truffé}
\index{Conserve de foie gras d'oie truffé}
\index{Foie gras d'oie truffé, en conserve}


Prenez de beaux foies gras d'oie bien sains et coupez-les en morceaux assez
gros.

Prenez de belles truffes noires du Périgord (la proportion dépend du goût et de
la bourse), pelez-les, passez-les légèrement à la poêle, sur un feu assez vif,
dans de la graisse d'oie bouillante, puis coupez-les en morceaux de la grosseur
d'une noisette.

Mélangez foie et truffes, assaisonnez avec {\ppp2\mmm} grammes de poivre fraîchement
moulu et {\ppp18\mmm} grammes de sel blanc par kilogramme de foie, puis
emplissez avec ce mélange des boîtes en fer-blanc. Soudez soigneusement les
boîtes, immergez-les ensuite dans un bain-marie et laissez cuire pendant une
heure environ.

La préparation est aussi simple que possible et la qualité du produit dépend
exclusivement de la qualité des matières premières employées.

\sk

\index{Conserve de foie gras de canard}
\index{Foie gras de canard en conserve}
On peut préparer de même le pâté de foie gras de canard : mais ce dernier,
toutes choses égales d'ailleurs, est incontestablement inférieur au pâté de
foie d'oie.

Le foie de canard triomphe lorsqu'il est préparé au naturel,
\hyperlink{p0592}{p. \pageref{pg0592}}.

\section*{\centering Mousse de foie gras d'oie en aspic.}
\phantomsection
\addcontentsline{toc}{section}{ Mousse de foie gras d'oie en aspic.}
\index{Mousse de foie gras d'oie en aspic}
\index{Aspic de mousse de foie gras}

Préparez et faites cuire un foie gras comme il est dit
\hyperlink{p0596}{p. \pageref{pg0596}}, après l'avoir assaisonné un peu plus
fortement avec sel, poivre et quatre épices.

Retirez-le de la cocote dès que la cuisson est achevée et passez-le au tamis
fin.

Clarifiez le liquide de cuisson ; réservez-le.

Travaillez la purée de foie sur glace, en y incorporant, par petites quantités,
un cinquième de son poids de crème fouettée très ferme.

Chemisez un moule avec une partie du liquide de cuisson réservé, décorez avec
des rondelles de truffes cuites dans du madère et des rondelles d'œufs durs,
faites prendre à la glace, puis garnissez l'intérieur du moule avec l'appareil
foie gras et crème jusqu'à deux centimètres environ du sommet du moule, et
achevez d'emplir avec du liquide de cuisson.

Mettez à la glace pendant deux heures.

Démoulez sur un socle en glace dressé sur un plat garni d'une serviette,
décorez avec le reste de la cuisson prise en gelée, découpée ou hachée, et
servez.

\sk

On peut préparer dans le même esprit des mousses de volaille, de gibier à plume
et à poil.

\section*{\centering Pain de foie gras.}
\phantomsection
\addcontentsline{toc}{section}{ Pain de foie gras.}
\index{Pain de foie gras}

Faites cuire des truffes dans du champagne, du porto ou du madère ; réservez la
cuisson.

Prenez un beau foie gras, parez-le, truffez-le à volonté et faites-le cuire
comme il est dit \hyperlink{p0596}{p. \pageref{pg0596}} ; laissez-le refroidir dans sa cuisson.
Lorsqu'il est bien froid, prélevez dessus quelques belles escalopes ; passez le
reste au tamis.

Réunissez la cuisson du foie et celle des truffes ; liez le mélange avec
4 jaunes d'œufs ; montez la sauce, au fouet, avec {\ppp175\mmm} grammes de
beurre frais, puis incorporez-y, très légèrement, la purée de foie.

Chemisez un moule à charlotte de gelée de veau et volaille, décorez les parois
avec des émincés de truffes, versez dedans la composition ci-dessus préparée,
disposez dessus les escalopes de foie réservées et des rondelles de truffes,
masquez avec une couche de gelée et mettez à refroidir.

Au moment de servir, démoulez le pain sur un plat et entourez-le de morceaux de
gelée, découpés de jolies formes.

\sk

On peut préparer dans le même esprit des pains de volaille et de gibier à plume
et à poil.

\section*{\centering Pigeons rôtis, sur canapés.}
\phantomsection
\addcontentsline{toc}{section}{ Pigeons rôtis, sur canapés.}
\index{Pigeons rôtis, sur canapés}
\index{Canapés de pigeons rôtis}

Prenez de jeunes pigeons ayant été soumis à un engraissement méthodique, comme
cela se pratique dans les pays d'élevage de poulardes, notamment en Bresse ;
étouffez\footnote{Les pigeons ne doivent jamais être saignés.}—les, plumez-les,
videz-les, habillez-les, bardez-les, puis faites-les rôtir à la broche.

Pendant leur cuisson, préparez des canapés de pain de mie que vous ferez dorer
dans du beurre.

Garnissez les canapés avec de la purée de truffes et de foie gras cuits dans du
madère ; dressez dessus les pigeons rôtis, dont vous aurez enlevé les bardes.

Servez avec une salade verte.

Lorsque les perdreaux manquent, ces pigeons font réellement plaisir.

\section*{\centering Pigeons aux petits pois.}
\phantomsection
\addcontentsline{toc}{section}{ Pigeons aux petits pois.}
\index{Pigeons aux petits pois}

Pour quatre personnes prenez :

\medskip

\footnotesize
\begin{longtable}{rrrp{16em}}
  1 500 & grammes & de & petits pois en cosses,                                                           \\
    125 & grammes & de & lard de poitrine maigre,                                                         \\
     30 & grammes & de & beurre,                                                                          \\
        &         &  6 & petits oignons,                                                                  \\
        &         &  2 & jeunes pigeons,                                                                  \\
        &         &  2 & ou 3 carottes nouvelles (facultatif),                                            \\
        &         &  1 & cœur de laitue,                                                                  \\
        &         &    & sel.                                                                             \\
\end{longtable}
\normalsize

Videz, flambez et bridez les pigeons ; coupez le lard en petits morceaux ;
faites revenir ensemble, pendant vingt minutes, pigeons et lard dans le
beurre ; puis ajoutez les petits pois écossés, les oignons entiers, la laitue
grossièrement hachée et les carottes coupées en morceaux, si les petits pois ne
sont pas suffisamment sucrés par eux-mêmes. Couvrez la casserole et laissez
cuire le tout ensemble à l'étouffée, à petit feu, pendant une heure.
Dégraissez, goûtez et ajoutez un peu de sel, s'il est nécessaire, avant de
servir.

\sk

Si, au moment de l'ouverture de la chasse, on peut avoir des petits pois frais,
on préparera de même des perdreaux aux petits pois, de beaucoup supérieurs aux
pigeons.

\sk

\index{Caneton aux petits pois}
On peut aussi accommoder un caneton aux petits pois : le procédé est le même,
mais la cuisson demande une demi-heure de plus.

Malgré la vieille réputation du pigeon aux petits pois, je ne crains pas de dire
que je trouve bien meilleur le caneton aux petits pois.

\sk

Comme variante, on pourra faire rôtir pigeons, perdreaux ou canetons ; on les
servira avec des petits pois cuits dans un bon fond auquel on ajoutera le
contenu de la lèchefrite.

Cest ainsi que prépareront le plat ceux qui préfèrent les rôtis aux braisés,

\section*{\centering Pigeons farcis rôtis, en turban de riz.}
\phantomsection
\addcontentsline{toc}{section}{ Pigeons farcis rôtis, en turban de riz.}
\index{Pigeons farcis rôtis, en turban de riz}

Préparez du riz au gras, comme il est dit \hyperlink{p0708}{p. \pageref{pg0708}} ;
incorporez-y un hachis de truffes cuites dans du madère, puis mettez-le dans un
moule annulaire légèrement beurré ; placez le moule au bain-marie, le riz s'y
tiendra chaud et prendra la forme d'un turban.

\index{Farce pour pigeons}
Préparez une farce avec du lard de poitrine, des foies de poulets, des truffes,
un peu d'échalote hachée, du sel et du poivre au goût.

Videz les pigeons, coupez-leur le cou, flambez-les, farcissez-les, troussez-les,
bardez-les et faites-les rôtir à la broche pendant vingt à vingt-cinq minutes.

En même temps, faites cuire les débris des pigeons, passez-les à la presse ;
recueillez le jus.

Préparez une sauce madère, aux truffes, comme il est dit
\hyperlink{p0459}{p. \pageref{pg0459}}, incorporez-y le jus de la lèchefrite et
celui des débris de pigeons ; dégraissez la sauce.

Démoulez le turban de riz sur un plat, dressez à l'intérieur les pigeons coupés
en deux, masquez avec la sauce madère, décorez avec des rondelles de truffes et
servez.

\sk

On peut préparer un plat semblable avec des pigeons braisés dans un consommé de
volaille additionné de madère, mais alors la cuisson se fait à tout petit feu
et dure de une heure à une heure et demie.

\section*{\centering Lapin domestique.}
\phantomsection
\addcontentsline{toc}{section}{ Lapin domestique.}
\index{Lapin domestique}

Ce modeste animal de nos basses-cours devrait, à cause de sa vertu prolifique,
jouer un rôle important dans l'alimentation. Mais il est peu apprécié parce
que, le plus souvent mal nourri, il n'offre qu'une chair un peu sèche qui se
ressent trop du chou qui fait la base ordinaire de sa subsistance. Boileau,
dans son « Repas ridicule », s'en est plaint amèrement. Pourtant, lorsqu'on lui
donne une nourriture variée et abondante, à laquelle on prend soin de mélanger
des herbes aromatiques, il acquiert du moelleux et il se parfume agréablement,
sans arriver jamais à valoir, il est vrai, le lapin de garenne, qui doit ses
qualités à sa liberté.

\medskip

On peut préparer le lapin de différentes façons : grillé, rôti, sauté, en
gibelotte, en fricassée, en ragoût, en civet, au jambon, aux petits pois, aux
fines herbes, gratiné, en pâté, etc., etc.

\section*{\centering Civet de lapin domestique.}
\phantomsection
\addcontentsline{toc}{section}{ Civet de lapin domestique.}
\index{Civet de lapin domestique}

Voici une formule de civet de lapin soigné pouvant figurer dans le menu d'un
repas sans prétention.

Pour six personnes prenez :

\smallskip

1° pour le civet :

\footnotesize
\begin{longtable}{rrrp{16em}}
    250 & grammes & de & lard de poitrine,                                                                \\
    250 & grammes & de & champignons,                                                                     \\
        &         & 12 & petits oignons,                                                                  \\
        &         &  5 & échalotes,                                                                       \\
        &         &  2 & lapins moyens,                                                                   \\
        &         &  1 & fort bouquet garni,                                                              \\
        &         &    & légumes de pot-au-feu,                                                           \\
        &         &    & cognac,                                                                          \\
        &         &    & crème,                                                                           \\
        &         &    & farine,                                                                          \\
        &         &    & beurre ;                                                                         \\
\end{longtable}
\normalsize

2° pour la marinade :

\footnotesize
\begin{longtable}{rrrp{16em}}
     10 & grammes & de & serpolet,                                                                        \\
     10 & grammes & de & baies de genièvre,                                                               \\
     10 & grammes & de & bourgeons de sapin,                                                              \\
      4 & grammes & de & thym,                                                                            \\
      2 & grammes & d' & origan,                                                                          \\
      2 & grammes & de & sarriette.                                                                       \\
      2 & grammes & de & safran,                                                                          \\
      1 & gramme  & de & menthe,                                                                          \\
    1/2 & gramme  & de & fleurs de lavande,                                                               \\
        &         &  6 & beaux oignons coupés en rouelles,                                                \\
        &         &    & vin rouge,                                                                       \\
        &         &    & vinaigre.                                                                        \\
        &         &    & sel et poivre.                                                                   \\
\end{longtable}
\normalsize

Lavez les bourgeons de sapin ; échaudez-les ; rafraîchissez-les.

Dépouillez les lapins ; videz-les ; détachez-en les râbles ; réservez le reste
et conservez-le à la glacière avec le sang et les foies.

Découpez les râbles en morceaux ; mettez-les à mariner, pendant deux heures,
d'abord dans du vinaigre, puis, après les avoir essuyés, dans du bon vin rouge,
du mâcon par exemple, et ajoutez-y les oignons coupés en rouelles, les aromates
indiqués au chapitre marinade, du sel, du poivre ; laissez en contact pendant
deux jours.

\index{Fonds de lapin}
En temps voulu, préparez un fond de lapin en faisant cuire, à bouilli perdu, le
reste des lapins avec des légumes de pot-au-feu, dans de l'eau salée ;
dégraissez-le, passez-le, concentrez-le.

Enlevez les morceaux de râbles de la marinade ; essuyez-les.

Faites roussir un peu de farine dans du beurre, puis mettez : d'abord les 12
petits oignons, les échalotes et le lard de poitrine coupé en dés, laissez
dorer ; ensuite les morceaux de lapin que vous ferez bien revenir ; flambez le
tout avec du cognac ; mouillez avec le fond de lapin auquel vous ajouterez plus
ou moins de la marinade, passée, au goût, et le bouquet garni. Laissez cuire
à petit feu pendant deux heures environ. Une vingtaine de minutes avant la fin,
mettez les champignons et, au dernier moment, achevez la liaison de la sauce
avec le sang et les foies pilés des lapins. Goûtez, complétez
l'assaisonnernent, s'il y a lieu, et finissez la sauce avec plus ou moins de
crème pour corriger tout excès de saveur de vinaigre ou de bourgeon de sapin.

Servez en envoyant à part un légumier de pommes de terre, de marrons, de
topinambours ou de patates, en purée.

\section*{\centering Pâté de lapin à l'anglaise.}
\phantomsection
\addcontentsline{toc}{section}{ Pâté de lapin à l'anglaise.}
\index{Pâté de lapin à l'anglaise}
\index{Civet de lapin domestique en croûte}

\centering\textit{(Rabbit pie.)}

\bigskip

\justifying

Pour quatre personnes prenez :

\medskip

\footnotesize
\begin{longtable}{rrrp{16em}}
    400 & grammes & de & pâte feuilletée,                                                                 \\
    125 & grammes & de & bacon,                                                                           \\
    125 & grammes & de & champignons,                                                                     \\
    125 & grammes & d' & olives privées de leurs noyaux,                                                  \\
        &         &  4 & œufs durs,                                                                       \\
        &         &  1 & lapin,                                                                           \\
        &         &    & légumes de pot-au-feu,                                                           \\
        &         &    & lard à piquer,                                                                   \\
        &         &    & beurre,                                                                          \\
        &         &    & vin blanc,                                                                       \\
        &         &    & fine champagne,                                                                  \\
        &         &    & sauge, thym, serpolet, marjolaine,                                               \\
        &         &    & sel, poivre, paprika.                                                            \\
\end{longtable}
\normalsize

Dépouillez et videz le lapin ; mettez de côté le râble, le foie et le sang.

Préparez un jus en faisant cuire, dans de l'eau salée, le reste du lapin et des
légumes de pot-au-feu.

Coupez le râble en quatre tranches que vous piquerez de lardons assaisonnés
avec du paprika ; salez-les, poivrez-les et mettez-les, avec du beurre et le
bacon coupé en quatre tranches, dans un plat creux en porcelaine allant au
feu ; faites revenir le tout ; flambez à la fine champagne, mouillez avec du
vin blanc et les trois quarts du jus de lapin ; aromatisez avec sauge, thym,
serpolet et marjolaine en poudre ; laissez cuire pendant une demi-heure en plat
couvert. Ajoutez ensuite les champignons pelés et passés dans du beurre, les
olives, les œufs durs coupés en quatre ; goûtez et complétez l'assaisonnement
s'il est nécessaire.

Remplacez le couvercle par une abaisse en pâte feuilletée que vous fixerez avec
les doigts sur le bord du plat, rayez le dessus avec un couteau, décorez-le et
réservez au milieu une petite cheminée, comme dans la préparation des pâtés.
Poussez au four, où vous laisserez le plat pendant une demi-heure environ.

Pendant ce temps, concentrez le reste du jus de lapin, écrasez dedans le foie
el liez avec le sang.

Au moment de servir, introduisez cette sauce, chaude, dans le plat, par la
cheminée que vous boucherez ensuite avec un bouchon de pâte cuit à part.

À table, levez le couvercle de pâte, mettez-le sur un plat à part, découpez-le
et servez à chaque convive du ragoût et de la croûte feuilletée qui
l'accompagnera très agréablement.

Ce plat pourrait être appelé « Civet de lapin en croûte ».

\sk

En s'inspirant de cette formule et de celles du Beef pie et du Chicken pie,
pp. \hyperlink{p0471}{\pageref{pg0471}} et \hyperlink{p0548}{\pageref {pg0548}}, 
il est facile de préparer d'autres plats en croûte.

\sk
