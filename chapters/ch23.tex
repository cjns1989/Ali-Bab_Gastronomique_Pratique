\section*{\centering Salades vertes.}
\phantomsection
\addcontentsline{toc}{section}{ Salades vertes.}
\index{Salades vertes}

Les salades vertes sont le plus souvent accommodées à l'huile et au vinaigre,
mais on peut aussi les apprêter à la crème qui, alors, remplace l'huile. Leur
goût dépend non seulement de la nature et de l'état des substances employées
(fraîcheur de la salade, qualité de l'huile et du vinaigre), mais aussi de leur
préparation préliminaire. Les salades, telles que laitue, romaine, scarole,
chicorée gagnent beaucoup de légèreté et de finesse à être débarrassées des
côtes et à être revivifiées longuement dans de l’eau fraîche lorsqu'elles ne
sont pas nouvellement cueillies. Inutile de dire qu'on ne doit se servir que
des parties tendres de toutes les salades pour obtenir un plat réellement bon.

L'assaisonnement se fait, au goût, avec huile, vinaigre, sel et poivre auxquels
on peut ajouter un jaune d'œuf frais, mais la salade ne doit être accommodée
que juste au moment d'être servie et, contrairement à la façon de faire de bien
des personnes, on doit la retourner légèrement, délicatement, sans la froisser.

\section*{\centering Salade de laitue à la crème.}
\phantomsection
\addcontentsline{toc}{section}{ Salade de laitue à la crème.}
\index{Salade de laitue à la crème}

Prenez de belles laitues bien pommées, lavez-les, égouttez-les bien. Préparez,
dans un saladier, un mélange homogène de crème, jus de citron ou vinaigre,
jaunes d'œufs durs tamisés, sel et poivre, le tout au goût ; mettez dedans les
laitues et remuez jusqu'à assaisonnement uniforme.

Servez aussitôt.

\sk

En supprimant les jaunes d'œufs durs et en remplaçant la crème ordinaire par
de la crème fouettée, on obtient une salade que certaines personnes trouvent plus
légère que la précédente.

\sk

\index{Salade de scarole à la crème}
\index{Salade de romaine à la crème}
\index{Salade d'endives à la crème}
\index{Chicorée en salade}
\index{Scarole en salade}
\index{Romaine en salade}
\index{Endives en salade}
Ces formules sont applicables à la romaine, à la chicorée frisée, à la scarole,
aux endives, etc.

\section*{\centering Salade de haricots verts et de tomates.}
\phantomsection
\addcontentsline{toc}{section}{ Salade de haricots verts et de tomates.}
\index{Salade de haricots verts et de tomates}

Pour six personnes prenez :

\footnotesize
\begin{longtable}{rrrp{16em}}
    750 & grammes & de & haricots verts,                                                                  \\
    750 & grammes & de & tomates,                                                                         \\
     45 & grammes & d' & huile d'olive ou de noix,                                                        \\
     15 & grammes & de & vinaigre de vin,                                                                 \\
     15 & grammes & de & Worcestershire sauce,                                                            \\
        &         &    & sel.                                                                             \\
\end{longtable}
\normalsize

Épluchez les haricots verts et faites-les cuire dans de l'eau salée.

Ébouillantez les tomates, pelez-les, épépinez-les, coupez-les en tranches,
faites-les dégorger dans du gros sel ; sortez les tranches de tomates du
liquide qu'elles ont rendu ; secouez-les pour les débarrasser de tout excès de
sel.

Mettez dans un saladier l'huile, le vinaigre et la sauce anglaise, mélangez,
ajoutez haricots verts et tranches de tomates, mélangez encore, laissez en contact
pendant une heure et servez.

Cette salade est très fraîche.

\section*{\centering Salade de salsifis.}
\phantomsection
\addcontentsline{toc}{section}{ Salade de salsifis.}
\index{Salade de salsifis}

Pour six à huit personnes prenez :

\footnotesize
\begin{longtable}{rrrp{16em}}
    600 & grammes & de & salsifis épluchés,                                                               \\
    300 & grammes & de & céleri-rave épluché,                                                             \\
    300 & grammes & de & tomates,                                                                         \\
    250 & grammes & d' & huile d'olive,                                                                   \\
        &         & 12 & noix,                                                                            \\
        &         &  2 & jaunes d'œufs frais,                                                             \\
        &         &  1 & citron,                                                                          \\
        &         &    & huile.                                                                           \\
        &         &    & vinaigre,                                                                        \\
        &         &    & sel et poivre.                                                                   \\
\end{longtable}
\normalsize

Coupez les salsifis en morceaux et le céleri-rave en bâtonnets de mêmes
dimensions,

Épluchez les noix, hachez-les grossièrement.

Ébouillantez les tomates, pelez-les, épépinez-les, mettez-les à mariner, pendant
une demi-heure, dans de l'huile et du vinaigre avec sel et poivre.

Faites blanchir le céleri pendant quelques minutes dans de l'eau salée
vinaigrée, enlevez-le, laissez-le refroidir ; puis, dans la même eau, faites
cuire les salsifis en les tenant fermes ; retirez-les ; laissez-les refroidir.

Préparez une mayonnaise avec les jaunes d'œufs, l'huile d'olive, le jus du
citron, un peu de sel et de poivre, mettez dedans les salsifis, le céleri et
les noix ; laissez en contact pendant une demi-heure. Au moment de servir,
ajoutez les tomates et leur marinade, mélangez bien, goûtez et complétez
l'assaisonnement sil y a lieu.

\sk

On peut préparer dans le même esprit des salades de fonds d'artichauts, de
crosnes, de germes de soja, etc.

\sk

On peut encore relever ces salades, au goût, avec plus ou moins de
Worcestershire sauce.

\section*{\centering Salade d'épinards.}
\phantomsection
\addcontentsline{toc}{section}{ Salade d'épinards.}
\index{Salade d'épinards}

Pour six personnes prenez :

\footnotesize
\begin{longtable}{rrrrp{16em}}
  &   1 500 & grammes & d' & épinards,                                                                    \\
  &      40 & grammes & d’ & huile d'olive fine,                                                          \\
  &      25 & grammes & de & vinaigre de vin,                                                             \\
  &      20 & grammes & de & moutarde ordinaire,                                                          \\
  &      10 & grammes & de & sel blanc,                                                                   \\
  & \multicolumn{2}{r}{2 décigramme} & de & poivre.                                                       \\
\end{longtable}
\normalsize

Épluchez les épinards et lavez-les sans déchirer les feuilles ; puis
ébouillantez-les vivement ; égouttez-les ; laissez-les refroidir.

Mettez dans un saladier l'huile, le vinaigre, la moutarde, le sel, le poivre ;
mélangez ; ajoutez ensuite les épinards et laissez-les s'imprégner pendant un
quart d'heure environ. Retournez-les fréquemment pendant ce temps, mais en
évitant autant que possible de les briser.

Servez.

\medskip

\index{Épinards en salade}
Les épinards en salade se marient bien avec les viandes blanches froides,
notamment avec le veau.

\section*{\centering Salade de chou rouge.}
\phantomsection
\addcontentsline{toc}{section}{ Salade de chou rouge.}
\index{Salade de chou rouge}
\index{Chou rouge en salade}

Pour six personnes prenez :

\medskip

\footnotesize
\begin{longtable}{rrrp{16em}}
    250 & grammes & de & crème épaisse,                                                                   \\
     20 & grammes & de & cerfeuil, ciboule, fenouil, en parties égales,                                   \\
        &         &  6 & jaunes d'œufs durs,                                                              \\
        &         &  3 & petits concombres saumurés,                                                      \\
        &         &  1 & chou rouge pesant 1 kilogramme environ,                                          \\
        &         &    & radis,                                                                           \\
        &         &    & vinaigre à l'estragon,                                                           \\
        &         &    & jus de citron,                                                                   \\
        &         &    & sel et poivre.                                                                   \\
\end{longtable}
\normalsize

Épluchez le chou, enlevez-en les côtes ; lavez-le, émincez-le en julienne que
vous ferez blanchir dans de l’eau salée bouillante. Rafraîchissez-le,
égouttez-le et mettez-le à mariner dans du vinaigre à l'estragon avec du sel et
du poivre ; laissez en contact pendant une heure en le retournant fréquemment.
Retirez ensuite le chou de la marinade et égouttez-le.

Passez les jaunes d'œufs au tamis, mélanges-les à la crème, ajoutez du jus de
citron, du sel, du poivre au goût, le cerfeuil, la ciboule, le fenouil et le
chou ; mélangez.

Décorez la salade avec des émincés de concombres et de radis et servez.

Cette formule de chou rouge en salade sort un peu de l'ordinaire.

\section*{\centering Asperges en salade.}
\phantomsection
\addcontentsline{toc}{section}{ Asperges en salade.}
\index{Asperges en salade}

Faites cuire dans de l'eau salée bouillante, d'une part, de belles asperges
d'Argenteuil et, d'autre part, de petites asperges vertes appelées balais.
Tenez les grosses asperges un peu fermes ; passez les balais en purée. Laissez
refroidir le tout.

Préparez une mayonnaise au citron ; incorporez-y la purée d'asperges vertes.

Servez les asperges d'Argenteuil sur un plat et la sauce dans une saucière.

Vous m'en direz des nouvelles.

\sk

On peut accommoder avec la même sauce d’autres salades.

\section*{\centering Salade d'aubergines.}
\phantomsection
\addcontentsline{toc}{section}{ Salade d'aubergines.}
\index{Salade d'aubergines}
\index{Aubergines en salade}

Pour quatre personnes prenez :

\footnotesize
\begin{longtable}{rrrp{16em}}
        &         &  4 & aubergines moyennes,                                                             \\
        &         &  4 & piments longs,                                                                   \\
        &         &    & huile, vinaigre,                                                                 \\
        &         &    & sel, poivre.                                                                     \\
\end{longtable}
\normalsize

Faites griller légèrement les piments ; enlevez-en l'enveloppe superficielle.

Essuyez les aubergines ; faites-les griller telles quelles jusqu'à ce qu'elles
soient complètement ramollies ; débarrassez-les de leur peau et écrasez-les au
moyen d'une cuiller en bois pour éviter que la puréc noircisse. Laissez
refroidir, puis assaisonnez au goût avec sel, poivre, huile, vinaigre.

Mettez la purée dans un plat ou dans un ravier, décorez avec les piments et
servez.

\medskip

En Orient, cette salade est présentée comme hors-d'œuvre.

\section*{\centering Salade de fonds d'artichauts au foie gras.}
\phantomsection
\addcontentsline{toc}{section}{ Salade de fonds d'artichauts au foie gras.}
\index{Salade de fonds d'artichauts au foie gras}
\index{Artichauts au foie gras, en salade}

Prenez autant de beaux artichauts qu'il y a de convives, faites-les cuire dans
de l'eau salée ; enlevez-en les feuilles, parez les fonds puis mettez-les
à mariner pendant vingt minutes dans du jus de citron additionné ou non d'huile
d'olive.

Disposez dans un saladier les fonds d’artichauts, couronnez chacun d'une
tranche de foie gras truffé, cuit dans du madère, ou, à défaut, de foie gras
truffé en terrine, de même grandeur et de même épaisseur que le fond ;
intercalez entre les fonds d'artichauts des petits cœurs de laitue, de chicorée
frisée ou de romaine, assaisonnés au goût ; ou garnissez avec des petits
bouquets de chou-palmiste masqués avec une mayonnaise peu serrée.

Servez aussitôt.

C'est une salade vraiment remarquable.

\section*{\centering Salade de cervelle et de laitue.}
\phantomsection
\addcontentsline{toc}{section}{ Salade de cervelle et de laitue.}
\index{Salade de cervelle et de laitue}

Faites blanchir une cervelle de veau parée dans de l'eau salée, vinaigrée et
aromatisée avec un peu de persil et d'oignon ; laissez-la refroidir dans sa
cuisson ; escalopez-la ensuite et trempez les tranches dans de la sauce
vinaigrette.

Préparez une salade de laitue que vous assaisonnerez avec une mayonnaise
relevée par de la moutarde ; disposez dessus les escalopes de cervelle, décorez
avec des concombres saumurés, des petits melons, des cornichons, des
champignons confits et émincés, des queues de crevettes ou d'écrevisses, des
câpres, etc. ; servez sans attendre.

\sk

On peut préparer de même une salade de ris de veau et de laitue.

\section*{\centering Salade de pommes de terre.}
\phantomsection
\addcontentsline{toc}{section}{ Salade de pommes de terre.}
\index{Salade de pommes de terre}
\index{Pommes de terre en salade}

Pour six personnes prenez :

\footnotesize
\begin{longtable}{rrrp{16em}}
  1 000 & grammes & de & pommes de terre de Hollande,                                                     \\
    100 & grammes & de & vin blanc,                                                                       \\
    100 & grammes & de & bouillon,                                                                        \\
    100 & grammes & de & vinaigre de vin,                                                                 \\
     60 & grammes & de & pommes reinettes,                                                                \\
     25 & grammes & d’ & huile d'olive,                                                                   \\
      5 & grammes & de & civette,                                                                         \\
      3 & grammes & de & persil,                                                                          \\
      3 & grammes & d' & estragon,                                                                        \\
      2 & grammes & de & cerfeuil,                                                                        \\
      2 & grammes & de & poivre fraîchement moulu,                                                        \\
      2 & grammes & de & moutarde,                                                                        \\
        &         &  1 & beau hareng de Hollande laité et salé.                                           \\
\end{longtable}
\normalsize

Dépouillez le hareng sans le dessaler, enlevez la tête et les arêtes, passez au
tamis la chair et la laitance.

Hachez pommes et fines herbes.

Mettez dans un saladier le vinaigre, le vin, le bouillon, le poivre, la moutarde
et l'huile, mélangez bien, ajoutez ensuite la purée de hareng et le hachis de pommes
et de fines herbes,

Faites cuire les pommes de terre à la vapeur, pelez-les pendant qu'elles sont
chaudes et coupez-les en tranches minces.

Battez la sauce pour la rendre homogène, mettez dedans les tranches de pommes
de terre encore chaudes et retournez-les avec précaution afin d'éviter de les
briser ; laissez en contact pendant trois heures au moins, retournez de nouveau
la salade, goûtez et servez.

Cette salade, très simple, est agréable.

\section*{\centering Salade de choucroute et de pommes de terre.}
\phantomsection
\addcontentsline{toc}{section}{ Salade de choucroute et de pommes de terre.}
\index{Salade de choucroute et de pommes de terre}
\index{Choucroute en salade}

Pour huit personnes prenez :

\footnotesize
\begin{longtable}{rrrp{16em}}
  1 000 & grammes & de & pommes de terre,                                                                 \\
    500 & grammes & de & choucroute,                                                                      \\
    200 & grammes & de & vin blanc,                                                                       \\
    200 & grammes & de & pommes reinettes,                                                                \\
    150 & grammes & de & sauce douce, \hyperlink{p0415}{p. \pageref{pg0415}},                             \\
        &         &    & cornichons hachés,                                                               \\
        &         &    & fines herbes hachées,                                                            \\
        &         &    & huile,                                                                           \\
        &         &    & vinaigre,                                                                        \\
        &         &    & sel et poivre.                                                                   \\
\end{longtable}
\normalsize

Lavez la choucroute, égouttez-la.

Pelez les pommes de terre, faites-les cuire à la vapeur, coupez-les en tranches
pendant qu'elles sont chaudes, arrosez-les aussitôt avec le vin blanc, puis mettez
la choucroute crue, mélangez.

Pelez les pommes, coupez-les en dés.

Deux heures au moins avant de servir, assaisonnez le mélange pommes de terre et
choucroute avec la sauce douce, ajoutez les pommes, les fines herbes, les
cornichons, mélangez encore, goûtez, complétez l’assaisonnement s'il y a lieu
avec huile, vinaigre, sel, poivre, puis servez.

Cette salade convient plutôt aux estomacs robustes.

\section*{\centering Salade de moules et de pommes de terre.}
\phantomsection
\addcontentsline{toc}{section}{ Salade de moules et de pommes de terre.}
\index{Salade de moules et de pommes de terre}

Pour six personnes prenez :

\footnotesize
\begin{longtable}{rrrrp{16em}}
  &   1 000 & grammes & de & pommes de terre,                                                             \\
  &     200 & grammes & d' & huile d'olive,                                                               \\
  &     100 & grammes & de & vin blanc,                                                                   \\
  &      10 & grammes & de & cornichons,                                                                  \\
  &       5 & grammes & de & fines herbes hachées,                                                        \\
  & \multicolumn{2}{r}{1 gramme 1/2} & de & moutarde en poudre,                                           \\
  & \multicolumn{2}{r}{3 litres} & de & moules,                                                           \\
  &         &         &  1 & jaune d'œuf cru,                                                             \\
  &         &         &  1 & oignon,                                                                      \\
  &         &         &  1 & citron,                                                                      \\
  &         &         &  1 & bouquet garni,                                                               \\
  &         &         &    & sel et poivre.                                                               \\
\end{longtable}
\normalsize

Nettoyez les moules, puis mettez-les sur le feu, dans une casserole, avec un
peu d'eau salée, l'oignon et le bouquet garni ; laissez-les s'ouvrir ; enlevez-les des
coquilles.

Faites cuire les pommes de terre au diable Rousset ou à la vapeur, pelez-les
pendant qu'elles sont chaudes, coupez-les en tranches et arrosez-les avec le
vin.

Préparez une sauce mayonnaise avec l'huile et le jaune d'œuf, ajoutez-y le jus
du citron, les fines herbes, {\ppp5\mmm} grammes de cornichons hachés, la
moutarde, du sel et du poivre au goût.

Disposez dans un saladier des couches alternées de pommes de terre, de
mayonnaise et de moules, mélangez bien, décorez le plat avec un peu de
mayonnaise, le reste des cornichons émincés, quelques moules que vous aurez mis
de coté et servez.

\section*{\centering Salade de légumes.}
\phantomsection
\addcontentsline{toc}{section}{ Salade de légumes.}
\index{Salade de légumes}

Pour six personnes prenez :

\footnotesize
\begin{longtable}{rrrp{16em}}
    500 & grammes & de  & fonds d'artichauts,                                                             \\
    250 & grammes & de  & pommes de terre vitelotte,                                                      \\
    250 & grammes & de  & céleri (les parties tendres seulement),                                         \\
    250 & grammes & de  & truffes noires du Périgord,                                                     \\
    125 & grammes & de  & bon vin de Bourgogne blanc,                                                     \\
        &         & 1/2 & cuillerée à café de chartreuse jaune,                                           \\
        &         &     & mayonnaise au citron, relevée par un peu de Worcestershire sauce,               \\
        &         &     & vin de Champagne,                                                               \\
        &         &     & lard,                                                                           \\
        &         &     & huile d'olive, d'œillette ou de noix,                                           \\
        &         &     & vinaigre de vin,                                                                \\
        &         &     & moutarde à l'estragon,                                                          \\
        &         &     & sel et poivre.                                                                  \\
\end{longtable}
\normalsize

Nettoyez, lavez et coupez le céleri en petits fragments.

Faites cuire séparément les fonds d'artichauts et les pommes de terre dans de
l'eau salée.

Pelez et coupez en tranches les pommes de terre ; coupez les fonds d'artichauts
chacun en huit morceaux.

Brossez soigneusement les truffes, lavez-les, séchez-les, faites-les cuire
pendant un quart d'heure, avec du champagne, dans une casserole foncée de lard.
Emincez-les en rondelles.

Préparez une sauce relevée avec huile d'olive, d'œillette ou de noix, vinaigre,
sel, poivre et moutarde.

Six heures avant le repas, mettez dans cette sauce les pommes de terre et le vin
de Bourgogne blanc. Retournez avec précaution.

Deux heures plus tard, ajoutez le céleri et la chartreuse. Mélangez, toujours
avec précaution.

Enfin, une heure après, mettez les fonds d'artichauts et remuez encore.

N'ajoutez les truffes qu'à la fin.

Mélangez bien le tout, masquez avec la mayonnaise au citron et décorez avec
quelques rondelles de truffes mises en réserve.

C'est une salade royale.

\section*{\centering Salade de légumes.}
\phantomsection
\addcontentsline{toc}{section}{ Salade de légumes.}
\index{Salade de légumes}

\begin{center}
\textit{(Autre formule).}
\end{center}

Pour six à huit personnes prenez :

\footnotesize
\begin{longtable}{rrrp{16em}}
    400 & grammes & de & céleri,                                                                          \\
    200 & grammes & de & truffes noires du Périgord,                                                      \\
        &         &  8 & fonds d'artichauts,                                                              \\
        &         &    & mayonnaise à la moutarde,                                                        \\
        &         &    & fine champagne,                                                                  \\
        &         &    & jus de citron.                                                                   \\
\end{longtable}
\normalsize

Coupez les fonds d'artichauts en tranches minces, les truffes en rondelles, le
céleri en petits morceaux.

Faites mariner pendant {\ppp24\mmm} heures les tranches de fonds d'artichauts
dans du jus de citron, les rondelles de truffes dans de la fine champagne.

Mélangez ensuite céleri, truffes et fonds d'artichauts et assaisonnez le tout avec
de la mayonnaise à la moutarde,

\section*{\centering Salade de légumes en surprise.}
\phantomsection
\addcontentsline{toc}{section}{ Salade de légumes en surprise.}
\index{Salade de légumes en surprise}

Cette salade se présente sous la forme d'une pièce montée, composée de deux
parties distinctes.

1° une couronne formée avec une salade de pommes de terre hachées
grossièrement, mélangées à du céleri haché fin, le tout assaisonné avec une
mayonnaise très claire relevée par de la moutarde anglaise. Cette salade, mise
dans un moule en couronne, est rafraîchie très fortement dans de la glace ;

2° un pain central constitué par une salade de chou-fleur passé ou non,
assaisonnée avec de l'huile et du vinaigre, mise dans un moule à charlotte et
également très rafraîchie, qu'on masque au moment de servir avec une mayonnaise
verte très épaisse et très froide, qui lui donne l'aspect d’une glace à la
pistache.

Pour embellir le plat, on peut le décorer, en dehors de la couronne, avec des
rondelles de betterave, de tomate et des piments rouges.

Cette salade en surprise, excellente au goût, est une agréable fumisterie
culinaire.

\section*{\centering Salade aux noix.}
\phantomsection
\addcontentsline{toc}{section}{ Salade aux noix.}
\index{Salade aux noix}

Pour dix à douze personnes prenez :

\footnotesize
\begin{longtable}{rrrp{16em}}
  1 000 & grammes & de & noix sèches,                                                                     \\
    500 & grammes & de & céleri-rave épluché,                                                             \\
    500 & grammes & d’ & huile d'olive,                                                                   \\
    400 & grammes & de & betterave,                                                                       \\
    250 & grammes & de & truffes du Périgord,                                                             \\
        &         &  3 & jaunes d'œufs frais,                                                             \\
        &         &  2 & belles pommes reinettes,                                                         \\
        &         &  1 & beau citron,                                                                     \\
        &         &    & sel, poivre,                                                                     \\
        &         &    & sauce Worcestershire.                                                            \\
\end{longtable}
\normalsize

Sortez les noix des coquilles, ébouillantez-les, épluchez-les, séparez-les en quatre.

Pelez les pommes de terre et la betterave ; nettoyez les truffes.

Emincez en julienne séparément céleri, pommes, betterave et truffes.

Ebouillantez le céleri ; laissez-le refroidir.

Mettez dans un saladier noix, céleri, pommes, betterave et truffes, moins une
petite partie que vous réserverez pour décorer le plat.

Préparez une mayonnaise un peu fluide avec l'huile, les jaunes d'œufs et le jus
du citron, assaisonnez-la avec sel et poivre, relevez-la au goût avec de la
sauce Worcestershire, puis mélangez-la avec le contenu du saladier.

Décorez le dessus avec les éléments réservés et servez.

Cette salade, qui a une saveur douce et agréable, doit aux noix son goût spécial.

\section*{\centering Salade aux œufs.}
\phantomsection
\addcontentsline{toc}{section}{ Salade aux œufs.}
\index{Salade aux œufs}

Disposez en forme de couronne, sur un plat, des fonds d’artichauts cuits ;
mettez sur chaque fond soit un œuf dur farci d'un mélange d'anchois et de
jaunes d'œufs, soit un œuf à jaune liquide cuit pendant trois minutes et
refroidi.

A l'intérieur de la couronne dressez un petit monticule de queues de crevettes
masqué par une mayonnaise.

Garnissez l'extérieur de la couronne avec des cœurs de laitue assaisonnés et
décorez avec des cornichons et des câpres.

Le plat ainsi préparé est très élégant.

\section*{\centering Salade de pointes d'asperges.}
\phantomsection
\addcontentsline{toc}{section}{ Salade de pointes d'asperges.}
\index{Salade de pointes d'asperges}

Pour six personnes prenez :

\footnotesize
\begin{longtable}{rrrp{16em}}
    500 & grammes & de & pointes d'asperges,                                                              \\
    250 & grammes & de & crevettes grises,                                                                \\
    125 & grammes & de & caviar frais,                                                                    \\
        &         &  6 & petits fonds d'artichauts,                                                       \\
        &         &  2 & jaunes d'œufs frais,                                                             \\
        &         &  1 & œuf dur,                                                                         \\
        &         &    & quelques belles crevettes roses,                                                 \\
        &         &    & huile d'olive,                                                                   \\
        &         &    & jus de citron,                                                                   \\
        &         &    & sel et poivre.                                                                   \\
\end{longtable}
\normalsize

Faites cuire séparément, dans de l’eau salée bouillante, les pointes d'asperges
pendant un quart d'heure et les fonds d'artichauts jusqu'à ce qu'ils soient
tendres. Laissez-les refroidir.

Réservez une demi-douzaine des plus belles pointes d'asperges pour décorer le
plat.

Court-bouillonnez les crevettes ; mettez de côté les crevettes roses entières ;
épluchez les crevettes grises,

Préparez une mayonnaise avec les jaunes d'œufs frais, de l'huile, du jus de
citron, du sel et du poivre.

Hachez l'œuf dur.

Mettez dans un saladier les pointes d'asperges et les crevettes grises,
assaisonnez-les avec de la mayonnaise.

Dressez la salade en monticule au milieu d'un plat creux, saupoudrez avec le
hachis d'œuf dur, décorez avec les pointes d'asperges réservées et les crevettes
roses, disposez autour les fonds d'artichauts garnis de mayonnaise et de caviar,
puis servez.

\section*{\centering Salade de riz.}
\phantomsection
\addcontentsline{toc}{section}{ Salade de riz.}
\index{Salade de riz}

Pour douze personnes prenez :

\footnotesize
\begin{longtable}{rrrp{16em}}
    500 & grammes & de & riz,                                                                             \\
    500 & grammes & de & tomates,                                                                         \\
    100 & grammes & de & truffes cuites au madère et coupées en rondelles,                                \\
    100 & grammes & de & câpres,                                                                          \\
    100 & grammes & de & noix ou de noisettes épluchées et hachées,                                       \\
     60 & grammes & de & cornichons hachés,                                                               \\
     60 & grammes & d' & huile,                                                                           \\
     30 & grammes & de & vinaigre,                                                                        \\
      5 & grammes & de & sel,                                                                             \\
      3 & grammes & de & paprika,                                                                         \\
        &         &  1 & homard œuvé, pesant 1 200 grammes environ,                                       \\
        &         &  1 & piment rouge Maille.                                                             \\
\end{longtable}
\normalsize

Faites cuire le riz sec, \hyperlink{p0707}{p. \pageref{pg0707}}. Laissez-le
refroidir.

Court-bouillonnez le homard comme il est dit
\hyperlink{p0284}{p. \pageref{pg0284}} ; laissez-le refroidir dans sa cuisson,
puis escalopez la queue et décortiquez les pattes.

Écrasez le piment dans le vinaigre, laissez en contact pendant un moment ;
passez au chinois.

Passez les œufs du homard au tamis ; recueillez le jus.

Mettez dans un saladier le vinaigre pimenté, le jus des œufs de homard,
l'huile, le sel, le paprika, battez, puis ajoutez le riz, les câpres, les noix
ou les noisettes, les cornichons, les tomates crues, pelées, épépinées et
coupées en morceaux ; mélangez bien ; enfin laissez tomber dans la salade la
queue de homard escalopée et les débris des pattes ; mélangez encore un peu,
décorez avec les truffes et servez.

C'est original, bon et frais.

\sk

Cette salade peut être modifiée à volonté. Elle peut être simplifiée par la
suppression des truffes, ou rendue plus riche par l'augmentation de la
proportion de cet élément et par l’adjonction d'éléments nouveaux : saumon,
queues de crevettes ou d'écrevisses, etc.

\section*{\centering Salade de crosnes.}
\phantomsection
\addcontentsline{toc}{section}{ Salade de crosnes.}
\index{Salade de crosnes}
\index{Crosnes en salade}

Pour quatre personnes prenez :

\footnotesize
\begin{longtable}{rrrp{16em}}
    150 & grammes & de & crosnes,                                                                         \\
    100 & grammes & de & langue à l'écarlate,                                                             \\
    100 & grammes & de & blanc de poulet rôti,                                                            \\
    100 & grammes & de & céleri cru (le blanc seulement),                                                 \\
        &         & 12 & noisettes échaudées et hachées,                                                  \\
        &         &  2 & petites pommes de terre cuites à l'eau,                                          \\
        &         &  2 & œufs durs,                                                                       \\
        &         &    & mayonnaise,                                                                      \\
        &         &    & cerfeuil haché.                                                                  \\
\end{longtable}
\normalsize

Débarrassez les crosnes de leurs radicelles, lavez-les, mettez-les dans un
torchon avec du gros sel, secouez vivement, vous enlèverez ainsi la pellicule
qui les entoure. Lavez-les de nouveau, plongez-les ensuite dans de l’eau
bouillante légèrement salée, laissez-les cuire pendant quinze à vingt minutes,
suivant leur grosseur, en ayant soin d'arrêter la cuisson à temps pour qu'ils
gardent une certaine consistance ; enfin égouttez-les et laissez-les refroidir.
Ils sont prêts à entrer dans la salade.

Émincez en julienne la langue, le poulet, le céleri, les pommes de terre ;
coupez les œufs en tranches, mettez le tout avec les crosnes et les noisettes
dans un saladier, saupoudrez de cerfeuil haché et assaisonnez avec de la
mayonnaise.

Cette salade, très agréable, est une surprise\footnote{\index{Apologue de la soupe au fer à cheval}
Elle me rappelle
l'apologue suivant :
                \protect\endgraf

\begin{center}
\textit{Apologue de la soupe au fer à cheval.}
\end{center}
                \protect\endgraf

Un pauvre diable, sans le sou et mourant de faim, arriva un jour dans un
village peu hospitalier. Il frappa en vain à toutes les portes. Éconduit de
partout avec ces mots : « Nous n'avons rien à vous donner », il se désespérait,
lorsque, étant arrivé sur la Grand'Place où de nombreux villageois étaient
réunis, il aperçut à terre un fer à cheval. Brusquement une idée lui vint. Il
ramassa le fer et, s'avançant vers un groupe, il dit : « Avez-vous jamais vu
faire une soupe avec un fer à cheval ? » Tout le monde se mit à rire. « Mais
oui une soupe, une bonne soupe même », ajouta le pauvre, « donnez-moi seulement
une marmite, de l'eau, de quoi faire du feu et vous verrez. » Les villageois
rirent de plus belle ; mais quelques-uns d'entre eux, désireux de voir le
prodige, lui apportèrent ce qu'il demandait. Notre homme mit la marmite sur le
feu, le fer dans la marmite et attendit.
                \protect\endgraf

Cependant, à l'annonce de cette merveille, tout le village était accouru.
                \protect\endgraf

« J'pourrai-t-y y goûter à ta soupe ? » demanda bientôt quelqu'un. — « Oui »,
répondit le vagabond et il ajouta : « Tu aimerais peut-être qu'elle ait un
petit goût de lard, mais voilà, il n'y en a pas. »
                \protect\endgraf

« J'en pourrai-t-y avoir aussi ? » dit un autre. — « Bien sûr et tu verras
comme elle sera bonne ; elle serait encore meilleure s'il y avait un peu de
légumes, mais enfin tant pis, on s'en passera. »
                \protect\endgraf

Les paysans tombèrent dans le piège et bientôt notre psychologue eut du lard,
des choux, des pommes de terre, etc. Par son subterfuge, il avait obtenu tout
ce qu'on lui refusait d'abord,
                \protect\endgraf

Ma salade de crosnes tient du même procédé ; elle est excellente, mais les
crosnes y jouent un peu le rôle du fer à cheval ; Les autres éléments en sont
la véritable base.}.

\section*{\centering Salade exotique.}
\phantomsection
\addcontentsline{toc}{section}{ Salade exotique.}
\index{Salade exotique}

On peut faire entrer dans celle salade tout ou partie des substances
suivantes : patates, ignames, topinambours, crosnes, oxalis, artocarpes,
chou-palmiste, avocats, gombos, noix de coco, noix d'Amérique ou du Brésil,
pistaches, le tout au goût.

Pour préparer celle salade, faites cuire, sous la cendre ou au four, patates,
ignames, topinambours et artocarpes ; dans de l'eau salée, crosnes, oxalis et
gombos, en les tenant un peu fermes et en les séchant ensuite à l'entrée d'un
four. Laissez refroidir.

Coupez en morceaux patates, ignames, topinambours, artocarpes et oxalis ;
mettez-les dans un saladier, ajoutez les crosnes, les gombos entiers ou coupés
en deux, des avocats, dont vous aurez enlevé le noyau, coupés en rouelles, de
la noix de coco en languettes, quelques noix d'Amérique et des pistaches
émincées ; assaisonnez avec une mayonnaise un peu fluide faite avec des jaunes
d'œufs, de l'huile de noix fraîchement pressée ou de l'huile d'olive fine, du
sel, du poivre, un peu de vinaigre. du jus de limon des Antilles en quantité
suffisante, et relevée ou non avec des fines herbes. Mélangez bien : laissez en
contact pendant une vingtaine de minutes et servez.

\section*{\centering Salade japonaise.}
\phantomsection
\addcontentsline{toc}{section}{ Salade japonaise.}
\index{Salade japonaise}

La salade dite japonaise, dont Alexandre Dumas fils parle dans « Francillon »,
est tout simplement une salade de moules et de pommes de terre, garnie de
truffes, et assaisonnée à l'huile et au vinaigre.

\medskip

En voici une formule concrète.

\medskip

Pour six personnes prenez :

\footnotesize
\begin{longtable}{rrrrp{16em}}
  &   1 000 & grammes & de & pommes de terre de Hollande,                                                 \\
  &     180 & grammes & d' & huile d'olive légèrement fruitée,                                            \\
  &     100 & grammes & de & vin de Château Yquem,                                                        \\
  &      60 & grammes & de & vinaigre d'Orléans,                                                          \\
  &      20 & grammes & de & fines herbes hachées menu,                                                   \\
  &       2 & grammes & de & poivre,                                                                      \\
  & \multicolumn{2}{r}{3 litres} & de & moules pouvant donner 600 grammes de chair,                       \\
  & \multicolumn{2}{r}{1 litre } & de & bouillon,                                                         \\
  &         &         &    & truffes cuites au champagne, à volonté,                                      \\
  &         &         &    & céleri,                                                                      \\
  &         &         &    & sel.                                                                         \\
\end{longtable}
\normalsize

Pelez les pommes de terre, faites-les cuire dans le bouillon, égouttez-les,
séchez-les, coupez-les en tranches pendant qu'elles sont encore tièdes et
assaisonnez-les avec sel, poivre, vinaigre et vin de Château Yquem.

Nettoyez les moules, mettez-les sur le feu dans une casserole avec un peu d'eau
salée et de céleri, laissez-les s'ouvrir, puis retirez-les des coquilles et
ajoutez-les aux pommes de terre ainsi que l'huile et les fines herbes. Mélangez
bien en remuant légèrement ; couvrez ensuite avec des tranches de truffes et
laissez refroidir lentement, sans mettre à la glace.

\sk

Pour donner à celle salade un cachet japonais, on y fera entrer des crosnes et
aussi des pétales de fleurs de chrysanthème jaune blanchis dans de l’eau salée
et acidulée, qu'on sèchera dans un linge et qu'on incorporera à la salade au
dernier moment.

\section*{\centering Salade russe, à la française.}
\phantomsection
\addcontentsline{toc}{section}{ Salade russe, à la française.}
\index{Salade russe, à la française}

La salade connue en France sous le nom de « Salade russe » n'a de russe que le
nom. C'est une salade de légumes variés, tels que carottes, fonds d'artichauts,
haricots verts, navets, petits pois, pointes d'asperges, pommes de terre,
petits bouquets de chou-fleur, auxquels on ajoute des truffes, des filets de
volaille, de la langue à l'écarlate, du jambon, parfois aussi de la langouste
et du saumon fumé, le tout assaisonné avec de la mayonnaise ou de la sauce
verte.

\section*{\centering Salade russe.}
\phantomsection
\addcontentsline{toc}{section}{ Salade russe.}
\index{Salade russe}

Pour préparer cette salade, prenez tout ou partie des éléments indiqués dans la
formule précédente et, en plus, comme légumes : des concombres saumurés, de la
betterave au raifort, du chou rouge et des champignons russes confits ; comme
viandes : des filets de gélinotte, de la langue de renne, du jambon d'ours ;
comme poissons : du sterlet mariné, de l’esturgeon fumé et du caviar, toutes
substances qu'on trouve aujourd'hui aisément à Paris et qui donnent à la salade
un cachet véritablement en rapport avec son nom.

Décorez cette salade avec des truffes cuites au champagne et des œufs durs
coupés en deux, dans lesquels le jaune sera remplacé par du caviar frais.

\section*{\centering Salade belge.}
\phantomsection
\addcontentsline{toc}{section}{ Salade belge.}
\index{Salade belge}

Pour six à huit personnes prenez :

\footnotesize
\begin{longtable}{rrrp{16em}}
    500 & grammes & de & pommes de terre de Hollande,                                                     \\
    300 & grammes & de & céleri-rave épluché,                                                             \\
    250 & grammes & de & betterave cuite,                                                                 \\
    125 & grammes & de & truffes noires du Périgord,                                                      \\
        &         &  4 & œufs durs,                                                                       \\
        &         &  1 & lapereau de garenne,                                                             \\
        &         &  1 & abatis de volaille,                                                              \\
        &         &  1 & piment Maille,                                                                   \\
        &         &    & mayonnaise au citron et à la moutarde,                                           \\
        &         &    & sauce tomate épaisse,                                                            \\
        &         &    & légumes,                                                                         \\
        &         &    & champignons,                                                                     \\
        &         &    & madère,                                                                          \\
        &         &    & vin blanc,                                                                       \\
        &         &    & fine champagne,                                                                  \\
        &         &    & sel et poivre.                                                                   \\
\end{longtable}
\normalsize

Dépouillez le lapereau, videz-le, mettez de côté le râble.

Faites cuire, à bouilli perdu, dans suffisamment de vin blanc et d'eau, en
parties égales, le reste du lapereau avec l'abatis de volaille, des légumes,
des champignons, le piment, du sel et du poivre. Passez le bouillon et faites
cuire dedans le râble du lapereau. Laissez-le refroidir dans la cuisson :
désossez-le ensuite et émincez la chair en languettes.

Coupez le céleri-rave en petits bâtonnets ; blanchissez-le pendant quelques
minutes dans de l’eau salée bouillante ; laissez-le refroidir.

Faites cuire les pommes de terre à la vapeur, pelez-les, coupez-les en tranches ;
mettez-les, avec le céleri-rave, à mariner pendant quelques heures dans le jus de
cuisson du lapereau qu'ils absorberont.

Nettoyez et pelez les truffes ; mettez pelures et truffes pelées dans du
madère ; laissez cuire ; coupez les truffes en rondelles ; hachez les pelures.

Séparez les blancs d'œufs des jaunes ; coupez les blancs en lamelles ; passez
les jaunes au tamis.

Coupez la betterave en tranches.

Mettez dans un saladier les pommes de terre, le céleri-rave, les émincés de
lapereau, le blanc des œufs, les pelures de truffes, la betterave, un peu de
fine champagne et de la mayonnaise ; mélangez, puis masquez la salade avec de
la mayonnaise et décorez-la avec les rondelles de truffes, les jaunes d'œufs
passés au tamis et de la sauce tomate, dont les couleurs figureront celles du
drapeau belge.

Cette salade a été créée en l'honneur de nos valeureux voisins et alliés, en
souvenir de leur admirable conduite pendant la guerre.

\section*{\centering Salade de chou-palmiste.}
\phantomsection
\addcontentsline{toc}{section}{ Salade de chou-palmiste.}
\index{Salade de chou-palmiste}

Pour six personnes prenez une boîte de chou-palmiste en conserve, de
préparation récente, ouvrez-la, égouttez l'eau, retirez le chou qui se présente
sous la forme d’un cylindre constitué par des feuilles enroulées les unes sur
les autres ; séparez-les soigneusement sans les briser et mettez-les dans un
saladier.

Préparez une mayonnaise fluide aux pointes d'asperges avec laquelle vous
assaisonnerez le chou au moment de servir.

Cette salade, à goût de noisette, dans laquelle, à la vérité, la plante a perdu
une partie de son goût par suite de la cuisson prolongée faite dans le but de sa
conservation, est très intéressante.

\sk

On peut y ajouter des feuilles blanches de cœurs de laitues ; cette addition
lui donnera de la fraîcheur.

\sk

Dans les pays chauds, avec du chou-palmiste frais, cette salade est tout à fait
remarquable ; on y trouve la finesse du bourgeon dans toute sa pureté. Mais, le
plus souvent alors, on est conduit à supprimer de l'assaisonnement les pointes
d'asperges fraîches qui y font défaut. Ce n'est que dans quelques rares
localités privilégiées, où le climat permet la culture en pleine terre des
fruits et des légumes des pays tempérés aussi bien que ceux des pays tropicaux,
qu'on peut goûter dans toute sa perfection l'inoubliable saveur de la salade de
chou-palmiste à la sauce aux pointes. J'ai eu ce plaisir à Park Lodge,
charmante villégiature située sur une hauteur au-dessus de Kingston, à la
Jamaïque, qui serait le paradis si l’on y trouvait des truffes. C'est, dans sa
simplicité, la plus distinguée des salades et, si je ne craignais d'être taxé
d'exagération, je dirais que, pour les amateurs, elle vaut le voyage.

\section*{\centering Salade sans nom.}
\phantomsection
\addcontentsline{toc}{section}{ Salade sans nom.}
\index{Salade sans nom}

Pour dix à douze personnes prenez :

\footnotesize
\begin{longtable}{rrrp{16em}}
    300 & grammes & de & pommes de terre,                                                                 \\
    300 & grammes & de & céleri-rave épluché,                                                             \\
    250 & grammes & de & crevettes grises,                                                                \\
    200 & grammes & d' & olives vertes,                                                                   \\
    200 & grammes & de & jambon de Bayonne,                                                               \\
    150 & grammes & de & saucisson de foie gras en six rondelles,                                         \\
    100 & grammes & de & truffes,                                                                         \\
    100 & grammes & de & vin blanc,                                                                       \\
     60 & grammes & de & saucisson de Lyon en tranches minces,                                            \\
        &         &  3 & laitues,                                                                         \\
        &         &    & madère,                                                                          \\
        &         &    & huile d'olive,                                                                   \\
        &         &    & vinaigre,                                                                        \\
        &         &    & sel et poivre.                                                                   \\
\end{longtable}
\normalsize

Décortiquez les crevettes.

Faites cuire les truffes et les olives dans du madère ; partagez les olives en
deux ; enlevez-en les noyaux. Coupez les truffes en rondelles.

Émincez le céleri en julienne, ébouillantez-le.

Épluchez les laitues, lavez-les, réservez les cœurs.

Faites cuire les pommes de terre à la vapeur, pelez-les, coupez-les en tranches
et assaisonnez-les, pendant qu'elles sont encore un peu chaudes, avec huile,
vinaigre, vin blanc, sel et poivre ; laissez-les refroidir complètement.
Ajoutez-y ensuite les crevettes, le céleri, les truffes. les olives, le
saucisson de Lyon, la moitié du jambon coupé en petits morceaux, les feuilles
de laitue ; mélangez ; goûtez et complétez l'assaisonnement, s'il y a lieu,
avec sel, poivre, huile et vinaigre.

Au dernier moment, décorez la salade avec le reste du jambon coupé en lamelles,
les rondelles de saucisson de foie gras et les cœurs de laitue coupés en
quatre, puis servez.

\section*{\centering Salade méli-mélo.}
\phantomsection
\addcontentsline{toc}{section}{ Salade méli-mélo.}
\index{Salade méli-mélo}

Pour douze personnes prenez :

\footnotesize
\begin{longtable}{rrrp{16em}}
    500 & grammes & de & crevettes grises,                                                                \\
    500 & grammes & de & lait,                                                                            \\
    330 & grammes & d' & huile d'olive,                                                                   \\
    300 & grammes & de & filets de poularde rôtie,                                                        \\
    250 & grammes & de & jambon d’York,                                                                   \\
     30 & grammes & de & câpres,                                                                          \\
     30 & grammes & de & cornichons coupés en rondelles,                                                  \\
     15 & grammes & de & vinaigre de vin,                                                                 \\
      4 & grammes & de & poivre fraîchement moulu,                                                        \\
      1 & gramme  & de & moutarde,                                                                        \\
        &         & 24 & écrevisses moyennes, œuvées,                                                     \\
        &         &  4 & cœurs de laitues,                                                                \\
        &         &  2 & homards œuvés pesant ensemble 1 800 grammes environ,                             \\
        &         &  2 & jaunes d'œufs frais,                                                             \\
        &         &  1 & petite botte de pointes d'asperges,                                              \\
        &         &    & truffes (ad libitum),                                                            \\
        &         &    & madère,                                                                          \\
        &         &    & jus de citron,                                                                   \\
        &         &    & sel.                                                                             \\
\end{longtable}
\normalsize

Court-bouillonnez les homards pendant une demi-heure comme il est indiqué
\hyperlink{p0284}{p. \pageref{pg0284}}.

Mettez à dégorger les écrevisses pendant une heure dans le lait, arrachez-leur
le boyau et faites-les cuire pendant dix minutes dans le court-bouillon indiqué
\hyperlink{p0287}{p. \pageref{pg0287}}.

Faites cuire ensuite les crevettes dans le même court-bouillon pendant trois
minutes.

Blanchissez pendant cinq minutes, dans de l'eau salée, les pointes d'asperges
coupées en morceaux de deux centimètres de longueur.

Faites cuire les truffes dans du madère, puis coupez-les en rondelles.

Escalopez les queues des homards, épluchez les pattes, réservez les œufs, mettez
à part les parures.

Épluchez les écrevisses ; mettez à part et séparément les queues entières, les
œufs et les parures.

Épluchez les crevettes ; réservez séparément les queues et les parures.

Coupez en languettes le jambon et les filets de poularde.

Mettez dans un saladier les escalopes de homard, les débris des pattes, les
queues d'écrevisses et de crevettes, les émincés de poularde et de jambon, les
pointes d'asperges, les câpres, les cornichons, les truffes, en réservant une
partie de ces éléments pour décorer le plat ; ajoutez {\ppp30\mmm} grammes
d'huile, le vinaigre, mélangez bien et laissez le tout confire pendant douze
heures.

Préparez une mayonnaise à la moutarde avec les jaunes d'œufs, le reste de
l'huile, le poivre, la moutarde, du sel et le jus de la moitié d'un citron.

Passez au tamis les œufs des homards et des écrevisses.

Écrasez à la presse les parures des homards, des écrevisses et des crevettes et
recueillez-en le jus.

Joignez ce jus aux œufs que vous venez de tamiser, incorporez le tout à la
mayonnaise, dont vous réserverez une partie pour décorer le plat.

Au moment de servir, ajoutez les cœurs de laitue et la mayonnaise aux éléments
confits, mêlez bien le tout, puis décorez avec la mayonnaise mise à part et les
éléments réservés.

Cette salade, très savoureuse, est réellement digne de figurer dans le menu d'un
repas soigné.

\medskip

Elle peut également être servie comme hors-d'œuvre.

\section*{\centering Salade Sémonville.}
\phantomsection
\addcontentsline{toc}{section}{ Salade Sémonville.}
\index{Salade Sémonville}

Cette salade a été composée par M. de Sémonville, ami de M. de Talleyrand. Elle
avait la prétention de symboliser l'état dans lequel se trouvait la France
lorsque Bonaparte prit le pouvoir en l'an VIII.

Le principe de cette salade consiste dans la multiplicité des éléments qui la
composent. Les personnes qui se piquent de mathématiques disent qu'on doit la
préparer avec un nombre d'éléments représenté par les puissances croissantes du
nombre {\ppp2\mmm}, en commençant par la {\ppp3\mmm}\textsuperscript{e}, tant
dans le règne végétal que dans le règne animal ; il faudrait dans ces
conditions pour le moins {\ppp8\mmm} éléments végétaux et {\ppp8\mmm} éléments
animaux ; puis viendrait la salade {\ppp4\mmm}\textsuperscript{e} puissance
composée de {\ppp16\mmm} éléments de chaque règne, la salade
{\ppp5\mmm}\textsuperscript{e} puissance : {\ppp32\mmm} éléments, etc. Je ne
crois pas que l'on soit arrivé à composer la salade puissance {\ppp7\mmm}, mais
je conçois la salade puissance {\ppp6\mmm}.

Le règne animal fournit le ris de veau, le museau de bœuf, la langue, les
cervelles, le jambon, le saucisson de Lyon haché (qui est d'une importance
capitale), la volaille, le gibier, le gruyère en lamelles, le parmesan en
poudre, le blanc et le jaune d'œuf dur hachés, les sardines, les anchois, les
filets de harengs, le saumon fumé, les petites truites fumées, le homard, la
langouste, les écrevisses, les crevettes, les coquillages, etc., etc.

Le règne végétal fournit du cru et du cuit.

Comme crudités, les salades vertes, les fines herbes, les fragments de radis
qui sont essentiels parce qu'ils donnent de la fraîcheur, les noix, la pomme
acide qui déroute un instant l'amateur, la truffe, la tomate, les olives, les
cornichons, les câpres, les piments les plus variés, tous les pickles, etc.,
etc.

Comme légumes cuits, cette salade comprend tous ceux des salades de légumes
habituelles ; le fond d’artichaut et les pointes d'asperges sont chargés de
donner à la macédoine le caractère aristocratique que la pomme de terre et
l'oignon risqueraient de compromettre, la betterave donne de la couleur, etc.,
etc.

L'assaisonnement de la salade est constitué par un mélange savant d'huile
d'olive, de vinaigre de vin, de jus de citron, de jaunes d'œufs délayés, de
poivres de toute espèce, de sel et de Worcestershire sauce, condiment anglais
qui est chargé de lui communiquer une saveur ardente quoique douce.

Tout le succès de la salade dépend de l'assaisonnement et de la préparation.

Les différents éléments doivent être mélangés à l'avance, par catégories, puis
ensuite étalés dans le ou les saladiers, que l'on garnira par couches
régulières, en saupoudrant d'œuf dur haché, de parmesan râpé, de poudre de
moutarde et en versant dessus l'assaisonnement.

Si ce plan méthodique n'est pas bien suivi, au lieu de la salade Sémonville, on
n'a guère qu'une macédoine burlesque et indigeste. \textit{Mens agitat molem}.

Cette salade peut d'autre part se décomposer, en quelque sorte par extraction
de racines carrées, en salades plus simples orientées vers le poisson, les
viandes blanches, la charcuterie, le gibier, les légumes crus ou les légumes
cuits.

En réalité, dans l’ordre de la salade, on se trouve là en présence d'une
véritable encyclopédie gastronomique.

\section*{\centering Salades en aspic.}
\phantomsection
\addcontentsline{toc}{section}{ Salades en aspic.}
\index{Salades en aspic}
\index{Aspic de salades}

La plupart des salades composées peuvent être présentées en aspic. Il convient
simplement d'accorder la composition de la gelée avec la note dominante de la
salade. C'est ainsi que les gelées maigres blondes seront employées avec les
salades de poissons ou de crustacés ; les gelées grasses, blondes ou brunes,
conviendront aux salades de viandes blanches ou rouges, de charcuterie ou de
légumes ; enfin les gelées brunes au fumet de gibier accompagneront les salades
au gibier.

\medskip

Les salades en aspic sont des hors-d'œuvre appétissants.

\section*{\centering Gaspacho de l'estramadoure.}
\phantomsection
\addcontentsline{toc}{section}{ Gaspacho de l'estramadoure.}
\index{Gaspacho de l'estramadoure}

Le gaspacho est un plat espagnol.

C'est une sorte de salade de tomates et de concombres, très rafraîchissante
pendant les fortes chaleurs.

On la prépare de différentes façons. Voici la formule employée en Estramadoure.

Pilez dans un mortier de marbre une gousse d'ail, quelques amandes douces, de
la ciboulette hachée fin, un peu de mie de pain rassis tamisée, mouillée et
pressée pour en exprimer tout excès de liquide ; assaisonnez avec sel et poivre
au goût et, lorsque le contenu du mortier paraît bien homogène, versez-y goutte
à goutte de l'huile d'olive. Faites monter l'ensemble comme une mayonnaise,
puis, quand le tout est pris, mettez-le à rafraîchir dans un saladier entouré
de glace. Lorsque la température est suffisamment abaissée, ajoutez de la mie
de pain rassis tamisée, des tomates mûres, des concombres blancs pelés,
épépinés et coupés en cubes, enfin relevez le tout avec des piments verts
d'Espagne, épépinés et hachés, et du vinaigre à l’estragon.
