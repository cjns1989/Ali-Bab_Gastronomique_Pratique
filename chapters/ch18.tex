\section*{\centering Grillade de filet de bœuf.}
\phantomsection
\addcontentsline{toc}{section}{ Grillade de filet de bœuf.}
\index{Grillade de filet de bœuf}
\index{Filet de bœuf (Grillade de)}
\index{Filet de bœuf grillé, au beurre Bercy}
\index{Filet de bœuf grillé, au beurre d'anchois}
\index{Filet de bœuf grillé, au beurre marchand de vin}
\index{Filet de bœuf grillé, au beurre maître-d'hôtel}

La grillade est le procédé de cuisson qui conserve le mieux aux viandes rouges
leur goût propre.

Pour quatre personnes prenez une belle tranche de filet de bœuf épaisse de
{\ppp4\mmm} à {\ppp5\mmm} centimètres au moins et pesant {\ppp1\mmm}
kilogramme environ.

Pour la cuisson, servez-vous d'un gril à feu dessus et chauffez-le au préalable
pour éviter l’adhérence de la viande, ou prenez un appareil combiné pour
retenir la graisse.

Mettez le filet sur le gril ; saisissez-le à feu vif, puis continuez la cuisson
pendant {\ppp8\mmm} à {\ppp10\mmm} minutes à feu plus doux ; retournez-le et
opérez de la même manière pour l'autre côté. Un peu avant la fin, salez et
poivrez au goût. La cuisson est à point lorsque le sang sort à la surface de la
viande.

La grillade de filet de bœuf peut être servie telle quelle, ou avec un beurre
maître d'hôtel, un beurre marchand de vin\footnote{On prépare le beurre
marchand de vin de la façon suivante,
\index{Beurre marchand de vin}


\label{pg0453} \hypertarget{p0453}{}
\protect\endgraf
Pour quatre personnes prenez :
\protect\endgraf
\smallskip
\begin{longtable}{rrrp{16em}}
    250 & grammes & de & vin rouge,                                                                       \\
    200 & grammes & de & beurre,                                                                          \\
     30 & grammes & d' & échalotes,                                                                       \\
     30 & grammes & de & glace de viande,                                                                 \\
        &         &    & bouillon,                                                                        \\
        &         &    & jus de citron,                                                                   \\
        &         &    & persil haché,                                                                    \\
        &         &    & sel et poivre.                                                                   \\
\end{longtable}
\protect\endgraf

Hachez les échalotes, mettez-les dans le vin ; faites cuire ; ajoutez ensuite
la glace de viande dissoute dans un peu de bouillon ; salez, poivrez ;
concentrez la cuisson ; montez-la avec le beurre ; relevez-la avec du jus de
citron ; saupoudrez d'un peu de persil haché.}, un beurre Bercy\footnote{
\index{Beurre Bercy}
On prépare d'une manière analogue le beurre Bercy, en remplaçant dans la
formule précédente le vin rouge par du vin blanc et la glace de viande par de
la moelle de bœuf pochée et coupée en tranches.}, un beurre d’anchois, de
l'huile d'anchois\footnote{On obtient l'huile d'anchois en pilant des anchois
dans un mortier, y incorporant de la bonne huile d'olive et passant le tout au
tamis.}, une sauce béarnaise ordinaire, une sauce Choron\footnote{La sauce
Choron est une béarnaise à la tomate.}, une sauce Foyot\footnote{La sauce Foyot
est une béarnaise à la glace de viande.}.

Les garnitures les meilleures avec la grillade de filet de bœuf sont les pommes
de terre sautées, les pommes de terre Anna, les pommes de terre Champs-Élysées,
les pommes de terre paille, les pommes de terre Chip, les pommes de terre
soufflées, etc.

\sk

\index{Entrecôte (Grillade d')}
\index{Entrecôte grillé, au beurre d'anchois}
\index{Entrecôte grillé, au beurre marchand de vin}
\index{Entrecôte grillé, sauce Choron}
\index{Entrecôte grillé, sauce aux huîtres}
\index{Entrecôte grillé, sauce béarnaise}
\index{Entrecôte grillé, à l'huile d'anchois}
\index{Entrecôte grillé, sauce Foyot}
\index{Eutrecôte grillé, au beurre Bercy}
\index{Faux filet (Grillade de)}
\index{Faux filet grillé, au beurre maître-d'hôtel}
\index{Faux filet grillé, au beurre Bercy}
\index{Faux filet grillé, au beurre d'anchois}
\index{Faux filet grillé, au beurre marchand de vin}
\index{Faux filet grillé, sauce Choron}
\index{Faux filet grillé, sauce Foyot}
\index{Faux filet grillé, sauce aux huîtres}
\index{Faux filet grillé, sauce béarnaise}
\index{Faux filet grillé, à l'huile d'anchois}
\index{Filet de bœuf grillé, sauce Choron}
\index{Filet de bœuf grillé, sauce Fovot}
\index{Filet de bœuf grillé, sauce béarnaise}
\index{Filet de bœuf grillé, à l'huile d'anchois}
Le faux filet et l'entrecôte peuvent être préparés de même.

\section*{\centering Filet Boston.}
\phantomsection
\addcontentsline{toc}{section}{ Filet Boston.}
\index{Filet Boston}

Pour six personnes prenez une tranche de filet de bœuf épaisse de {\ppp4\mmm}
à {\ppp5\mmm} centimètres, pesant {\ppp1\mmm} kilogramme environ, et
{\ppp36\mmm} belles huîtres.

Faites cuire le filet comme dans la formule précédente.

En même temps. préparez une sauce aux huîtres.

Ouvrez les huîtres, recueillez-les avec leur eau dans une casserole ;
mettez-les sur le feu ; donnez un bouillon, écumez ; enlevez les mollusques ;
tenez-les au chaud ; concentrez leur eau.

Préparez une sauce hollandaise ou une sauce béarnaise, ou une sauce allemande ;
incorporez-y l’eau réduite des huîtres ; goûtez et complétez l'assaisonnement
avec jus de citron, sel et poivre, au goût.

Foncez avec cette sauce un plat de service ; disposez dessus le filet grillé ;
entourez-le avec les huîtres cuites ; garnissez le plat avec des pommes de
terre soufflées et servez.

\sk

\index{Entrecôte Boston}
\index{Faux filet Boston}
On peut apprêter de même le faux filet et l'entrecôte.

\sk

\index{Barbue court-bouillonnée, sauce aux huîtres}
\index{Langouste court-bouillonnée, sauce aux huîtres}
\index{Langoustines court-bouillonnées, sauce aux huîtres}
\index{Turbot court-bouillonné, sauce aux huîtres}
\index{Cigales de mer grillées, saucé aux huîtres}
La sauce aux huîtres accompagne très bien les poissons court-bouillonnés,
notamment le turbot et la barbue, les langoustes, les langoustines et les
cigales de mer grillées ; mais il est bon, dans ces cas, d'employer un velouté
maigre comme base de la sauce,

\section*{\centering Entrecôte grillé, sauce à la moelle\footnote{La sauce à la
moelle classique, dite sauce bordelaise, est préparée de la façon suivante.
\protect\endgraf
Pour six personnes, prenez :
\protect\endgraf
% \medskip
\begin{longtable}{rrrp{16em}}
    350 & grammes & de & vin de Bordeaux rouge,                                                           \\
    135 & grammes & de & sauce espagnole,                                                                 \\
    100 & grammes & de & moelle de bœuf,                                                                  \\
     30 & grammes & de & glace de viande,                                                                 \\
     30 & grammes & d' & échalotes,                                                                       \\
      5 & grammes & de & jus de citron,                                                                   \\
        &         &    & laurier,                                                                         \\
        &         &    & thym,                                                                            \\
        &         &    & poivre mignonnette.                                                              \\
\end{longtable}
\protect\endgraf
Mettez dans le vin les échalotes hachées, du laurier, du thym, du poivre mignonnette au goût, concentrez
au quart ; ajoutez ensuite la sauce espagnole, dépouillez pendant une vingtaine de minutes, puis passez à
l'étamine.
\protect\endgraf
Finissez la sauce avec la glace de viande, le jus de citron et la moelle pochée coupée en dés.}.}

\phantomsection
\addcontentsline{toc}{section}{ Entrecôte grillé, sauce à la moelle.}
\index{Entrecôte grillé, sauce à la moelle}
\index{Entrecôte grillé, au beurre maitre-d'hôtel}
\index{Entrecôte grillé, sauce bordelaise}

Pour six personnes prenez :

\medskip

\footnotesize
\begin{longtable}{rrrp{16em}}
  1 200 & grammes & d' & entrecôte persillé de graisse, en une tranche de 4 centimètres
                         d'épaisseur,                                                                     \\
    200 & grammes & de & vin de Bordeaux rouge,                                                           \\
    100 & grammes & de & moelle de bœuf,                                                                  \\
    100 & grammes & de & jus de viande ou 75 grammes de glace de viande dissoute dans du
                         bouillon,                                                                        \\
     60 & grammes & de & beurre,                                                                          \\
     15 & grammes & d’ & échalote ciselée\footnote{On peut mettre plus ou moins d'échalote
                                  et d'ail ; les proportions indiquées sont des proportions
                                  moyennes.},                                                             \\
      5 & grammes & d' & essence d'anchois ou, à défaut, 1 anchois dessalé, pilé avec du
                         beurre et passé au tamis,                                                        \\
      5 & grammes & de & jus de citron,                                                                   \\
      2 & grammes & d' & ail,                                                                             \\
        &     1/4 & d' & une feuille de laurier,                                                          \\
        &         &    & farine,                                                                          \\
        &         &    & huile d'olive,                                                                   \\
        &         &    & persil haché,                                                                    \\
        &         &    & sel, poivre blanc,                                                               \\
        &         &    & cayenne.                                                                         \\
\end{longtable}
\normalsize

Faites dégorger la moelle pendant quelques heures dans de l'eau fraîche, que
vous changerez plusieurs fois ; puis, faites-la pocher pendant une demi-heure
dans de l’eau salée ; tenez-la au chaud.

Préparez la sauce. Mettez l'ail, l'échalote et le laurier dans le vin ;
réduisez le volume du liquide au quart ; passez ; ajoutez le jus de viande, le
jus de citron, l'essence d'anchois, du sel, du poivre, du cayenne, au goût.
Maniez le beurre avec suffisamment de farine ; incorporez-le à la sauce ;
laissez cuire pendant un instant. Goûtez, ajoutez encore un peu de cayenne pour
donner du montant et tenez au chaud.

Passez l'entrecôte dans de l'huile d'olive ; mettez-le sur un gril chauffé au
préalable, saisissez-le et continuez la cuisson pendant {\ppp7\mmm}
à {\ppp8\mmm} minutes ; retournez-le et opérez de même pour l'autre côté.

Découpez-le en tranches que vous dresserez sur un plat tenu au chaud.

Coupez la moelle en rondelles avec un couteau trempé dans de l’eau chaude,
disposez-les sur les tranches de viande, saupoudrez avec du persil haché, si
vous l'aimez, masquez avec la sauce et servez.

Les trois opérations : blanchir la moelle, préparer la sauce et faire griller
l'entrecôte sont menées plus ou moins simultanément, mais la viande ne doit pas
attendre : il importe donc que les deux autres opérations soient achevées
lorsque l’entrecôte est cuit.

Les légumes qui accompagnent le mieux l'entrecôte grillé sont les pommes de
terre soufflées ou les pommes de terre Chip.

\section*{\centering Filet de bœuf truffé, sauce demi-glace.}
\phantomsection
\addcontentsline{toc}{section}{ Filet de bœuf truffé, sauce demi-glace.}
\index{Filet de bœuf truffé, sauce demi-glace}
\index{Bœuf truffé, sauce demi-glace}

Piquez du filet de bœuf paré avec du lard gras et des languettes de truffes.
Ficelez-le, puis faites-le rôtir à la broche de manière à conserver l'intérieur
rosé.

Dressez-le sur un plat que vous garnirez avec de petites escalopes de foie
gras, ou médaillons, cuit au naturel, des rognons de coq sautés au beurre et
des pommes de terre Champs-Élysées truffées.

Servez avec une sauce demi-glace au porto.

\sk

\label{pg0456} \hypertarget{p0456}{}
La sauce demi-glace est de la sauce espagnole très fine additionnée, hors du
feu, de {\ppp100\mmm} grammes de madère ou de porto par litre de sauce.

\sk

On prépare l’espagnole de la façon suivante.

\medskip

Pour un litre d'espagnole prenez :

\medskip

\footnotesize
\begin{longtable}{rrrp{16em}}
    100 & grammes & de & lard de poitrine ou de jambon salé, non fumé,                                    \\
     75 & grammes & de & farine,                                                                          \\
     60 & grammes & de & beurre clarifié,                                                                 \\
     50 & grammes & de & carottes,                                                                        \\
     50 & grammes & de & vin blanc,                                                                       \\
     30 & grammes & d' & oignons,                                                                         \\
  2  & litres 1/2 & de & fond brun, \hyperlink{p0203}{p. \pageref{pg0203}},                               \\
        &         &    & thym,                                                                            \\
        &         &    & laurier.                                                                         \\
\end{longtable}
\normalsize

Préparez un mirepoix\footnote{
\index{Définition du mirepoix}
\index{Mirepoix (Définition du)}
On désigne sous le nom de mirepoix, un appareil
aromatisé, le plus souvent à base de jambon, de veau et de légumes, imaginé par
le cuisinier du maréchal de Mirepoix, et qui sert à corser les sauces. Les
personnes qui considèrent le mirepoix comme une essence font le mot féminin.} :
faites revemir le lard ou le jambon coupé en petits morceaux, mettez carottes,
oignons, thym et laurier au goût ; laissez prendre couleur ; égouttez la
graisse ; mouillez avec le vin ; réduisez de moitié.

Faites dorer lentement la farine dans le beurre de façon à obtenir un roux
brillant et homogène ; mouillez en remuant avec un litre et demi de fond brun,
amenez à ébullition, ajoutez l'appareil mirepoix et laissez cuire doucement
pendant trois heures, en dépouillant fréquemment la sauce.

Passez au chinois en pressant un peu ; mouillez de nouveau avec {\ppp400\mmm} grammes de
fond ; laissez bouillir à petit feu pendant trois heures encore, puis
refroidissez la sauce en la soulevant constamment, c'est-à-dire en vannant.

Le lendemain, remettez la sauce dans une casserole avec le reste du fond et 400
grammes de tomates fraîches concassées ; faites cuire en fouettant et en
dépouillant encore la sauce jusqu'à réduction au volume d'un litre.

Passez et refrordissez la sauce en la vannant de nouveau.

\sk

On prépare l'espagnole maigre d'une façon analogue, en remplaçant le fond brun
par du fond de poisson, le lard par du beurre et en ajoutant des champignons.

\section*{\centering Contrefilet grillé ou rôti.}
\phantomsection
\addcontentsline{toc}{section}{ Contrefilet grillé ou rôti.}
\index{Contrefilet grillé}
\index{Contrefilet rôti}

Piquez le contrefilet avec de fins lardons de lard gras et de langue ou de lard
et de truffe, assaisonnez-le, faites-le griller ou rôtir à la broche en tenant
l'intérieur rosé.

Dressez-le sur un plat tel quel ou glacez-le.

Servez à part le jus de cuisson dégraissé ou un bon jus lié, ou encore une des
sauces usuelles.

Toutes les garnitures de légumes ou de pâtes accompagnent très bien le
contrefilet grillé ou rôti.

\section*{\centering Aloyau.}
\phantomsection
\addcontentsline{toc}{section}{ Aloyau.}
\index{Aloyau}

L'aloyau est la partie du bœuf située entre la hanche et les premières côtes ;
il comprend le filet et le contrefilet.

Chez les anciens, on le servait entier, non désossé. Aujourd'hui, on le désosse,
le plus souvent, et on le roule.

L'aloyau est la grosse pièce de choix pour rôti ou pour braisé dans les repas
où les convives sont nombreux. C'est l'élément du classique roast-beef anglais,
que le roi Henri VIII, grand amateur de viande, a anobli en le faisant
\textit{Sir}\footnote{C'est-à-dire baron de bœuf ; d'où le nom de baron de
mouton et d'agneau donné aux pièces composées de la selle et des deux
gigots.} : \textit{Sir loin of beef}\footnote{On écrit aujourd’hui Sirloin of
beef.}, à l'instar de Caligula qui avait élevé son cheval favori au rang de
chevalier,

Les garnitures et les sauces les plus variées conviennent comme accompagnement
de l'aloyau. En voici quelques-unes.

\medskip

\index{Garniture à la Clamart}
\textit{Garniture à la Clamart} : barquettes de petits pois et de laitue au jus
lié :

\medskip

\index{Garniture à la du Barry}
\textit{Garniture à la du Barry} : croustades de crème de chou-fleur et de
pommes de terre montée au beurre, gratinées ou non : sauce demi-glace ;

\medskip

\index{Garniture à la forestière}
\textit{Garniture à la forestière} : morilles sautées avec petits cubes de
lard ; sauce à base de d'Uxel et de jus du rôti ;

\medskip

\index{Garniture à la Godard}
\textit{Garniture à la Godard} : salpicon de fonds d’artichauts, de ris de
veau, de champignons et de quenelles ; sauce demi-glace ;

\medskip

\index{Garniture Lucullus}
\textit{Garniture Lucullus} : grosses truffes au naturel ou farcies de foie
gras ; sauce madère ;

\medskip

\index{Garniture à l'algérienne}
\textit{Garniture à l'algérienne} : croquettes de patates et de tomates ; sauce tomate
relevée par des émincés de piment ;

\medskip

\index{Garniture à l'anglaise}
\textit{Garniture à l'anglaise} : yorkshire pudding ou pommes de terre
à l'eau ; sauce au raifort ;

\medskip

\index{Garniture à l'andalouse}
\textit{Garniture à l'andalouse} : poivrons grillés garnis de riz et aubergines
frites ; sauce tomate au jus ;

\medskip

\index{Garniture à la napolitaine}
\textit{Garniture à la napolitaine} : salpicon de spaghetti, de jambon, de
truffes du Piémont, avec parmesan râpé ; sauce tomate au jus ;

\medskip

\index{Garniture à la polonaise} 
\textit{Garniture à la polonaise} : cèpes farcis grillés ; sauce demi-glace
additionnée légèrement de purée Soubise.

\index{Aloyau rôti, financière}
\section*{\centering Aloyau rôti, financière.}
\phantomsection
\addcontentsline{toc}{section}{ Aloyau rôti, financière.}
\index{Aloyau rôti, financière}

Pour {\ppp12\mmm} personnes prenez :

\medskip

1° une tranche d'aloyau désossée, parée et roulée, prise dans le milieu et
pesant {\ppp4\mmm} kilogrammes environ ;

\medskip

\index{Garniture financiere}
2° pour la garniture :

\medskip

\footnotesize
\begin{longtable}{rrrp{16em}}
    300 & grammes & de & petits champignons cannelés,                                                     \\
    250 & grammes & de & rognons de coq,                                                                  \\
    125 & grammes & de & truffes cuites dans du madère et émincées,                                       \\
     60 & grammes & de & beurre,                                                                          \\
        &         & 60 & petites quenelles de veau,                                                       \\
        &         & 30 & olives verdales tournées et blanchies ;                                          \\
\end{longtable}
\normalsize

3° pour la sauce madère\footnote{D'une façon générale, la sauce madère peut
être préparée à base de consommé ou à base de fond de veau, suivant la nature
de la viande qu'elle doit accompagner.} :

\medskip

\footnotesize
\begin{longtable}{rrrp{16em}}
  1 000 & grammes & de & bon consommé,                                                                    \\
    400 & grammes & de & madère,                                                                          \\
     60 & grammes & de & beurre,                                                                          \\
     40 & grammes & de & glace de viande,                                                                 \\
     40 & grammes & de & farine,                                                                          \\
        &         &    & poivre.                                                                          \\
\end{longtable}
\normalsize

\label{pg0459} \hypertarget{p0459}{}
Préparez d'abord la sauce. Mettez la glace de viande dans le madère, poivrez au
goût ; chauffez ; réduisez de moitié.

Faites un roux avec le beurre et la farine, mouillez avec le consommé, ajoutez
le madère réduit et concentrez la sauce, en la dépouillant, jusqu'à ce qu'elle ait
une consistance suffisante pour masquer une cuiller.

Enveloppez l'aloyau avec du papier beurré, ficelez, faites rôtir à la broche en
commençant à feu vif, puis continuez la cuisson à feu plus modéré, de façon que
l'intérieur de la viande reste rose.

Au dernier moment, enlevez le papier, faites prendre couleur au rôti.

En même temps, faites pocher les quenelles dans de l’eau salée, les rognons de
coq dans du consommé et cuisez les champignons dans le beurre avec un peu de
jus de citron.

Mettez quenelles, rognons de coq, olives, champignons, émincés de truffe et le
jus de cuisson de l'aloyau dégraissé dans la sauce madère et achevez la cuisson
de l'ensemble.

Dressez l'aloyau sur un plat, garnissez avec le salpicon financière et servez.

\section*{\centering Paupiettes de bœuf rôties à la broche.}
\phantomsection
\addcontentsline{toc}{section}{ Paupiettes de bœuf rôties à la broche.}
\index{Paupiettes de bœuf rôties à la broche}

Pour trois personnes prenez :

\medskip

\footnotesize
\begin{longtable}{rrrp{16em}}
    500 & grammes & de & faux filet paré, coupé en trois tranches régulières,                             \\
    125 & grammes & de & champignons de couche,                                                           \\
     50 & grammes & de & truffes noires,                                                                  \\
     50 & grammes & de & vin blanc,                                                                       \\
     50 & grammes & de & beurre,                                                                          \\
     10 & grammes & d' & oignon épluché et haché,                                                         \\
      5 & grammes & d' & huile d'olive,                                                                   \\
        &         &  3 & bardes de lard,                                                                  \\
        &         &  1 & jaune d'œuf cru,                                                                 \\
        &         &    & jus de citron,                                                                   \\
        &         &    & poivre fraîchement moulu,                                                        \\
        &         &    & sel.                                                                             \\
\end{longtable}
\normalsize

Faites cuire l'oignon dans le beurre sans lui laisser prendre couleur ;
réservez la cuisson.

Épluchez les champignons, passez-les dans du jus de citron, coupez-les en
morceaux et faites-les cuire dans la cuisson de l'oignon réservée.

Pelez les truffes et coupez-les en morceaux.

Mélangez oignon, champignons et truffes, assaisonnez avec sel, poivre et liez
avec le jaune d'œuf.

Disposez cette farce sur les tranches de faux filet assaisonnées elles-mêmes,
roulez-les en paupiettes, bardez les paupiettes de lard, ficelez-les et
mettez-les à mariner pendant six heures dans le vin mélangé avec l'huile et
assaisonné avec du poivre.

Enfin, faites rôtir les paupiettes à la broche, à feu assez vif, en arrosant
avec la marinade ; une vingtaine de minutes de cuisson suffit généralement.
Enlevez les ficelles et ce qui reste des bardes, puis servez, en envoyant en
même temps un légumier de pommes de terre sautées.

\section*{\centering Rump-steak\footnote{ou romsteck, par altération du mot
anglais.} à la poêle, au riz.}
\phantomsection
\addcontentsline{toc}{section}{ Rump-steak à la poêle, au riz.}
\index{Rump-steak à la poêle, au riz}
\index{Bœuf à la poêle}

Pour quatre personnes prenez :

\medskip

\footnotesize
\begin{longtable}{rrrrp{16em}}
  & 800 & grammes & de & rump-steak, en une tranche de 2 centimètres d'épaisseur,                         \\
  & 300 & grammes & de & riz,                                                                             \\
  & 150 & grammes & de & beurre,                                                                          \\
  &   5 & grammes & de & sel,                                                                             \\
2 & \multicolumn{2}{r}{décigrammes}  & de & poivre,                                                       \\
  &     &         &    & persil haché.                                                                    \\
\end{longtable}
\normalsize

Faites cuire le riz sec, comme il est dit \hyperlink{p0707}{p. \pageref{pg0707}} ;
tenez-le au chaud.

Foncez une poêle avec {\ppp20\mmm} grammes de beurre ; chauffez, puis mettez la viande ;
laissez la cuire pendant six minutes de chaque côté ; salez et poivrez avant la
fin de la cuisson.

Faites sauter, pendant un instant, dans une casserole, le riz sec avec 110
grammes de beurre, sans le laisser dorer.

Dressez la viande sur un plat chaud ; maniez le reste du beurre avec du persil
haché, laissez-le fondre, versez-le sur la viande et servez, en envoyant à part
le riz dans un légumier.

\section*{\centering Filets mignons, au madère, avec pommes de terre à la crème.}
\phantomsection
\addcontentsline{toc}{section}{ Filets mignons, au madère, avec pommes de terre à la crème.}
\index{Filets mignons, au madère, avec pommes de terre à la crème}

Pour six personnes prenez :

\medskip

\footnotesize
\begin{longtable}{rrrp{16em}}
  1 000 & grammes & de & pommes de terre,                                                                 \\
    400 & grammes & de & crème,                                                                           \\
    200 & grammes & de & madère,                                                                          \\
    125 & grammes & de & beurre,                                                                          \\
    125 & grammes & de & consommé,                                                                        \\
    125 & grammes & de & champignons de couche,                                                           \\
     20 & grammes & de & farine,                                                                          \\
        &         &  6 & filets mignons,                                                                  \\
        &         &  1 & truffe noire du Périgord,                                                        \\
        &         &    & muscade,                                                                         \\
        &         &    & sel et poivre.                                                                   \\
\end{longtable}
\normalsize

Épluchez les champignons.

Nettoyez la truffe ; coupez-la en tranches minces.

Prenez une casserole de dimensions convenables, mettez dedans {\ppp80\mmm} grammes de
beurre et la farine, faites roussir, mouillez avec le madère et le consommé ;
laissez cuire à petit feu pendant une demi-heure en tournant de temps en temps,
puis ajoutez champignons et truffes. Continuez la cuisson encore pendant une
demi-heure, toujours en tournant.

En même temps, faites cuire dans de l’eau salée les pommes de terre en robe
de chambre, pelez-les, coupez-les en morceaux si elles sont grosses, laissez-les
entières si elles sont petites, puis mettez-les dans une casserole avec la crème,
du sel, du poivre et de la muscade, au goût. Laissez mijoter à tout petit feu
pendant un quart d'heure.

Chauffez le reste du beurre dans une poêle, mettez dedans les filets mignons
que vous laisserez cuire pendant cinq minutes de chaque côté, puis dressez-les
sur un plat, masquez-les avec la sauce au madère et servez.

Envoyez, en même temps, les pommes de terre dans un légumier.

C'est un plat excellent, qui a généralement beaucoup de succès.

\sk

\index{Filets mignons sur canapés garnis de foie gras}
Comme variante on pourra dresser les filets mignons sur des canapés de pain
dorés dans du beurre et garnis de foie gras.

\section*{\centering Filets mignons sur canapés, aux champignons farcis de bacon.}
\phantomsection
\addcontentsline{toc}{section}{ Filets mignons sur canapés, aux champignons farcis de bacon.}
\index{Filets mignons sur canapés, aux champignons farcis de bacon}
\index{Filets mignons sur canapés, aux cépes, aux morilles ou aux truffes, farcis de jambon}
\index{Champignons farcis de bacon}

Pour quatre personnes prenez :

\medskip

\footnotesize
\begin{longtable}{rrrp{16em}}
    500 & grammes & de & fond de veau,                                                                    \\
    250 & grammes & de & bacon,                                                                           \\
    150 & grammes & de & beurre,                                                                          \\
    100 & grammes & de & glace de viande,                                                                 \\
     50 & grammes & de & porto,                                                                           \\
     15 & grammes & de & farine,                                                                          \\
        &         &  4 & filets mignons,                                                                  \\
        &         &  4 & gros champignons de couche ou 8 moyens,                                          \\
        &         &  2 & jaunes d'œufs frais,                                                             \\
        &         &  1 & bouquet garni,                                                                   \\
        &         &    & pain de mie anglais,                                                             \\
        &         &    & jus de citron,                                                                   \\
        &         &    & sel et poivre.                                                                   \\
\end{longtable}
\normalsize

Coupez du pain anglais en quatre tranches de même surface que les filets
mignons et de {\ppp1\mmm}/{\ppp2\mmm} centimètre d'épaisseur environ.

Pelez les champignons, passez-les dans du jus de citron ; hachez les queues ;
réservez les chapeaux.

Faites un roux avec {\ppp25\mmm} grammes de beurre et la farine, mouillez avec
le porto et le fond de veau dans lequel vous aurez fait dissoudre la glace de
viande, ajoutez le bouquet ; laissez cuire doucement de façon à amener la sauce
à bonne consistance, ce qui demande trois quarts d'heure environ. Goûtez,
rectifiez l'assaisonnement, s'il est nécessaire.

En même temps, faites revenir le bacon coupé en tranches, à petit feu, dans une
poêle ; hachez-le.

Réunissez hachis de champignons et hachis de bacon, ajoutez les jaunes d'œufs,
du poivre et du jus de citron au goût, mélangez bien.

Farcissez avec ce mélange les chapeaux des champignons, puis mettez-les dans la
sauce, farce en l'air ; laissez-les cuire ; arrosez-les avec la sauce pendant
la cuisson.

Un peu avant la fin, faites sauter les filets mignons dans {\ppp50\mmm} grammes
de beurre, salez, poivrez.

Préparez les canapés en faisant dorer les tranches de pain anglais dans 50
grammes de beurre, de façon à les bien imbiber.

Au dernier moment, retirez les champignons de leur cuisson ; passez la sauce et
montez-la avec le reste du beurre.

Les quatre éléments du plat : filets, canapés, champignons et sauce doivent
être prêts en même temps.

Dressez les canapés sur un plat de service tenu au chaud, placez dessus les filets
mignons, disposez autour d'eux les champignons, masquez avec la sauce et servez.

\sk

\index{Canapés de filets mignons aux cèpes}
\index{Canapés de filets mignons aux truffes}
\index{Canapés de filets mignons aux champignons}
\index{Canapés de filets mignons aux morilles}
Comme variantes, on pourra remplacer le bacon par du jambon ou de la langue ;
les champignons de couche par des cèpes, des morilles ou des truffes ; le fond
de veau par du jus de bœuf, \hyperlink{p0201}{p. \pageref{pg0201}} ; le porto par
du madère ou du xérès ; et les canapés de pain anglais par des canapés de
brioche, de galette feuilletée ou de galette de plomb.

\section*{\centering Filet de bœuf poché, aux tomates.}
\phantomsection
\addcontentsline{toc}{section}{ Filet de bœuf poché, aux tomates.}
\index{Filet de bœuf poché, aux tomates}
\index{Bœuf poché, aux tomates}

\label{pg0464} \hypertarget{p0464}{}

Pour six à huit personnes prenez :

\medskip

\footnotesize
\begin{longtable}{rrrp{16em}}
  1 500 & grammes & de & filet de bœuf paré,                                                              \\
  1 500 & grammes & de & bon bouillon,                                                                    \\
    200 & grammes & de & purée de tomates aromatisée,                                                     \\
    125 & grammes & de & madère,                                                                          \\
     60 & grammes & de & beurre,                                                                          \\
     20 & grammes & de & farine,                                                                          \\
      6 & grammes & de & poivre blanc fraîchement moulu,                                                  \\
      5 & grammes & de & sel blanc,                                                                       \\
        &         & 12 & tomates, pesant ensemble 1 kilogramme environ.                                   \\
\end{longtable}
\normalsize

Ébouillantez les tomates, pelez-les, retirez-en les pépins, et assaisonnez-les
avec {\ppp5\mmm} grammes de poivre et {\ppp3\mmm} grammes de sel mélangés, que vous insérerez dans
leur intérieur.

Faites bouillir le bouillon ; plongez dedans le filet et laissez-le cuire
à raison d'un quart d'heure par {\ppp500\mmm} grammes de viande, soit pendant {\ppp45\mmm} minutes.

En même temps, faites un roux avec le beurre et la farine ; mouillez avec le
madère et {\ppp125\mmm} grammes de bouillon de cuisson ; ajoutez la purée de tomates ;
concentrez la sauce.

Dès que le filet sera cuit, découpez-le en tranches que vous assaisonnerez des
deux côtés avec le reste du sel et du poivre mélangés ; dressez-les sur un
plat, masquez-les avec la sauce, tenez-les au chaud.

Mettez les tomates pendant une minute dans le reste du liquide de cuisson
bouillant, puis disposez-les autour du filet et servez.

Dans cette préparation, la viande ne présente pas, comme dans un rôti, des
parties plus ou moins racornies par le feu ; elle est juteuse ; les tomates
conservent tout leur arome, et l'ensemble constitue un plat qui mérite
l'attention.

\section*{\centering Daube\footnote{Le mot daube est synonyme de viande
braisée.
\protect\endgraf
\index{Daubes du Midi}
Dans le Midi, la cuisson est faite au vin, dans une braisière foncée
de couenne, la viande est généralement piquée de lard et l'assaisonnement,
relevé, contient toujours de l'ail et de l'écorce d'orange.
\protect\endgraf
On ajoute souvent aussi de la langue, des tripes, des pieds de porc, de la
volaille, qu'on fait cuire avec le bœuf ou à part.} de faux filet, aux pommes
de terre à l'étuvée.}

\phantomsection
\addcontentsline{toc}{section}{ Daube de faux filet, aux pommes de terre à l'étuvée.}
\index{Daube de faux filet, aux pommes de terre à l'étuvée}
\index{Boeuf en daube}
\index{Faux filet en daube}
\label{pg0464-2} \hypertarget{p0464-2}{}

Pour six personnes prenez :

\medskip

\footnotesize
\begin{longtable}{rrrrp{16em}}
  & \multicolumn{2}{r}{2 kilogrammes} & de & faux filet,                                                 \\
  & 625 & grammes & de & pommes de terre pelées et coupées
                         en tranches d'épaisseur uniforme,                                                \\
  & 375 & grammes & de & beurre,                                                                          \\
  & 375 & grammes & d' & un mélange \textit{ad libitum} d'olives
                         dont on aura retiré les noyaux, de truffes
                         et de champignons épluchés,                                                      \\
  &     &         &  1 & bouquet garni,                                                                   \\
  &     &         &    & sel et poivre.                                                                   \\
\end{longtable}
\normalsize

L'ustensile qui convient le mieux pour préparer le faux filet en daube est une
gamelle tronconique en cuivre étamé, munie d'un couvercle, pouvant être séparée
en deux compartiments par une tôle perforée de même métal, qui est elle-même
munie d'un anneau.

Placez au fond de la gamelle le faux filet, le mélange d'olives, champignons et
truffes, le beurre, le bouquet garni ; salez et poivrez ; mettez en place la
tôle perforée et disposez dessus les pommes de terre ; salez, poivrez ; puis,
coiffez la gamelle de son couvercle et faites cuire à feu doux pendant trois
heures.

Dressez la viande braisée sur un plat foncé avec une purée obtenue en passant
au tamis jus, champignons, truffes et olives, garnissez avec les pommes de
terre et servez.

Les pommes de terre, cuites ainsi à l'étuvée, engraissées par les vapeurs du
beurre, nourries par les émanations de la viande et parfumées par les aromes
des champignons, des truffes et des olives, sont exquises.

\sk

On peut encore passer à la presse tout le contenu du compartiment du bas :
viande, légumes, etc. ; on servira alors les pommes de terre seules, dans un
légumier, et le jus obtenu à la presse, dans une saucière.

On aura ainsi un savoureux entremets de légumes qu'on pourra désigner sous le
nom de pommes de terre à l'étuvée, au jus.

\section*{\centering Rump-steak braisé, aux pommes de terre à la crème.}
\phantomsection
\addcontentsline{toc}{section}{ Rump-steak braisé, aux pommes de terre à la crème.}
\index{Rump-steak braisé, aux pommes de terre à la crème}

Pour six personnes prenez :

\medskip

\footnotesize
\begin{longtable}{rrrp{16em}}
  1 200 & grammes & de  & rump-steak,                                                                     \\
  1 000 & grammes & de  & pommes de terre,                                                                \\
    375 & grammes & de  & bouillon ou 60 grammes de glace de viande dissoute dans 325 grammes d'eau,      \\
    250 & grammes & de  & crème,                                                                          \\
    250 & grammes & de  & vin rouge,                                                                      \\
     90 & grammes & de  & beurre,                                                                         \\
     60 & grammes &  de & chapelure,                                                                      \\
     20 & grammes &  de & farine,                                                                         \\
        &         & 100 & câpres,                                                                         \\
        &         &   2 & oignons moyens,                                                                 \\
        &         &   2 & carottes moyennes,                                                              \\
        &         &   2 & clous de girofle,                                                               \\
        &         &   1 & bouquet garni, composé de 5 grammes de persil et de 5 grammes de céleri,        \\
        &         &     & le jus d'un demi-citron,                                                        \\
        &         &     & persil haché,                                                                   \\
        &         &     & sel et poivre.                                                                  \\
\end{longtable}
\normalsize

Faites aigrir la moitié de la crème en la tenant au chaud ; réservez le reste.

Coupez la viande en six tranches, chacune de l'épaisseur d'un doigt ;
battez-les pour les attendrir, salez-les et laites-les revenir dans une
casserole avec {\ppp60\mmm} grammes de beurre, les oignons, les carottes et le bouquet ;
puis, ajoutez la chapelure revenue dans le reste du beurre, {\ppp1\mmm} gramme de poivre,
les clous de girofle ; mouillez avec le vin et le bouillon ; couvrez, puis
laissez mijoter pendant une heure et demie, en arrosant avec le jus pendant la
cuisson.

En même temps, faites cuire à part les pommes de terre à la vapeur.

Passez la sauce, ajoutez la crème aigrie, le jus de citron, la farine
(seulement dans le cas où la sauce serait trop liquide) et les câpres ; couvrez
et laissez mijoter encore pendant un quart d'heure.

Lorsque les pommes de terre seront cuites, ajoutez-y le reste de la crème
fraîche, du sel, du poivre, au goût, et laissez mijoter aussi pendant un quart
d'heure.

Dressez les tranches de viande sur un plat, masquez-les avec la sauce et servez
en envoyant à part, dans un légumier, les pommes de terre saupoudrées ou non de
persil haché.

\section*{\centering Paupiettes de bœuf braisées, à la crème.}
\phantomsection
\addcontentsline{toc}{section}{ Paupiettes de bœuf braisées, à la crème.}
\index{Paupiettes de bœuf braisées, à la crème}

Pour quatre personnes prenez :

\medskip

\footnotesize
\begin{longtable}{rrrp{16em}}
    750 & grammes & de & faux filet persillé de graisse, paré et coupé en quatre tranches,                \\
    250 & grammes & de & fond de veau,                                                                    \\
    150 & grammes & de & crème,                                                                           \\
    125 & grammes & d' & anchois,                                                                         \\
     50 & grammes & de & beurre,                                                                          \\
     25 & grammes & d' & oignons,                                                                         \\
      5 & grammes & de & persil,                                                                          \\
        &         &    & jus de citron,                                                                   \\
        &         &    & sel et poivre.                                                                   \\
\end{longtable}
\normalsize

Lavez, essuyez les anchois, levez-en les filets ; hachez fin le persil et les
oignons. Faites une pâte avec les filets d'anchois, le persil et les oignons ;
étendez cette pâte sur l’une des faces des tranches de viande ; roulez-les en
paupiettes ; ficelez-les.

Mettez le beurre dans une casserole, laissez-le fondre ; faites revenir dedans
les paupiettes ; mouillez avec le fond de veau, poivrez au goût ; laissez
mijoter à petit feu pendant une heure. Un quart d'heure avant la fin, ajoutez
la crème acidulée par un peu de jus de citron. Achevez la cuisson.

Le sel des anchois suffit généralement ; goûtez cependant et complétez
l'assaisonnement si c'est utile.

Enlevez les ficelles et servez.

Envoyez en même temps un légumier de pommes de terre cuites à la vapeur ou au
diable Rousset et un ravier de beurre frais.

\section*{\centering Paupiettes de bœuf braisées, à la crème.}
\phantomsection
\addcontentsline{toc}{section}{ Paupiettes de bœuf braisées, à la crème.}
\index{Paupiettes de bœuf braisées, à la crème}

\begin{center}
\textit{(Autre formule).}
\end{center}

Pour quatre personnes prenez :

\medskip

\footnotesize
\begin{longtable}{rrrp{16em}}
    750 & grammes & de & faux filet persillé de graisse, paré et coupé en quatre tranches.                \\
    250 & grammes & de & fond de veau,                                                                    \\
    250 & grammes & de & crème,                                                                           \\
    100 & grammes & de & raifort râpé,                                                                    \\
     50 & grammes & de & mie de pain rassis tamisée,                                                      \\
     50 & grammes & de & beurre,                                                                          \\
        &         &  2 & jaunes d'œufs frais,                                                             \\
        &         &    & jus de citron,                                                                   \\
        &         &    & farine.                                                                          \\
        &         &    & sel et poivre.                                                                   \\
\end{longtable}
\normalsize

Mélangez intimement raifort, mie de pain et jaunes d'œufs ; assaisonnez avec
sel et poivre.

Étendez un quart du mélange sur une face des quatre tranches de viande ; roulez
en paupiettes ; ficelez ; saupoudrez de farine.

Faites revenir les paupiettes dans le beurre, mouillez avec le fond, salez,
poivrez au goût, puis laissez mijoter à tout petit feu pendant une heure.

Un quart d'heure avant la fin, ajoutez la crème et du jus de citron, au goût.
Ne faites plus bouillir.

La cuisson achevée, retirez les ficelles et servez les paupiettes, en envoyant
en même temps un légumier de riz sauté, par exemple.

\section*{\centering Paupiettes de bœuf braisées, sauce à la crème.}
\phantomsection
\addcontentsline{toc}{section}{ Paupiettes de bœuf braisées, sauce à la crème.}
\index{Paupiettes de bœuf braisées, sauce à la crème}

Pour quatre personnes prenez :

\medskip

\footnotesize
\begin{longtable}{rrrp{16em}}
    750 & grammes & de & faux filet persillé de graisse, paré et coupé
                         en quatre tranches régulières, sans déchirures,                                  \\
    250 & grammes & de & champignons de couche,                                                           \\
    200 & grammes & de & crème,                                                                           \\
    150 & grammes & de & beurre,                                                                          \\
     20 & grammes & de & chapelure,                                                                       \\
        & 1 litre & de & bouillon,                                                                        \\
        &         &  2 & beaux oignons hachés fin,                                                        \\
        &         &  2 & jaunes d'œufs crus,                                                              \\
        &         &    & jus de citron,                                                                   \\
        &         &    & muscade,                                                                         \\
        &         &    & sel et poivre.                                                                   \\
\end{longtable}
\normalsize

Aplatissez les tranches de bœuf, salez-les, poivrez-les.

Épluchez les champignons ; faites-les cuire dans du beurre et du jus de
citron ; hachez-les fin.

Faites revenir dans du beurre les oignons et une partie de la chapelure ;
salez, poivrez, ajoutez de la muscade râpée, au goût, les champignons hachés,
liez avec les jaunes d'œufs ; mélangez.

Garnissez les tranches de viande avec cette farce ; roulez-les en paupiettes,
ficelez-les ; mettez les paupiettes dans une casserole avec le reste du
beurre ; faites-les revenir pendant une demi-heure ; puis mouillez avec le
bouillon, couvrez et laissez cuire à petit feu pendant trois quarts d'heure.

Retirez les paupiettes ; enlevez les ficelles.

Dressez les paupiettes sur un plat tenu au chaud. Mettez dans la sauce la crème
et le reste de la chapelure, laissez mijoter pendant le temps nécessaire pour
amener la sauce à une consistance convenable, versez-la sur les paupiettes et
servez.

Envoyez en même temps soit du riz aux cèpes, soit du gruau de sarrasin, soit
de la purée de pommes de terre.

\section*{\centering Paupiettes de bœuf braisées au porto.}
\phantomsection
\addcontentsline{toc}{section}{ Paupiettes de bœuf braisées au porto.}
\index{Paupiettes de bœuf braisées au porto}

Pour quatre personnes prenez :

\medskip

\footnotesize
\begin{longtable}{rrrp{16em}}
    500 & grammes & de & faux filet paré, coupé en quatre tranches régulières,                            \\
    300 & grammes & de & vin de Porto rouge,                                                              \\
    250 & grammes & de & pâté de gibier, en terrine, truffé ou non,                                       \\
    250 & grammes & de & fond de gibier,                                                                  \\
     50 & grammes & de & beurre,                                                                          \\
        &         &    & sel, poivre.                                                                     \\
\end{longtable}
\normalsize

Préparez le fond de gibier avec les déchets du gibier ayant servi à la confection
de la terrine, des légumes, des oignons, un bouquet garni, du sel, du poivre et de
l'eau en quantité suffisante.

Étalez sur chaque tranche de faux filet une couche de pâté, roulez-les en
paupiettes, ficelez les paupiettes et faites-les revenir pendant un quart
d'heure dans le beurre ; assaisonnez très légèrement avec sel et poivre,
mouillez avec le porto et le fond de gibier ; couvrez et laissez mijoter
pendant une heure et demie à deux heures.

Au moment de servir, goûtez la sauce pour l'assaisonnement et complétez-le s'il
y a lieu.

Retirez les ficelles, dressez les paupiettes sur un plat, masquez-les avec la
sauce et envoyez en même temps des pommes de terre Champs-Élysées ou une purée
de pommes de terre et de cerfeuil bulbeux.

Le fumet de gibier parfume agréablement le plat, dont tous les éléments se
marient à souhait.

\section*{\centering Côte de bœuf braisée au champagne.}
\phantomsection
\addcontentsline{toc}{section}{ Côte de bœuf braisée au champagne.}
\index{Côte de bœuf braisée au champagne}
\index{Bœuf braisé au champagne}

Pour douze personnes prenez :

\medskip

\footnotesize
\begin{longtable}{rrrrp{16em}}
  & \multicolumn{2}{r}{5 kilogrammes} & de & côte de bœuf, non parée,                                     \\
  & \multicolumn{2}{r}{1 bouteille}   & de & champagne demi-sec,                                          \\
  &     &             &  1 & pied de veau,                                                                \\
  &     &             &  1 & bouquet garni,                                                               \\
  &     &             &    & graisse de rôti,                                                             \\
  &     &             &    & carottes,                                                                    \\
  &     &             &    & oignons,                                                                     \\
  &     &             &    & bouillon,                                                                    \\
  &     &             &    & sel et poivre.                                                               \\
\end{longtable}
\normalsize

Faites dorer la viande des deux côtés dans de la graisse de rôti ; mettez-la
ensuite dans une braisière avec le pied de veau ; mouillez avec les quatre
cinquièmes du champagne ; laissez cuire pendant une heure et demie ; puis,
ajoutez du bouillon en quantité suffisante, des carottes, des oignons, le
bouquet garni et laissez mijoter à tout petit feu pendant huit heures en
remplaçant le liquide (vin et bouillon) au fur et à mesure de son évaporation.

A la fin, retirez les os du pied de veau, réduisez la cuisson, dégraissez,
passez, goûtez pour l'assaisonnement et complétez-le s'il y a lieu avec sel et
poivre ; servez.

Envoyez en même temps, mais à part, des légumes blanchis dans du bouillon et
sautés au beurre, ou encore des pâtes, par exemple des nouilles sautées.

\section*{\centering Côte de bœuf gratinée.}
\phantomsection
\addcontentsline{toc}{section}{ Côte de bœuf gratinée.}
\index{Côte de bœuf gratinée}
\index{Bœuf gratiné}

Pour six personnes prenez :

\medskip

\footnotesize
\begin{longtable}{rrrp{16em}}
  1 500 & grammes & de & côte de bœuf persillée de graisse et épaisse de 4 centimètres,                   \\
    200 & grammes & de & bouillon,                                                                        \\
    100 & grammes & de & vin rouge,                                                                       \\
    100 & grammes & de & champignons de couche épluchés,                                                  \\
     75 & grammes & de & fromage râpé, moitié gruyère, moitié parmesan,                                   \\
     65 & grammes & de & lard frais,                                                                      \\
     40 & grammes & d' & oignons,                                                                         \\
     15 & grammes & de & mie de pain rassis tamisée,                                                      \\
     10 & grammes & d' & huile d'olive,                                                                   \\
     10 & grammes & de & sel blanc,                                                                       \\
      5 & grammes & de & persil,                                                                          \\
      2 & grammes & de & poivre,                                                                          \\
     1/2&  gramme & d’ & ail,                                                                             \\
     1/2&  gramme & de & quatre épices,                                                                   \\
        &         &  1 & œuf dur,                                                                         \\
        &         &  1 & barde de lard,                                                                   \\
        &         &  1 & bouquet garni, composé de 5 grammes de persil,
                         1 gramme de thym et 2 décigrammes de laurier,                                    \\
        &         &    & beurre,                                                                          \\
        &         &    & crépine.                                                                         \\
\end{longtable}
\normalsize

Désossez la viande ; faites-la dorer dans un peu de beurre ; réservez les os.
Hachez ensemble œuf dur, champignons, lard, ail, persil, assaisonnez avec le
sel, le poivre et les quatre épices, ajoutez l'huile, mélangez ; enrobez la
viande avec cette farce et enveloppez le tout dans de la crépine.

Foncez une casserole avec la barde de lard, mettez dessus la côte, mouillez
avec le bouillon et le vin, ajoutez les oignons, le bouquet garni et les os
réservés ; faites cuire pendant une heure à une heure un quart, puis retournez
la viande et laissez cuire encore pendant le même laps de temps.

Dressez la viande sur un plat de service allant au feu ; tenez-la au chaud.
Réduisez la cuisson, dégraissez-la et passez-la sur la viande.

Mélangez mie de pain et fromages râpés, couvrez la viande avec ce mélange et
faites gratiner au four pendant quelques minutes. Servez.

\sk

\index{Filet de bœuf gratiné}
On peut préparer de même un filet de bœuf.

\section*{\centering Bœuf en cocote.}
\phantomsection
\addcontentsline{toc}{section}{ Bœuf en cocote.}
\index{Bœuf en cocote}

Pour quatre personnes prenez :

\medskip

\footnotesize
\begin{longtable}{rrrrp{16em}}
  & 750 & grammes & de & rump-steak en une tranche de 2 centimètres d'épaisseur,                          \\
  & 300 & grammes & de & pommes de terre,                                                                 \\
  & 150 & grammes & de & carottes,                                                                        \\
  & 150 & grammes & de & moelle de bœuf,                                                                  \\
  &  50 & grammes & d' & oignons,                                                                         \\
  &  20 & grammes & de & céleri,                                                                          \\
  &   7 & grammes & de & sel,                                                                             \\
  &   5 & grammes & de & persil,                                                                          \\
  & \multicolumn{2}{r}{6 décigrammes} & de & poivre.                                                       \\
\end{longtable}
\normalsize

Coupez le rump-steak en morceaux carrés de {\ppp6\mmm} centimètres de côté, les
pommes de terre, les carottes et la moelle en tranches minces, hachez les
oignons, le céleri et le persil.

Mettez au fond d'une cocote en porcelaine épaisse allant au feu,
successivement, la moitié des éléments dans l'ordre suivant : moelle, viande,
sel, poivre, pommes de terre, carottes, oignons, céleri, persil et recommencez
les mêmes alternances. Couvrez avec un papier beurré.

Couvrez la cocote avec son couvercle ; faites cuire au four doux pendant une
heure et demie ; dégraissez.

Servez dans la cocote.

\sk

\label{pg0471} \hypertarget{p0471}{}
\index{Bee pie (Bœuf en croûte)}
\index{Bœuf en croûte (Beef pie)}
Le « beef-pie » est une variante anglaise de bœuf en cocote. Dans sa
préparation, on couvre la cocote avec une abaisse en pâte feuilletée décorée
comme le dessus d’un pâté et munie d'un orifice qu'on obstrue après la cuisson
avec un bouchon de pâte.

Lorsque le beef-pie doit être mangé froid, on introduit par l'orifice, avant le
refroidissement, quelques cuillerées de bon jus qui se prend en gelée.

On ajoute souvent au bœuf du rognon de veau ou des rognons de mouton.

\section*{\centering Bœuf à la mode, en aspic.}
\phantomsection
\addcontentsline{toc}{section}{ Bœuf à la mode, en aspic.}
\index{Bœuf à la mode, en aspic}
\index{Aspic de bœuf à la mode}
\index{Bœuf à la mode chaud}
\index{Bœuf à la mode froid}

Le bœuf à la mode bien préparé est un excellent plat de famille qu'on peut
servir aussi bien chaud\footnote{Le bœuf à la mode, chaud, peut être servi,
sans carottes, avec une purée d'oignons : c'est le bœuf Soubise.
\protect\endgraf
\index{Bœuf Soubise}
\index{Bœuf au riz}
On peut encore le présenter, sans carottes, avec du riz : ris sec, ris au gras,
risotto ou pilaf.} que froid.

Présenté froid, en aspic, il a très bonne allure, et il convient
particulièrement pour les pique-niques à la campagne.

\medskip

Pour dix à douze personnes prenez :

\medskip

\footnotesize
\begin{longtable}{rrrp{16em}}
  2 500 & grammes & de & culotte de bœuf désossée et parée,                                               \\
    800 & grammes & de & bon bouillon,                                                                    \\
    700 & grammes & de & vin blanc de Sauternes,                                                          \\
    700 & grammes & de & vin de Madère,                                                                   \\
    250 & grammes & de & carottes épluchées et coupées en tranches,
                         dont on ne conservera que les parties rouges,                                    \\
    200 & grammes & de & lard à piquer, en lardons ayant une section carrée de 5 millimètres de côté.     \\
    100 & grammes & d' & oignons épluchés et coupés en tranches,                                          \\
     50 & grammes & de & fine champagne,                                                                  \\
     40 & grammes & de & graisse de rôti.                                                                 \\
        &         &  3 & clous de girofle,                                                                \\
        &         &  2 & blancs d'œufs,                                                                   \\
        &         &  1 & fort pied de veau lavé, nettoyé et coupé en morceaux,                            \\
        &         &  1 & morceau carré de couenne de lard de 2 décimètres de côté,                        \\
        &         &  1 & bouquet garni (persil, thym, laurier),                                           \\
        &         &    & quatre épices,                                                                   \\
        &         &    & sel et poivre.                                                                   \\
\end{longtable}
\normalsize

Mettez le bœuf assaisonné avec sel, poivre et quatre épices, les lardons, 400
grammes de vin blanc et {\ppp400\mmm} grammes de madère dans une terrine de dimensions
convenables pour que la viande baigne dans le liquide.

Au bout de vingt-quatre heures, prenez les lardons, assaisonnez-les, piquez-en
la viande parallèlement aux fibres : mettez-la dans une casserole avec la
graisse de rôti et faites-la revenir de tous les côtés pendant {\ppp20\mmm} minutes ;
flambez-la ensuite avec la fine champagne.

Foncez la casserole avec la couenne, placez le bœuf dessus, ajoutez le pied de
veau, les carottes, les oignons, le bouquet garni, les clous de girofle,
mouillez avec le bouillon, le reste du vin blanc et le reste du madère, salez
et poivrez légèrement, couvrez, donnez un bouillon, puis laissez cuire à petit
feu pendant quatre heures et demie. Avant la fin de la cuisson, goûtez et
complétez l'assaisonnement s'il y a lieu.

La cuisson achevée, retirez le bœuf et les carottes. Passez le jus,
dégraissez-le, clarifiez-le avec les blancs et deux coquilles d'œufs, en
chauffant sur un feu modéré, retirez-le du feu au premier bouillon et passez-le
au travers d'un torchon légèrement mouillé.

Coupez le bœuf en tranches perpendiculairement aux lardons.

Prenez un moule, décorez-en les parois avec des tranches de carottes non
désagrégées, cuites à part, coulez-y du jus ; laissez prendre en gelée, puis mettez
une couche de tranches de bœuf, au-dessus une couche de carottes, noyez le tout
dans du jus, continuez ainsi les alternances et terminez par une couche de bœuf.
Mettez à refroidir, puis démoulez.

Servez en découpant l'aspic comme si vous aviez affaire à un pâté.

Les convives trouveront dans chaque bouchée du bœuf, du lard, des carottes et
de la gelée.

\sk
\index{Canard en aspic}
\index{Canard à la mode}
\index{Aspic de canard}

On peut préparer de même un canard à la mode, chaud, froid ou en aspic.

\section*{\centering Bœuf à la bourguignonne.}
\phantomsection
\addcontentsline{toc}{section}{ Bœuf à la bourguignonne.}
\index{Bœuf à la bourguignonne}

Pour douze personnes prenez :

\medskip

\footnotesize
\begin{longtable}{rrrp{16em}}
  2 500 & grammes & de & culotte de bœuf, désossée et parée,                                              \\
  1 000 & grammes & de & jarret de veau,                                                                  \\
    750 & grammes & de & vin rouge de Bourgogne, du vin de Beaune, par exemple,                           \\
    700 & grammes & de & carottes,                                                                        \\
    625 & grammes & de & petits champignons de couche,                                                    \\
    500 & grammes & de & lard de poitrine,                                                                \\
    200 & grammes & de & lard à piquer,                                                                   \\
     50 & grammes & de & fine champagne,                                                                  \\
        &         & 48 & petits oignons blancs,                                                           \\
        &         &  3 & abatis de poulets,                                                               \\
        &         &  3 & oignons moyens,                                                                  \\
        &         &  1 & pied de veau,                                                                    \\
        &         &  1 & morceau carré de couenne maigre, de 15 centimètres de côté,                      \\
        &         &  1 & bouquet garni,                                                                   \\
        &         &  1 & clou de girofle,                                                                 \\
        &         &    & beurre ou graisse de rôti,                                                       \\
        &         &    &jus de citron,                                                                    \\
        &         &    &sucre en poudre,                                                                  \\
        &         &    &sel et poivre.                                                                    \\
\end{longtable}
\normalsize

Préparez la veille un fond de veau, comme il est dit,
\hyperlink{p0426}{p. \pageref{pg0426}}, avec le jarret, le pied de veau, les
abatis, les carottes, les trois oignons moyens, la couenne maigre, le bouquet
garni. le clou de girofle, du sel, du poivre et de l'eau en quantité
suffisante.

Le lendemain, pelez les champignons, passez les chapeaux dans du jus de citron,
réservez-les.

Mettez dans le fond de veau les pieds et les épluchures lavées des
champignons ; faites réduire de façon à obtenir {\ppp900\mmm} grammes environ de liquide
concentré ; passez-le.

Coupez le lard à piquer en aiguillettes ; assaisonnez-les ; piquez-en la viande
que vous ferez revenir ensuite dans du beurre ou dans de la graisse de rôti,
pendant {\ppp20\mmm} à {\ppp30\mmm} minutes en la retournant de tous côtés.

Mettez la viande revenue dans une braisière ; flambez-la avec la fine
champagne ; assaisonnez avec sel et poivre ; mouillez avec le fond de veau
concentré et le vin ; chauffez, puis faites braiser au four, à petit feu,
pendant six heures environ.

Préparez la garniture :

\index{Garniture bourguignonne}
Épluchez les petits oignons, mettez-les dans une sauteuse avec du beurre ;
saupoudrez avec du sucre et salez un peu ; chauffez, puis faites sauter les
oignons de manière que le caramel fourni par le sucre en recouvre la surface.
Mouillez avec {\ppp150\mmm} grammes environ de jus de la braisière et laissez cuire
jusqu'à ce que la cuisson soit réduite à l'état de demi-glace. Les oignons
doivent rester entiers.

Coupez le lard de poitrine en petites tranches ou en dés ; faites-les rissoler.

Faites sauter les chapeaux de champignons dans du beurre.

Dégraissez le liquide de la braisière ; concentrez-le s'il est nécessaire.

Dressez le bœuf sur un plat de service, masquez-le avec la sauce, entourez-le
avec sa garniture d'oignons glacés, de lardons rissolés et de champignons
sautés, puis servez.

\sk

\index{Bœuf à la parisienne}
\index{Garniture parisienne}
Cette pièce de bœuf, dressée sur un socle de riz et entourée de croquettes de
pommes de terre, de petites timbales aux choux, de laitues braisées, de fonds
d'artichauts garnis de pointes d'asperges et de petits pois, prend le nom de
« bœuf à la parisienne ».

\sk

\index{Bœuf à la flamande}
\index{Garniture flamande}
En remplaçant la garniture du bœuf à la bourguignonne par des choux braisés,
des carottes, des navets, des pommes de terre, des tranches de saucisson et de
cervelas, on aura le « bœuf à la flamande ».

\sk

\index{Bœuf à l'italienne}
\index{Garniture italienne}
En remplaçant dans la formule du bœuf à la bourguignonne le beaune par du
chianti rouge et en accompagnant la viande avec des pâtes, notamment des
spaghetti aux tomates et au parmesan, le tout arrosé avec le jus de cuisson, on
aura un excellent « bœuf à l'italienne ».

\section*{\centering Culotte de bœuf aux pâtes.}
\phantomsection
\addcontentsline{toc}{section}{ Culotte de bœuf aux pâtes.}
\index{Culotte de bœuf aux pâtes}
\index{Bœuf aux pâtes}

Pour dix à douze personnes prenez :

\medskip

\footnotesize
\begin{longtable}{rrrp{16em}}
  2 500 & grammes & de & culotte\footnote{ La culotte est la partie qui commence à
                          l'aloyau et finit à la queue.} de bœuf désossée et parée,                       \\
    500 & grammes & de & pâtes,                                                                           \\
    500 & grammes & de & parmesan râpé,                                                                   \\
    500 & grammes & de & bouillon,                                                                        \\
    250 & grammes & de & purée de tomates,                                                                \\
    250 & grammes & de & lard gras haché,                                                                 \\
    200 & grammes & de & lard à piquer, en lardons ayant une section carrée de 5 millimètres de côté,     \\
    200 & grammes & d' & oignons épluchés et hachés,                                                      \\
     50 & grammes & de & fine champagne,                                                                  \\
        &         &  3 & clous de girofle,                                                                \\
        &         &    & pied de veau,                                                                    \\
        &         &    & bouquet garni,                                                                   \\
        &         &    & gousse d'ail,                                                                    \\
        &         &    & sel et poivre.                                                                   \\
\end{longtable}
\normalsize

Assaisonnez les lardons : piquez-en la viande.

Mettez dans une casserole le lard haché, laissez-le fondre, puis ajoutez la
viande et les oignons : faites revenir le tout à découvert, sur un bon feu,
pendant {\ppp20\mmm} à {\ppp30\mmm} minutes, en remuant fréquemment. Flambez
ensuite avec la fine champagne, mouillez avec le bouillon, ajoutez pied de
veau, ail, bouquet garni, clous de girofle, purée de tomates, sel et poivre.
Couvrez avec un papier sous le couvercle et laissez cuire à petit feu pendant
six heures. Goûtez, complétez l'assaisonnement s'il est nécessaire, dégraissez
et passez le jus ; il servira à accommoder les pâtes.

Faites cuire les pâtes pendant {\ppp20\mmm} minutes dans du bouillon ou,
à défaut, dans de l'eau salée ; égouttez-les, puis mettez-les dans un légumier,
par couches, en alternant avec du parmesan et en arrosant avec le jus, terminez
par du parmesan et servez en même temps que le bœuf.

Parmi les pâtes du commerce, celles que je préfère sont les spaghetti de Naples
et les vermicelli de Palerme.

Je conseille de faire cuire le bœuf la veille et de le réchauffer le lendemaim
au moment de le servir ; je le trouve meilleur ainsi.

Les pâtes ne doivent jamais être accommodées d'avance.

\section*{\centering Civet de bœuf.}
\phantomsection
\addcontentsline{toc}{section}{ Civet de bœuf.}
\index{Civet de bœuf}
\index{Bœuf en civet}

Pour cinq personnes prenez :

\medskip

\footnotesize
\begin{longtable}{rrrp{16em}}
  1 000 & grammes & de & rump-steak,                                                                      \\
    750 & grammes & de & vin rouge,                                                                       \\
    250 & grammes & de & champignons de couche,                                                           \\
    150 & grammes & de & bacon,                                                                           \\
    100 & grammes & de & beurre,                                                                          \\
    100 & grammes & de & sang de porc,                                                                    \\
     25 & grammes & de & farine,                                                                          \\
     15 & grammes & de & fine champagne,                                                                  \\
        &         & 12 & petits oignons,                                                                  \\
        &         &  1 & échalote,                                                                        \\
        &         &  1 & bouquet garni (persil, thym, laurier),                                           \\
        &         &    & jus de citron,                                                                   \\
        &         &    & sel et poivre.                                                                   \\
\end{longtable}
\normalsize

Coupez le rump-steak en morceaux de la grosseur de petits filets mignons, le
lard en gros dés ; puis, faites revenir viande de bœuf et lard dans une partie du
beurre ; flambez avec la fine champagne et réservez.

Faites, avec le reste du beurre et la farine, un roux dans lequel vous ferez
dorer les oignons, mettez ensuite l'échalote, le bouquet garni, du sel, du
poivre ; mouillez avec le vin ; ajoutez bœuf et lard réservés ; laissez cuire
à feu modéré pendant deux heures.

Un quart d'heure avant la fin, ajoutez les champignons épluchés et passés dans
du jus de citron.

Dressez les morceaux de viande sur un plat, entourez-les avec les champignons
et les petits oignons ; tenez au chaud.

Versez le sang dans la sauce, mélangez bien, chauffez sans laisser bouillir,
goûtez, complétez l'assaisonnement s'il est nécessaire, passez la sauce sur la
viande et servez,

Envoyez en même temps. dans un légumier, du riz sec ou des pommes de
terre bouillies. écrasées, maniées avec du beurre et gratinées.

\sk

\index{Civet de mouton}
\index{Civat de porc}
On peut faire de même des civets de mouton ou des civets de porc.

\sk

\index{Civet de bœuf mariné}
\index{Civet de viandes marinées}
\index{Civet de mouton mariné}
\index{Civet de porc mariné}
On pourra donner à tous ces civets un gout de venaison en faisant mariner, au
préalable, les viandes dans une marinade telle que celle de la
\hyperlink{p0514}{p. \pageref{pg0514}} et en ajoutant à la sauce un peu de fumet
de gibier ou de venaison.

\section*{\centering Goulach de bœuf.}
\phantomsection
\addcontentsline{toc}{section}{ Goulach de bœuf.}
\index{Goulach de bœuf}
\index{Bœuf (Goulach de)}

Le goulach est un ragoût hongrois au paprika, qu'on peut préparer avec de la
viande de boucherie, du porc ou de la volaille, et qu'on accompagne de pâtes ou
de pommes de terre.

Voici une formule de goulach de bœuf avec des pommes de terre.

\medskip

Pour huit personnes prenez :

\medskip

\footnotesize
\begin{longtable}{rrrp{16em}}
  1 500 & grammes & de & faux filet,                                                                      \\
  1 000 & grammes & de & pommes de terre,                                                                 \\
    400 & grammes & de & bouillon,                                                                        \\
    250 & grammes & de & lard de poitrine,                                                                \\
    250 & grammes & d' & oignons,                                                                         \\
    100 & grammes & de & glace de viande,                                                                 \\
      4 & grammes & de & paprika,                                                                         \\
        &         &    & sel.                                                                             \\
\end{longtable}
\normalsize

Coupez le lard en petits dés, mettez-le dans une casserole, laissez-le fondre ;
ajoutez ensuite les oignons coupés en rouelles minces et faites cuire, en
casserole couverte, pendant cinq minutes.

Coupez le bœuf en morceaux gros comme de belles noix, mettez-le dans la
casserole, assaisonnez avec le paprika, salez légèrement, mouillez avec le
bouillon dans lequel vous aurez fait dissoudre la glace de viande, couvrez,
amenez à ébullition, puis laissez mijoter. Au bout d'une heure, ajoutez les
pommes de terre pelées et coupées en dés, laissez cuire jusqu'à ce que les
pommes de terre soient à point, ce qui demande en moyenne une vingtaine de
minutes, goûtez, complétez l'assaisonnement s'il y a lieu et servez.

Il convient de ne saler que légèrement au début, car il est bien difficile
d'apprécier par avance la proportion du sel contenu dans le bouillon et dans la
glace de viande.

\sk

On peut préparer de la même façon un goulach dans lequel il entrera parties
égales de faux filet de bœuf, filet de mouton et filet de porc.

\section*{\centering Miroton.}
\phantomsection
\addcontentsline{toc}{section}{ Miroton.}
\index{Miroton}
\index{Bœuf miroton}
\index{Bœuf (Miroton de)}

Le miroton est un plat composé de tranches de bœuf cuit, notamment du bœuf
bouilli, qu'on fait simplement réchauffer dans une sauce à base d'oignons.

Émincez des oignons ; faites-les roussir à la poêle dans du beurre ; lorsqu'ils
auront bien pris couleur, mouillez avec du vinaigre de vin, poivrez et laissez
cuire, puis mettez un bouquet garni, de la sauce tomate, de la glace de viande
dissoute dans du bouillon, réduisez jusqu'à consistance convenable ; retirez
ensuite le bouquet, goûtez, complétez l'assaisonnement, qui doit être relevé,
ajoutez le bœuf coupé en tranches, chauffez et servez.

\sk

Pour faire le miroton gratiné, saupoudrez de chapelure un miroton ordinaire
préparé comme il vient d'être dit, mettez par-dessus un peu de beurre coupé en
petits morceaux et poussez au four.

\sk

On peut préparer de même du mouton cuit.

\section*{\centering Galettes de bœuf.}
\phantomsection
\addcontentsline{toc}{section}{ Galettes de bœuf.}
\index{Galettes de bœuf}
\index{Bœuf en galettes}


En France, la viande de boucherie est généralement tendre ; aussi ne sert-on
guère sous forme de galettes ou de boulettes que des hachis faits avec des
restes de viandes rôties ou bouillies. A l'étranger, au contraire, où la viande
de boucherie est fréquemment dure, on fait souvent des hachis de viandes crues.

Voici deux exemples de composition de galettes de bœuf :

\medskip

1° Prenez de la tranche grasse. coupez-la au couteau\footnote{Ce procédé est de
beaucoup supérieur au hachage qui a l'inconvénient de faire perdre du sang.} en
morceaux gros comme des têtes d'épingle, ou hachez-la ; ajoutez de la mie de
pain rassis tamisée trempée dans du lait et égouttée, de l'oignon ciselé revenu
dans du beurre sans avoir pris couleur, des jaunes d'œufs ; assaisonnez avec
sel et poivre. Pétrissez le tout ; faites-en des galettes, auxquelles vous
pourrez donner la forme de biftecks, et que vous ferez cuire, à la poêle, dans
du beurre. Servez-les avec de la purée de pommes de terre.

\sk

Comme variante, on pourra passer les galettes dans du blanc d'œuf et dans de la
mie de pain rassis tamisée, les faire cuire ensuite dans du beurre, les arroser
de jus de citron et les servir avec des pommes de terre sautées.

\sk

On peut, il va sans dire, faire cuire les galettes sur le gril ; mais il
convient alors d'augmenter la proportion de jaunes d'œufs afin de leur donner
une consistance suffisante.

Le raifort râpé accompagne très bien ces différentes galettes panées ou non
panées.

\sk

2° Coupez ou hachez, comme ci-dessus, de la viande maigre de bœuf, ajoutez-y un
tiers de son poids de beurre et un peu de madère ; assaisonnez avec sel, poivre
et muscade ; mélangez bien.

Faites avec le mélange des galettes, que vous lisserez avec la lame mouillée
d'un couteau, passez-les d'abord dans de la farine, ensuite dans de l'œuf
battu, enfin dans de la mie de pain rassis tamisée ; faites-les cuire à la
poêle dans du beurre. Si vous préférez les faire griller, enveloppez-les au
préalable dans de la crépine de porc.

En même temps, préparez une sauce béarnaise ou une sauce composée de jus de
viande et de crème aigrie par du jus de citron ; chauffez, garnissez le dessus
des galettes avec des émincés d'oignons revenus dans du beurre et masquez le
tout avec la sauce.

Ces galettes, comme les précédentes, peuvent être accompagnées de purée de
pommes de terre, de pommes de terre sautées, ou encore de légumes quelconques,
au goût.

\sk

On peut naturellement faire des galettes d'autres viandes. C'est ainsi qu'on
pourra présenter :

\medskip

\index{Galettes de mouton}
\index{Galettes de porc}
\index{Galettes de volaille}

\textit{a}) des galettes de mouton, avec garnitures de haricots sautés ou en purée ;

\textit{b}) des galettes de porc, avec une purée de pois ou de lentilles au lard ;

\textit{c}) des galettes de volaille, exquises, composées de blanc de poulet, jambon,
ris de veau, crème, servies avec une sauce Nantua, ou une purée de champignons
de couche, ou encore avec des morilles au jus.

\sk

Toutes ces préparations seront appréciées par les vieillards et les personnes qui
mastiquent difficilement.

\section*{\centering Filet de veau rôti.}
\phantomsection
\addcontentsline{toc}{section}{ Filet de veau rôti.}
\index{Filet de veau rôti}
\index{Filet de veau piqué, rôti}

Parez le filet, enlevez les parties nerveuses. Piquez-le de fins lardons de
lard gras et de petites baguettes de truffe, ou de fines languettes de jambon,
de langue et de truffe ; assaisonnez-le, bardez-le entièrement et faites-le
rôtir à la broche.

Dressez le filet sur un plat ; glacez-le avec le jus de cuisson, dégraissé,
très réduit ; garnissez le plat soit avec des petites tartelettes ou des
petites croustades de fins salpicons ; soit avec des petites timbales de
nouilles ou de macaroni au jambon, à la langue et aux champignons ; soit encore
avec des croquettes de pommes de terre duchesse, etc.

Servez, en envoyant en même temps de la sauce demi-glace.

\sk

\index{Filet de veau piqué, braisé}
Le filet de veau piqué peut être braisé. On le servira accompagné de garnitures
fines avec sauces adéquates.

\section*{\centering Longe de veau rôtie.}
\phantomsection
\addcontentsline{toc}{section}{ Longe de veau rôtie.}
\index{Longe de veau rôtie}

Enlevez entièrement les os de l'échine, aplatissez bien la bavette en la tenant
un peu longue, retirez une partie de la graisse du rognon, mettez-le contre le
filet mignon, assaisonnez de sel et de poivre, roulez autour la bavette de
façon à couvrir le rognon et le filet mignon ; ficelez en conservant à la longe
une forme de carré allongé, embrochez, couvrez d'un papier beurré et faites
rôtr.

Dressez la longe sur un plat, garnissez les extrémités avec un bouquet de
cresson. Servez, en envoyant à part du jus de viande ou une sauce demi-glace.

Les légumes qui accompagnent le mieux la longe rôtie sont : les pointes
d’asperges, les fonds d’artichauts, les cardons, les laitues, les chicorées,
les pieds de fenouil braisés, les pâtes et le risotto.

\section*{\centering Carré de veau piqué, rôti.}
\phantomsection
\addcontentsline{toc}{section}{ Carré de veau piqué, rôti.}
\index{Carré de veau piqué, rôti}

Prenez le haut du carré, sciez les os des côtes à moitié de leur longueur.
Parez le dessus et piquez-le de lard fin ; salez, poivrez ; couvrez d'un papier
beurré ou bardez entièrement le carré ; embrochez-le et faites-le rôtir.
Quelques instants avant la fin, enlevez le papier ou la barde et donnez de la
couleur au rôti.

Dressez le carré sur un plat foncé d'un bon fond de veau et volaille bien réduit.

Servez, en envoyant en même temps, au choix, un plat de chicorée, une macédoine
de légumes, des épinards, des carottes au jus, des nouilles, du macaroni ou du
riz.

\section*{\centering Escalopes de veau panées.}
\phantomsection
\addcontentsline{toc}{section}{ Escalopes de veau panées.}
\index{Escalopes de veau panées}

On peut préparer des escalopes de veau panées de plusieurs manières. En voici
deux qui diffèrent essentiellement l'une de l'autre par l'épaisseur des escalopes et
par la sauce qui les accompagne.

\medskip

A. — Pour quatre personnes prenez :

\footnotesize
% \begin{longtable}{rrrp{16em}}
\begin{longtable}{rrr>{\raggedright\arraybackslash}p{21em}}
    400 & grammes & de & noix de veau pâtissière, en deux escalopes
                         épaisses de {\ppp10\mmm} à {\ppp12\mmm} millimètres,                                                 \\
     60 & grammes & de & beurre,                                                                          \\
     15 & grammes & de & lait,                                                                            \\
        &         &  1 & œuf,                                                                             \\
        &         &    & farine.                                                                          \\
        &         &    & mie de pain rassis tamisée,                                                      \\
        &         &    & sel et poivre.                                                                   \\
\end{longtable}
\normalsize

Battez l'œuf avec le lait, un gramme de sel et un demi-gramme de poivre ;
passez les escalopes d'abord dans la farine, puis dans l'œuf battu, ensuite
dans la mie de pain et répétez l'opération, si cela est nécessaire, pour
assurer une adhérence convenable du pain.

Faites fondre le beurre dans une poêle, mettez dedans les escalopes, saupoudrez
de sel et de poivre et laissez cuire pendant {\ppp20\mmm} à {\ppp25\mmm} minutes sur un feu
d'intensité moyenne.

Servez avec une sauce tomate.

\medskip

\index{Escalopes de veau panées (autre formule)}
B. — Pour quatre personnes prenez :

\medskip

\footnotesize
\begin{longtable}{rrrp{16em}}
    400 & grammes & de & noix de veau pâtissière, en quatre tranches de
                         5 à 6 millimètres d'épaisseur environ,                                           \\
    100 & grammes & de & beurre,                                                                          \\
     15 & grammes & de & lait,                                                                            \\
        &         &  1 & œuf,                                                                             \\
        &         &    & farine,                                                                          \\
        &         &    & mie de pain rassis tamisée,                                                      \\
        &         &    & jus de citron,                                                                   \\
        &         &    & sel et poivre.                                                                   \\
\end{longtable}
\normalsize

Panez et faites cuire les escalopes comme précédemment, mais pendant {\ppp12\mmm} à 15
minutes seulement, arrosez-les avec du jus de citron et servez.

\section*{\centering Escalopes de veau au jambon, gratinées.}
\phantomsection
\addcontentsline{toc}{section}{ Escalopes de veau au jambon, gratinées.}
\index{Escalopes de veau au jambon, gratinées}

Pour quatre personnes prenez :

\medskip

\footnotesize
\begin{longtable}{rrrp{16em}}
    500 & grammes & de & veau, en quatre escalopes très minces,                                           \\
    250 & grammes & de & jambon de Parme ou de Bologne, en quatre
                         tranches minces de même surface que les escalopes,                               \\
     50 & grammes & de & beurre,                                                                          \\
        &         &    & parmesan râpé,                                                                   \\
        &         &    & sel et poivre.                                                                   \\
\end{longtable}
\normalsize

Faites cuire à la poêle les escalopes dans le beurre ; salez et poivrez
légèrement ; enlevez-les.

Passez les tranches de jambon pendant une minute dans le même beurre simplement
chaud.

Disposez les escalopes dans un plat allant au feu : mettez sur chacune d'elles
une tranche de jambon ; saupoudrez avec du parmesan ; arrosez avec le beurre de
cuisson et faites gratiner au four pendant {\ppp3\mmm} à {\ppp4\mmm} minutes.

Servez dans le plat et envoyez en même temps, mais à part, des gnocchi à la
semoule.

Ce mets, qui est une spécialité bolognaise, surprend agréablement ceux qui ne
connaissent pas la cuisine italienne.

\section*{\centering Côtelettes de veau grillées, sauce Mornay.}
\phantomsection
\addcontentsline{toc}{section}{ Côtelettes de veau grillées, sauce Mornay.}
\index{Côtelettes de veau grillées, sauce Mornay}

Faites griller des côtelettes de veau ; un instant avant la fin, arrêtez la
cuisson.

Foncez un plat allant au feu de sauce Mornay serrée, à base de fond de veau
et volaille, disposez dessus les côtelettes, masquez-les avec de la sauce Mornay,
saupoudrez d'un peu de chapelure ou de mie de pain rassis tamisée, parsemez le
dessus de petits morceaux de beurre frais, faites gratiner au four pendant une
dizaine de minutes, puis servez.

\sk

Comme variantes, fendez sur le côté les côtelettes incomplètement grillées et
farcissez-les de purée de champignons ou de purée Soubise avant de les entourer
de sauce.

\section*{\centering Côtelettes de veau au paprika.}
\phantomsection
\addcontentsline{toc}{section}{ Côtelettes de veau au paprika.}
\index{Côtelettes de veau au paprika}

Pour quatre personnes prenez :

\medskip

\footnotesize
\begin{longtable}{rp{4em}rrrp{16em}}
& & 300 & grammes & de & crème épaisse,                                                                   \\
& & 250 & grammes & de & mie de pain rassis tamisée,                                                      \\
& & 200 & grammes & de & fond de veau,                                                                    \\
& &  60 & grammes & de & beurre,                                                                          \\
& &  50 & grammes & de & glace de viande,                                                                 \\
& &  30 & grammes & de & farine,                                                                          \\
& \multicolumn{3}{r}{6 à 8 centigrammes} & de & paprika,                                                  \\
& &     &         &  4 & côtelettes de veau, de 2 centimètres d'épaisseur
                         et pesant chacune 300 grammes environ,                                           \\
& &     &         &  2 & oignons,                                                                         \\
& &     &         &    & sel.                                                                             \\
\end{longtable}
\normalsize

Faites dorer dans le beurre la farine et les oignons hachés fin ; mouillez avec
le fond de veau dans lequel vous aurez fait dissoudre la glace de viande, ajoutez
sel et paprika au goût ; laissez cuire.

Roulez les côtelettes dans la mie de pain, de façon à la bien faire adhérer ;
faites-les cuire incomplètement sur le gril, puis achevez leur cuisson, pendant
cinq minutes dans la sauce.

Au dernier moment, montez la sauce à la crème ; goûtez ; l'assaisonnement doit
être un peu chaud, mais sans excès.

Servez, en envoyant en même temps un légumier de betteraves au beurre, \hyperlink{p0780}{p. \pageref{pg0780}}.

\section*{\centering Côtelettes de veau braisées, au parmesan.}
\phantomsection
\addcontentsline{toc}{section}{ Côtelettes de veau braisées, au parmesan.}
\index{Côtelettes de veau braisées, au parmesan}

Pour quatre personnes prenez :

\medskip

\footnotesize
\begin{longtable}{rrrp{16em}}
    600 & grammes & de & bon vin blanc,                                                                   \\
    100 & grammes & de & beurre,                                                                          \\
    100 & grammes & de & mie de pain rassis tamisée,                                                      \\
     80 & grammes & de & parmesan râpé,                                                                   \\
        &         &  4 & côtelettes de veau de 2 centimètres d'épaisseur
                         et pesant chacune 350 grammes environ,                                           \\
        &         &  2 & oignons,                                                                         \\
        &         &    & sel et poivre.                                                                   \\
\end{longtable}
\normalsize

Foncez une casserole avec {\ppp50\mmm} grammes de beurre et faites dorer dedans les
côtelettes ; salez, poivrez des deux côtés ; enlevez les côtelettes ;
remplacez-les par les oignons hachés très fin ; laissez-leur prendre légèrement
couleur ; puis remettez la viande, mouillez avec le vin qui ne doit pas la
recouvrir ; étalez sur les côtelettes le parmesan et la mie de pain mélangés,
mettez par-dessus le reste du beurre coupé en petits morceaux ; couvrez et
laissez cuire à petit feu pendant une heure environ. Arrosez pendant la cuisson
et prenez soin que la viande ne s'attache pas au fond de la casserole.

Au dernier moment, réduisez la sauce, qui doit être courte.

Dressez les côtelettes sur un plat, masquez-les avec la sauce et servez, en
envoyant en même temps, dans un légumier, des petits pois, des épinards ou des
nouilles, par exemple.

\section*{\centering Côtelettes de veau gratinées.}
\phantomsection
\addcontentsline{toc}{section}{ Côtelettes de veau gratinées.}
\index{Côtelettes de veau gratinées}

Pour huit personnes prenez :

\medskip

\footnotesize
\begin{longtable}{rrrp{16em}}
    300 & grammes & de & crème,                                                                           \\
    275 & grammes & de & beurre,                                                                          \\
    240 & grammes & de & chapelure,                                                                       \\
     30 & grammes & d' & oignons,                                                                         \\
      4 & grammes & de & persil haché,                                                                    \\
        &         &  8 & côtelettes de veau à chair blanche,                                              \\
        &         &  2 & belles cervelles de veau,                                                        \\
        &         &  2 & œufs frais,                                                                      \\
        &         &  2 & jaunes d'œufs frais,                                                             \\
        &         &    & vinaigre,                                                                        \\
        &         &    & muscade,                                                                         \\
        &         &    & sel et poivre.                                                                   \\
\end{longtable}
\normalsize

Parez les côtelettes, battez-les, raccourcissez-en les manches. Trempez les
côtelettes dans les œufs entiers battus, roulez-les dans la chapelure.

Foncez une sauteuse avec {\ppp150\mmm} grammes de beurre, chauffez, mettez dedans les
côtelettes et faites-les cuire en les retournant ; assaisonnez légèrement.

En même temps, parez les cervelles et faites-les cuire dans de l'eau salée
vinaigrée et aromatisée ({\ppp3\mmm} à {\ppp4\mmm} minutes d'ébullition, {\ppp10\mmm} à {\ppp12\mmm} minutes de pochage).

Coupez tout le tour des cervelles de façon à ne conserver que deux masses
centrales homogènes représentant à peu près la moitié de l'ensemble ; faites-en
huit escalopes ; réservez les déchets.

Préparez la sauce : faites revenir les oignons hachés fin et le persil dans le
reste du beurre ; ajoutez les déchets de cervelle ; triturez et passez le
tout ; mettez ensuite la crème que vous aurez fait aigrir un peu en la tenant
au chaud ou en l'additionnant d'un filet de vinaigre ; assaisonnez avec sel,
poivre et muscade au goût ; mélangez bien jusqu'à épaississement ; puis liez la
sauce, hors du feu, avec les deux jaunes d'œufs.

Disposez les côtelettes dans un plat allant au feu ; mettez sur chacune une
escalope de cervelle, un peu de chapelure ; versez dessus la sauce, saupoudrez
avec le reste de la chapelure, poussez au four et faites gratiner rapidement.

Ornez de manchettes les manches des côtes et servez en envoyant en même
temps, dans un légumier, des pointes d'asperges à la crème.

Ces côtes de veau sont tout simplement exquises ; elles doivent être mangées
sans que l'édifice culinaire soit détruit, de façon que les convives trouvent
dans chaque bouchée la gamme complète de la préparation.

\section*{\centering Quasi\footnote{Le quasi de veau est le morceau qui entoure
l'os iliaque, entre la noix pâtissière et le cul de veau ; il est
particulièrement moelleux.} de veau braisé.}
\phantomsection
\addcontentsline{toc}{section}{ Quasi de veau braisé.}
\index{Quasi de veau braisé}

Le quasi de veau, simplement assaisonné de sel et de poivre, est braisé au
beurre, à petit feu, sans aucun mouillement. On le retourne fréquemment et on
l'arrose avec son propre jus.

Lorsqu'il est cuit, on le dresse sur un plat foncé avec le jus dégraissé.

Toutes les garnitures indiquées pour les différentes pièces de veau conviennent
parfaitement pour accompagner le quasi.

\sk

On apprête de la même façon la rouelle de veau.

\section*{\centering Quasi de veau gratiné.}
\phantomsection
\addcontentsline{toc}{section}{ Quasi de veau gratiné.}
\index{Quasi de veau gratiné}

Prenez une belle rouelle de quasi de veau désossé et paré ; assaisonnez-la avec
sel et poivre ; enfermez dedans deux rognons de veau parés ; ficelez.

Mettez le tout, avec du beurre, dans une braisière et faites suer la viande au
four ; ne couvrez pas.

Déglacez avec du chablis, couvrez la braisière et achevez la cuisson à petit feu.
Réduisez la sauce, dégraissez-la.

Saupoudrez alors avec un mélange abondant de gruyère râpé et de mie de pain
rassis tamisée ; mettez par-dessus quelques morceaux de beurre et faites
gratiner.

Servez, en envoyant en même temps des pommes de terre sautées.

Découpez à part le veau et les rognons. Donnez à chaque convive du veau, du
rognon et de la croûte gratinés, en accompagnant le tout avec du jus de
cuisson.

\section*{\centering Quasi de veau mariné, gratiné.}
\phantomsection
\addcontentsline{toc}{section}{ Quasi de veau mariné, gratiné.}
\index{Quasi de veau mariné, gratiné}

Pour six personnes prenez :

\medskip

\footnotesize
\begin{longtable}{rp{2em}rrrp{16em}}
& \multicolumn{3}{r}{2 kilogrammes} & de & quasi de veau non paré, en une tranche de
                                             3 centimètres d'épaisseur environ,                           \\
& &   500 & grammes & de & vin blanc sec,                                                                 \\
& &    50 & grammes & de & chapelure fine,                                                                \\
& &    30 & grammes & de & carotte coupée en rondelles,                                                   \\
& &    25 & grammes & d' & huile d'olive,                                                                 \\
& &    15 & grammes & d' & oignon coupé en rondelles,                                                     \\
& &    10 & grammes & de & sel,                                                                           \\
& &     5 & grammes & de & poivre,                                                                        \\
& &       &         &  2 & clous de girofle,                                                              \\
& &       &         &  1 & gousse d'ail,                                                                  \\
& &       &         &  1 & feuille de laurier,                                                            \\
& &       &         &  1 & brindille de thym,                                                             \\
& &       &         &    & beurre,                                                                        \\
& &       &         &    & persil.                                                                        \\
\end{longtable}
\normalsize

Désossez, parez et ficelez la viande.

Mettez-la pendant une douzaine d'heures dans une marinade composée avec le vin
blanc, l'huile d'olive, la carotte, l'oignon, l'ail, les clous de girofle, le
laurier, le thym, un peu de persil, le sel et le poivre ; retournez-la dans la
marinade.

Foncez de beurre un plat allant au feu ; mettez dedans le quasi de veau,
saupoudrez avec la chapelure, mouillez avec la marinade et faites cuire au
four, à feu doux, pendant une heure et demie, en arrosant souvent.

Les garnitures qui vont le mieux avec le veau ainsi préparé sont les épinards,
la chicorée, l'oseille, les petits pois cuits au beurre ou à la crème, les
betteraves à la crème, etc.

\section*{\centering Noix de veau braisée.}
\phantomsection
\addcontentsline{toc}{section}{ Noix de veau braisée.}
\index{Noix de veau braisée}

Pour huit personnes prenez :

\medskip

\footnotesize
\begin{longtable}{rp{2em}rrrp{16em}}
& \multicolumn{3}{r}{2 kilogrammes} & de & noix de veau parée et lardée,                                  \\
& & 125 & grammes   & d' & armagnac,                                                                      \\
& & 100 & grammes   & de & glace de viande,                                                               \\
& &  30 & grammes   & de & beurre,                                                                        \\
& &   1 & bouteille & de & bourgogne blanc sec,                                                           \\
& &     &           &    & carottes,                                                                      \\
& &     &           &    & oignons,                                                                       \\
& &     &           &    & couenne,                                                                       \\
& &     &           &    & bouquet garni,                                                                 \\
& &     &           &    & sel et poivre.                                                                 \\
\end{longtable}
\normalsize

Mettez le beurre dans une cocote ; faites revenir dedans la noix de veau, des
oignons ciselés, de la couenne et les déchets de veau ; flambez avec
l'armagnac ; mouillez avec le vin, ajoutez ensuite la glace de viande, des
carottes, un bouquet garni ; salez, poivrez. Faites cuire au four, à très petit
feu, pendant quatre heures, en arrosant fréquemment pendant la cuisson,

Dressez la viande sur un plat ; dégraissez la sauce ; passez-la et envoyez-la,
dans une saucière, en même temps que le veau.

Des nouilles aux tomates ou des petits pois aux pointes d'asperges accompagnent
très bien le veau braisé.

\sk

\index{Longe de veau braisée}
\index{Selle de veau braisée}
\index{Carré de veau braisé}
On peut apprêter de même la longe, la selle et le carré de veau.

\section*{\centering Côte de veau braisée, à la purée de tomates.}
\phantomsection
\addcontentsline{toc}{section}{ Côte de veau braisée, à la purée de tomates.}
\index{Côte de veau braisée, à la purée de tomates}

Pour six personnes prenez :

\medskip

\footnotesize
\begin{longtable}{rrrp{16em}}
  1 500 & grammes & de & côte de veau,                                                                    \\
    250 & grammes & de & jus de viande,                                                                   \\
    150 & grammes & de & purée de tomates aromatisée,                                                     \\
    100 & grammes & de & beurre,                                                                          \\
     75 & grammes & de & fine champagne,                                                                  \\
     10 & grammes & de & farine,                                                                          \\
      9 & grammes & d' & huile d'olive,                                                                   \\
      6 & grammes & d’ & échalote hachée,                                                                 \\
        &         &    & bouquet garni,                                                                   \\
        &         &    & sel et poivre.                                                                   \\
\end{longtable}
\normalsize

Graissez une casserole avec l'huile\footnote{L'emploi préliminaire de l'huile
a pour effet d'empêcher le beurre de brûler.}, chauffez, mettez ensuite
{\ppp20\mmm} grammes de beurre, le veau ; faites revenir pendant {\ppp20\mmm}
à {\ppp30\mmm} minutes, puis flambez à la fine champagne, saupoudrez avec la
farine, mouillez avec le jus de viande, ajoutez l'échalote, le bouquet garni,
du sel, du poivre au goût ; laissez mijoter pendant une heure et demie.

Un quart d'heure avant de servir, incorporez à la sauce la purée de tomates et
le reste du beurre.

Servez la côte de veau masquée avec la sauce passée au tamis.

\section*{\centering Côte de veau aux morilles.}
\phantomsection
\addcontentsline{toc}{section}{ Côte de veau aux morilles.}
\index{Côte de veau aux morilles}

Pour quatre personnes prenez :

\medskip

\footnotesize
\begin{longtable}{rrrp{16em}}
    800 & grammes & de & côte de veau,                                                                    \\
    500 & grammes & de & morilles,                                                                        \\
    100 & grammes & de & beurre,                                                                          \\
     50 & grammes & de & glace de viande,                                                                 \\
     50 & grammes & de & fond de veau,                                                                    \\
     30 & grammes & de & vin de Madère,                                                                   \\
      9 & grammes & de & sel,                                                                             \\
      3 & grammes & de & poivre,                                                                          \\
        &         &    & fécule.                                                                          \\
\end{longtable}
\normalsize

Nettoyez, lavez les morilles, séchez-les dans un linge ; fendez-les en deux dans le
sens de leur longueur.

Faites revenir la côte de veau dans le beurre, pendant une demi-heure, salez et
poivrez ; mouillez avec le madère et le fond de veau dans lequel vous aurez
fait dissoudre la glace de viande, ajoutez les morilles, puis continuez la
cuisson de l'ensemble, à feu doux, pendant une demi-heure ; concentrez le jus
et, s'il est nécessaire, épaississez-le avec un peu de fécule.

Servez la côte masquée avec la sauce et les morilles autour comme garniture.

\section*{\centering Paupiettes de veau au jambon.}
\phantomsection
\addcontentsline{toc}{section}{ Paupiettes de veau au jambon.}
\index{Paupiettes de veau au jambon}

Pour six personnes prenez :

\medskip

\footnotesize
\begin{longtable}{rrrp{16em}}
  1 000 & grammes & de & veau, en six escalopes minces, régulières, sans déchirures,                      \\
    600 & grammes & de & jambon de Bayonne fumé, en six tranches très minces
                         et de dimensions plus petites que celles des escalopes,                          \\
    250 & grammes & de & sauce tomate,                                                                    \\
    200 & grammes & de & champignons de couche,                                                           \\
    150 & grammes & de & beurre,                                                                          \\
     20 & grammes & de & sel blanc,                                                                       \\
     15 & grammes & de & farine,                                                                          \\
      8 & grammes & de & persil haché,                                                                    \\
      2 & grammes & de & poivre fraîchement moulu,                                                        \\
        &         &  1 & belle échalote.                                                                  \\
\end{longtable}
\normalsize

Mettez sur chaque escalope une tranche de jambon, après en avoir enlevé la
couenne et l'excès de gras que vous réserverez.

Préparez une farce de la façon suivante.

Hachez l'échalote, faites-la cuire pendant {\ppp2\mmm} à {\ppp3\mmm} minutes
dans {\ppp20\mmm} grammes de beurre, sans la laisser roussir, ajoutez la
couenne et le gras de jambon réservés, le persil haché, les champignons pelés
et hachés, {\ppp60\mmm} grammes de sauce tomate ; laissez cuire le tout pendant
dix minutes, puis mettez {\ppp30\mmm} grammes de beurre manié avec la farine ;
continuez la cuisson pendant dix autres minutes, en remuant constamment : vous
obtiendrez ainsi une farce suffisamment épaisse.

Étalez sur chaque escalope garnie de jambon une couche de cette farce, laissez
refroidir, puis roulez en paupiettes et ficelez.

Faites revenir les paupiettes, à petit feu, dans le reste du beurre, pendant
une demi-heure, ajoutez le reste de la sauce tomate, le reste de la farce, le
sel, le poivre et continuez la cuisson pendant une demi-heure encore.

Retirez alors les paupiettes, dressez-les sur un plat, après les avoir
débarrassées des ficelles, passez la sauce\footnote{En ajoutant à cette sauce
des fines herbes hachées, on obtient ce qu'on appelle la \label{pg0490}
\hypertarget{p0490}{}sauce italienne.} au tamis à l’aide d’un pilon, masquez-en
les paupiettes, garnissez le plat de tomates farcies de champignons grillés et
servez.

C'est un plat très fin. L'heureuse association des matières premières qui le
composent et le mode de cuisson à petit feu donnent un ensemble admirablement
fondu, qui produit sur le palais des gourmands l'effet qu'une symphonie de
Beethoven produit sur l'oreille des dilettanti.

\section*{\centering Paupiettes de veau au bacon.}
\phantomsection
\addcontentsline{toc}{section}{ Paupiettes de veau au bacon.}
\index{Paupiettes de veau au bacon}

Pour quatre personnes prenez :

\medskip

\footnotesize
\begin{longtable}{rrrp{16em}}
    500 & grammes & de & noix de veau,                                                                    \\
    300 & grammes & de & bacon,                                                                           \\
    300 & grammes & de & fond de veau,                                                                    \\
    100 & grammes & d' & oignons,                                                                         \\
     30 & grammes & de & beurre,                                                                          \\
     20 & grammes & d' & échalotes,                                                                       \\
        &         &    & bouquet de persil et céleri,                                                     \\
        &         &    & farine,                                                                          \\
        &         &    & sel et poivre.                                                                   \\
\end{longtable}
\normalsize

Coupez le veau en quatre tranches minces ; parez-les ; recouvrez les trois
quarts de leur surface avec de minces lames de bacon ; salez, poivrez ; roulez
en paupiettes. Ficelez-les ; passez-les dans de la farine.

Faites revenir légèrement dans le beurre le reste du bacon coupé en petits dés,
les déchets de veau, les oignons, les échalotes ; mettez ensuite les paupiettes
et laissez-les dorer, puis mouillez avec le fond de veau, ajoutez le bouquet et
faites braiser à petit feu pendant une heure et demie environ.

Dressez les paupiettes, débarrassées de leur ficelle, sur un plat, passez
dessus la sauce et servez en envoyant en même temps des pommes de terre
écrasées et gratinées ou des pommes de terre en purée.

Le bacon, avec son goût de fumé, relève beaucoup le plat.

\section*{\centering Paupiettes de veau au pâté de lièvre.}
\phantomsection
\addcontentsline{toc}{section}{ Paupiettes de veau au pâté de lièvre.}
\index{Paupiettes de veau au pâté de lièvre}

Pour six personnes prenez :

\medskip

\footnotesize
\begin{longtable}{rrrp{16em}}
    800 & grammes & de & veau, en six escalopes minces, régulières, sans déchirures,                      \\
    400 & grammes & de & pâté de lièvre, en terrine, pp. \hyperlink{p0672}{\pageref{pg0672}}, 
                         \hyperlink{p0673}{\pageref{pg0673}}, coupé en six tranches,                      \\
    400 & grammes & de & vin de Porto blanc,                                                              \\
    200 & grammes & de & champignons de couche,                                                           \\
    100 & grammes & de & chapelure,                                                                       \\
     40 & grammes & de & beurre,                                                                          \\
     15 & grammes & de & farine,                                                                          \\
     10 & grammes & de & glace de viande dissoute dans un peu de bouillon,                                \\
        &         &  1 & jaune d'œuf frais,                                                               \\
        &         &    & jus d'un citron,                                                                 \\
        &         &    & sel et poivre.                                                                   \\
\end{longtable}
\normalsize

Mettez sur chaque escalope une tranche de pâté ; roulez-les en paupiettes,
ficelez-les et passez-les dans la chapelure mélangée avec le jaune d'œuf.

Faites revenir les paupiettes à petit feu dans {\ppp30\mmm} grammes de beurre, pendant
une demi-heure, mouillez avec le porto et la glace de viande dissoute dans le
bouillon, maniez la farine avec le reste du beurre, ajoutez-la à la sauce et
laissez mijoter pendant une heure.

Un quart d'heure avant la fin, mettez les champignons préalablement épluchés
et passés dans du jus de citron.

Concentrez le jus de cuisson, dégraissez-le, goûtez, ajoutez sel et poivre s'il
est nécessaire.

Dressez les paupiettes sur un plat, garnissez avec les champignons, masquez
avec la réduction et servez.

\section*{\centering Grenadins\footnote{
\index{Définition des grenadins}
\index{Grenadins (Définition des)}
On désigne sous le nom de grenadins des
tranches minces, c’est-à-dire des escalopes de viande, piquées de lard.
\protect\endgraf
On fait des grenadins de veau, de dinde, de volaille, de chevreuil, de poisson,
etc.} de veau au jus, sauce à la crème.}
\phantomsection
\addcontentsline{toc}{section}{ Grenadins de veau au jus, sauce à la crème.}
\index{Grenadins de veau au jus, sauce à la crème}

Pour quatre personnes prenez :

\medskip

\footnotesize
\begin{tabular}{@{}lrrrp{16em}}
\normalsize1°\footnotesize &     &         &  8 & petites escalopes de veau,                              \\
   &     &         &    & fond de veau concentré,                                                         \\
   &     &         &    & lard à piquer,                                                                  \\
   &     &         &    & fécule,                                                                         \\
   &     &         &    & sel et poivre,                                                                  \\
   &     &         &    &                                                                                 \\
\normalsize 2° & \multicolumn{4}{l}{\normalsize   pour la sauce :}                                        \\
\footnotesize
   &     &         &    &                                                                                 \\
   & 250 & grammes & de & crème épaisse,                                                                  \\
   &  65 & grammes & de & beurre,                                                                         \\
   &  10 & grammes & de & farine,                                                                         \\
   &     &         &    & lait,                                                                           \\
   &     &         &    & sel et poivre.                                                                  \\
\end{tabular}
\normalsize

\medskip

Piquez chaque escalope de trois ou quatre petits lardons assaisonnés,
salez-les, poivrez-les. Mettez-les dans le fond de veau concentré et faites-les
cuire doucement en les retournant fréquemment.

En même temps, préparez une sauce à la crème de la façon suivante : mettez dans
une casserole le beurre et la farine : faites cuire en tournant sans laisser
roussir ; mouillez avec un peu de lait, puis ajoutez la crème par petites
quantités ; salez, poivrez au goût ; laissez mijoter jusqu'à ce que la sauce
ait pris la consitance d'une sauce mayonnaise.

(Je n'indique pas la proportion de lait, parce qu'en réalité tout dépend de la
consistance de la crème : si elle est liquide, le lait est inutile ; si elle
est très épaisse, et cela est préférable, le mouillement préalable avec un peu
de lait a pour résultat de l'empêcher de tourner, ce qui pourrait arriver avec
une cuisinière inexpérimentée qui mettrait trop brusquement la crème.)

Versez la sauce dans un plat, dressez dessus les grenadins, tenez le tout au
chaud. Donnez du corps au jus de cuisson des grenadins avec un peu de fécule,
masquez-en les grenadins et servez.

Je connais des gens auxquels ce plat a fait aimer le veau.

\sk

On peut préparer de la même façon du ris de veau ou des aiguillettes de
poulet ; c'est également très agréable.

\section*{\centering Quasi de veau braisé au lait.}
\phantomsection
\addcontentsline{toc}{section}{ Quasi de veau braisé au lait.}
\index{Quasi de veau braisé au lait}

Pour six personnes prenez :

\medskip

\footnotesize
\begin{longtable}{rrrp{16em}}
  1 500 & grammes & de & quasi de veau paré,                                                              \\
    700 & grammes & de & lait,                                                                            \\
    350 & grammes & d' & os de veau,                                                                      \\
    200 & grammes & de & carottes,                                                                        \\
    125 & grammes & de & champignons de couche,                                                           \\
    125 & grammes & de & beurre,                                                                          \\
     60 & grammes & d' & oignons,                                                                         \\
     25 & grammes & d’ & échalotes,                                                                       \\
     20 & grammes & de & farine,                                                                          \\
     12 & grammes & de & sel,                                                                             \\
     10 & grammes & de & persil,                                                                          \\
      2 & grammes & de & poivre,                                                                          \\
        &         &  1 & feuille de laurier,                                                              \\
        &         &    & un peu de thym,                                                                  \\
        &         &    & glace de viande,                                                                 \\
        &         &    & gelée de groseilles (facultatif).                                                \\
\end{longtable}
\normalsize

Épluchez carottes, champignons, oignons et échalotes.

Mettez dans une braisière {\ppp75\mmm} grammes de beurre, le veau et les os, faites
dorer, salez, poivrez, ajoutez la glace de viande, couvrez et laissez cuire
à petit feu pendant une heure.

En même temps, faites bouillir le lait, mettez dedans les carottes, les
champignons, les oignons, les échalotes, le laurier, le thym, le persil ;
laissez cuire à petit feu pendant une heure de façon à obtenir un demi-litre de
lait aromatisé. Passez-le.

Maniez la farine avec le reste du beurre, mouillez avec le lait, de manière
à obtenir une sauce bien liée, sans grumeaux.

Retirez la viande et les os de la braisière, passez le jus ; puis remettez
viande et jus dans la braisière, ajoutez-y la sauce ; laissez cuire le tout
pendant une demi-heure.

Au moment de servir, délayez dans la sauce un peu de gelée de groseilles (pas
plus de {\ppp10\mmm} grammes), si vous l’aimez.

Ce plat, qui peut paraître bizarre, relève de la cuisine danoise : je l'ai
goûté à Copenhague et il n'est pas sans charme.

\section*{\centering Épaule de veau braisée au chablis.}
\phantomsection
\addcontentsline{toc}{section}{ Épaule de veau braisée au chablis.}
\index{Épaule de veau braisée au chablis}

Pour six à huit personnes prenez :

\medskip

\footnotesize
\begin{longtable}{rrrp{16em}}
  2 500 & grammes & d' & épaule de veau, à chair fine, blanche et grasse,                                 \\
    300 & grammes & de & chablis,                                                                         \\
     50 & grammes & de & beurre,                                                                          \\
        &         &  4 & carottes moyennes,                                                               \\
        &         &  4 & oignons moyens,                                                                  \\
        &         &  3 & tomates moyennes,                                                                \\
        &         &  1 & feuille de laurier,                                                              \\
        &         &    & fécule,                                                                          \\
        &         &    & paprika,                                                                         \\
        &         &    & sel et poivre.                                                                   \\
\end{longtable}
\normalsize

Désossez l'épaule, roulez-la ; ficelez-la : réservez les os.

Faites revenir doucement l'épaule, les carottes et les oignons dans le beurre ;
assaisonnez avec sel et poivre ; mettez le tout dans une braisière ; ajoutez
les os, les tomates et le laurier ; mouillez avec le chablis et laissez braiser
au four, à petit feu, pendant quatre heures. Arrosez de temps en temps avec la
cuisson.

Retirez la viande ; dressez-la sur un plat et tenez-la au chaud.

Dégraissez la cuisson, passez-la, concentrez-la, liez-la avec un peu de fécule
en donnant un bouillon à l'ensemble ; goûtez et complétez l'assaisonnement avec
un peu de paprika.

Versez la sauce sur la viande et servez en envoyant en même temps des nouilles
ordinaires ou des nouilles à la tomate.

\sk

On peut faire cuire de même l'épaule farcie. Un hachis de porc frais, bacon,
rognon de veau, champignons, échalote et persil, assaisonné avec sel et poivre,
constitue une excellente farce.

\section*{\centering Épaule de veau farcie, braisée.}
\phantomsection
\addcontentsline{toc}{section}{ Épaule de veau farcie, braisée.}
\index{Épaule de veau farcie, braisée}
\index{Farce pour veau}
\index{Épaule de veau farcie, braisée au chablis}

Désossez l'épaule, battez la chair de l'intérieur, assaisonnez-la et
garnissez-la d'une farce faite avec du rognon de veau, de la langue, du jambon,
des champignons, une belle truffe, du persil, un peu de ciboule hachée, liée
avec des jaunes d'œufs frais.

Roulez et ficelez l'épaule, puis faites-la braiser doucement dans un bon fond
de veau et volaille, à court mouillement, et en arrosant fréquemment pendant la
cuisson.

Disposez l'épaule sur un plat, dégraissez la cuisson, passez-la et servez-la,
dans une saucière, en même temps que l'épaule.

Les choux-fleurs, les épinards, la chicorée, les champignons, les petits pois,
les carottes, les pommes de terre, les jardinières de légumes sont de bonnes
garnitures pour pièces de veau braisées.

\sk

On apprêtera de même la longe de veau et la poitrine de veau farcies.

\section*{\centering Jarret de veau au jus.}
\phantomsection
\addcontentsline{toc}{section}{ Jarret de veau au jus.}
\index{Jarret de veau au jus}

Pour quatre personnes prenez :

\medskip

\footnotesize
\begin{longtable}{rrrp{16em}}
  1 500 & grammes & de & jarret de veau, en quatre rouelles de 3 à 4 centimètres d'épaisseur,              \\
    750 & grammes & de & fond de veau,                                                                    \\
    500 & grammes & de & tomates,                                                                         \\
    300 & grammes & de & vin blanc,                                                                       \\
    150 & grammes & d' & oignons hachés,                                                                  \\
     60 & grammes & de & beurre,                                                                          \\
        &         &  1 & bouquet garni,                                                                   \\
        &         &    & farine,                                                                          \\
        &         &    & persil,                                                                          \\
        &         &    & citron,                                                                          \\
        &         &    & sel et poivre.                                                                   \\
\end{longtable}
\normalsize

Passez les rouelles de veau dans de la farine ; faites-les revenir dans le
beurre ; retirez-les et remplacez-les par les oignons que vous laisserez
dorer ; mettez ensuite les tomates pelées, épépinées et concassées, la viande
revenue, le bouquet garni, du zeste de citron râpé, au goût ; mouillez avec le
fond de veau et le vin blanc. Laissez mijoter jusqu'à ce que la sauce ait une
bonne consistance ; deux heures et demie sont généralement nécessaires pour
obtenir ce résultat. Goûtez et complétez l'assaisonnement, s'il y a lieu, avec
sel, poivre et jus de citron.

Disposez les rouelles de jarret sur un plat ; masquez-les avec la sauce passée,
saupoudrez de persil blanchi haché et servez en envoyant en même temps soit du
riz sec, soit du riz sauté au beurre, ou encore des épinards au jus.

\sk

Le jarret de veau, ainsi préparé, ne diffère d’un plat de la cuisine milanaise
connu sous le nom d' « osso buco » que par l'addition du vin et du fond de veau
qui le bonifient.

\sk

En remplaçant le jarret par de l'épaule et le citron par de l'ail, on obtient le
plat classique connu sous le nom de « veau Marengo ».

\sk

On pourra préparer, dans le même esprit, du tendron de veau.

\section*{\centering Ragoût de veau et issues de veau aux champignons de couche ou aux morilles.}
\phantomsection
\addcontentsline{toc}{section}{ Ragoût de veau et issues de veau aux champignons de couche ou aux morilles.}
\index{Ragoût de veau et issues de veau aux champignons de couche ou aux morilles}

Pour huit personnes prenez :

\medskip

\footnotesize
\begin{longtable}{rrrp{16em}}
  1 000 & grammes & de & quasi ou de sous-noix de veau,                                                   \\
    500 & grammes & de & champignons de couche ou de morilles,                                            \\
    125 & grammes & de & beurre,                                                                          \\
     20 & grammes & de & farine,                                                                          \\
        & 1 litre & de & gelée de veau et de volaille, \hyperlink{p0418}{p. \pageref{pg0418}},            \\
  1/2 & bouteille & de & sauternes,                                                                       \\
        &         & 8 & foies de poulets,                                                                 \\
        &         & 2 & rognons de veau,                                                                  \\
        &         & 1 & beau ris de veau,                                                                 \\
        &         &   & carottes,                                                                         \\
        &         &   & oignons,                                                                          \\
        &         &   & bouquet garni,                                                                    \\
        &         &   & jus de citron,                                                                    \\
        &         &   & sel et poivre.                                                                    \\
\end{longtable}
\normalsize

Parez le ris, mettez-le à dégorger dans de l'eau un peu salée pendant une heure.

Faites revenir doucement veau, carottes et oignons dans {\ppp40\mmm} grammes de beurre,
salez, poivrez, mouillez avec le sauternes, ajoutez la gelée de veau et
volaille, le bouquet garni ; laissez cuire à petit feu pendant une heure
environ.

Épluchez les champignons, passez-les dans du jus de citron, ou nettoyez
soigneusement les morilles.

Faites dorer légèrement le ris dans du beurre pendant un quart d'heure ; passez
les rognons pendant quelques minutes dans le même beurre ; ajoutez ris et
rognons dans la casserole contenant le veau. Laissez cuire ensemble pendant un
quart d'heure, puis mettez les champignons ou les morilles et continuez la cuisson
pendant un quart d'heure encore.

Passez les foies de poulets dans le beurre qui à servi à faire revenir le ris,
ajoutez-les à l'ensemble et laissez cuire encore pendant cinq minutes environ.

Faites un roux avec le reste du beurre et la farine, mouillez avec le jus de
cuisson, réduisez à bonne consistance, goûtez et complétez l'assaisonnement sil
est nécessaire avec sel, poivre et jus de citron. au goût.

Découpez veau, ris et rognons ; dressez ces trois éléments sur un plat,
entourez-les avec les foies de poulets, les champignons ou les morilles,
masquez le tout avec la sauce passée et servez.

C'est un ragoût royal.

\section*{\centering Ragoût de veau aux écrevisses.}
\phantomsection
\addcontentsline{toc}{section}{ Ragoût de veau aux écrevisses.}
\index{Ragoût de veau aux écrevisses}

Pour quatre personnes prenez :

\medskip

\footnotesize
\begin{longtable}{rrrp{16em}}
  1 000 & grammes & d' & épaule de veau,                                                                  \\
    125 & grammes & de & fond de veau,                                                                    \\
     90 & grammes & de & beurre,                                                                          \\
     10 & grammes & de & farine,                                                                          \\
        &         & 24 & écrevisses,                                                                      \\
        &         &  1 & chou-fleur,                                                                      \\
        &         &  1 & belle carotte,                                                                   \\
        &         &  1 & petite branche de céleri,                                                        \\
        &         &  1 & petit poireau,                                                                   \\
        &         &    & pointes d'asperges,                                                              \\
        &         &    & zeste de citron,                                                                 \\
        &         &    & persil,                                                                          \\
        &         &    & sel et poivre.                                                                   \\
\end{longtable}
\normalsize

Faites revenir dans {\ppp40\mmm} grammes de beurre le veau, la carotte coupée en
morceaux, le céleri et le poireau : saupoudrez avec la farine : mouillez avec
le fond de veau ; ajoutez du zeste de citron et du persil, salez, poivrez ;
laissez cuire.

En même temps, faites cuire les écrevisses dans un bon court-bouillon, le
chou-fleur et les pointes d'asperges dans de l'eau salée.

Décortiquez les écrevisses ; tenez les queues au chaud ; préparez un beurre
d'écrevisses avec le reste du beurre et les parures.

Dressez la viande au milieu d'un plat foncé avec le jus de cuisson passé ;
masquez-la avec le beurre d'écrevisses ; disposez autour comme garniture les
queues d'écrevisses, les pointes d'asperges, le chou-fleur et servez.

\section*{\centering Blanquette de veau.}
\phantomsection
\addcontentsline{toc}{section}{ Blanquette de veau.}
\index{Blanquette de veau}

On peut préparer la blanquette de veau soit avec de la poitrine, nommée
vulgairement tendron, soit avec du rôti (pointe de culotte ou noix patissière).

Voici une formule avec du tendron.

\medskip

Pour quatre personnes prenez :

\footnotesize
\begin{longtable}{rrrp{16em}}
  1 000 & grammes & de & poitrine de veau coupée en morceaux,                                             \\
    250 & grammes & de & champignons de couche,                                                           \\
    100 & grammes & de & crème,                                                                           \\
\end{longtable}
\normalsize

\medskip

\footnotesize
\begin{longtable}{rrrp{16em}}
  100 & grammes & de & beurre,                                                                            \\
   25 & grammes & de & farine,                                                                            \\
   20 & grammes & de & sel,                                                                               \\
    2 & grammes & de & poivre fraîchement moulu,                                                          \\
      &         & 10 & petits oignons,                                                                    \\
      &         &  3 & jaunes d'œufs frais,                                                               \\
      &         &  1 & bouquet garni,                                                                     \\
      &         &    & bouillon de veau,                                                                  \\
      &         &    & jus de citron,                                                                     \\
      &         &    & persil.                                                                            \\
\end{longtable}
\normalsize

Faites blondir la farine dans le beurre, puis mettez les morceaux de veau, les
oignons, le bouquet garni, le sel, le poivre, mouillez avec suffisamment de
bouillon de veau pour couvrir la viande ; laissez cuire pendant une heure et
demie, à petit feu. Enlevez la viande, passez le jus, dégraissez-le. Remettez
dans la casserole la viande et le jus passé, ajoutez les champignons épluchés,
continuez la cuisson de l’ensemble pendant une vingtaine de minutes ; liez la
sauce avec les jaunes d'œufs délayés dans la crème, relevez avec du jus de
citron, chauffez sans laisser bouillir.

Dressez la viande sur un plat, entourez-la avec les champignons, versez dessus
la sauce, saupoudrez ou non de persil haché et servez.

\sk

En opérant avec du veau rôti, il convient, après avoir fait blondir la farine
dans le beurre, de remplacer le bouillon de veau ordinaire par du fond
blanc\footnote{ 
\index{Fond blanc}
Pour faire un litre de fond blanc, prenez :
\protect
\begin{tabular}{rrrp{16em}}
\hspace{4em} 1 250 & grammes & de & jarret et de bas morceaux de veau,                                    \\
\hspace{4em}   100 & grammes & de & carottes,                                                             \\
\hspace{4em}    50 & grammes & d' & oignons,                                                              \\
\hspace{4em}    50 & grammes & de & sel,                                                                  \\
\hspace{4em}    20 & grammes & de & poireau,                                                              \\
\hspace{4em}    15 & grammes & de & céleri,                                                               \\
\hspace{4em}     2 & grammes & de & poivre,                                                               \\
\hspace{4em}       & 1 litre & d' & eau,                                                                  \\
\hspace{4em}       &         &  1 & clou de girofle,                                                      \\
\hspace{4em}       &         &  1 & petit bouquet garni (persil, thym et laurier),                        \\
\hspace{4em}       &         &    & beurre ou graisse.                                                    \\
\end{tabular}
\smallskip
\protect\endgraf
Faites revenir légèrement carottes, oignons, poireau, céleri dans un peu de
beurre ou de graisse ; enlevez les légumes et remplacez-les par le veau coupé
en morceaux ; laissez dorer ; égouttez la graisse. Mouillez alors avec un peu
d'eau, laissez tomber à glace deux ou trois fois en déglaçant chaque fois avoc
un peu d'eau bouillante. Déglacez définitivement avec le reste de l'eau,
ajoutez les légumes revenus, le bouquet, le clou de girofle, le sel et le
poivre ; faites cuire pendant {\ppp5\mmm} à {\ppp6\mmm} heures en maintenant
toujours le volume du liquide à un litre par des additions successives d'eau
bouillante. Écumez, dégraissez et passez au chinois.}, car la viande employée
ne donnera pas de jus ; on ajoutera seulement les champignons qu'on laissera
cuire pendant une vingtaine de minutes. Au dernier moment, on mettra dans la
casserole le veau rôti, coupé en tranches, pour le réchauffer, sans faire
bouillir ; on liera la sauce avec les jaunes d'œufs délayés dans la crème, on
la relèvera avec du jus de citron, on dressera et on servira comme ci-dessus.

\sk

La blanquette faite avec du rôti de veau a meilleur aspect que celle qui est
faite avec du tendron ; cependant, ce morceau donne sous la dent une sensation
particulière qui a bien son charme. Aussi, pour réunir tous les suffrages, je
conseille de préparer la blanquette mi-partie avec du tendron, mi-partie avec
du rôti de veau, en observant les recommandations formulées dans les deux cas.

\sk

\index{Agneau on blanquette}
\index{Blanquette d'agneau de lait}
\index{Blanquette de chevreau}
\index{Chevreau en blanquette}
On apprêtera d'une façon analogue la blanquette d'agneau de lait et celle de
chevreau, mais la durée de la cuisson sera naturellement moins longue.

\section*{\centering Fricandeau.}
\phantomsection
\addcontentsline{toc}{section}{ Fricandeau.}
\index{Fricandeau}

\label{pg0499} \hypertarget{p0499}{}

Pour six à huit personnes prenez :

\medskip

\footnotesize
\begin{longtable}{rrrp{18em}}
  1 500 & grammes & de & noix ou de sous-noix de veau, ou encore une belle
                         rouelle de veau, désossée, du même poids,                                        \\
    500 & grammes & de & bouillon,                                                                        \\
    200 & grammes & de & bon vin blanc,                                                                   \\
    200 & grammes & de & lard à piquer,                                                                   \\
        &         &    & carotte,                                                                         \\
        &         &    & oignon,                                                                          \\
        &         &    & beurre,                                                                          \\
        &         &    & sel et poivre.                                                                   \\
\end{longtable}
\normalsize

Piquez la viande de fins lardons assaisonnés, mettez-la ensuite dans une
braisière avec un peu de beurre ; retournez-la en tous sens pendant quelques
instants sans lui laisser prendre couleur, ajoutez quelques rondelles de
carotte et d'oignon, poivrez, mouillez avec {\ppp100\mmm} grammes de bouillon
et {\ppp100\mmm} grammes de vin blanc ; faites bouillir jusqu'à réduction, en évitant que
la viande s'attache à la casserole. Lorsque le jus de cuisson a pris une
consistance sirupeuse, mouillez de nouveau avec le reste du bouillon et le
reste du vin blanc, puis laissez cuire, à petit feu, en casserole entr'ouverte,
pendant {\ppp1\mmm} heure un quart à {\ppp1\mmm} heure et demie. Arrosez
pendant la cuisson.

Goûtez et complétez l'assaisonnement avec sel et poivre s'il est nécessaire.

Un peu avant la fin, poussez la casserole au four et glacez la viande,
c'est-à-dire faites-lui prendre une belle teinte jaunâtre brillante. Arrosez
fréquemment pendant cette dernière opération.

\index{Fricandeau à l'oseille}
\index{Fricandeau à la chicorée}
\index{Fricandeau aux épinards}
Dressez le fricandeau sur un plat, passez dessus la sauce et servez en envoyant
en même temps un légumier d'oseille sans jus ou encore des épinards ou de la
chicorée à la crème.

\sk

\index{Fricandeau aux champignons de couche}
\index{Fricandeau aux morilles}
Comme variantes, on pourra servir le fricandeau garni de tartelettes de morilles
à la crème, ou encore de champignons ou de morilles farcis.

\sk

\index{Fricandeau d'alose}
\index{Fricandeau d'esturgeon}
\index{Fricandeau de thon}
On peut apprêter de même de l'alose, de l’esturgeon, du thon, etc., avec cette
différence qu'ici on fera le mouillement avec du bouillon de poisson et qu'on
augmentera un peu la quantité du vin.

\section*{\centering Fricandeau aux cèpes.}
\phantomsection
\addcontentsline{toc}{section}{ Fricandeau aux cèpes.}
\index{Fricandeau aux cèpes}

Mettez dans une casserole des poireaux, des carottes, des oignons, de l'ail au
goût, du sel, du poivre, un os de veau, de la graisse d'oie ; faites rissoler,
flambez avec un peu d'armagnac, ajoutez de la sauce tomate, puis mouillez avec
du bouillon de veau en quantité suffisante pour permettre ultérieurement de
couvrir la viande.

Laissez cuire ; passez le jus.

Piquez de la noix de veau avec du jambon gras, puis faites-la cuire en opérant
comme il est dit dans la formule précédente, mais en mouillant avec le jus
préparé ci-dessus.

Dressez le fricandeau sur un plat, masquez-le avec la sauce très réduite,
garnissez avec des tartelettes ou des petites timbales de cèpes à la crème, et
servez.

\section*{\centering Goulach de veau.}
\phantomsection
\addcontentsline{toc}{section}{ Goulach de veau.}
\index{Goulach de veau}

Pour cinq personnes prenez :

\medskip

\footnotesize
\begin{longtable}{rrrp{16em}}
  1 000 & grammes & de & veau, de préférence un morceau maigre,                                           \\
    250 & grammes & de & crème,                                                                           \\
    100 & grammes & d' & oignons,                                                                         \\
     60 & grammes & de & vin blanc,                                                                       \\
     60 & grammes & de & beurre,                                                                          \\
     30 & grammes & de & saindoux,                                                                        \\
     30 & grammes & de & jus de viande,                                                                   \\
      5 & grammes & de & sel,                                                                             \\
      2 & grammes & de & paprika,                                                                         \\
        &         &    & jus de citron.                                                                   \\
\end{longtable}
\normalsize

Émincez les oignons, mettez-les dans une casserole avec le beurre et le
saindoux ; faites-leur prendre couleur ; puis ajoutez la viande coupée en
morceaux carrés de {\ppp4\mmm} à {\ppp5\mmm} centimètres de côté ; mouillez avec le jus,
assaisonnez avec le sel et le paprika ; laissez mijoter pendant une heure.

Dix minutes avant de servir, mettez le vin. puis la crème aigrie avec du jus de
citron, chauffez sans laisser bouillir, goûtez pour l'assaisonnement et servez
en envoyant en même temps un plat de nouilles ou un plat de riz.

\sk

On peut préparer de même un goulach de porc, mais on ajoutera alors
à l'assaisonnement un peu de cumin et de marjolaine.

Le goulach de porc est généralement accompagné par de la choucroute.

\sk

On peut aussi faire un goulach de volaille.

La préparation est la même que celle du goulach de veau ; mais, après avoir
coupé le poulet en morceaux, on lui ajoute un cinquième de son poids de lard
coupé en petits cubes.

Ce plat, auquel on donne en Hongrie le nom de \textit{pörckel}, est servi avec
du riz.

\section*{\centering Kalalou à la parisienne.}
\phantomsection
\addcontentsline{toc}{section}{ Kalalou à la parisienne.}
\index{Kalalou à la parisienne}

Le kalalou est un mets créole. C'est un ragoût de viandes et de légumes parmi
lesquels le plus original est le gombo. Il est à peu près impossible de
réaliser en France un kalalou créole ; mais, grâce aux excellentes conserves de
gombos qu'on peut se procurer en toute saison, il est facile de préparer un
plat qui rappellera agréablement le mets original.

Pour huit personnes prenez :

\medskip

\footnotesize
\begin{longtable}{rrrp{16em}}
  1 500 & grammes & de & fond de veau,                                                                    \\
    600 & grammes & de & noix de veau,                                                                    \\
    600 & grammes & de & filet de porc,                                                                   \\
    400 & grammes & de & gombos (deux boîtes),                                                            \\
    300 & grammes & de & purée de tomates concentrée,                                                     \\
     60 & grammes & de & beurre,                                                                          \\
        &         &    & cayenne,                                                                         \\
        &         &    & sel et poivre.                                                                   \\
\end{longtable}
\normalsize

Mettez les gombos dans une passoire ; rafraîchissez-les à l'eau froide.

Coupez le veau et le porc en gros dés ; faites-les revenir dans le beurre ;
mouillez avec {\ppp1\mmm} {\ppp300\mmm} grammes de fond de veau ; laissez cuire
pendant deux heures. Ajoutez alors la moitié des gombos et la purée de
tomates ; continuez la cuisson pendant une demi-heure en évitant que les gombos
s'attachent au fond de la casserole. Pendant la cuisson goûtez et complétez
l'assaisonnement avec cayenne, sel et poivre.

En même temps, chauffez au bain-marie le reste des gombos dans le reste du fond
de veau de manière qu'ils s'imbibent de jus tout en restant entiers ; les
autres, qui ont cuit dans le ragoût, sont fondus et ils ont communiqué leur
goût et leur onctuosité à la sauce.

Au moment de servir, ajoutez au ragoût les gombos entiers et le reste du fond
de veau. Servez, en envoyant à part un légumier de riz sauté au beurre.


\sk

On peut présenter aussi le riz sauté moulé en turban. On mettra le kalalou dans
l'intérieur du turban.

\section*{\centering Aillade\footnote{ Au sens littéral du terme, l'aillade est
une sauce à l'ail ; par extension, on donne ce nom dans le Midi à des plats
fortement alliacés.} de veau.}
\phantomsection
\addcontentsline{toc}{section}{ Aillade de veau.}
\index{Aillade de veau}

Dans le Midi, on adore les aillades et, en particulier, le veau à l'ail. Voici la
formule de l'aillade de veau de Casteljaloux.

\medskip

Pour cinq personnes prenez :

\footnotesize
\begin{longtable}{rrrp{16em}}
  1 000 & grammes & d' & épaule ou de sous-noix de veau,                                                  \\
    300 & grammes & de & purée de tomates aromatisée,                                                     \\
     60 & grammes & de & saindoux,                                                                        \\
     60 & grammes & de & bon jus de viande,                                                               \\
     50 & grammes & d' & ail en gousses,                                                                  \\
      5 & grammes & de & mie de pain rassis tamisée,                                                      \\
        &         &    & sel, poivre.                                                                     \\
\end{longtable}
\normalsize


Faites dorer dans le saindoux, à la casserole, la viande coupée en morceaux
carrés de {\ppp4\mmm} à {\ppp5\mmm} centimètres de côté, puis ajoutez la mie de pain, l'ail, la
purée de tomates, le jus de viande, assaisonnez avec sel et poivre ; laissez
mijoter pendant une heure ; retirez l'ail si vous n'aimez pas le trouver et
servez en envoyant en même temps un plat de riz sauté, par exemple.

\sk

\index{Aillade de mouton}
On peut préparer de même une aillade de mouton.

\section*{\centering Galettes de veau à la moelle de bœuf.}
\phantomsection
\addcontentsline{toc}{section}{ Galettes de veau à la moelle de bœuf.}
\index{Galettes de veau à la moelle de bœuf}

Pour six personnes prenez :

\medskip

\footnotesize
\begin{longtable}{rrrp{16em}}
    300 & grammes & de & noix de veau, hachée fin,                                                        \\
    125 & grammes & de & jambon de Bayonne fumé, haché fin,                                               \\
    100 & grammes & de & moelle de bœuf,                                                                  \\
    100 & grammes & de & mie de pain rassis tamisée,                                                      \\
        &         &    & beurre,                                                                          \\
        &         &    & lait,                                                                            \\
        &         &    & jaune d'œuf frais,                                                               \\
        &         &    & chapelure,                                                                       \\
        &         &    & sel.                                                                             \\
\end{longtable}
\normalsize

Faites pocher la moelle dans de l'eau salée.

Mélangez intimement veau, jambon, moelle, mie de pain ramollie dans du lait et
égouttée, du sel ; préparez avec le tout six galettes plates. Dorez-les au
jaune d'œuf, saupoudrez-les de chapelure, puis faites-les cuire à la poêle dans
du beurre.

Couvrez la poêle, les galettes gonfleront ; retournez-les et, quand elles
auront pris de tous côtés une belle couleur orangée, servez-les.

Ges galettes sont excellentes ; on les accompagne de tomates farcies de
champignons grillés, de purée d'oseille, ou encore de sauce tomate, de sauce
veloutée grasse,

\sk

On prépare la sauce veloutée grasse, ou velouté, de la manière suivante.

\medskip

Pour six personnes prenez :

\footnotesize
\begin{longtable}{rrrp{16em}}
    500 & grammes & de & gelée de veau et de volaille,                                                    \\
     40 & grammes & de & beurre frais,                                                                    \\
     40 & grammes & de & beurre clarifié,                                                                 \\
     30 & grammes & de & farine.                                                                          \\
\end{longtable}
\normalsize

Faites cuire la farine dans le beurre clarifié sans laisser prendre couleur,
mouillez avec la gelée fondue, tournez jusqu'à ébullition. Au premier bouillon,
éloignez un peu la casserole du feu et laissez mijoter pendant une heure. Au
dernier moment, montez la sauce avec le beurre frais coupé en petits morceaux ;
goûtez pour l'assaisonnement.

\section*{\centering Pâté de veau et jambon en croûte.}
\phantomsection
\addcontentsline{toc}{section}{ Pâté de veau et jambon en croûte.}
\index{Pâté de veau et jambon en croûte}
\index{Croûte pour pâtés}
\index{Garniture pour pâtés}

On peut préparer des pâtés en moule ou sans moule. Les premiers sont plus jolis
à l'œil : les seconds permettent l'emploi d'une pâte donnant une croûte plus
fine, mais ils ont l'inconvénient de rendre difficile l'introduction de la
gelée, qu'on sert alors comme garniture, autour du pâté.

Voici une excellente formule de pâté sans moule.

\medskip

Pour huit personnes prenez :

\smallskip

\label{pg504} \hypertarget{p504}{}
1° pour la pâte :

\footnotesize
\begin{longtable}{rp{2em}rrrp{16em}}
& & 400 & grammes & de & farine de blé dur de Hongrie,                                                    \\
& & 150 & grammes & de & beurre,                                                                          \\
& &  80 & grammes & d' & eau,                                                                             \\
& &  15 & grammes & de & sel,                                                                             \\
& &     &         &  2 & jaunes d'œufs frais ;                                                            \\
\end{longtable}
\normalsize

2° pour le corps du pâté :

\footnotesize
\begin{longtable}{rp{2em}rrrp{16em}}
& & 400 & grammes & de & rouelle de veau,                                                                 \\
& & 400 & grammes & de & jambon fumé ou non, au goût,                                                     \\
& & 250 & grammes & de & fines bardes de lard,                                                            \\
& & 100 & grammes & de & cognac,                                                                          \\
& &   1 & gramme  & de & quatre épices,                                                                   \\
& &     &         &  1 & jaune d'œuf frais,                                                               \\
& &     &         &    & sel ;                                                                            \\
\end{longtable}
\normalsize

\index{Farce pour pâtés de viande}
3° pour la farce :

\footnotesize
\begin{longtable}{rp{2em}rrrp{16em}}
& & 250 & grammes & de & foie de veau,                                                                    \\
& & 250 & grammes & de & poitrine de porc maigre,                                                         \\
& & 250 & grammes & de & champignons,                                                                     \\
& & 150 & grammes & de & jambon de Bayonne,                                                               \\
& &  60 & grammes & de & beurre,                                                                          \\
& &  20 & grammes & d' & échalotes,                                                                       \\
& \multicolumn{3}{r}{2 décigrammes} & de & poivre,                                                        \\
& &     &         &  1 & feuille de laurier,                                                              \\
& &     &         &  1 & brindille de thym ;                                                              \\
\end{longtable}
\normalsize

4° pour la gelée :

\footnotesize
\begin{longtable}{rp{2em}rrrp{16em}}
& & 500 &  grammes & de & gîte de bœuf,                                                                   \\
& & 100 &  grammes & de & jarret de veau,                                                                 \\
& & 200 &  grammes & de & vin blanc,                                                                      \\
& & 150 &  grammes & de & carottes,                                                                       \\
& & 100 &  grammes & de & couenne fraîche,                                                                \\
& &  15 &  grammes & de & céleri,                                                                         \\
& &     & 2 litres & d' & eau,                                                                            \\
& &     &          &  2 & blancs d'œufs,                                                                  \\
& &     &          &  1 & abatis de volaille,                                                             \\
& &     &          &  1 & pied de veau,                                                                   \\
& &     &          &  1 & poireau (le blanc seulement),                                                   \\
& &     &          &  1 & oignon piqué d'un clou de girofle,                                              \\
& &     &          &    & os de porc,                                                                     \\
& &     &          &    & madère,                                                                         \\
& &     &          &    & sel, quatre épices, cayenne, au goût.                                           \\
\end{longtable}
\normalsize

Mettez à mariner dans le cognac le veau et le jambon destinés au corps du pâté ;
assaisonnez avec sel au goût et quatre épices.

\smallskip
\textit{Préparation de la pâte}. — Faites une pâte homogène avec les éléments
indiqués : laissez-la reposer au moins trois heures avant de l'employer.

\smallskip
\textit{Préparation de la farce}. — Hachez champignons, échalotes, thym et
laurier : faites revenir le tout dans le beurre. Hachez le foie, la poitrine de
porc et le jambon de Bayonne, passez-les au tamis, ajoutez le poivre, le cognac
de la marinade et le hachis de champignons : mélangez bien.

\smallskip
\textit{Préparation de la gelée}. — Coupez en morceaux le gîte de bœuf, le
jarret et le pied de veau et faites-les revenir avec l'abatis dans un peu de
graisse,

Cassez fin les os de bœuf, de veau et de porc.

Faites dorer, dans un peu de graisse, carottes, céleri, poireau, oignon ; laissez
pincer légèrement.

Foncez une marmite avec la couenne ; mettez dessus les os brisés, les viandes
et les légumes revenus, moins la graisse, assaisonnez avec sel, cayenne et
quatre épices, mouillez avec l'eau et le vin ; laissez cuire pendant six
heures ; dégraissez. Dépouillez pendant la cuisson. un peu avant la fin,
ajoutez plus ou moins de madère au goût. La cuisson achevée, clarifiez avec les
blancs d'œufs.

\smallskip
\textit{Dressage du pâté}. — Abaissez la pâte ; placez l'abaisse sur un linge
saupoudré de farine ; mettez sur cette abaisse une couche de farce, au-dessus
du jambon bardé de lard, puis une autre couche de farce, ensuite le veau
également bardé de lard, finissez par une couche de farce,

Repliez la pâte de façon à former le pâté et ménagez deux ou trois ouvertures
pour l’échappement des gaz. Dorez au jaune d'œuf.

\smallskip
\textit{Cuisson}. — Au four chaud, pendant {\ppp40\mmm} à {\ppp50\mmm} minutes.

\smallskip
\textit{Finissage}. — Coulez dans le pâté un peu de gelée à une température
voisine de son point de solidification ; laissez refroidir.

Dressez le pâté sur un plat : garnissez-en le pourtour avec le reste de la
gelée découpée en losanges ; servez.

\sk

Pour faire le pâté en moule, on préparera une pâte un peu plus ferme que la
précédente en employant, par exemple :

\medskip

\footnotesize
\begin{longtable}{rrrp{16em}}
    500 & grammes & de & farine,                                                                          \\
    150 & grammes & de & beurre,                                                                          \\
    100 & grammes & d' & eau,                                                                             \\
     15 & grammes & de & sel,                                                                             \\
        &         &  3 & jaunes d'œufs.                                                                   \\
\end{longtable}
\normalsize

On préparera la pâte comme ci-dessus et on en chemisera le moule. On garnira et
on finira le pâté comme précédemment. La cuisson demandera ici plus de temps.

\section*{\centering Tourte aux quenelles de godiveau.}
\phantomsection
\addcontentsline{toc}{section}{ Tourte aux quenelles de godiveau.}
\index{Tourte aux quenelles de godiveau}

Les tourtes\footnote{
\index{Définition des tourtes}
\index{Tourtes (Définition des)}
On donne aussi le nom de tourtes à des préparations de
marmelades de fruits, de gelées de fruits et de crèmes servies dans des croûtes
au sirop.}, précurseurs des vol-au-vent, sont des croûtes de forme basse
garnies de ragoûts, de godiveau (farce de veau et de rognon de bœuf) plus ou
moins relevé par de l'échalote, de quenelles de godiveau dans une sauce, etc.
À l'origine, la croûte des tourtes était faite en pâte à pain. Aujourd'hui, on
la fait soit en pâte brisée comme les timbales, à la hauteur près et sans
couvercle, soit en pâte feuilletée avec couvercle.

Voici un exemple de tourte aux quenelles de godiveau dans une sauce béchamel
garnie de champignons, en croûte feuilletée.

\medskip

Pour six à huit personnes prenez :

\medskip

1° pour la croûte :

\medskip

\footnotesize
\begin{longtable}{rrrp{16em}}
    500 & grammes & de & farine,                                                                          \\
    500 & grammes & de & beurre,                                                                          \\
    250 & grammes & d' & eau,                                                                             \\
     10 & grammes & de & sel ;                                                                            \\
\end{longtable}
\normalsize

\index{Garniture pour tourtes}
2° pour la garniture :

\medskip

\footnotesize
\begin{longtable}{rrrp{16em}}
    750 & grammes & de & graisse de rognon de bœuf, épluchée,                                             \\
    500 & grammes & de & noix de veau dénervée,                                                           \\
    250 & grammes & de & petits champignons de couche,                                                    \\
    100 & grammes & de & glace,                                                                           \\
     50 & grammes & de & beurre,                                                                          \\
        &         &  5 & œufs frais,                                                                      \\
        &         &    & farine,                                                                          \\
        &         &    & jus de citron,                                                                   \\
        &         &    & épices,                                                                          \\
        &         &    & sel et poivre ;                                                                  \\
\end{longtable}
\normalsize

3° pour la sauce :

\medskip

\footnotesize
\begin{longtable}{rrrp{16em}}
    750 & grammes & de & sous-noix ou de rouelle de veau.                                                 \\
    250 & grammes & de & beurre,                                                                          \\
    150 & grammes & de & crème épaisse,                                                                   \\
    125 & grammes & de & champignons,                                                                     \\
     75 & grammes & de & farine,                                                                          \\
    & 1 litre 1/2 & de & consommé,                                                                        \\
        &         &  3 & carottes moyennes,                                                               \\
        &         &  2 & oignons moyens,                                                                  \\
        &         &  1 & bouquet garni,                                                                   \\
        &         &    & céleri,                                                                          \\
        &         &    & sel et poivre.                                                                   \\
\end{longtable}
\normalsize

Hachez fin la noix de veau ; assaisonnez-la avec sel, poivre et épices.

Hachez fin la graisse de rognon de bœuf.

Mélangez ces deux éléments : pilez-les dans un mortier en marbre tenu sur
glace ; ajoutez-y un à un trois œufs, en pilant toujours.

Étalez la farce sur un plat tenu sur glace ; laissez-la reposer jusqu'au
lendemain,

Préparez la sauce. Coupez le veau en morceaux gros comme des petites noix et
faites-les revenir dans le beurre avec les oignons, les carottes et du céleri
émincés, sans laisser prendre couleur. Au bout de dix minutes, ajoutez la
farine, tournez pendant cinq minutes, mouillez avec le consommé, mettez les
champignons épluchés et émincés, le bouquet garni, du sel et du poivre ;
mélangez en tournant. Au premier bouillon, éloignez la casserole sur le coin du
fourneau et continuez la cuisson à tout petit feu, pendant deux heures. Écumez
et dégraissez pendant l'opération.

Passez la sauce à l'étamine, remettez-la sur le feu, réduisez-la, ajoutez la
crème épaisse et amenez le tout à la consistance voulue pour masquer une
cuiller. Passez de nouveau.

En même temps, avec les éléments du premier paragraphe préparez une pâte
feuilletée, \hyperlink{p0319}{p. \pageref{pg0319}}. et faites-en une abaisse de
{\ppp3\mmm} centimètres d'épaisseur. Découpez dedans une bande de {\ppp2\mmm}
centimètres de largeur et de {\ppp63\mmm} centimètres de longueur qui puisse
limiter le pourtour de la tourte ; séparez le reste de la pâte en deux parles ;
donnez à chacune deux tours et faites-en deux abaisses ; découpez dans chacune
un disque de {\ppp20\mmm} centimètres de diamètre : l'un formera le fond,
l'autre le couvercle de la tourte.

Disposez l'un des disques sur un plafond, mouillez-en le tour, collez la bande
sur le pourtour de manière qu'elle adhère bien partout, puis mettez au fond un
papier beurré, dessus un tampon de papier blanc qui emplisse le vide et dépasse
l'orifice de quelques centimètres. Posez l'autre disque sur le papier et
soudez-le sur le bord supérieur de la bande, en l'amincissant un peu. Décorez
le couvercle, dorez la croûte au jaune d'œuf et faites cuire au four.

Pendant la cuisson de la croûte, retravaillez au mortier la farce préparée la
veille, en y ajoutant les deux œufs frais qui restent et la glace concassée
finement, jusqu'à ce que le tout soit lisse et consistant.

Étalez la farce sur une table ou sur une planche farinée et faites un essai en
moulant une petite quenelle que vous plongerez ensuite dans de l'eau
bouillante. Si l'appareil est trop ferme, ramollissez-le avec un peu d'eau
glacée ; moulez alors toute la farce en quenelles que vous ferez pocher dans de
l'eau salée bouillante.

Épluchez les petits champignons, passez-les dans du jus de citron et faites-les
cuire dans le beurre pendant un quart d'heure.

Lorsque la croûte est cuite, décollez le couvercle, enlevez le papier ;
emplissez la tourte avec quenelles, sauce, champignons avec leur cuisson et
servez.

\section*{\centering Vol-au-vent au gras.}
\phantomsection
\addcontentsline{toc}{section}{ Vol-au-vent au gras.}
\index{Vol-au-vent au gras}

L'une des meilleures garnitures pour vol-au-vent au gras consiste en un
salpicon de quenelles de godiveau truffé, de cervelles de mouton ou de veau, de
ris de veau ou d'agneau, de crêtes et de rognons de coq, de champignons, de
truffes, le tout réuni dans une sauce suprême.

\clearpage

Pour huit personnes prenez :

\medskip

1° pour la croûte :

\medskip

\footnotesize
\begin{longtable}{rrrp{16em}}
    600 & grammes & de & farine,                                                                          \\
    600 & grammes & de & beurre,                                                                          \\
    250 & grammes & d' & eau,                                                                             \\
     10 & grammes & de & sel,                                                                             \\
        &         &  2 & jaunes d'œufs frais ;                                                            \\
\end{longtable}
\normalsize

2° pour la garniture :

\medskip

\footnotesize
\begin{longtable}{rrrp{16em}}
    150 & grammes & de & crêtes et de rognons de coq,                                                     \\
    125 & grammes & de & champignons tournés,                                                             \\
     25 & grammes & de & beurre,                                                                          \\
        &         & 40 & petites quenelles de godiveau truffé\protect\footnote{ou de volaille.      \\
                          \protect\endgraf
                          Pour préparer des quenelles de volaille, prenez :                         \\
                          \protect\endgraf
                          \begin{tabular}{rrrl}
                          \hspace{4em} 600 & grammes & de & chair de poulet sans peau ni nerfs,     \\
                          \hspace{4em} 250 & grammes & de & ris de veau épluché,                    \\
                          \hspace{4em} 200 & grammes & de & beurre fin,                             \\
                          \hspace{4em} 200 & grammes & de & sauce suprême très concentrée,          \\
                          \hspace{4em}     &         &  5 & jaunes d'œufs frais,                    \\
                          \hspace{4em}     &         &    & sel et épices, au goût.                 \\
                          \end{tabular}
                          \protect\endgraf
                          Pilez vivement la chair de poulet et le ris de veau au mortier,
                          incorporez-y ensuite les jaunes d'œufs et le beurre en plusieurs
                          fois, assaisonnez au goût, puis ajoutez la sauce par petites
                          quantités, triturez encore, amenez à bonne consistance, goûtez,
                          passez l'appareil au tamis et faites-en l'essai.
                          \protect\endgraf
                          Roulez la farce en petites quenelles et pochez-les dans du fond
                          de veau.
                          \protect\endgraf
                          On peut, dans le même esprit, faire toutes sortes de quenelles ;
                          suivant la viande employée, on préparera avec les débris un fond
                          qu'on liera avec un peu de fécule et qu'on réduira à demi-glace,
                          en le tenant très blanc pour les viandes blanches, et en lui faisant
                          prendre couleur s'il s'agit de gibier.
                          \protect\endgraf
                          On pourra aussi faire entrer dans la pâte des quenelles de la crème
                          double, de la cervelle, de la moelle, des champignons, des truffes.
                          Les quenelles ont leur place dans la préparation des vol-au-vent,
                          des timbales ; elles font partie des garnitures ; elles peuvent
                          aussi être servies seules avec sauce adéquate. Suivant leur destination,
                          on les fait plus ou moins grosses et on les tient plus ou moins fermes.},       \\
        &         &  3 & cervelles de mouton,                                                             \\
        &         &  1 & beau ris de veau,                                                                \\
        &         &    & truffes,                                                                         \\
        &         &    & jus de citron ;                                                                  \\
\end{longtable}
\normalsize

3° pour la sauce :

\medskip

\footnotesize
\begin{longtable}{rrrp{16em}}
    200 & grammes & de & madère,                                                                          \\
    150 & grammes & de & glace de volaille,                                                               \\
    125 & grammes & de & crème épaisse,                                                                   \\
     30 & grammes & de & beurre,                                                                          \\
     20 & grammes & de & farine,                                                                          \\
        & 1 litre & de & fond de veau concentré,                                                          \\
        &         &  3 & jaunes d'œufs frais,                                                             \\
        &         &    & essence de champignons,                                                          \\
        &         &    & essence de truffe,                                                               \\
        &         &    & poivre.                                                                          \\
\end{longtable}
\normalsize

Avec les éléments du premier paragraphe, faites une pâte feuilletée ;
donnez-lui huit tours ; puis, avec cette pâte, préparez une croûte de
vol-au-vent comme il est dit \hyperlink{p0319}{p. \pageref{pg0319}}.

Parez le ris, mettez-le à dégorger dans de l'eau fraîche, puis faites-le cuire
dans une partie du fond de veau.

Nettoyez les truffes.

Faites cuire les cervelles dans de l'eau salée, vinaigrée et aromatisée, les
champignons dans du beurre avec du jus de citron, les truffes dans le madère.

Faites pocher les quenelles dans de l’eau salée bouillante.

Blanchissez les crêtes et les rognons de coq.

Tenez tous ces éléments au chaud.

\label{pg0510} \hypertarget{p0510}{}
En même temps, préparez la sauce suprême ; mettez le madère de cuisson des
truffes, la cuisson des champignons et du poivre, au goût, dans une casserole ;
réduisez de moitié. Maniez la farine avec le beurre, laissez-la dorer, mouillez
avec le reste du fond de veau, le jus de cuisson du ris et le madère réduit ;
concentrez, dépouillez pendant la cuisson, ajoutez ensuite la glace de
volaille ; réduisez encore jusqu'à consistance suffisante pour masquer une
cuiller ; parfumez, hors du feu, avec de l'essence de champignons et de
l'essence de truffe au goût ; passez la sauce à l'étamine ; tenez-la au chaud.
au bain-marie.

Escalopez les cervelles, coupez le ris en petits cubes, émincez les truffes.

Réunissez sauce. truffes, champignons, cervelles, ris, quenelles, crêtes et
rognons de coq, finissez leur cuisson pendant quelques minutes, puis liez
l'ensemble avec les jaunes d'œufs et la crème.

Chauffez pendant cinq minutes la croûte de vol-au-vent à la bouche du four,
garnissez-la avec le salpicon et servez.

\sk

\index{Bouchées au gras}
\index{Farce pour bouchées au gras}
On peut préparer dans le méme esprit des bouchées,

\sk

\index{Bouchées Périgueux}
\index{Bouchées financière}
On pourra apprêter d'une façon analogue des vol-au-vent et des bouchées
Périgueux et financière.

\section*{\centering Grillade de pré-salé.}
\phantomsection
\addcontentsline{toc}{section}{ Grillade de pré-salé.}
\index{Grillade de pré-salé}

Pour faire des grillades de mouton, taillez dans la noix d'un gigot de pré-salé
des tranches de l'épaisseur de {\ppp4\mmm} centimètres environ et faites-les griller au
charbon de bois, à feu dessus.

Donnez d'abord un coup de feu très vif sur l'une des faces, continuez la cuisson
à feu vif pendant {\ppp7\mmm} minutes environ ; puis retournez la viande et traitez l'autre
face de la même manière. La cuisson achevée, salez et poivrez au goût les deux
faces des grillades et servez aussitôt, avec une garniture de cresson.

Ces grillades de pré-salé sont délicicuses.

\sk

\index{Côtelettes de mouton grillée}
On peut griller de même des tranches de selle « Mutton chops » de même
épaisseur ; mais il sera bon alors de ficeler la viande avant de la mettre sur
le gril pour éviter qu'elle se déforme à la cuisson ; on relirera naturellement
la ficelle avant de servir.

\section*{\centering Carré de mouton rôti.}
\phantomsection
\addcontentsline{toc}{section}{ Carré de mouton rôti.}
\index{Carré de mouton rôti}

Prenez un carré de mouton raccourci, c'est-à-dire prêt à être découpé en
côtelettes, formé par la réunion des côtes premières et secondes, enlevez les
os de l'échine, dégagez le bout des os comme l’on fait pour les côtelettes
parées ; bardez la noix, puis faites rôtir le carré ainsi apprêté.

Lorsqu'il est cuit à point, dressez-le sur un plat, garnissez d’une papillote
le bout de chaque os et servez en envoyant le jus de cuisson dégraissé dans une
saucière.

Toutes les garnitures de légumes usuelles accompagnant les côtelettes, la selle,
le baron de mouton conviennent parfaitement pour le carré rôti.

\sk

\index{Agneau grillé}
\index{Carré d'agneau grillé}
Le carré d'agneau est généralement grillé au lieu d'être rôti.

\section*{\centering Brochettes de pré-salé.}
\phantomsection
\addcontentsline{toc}{section}{ Brochettes de pré-salé.}
\index{Brochettes de pré-salé}

Pour quatre personnes prenez :

\medskip

\footnotesize
\begin{longtable}{rrrp{16em}}
    100 & grammes & de & lard gras et maigre en douze petites lames,                                      \\
        &         &  1 & carré de huit côtelettes de pré-salé,                                            \\
        &         &  1 & barde de lard.                                                                   \\
        &         &    & sauce béarnaise, \hyperlink{p0433}{p. \pageref{pg0433}},                         \\
        &         &    & curry,                                                                           \\
        &         &    & sel et poivre.                                                                   \\
\end{longtable}
\normalsize

Levez la noix du carré\footnote{Le reste du carré pourra être utilisé dans un
ragoût.}, salez-la, poivrez-la, bardez-la et découpez-la en huit noisettes.

Préparez la sauce béarnaise, relevez-la avec du curry.

Enfilez sur une petite brochette deux noisettes de pré-salé encastrées dans
trois lames de lard. Apprêtez quatre brochettes semblables et faites-les
griller pendant une dizaine de minutes.

Servez les brochettes sur un plat chaud et la sauce dans une saucière.

\section*{\centering Chachlik à la parisienne.}
\phantomsection
\addcontentsline{toc}{section}{ Chachlik à la parisienne.}
\index{Chachlik à la parisienne}

Le chachlik est un plat caucasien constitué par des brochettes de mouton avec
alternances de lard et de champignons, qu'on fait griller en les arrosant de
graisse de mouton.

Voici un mode de préparation légèrement différent du plat original, mais plus fin.

Pour six personnes prenez :

\medskip

\footnotesize
\begin{tabular}{rp{22em}}
\hspace{4em}18 & \hangindent=1em noisettes de pré-salé de 1 centimètre d'épaisseur environ,               \\
\hspace{4em}18 & \hangindent=1em tranches de bacon de même surface que les noisettes de
                                 pré-salé, mais de 1/2 centimètre d'épaisseur,                            \\
\hspace{4em}18 & \hangindent=1em champignons de couches ou 18 morilles ayant des chapeaux
                                 d’une surface sensiblement semblable à celle des tranches de viande,     \\
\hspace{4em}   & huile d'olive,                                                                           \\
\hspace{4em}   & beurre,                                                                                  \\
\hspace{4em}   & mie de pain rassis tamisée,                                                              \\
\hspace{4em}   & sel et poivre.                                                                           \\
\end{tabular}
\normalsize

\medskip

Épluchez les champignons ou nettoyez les morilles.

Mettez à mariner pré-salé, bacon et chapeaux de champignons ou de morilles dans
de l'huile d'olive ; salez, poivrez ; laissez en contact pendant une
demi-heure.

Enfilez sur six brochettes les trois éléments en les alternant ; passez-les
dans de la mie de pain rassis tamisée, beurrez-les légèrement et faites-les
griller, de préférence sur un feu doux de braise, en les arrosant avec du
beurre fondu pendant la cuisson.

Dressez les brochettes sur un plat de riz pilaf simple ou au parmesan et servez.

C'est un excellent plat qui diffère entièrement des autres préparations de
mouton.

\section*{\centering Filet de pré-salé au bacon et au rognon de veau, sauce moutarde.}
\phantomsection
\addcontentsline{toc}{section}{ Filet de pré-salé au bacon et au rognon de veau, sauce moutarde.}
\index{Filet de pré-salé au bacon et au rognon de veau, sauce moutarde}

Pour six personnes prenez :

\medskip

\footnotesize
\begin{longtable}{rrrp{16em}}
    500 & grammes & de & bacon,                                                                           \\
    125 & grammes & de & beurre,                                                                          \\
        & 1 litre & de & consommé,                                                                        \\
        &         &  2 & rognons de veau,                                                                 \\
        &         &  1 & selle de pré-salé,                                                               \\
        &         &    & légumes,                                                                         \\
        &         &    & farine,                                                                          \\
        &         &    & moutarde,                                                                        \\
        &         &    & sel et poivre.                                                                   \\
\end{longtable}
\normalsize

Levez les filets du mouton ; réservez les déchets.

Épluchez les légumes ; coupez-les en morceaux ; faites-les revenir dans une
partie du beurre ; saupoudrez-les de farine, mouillez avec le consommé, mettez
les déchets réservés et le bacon. Laissez cuire doucement pendant deux heures.

Retirez le lard dès qu'il est cuit ; tenez-le au chaud.

Passez la sauce ; réduisez-la à bonne consistance ; tenez-la au chaud.

Escalopez les rognons ; faites-les cuire dans le reste du beurre ;
assaisonnez-les avec sel et poivre.

Faites rôtir les filets de mouton en les gardant saignants ; assaisonnez-les.

Coupez en tranches de surfaces semblables filets de mouton et lard,

Dressez les filets sur un plat tenu sur un réchaud, en les reconstituant et en
intercalant les tranches de lard entre les tranches de mouton ; disposez autour
les escalopes de rognon : masquez le tout avec la sauce, montée à la moutarde
au goût, puis servez en envoyant en même temps des croquettes de pommes de
terre duchesse ou des épinards gratinés, par exemple.

\section*{\centering Gigot de mouton mariné.}
\phantomsection
\addcontentsline{toc}{section}{ Gigot de mouton mariné.}
\index{Gigot de mouton mariné}

Pour huit personnes prenez :

\footnotesize
\begin{longtable}{@{}lrrrp{16em}}
\setlength\LTleft\parindent
\hspace{4em}   & 1 500 & grammes & de & vin rouge,                                                        \kill
\normalsize1°\footnotesize \hspace{4em} & &  & 1 & beau gigot de mouton,                                  \\
\hspace{4em}   &       &         &    & lard à piquer,                                                    \\
\hspace{4em}   &       &         &    & sel et poivre ;                                                   \\
\end{longtable}
\normalsize

\label{pg0514} \hypertarget{p0514}{}
\label{pg0515} \hypertarget{p0515}{}
\footnotesize
\begin{longtable}{@{}lrrrp{16em}}
\setlength\LTleft\parindent
\normalsize 2° & \multicolumn{4}{l}{\normalsize   pour la marinade :}                                     \\
\footnotesize
\hspace{4em}   &       &         &    &                                                                   \\
\hspace{4em}   & 1 500 & grammes & de & vin rouge,                                                        \\
\hspace{4em}   &   250 & grammes & de & vinaigre de vin rouge,                                            \\
\hspace{4em}   &   200 & grammes & d’ & huile,                                                            \\
\hspace{4em}   &   100 & grammes & d' & oignons coupés en rondelles,                                      \\
\hspace{4em}   &   100 & grammes & de & carottes coupées en rondelles,                                    \\
\hspace{4em}   &    40 & grammes & d’ & échalotes,                                                        \\
\hspace{4em}   &    30 & grammes & de & céleri,                                                           \\
\hspace{4em}   &    15 & grammes & de & sucre,                                                            \\
\hspace{4em}   &    10 & grammes & d' & ail,                                                              \\
\hspace{4em}   &    10 & grammes & de & persil,                                                           \\
\hspace{4em}   &    10 & grammes & de & sel,                                                              \\
\hspace{4em}   &     1 & gramme  & de & poivre en grains,                                                 \\
\hspace{4em}   &     1 & gramme  & de & baies de genévrier,                                               \\
\hspace{4em}   &     1 & gramme  & de & romarin,                                                          \\
\hspace{4em}   & \multicolumn{2}{r}{5 décigrammes} & de & sauge,                                          \\
\hspace{4em}   & \multicolumn{2}{r}{5 décigrammes} & de & basilic,                                        \\
\hspace{4em}   & \multicolumn{2}{r}{5 décigrammes} & de & quatre épices,                                  \\
\hspace{4em}   & \multicolumn{2}{r}{1 décigramme } & de & laurier,                                        \\
\hspace{4em}   & \multicolumn{2}{r}{1 décigramme } & de & thym,                                           \\
\hspace{4em}   & \multicolumn{2}{r}{1 décigramme } & de & girofle,                                        \\
\hspace{4em}   & \multicolumn{2}{r}{1 décigramme } & de & poivre de Cayenne ;                             \\
\end{longtable}
\normalsize

\footnotesize
\begin{longtable}{@{}lrrrp{16em}}
\setlength\LTleft\parindent
\hspace{4em}   & 1 500 & grammes & de & vin rouge,                                                        \kill
\normalsize 3° & \multicolumn{4}{l}{\normalsize   pour la sauce :}                                        \\
\footnotesize
\hspace{4em}   &       &         &    &                                                                   \\
\hspace{4em}   &   110 & grammes & de & beurre,                                                           \\
\hspace{4em}   &   110 & grammes & de & glace de viande,                                                  \\
\hspace{4em}   &    10 & grammes & de & farine.                                                           \\
\end{longtable}
\normalsize

Enlevez la peau et les aponévroses\footnote{ Les aponévroses sont des membranes
luisantes, très résistantes, qui enveloppent les muscles. La suppression des
aponévroses superficielles, très facile à exécuter au moyen d'un couteau à lame
mince et effilée, a pour effet de rendre la chair beaucoup plus tendre.}
superficielles du gigot, retirez l'os du quasi ; réservez ces déchets.

Coupez le lard en lardons ; assaisonnez-les avec sel et poivre ; piquez-en le
gigot.

Préparez la marinade : faites revenir dans l'huile les oignons, les carottes,
le céleri, les échalotes et l'ail : ajoutez le vin, le vinaigre, le sel, le
sucre, le persil, les épices et les aromates ; continuez la cuisson pendant dix
minutes.

Laissez refroidir, dégraissez, puis mettez dedans le gigot et les déchets
réservés et laissez mariner pendant deux à sept jours, suivant la température
ambiante.

Retirez ensuite le gigot de la marinade, essuyez-le et faites-le rôtir à la broche,
à feu vif, pendant autant de quarts d'heure qu'il pèse de livres.

En même temps, préparez la sauce.

Faites blondir la farine dans {\ppp50\mmm} grammes de beurre, mouillez avec la
marinade, mettez les déchets du gigot. Laissez cuire pendant une heure ;
passez, ajoutez la glace de viande, dépouillez la sauce en la faisant mijoter
pendant une demi-heure, puis montez-la au fouet avec le reste du beurre.

La sauce, telle quelle, est une \textit{sauce marinade}.

On obtiendra une sauce \textit{marinade à la russe}, en adoucissant la sauce
marinade avec de la crème.

On aura une \textit{sauce venaison}, en mettant dans la sauce marinade une
cuillerée à café, soit {\ppp15\mmm} grammes, de gelée de groseilles.

Enfin, on aura différentes \textit{sauces poivrades}, en relevant la sauce
marinade avec du poivre fraîchement moulu, du cayenne ou du paprika.

Quelle que soit la sauce choisie, servez le gigot à part et la sauce dans une
saucière.

Le gigot de mouton mariné a un goût de venaison remarquable et il constitue un
plat très apprécié quand le gibier fait défaut. Il est encore meilleur froid
que chaud.

\sk

\index{Chevreau mariné (Gigot de)}
\index{Chèvre marinée (Gigot de)}
\index{Agneau mariné (Gigot de)}
On peut préparer de même le gigot d'agneau, le gigot de chevreau et le gigot de
chèvre. Ce dernier imite très bien le cuissot de chevreuil.

\section*{\centering Gigot de mouton au four, avec des pommes de terre.}
\phantomsection
\addcontentsline{toc}{section}{ Gigot de mouton au four, avec des pommes de terre.}
\index{Gigot de mouton au four, avec des pommes de terre}

On peut préparer ce plat de différentes manières :

1° Faire braiser le gigot avec les pommes de terre dans du jus, sans autre
assaisonnement que du sel et du poivre ;

2° Faire cuire le gigot et les pommes de terre sans mouillement, simplement
avec du beurre ({\ppp125\mmm} grammes de beurre environ pour un gigot et {\ppp1\mmm} kilogramme de
pommes de terre). C'est le gigot dit « à la boulangère » :

3° Mettre les pommes de terre seules dans le plat avec du beurre et placer le
gigot sur un gril disposé au-dessus du plat. Le goût du gigot rappellera alors
celui du gigot grillé.

\sk

\index{Filet de porc au four, avec pommes de terre}
On pourra préparer dans le même esprit un filet de porc ou une tranche épaisse
de jambon frais ; mais ici la graisse du porc suffira ; le beurre est inutile.

\medskip

Ce sont d'excellents plats de famille.

\section*{\centering Noisettes de pré-salé marinées, en aspic.}
\phantomsection
\addcontentsline{toc}{section}{ Noisettes de pré-salé marinées, en aspic.}
\index{Noisettes de pré-salé marinées, en aspic}
\index{Aspic de noisettes de pré-salé marinées}
\index{Agneau mariné}

Pour douze personnes prenez :

\footnotesize
% \begin{longtable}[l]{@{}lrrrp{16em}}
\begin{longtable}{@{}lrrrp{16em}}
\setlength\LTleft\parindent
\normalsize1°\footnotesize \hspace{4em} & 100 & grammes & de & glace de viande,                           \\
\hspace{4em}   &   50 &  grammes & de & beurre,                                                           \\
\hspace{4em}   &   15 &  grammes & de & gelée de groseilles,                                              \\
\hspace{4em}   &   10 &  grammes & de & farine,                                                           \\
\hspace{4em}   &      &          &  4 & bardes de lard,                                                   \\
\hspace{4em}   &      &          &  2 & selles de pré-salé,                                               \\
\hspace{4em}   &      &          &    & lard à piquer,                                                    \\
\hspace{4em}   &      &          &    & truffes à volonté,                                                \\
\hspace{4em}   &      &          &    & madère,                                                           \\
\hspace{4em}   &      &          &    & marinade,                                                         \\
\hspace{4em}   &      &          &    & sel et poivre ;                                                   \\
\end{longtable}
\normalsize

\footnotesize
% \begin{longtable}[l]{@{}lrrrp{16em}}
\begin{longtable}{@{}lrrrp{16em}}
\setlength\LTleft\parindent
\normalsize 2° & \multicolumn{4}{l}{\normalsize   pour la gelée d'aspic :}                                \\
\footnotesize
\hspace{4em}   &      &          &    &                                                                   \\
\hspace{4em}   &  750 &  grammes & de & gîte-gîte,                                                        \\
\hspace{4em}   &  750 &  grammes & de & jarret de veau,                                                   \\
\hspace{4em}   &  200 &  grammes & de & bœuf maigre haché,                                                \\
\hspace{4em}   &  200 &  grammes & de & carottes,                                                         \\
\hspace{4em}   &  100 &  grammes & de & vin blanc sec,                                                    \\
\hspace{4em}   &   50 &  grammes & de & cognac,                                                           \\
\hspace{4em}   &   50 &  grammes & de & couenne maigre,                                                   \\
\hspace{4em}   &   30 &  grammes & de & sel,                                                              \\
\hspace{4em}   &      & 3 litres & d’ & eau,                                                              \\
\hspace{4em}   &      &          &  6 & poireaux (le blanc seulement),                                    \\
\hspace{4em}   &      &          &  2 & abatis de volaille,                                               \\
\hspace{4em}   &      &          &  2 & blancs d'œufs,                                                    \\
\hspace{4em}   &      &          &  1 & beau pied de veau,                                                \\
\hspace{4em}   &      &          &  1 & oignon.                                                           \\
\end{longtable}
\normalsize

\medskip

Préparez la marinade comme il est dit \hyperlink{p0514}{p. \pageref{pg0514}}.

Faites cuire les truffes dans du madère ; réservez-les ; ajoutez le madère de
cuisson à la marinade.

Désossez les selles, enlevez-en la peau et la graisse extérieure ; retirez les
aponévroses superficielles ; coupez les os.

Piquez les demi-selles de lardons assaisonnés avec sel et poivre, roulez-les et
mettez-les ainsi apprêtées, avec les os, dans la marinade où vous les laisserez
pendant deux à trois jours, suivant la température.

Le dernier jour, préparez la gelée d'aspic.

Mettez dans une marmite l'eau, le gîte-gîte, le jarret et le pied de veau, la
couenne maigre, les abatis de volaille, les carottes, les poireaux, l'oignon,
le sel ; faites cuire pendant cinq heures, passez le bouillon, laissez-le
refroidir, dégraissez- le. Remettez le bouillon sur le feu, ajoutez le bœuf
maigre haché et les blancs d'œufs battus, en tournant jusqu'à ébullition, puis
le vin blanc et le cognac ; faites cuire encore pendant une demi-heure. Passez
à l'étamine ; laissez prendre en gelée.

Sortez la viande de la marinade, essuyez-la, bardez chaque demi-selle à part ;
puis faites-les rôtir à feu vif pendant un quart d'heure environ de façon que
l'intérieur reste saignant.

Laissez-les refroidir, retirez les bardes, enlevez le reste de la graisse et
parez la viande ; réservez les déchets.

Coupez les filets en noisettes ; disposez-les isolément sur un marbre.

Concentrez la marinade en faisant cuire dedans les déchets maigres de viande et
les os coupés, pendant une heure environ ; dégraissez-la ; passez-la.

Faites un roux peu coloré avec le beurre et la farine ; mouillez avec la
marinade passée, ajoutez la glace de viande ; laissez cuire pendant une
demi-heure environ en dépouillant la sauce ; puis mettez la gelée de
groseilles.

Laissez refroidir la marinade jusqu'à obtention d'une consistance serrée et
masquez-en les noisettes. Tenez au frais pour que la marinade se solidifie.

Moulez, avec une partie de la gelée, un tronc de pyramide dans lequel vous
aurez inséré une truffe (un petit pot à confitures peut servir de moule),
disposez ce bloc au milieu d'un plat garni de gelée hachée : dressez en
couronne sur cette gelée les noisettes masquées avec la marinade et glacez-les
avec le reste de la gelée fondue. Laissez prendre.

Décorez le pourtour du tronc de pyramide et les intervalles existant entre les
noisettes avec un hachis de truffes et servez.

Ces noisettes de pré-salé dépassent comme finesse le filet de chevreuil le plus
délicat. Le plat fait excellente figure dans les repas les plus soignés.

\section*{\centering Gigot de mouton poché, sauce tomate alliacée au mirepoix.}
\phantomsection
\addcontentsline{toc}{section}{ Gigot de mouton poché, sauce tomate alliacée au mirepoix.}
\index{Gigot de mouton poché, sauce tomate alliacée au mirepoix}

Pour six personnes prenez :

\medskip

\footnotesize
\begin{longtable}{rrrp{16em}}
    250 & grammes & de & consommé,                                                                        \\
    250 & grammes & de & purée de tomates concentrée,                                                     \\
    100 & grammes & de & jambon salé, non fumé,                                                           \\
    100 & grammes & de & noix de veau,                                                                    \\
    100 & grammes & de & beurre,                                                                          \\
     30 & grammes & de & sel gris,                                                                        \\
     30 & grammes & d’ & ail en gousses,                                                                  \\
     20 & grammes & de & carotte,                                                                         \\
     20 & grammes & d' & oignon,                                                                          \\
     10 & grammes & de & farine,                                                                          \\
       & 5 litres & d' & eau,                                                                             \\
        &         &  1 & gigot de pré-salé pesant 2 kilogr. 500 environ,                                  \\
        &         &  1 & bouquet garni,                                                                   \\
        &         &    & sel et poivre.                                                                   \\
\end{longtable}
\normalsize

Piquez le gigot avec cinq gousses d'ail pesant ensemble {\ppp12\mmm} grammes environ,
frottez-le extérieurement avec {\ppp3\mmm} grammes d'ail.

Mettez l’eau et le sel dans une marmite, faites bouillir ; puis plongez dedans le
gigot et laissez-le cuire pendant une heure un quart, c'est-à-dire à raison d’une
demi-heure de pochage par kilogramme de viande.

La sauce est une sauce tomate, corsée par un mirepoix fait avec veau, jambon,
oignon, carotte, bouquet, relevée avec de l'ail, qu'on prépare de la façon
suivante.

Coupez le veau et le jambon en petits morceaux.

Hachez fin oignon et carotte.

Mettez {\ppp50\mmm} grammes de beurre dans une casserole ; faites revenir dedans veau,
jambon et légumes, saupoudrez avec la farine, ajoutez le bouquet, le reste de
l'ail en gousses, salez, poivrez, mouillez avec le consommé, faites cuire
pendant une heure et demie, puis passez le tout à la passoire fine ; ajoutez
alors la purée de tomates et concentrez. Un instant avant de servir, mettez le
reste du beurre coupé en petits morceaux, goûtez et complétez l'assaisonnement
s'il y a lieu.

Servez le gigot au sortir de l'eau et envoyez à part la sauce dans une saucière.

Les haricots blancs nouveaux et les haricots verts accompagnent très bien ce
plat ; à leur défaut, on peut encore donner des spaghetti.

\sk

On pourra faire pocher de même des côtelettes de mouton, mais alors il vaudra
mieux ne pas les piquer d'ail ; on se bornera à les en frotter et à en mettre
dans l’eau qui servira à les cuire.

Pour des côtelettes pesant {\ppp200\mmm} grammes environ, dix minutes de cuisson sont
nécessaires et suffisantes.

\section*{\centering Épaule de pré-salé au risotto.}
\phantomsection
\addcontentsline{toc}{section}{ Épaule de pré-salé au risotto.}
\index{Épaule de pré-salé au risotto}
\index{Épaule de pré-salé au risotto, gratinée}

Pour six personnes prenez :

\medskip

\footnotesize
\begin{longtable}{rrrp{16em}}
  1 500 & grammes & d' & épaule de pré-salé désossée et coupée en tranches,                               \\
  1 500 & grammes & de & bon bouillon,                                                                    \\
    250 & grammes & de & riz,                                                                             \\
    200 & grammes & de & purée de tomates aromatisée,                                                     \\
    100 & grammes & de & beurre,                                                                          \\
     65 & grammes & de & parmesan râpé,                                                                   \\
     65 & grammes & de & gruyère râpé,                                                                    \\
     10 & grammes & d’ & oignon haché,                                                                    \\
      2 & grammes & de & paprika,                                                                         \\
        &         &    & sel.                                                                             \\
\end{longtable}
\normalsize

Mettez le mouton dans une marmite avec le bouillon froid, placez la marmite sur
le feu, amenez rapidement le liquide à l'ébullition, puis continuez lentement
la cuisson, qui doit durer en tout une heure et demie.

Retirez la viande, tenez-la au chaud ; réservez le bouillon.

En même temps, préparez le risotto. Lavez le riz, laissez-le tremper pendant
une heure dans de l'eau froide, essuyez-le, séchez-le dans un linge.

Mettez dans une casserole le beurre et l'oignon ; faites cuire sans laisser
prendre couleur, passez, puis ajoutez le riz, étalez-le au fond de la
casserole, arrosez-le avec quelques cuillerées de bouillon ; chauffez en
casserole découverte.

Lorsque le bouillon sera absorbé, agitez de façon à détacher les grains qui
pourraient adhérer au fond, mouillez de nouveau avec un peu de bouillon et
continuez ainsi jusqu'à ce que tout le liquide soit absorbé. Pendant la
cuisson, assaisonnez avec le paprika et n'ajoutez de sel qu'après avoir goûté,
car l'assaisonnement du bouillon peut être suffisant. Mettez enfin la purée de
tomates et les fromages ; laissez cuire encore pendant quelques instants.

Le riz doit être moelleux et les grains entiers, non crevés.

Dressez sur un plat des couches alternées de risotto et de mouton, en terminant
par une couche de risotto et servez.

\sk

Comme variante, on peut réserver une partie des deux fromages, en saupoudrer,
à la fin, la couche supérieure de risotto et faire gratiner au four.

\medskip

Le plat, gratiné ou non, diffère absolument aussi bien du ragoût d'épaule de
mouton au riz, que de l'épaule de mouton au riz, et que du pilaf de mouton.

La viande non revenue, bouillie dans du bouillon, a un goût spécial. Le riz,
aromatisé avec les tomates et le paprika, engraissé par le beurre, les fromages
et le suc de la viande, l'accompagne on ne peut mieux, et l'ensemble constitue
un plat délicieux, qu'on pourra rendre plus riche en incorporant au riz des
émincés de truffe noire cuite dans du madère.

\section*{\centering Côtelettes de mouton braisées.}
\phantomsection
\addcontentsline{toc}{section}{ Côtelettes de mouton braisées.}
\index{Côtelettes de mouton braisées}

Pour six personnes prenez :

\medskip

\footnotesize
\begin{longtable}{rrrp{16em}}
    200 & grammes & de & bouillon,                                                                        \\
    200 & grammes & de & jarret de veau,                                                                  \\
    200 & grammes & de & carottes coupées en rondelles,                                                   \\
    100 & grammes & d' & oignons coupés en rondelles,                                                     \\
    100 & grammes & de & lard à piquer,                                                                   \\
     30 & grammes & de & beurre,                                                                          \\
        &         &  6 & côtelettes de pré-salé parées, pesant ensemble 1 200 grammes environ,            \\
        &         &  3 & clous de girofle,                                                                \\
        &         &  2 & bardes de lard, pesant ensemble 100 grammes environ,                             \\
        &         &  1 & feuille de laurier,                                                              \\
        &         &  1 & brindille de thym,                                                               \\
        &         &    & quatre épices,                                                                   \\
        &         &    & sel et poivre.                                                                   \\
\end{longtable}
\normalsize

Coupez le lard à piquer en lardons ; assaisonnez-les avec sel, poivre et quatre
épices.

Piquez les côtelettes avec les lardons ; faites-les revenir légèrement, dans le
beurre, sans les dorer.

Foncez une casserole avec l’une des bardes de lard, disposez dessus les
côtelettes, le jarret de veau, les carottes, les oignons et les aromates,
assaisonnez avec sel et poivre, au goût, mouillez avec le bouillon, couvrez
avec la deuxième barde et laissez mijoter, en casserole fermée, à tout petit
feu, pendant deux heures.

Retirez les côtelettes, dressez-les sur un plat, passez le jus, dégraissez-le,
concentrez-le, goûtez-le, complétez l'assaisonnement s'il est nécessaire,
masquez-en les côtelettes et servez.

L'accompagnement indiqué de ces côtelettes est une purée d'oignons à la crème.

\section*{\centering Gigot de mouton braisé, aux haricots.}
\phantomsection
\addcontentsline{toc}{section}{ Gigot de mouton braisé, aux haricots.}
\index{Gigot de mouton braisé, aux haricots}

Pour six personnes prenez :

\medskip

\footnotesize
\begin{longtable}{rrrp{16em}}
        1 & litre 1/2 & de & haricots de Soissons frais,                                                  \\
        1 & litre 1/2 & de & bon jus ou de consommé,                                                      \\
          &           &  1 & gigot de pré-salé,                                                           \\
          &           &    & bouquet garni,                                                               \\
          &           &    & beurre,                                                                      \\
          &           &    & ail, oignon, échalote, au goût,                                              \\
          &           &    & sel et poivre.                                                               \\
\end{longtable}
\normalsize

Mettez le gigot dans un plat allant au feu ; disposez autour les haricots, salez,
poivrez, ajoutez bouquet garni, ail, oignon, échalote, au goût, mouillez avec le jus
ou le consommé. Faites cuire au four pendant autant de quarts d'heure que le
gigot pèse de livres, en arrosant la viande à partir du moment où le liquide de
cuisson ne la couvre plus. Lorsque le jus est presque complètement épuisé, ajoutez
du beurre, par petites quantités, autant que les haricots pourront en absorber.

Servez dans le plat.

\sk

On pourrait à la rigueur préparer le plat avec des haricots secs, mais il
n'aurait plus la même finesse. De plus, il faudrait, après avoir échaudé les
haricots, les faire cuire en partie séparément avant de les mettre avec le
gigot, pour que leur cuisson fût achevée au four en même temps que celle de la
viande.

\section*{\centering Épaule de pré-salé farcie, braisée au madère.}
\phantomsection
\addcontentsline{toc}{section}{ Épaule de pré-salé farcie, braisée au madère.}
\index{Épaule de pré-salé farcie, braisée au madère}

Pour six personnes prenez :

\medskip

\footnotesize
\begin{longtable}{rp{2em}rrrp{16em}}
& \multicolumn{3}{r}{2 kilogrammes} & d' & épaule de pré-salé non parée,                                  \\
& & 400 & grammes & de & madère,                                                                          \\
& & 250 & grammes & de & jambon fumé, cru, de Bayonne, de Prague ou d'York,                               \\
& & 100 & grammes & de & beurre,                                                                          \\
& & 100 & grammes & de & carottes,                                                                        \\
& &  65 & grammes & d' & oignons,                                                                         \\
& &  60 & grammes & de & glace de viande,                                                                 \\
& &  30 & grammes & de & fine champagne,                                                                  \\
& &   2 & grammes & d' & ail,                                                                             \\
& &     &         &    & bouillon,                                                                        \\
& &     &         &    & crépine de porc,                                                                 \\
& &     &         &    & sel et poivre.                                                                   \\
\end{longtable}
\normalsize

Parez l'épaule, désossez-la ; réservez les déchets.

\index{Farce pour pré-salé}
Hachez le jambon, ajoutez-y {\ppp5\mmm} grammes d’oignon et un demi-gramme
d'ail hachés, saupoudrez d'un demi-gramme de poivre et farcissez avec ce
mélange l'épaule désossée.

Garnissez d'une crépine la partie de la viande non recouverte de peau, ficelez,
piquez avec le reste de l'ail. Faites revenir l'épaule ainsi apprêtée pendant
une demi-heure dans le beurre ; flambez ensuite avec la fine champagne, ajoutez
les déchets de l'épaule, les carottes et le reste des oignons coupés en
tranches ; laissez dorer ; mouillez avec le madère et du bouillon dans lequel
vous aurez fait dissoudre la glace de viande, couvrez et faites braiser, au
four, à feu modéré, pendant deux heures, en arrosant de temps en temps avec la
cuisson. Dégraissez alors la sauce, passez-la, goûtez pour l’assaisonnement,
rectifiez-le s'il est nécessaire, retirez les ficelles, les clous d'ail et la
crépine, dressez l'épaule sur plat, masquez-la avec la sauce et servez.

C'est un plat de famille soigné.

\section*{\centering Dolma ou Warack Malfouff à la parisienne.}
\phantomsection
\addcontentsline{toc}{section}{ Dolma ou Warack Malfouff à la parisienne.}
\index{Dolma ou Warack Malfouff à la parisienne}

Ce plat, qui appartient à la cuisine orientale, porte en Turquie le nom de
« Dolma » et en Algérie celui de « Warack Malfouff ».

Pour six personnes prenez :

\medskip

\footnotesize
\begin{longtable}{rrrrp{16em}}
 & 1 000 & grammes & de & filet de pré-salé non désossé,                                                  \\
 &   300 & grammes & de & purée de tomates concentrée,                                                    \\
 &   125 & grammes & de & riz,                                                                            \\
 &   125 & grammes & de & beurre,                                                                         \\
 &    10 & grammes & d' & ail,                                                                            \\
 &    10 & grammes & de & sel,                                                                            \\
 &     2 & grammes & de & paprika,                                                                        \\
 &  \multicolumn{2}{r}{2 décigrammes} & de & poivre fraîchement moulu,                                    \\
 &       & 1 litre & de & consommé,                                                                       \\
 &       &         & 12 & feuilles blanches de chou frisé saines et sans déchirure,                       \\
 &       &         &    & muscade.                                                                        \\
\end{longtable}
\normalsize

Désossez le filet sans le dégraisser ; hachez fin viande et graisse ; réservez
les os. Ajoutez au hachis le riz cru, le beurre, le sel, le poivre, le paprika
et de la muscade au goût, mélangez bien, puis divisez la masse obtenue en douze
parties que vous roulerez en forme de saucisses.

Faites blanchir les feuilles de chou pendant trois minutes dans de l'eau
bouillante ; enlevez-en les grosses côtes ; enrobez chaque saucisse dans une
feuille de chou.

Mettez au fond d'une casserole les os et l'ail, disposez dessus les dolma,
masquez-les avec la purée de tomates, mouillez avec le consommé de manière que
tout baigne dans le liquide. Couvrez la préparation avec une assiette plus
petite que la casserole afin d'empêcher les warack de monter à la surface et
mettez le couvercle de la casserole.

Faites cuire pendant une demi-heure à feu vif, puis continuez la cuisson
à petit feu pendant quatre heures ; le jus doit alors être très réduit.

Retirez les warack avec précaution afin de ne pas les abîmer, dressez-les sur
un plat chaud, versez dessus le jus bien dégraissé et servez aussitôt.

La préparation ci-dessus diffère de la préparation orientale par la qualité du
mouton, la finesse du jus employé et par l'assaisonnement dans lequel le paprika
remplace le cayenne mal supporté par beaucoup d’estomacs.

\section*{\centering Épaule de mouton marinée, braisée à la crème.}
\phantomsection
\addcontentsline{toc}{section}{ Épaule de mouton marinée, braisée à la crème.}
\index{Épaule de mouton marinée, braisée à la crème}

Pour quatre personnes prenez :

\medskip

\footnotesize
\begin{longtable}{rrrp{16em}}
  1 500 & grammes & d' & épaule de mouton, désossée et roulée,                                            \\
  1 000 & grammes & de & crème épaisse,                                                                   \\
     30 & grammes & de & beurre,                                                                          \\
     10 & grammes & de & farine,                                                                          \\
        &         &    & la moitié des éléments de la marinade, \hyperlink{p0514}{p. \pageref{pg0514}},   \\
        &         &    & sel et poivre.                                                                   \\
\end{longtable}
\normalsize

Préparez la marinade comme il est dit \hyperlink{p0515}{p. \pageref{pg0515}} ;
mettez dedans l'épaule ; laissez-la pendant {\ppp48\mmm} heures.

Retirez la viande de la marinade, essuyez-la, faites-la revenir dans le beurre,
saupoudrez-la avec la farine, puis mouillez avec la marinade passée. Faites braiser
à petit feu jusqu'à la moitié de la cuisson, en retournant la viande plusieurs fois.

Mettez alors la crème ; achevez la cuisson de l'ensemble à petit feu. Goûtez la
sauce et complétez l’assaisonnement avec sel et poivre, sil y a lieu.

Servez avec accompagnement de riz aux cèpes ou aux morilles.

\section*{\centering Salmis de pré-salé, sauce piquante.}
\phantomsection
\addcontentsline{toc}{section}{ Salmis de pré-salé, sauce piquante.}
\index{Salmis de pré-salé, sauce piquante}

Pour six personnes prenez :

\medskip

\footnotesize
\begin{longtable}{rrrrp{16em}}
  & \multicolumn{2}{r}{2 kilogrammes} & de & pré-salé désossé : gigot, selle ou épaule,                   \\
  & 500 & grammes & de & bon consommé,                                                                    \\
  & 200 & grammes & de & vin blanc,                                                                       \\
  &  50 & grammes & de & glace de viande,                                                                 \\
  &  40 & grammes & de & cornichons hachés,                                                               \\
  &  30 & grammes & d’ & échalotes hachées,                                                               \\
  &  25 & grammes & de & beurre,                                                                          \\
  &  25 & grammes & de & farine,                                                                          \\
  &  15 & grammes & de & vinaigre,                                                                        \\
  & \multicolumn{2}{r}{3 décigrammes} & d' & ail,                                                         \\
  & \multicolumn{2}{r}{2 décigrammes} & de & poivre fraîchement moulu,                                    \\
  & \multicolumn{2}{r}{1 décigramme } & de & cayenne.                                                     \\
\end{longtable}
\normalsize

Faites revenir les échalotes et l'ail dans le beurre, mettez ensuite la farine,
laissez dorer ; puis mouillez avec le vinaigre et le vin ; enfin, ajoutez le
consommé. Laissez bouillir pendant une vingtaine de minutes, passez, ajoutez la
glace de viande, le poivre et le cayenne ; remettez la sauce sur le feu pendant
trois quarts d'heure environ ; dépouillez-la soigneusement.

En même temps, faites rôtir la viande et arrêtez l'opération un peu avant la fin
de la cuisson.

Dégraissez le jus de cuisson et ajoutez-le à la sauce.

Coupez la viande en tranches que vous disposerez dans un plat allant au feu ;
versez dessus la sauce passée au chinois ; chauffez le tout pendant un quart
d'heure sans laisser bouillir. Au dernier moment, ajoutez les cornichons.
Servez dans le plat.

Envoyez en même temps un légumier de riz sauté au beurre,
\hyperlink{p0710}{p. \pageref{pg0710}}.

Ce salmis diffère absolument comme goût de toutes les autres préparations de
mouton ; la sauce est fine et moelleuse et le riz est un excellent
accompagnement.

\section*{\centering Civet de mouton mariné.}
\phantomsection
\addcontentsline{toc}{section}{ Civet de mouton mariné.}
\index{Civet de mouton mariné}

Pour douze personnes prenez :

\medskip

\footnotesize
\begin{longtable}{rrrp{16em}}
    500 & grammes & de & champignons de couche,                                                           \\
    250 & grammes & de & lard de poitrine,                                                                \\
    200 & grammes & de & sang de porc,                                                                    \\
    150 & grammes & de & beurre,                                                                          \\
    100 & grammes & de & fumet de venaison\footnote{\index{Fumet de gibier}
                                                    \index{Fumet de venaison}
                                                    On prépare le fumet de venaison pendant
                                                    la période de la chasse avec des débris de
                                                    venaison rôtie au four. On fait cuire ces
                                                    débris dans de l’eau avec des légumes,
                                                    pendant plusicurs heures ; on passe le jus
                                                    et on le fait réduire à glace ; enfin, on
                                                    le conserve en boîtes soudées.
                                                    \protect\endgraf
                                                    \smallskip
                                                    On prépare de la même façon le fumet de
                                                    gibier.},                                             \\
     70 & grammes & de & farine,                                                                          \\
        &         &  2 & foies de lapin,                                                                  \\
        &         &  1 & beau gigot de mouton,                                                            \\
        &         &    & éléments de la marinade, \hyperlink{p0514}{p. \pageref{pg0514}},                 \\
        &         &    & jus de citron,                                                                   \\
        &         &    & sucre,                                                                           \\
        &         &    & sel et poivre.                                                                   \\
\end{longtable}
\normalsize

Préparez la marinade.

Coupez le gigot en morceaux pesant {\ppp100\mmm} grammes en moyenne ; mettez-les
à mariner pendant le temps nécessaire, suivant la température, puis sortez-les,
égouttez-les ; passez la marinade, réservez-la.

Coupez le lard en petites languettes.

Faites revenir dans le beurre le mouton et le lard pendant une vingtaine de
minutes, saupoudrez avec la farine, laissez mijoter pendant vingt-cinq minutes,
en remuant fréquemment ; mouillez ensuite avec la marinade réservée, salez,
poivrez au goût, et continuez la cuisson à petit feu pendant une heure et
demie.

Un quart d'heure avant la fin, mettez les champignons épluchés et passés au
jus de citron.

Écrasez les foies de lapin, mélangez-les avec le sang de porc et le fumet de
venaison, ajoutez le tout à la cuisson, goûtez, complétez l'assaisonnement,
mitigez, s'il y a lieu, l’âcreté due au vin de la marinade avec un peu de
sucre, donnez un léger bouillon, puis servez en envoyant en même temps un plat
de riz sec.

Ce civet, qui peut soutenir la comparaison avec les meilleurs civets de
venaison, est très bon le jour même, mais il est encore meilleur réchauffé au
bain-marie, le lendemain.

\section*{\centering Épaule de mouton aux navets.}
\phantomsection
\addcontentsline{toc}{section}{ Épaule de mouton aux navets.}
\index{Épaule de mouton aux navets}

Pour quatre personnes prenez :

\medskip

\footnotesize
\begin{longtable}{rrrp{16em}}
  1 500 & grammes & d' & épaule de mouton désossée et roulée,                                             \\
  1 500 & grammes & de & navets,                                                                          \\
    100 & grammes & de & beurre,                                                                          \\
     30 & grammes & de & cognac,                                                                          \\
        & 1 litre & de & bon bouillon,                                                                    \\
        &         &    & sucre en poudre,                                                                 \\
        &         &    & sel et poivre.                                                                   \\
\end{longtable}
\normalsize

Faites revenir la viande dans {\ppp30\mmm} grammes de beurre, enlevez la graisse, flambez
avec le cognac, puis mouillez avec le bouillon, salez, poivrez et laissez cuire
pendant deux heures environ (avec du pré-salé tendre, une heure trois quarts
suffit ; avec du mouton un peu ferme, il faut un peu plus de deux heures).

En même temps, épluchez les navets, roulez-les dans du sucre en poudre,
mettez-les dans une sauteuse avec le reste du beurre et faites-les dorer, à feu
vif, en surveillant l'opération pour éviter un coup de feu intempestif.

Si vous disposez de navets nouveaux, ajoutez-les dans la casserole une
demi-heure seulement avant de servir ; si les navets sont vieux, il peut
falloir une heure de cuisson et même davantage.

Inutile de dire que le plat est particulièrement bon quand il est préparé avec
du pré-salé et des navets nouveaux.

\section*{\centering Épaule de mouton au riz.}
\phantomsection
\addcontentsline{toc}{section}{ Épaule de mouton au riz.}
\index{Épaule de mouton au riz}

Pour quatre personnes prenez :

\medskip

\footnotesize
\begin{longtable}{rrrp{16em}}
  1 500 & grammes & d' & épaule de mouton désossée et roulée,                                             \\
    250 & grammes & de & riz,                                                                             \\
     30 & grammes & de & beurre,                                                                          \\
     30 & grammes & de & cognac,                                                                          \\
        & 1 litre & de & bon bouillon,                                                                    \\
        &         &    & sel et poivre.                                                                   \\
\end{longtable}
\normalsize

Faites revenir la viande, dans le beurre, dégraissez après l'opération, puis
flambez la viande avec le cognac, mouillez avec le bouillon, salez, poivrez et
laissez cuire pendant deux heures environ.

Faites blanchir le riz pendant un quart d'heure dans {\ppp3\mmm} litres d'eau additionnée
de sel gris, rafraîchissez-le ensuite à l'eau froide, puis mettez-le avec le
mouton un quart d'heure avant de servir : il absorbera tout le jus de cuisson
du mouton. Goûtez, rectifiez l'assaisonnement s'il y a lieu, et servez.

Plat de famille très recommandable.

\sk

\index{Épaule de mouton au riz et aux petits pois}
Comme variante, on pourra, dans la saison, mélanger des petits pois au riz. On
prendra les mêmes éléments que précédemment, plus {\ppp1\mmm} kilogramme de petits pois
en cosses. On commencera de la même manière la confection du plat ; une heure
avant de servir, on mettra les pois écossés dans la casserole où cuit le mouton
et l'on continuera l'opération jusqu'à la fin, conformément aux indications
données ci-dessus.

\section*{\centering Épaule de mouton au pilaf, ou pilaf de mouton.}
\phantomsection
\addcontentsline{toc}{section}{ Épaule de mouton au pilaf, ou pilaf de mouton.}
\index{Épaule de mouton au pilaf, ou pilaf de mouton}

On peut faire du mouton au pilaf des deux façons suivantes :

1° Préparez des brochettes de chair de mouton, en alternant des parties grasses
avec des parties maigres et en mettant entre elles des feuilles de sauge ou de
laurier, puis faites-les rôtir devant un bon feu en les graissant avec du
beurre assaisonné et aromatisé, plus ou moins, avec de l'ail et de l'oignon.

2° Faites revenir dans une poêle, avec du beurre, d'abord de l'oignon haché,
puis de la chair de mouton coupée en petits cubes ; mouillez avec du bouillon
de bœuf, de mouton et de volaille, ajoutez de la pulpe de tomates, assaisonnez
au goût et achevez la cuisson.

Dans les deux cas, pendant que le mouton cuit, apprêtez du riz en pilaf, comme
il est dit \hyperlink{p0712}{p. \pageref{pg0712}}, dressez-le sur un plat,
disposez au milieu les morceaux de mouton, arrosez avec la sauce et servez.

\section*{\centering Épaule de mouton à l'ail.}
\phantomsection
\addcontentsline{toc}{section}{ Épaule de mouton à l'ail.}
\index{Épaule de mouton à l'ail}

Piquez une épaule de mouton avec des lardons assaisonnés et de l'ail. Faites-la
revenir dans du beurre, flambez-la avec du cognac, ajoutez ensuite des
carottes, des navets, du céleri, du vin de Porto, un peu de cognac, du sel, du
poivre, des épices, un bouquet garni et laissez cuire à petit feu, en casserole
couverte, pendant deux heures. Mettez alors une vingtaine de gousses d'ail, que
vous aurez fait bouillir au préalable dans de l'eau et que vous aurez
égouttées ; continuez la cuisson pendant une heure encore, réduisez la sauce,
passez-la ; dressez l'épaule sur un plat et servez.

Envoyez en même temps un légumier de haricots blancs au beurre.

\section*{\centering Ragoût de mouton aux pommes de terre ou aux navets.}
\phantomsection
\addcontentsline{toc}{section}{ Ragoût de mouton aux pommes de terre ou aux navets.}
\index{Ragoût de mouton aux pommes de terre ou aux navets}

Le ragoût de mouton est essentiellement un plat de famille. Certaines personnes
prétendent en faire un plat distingué en employant pour sa préparation des
morceaux de choix, tels que du gigot ou des côtelettes premières. À mon avis,
cest une erreur, les côtelettes et le gigot étant meilleurs rôtis ou grillés.
Les morceaux les plus indiqués pour le ragoût sont l'épaule et la poitrine
(hauts de côtelettes), auxquels on peut adjoindre quelques côtelettes secondes.

Le ragoût de mouton peut être préparé aux pommes de terre ou aux navets.

\medskip

Voici une formule de ragoût aux pommes de terre.

\medskip

Pour six personnes prenez :

\medskip

\footnotesize
\begin{longtable}{rrrp{16em}}
  1 500 & grammes & de & hauts de côtelettes pas trop gras, ou bien le même poids
                         d'un mélange de hauts de côtelettes et d'épaule, ou encore
                         le même poids d'un mélange de hauts de côtelettes et de
                         côtelettes secondes coupées en morceaux,                                         \\
  1 000 & grammes & de & pommes de terre,                                                                 \\
  1 000 & grammes & de & bouillon,                                                                        \\
    125 & grammes & de & lard de poitrine maigre, coupé en petits cubes,                                  \\
    125 & grammes & d’ & oignons,                                                                         \\
     60 & grammes & de & beurre,                                                                          \\
     40 & grammes & de & graisse de porc,                                                                 \\
     25 & grammes & de & sel gris,                                                                        \\
     20 & grammes & de & farine,                                                                          \\
        & 1 gr. 50& de & poivre fraîchement moulu,                                                        \\
        &         &  1 & bouquet garni, comprenant 30 grammes de persil en branche,
                         2 grammes de thym et 2 grammes de laurier,                                       \\
        &         &  1 & petite gousse d'ail.                                                             \\
\end{longtable}
\normalsize

Enlevez la peau des morceaux de mouton et mettez la viande seule dans une
sauteuse afin de la dégraisser à fond, à la chaleur.

Faites revenir le lard dans la graisse de porc.

Mettez dans une casserole le beurre et les oignons, retirez-les lorsqu'ils
auront pris couleur\footnote{On peut également ne pas faire dorer les oignons ;
c'est une question de goût.}, ajoutez alors la farine, laissez roussir,
mouillez avec le bouillon, remettez les oignons, ajoutez la viande dégraissée,
les lardons revenus, le bouquet garni, l'ail, le sel et le poivre.

Faites cuire à feu vif pendant une demi-heure, en ayant soin que le tout soit
couvert par le liquide ; puis mettez les pommes de terre pelées et laissez
cuire encore pendant une demi-heure en maintenant toujours suffisamment de
liquide. Un quart d'heure avant la fin, réduisez la sauce, dégraissez encore,
passez, goûtez et complétez l'assaisonnement qui doit être légèrement relevé.

Donnez des assiettes chaudes et servez.

J'emploie de préférence pour le ragoût de mouton aux pommes de terre ou aux
navets les hauts de côtelettes qui donnent au plat une saveur toute
particulière et nourrissent bien les légumes qui les accompagnent : le seul
inconvénient de ces morceaux est d'être gras, mais il est facile d'y remédier
en les dégraissant soigneusement dès le début de l'opération.

Les côtelettes secondes et l'épaule fournissent des morceaux plus charnus.

Le lard de poitrine qui entre dans la préparation plait généralement et
l'assaisonnement un peu corsé du plat le relève agréablement.

Enfin, les pommes de terre, qui accompagnent la viande, donnent à d'ensemble un
caractère confortable, qui convient à la cuisine familiale et attire sur lui
l'attention bienveillante des amateurs sans prétention.

\sk

On peut également préparer le ragoût de mouton aux navets.

On l'apprête comme le ragoût aux pommes de terre, mais avec les différences
suivantes : les navets, après avoir été épluchés, doivent être roulés dans du
sucre en poudre, puis dorés à la poêle dans du beurre. On les ajoute alors au
ragoût ; mais ici la cuisson de l'ensemble doit durer une heure au moins et
quelquefois plus, si les navets sont un peu fermes. Tout le reste de la
préparation reste le même.

\sk

Enfin, on peut faire le ragoût de mouton jardinière, en employant comme légumes
un mélange de pommes de terre, carottes, navets, haricots verts, petits pois,
petits champignons, pointes d'asperges, etc.

C'est particulièrement bon avec des légumes nouveaux.

\section*{\centering Ragoût de mouton au riz.}
\phantomsection
\addcontentsline{toc}{section}{ Ragoût de mouton au riz.}
\index{Ragoût de mouton au riz}

Si les hauts de côtelettes conviennent pour le ragoût aux pommes de terre ou
aux navets, l'épaule de mouton, qui présente incontestablement l'avantage de
fournir des morceaux très charnus, semble plus indiquée pour le ragoût au riz,
dans lequel le riz est cuit à part.

\medskip

Voici deux façons de le préparer.

\medskip

1° Pour quatre à six personnes prenez :

\footnotesize
\begin{longtable}{rrrp{16em}}
  1 000 & grammes  & d' & épaule de mouton,                                                               \\
  1 000 & grammes  & de & bouillon,                                                                       \\
    500 & grammes  & de & champignons de couche,                                                          \\
    125 & grammes  & d' & oignons,                                                                        \\
    100 & grammes  & de & beurre,                                                                         \\
     20 & grammes  & de & farine,                                                                         \\
     20 & grammes  & de & sel gris,                                                                       \\
        & 1 gr. 25 & de & poivre fraîchement moulu,                                                       \\
        &          &  1 & bouquet garni,                                                                  \\
        &          &  1 & petite gousse d'ail.                                                            \\
\end{longtable}
\normalsize

Faites revenir l'épaule de mouton coupée en morceaux dans {\ppp30\mmm} grammes
de beurre.

Faites dorer les oignons dans le reste du beurre, enlevez-les ensuite, puis
mettez la farine, laissez roussir, mouillez avec le bouillon, remettez les
oignons, l'épaule revenue à part, le bouquet garni, l'ail, le sel, le poivre,
et continuez la cuisson pendant une heure.

Dégraissez, passez la sauce, mettez les champignons épluchés, goûtez, complétez
l'assaisonnement et faites cuire encore pendant une vingtaine de minutes.

Servez avec un plat de riz sec, \hyperlink{p0707}{p. \pageref{pg0707}}.

\medskip

2° Remplacez les {\ppp500\mmm} grammes de champignons de couche par
{\ppp500\mmm} grammes de groseilles à maquereau vertes.

Commencez l'opération comme ci-dessus. Après une heure de cuisson, dégraissez,
passez la sauce, ajoutez les groseilles et laissez cuire encore pendant une
demi-heure.

Servez avec un plat de riz aux cèpes, \hyperlink{p0708-1}{p. \pageref{pg0708-1}}.

Ce ragoût diffère essentiellement du précédent par l'introduction des
groseilles vertes, qui pourraient du reste être remplacées par d'autres baies
acidulées ou même par du jus de citron. Le rôle de l'élément acide est de
combattre les fâcheux effets de la graisse de mouton que certains estomacs
digèrent difficilement et de rendre le plat plus sapide.

Les groseilles donnent à la préparation une petite allure exotique, le riz aux
cèpes l'accompagne parfaitement et le plat ainsi combiné est à la fois original
et bon.

\section*{\centering Agneau sauté, à la macédoine de légumes.}
\phantomsection
\addcontentsline{toc}{section}{ Agneau sauté, à la macédoine de légumes.}
\index{Agneau sauté, à la macédoine de légumes}

Pour six personnes prenez :

\medskip

\footnotesize
\begin{longtable}{rp{2em}rrrp{16em}}
& \multicolumn{3}{r}{2 kilogrammes} & d' & épaule d'agneau,                                               \\
& &    75 & grammes & de & gelée de veau et volaille,                                                     \\
& &    60 & grammes & d' & oignons,                                                                       \\
& &    50 & grammes & de & beurre,                                                                        \\
& &       &         &    & macédoine de légumes,                                                          \\
& &       &         &    & sel et poivre.                                                                 \\
\end{longtable}
\normalsize

Préparez une macédoine de légumes à la crème, comme il est dit \hyperlink{p0773}{p. \pageref{pg0773}}.

Coupez l'agneau en morceaux, faites-le sauter dans le beurre, avec les oignons,
pendant une demi-heure environ ; assaisonnez avec sel et poivre. Lorsque la
viande est à point, dégraissez, puis mettez la gelée de veau et volaille.
Chauffez sans laisser bouillir.

Dressez les morceaux d'agneau sur un plat chaud, masquez-les avec la sauce
passée, disposez autour la macédoine de légumes et servez.


\section*{\centering Côtelettes d'agneau à la Villeroi\footnote{On désigne sous le qualificatif de
                                                                « à la Villeroi » des préparations
                                                                culinaires dont la création est
                                                                attribuée au chef des cuisines du
                                                                maréchal de Villeroi.
                                                                \protect\endgraf
                                                                % \newline
                                                                Les éléments du plat, cuits ou non
                                                                au préalable, sont d'abord enrobés
                                                                dans une sauce serrée, puis panés et
                                                                dorés dans du beurre, ou frits.}.}
\phantomsection
\addcontentsline{toc}{section}{ Côtelettes d'agneau à la Villeroi.}
\index{Côtelettes d'agneau à la Villeroi}
\index{Agneau à la Villeroi}

Prenez un carré d'agneau de Pauillac de deux mois, bien blanc et bien gras ;
faites-le rôtir ; débitez-le en côtelettes ; laissez-les refroidir. Enrobez
chaque côtelette dans de la sauce allemande concentrée et rendue plus
consistante par l'addition de pied de veau aux éléments qui la constituent ;
laissez prendre. Passez les côtelettes, ainsi enrobées, successivement dans de
l'œuf battu et dans de la mie de pain rassis tamisée additionnée ou non de
fromage de Gruyère râpé et faites-les dorer dans du beurre des deux côtés.

Dressez les côtelettes en couronne sur un plat décoré ou non de persil frit et
servez-les telles quelles ou accompagnées d'une sauce, au goût.

\sk

Comme variantes, on pourra ajouter à la sauce Villeroi ordinaire ci-dessus des
truffes, de la purée de tomates, de la purée d'oignons, de l'appareil d'Uxel, du
mirepoix, etc.

\index{Farce à la Villeroi}
On pourra aussi transformer la sauce Villeroi en une véritable farce en
y incorporant du coulis de volaille.

\sk

\index{Chevreau à la Villeroi}
\index{Côtelettes de chevreau à la Villeroi}
\index{Poulet à la Villeroi}
On peut apprêter de même du chevreau et du poulet.

\sk

\index{Fonds d'artichauts à la Villeroi}
Certains légumes tels que fonds d'artichauts, pommes de terre farcies ou non,
pourront être traités de même.

\sk

On peut également préparer dans le même esprit des filets ou des tranches
minces de poissons, des huîtres, etc., mais, dans ces cas, la sauce qui les
enrobera sera constituée par une réduction d'un mélange de béchamel maigre et
de fond de poisson liée avec des jaunes d'œufs. Les filets ou les tranches
minces ainsi enrobés et panés seront cuits rapidement dans de la friture très
chaude.

\section*{\centering Gigot d'agneau braisé, aux oignons.}
\phantomsection
\addcontentsline{toc}{section}{ Gigot d'agneau braisé, aux oignons.}
\index{Gigot d'agneau braisé, aux oignons}

Pour quatre à cinq personnes prenez :

\medskip

\footnotesize
\begin{longtable}{rrrp{16em}}
    800 & grammes  & de & bouillon,                                                                       \\
    100 & grammes  & de & sauce Lomale,                                                                   \\
     90 & grammes  & de & beurre frais,                                                                   \\
     50 & grammes  & de & jus de viande,                                                                  \\
     50 & grammes  & de & fine champagne,                                                                 \\
      6 & grammes  & de & sucre en poudre,                                                                \\
        & 2 gr. 50 & de & fécule,                                                                         \\
        &          & 30 & petits oignons, pesant chacun 5 à 6 grammes,                                    \\
        &          &  3 & gros oignons, pesant ensemble 250 grammes,                                      \\
        &          &  1 & gigot d'agneau de Pauillac, pesant, non paré, 1 kilogramme environ,             \\
        &          &    & sel et poivre.                                                                  \\
\end{longtable}
\normalsize

Mettez le gigot dans une braisière avec {\ppp30\mmm} grammes de beurre, les
trois gros oignons et {\ppp200\mmm} grammes de bouillon ; ajoutez un peu de sel
et de poivre ; laissez mijoter.

Après une heure et demie, retirez les oignons, réduisez le jus et faites dorer
la viande de tous les côtés. Lorsque le gigot a bien pris couleur, ajoutez 400
grammes de bouillon, la fine champagne, le jus de viande et la sauce tomate ;
laissez mijoter encore pendant une heure et demie.

Durant la cuisson du gigot, faites sauter dans une casserole les {\ppp60\mmm}
petits oignons avec le reste du beurre, saupoudrez avec le sucre, mouillez avec
le reste du bouillon, moins {\ppp20\mmm} grammes que vous réserverez, et
laissez cuire de façon à conserver les oignons entiers.

Lorsqu'ils seront tendres, activez le feu, réduisez le jus, faites prendre aux
oignons une belle couleur jaune, tout en les remuant constamment pour empêcher
qu'ils s'attachent à la casserole ; mouillez ensuite avec {\ppp200\mmm} grammes
de jus de cuisson du gigot, liez la sauce avec la fécule délayée dans le
bouillon froid réservé et, dix minutes avant de servir, versez le tout dans la
braisière,

Dressez le gigot sur un plat, masquez-le avec la sauce et disposez autour les
oignons comme garniture.

Ce gigot est tellement fondant qu'on peut le manger à la cuiller.

\section*{\centering Gigot d'agneau de lait à l'ananas, braisé au porto.}
\phantomsection
\addcontentsline{toc}{section}{ Gigot d'agneau de lait à l'ananas, braisé au porto.}
\index{Gigot d'agneau de lait à l'ananas, braisé au porto}
\index{Agneau de lait braisé au porto}
\index{Agneau de lait à l'ananas}

Prenez un gigot d'agneau bien fin ; désossez-le, parez-le et, avec un
instrument en corne, soulevez-en la peau et introduisez des tranches d'ananas
frais entre cuir et chair.

Mettez-le dans une casserole avec du beurre, des carottes et un peu d'oignon ;
faites prendre couleur ; puis disposez le tout dans une braisière ; mouillez
avec du porto blanc ; salez, poivrez et laissez cuire à petit feu pendant deux
heures en arrosant avec le jus durant la cuisson.

Dressez le gigot sur un plat, passez dessus la sauce et servez avec du riz sec.

C'est un plat curieux qui procure une sensation inédite.

\section*{\centering Chaud-froid\footnote{\index{Chauds-froids (Définition des)}
                                          \index{Définition des chauds-froids}
                                          L'étymologie du mot chaud-froid est incertaine. Les uns
                                          prétendent que ce serait un nommé Chaufroix ou Chaufroid
                                          qui le premier aurait préparé le plat, et ils écrivent
                                          le mot comme le nom ; d’autres soutiennent que ce vocable
                                          bizarre signifie que le mets, préparé à chaud, doit être
                                          servi froid, étymologie simpliste et manquant de précision
                                          car, dans ces conditions, il serait applicable à toutes
                                          les substances cuites qu'on mange froides,
                                          \protect\endgraf
                                          \index{Chaud-froid de petits gibiers}
                                          \index{Aspic de petits gibiers}
                                          En réalité, le mot chaud-froid, dont nous ignorerons
                                          probablement toujours l'origine, sert à désigner des pièces
                                          montées, plus ou moins décorées, composées le plus souvent
                                          de membres d'agneau, de volaille, de gibier ou encore de
                                          petits gibiers entiers, en aspic.} d'agneau de lait.}
\phantomsection
\addcontentsline{toc}{section}{ Chaud-froid d'agneau de lait.}
\index{Chaud-froid d'agneau de lait}
\index{Agneau de lait en chaud-froid}
\index{Agneau rôti}

Préparez d'abord une bonne gelée de la façon suivante : mettez dans une
casserole du beurre, du veau et des os de veau, des oignons, des carottes, un
bouquet garni ; faites pincer au four, mouillez avec du consommé ; laissez
mijoter pendant {\ppp5\mmm} à {\ppp6\mmm} heures. Ajoutez ensuite du porto blanc et continuez la
cuisson pendant le même laps de temps. Concentrez et dépouillez bien le jus. Un
quart d'heure avant la fin, corsez avec un petit verre de vieille fine
champagne par litre de jus.

Prenez un jeune agneau ; videz-le, coupez-lui tête et pattes : mettez dans
l'intérieur des herbes aromatiques hachées : thym, serpolet, marjolaine, etc.,
salez, poivrez ; cousez les orifices et faites rôtir l'agneau à la broche
devant un grand feu de bois en le caressant simplement avec un petit drapeau
trempé dans de la graisse additionnée de jus de citron.

Faites cuire du foie gras dans du porto blanc ; passez-le en purée avec des
feuilles d'estragon.

Lorsque l'agneau est cuit, détachez-en la selle et mettez à sa place la purée
de foie gras ; découpez la selle ; remettez-la en place au-dessus de la purée
de foie gras ; masquez le tout avec la gelée, décorez avec des truffes cuites
dans du porto et laissez prendre.

\index{Barquettes de salade de légumes à la mayonnaise}
Garnissez le plat avec des barquettes de salade de légumes à la mayonnaise.

C'est très délicat.

\section*{\centering Filets mignons de porc panés, grillés, sauce Robert\footnote{La sauce Robert était
                                                 connue déjà sous Charles VI. Les préparations modernes
                                                 de cette sauce différent sensiblement comme finesse de
                                                 la sauce primitive.}.}
\phantomsection
\addcontentsline{toc}{section}{ Filets mignons de porc panés, grillés, sauce Robert.}
\index{Filets mignons de porc panés, grillés, sauce Robert}

Pour six personnes prenez :

\medskip

\footnotesize
\begin{longtable}{rrrp{16em}}
    300 & grammes & de & sauce demi-glace, \hyperlink{p0456}{p. \pageref{pg0456}},                        \\
    200 & grammes & de & chablis ou de pouilly sec,                                                       \\
     50 & grammes & de & beurre,                                                                          \\
        &         & 12 & filets mignons de porc, parés,                                                   \\
        &         &  1 & oignon moyen,                                                                    \\
        &         &    & lard à piquer,                                                                   \\
        &         &    & mie de pain rassis tamisée,                                                      \\
        &         &    & moutarde,                                                                        \\
        &         &    & cayenne ou paprika,                                                              \\
        &         &    & sel, poivre.                                                                     \\
\end{longtable}
\normalsize

Piquez les filets mignons de fins lardons assaisonnés de sel et de poivre ;
passez-les d'abord dans {\ppp25\mmm} grammes de beurre fondu et légèrement tiède, puis
dans de la mie de pain rassis tamisée.

Préparez la sauce. Ciselez l'oignon, faites-le dorer doucement dans le reste du
beurre, mouillez ensuite avec le vin, réduisez le liquide au quart de son
volume, puis ajoutez la sauce demi-glace. Laissez mijoter le tout pendant
environ une demi-heure. Passez la sauce au chinois, mettez-la au point, hors du
feu, avec moutarde, cayenne ou paprika, au goût, mélangez bien. Tenez la sauce
au chaud au bain-marie.

Faites griller les filets mignons, dressez-les sur un plat, entourez-les de
{\ppp12\mmm} tomates farcies de champignons grillés,
\hyperlink{p0766}{p. \pageref{pg0766}} et servez. Envoyez en même temps la sauce
dans une saucière.

\sk

On peut préparer de même une sauce Robert maigre pour le poisson, en remplaçant
la sauce demi-glace par du fond de poisson très concentré et lié.

\section*{\centering Côtelettes de porc grillées, à la purée de poireaux.}
\phantomsection
\addcontentsline{toc}{section}{ Côtelettes de porc grillées, à la purée de poireaux.}
\index{Côtelettes de porc grillées, à la purée de poireaux}

Pour quatre personnes prenez :

\medskip

\footnotesize
\begin{longtable}{rrrp{16em}}
    600 & grammes & de & poireaux (le blanc seulement),                                                   \\
    100 & grammes & de & beurre,                                                                          \\
     45 & grammes & de & bon jus de viande,                                                               \\
     10 & grammes & de & farine,                                                                          \\
        &         &  4 & côtelettes de porc panées,                                                       \\
        &         &    & sel et poivre.                                                                   \\
\end{longtable}
\normalsize

Faites cuire les poireaux dans {\ppp3\mmm} litres d'eau salée à raison de {\ppp15\mmm} grammes de
sel gris par litre, pendant le temps nécessaire pour les bien cuire ;
pressez-les pour leur enlever tout excès d’eau ; hachez-les ; séchez-les
ensuite un peu dans une casserole, sur le coin du fourneau, pendant une heure
environ ; puis, réduisez-les en purée, passez-les ; réservez-en {\ppp30\mmm} grammes pour
la sauce.

Incorporez au reste de la purée {\ppp75\mmm} grammes de beurre, salez et poivrez, mettez
par exemple {\ppp5\mmm} grammes de sel et {\ppp1\mmm} gramme de poivre.

Faites un roux avec le reste du beurre et la farine ; mouillez avec le jus de
viande, assaisonnez avec {\ppp3\mmm} grammes de sel et {\ppp2\mmm} décigrammes de poivre, si le jus
n'est ni trop salé, ni trop poivré, et mélangez-y les {\ppp30\mmm} grammes de purée
réservée.

Pendant les dernières opérations, faites griller les côtelettes telles
quelles ; assaisonnez-les.

Dressez les côtelettes sur la purée et envoyez la sauce à part.

La purée de poireaux remplace ici la classique purée d'oignons et la sauce aux
poireaux, que j'appellerai sauce « Mérite agricole », remplace la sauce
Soubise\footnote{La sauce Soubise est une sauce au jus, à la purée d'oignons.}.

\section*{\centering Paupiettes de porc rôties.}
\phantomsection
\addcontentsline{toc}{section}{ Paupiettes de porc rôties.}
\index{Paupiettes de porc rôties}

Pour quatre personnes prenez :

\medskip

\footnotesize
\begin{longtable}{rrrp{16em}}
    750 & grammes & de & \hangindent=1em filet de porc ou de jambon frais, en quatre tranches parées,     \\
    150 & grammes & de & crème,                                                                           \\
        &         &  4 & bardes de lard,                                                                  \\
        &         &    & moutarde,                                                                        \\
        &         &    & mie de pain rassis tamisée,                                                      \\
        &         &    & jus de citron,                                                                   \\
        &         &    & sel, poivre, paprika.                                                            \\
\end{longtable}
\normalsize

Enduisez de moutarde chaque tranche de porc sur les deux faces ; saupoudrez-les
de mie de pain ; roulez-les en paupiettes ; bardez-les, ficelez-les.

Tenez-les au frais pendant {\ppp24\mmm} heures, puis faites-les rôtir à la broche en les
arrosant avec la crème acidulée par du jus de citron et assaisonnée avec sel,
poivre et paprika, au goût.

Dégraissez la sauce ; enlevez des paupiettes les ficelles et les résidus de
bardes.

Servez, avec accompagnement de choucroute au naturel, par exemple.

\section*{\centering Jambon en croûte.}
\phantomsection
\addcontentsline{toc}{section}{ Jambon en croûte.}
\index{Jambon en croûte}

Pour douze personnes prenez :

\footnotesize
\begin{longtable}{@{}lrrrp{16em}}
\setlength\LTleft\parindent
\normalsize1°\footnotesize \hspace{4em} & 200 & grammes   & d' & oignons,                                 \\
\hspace{4em}   &  200 & grammes   & de & carottes,                                                        \\
\hspace{4em}   &   50 & grammes   & de & persil,                                                          \\
\hspace{4em}   &    2 & grammes   & de & thym,                                                            \\
\hspace{4em}   &    1 & bouteille & de & vin blanc,                                                       \\
\hspace{4em}   &    1 & bouteille & de & madère, de xérès ou de marsala, au goût,                         \\
\hspace{4em}   &      &           &  6 & échalotes,                                                       \\
\hspace{4em}   &      &           &  4 & feuilles de laurier,                                             \\
\hspace{4em}   &      &           &  2 & gousses d'ail,                                                   \\
\hspace{4em}   &      &           &  1 & jambon pesant environ 4 kilogrammes ;                            \\
\end{longtable}
\normalsize

\footnotesize
\begin{longtable}{@{}lrrrp{16em}}
\setlength\LTleft\parindent
\normalsize 2° & \multicolumn{4}{l}{\normalsize   pour la croûte :}                                       \\
\footnotesize
\hspace{4em}   &      &         &    &                                                                    \\
\hspace{4em}   &  750 & grammes & de & farine,                                                            \\
\hspace{4em}   &  500 & grammes & de & lait,                                                              \\
\hspace{4em}   &  250 & grammes & de & beurre,                                                            \\
\hspace{4em}   &   20 & grammes & de & sel,                                                               \\
\hspace{4em}   &      &         &  2 & œufs frais.                                                        \\
\end{longtable}
\normalsize

\label{pg0537} \hypertarget{p0537}{}
\index{Fond de cuisson pour jambon}
Faites cuire le jambon aux trois quarts. Pour faire cette opération, mettez-le
dans une jambonnière avec de l'eau, amenez à ébullition ; au premier bouillon,
retirez l'eau qui emportera l'excès de sel du jambon ; remplacez-la par d'autre
eau bouillante, ajoutez le vin blanc, les légumes et les aromates, puis
continuez la cuisson pendant un temps calculé à raison d'une demi-heure par
kilogramme de jambon.

Laissez refroidir le jambon dans sa cuisson.

Retirez-le, parez-le, enlevez la couenne.

Préparez une pâte avec les éléments indiqués pour la croûte ; enveloppez le
jambon avec cette pâte en lui conservant sa forme, décorez la surface à la
pince et faites cuire au four pendant une heure et demie.

Décalottez alors le jambon, coupez en tranches les parties mises à nu,
arrosez-les avec un peu de madère, de xérès ou de marsala, tiédi au préalable,
puis glacez- les au four.

Servez avec des épinards ou des pointes d'asperges et envoyez en même temps
une saucière de sauce madère, xérès ou marsala, \hyperlink{p0459}{p. \pageref{pg0459}}.

\section*{\centering Porc aux choux.}
\phantomsection
\addcontentsline{toc}{section}{ Porc aux choux.}
\index{Porc aux choux}

Pour quatre personnes prenez :

\medskip

\footnotesize
\begin{longtable}{rrrp{16em}}
  1 500 & grammes & de & \hangindent=1em chou coupé comme pour la choucroute, mais en morceaux plus gros, \\
    600 & grammes & de & pointe de porc frais, coupée en morceaux,                                        \\
    200 & grammes & de & sel gris,                                                                        \\
     60 & grammes & de & cèpes secs,                                                                      \\
     50 & grammes & de & beurre,                                                                          \\
     50 & grammes & de & vin blanc,                                                                       \\
     15 & grammes & de & farine,                                                                          \\
      7 & grammes & de & poivre fraîchement moulu.                                                        \\
\end{longtable}
\normalsize

Mettez le chou dans un vase avec le sel ; gardez au frais pendant trente-six
heures. Au bout de ce temps, enlevez le chou et lavez-le rapidement à l'eau
froide.

Mettez le porc et le chou dans une casserole, chauffez, mélangez ; puis,
mouillez avec le vin blanc, poivrez et laissez mijoter à petit feu pendant sept
heures.

Réchauffez le tout le lendemain.

Faites tremper les cèpes, coupez-les en morceaux, ajoutez-les au chou au bout
d'une heure de réchauffage, et continuez la cuisson pendant deux heures
environ.

Un quart d'heure avant la fin, faites un roux avec le beurre et la farine,
incorporez-le au contenu de la casserole et servez.

\section*{\centering Saucisses au vin blanc.}
\phantomsection
\addcontentsline{toc}{section}{ Saucisses au vin blanc.}
\index{Saucisses au vin blanc}
\index{Canapés de saucisses}

Pour six personnes prenez :

\medskip

\footnotesize
\begin{longtable}{rrrp{16em}}
    500 & grammes & de & saucisses longues, ce qui correspond à une douzaine de saucisses, en moyenne.    \\
    400 & grammes & de & vin blanc,                                                                       \\
    100 & grammes & de & beurre,                                                                          \\
    100 & grammes & d' & oignons,                                                                         \\
     40 & grammes & de & glace de viande,                                                                 \\
     20 & grammes & de & purée de tomates aromatisées,                                                    \\
     10 & grammes & de & farine,                                                                          \\
        &         &    & pain anglais.                                                                    \\
\end{longtable}
\normalsize

Faites fondre {\ppp80\mmm} grammes de beurre dans une poêle, mettez dedans les saucisses
et les oignons pelés et émincés ; laissez cuire ensemble, à petit feu, pendant
une dizaine de minutes, puis retirez les saucisses et tenez-les au chaud.
Mettez alors la farine, mouillez avec le vin blanc, faites réduire, ajoutez
ensuite la glace de viande et la purée de tomates ; laissez mijoter et réduire
encore, puis passez et finissez la sauce avec le reste du beurre.

Dressez les saucisses sur des canapés de mie de pain anglais, dorés dans du
beurre et disposés sur un plat, masquez avec la sauce et servez.

\section*{\centering Petits sandwichs chauds au jambon et au fromage.}
\phantomsection
\addcontentsline{toc}{section}{ Petits sandwichs chauds au jambon et au fromage.}
\index{Petits sandwichs chauds au jambon et au fromage}

Pour huit personnes prenez :

\medskip

\footnotesize
\begin{longtable}{rrrp{16em}}
    250 & grammes & de & pain anglais,                                                                    \\
    125 & grammes & de & beurre fin,                                                                      \\
    100 & grammes & de & jambon anglais, coupé en feuilles minces,                                        \\
     50 & grammes & de & gruyère râpé.                                                                    \\
\end{longtable}
\normalsize

Enlevez la croûte du pain anglais et découpez la mie en huit tranches. Taillez
ensuite chacune de ces tranches en quatre, soit perpendiculairement aux axes en
morceaux carrés, soit diagonalement en morceaux triangulaires. Beurrez les
trente-deux tranches sur une face, saupoudrez de fromage râpé, puis interposez
entre deux tranches ainsi apprêtées, faces beurrées en dedans, une feuille de
jambon, et faites prendre couleur, dans du beurre, à la poêle, sur un feu doux.
Au bout de deux minutes et demie les sandwichs seront convenablement dorés d'un
côté ; retournez-les, faites-les dorer de même de l'autre côté et servez.

C'est un excellent hors-d'œuvre chaud,

\section*{\centering Croûte au jambon et au foie gras.}
\phantomsection
\addcontentsline{toc}{section}{ Croûte au jambon et au foie gras.}
\index{Croûte au jambon et au foie gras}

Préparez une croûte comme pour une tarte, mais pas sucrée ; emplissez-la
d'un corps inerte ; faites-la cuire.

Garnissez-en le fond d'une couche de purée de foie gras cuit au porto, mettez
dessus un mélange de petits cubes de jambon de Bayonne blanchi et revenu et de
petits cubes de truffe cuite dans du porto ; masquez le tout avec une béchamel
grasse additionnée ou non d'un peu de parmesan râpé et faites gratiner au four.

\sk

\index{Croûtes au ris de veau truffé et au foie gras}
On pourra faire dans le même esprit des petites croûtes ou des tartelettes de
ris de veau truffé au foie gras, à la sauce suprême.

\section*{\centering Chaussons au jambon.}
\phantomsection
\addcontentsline{toc}{section}{ Chaussons au jambon.}
\index{Chaussons au jambon}

Coupez en morceaux des tranches de jambon fumé, cuit au vin blanc.

Pelez des truffes et faites-les cuire dans du madère.

Prenez un beau foie gras d'oie ; piquez-le de baguettes de truffes cuites et
coupez-le en tranches.

Préparez la croûte des chaussons, comme il est dit \hyperlink{p0289}{p. \pageref{pg0289}}.

Mettez sur la moitié de chaque rond de pâte une couche de jambon et par-dessus
une tranche de foie gras. Fermez les chaussons ; réservez dans chacun une
ouverture de {\ppp2\mmm} centimètres et demi de diamètre avec couvercle, comme cela se
fait pour les bouchées et faites cuire au four.

Quand les chaussons seront cuits, emplissez-les, par l'orifice, d'une sauce
Périgueux\footnote{\label{pg0540} \hypertarget{p0540}{}Pour préparer la sauce Périgueux, prenez :
                   \protect\endgraf
                   \begin{longtable}{rrrp{16em}}
                       300 & grammes & de & jambon de Bayonne,                                            \\
                       200 & grammes & de & bon jus,                                                      \\
                       200 & grammes & de & madère,                                                       \\
                        40 & grammes & de & beurre,                                                       \\
                        30 & grammes & de & farine,                                                       \\
                           &         &  1 & oignon coupé en rouelles,                                     \\
                           &         &  1 & échalote,                                                     \\
                           &         &    & truffes à volonté,                                            \\
                           &         &    & sel et poivre.                                                \\
                   \end{longtable}
                   \protect\endgraf
                   Faites cuire les truffes dans Le madère ; hachez-les.
                   \protect\endgraf
                   Faites un roux avec le beurre et la farine, ajoutez l'oignon et
                   l'échalote, laissez dorer légèrement, puis mettez le jambon coupé
                   en petits morceaux, mouillez avec le madère et le jus ; laissez
                   cuire. Dépouillez la sauce, c'est-à-dire enlevez les peaux qui se
                   forment à la surface, concentrez-la, passez-la ; goûtez et rectifiez
                   l'assaisonnement s'il ya lieu avec sel et poivre, puis ajoutez les
                   truffes.}, à laquelle vous aurez ajouté du fumet de gibier, puis
                   servez.


\section*{\centering Petits pains à la saucisse.}
\phantomsection
\addcontentsline{toc}{section}{ Petits pains à la saucisse.}
\index{Petits pains à la saucisse}

Préparez de la pâte à pain avec de la farine, de l'eau, un peu de levain et de
sel ; pétrissez bien.

Divisez la pâte en morceaux pesant {\ppp50\mmm} grammes environ ; aplatissez-les un peu,
mettez sur chacun une saucisse crue que vous enfermerez dans la pâte.
Laissez-les reposer pendant une demi-heure, puis faites-les cuire au four
chaud.

C'est un excellent en cas pour excursions à la campagne.

\sk

On peut préparer de même des petits pains à la viande. Coupez des tranches de
viande crue, juteuse, de grandeur convenable ; assaisonnez-les ; enrobez-les
dans de la pâte et faites-les cuire au four.

\sk

\index{Friands}
En remplaçant la pâte à pain par de la pâte feuilletée et la saucisse par une
feuille de jambon enrobée dans un hachis de filet de porc, lard et veau, on aura
ce qu'on appelle les « Friands ».

\section*{\centering Boulettes au jambon.}
\phantomsection
\addcontentsline{toc}{section}{ Boulettes au jambon.}
\index{Boulettes au jambon}
\label{pg0541} \hypertarget{p0541}{}

Pour trois personnes prenez :

\medskip

\footnotesize
\begin{longtable}{rrrp{16em}}
    250 & grammes & de & mie de pain blanc ou bis,                                                        \\
    200 & grammes & de & beurre,                                                                          \\
    150 & grammes & de & lait,                                                                            \\
    120 & grammes & de & jambon cru,                                                                      \\
     10 & grammes & d' & un mélange en parties égales de cerfeuil, ciboule et persil hachés,              \\
        &         &  2 & œufs frais,                                                                      \\
        &         &    & chapelure,                                                                       \\
        &         &    & farine,                                                                          \\
        &         &    & sel et poivre.                                                                   \\
\end{longtable}
\normalsize

Faites bouillir le lait.

Découpez dans la mie de pain des petits cubes ; faites-les dorer dans 100
grammes de beurre, puis mettez-les dans le lait bouillant ; laissez tremper.
Écrasez la mie de pain dans le lait, incorporez-y les œufs l'un après l'autre,
triturez afin d'obtenir une pâte homogène.

Faites revenir, dans {\ppp50\mmm} grammes de beurre, le jambon coupé en petits morceaux,
le cerfeuil, la ciboule et le persil hachés ; mélangez le tout à la pâte
précédente ; goûtez et complétez l'assaisonnement, s'il est nécessaire, avec
sel et poivre.

Préparez avec ce mélange des boulettes de la grosseur d'un abricot ; roulez-les
dans de la farine et plongez-les pendant {\ppp5\mmm} minutes dans de l'eau salée
bouillante.

Égouttez-les, dressez-les sur un plat et masquez-les avec de la chapelure
revenue dans le reste du beurre.

Servez en envoyant en même temps une saucière de jus de viande et un ravier
de gruyère ou de parmesan râpé.

\sk

Comme variante, on peut faire frire les boulettes dans de la graisse et les
servir avec une sauce tomate, par exemple.

\section*{\centering Rillons.}
\phantomsection
\addcontentsline{toc}{section}{ Rillons.}
\index{Rillons}

Les rillons sont constitués par des morceaux plus ou moins cubiques de porc
entrelardé cuit dans sa propre graisse.

Pour cinq à six personnes prenez :

\medskip

\footnotesize
\begin{longtable}{rrrrp{16em}}
 &  1 500 & grammes & de & poitrine de porc frais, pas trop grasse,                                       \\
 &     20 & grammes & de & sel gris,                                                                      \\
 &      & 1 gr. 50  & de & poivre fraîchement moulu,                                                      \\
 &  \multicolumn{2}{r}{2 décigrammes} & de & sauge en poudre,                                             \\
 &  \multicolumn{2}{r}{1 décigramme}  & de & quatre épices.                                               \\
\end{longtable}
\normalsize

Coupez la poitrine de porc en carrés de {\ppp5\mmm} à {\ppp6\mmm} centimètres
de côté ; enlevez les os.

Faites revenir à petit feu les morceaux de poitrine dans une marmite de façon
à les dorer de tous les côtés, ce qui demande une demi-heure environ. Égouttez
la graisse ; salez, poivrez, ajoutez épices et sauge, couvrez la marmite et
faites cuire doucement pendant une heure à une heure et demie. Découvrez la
marmite, activez le feu afin de sécher les rillons. Enlevez-les, mettez-les sur
un plat, laissez-les refroidir.

\sk

On obtiendra des rillons plus fins en prenant :

\medskip

\footnotesize
\begin{longtable}{rrrrp{16em}}
 &  1 500 & grammes & de & porc frais entrelardé,                                                         \\
 &    125 & grammes & de & vin blanc sec,                                                                 \\
 &     30 & grammes & de & bon cognac,                                                                    \\
 &     20 & grammes & de & sel gris,                                                                      \\
 &      & 1 gr. 50  & de & poivre fraîchement moulu,                                                      \\
 &  \multicolumn{2}{r}{2 décigrammes} & de & sauge en poudre,                                             \\
 &  \multicolumn{2}{r}{1 décigramme}  & de & quatre épices,                                               \\
 &       &         &  1 & échalote hachée (facultatif).                                                   \\
\end{longtable}
\normalsize

Apprêtez et faites revenir la poitrine comme ci-dessus ; égouttez la graisse,
flambez ensuite avec le cognac, mouillez avec le vin et continuez le reste de
l'opération comme il est dit plus haut.

\sk

On sert les rillons comme hors-d'œuvre,

\section*{\centering Rillettes.}
\phantomsection
\addcontentsline{toc}{section}{ Rillettes.}
\index{Rillettes}

Les rillettes sont constituées par de la viande de porc entrelardée et
aromatisée cuite dans sa propre graisse, hachée et passée au tamis avec la
graisse qu'elle a rendue. Elles sont additionnées presque toujours de foie de
porc cuit à part et passé au tamis.

On peut encore ajouter aux rillettes d'autres viandes, de la volaille et en
particulier de l’oie.

\medskip

Voici une formule de rillettes avec de l'oie.

\medskip

\footnotesize
\begin{longtable}{rrrp{16em}}
  1 000 & grammes & de & porc frais maigre, filet ou jambon,                                              \\
  1 000 & grammes & de & poitrine de porc frais,                                                          \\
  1 000 & grammes & de & chair d'oie,                                                                     \\
    500 & grammes & de & foie gras d'oie,                                                                 \\
    250 & grammes & de & graisse de rôti de porc ou de volaille,                                          \\
     75 & grammes & de & sel,                                                                             \\
     65 & grammes & d' & échalotes,                                                                       \\
     15 & grammes & d' & épices,                                                                          \\
      1 & gramme  & de & thym,                                                                            \\
    1/2 & gramme  & de & laurier,                                                                         \\
    1/2 & gramme  & de & cayenne,                                                                         \\
        &         &  3 & clous de girofle,                                                                \\
        &         &    & ail (facultatif).                                                                \\
\end{longtable}
\normalsize

Mettez dans un sachet échalotes, laurier, thym, clous de girofle et ail.

Coupez en petits morceaux le porc et l’oie.

Faites cuire doucement dans une marmite les viandes coupées, la graisse et le
sachet d'aromates, pendant cinq heures, en remuant fréquemment avec une
cuiller en bois. Assaisonnez ensuite avec sel, épices, cayenne, puis ajoutez le foie
gras d'oie coupé en petits cubes. Mélangez bien et laissez cuire encore pendant
un quart d'heure.

Retirez le sachet, hachez fin les viandes ou passez le tout au tamis ; mettez
en pots ; laissez refroidir.

Couvrez les pots avec de la graisse fondue et du papier d'étain. Tenez les pots
au frais.

C'est un très bon hors-d'œuvre.

\section*{\centering Cochon de lait farci, rôti.}
\phantomsection
\addcontentsline{toc}{section}{ Cochon de lait farci, rôti.}
\index{Cochon de lait farci, rôti}

Prenez un petit cochon de lait bien blanc, de six à huit semaines ;
échaudez-le ; grattez-le, lavez-le, videz-le ; réservez la fressure. Flambez-le
pour enlever les derniers poils qui auraient pu rester sur la peau et
assaisonnez-en l'intérieur avec sel et poivre.

Hachez la fressure, faites-la revenir dans du beurre avec du persil, du
cerfeuil, de l'estragon, de la ciboulette, du thym et de la sauge hachés ;
mélangez le tout avec du foie gras d'oie et de la chair de porc frais hachée et
aromatisée comme la fressure ; assaisonnez avec sel et poivre ; cela
constituera la farce.

Mettez cette farce dans le corps de l'animal ; troussez-le en plaçant les pattes de
devant sous la tête, les pattes de derrière sous le corps, la queue dans le croupion,
et enveloppez les oreilles avec un papier beurré.

Faites-le rôtir, soit au four, sur une grille placée au-dessus d'un plat, ou
mieux à la broche, devant un grand feu de bois, à la mode ancienne, en le
caressant fréquemment, pendant la cuisson, avec un petit drapeau trempé dans du
saindoux fondu relevé par du jus de citron. La durée de la cuisson varie d'une
heure et demie à deux heures un quart suivant la grosseur de l'animal et la
conduite du feu. La peau doit être rissolée et très croustillante.

Retirez le papier qui couvre les oreilles, débridez l'animal ; mettez-lui, d'un
côté, dans le groin. une belle petite pomme d’api bien rouge et, de l'autre, dans
le ... parfaitement, un joli cornichon bien vert.

Dressez l'animal, ainsi habillé, sur un plat garni d'une serviette et envoyez
en même temps, d'une part, le jus de cuisson dégraissé, dans une saucière, et,
d'autre part, de la purée de pommes de terre et de haricots verts, par exemple,
dans un légumier.

Découpez la bête : faites que chaque convive ait un morceau de cochon recouvert
de peau et de la farce.

C'est vraiment très bon.

\sk

Comme variante, on peut farcir le cochon de lait avec du risotto incomplètement
cuit dans lequel on aura incorporé la fressure du cochon aromatisée comme
précédemment et passée en purée.

\sk

En Russie, on farcit le cochon de lait avec de la semoule de sarrasin et on
l'accompagne avec une sauce au raifort.
