\section*{\centering Pot-au-feu de famille.}
\addcontentsline{toc}{section}{Pot-au-feu de famille.}
\index{Pot-au-feu de famille}

\justify

\setlength\tabcolsep{.3em}

\label{pg0199} \hypertarget{p0199}{}
\index{Garnitures pour pot-au-feu}
\index{Bœuf bouilli}
Lorsqu'on prépare un pot-au-feu en famille, on désire généralement avoir à la
fois un bon bouillon et un bon bouilli. Voici une formule qui permet d'arriver
au résultat désiré.

\smallskip

Pour huit\footnote{L'indication du nombre de personnes auquel un plat déterminé
peut être servi est donnée pour des repas de famille. Lorsqu'on a un menu
copieux, on peut facilement doubler le nombre.} personnes, prenez :

\footnotesize
\setlength\tabcolsep{.15em}
\begin{longtable}{rrrrp{16em}}   
  & \multicolumn{2}{r}{2 kilogrammes} & de & poitrine de bœuf prise dans le milieu 
                                            du morceau, qui fait un excellent bouilli,                    \\
  &   500 & grammes & d' & os de crosse et gîte,                                                          \\
  &   500 & grammes & de & carottes,                                                                      \\
  &   150 & grammes & de & foie de bœuf\footnote{Le rôle du foie de bœuf  est d'éclaircir le bouillon.},  \\
  &   125 & grammes & de & navets,                                                                        \\
  &    60 & grammes & de & sel gris,                                                                      \\
  &     6 & grammes & de & cosses de pois\footnote{Le rôle des cosses
                                               de pois est de colorer le bouillon, tout en le          
                                               parfumant légèrement. Si l'on désire avoir un           
                                               bouillon très coloré, on ajoutera un peu de             
                                               caramel.}, séchées au four,                                \\
  &     1 & gramme  & de & poivre (facultatif),                                                           \\
  &      & 4 litres & d' & eau\footnote{La pureté de l’eau employée a une très grande importance 
                                    dans toutes les préparations.}                                        \\
  &       &         &  4 & poireaux moyens (le blanc seulement),                                          \\
  &       &         &  2 & abatis de volaille,                                                            \\
  &       &         &  2 & clous de girofle\footnote{ Boutons florifères de l'Eugenia aromatica ; 
                                             famille des Myrtacées.}, qu'on piquera dans le panais,       \\
  &       &         &  1 & petit morceau de panais,                                                       \\
  &       &         &  1 & petit morceau de céleri,                                                       \\
  &       &         &  1 & petite gousse d'ail,                                                           \\
  &       &         &  1 & oignon (facultatif),                                                           \\
  &       &         &  1 & petite feuille de laurier,                                                     \\
  &       &         &    & quelques brindilles de thym,                                                   \\
  &       &         &    & un peu de persil.                                                              \\
\end{longtable}
\normalsize

Nettoyez et parez les abatis.

Épluchez les légumes\footnote{Les quantités indiquées correspondent à des
légumes d'hiver.}, lavez-les. Faites un bouquet avec les cosses de pois, les
poireaux, le panais, le céleri, le thym, le laurier, le persil, l'ail et
l'oignon.

Mettez la viande, les os, le foie, l'eau, le sel et le poivre dans une marmite
en cuivre étamé ou dans une marmite en porcelaine épaisse allant au
feu\footnote{Je déconseille l'usage de la marmite classique en terre qui n'est
bonne que pendant très peu de temps ; au début, elle sent la terre ; au bout
d'un certain temps, elle sent le graillon.} ; amenez à ébullition, enlevez
l’écume grise au fur et à mesure qu'elle se forme, mais laissez l'écume blanche
qui se dissoudra et donnera bon goût, ajoutez les légumes et le bouquet ;
faites cuire à liquide légèrement frissonnant, sans couvrir hermétiquement la
marmite. Au bout de trois heures de cuisson, dégraissez, mettez les abatis et
laissez cuire encore pendant une heure.

Passez le bouillon au travers d'une passoire garnie d'un linge ; servez-le
soit avec des tranches de pain grillé ou non, soit avec du tapioca, des pâtes
alimentaires, du riz cuits dedans, ou encore avec des pommes de terre crues
râpées, passées à la passoire et séchées dans un linge, cuites dans le même
bouillon. On peut aussi servir le bouillon accompagné de toutes petites galettes
feuilletées qui sont délicieuses, de ravioli, de boulettes frites, de boulettes au
jambon, de pâtes au fromage, de quenelles, de profiteroles, de fromage de
Gruyère râpé, etc., etc.

Dressez ensuite le bouilli sur un plat, garnissez avec les abatis et les
légumes de la cuisson auxquels vous pourrez ajouter d'autres légumes tels que
pommes de terre, choux de Bruxelles, chou-fleur blanchis dans de l’eau salée et
dont vous aurez achevé la cuisson dans du beurre.

Servez avec l'accompagnement ordinaire de pickles, moutarde et gros sel, ou
avec des concombres verts en saumure, des betteraves au raifort confites, ou
encore avec une sauce au raifort que vous préparerez de la façon suivante.

\medskip

Pour huit personnes, prenez :

\footnotesize
\begin{longtable}{rlp{16em}}
  200 grammes                   & de              & crème,                                                \\
  200 grammes                   & de              & jus de viande\footnote{Le procédé le 
                                                    plus simple pour obtenir du jus de viande 
                                                    consiste à passer à la presse du bœuf grillé 
                                                    coupé en tranches.
                                                    \protect\endgraf
                                                    Je préfère le procédé suivant : Faire cuire 
                                                    dans une marmite dite « américaine » de la 
                                                    viande de bœuf coupée en morceaux gros comme 
                                                    des noix avec des légumes émincés, le tout 
                                                    assaisonné. La cuisson a lieu au bain-marie ; 
                                                    elle dure six heures, pendant lesquelles on 
                                                    doit avoir soin de maintenir l'eau bouillante 
                                                    du bain-marie au niveau du bouchon de la marmite. 
                                                    L'opération achevée, on recueille le jus et 
                                                    l'on y ajoute celui qu'on obtient en passant 
                                                    la viande à la presse pour l'épuiser complètement. 
                                                    1 kilogramme 1/2 de tranche donne environ 500 
                                                    grammes de jus. Pour corser les sauces, le mieux 
                                                    est d'employer de la viande sans os ni graisse 
                                                    et un mélange de légumes analogue à celui que 
                                                    j'ai indiqué dans la formule du 
                                                    bouillon corsé, \hyperlink{p0202}{p. \pageref{pg0202}}. 
                                                    Pour malades et pour convalescents, il 
                                                    est bon de ne mettre comme légumes que quelques 
                                                    rondelles de carotte et une rondelle de navet.}
                                                    ou de fond de veau et volaille, 
                                                    \hyperlink{p0418}{p. \pageref{pg0418}},               \\
  150 grammes                   & de              & raifort râpé,                                         \\
  60  grammes                   & de              & de beurre,                                            \\
  50  grammes                   & de              & de farine,                                            \\
                                &                 & jus de citron,                                        \\
                                &                 & sucre,                                                \\
                                &                 & sel et poivre.                                        \\
                                &                 &                                                       \\
\end{longtable}
\normalsize

Faites un roux avec le beurre et la farine ; mouillez avec le jus de viande ou
le fond de veau et volaille ; laissez cuire ; puis ajoutez, hors du feu, la
crème et Le raifort, assaisonnez avec sel, poivre, sucre et jus de citron au
goût ; chauffez sans laisser bouillir.

\smallskip

Une sauce hollandaise, dans laquelle on aura incorporé du raifort râpé,
accompagnera aussi très bien le bouilli chaud.

\smallskip

Avec le bouilli froid, on pourra servir : une sauce froide au raifort, préparée
avec de la crème aigrie par du jus de citron, du raifort râpé, de la moutarde,
du sel et du poivre au goût ; une sauce vinaigrette additionnée ou non de
raifort : une sauce douce ou encore une sauce mayonnaise à la ravigote ou au
raifort.

\section*{\centering Bouillon à bouilli perdu.}
\addcontentsline{toc}{section}{Bouillon à bouilli perdu.}
\index{Bouillon à bouilli perdu}
\label{pg0201} \hypertarget{p0201}{}

Lorsqu'on ne tient pas à utiliser le bouilli, il est facile de faire des
bouillons plus ou moins corsés, à volonté.

\medskip

Voici, comme exemples, trois formules de bouillons de forces croissantes.

\medskip

A — Bouillon léger, pour malades.

\index{Bouillon léger pour malades}
\footnotesize
\begin{longtable}{rrrp{16em}}
  500 & grammes    & d' & os de crosse et gîte,                                                           \\
  500 & grammes    & de & jarret de veau et nourrice,                                                     \\
   12 & grammes    & de & sel gris,                                                                       \\
    2 & litres 1/2 & d' & eau,                                                                            \\
      &            & 2  & abatis de poulets,                                                              \\
      &            & 1  & carotte moyenne,                                                                \\
      &            & 1  & navet moyen.                                                                    \\
\end{longtable}
\normalsize                                
\index{Bouillon pour convalescents}
B. — Bouillon pour convalescents.

\medskip

\footnotesize
\begin{longtable}{rrrp{16em}}
  500 & grammes    & d' & os de crosse et gîte,                                                           \\
  500 & grammes    & de & nourrice et tranche,                                                            \\
   15 & grammes    & de & sel gris,                                                                       \\
    2 & litres 1/2 & d' &  eau,                                                                           \\
      &            & 2  & abatis de poulets,                                                              \\
      &            & 1  & carotte moyenne,                                                                \\
      &            & 1  & navet moyen.                                                                    \\
      &            & 1  & blanc de poireau.                                                               \\
\end{longtable}
\normalsize                                

\label{pg0202} \hypertarget{p0202}{}
\index{Bouillon corsé, pour amateurs}
C. — Bouillon corsé, pour amateurs,

\medskip

\footnotesize
\begin{longtable}{rrrp{16em}}
  750 & grammes    & de  & tranche avec os,                                                               \\
  500 & grammes    & de  & crosse et gîte,                                                                \\
   30 & grammes    & de  & sel gris,                                                                      \\
  1/2 & gramme     & de  & poivre,                                                                        \\
    2 & litres 1/2 & d'  & eau,                                                                           \\
      &            &  3  & carottes moyennes,                                                             \\
      &            &  2  & abatis de volaille,                                                            \\
      &            &  1  & navet moyen,                                                                   \\
      &            &     & et en bouquet :                                                                \\
    3 & grammes    & de  & cosses de pois séchées au four,                                                \\
      &            &  2  & blancs de poireaux moyens,                                                     \\
      &            &  1  & petit morceau de panais,                                                       \\
      &            &  1  & petit morceau de célert,                                                       \\
      &            & 1/2 & feuille de laurier,                                                            \\
      &            & 1/2 & racine de persil,                                                              \\
      &            & 1/2 & petite gousse d'ail,                                                           \\
      &            & 1/2 & oignon (facultatif).                                                           \\
\end{longtable}
\normalsize                                

\index{Définition des gelées animales}
La cuisson se fera comme pour le pot-au-feu de famille, mais on la prolongera
quatre heures de plus. Au bout de huit heures, les matières premières seront
épuisées ; le bouillon sera passé et conservé seul. Avec les quantités
indiquées on obtiendra dans les trois cas un litre un quart de bouillon qui,
refroidi, se prendra en gelée\footnote{Les gelées sont le résultat du
refroidissement de bouillons concentrés de substances plus ou moins
gélatineuses. Leur consistance est élastique. Les gelées alimentaires sont
généralement plus ou moins aromatisées et clarifiées ; leur consistance doit
être suffisamment molle pour qu'elles puissent être mangées à la cuiller.}. Le
bouillon A donnera une gelée blonde, le bouillon B une gelée plus foncée, le
bouillon C une gelée brune.

\index{Croûte au pot}
\index{Consommé}
\section*{\centering Consommé\footnote{ On désigne sous le nom de
« croûte au pot » un consommé servi dans un pot en terre vernissée et contenant
des croûtes de flûte séchées au four et des émincés de légumes cuits dans le
consommé. 
\protect\endgraf
On appelle « petite marmite » un consommé servi dans une petite
marmite en terre vernissée, dressée sur un plat garni d'une serviette, et
contenant des émincés de viande (bœuf, queue de bœuf, volaille) et des légumes
cuits dans le consommé. On l'accompagne de petites tranches de pain grillé
garnies de lames de moelle de bœuf très chaude, salée et poivrée, ou de petites
rondelles de flûte, séchées au four, et de fromage de Gruyère râpé.} et bouilli
parfait.}

\addcontentsline{toc}{section}{Consommé et bouilli parfait.}
\index{Bœuf bouilli}
\index{Consommé et bouilli parfait}
Pour obtenir à la fois un consommé et un bouilli parfait, on préparera d'abord
un bouillon corsé à bouilli perdu, puis on mettra dans ce bouillon refroidi un
bon morceau de bœuf, poitrine grasse, plat de côtes, gîte à la noix ou langue,
suivant qu'on aime la viande plus ou moins grasse, ou encore une volaille ; on
fera bouillir, on écumera comme pour un pot-au-feu et on laissera cuire la
viande autant de demi-heures qu'elle pèse de livres.

On servira le consommé, tel que, ou garni, soit de pain, soit de pâtes, de riz
ou de tapioca ; et le bouill à part, sur un plat, entouré de légumes cuits dans du
bouillon.

\section*{\centering Consommé aux œufs pochés et au fromage.}
\addcontentsline{toc}{section}{ Consommé aux œufs pochés et au fromage.}
\index{Consommé aux œufs pochés et au fromage}

\medskip

Pour six personnes, prenez :

\medskip

\index{Fond brun}
\index{Fond de veau brun}

\setlength\tabcolsep{0.1em}
\footnotesize
\begin{longtable}{rrrp{16em}}
  500 & grammes    & d' & os de crosse et gîte,                                                           \\
  200 & grammes    & de & parmesan râpé fin,                                                              \\
   50 & grammes    & d' & excellente glace de viande\footnote{La glace de viande est un produit 
                                                      sirupeux ou demi-solide résultant de la 
                                                      concentration du fond brun.  
                                                     \protect\endgraf
                                                     \label{pg0203} \hypertarget{p0203}{}
                                                     Le fond brun classique, un peu long à préparer, 
                                                     s'obtient de la façon suivante.          
                                                     \protect\endgraf
                                                      
                                                     \protect\endgraf
                                                     Pour faire 2 litres de fond, prenez :    
                                                     \protect\endgraf
              \setlength\tabcolsep{.1em}                                                      
              \begin{tabular}{rrrrl}
              \hspace{8em}  &           &         &    &                                      \\ 
              \hspace{8em}  &     1 200 & grammes & de & gîte de bœuf,                        \\ 
              \hspace{8em}  &     1 200 & grammes & de & jarret de veau,                      \\ 
              \hspace{8em}  &       200 & grammes & d' & os de porc,                          \\ 
              \hspace{8em}  &       125 & grammes & de & couenne maigre,                      \\ 
              \hspace{8em}  &       125 & grammes & de & carottes,                            \\ 
              \hspace{8em}  &       100 & grammes & d' & oignons,                             \\ 
              \hspace{8em}  &        15 & grammes & de & persil,                              \\ 
              \hspace{8em}  &         6 & grammes & de & sel,                                 \\ 
              \hspace{8em}  & \multicolumn{2}{r}{1 gramme 1/2} & de & thym                    \\ 
              \hspace{8em}  & \multicolumn{2}{r}{1/2 gramme}   & de & laurier                 \\ 
              \hspace{8em}  & \multicolumn{2}{r}{1 décigramme} & d' & ail                     \\ 
              \hspace{8em}  &           &         &    &                                      \\ 
              \end{tabular}
                                                     \protect\endgraf
                                                    Désossez les viandes, cassez tous les os en petits 
                                                    morceaux ; faites-les brunir plus ou moins dans un 
                                                    peu de graisse ou au four, puis faites revenir les
                                                    légumes ; mouillez avec deux litres d'eau ; ajoutez 
                                                    le sel et laissez cuire doucement pendant 12 heures 
                                                    en maintenant toujours la même quantité de liquide
                                                    par des additions successives d'eau bouillante. 
                                                    Passez ce fond ; dégraissez-le. Faites revenir les 
                                                    viandes et la couenne coupées en petits morceaux ; 
                                                    enlevez la graisse ; mouillez avec une partie du 
                                                    fond ci-dessus ; laissez tomber à glace deux ou 
                                                    trois fois en mouillant chaque fois avec du fond. 
                                                    Déglacez une dernière fois avec Le reste du fond ; 
                                                    amenez à ébullition ; écumez, dégraissez, passez 
                                                    et réservez le liquide qui doit être très limpide.
                                                    \protect\endgraf
                                                    On préparera dans le même esprit le fond de veau 
                                                    brun.}                                                \\
      & 1 litre 1/4 & de & très bon consommé,                                                             \\
      &             & 12 & croûtons,                                                                      \\
      &             &  6 & œufs frais,                                                                    \\
      &             &    & beurre.                                                                        \\
\end{longtable}
\normalsize

Faites bouillir le consommé avec la glace de viande.

Beurrez les croûtons et faites-les dorer au four.

Mettez dans une soupière les croûtons de pain au sortir du four.

Cassez les œufs dans le consommé bouillant. Aussitôt qu'ils seront pris, versez
le tout avec précaution dans la soupière et servez, en envoyant en même temps
le parmesan dans un ravier.

Ce potage, très remontant et très nourrissant, ne saurait être trop recommandé
aux surmenés de tout genre.

\sk

On peut aussi servir ce potage dans autant de petites soupières qu'il ya de
convives. Les œufs seront cassés chacun dans une soupière, le consommé y étant
maintenu bouillant.

\section*{\centering Consommés froids.}
\addcontentsline{toc}{section}{ Consommés froids.}
\index{Consommés froids}

Les consommés froids peuvent être servis comme rafraîchissement dans les
réunions mondaines, ou comme potage dans les dîners, en été, et dans les
soupers. Ils sont toujours clarifiés et ils n'admettent jamais l'addition
d'élément solide. Ils doivent être très corsés et avoir une onctuosité analogue
à celle d'un consommé chaud au tapioca : plus clairs, ils sembleraient sans
consistance ; plus épais, ils seraient moins agréables. Ils peuvent être
aromatisés avec des éléments destinés à leur donner une tonalité spéciale.
Parmi ces éléments, je citerai les fumets de gibier ; les champignons, les
cèpes, les morilles et les truffes ; les tomates, le piment doux, le céleri,
l’estragon, etc. ; la fine champagne ; des liqueurs ; des vins : chypre,
madère, malvoisie, marsala, porto, samos, zucco, etc. Mais il est indispensable
que l’arome choisi soit employé avec discrétion : 150 grammes de champignons,
cèpes, morilles ou truffes, 200 grammes de tomates, 100 grammes de céleri, 15
grammes de piment doux, 75 centigrammes d'estragon, 70 à 80 grammes de vin, 30
à 40 grammes de fine champagne sont suffisants pour un litre de consommé. En ce
qui concerne les liqueurs et les fumets de gibier, il faudra n'en mettre que
par petites quantités et goûter après chaque addition, car il est impossible de
donner des proportions, tout dépendant, pour les liqueurs de leur degré
d'alcool et de leur parfum, pour les fumets de gibier de la nature du gibier
employé et de son état de fraîcheur.

\sk

Voici maintenant un exemple concret de préparation d'un consommé froid aux
tomates et au porto.                                     
\index{Consommé froid aux tomates et au porto}

Pour six personnes, prenez :

\footnotesize
\begin{longtable}{rrrp{18em}}
 1 200 & grammes  & de & gîte à la noix, de tranche ou de poitrine épaisse sans graisse ni os,            \\
 1 000 & grammes  & de & plat de côtes épais et couvert,                                                  \\
 1 000 & grammes  & de & carottes,                                                                        \\
   300 & grammes  & de & tranche hachée,                                                                  \\
   300 & grammes  & de & tomates,                                                                         \\
   250 & grammes  & d' & os de crosse,                                                                    \\
   110 & grammes  & de & porto,                                                                           \\
   100 & grammes  & de & navets,                                                                          \\
    75 & grammes  & de & blanc de poireaux,                                                               \\
    30 & grammes  & de & sel gris,                                                                        \\
   1/2 & gramme   & de & poivre,                                                                          \\
     4 & litres   & d' & eau.                                                                             \\
       &          & 3  & abatis de volaille,                                                              \\
       &          & 2  & blancs d'œufs,                                                                   \\
\setlength\tabcolsep{.15em}
       &          &    &  $\left.                   
                               \begin{tabular}{lll} 
                                panais,               \\
                                céleri,               \\ 
                                laurier,              \\ 
                                thym,                 \\ 
                                clous de girofle,     \\ 
                                ail.                  \\ 
                               \end{tabular}        
                             \right\} $ Plus ou moins, au goût.
\end{longtable}
\normalsize 

Avec tous les éléments indiqués ci-dessus, excepté la tranche hachée, le porto
et les blancs d'œufs, préparez un bouillon à bouilli perdu, que vous ferez
cuire le temps nécessaire pour obtenir 2 litres de bouillon. Pendant
l'opération, retirez les tomates avant qu'elles se soient défaites : laissez
refroidir ; dégraissez.

Mettez le bouillon dégraissé dans une casserole, ajoutez la tranche hachée et
les blancs d'œufs légèrement battus, amenez à ébullition, puis laissez cuire de
manière à réduire le liquide à 1 500 grammes environ. Passez le consommé à la
mousseline, ajoutez le porto et mettez au frais.

\sk

\index{Consommé de volaille froid, à la fine champagne et à l'essence de truffes}
\index{Consommé de volaille}
Comme consommé froid original, j'indiquerai un consommé de volaille\footnote{On
obtient un excellent consommé de volaille en faisant cuire, à bouilli perdu,
des abatis de poulardes dans un bouillon blanc de veau.} très corsé, clarifié
comme il convient, aromatisé avec de la fine champagne et de l'essence de
truffes et parsemé de petites paillettes d’or, semblables à celles que l'on
trouve dans l'eau-de-vie de Dantzig (aujourd'hui Gdansk) et qu'on obtiendra en
découpant une feuille d'or battu.

Ce consommé, servi dans des coupes de cristal, charme à la fois le goût par sa
finesse, l'odorat par son arome et la vue par le scintillement des paillettes
de métal dans le liquide transparent.

\section*{\centering Potage à la queue de bœuf.}
\addcontentsline{toc}{section}{ Potage à la queue de bœuf.}
\index{Potage à la queue de bœuf}

Pour six personnes, prenez :

\medskip

1° pour le potage :

\footnotesize
\begin{longtable}{rrrp{16em}}
 1 200 & grammes  & de & queue de bœuf.                                                                   \\
 1 000 & grammes  & de & carottes,                                                                        \\
   600 & grammes  & de & tranche maigre hachée,                                                           \\
   200 & grammes  & de & navets,                                                                          \\
   200 & grammes  & de & vin blanc,                                                                       \\
   100 & grammes  & de & céleri,                                                                          \\
   100 & grammes  & de & madère, porto ou malvoisie,                                                      \\
    15 & grammes  & de & sel gris,                                                                        \\
     4 & litres   & d' & eau,                                                                             \\
       &          &  3 & poireaux (le blanc seulement),                                                   \\
       &          &  2 & blancs d'œufs,                                                                   \\
       &          &  2 & clous de girofle,                                                                \\
       &          &  1 & bel abatis de poularde,                                                          \\
       &          &  1 & petit oignon, ou une demi-gousse d'ail, au goût,                                 \\
       &          &  1 & bouquet garni,                                                                   \\
       &          &    & sel blanc, poivre, cayenne.                                                      \\
\end{longtable}
\normalsize    

2° pour la garniture :

\footnotesize
\begin{longtable}{rrrp{16em}}
   200 & grammes  & de & petites carottes nouvelles,                                                      \\
   150 & grammes  & de & petits navets nouveaux,                                                          \\
   100 & grammes  & de & haricots verts.                                                                  \\
\end{longtable}
\normalsize 
                
Tronçonnez la queue aux articulations des vertèbres, puis faites-la dégorger
pendant plusieurs heures dans de l’eau froide : essuyez-la.

Mettez dans une casserole d'abord les morceaux de queue, au-dessus l'abatis,
puis les carottes, les navets et le céleri coupé en rondelles, les poireaux,
les condiments, le bouquet garni et 100 grammes de vin blanc ; faites réduire
et laissez tomber à glace. Ajoutez le reste du vin blanc ; faites réduire de
nouveau et laissez tomber à glace une deuxième fois : cela donnera du goût et
de la couleur au potage. Mouillez alors avec l'eau, mettez le sel gris et
portez à l'ébullition ; écumez avant le premier bouillon. Laissez cuire
à liquide frissonnant pendant cinq heures.

Retirez les morceaux de queue en évitant de les briser ; tenez-les au chaud.

Passez le bouillon à la passoire fine.

Mettez dans une casserole le bouillon passé, les blancs d'œufs battus et la tranche
hachée ; remuez pour clarifier ; puis faites cuire pendant une demi-heure environ
et passez au chinois.

Entre temps, épluchez les haricots verts ; coupez-les en morceaux. Pelez les
carottes et les navets nouveaux ; laissez-les entiers. Faites cuire les haricots verts
dans de l’eau salée, les carottes et les navets dans un peu de bouillon.

Un quart d'heure avant de servir, mettez dans le bouillon la queue de bœuf, les
carottes et les navets nouveaux, les haricots verts, le madère, le porto ou le
malvoisie, du sel, du poivre et du cayenne au goût ; chauffez sans laisser
bouillir.

Envoyez dans une soupière.

Ce potage est exquis ; il procède du vieil « Hochepot » français dans lequel le
bouillon était lié au lieu d’être clair.

\sk

Lorsqu'on n'a pas de légumes nouveaux pour faire la garniture on peut les
remplacer par des légumes tournés à la cuiller, aussi tendres que possible.

\sk

On pourra encore, pour corser la garniture, ajouter de la moelle de bœuf. Dans
ce cas, pour six personnes, on prendra 50 centimètres environ d'os à moelle
qu'on entourera d’une peau ou d'une mousseline et qu'on fera cuire pendant une
demi-heure dans de l’eau salée. On coupera la moelle en rondelles que l'on
salera légèrement et on la mettra, très chaude, dans le bouillon au moment de
servir.

\section*{\centering Potage mimosa.}
\addcontentsline{toc}{section}{ Potage mimosa.}
\index{Potage mimosa}

Émincez des haricots verts cuits au préalable dans de l’eau salée.

Passez au travers d'une passoire à gros trous des jaunes d'œufs durs.

Faites chauffer du très bon bouillon, puis, au moment de servir, mettez dedans
les émincés de haricots verts et les petites boules de jaunes d'œufs : ils
figureront le feuillage et la fleur du mimosa.

C'est un potage joli et agréable.

\section*{\centering Potage aux abatis.}
\addcontentsline{toc}{section}{ Potage aux abatis.}
\index{Potage aux abatis}

Pour six personnes prenez :

\footnotesize
\begin{longtable}{rrrp{16em}}
   150 & grammes  & de & carottes,                                                                        \\
    80 & grammes  & de & navets,                                                                          \\
    50 & grammes  & de & glace de viande,                                                                 \\
    25 & grammes  & de & riz,                                                                             \\
    10 & grammes  & de & blanc de poireau,                                                                \\
     5 & grammes  & de & céleri,                                                                          \\
       &          &  2 & litres d' eau,                                                                   \\
       &          &  3 & beaux abatis de poulardes,                                                       \\
       &          &    & sel et poivre.                                                                   \\
\end{longtable}
\normalsize 

Faites cuire les abatis et tous les légumes dans l'eau pendant deux heures,
comme pour faire un pot-au-feu ; passez le bouillon. Un quart d'heure avant de
servir, mettez le riz et la glace de viande, laissez cuire, assaisonnez au goût
et servez.

\section*{\centering Potage à la tortue.}
\addcontentsline{toc}{section}{ Potage à la tortue.}
\index{Potage à la tortue}

Les tortues (Testude) appartiennent au genre reptiles. Suivant leur habitat,
elles sont marines, terrestres, fluviales.

Le potage à la tortue devrait toujours être préparé avec de la tortue de mer
fraîche ; l'espèce la meilleure est la chélonée franche\footnote{Cette tortue,
qui se nourrit de fucus et de zostères, so trouve dans toutes les mers, excepté
dans la Méditerranée. Elle a pour caractéristique une carapace couverte
d'écailles cornées régulières et deux griffes à chaque patte.}, dont la taille
atteint 2 mètres de longueur et le poids 500 kilogrammes ; mais, étant donné
son volume qui rend difficiles les manipulations nécessitées par l'extraction
de sa chair, il est presque impossible de l'exécuter, comme il devrait l'être,
dans la cuisine bourgeoise.

En Angleterre, où ce potage (turtle soup) est particulièrement estimé, la
préparation en est monopolisée par des spécialistes qui le livrent à domicile
et qui l'expédient dans le monde entier, en boîtes soudées.

Je conseille aux amateurs désireux de goûter ce potage, l'emploi de la conserve
ainsi faite. Il suffira, pour obtenir un potage à point, d'allonger le contenu
d'une boîte de turtle soup avec un bon consommé de bœuf légèrement gélatineux
(ce qu'on obtient aisément par l'addition d'un pied de bœuf), de compléter
l'assaisonnement au goût et d'aromatiser avec un peu de vieux madère.

\sk

On peut également préparer le potage à la tortue avec de la tortue séchée,
qu'on trouve dans les magasins de comestibles anglais.

Voici une formule de préparation ;

\medskip

Pour huit personnes prenez :

\medskip

\setlength\tabcolsep{.1em}
\footnotesize
\begin{longtable}{rrrrp{16em}}   
  &      \multicolumn{2}{r}{700 grammes} & de & chair de tortue de mer séchée,                            \\
  &      \multicolumn{2}{r}{250 grammes} & de & poireaux,                                                 \\
  &       \multicolumn{2}{r}{80 grammes} & d' & oignons,                                                  \\
  &       \multicolumn{2}{r}{40 grammes} & de & céleri,                                                   \\
  &       \multicolumn{2}{r}{40 grammes} & d' & échalotes,                                                \\
  &       \multicolumn{2}{r}{30 grammes} & de & vieux madère,                                             \\
  &       \multicolumn{2}{r}{12 grammes} & de & poivre en grains,                                         \\
  &       \multicolumn{2}{r}{12 grammes} & de & coriandre,                                                \\
  &       \multicolumn{2}{r}{10 grammes} & d' & ail,                                                      \\
  &        \multicolumn{2}{r}{8 grammes} & de & sel gris,                                                 \\
  &  \multicolumn{2}{r}{10 centigrammes} & de & zeste de citron,                                          \\
  &   \multicolumn{2}{r}{4 centigrammes} & de & marjolaine,                                               \\
  &   \multicolumn{2}{r}{2 centigrammes} & de & basilic,                                                  \\
  &   \multicolumn{2}{r}{2 centigrammes} & de & sauge,                                                    \\
  &   \multicolumn{2}{r}{2 centigrammes} & de & fenouil,                                                  \\
  &   \multicolumn{2}{r}{2 centigrammes} & de & sarriette,                                                \\
  &   \multicolumn{2}{r}{2 centigrammes} & de & thym,                                                     \\
  &   \multicolumn{2}{r}{2 centigrammes} & de & laurier,                                                  \\
  &     &              &  3 & litres d' eau,                                                              \\
  &     &              &  1 & litre de consommé gélatineux de bœuf,                                       \\
  &     &              &    & 1 clou de girofle,                                                          \\
  &     &              &    & cayenne.                                                                    \\
\end{longtable}
\normalsize                             

Faites tremper, pendant 24 heures, la chair de tortue dans de l'eau fraîche.

Mettez dans une marmite les deux litres d'eau, la chair détrempée de tortue, le
sel gris, les poireaux, les oignons, le céleri, les échalotes, l'ail et laissez
cuire comme pour un pot-au-feu. Après deux heures de cuisson, retirez le tiers
de la chair de tortue ; réservez-la. Continuez à faire cuire à bouilli perdu.
Au bout de sept heures de cuisson depuis le début de l'opération, ajoutez le
zeste de citron, le poivre en grains, la coriandre, la marjolaine, le basilic,
la sauge, le fenouil, la sarriette, le thym, le laurier et le girofle, le tout
mis dans un sachet ; laissez cuire encore pendant une heure : vous devez avoir
alors un litre environ de bouillon de tortue. Passez-le à la serviette ;
ajoutez-y le consommé de bœuf ; goûtez ; corsez l'assaisonnement avec un peu de
cayenne et aromatisez avec le madère.

Mettez dans le potage la chair de tortue réservée que vous aurez émincée,
chauffez et servez.

En liant ce potage avec des jaunes d'œufs et de la crème, on aura un potage
crème veloutée de tortue, qu'on pourra garnir avec des émincés d'amandes
grillées ou non.

\sk

On peut aussi, à la rigueur, faute de tortue de mer, préparer un potage avec de
la tortue de terre. Voici comment il convient d'opérer : coupez le cou d'une
tortue de terre, recueillez-en le sang. Ébouillantez l'animal pour faciliter
son extraction de la carapace, qu'on devra briser soit à l'aide d'un marteau et
d'un ciseau à froid, soit au moyen d'un appareil mécanique.

Videz la tortue ; faites dégorger la chair dans de l’eau froide ou mieux dans
du lait ; puis, accommodez-la comme celle de la tortue de mer.

Ce potage pourra être lié avec le sang réservé de la tortue.

\sk

Enfin, dans le même esprit et par le même procédé, on fera un potage
interessant, appelé potage fausse tortue, en remplaçant dans sa préparation. la
tortue par de la tête de veau.

Les garnitures pour ce potage seront, par exemple, des boulettes frites ou des
profiteroles fourrées de cervelle de veau cuite au préalable, des émincés de
langue de veau ou des petits cubes de ris de veau cuits à l'avance.

\section*{\centering Bouillon de légumes.}
\addcontentsline{toc}{section}{ Bouillon de légumes.}
\index{Bouillon de légumes}
\label{pg0211} \hypertarget{p0211}{}

Ce bouillon, chaud ou froid, constitue un excellent breuvage, très hygiénique
pour les personnes qui ont l'estomac délabré. Il peut aussi servir de fond de
cuisson et de fond de sauce pour plats maigres.

\medskip

En voici une formule qui comporte beaucoup de variantes :

\medskip

Pour faire deux litres de bouillon, prenez :

\medskip

\footnotesize
\begin{longtable}{rrrp{16em}}
   250 & grammes  & de & carottes,                                                                        \\
   250 & grammes  & de & pommes de terre,                                                                 \\
   100 & grammes  & de & navets,                                                                          \\
    25 & grammes  & de & blanc de poireaux,                                                               \\
    20 & grammes  & de & haricots blancs secs,                                                            \\
    20 & grammes  & de & pois cassés,                                                                     \\
    20 & grammes  & de & lentilles,                                                                       \\
    20 & grammes  & de & sel gris,                                                                        \\
       & 4 litres & d' & eau.                                                                             \\
\end{longtable}
\normalsize                   

Mettez dans l'eau le sel et les légumes, faites bouillir pendant quatre heures,
de façon à réduire le liquide de moitié.

\sk

On peut n'employer que le bouillon filtré, mais il vaut mieux le passer au
travers d'une passoire demi-fine en écrasant les légumes, de manière à avoir un
bouillon contenant un peu de purée de légumes.

\sk

On peut naturellement lier ce bouillon et le garnir à volonté ; on obtient ainsi
d'excellents potages et soupes maigres.

\section*{\centering Potage aux légumes nouveaux.}
\addcontentsline{toc}{section}{ Potage aux légumes nouveaux.}
\index{Potage aux légumes nouveaux}

Prenez du bon consommé et faites cuire dedans pendant le temps nécessaire des
légumes nouveaux et variés, tels que carottes, navets, pommes de terre émincés
en julienne, chou-fleur, fonds d'artichauts coupés en morceaux, tomates,
haricots, petits pois. pointes d’asperges.

En même temps, faites cuire dans du beurre des champignons émincés, puis au
dernier moment ajoutez-les aux autres légumes. Chauffez le tout ensemble un
instant, versez dans une soupière et servez.

\sk

En liant le potage avec de la crème ; ou avec des jaunes d'œufs délayés dans de
la crème, on aura, dans le premier cas, un potage de légumes nouveaux à la
crème et, dans le second, un potage velouté de légumes nouveaux.

\section*{\centering Potage aux gombos\footnote{Le gombo est le fruit non
complètement développé de l'Hibiscus esculentus, de la famille des Malvacées,
originaire de l'Amérique méridionale.}.}
\addcontentsline{toc}{section}{ Potage aux gombos.}
\index{Potage aux gombos}

\medskip

Pour quatre personnes, prenez :

\medskip

\footnotesize
\begin{longtable}{rrrp{16em}}
   250 & grammes  & de & gombos,                                                                          \\
    50 & grammes  & de & riz,                                                                             \\
       & 1 litre  & de & consommé de volaille,                                                            \\
       &          &  2 & tomates pelées, débarrassées de leurs graines et coupées en dés,                 \\
       &          &    & cayenne.                                                                         \\
\end{longtable}
\normalsize                 

Lavez le riz, faites-le blanchir pendant quelques minutes dans de l'eau
bouillante, égouttez-le.

Mettez le consommé de volaille dans une casserole, amenez à ébullition, ajoutez
les gombos coupés en deux dans le sens de leur grand axe, laissez cuire pendant
dix minutes ; jetez dedans le riz en pluie, continuez la cuisson pendant le
temps nécessaire pour que le riz soit cuit, les grains restant entiers, non
crevés. Mettez alors les tomates et un peu de cayenne au goût, donnez quelques
bouillons et servez.

Dans les colonies, le potage aux gombos est préparé souvent sans addition de
riz. Le riz est cuit séparément à la créole, c'est-à-dire sec, et il est servi
à part en même temps que le potage.

\section*{\centering Potage à l'oignon.}
\addcontentsline{toc}{section}{ Potage à l'oignon.}
\index{Potage à l'oignon}

Pour préparer un potage à l'oignon, on peut employer l'oignon ciselé, plus ou
moins doré, ou non revenu, mouiller avec de l'eau ou un bouillon quelconque
gras ou maigre, ou encore avec du lait, enlever l'oignon ou le laisser dans le
potage.

Si l'on fait un potage clair, on peut le servir tel quel, le lier à l'œuf ou
autrement ; ou encore y ajouter d'autres substances, par exemple de la tomate,
du fromage, etc. On peut également préparer des potages à la purée d'oignon et
aussi des potages crème d'oignon ou crème veloutée d'oignon que l’on garnira
à volonté de croûtons frits ou de croûtons au parmesan.

Il est facile d'imaginer beaucoup de potages à l'oignon différant les uns
des autres.


\section*{\centering Krupnik.}
\addcontentsline{toc}{section}{ Krupnik.}
\index{Krupnik}

Le krupnik est un potage polonais à l'orge perlé.

\medskip
Pour huit personnes prenez :

\medskip

1° pour le potage :

\medskip

\footnotesize
\begin{longtable}{rrrp{16em}}                                                                             \\
   300 & grammes  &  de & jarret de veau,                                                                 \\
    30 & grammes  &  de & sel gris,                                                                       \\
   1/2 & gramme   &  de & poivre,                                                                         \\
       & 2 litres &  d' & eau,                                                                            \\
       &          &   3 & abatis de poulardes,                                                            \\
       &          &   3 & carottes moyennes,                                                              \\
       &          &   2 & blancs de poireaux moyens,                                                      \\
       &          &   1 & navet moyen,                                                                    \\
       &          &   1 & petit morceau de panais,                                                        \\
       &          &   1 & petit morceau de céleri,                                                        \\
       &          & 1/2 & feuille de laurier,                                                             \\
       &          & 1/2 & racine de persil,                                                               \\
       &          & 1/3 & oignon ;                                                                        \\
\end{longtable}
\normalsize                   

\medskip

2° pour la garniture :

\medskip

\footnotesize
\begin{longtable}{rrrp{16em}}   
   240 & grammes  & d’ & orge perlé,                                                                      \\
    60 & grammes  & de & rouge de carottes, coupé en cubes de 5 millimètres de côté,                      \\
    30 & grammes  & de & beurre,                                                                          \\
       &          & 1  & aile de poularde rôtié, émincée en julienne.                                     \\
\end{longtable}
\normalsize                   

\medskip

\index{Bouillon blanc}
\label{pg0214} \hypertarget{p0214}{}
Préparez avec les éléments du paragraphe I un bouillon blanc à bouilli perdu ;
passez-le.

Mettez dans une casserole l'orge avec le beurre, couvrez avec une quantité
suffisante de bouillon et laissez cuire à petit feu pendant deux heures et
demie, en remuant de temps à autre avec une cuiller en bois et en remplaçant le
liquide au fur et à mesure de son évaporation, pour éviter que l'orge brûle.

Faites cuire à part le rouge de carottes dans du même bouillon.

Tenez au chaud les émincés de poularde.

Au moment de servir, réunissez bouillon, orge, rouge de carottes et volaille.

\sk

Comme variante, on peut employer, au lieu de bouillon blanc, un bouillon de
légumes, mais dans ce cas il convient de lier le potage avec des jaunes d'œufs,
ou avec de la crème additionnée ou non de jus de citron.

\section*{\centering Barszcz\footnote{La prononciation figurée du mot est barchtch.}.}
\addcontentsline{toc}{section}{ Barszcz.}
\index{Barszcz}

Le barszcz est le type polonais des potages aigrelets, que l'on trouve dans la
plupart des cuisines slaves.

On le prépare de beaucoup de manières, mais sa saveur aigrelette est toujours
obtenue avec du jus de betteraves aigri. 

Voici une formule de préparation des plus simples.

\medskip

Pour quatre personnes prenez :

\medskip

\footnotesize
\begin{longtable}{rrrp{16em}}                                                      
   600 & grammes  & d' & excellent consommé de viande de boucherie et de volaille,                        \\
   100 & grammes  & de & carottes,                                                                        \\
   100 & grammes  & de & betterave crue,                                                                  \\
   100 & grammes  & de & betterave cuite au four,                                                         \\
    10 & grammes  & d' & oignon,                                                                          \\
    10 & grammes  & de & cèpes secs,                                                                      \\
     5 & grammes  & de & persil,                                                                          \\
     5 & grammes  & de & feuilles de fenouil hachées.                                                     \\
     5 & grammes  & de & sucre en poudre,                                                                 \\
       &          &    & sel, poivre,                                                                     \\
       &          &    & jus de betteraves aigri\footnote{On prépare le jus de betteraves                   
                         aigri avec des betteraves crues ou avec des betteraves cuites.                    
                         Les amateurs préfèrent la première solution. Les betteraves crues                 
                         ou cuites sont pelées, émincées et mises dans un pot de grès avec                 
                         de l’eau chaude. En maintenant le pot dans un endroit tiède, on                   
                         obtient un jus aigri à point au bout d'un nombre de jours qui varie               
                         de 8 à 15, avec la température.                                                   
                         \protect\endgraf
                         Si l'on ajoute dans le pot du pain de seigle émietté, on active                   
                         l'opération ; en procédant avec des betteraves cuites et à chaud,                 
                         on peut arriver au résultat désiré en 48 heures.                                  
                         \protect\endgraf
                         Quand on est très pressé, on emploie parfois, au lieu de jus de                   
                         betteraves aigri, du jus aigre de betteraves que l'on obtient en                  
                         faisant confire pendant quelques heures des tranches de betteraves                
                         cuites dans du vinaigre de vin.}.                                                \\
\end{longtable}
\normalsize  

\sk

Pelez et émincez les carottes et la betterave crue, saupoudrez-les de sucre et
laissez-les dégorger.

Faites bouillir le consommé, mettez dedans carottes et betterave crue dégorgées,
persil, oignon, cèpes, sel et poivre ; donnez un bouillon.

\index{Barszcz à la crème}

Coupez en tranches la betterave cuite. faites-la confire dans du jus de
betteraves fermenté que vous chaufferez sans faire bouillir et ajoutez le tout
à la première préparation, donnez un bouillon, goûtez, complétez
l'assaisonnement s'il est nécessaire, rectifiez l'acidité\footnote{Si l’acidité
est trop forte, on l'atténue avec un peu de crème ; si elle est insuffisante,
on l'augmente avec du jus de betteraves cuites additionné de vinaigre de vin
rouge.} et donnez de la couleur\footnote{La couleur du potage doit être d'un
beau rouge ; si elle est trop pâle, on la fonce en ajoutant du jus naturel de
betteraves cuites.} au potage. Passez, ajoutez le fenouil haché, des œufs
pochés ou des tranches de saucisse et de saucisson cuits à part, dont vous
aurez enlevé la peau, ou encore des ravioli. Servez très chaud.

\sk

Comme variante, on peut mettre dans le bouillon toute espèce de légumes émincés
en julienne : carottes, poireaux, céleri, rave, choux, betteraves, tomates, et
des haricots, mais alors on ne passe pas le potage.

\sk

\index{Barszcz aux émincés de viande}
Comme autre variante, on peut ajouter dans le potage passé ou non des émincés
de viande de boucherie, de porc, de lard fumé, de volaille ou de gibier, que
l’on aura fait cuire dans le bouillon.

\sk

\index{Barszcz à la jardinière de légumes}
Enfin, au lieu d'employer du bouillon de viande, on peut employer du bouillon
de légumes, mais dans ce cas il convient d'ajouter au potage de la crème pour
le lier et lui donner du corps.

\sk

\index{Borchtch au jus de concombres}
\index{Borchtch}
En Russie, on prépare des potages aigrelets appelés borchtch, les uns gras,
plus ou moins analogues aux variantes ci-dessus, les viandes étant servies
à part avec une saucière de crème aigre ; les autres maigres, à base de
bouillon de poissons aromatisé avec du fenouil, de la marjolaine et du girofle ;
on les sert avec des ravioli farcis de chair de poisson hachée.

\sk

On y prépare encore un autre potage aigrelet, dans lequel le jus de betteraves
est remplacé par du jus de concombres saumurés.

\sk

\index{Borge au jus de citron}
\index{Borge}
En Moldavie, il existe aussi des potages, appelés borge, qui appartiennent à la
même classe des potages aigrelets, mais qui diffèrent du barszcz en ce que le
jus de betteraves aigri est remplacé par une décoction aigrie de pâte à pain et
de son, ou par du jus de citron.

Voici quelques indications sur le borge au jus de citron.

\smallskip

Coupez en petits morceaux des choux, des piments doux, des tomates rouges et
vertes, des courges, des betteraves, des oignons, des pommes de terre ; faites
cuire longuement tous ces légumes dans du bouillon et acidulez au goût avec du
jus de citron. Dix minutes avant de servir, mettez dans une soupière une
cuillerée à soupe de crème double par personne, quelques jaunes d'œufs crus,
délayez avec un peu de bouillon, puis versez doucement, en remuant, le reste du
potage dans la soupière.

\section*{\centering Chlodnik.}
\addcontentsline{toc}{section}{ Chlodnik.}
\index{Chlodnik}

Le mot chlodnik signifie littéralement rafraîchissement. Et, en fait, les
chlodniki, qui appartiennent à la cuisine slave, sont plutôt des
rafraîchissements que des potages, mais comme on les sert au commencement du
repas, dans des soupières, je les classe dans les potages, tout en faisant des
réserves.

On peut préparer les chlodniki d'une infinité de manières ; je me bornerai à en
indiquer une très simple, qui permettra d'en concevoir bien d'autres.

Contrairement à mon habitude, je ne donnerai pas de formule dosimétrique, car
les différents éléments composant ce chlodnik y entrent, suivant les goûts,
dans des proportions variables.

Faites cuire dans du beurre une poignée de feuilles d'oseaille et de feuilles
de betterave hachées, ajoutez de la crème, des queues d'écrevisses cuites
suivant le rite, \hyperlink{p0287}{p. \pageref{pg0287}}, des tranches de petits
concombres verts, frais, que vous aurez fait dégorger dans du sel gris, des
œufs durs coupés en quatre, du fenouil et de l'estragon hachés, du jus de
concombre salé ou du jus aigri de betterave, assaisonnez au goût et mettez sur
glace.

Servez froid.

\section*{\centering Czernina.}
\addcontentsline{toc}{section}{ Czernina.}
\index{Czernina}

La czernina (tchernina) ou brouet noir est un potage au sang, d'origine
polonaise.

On la prépare à volonté avec du sang de canard, de poulet, de gibier ou de porc.

Pour faire la czernina, prenez un bon consommé de volaille, faites cuire dedans
de la semoule, du riz ou des pâtes, puis éloignez la casserole du feu,
ajoutez-y du sang frais, en quantité variable, au goût, chauffez sur feu doux
sans laisser bouillir, liez avec une purée de foies de volaille ou de gibier et
servez.

On peut aussi remplacer la semoule, le riz ou les pâtes par des croûtons frits
mis dans le potage au moment de servir.

\section*{\centering Panade.}
\addcontentsline{toc}{section}{ Panade.}
\index{Panade}

La panade est un potage lié, au pain, préparé essentiellement avec de l'eau, du
beurre et du pain qu'on laisse mitonner. On peut la confectionner de plusieurs
façons ; en voici une qui, je crois, satisfera tout le monde.

\medskip

Pour quatre personnes, prenez :

\medskip

\footnotesize
\begin{longtable}{rrrp{16em}}                                                                             \\
   125 & grammes  & de & pain rassis (mie et croûte),                                                     \\
    80 & grammes  & de & beurre,                                                                          \\
    12 & grammes  & de & sel gris,                                                                        \\
       &  1 litre & d' & eau,                                                                             \\
       &          &  3 & jaunes d'œufs frais.                                                             \\
\end{longtable}
\normalsize

Mettez dans une casserole le pain coupé en tranches minces, l’eau et le sel.
Faites cuire à petit feu, sans remuer, pendant trois quarts d'heure ; puis
mélangez avec une cuiller ; ajoutez le beurre coupé en petits morceaux, laissez
fondre, donnez deux ou trois bouillons, liez avec les jaunes d'œufs et servez.

\sk

On peut, comme variantes, ajouter de la crème aux jaunes d'œufs : on aura ainsi
une panade veloutée ; remplacer les jaunes d'œufs par de la crème, ou remplacer
les trois jaunes d'œufs par deux œufs entiers. Dans ce dernier cas, on séparera
les blancs des jaunes, on battra les blancs et on les ajoutera à la préparation
avant de mettre le beurre.

On peut aussi remplacer l’eau par du bouillon.

Ambroise Paré preserivait volontiers à ses malades « panade au pain gratté, avec
bouillon de chapon ».

\section*{\centering Potage aux huîtres.}
\addcontentsline{toc}{section}{ Potage aux huîtres.}
\index{Potage aux huîtres}

Faites un roux, mouillez avec du bon bouillon de
poissons\footnote{\label{pg0218} \hypertarget{p0218}{On prépare le bouillon de poissons} en
faisant cuire, dans de l'eau légèrement salée et aromatisée avec des légumes,
du merlan et du congre par exemple avec des têtes et des arêtes de poissons.}
et du vin du Rhin ; laissez cuire ; réservez.

Prenez six huîtres par convive, retirez-les de leurs coquilles, metttez-les,
avec leur eau, dans une casserole, faites-les rapidement blanchir dans cette
eau ; enlevez-les ; tenez-les au chaud ; puis filtrez le liquide de cuisson et
mettez-le de côté.

\index{Composition de la poudre de quatre épices}
Faites revenir de l'oignon dans du beurre, ajoutez sel, poivre blanc, quatre
épices\footnote{On donne ce nom au fruit pulvérisé du \textit{Myrtus pimenta},
dans lequel on trouve réunis les aromes de la cannelle, de la muscade, du
poivre noir et du girofle ; et aussi à un mélange de ces quatre épices. Voici
encore une autre composition d'une poudre dite des quatre épices : 70 pour 100
de poivre blanc, 17 pour 100 de gingembre, 8,5 pour 100 de muscade, 4,5 pour
100 de girofle.} au goût, passez le beurre ainsi aromatisé.

Réunissez alors le bouillon de poissons réservé, l’eau des huîtres, le beurre
aromatisé, liez avec des jaunes d'œufs, mettez les huîtres ; chauffez.

Au moment de servir, ajoutez un filet de jus de citron et des croûtons dorés
dans du beurre.

\section*{\centering Potage maigre au caviar.}
\addcontentsline{toc}{section}{ Potage maigre au caviar.}
\index{Potage maigre au caviar}

\medskip

Pour six personnes, prenez :

\medskip

\footnotesize
\begin{longtable}{rrrp{16em}}                                                    
  1 500 &  grammes & de  &  bon bouillon de poisson préparé avec du merlan ou du                 
                            congre, par exemple, qu'on fera cuire avec des légumes               
                            dans de l'eau salée,                                                          \\
    150 &  grammes & de  &  caviar frais\footnote{Il est indispensable que le caviar 
                            employé soit frais.},                                                         \\
        &          &  24 &  petits goujons ou 24 white bait,                                              \\
        &          &     &  blancs d'œufs.                                                                \\
\end{longtable}
\normalsize

Mettez dans le bouillon chaud le caviar écrasé et passé au tamis, fouettez,
passez le bouillon ; clarifiez-le avec des blancs d'œufs. Tenez-le au chaud.

Au moment de servir, faites frire les poissons dans un bain d'huile.

Versez le potage dans une soupière, garnissez-le avec les poissons frits et
servez immédiatement.

\section*{\centering Potage de Carême.}
\addcontentsline{toc}{section}{ Potage de Carême.}
\index{Potage de Carême}

Faites tremper des œufs de poisson dans de l'eau, après avoir retiré
l'enveloppe qui les entoure ; ébouillantez-les et divisez la masse en petits
cubes.

Faites cuire dans du beurre additionné de vin blanc des laitances de carpe et
des foies de raie ; passez le tout.

Préparez un bouillon de poissons en faisant cuire pendant quelques heures, dans
de l’eau légèrement salée, du merlan et du congre, par exemple, avec des têtes,
des arêtes de poissons et des légumes ; concentrez-le ; liez-le avec un roux ;
ajoutez le produit passé, les œufs ; donnez un bouillon et servez avec des
croûtons au parmesan, \hyperlink{p0259}{p. \pageref{pg0259}}.

\sk

En prenant comme œufs de poisson du caviar, on aura un potage de carême au
caviar qui sort tout à fait de l'ordinaire et qui pourrait figurer avec le
vol-au-vent de carême de la \hyperlink{p0319}{p. \pageref{pg0319}} dans le menu
d'un repas archiépiscopal.

\section*{\centering Potage aux choux et à la crème.}
\addcontentsline{toc}{section}{ Potage aux choux et à la crème.}
\index{Potage aux choux et à la crème}

\medskip

Pour six personnes, prenez :

\medskip

\footnotesize
\begin{longtable}{rrrp{16em}}                                                    
    800 & grammes  & de & chou coupé en petits morceaux,                                                  \\
    250 & grammes  & de & crème épaisse,                                                                  \\
     50 & grammes  & de & beurre,                                                                         \\
     15 & grammes  & de & farine,                                                                         \\
      1 & gramme   & de & poivre blanc fraîchement moulu,                                                 \\
      1 & gramme   & de & paprika\footnote{Le paprika ou poivre de Hongrie                               
                         est en réalité un piment.},                                                      \\
        & 2 litres & de & consommé,                                                                       \\
        &          &    & vinaigre,                                                                       \\
        &          &    & sel.                                                                            \\
\end{longtable}
\normalsize

Mettez dans le consommé le chou, le poivre et le paprika et faites cuire
pendant une heure et demie.

Cinq minutes avant de servir, maniez le beurre avec la farine, délayez le tout
dans la casserole où cuit le potage. Éloignez la casserole du feu, ajoutez la
crème et plus ou moins de vinaigre\footnote{15 grammes de vinaigre doux de vin
suffisent généralement.}, de façon à aciduler légèrement, goûtez, salez s'il
est nécessaire, puis servez.

\sk

On peut facilement donner à ce potage un aspect plus distingué en le passant au
tamis et en le servant avec des croûtons frits ; on aura alors un potage crème
de chou,

\section*{\centering Potage aux concombres et à la crème.}
\addcontentsline{toc}{section}{ Potage aux concombres et à la crème.}
\index{Potage aux concombres et à la crème}

Faites cuire, dans du bon consommé, jusqu'à ramollissement convenable, des
tranches pelées de concombres saumurés ; éloignez la casserole du feu ; liez
avec de la crème aigrie par du jus de concombres saumurés, goûtez, rectifiez
l'acidité s'il est nécessaire, soit avec du jus de concombres saumurés pour
l'accentuer ou avec de la crème pour l'adoucir, puis servez.

Le potage doit avoir une saveur aigrelette agréable.

\section*{\centering Consommé velouté au parmesan.}
\addcontentsline{toc}{section}{ Consommé velouté au parmesan.}
\index{Consommé velouté au parmesan}

\medskip

Pour huit personnes prenez :

\medskip

\footnotesize
\begin{longtable}{rrrp{16em}}                                                    
    100 & grammes  & de & crème,                                                                          \\
    100 & grammes  & de & parmesan râpé,                                                                  \\
    100 & grammes  & de & vermicelle ou de nouilles,                                                      \\
        & 2 litres & de & consommé,                                                                       \\
        &          &  4 & jaunes d'œufs.                                                                  \\
\end{longtable}
\normalsize

Faites cuire le vermicelle ou les nouilles dans le consommé.

Délayez dans une soupière les jaunes d'œufs avec la crème, versez dessus le
consommé contenant le vermicelle ou les nouilles et servez, en envoyant en même
temps le parmesan dans un ravier.

\section*{\centering Potage velouté de grenouilles, aux perles.}
\addcontentsline{toc}{section}{ Potage velouté de grenouilles, aux perles.}
\index{Potage velouté de grenouilles, aux perles}
\label{pg0221} \hypertarget{p0221}{}

\medskip

Pour six personnes prenez :

\medskip

\footnotesize
\begin{longtable}{rrrp{16em}}                                                    
    100 & grammes     & de & crème,                                                                       \\
    100 & grammes     & de & carottes,                                                                    \\
     30 & grammes     & de & poireau,                                                                     \\
     30 & grammes     & de & navet,                                                                       \\
     30 & grammes     & de & perles du Japon,                                                             \\
     15 & grammes     & de & sel gris,                                                                    \\
     10 & grammes     & de & panais,                                                                      \\
        & 1 litre 1/2 & d' & eau,                                                                         \\
        &             & 36 & grenouilles moyennes, parées,                                                \\
        &             &  2 & jaunes d'œufs,                                                               \\
        &             &  1 & abatis de poularde.                                                          \\
\end{longtable}
\normalsize       

\index{Bouillon de grenouilles}
Préparez un bouillon à bouilli perdu avec les grenouilles, l'abatis, les
légumes, l'eau et le sel, en suivant exactement les prescriptions formulées
dans l’article « Pot-au-feu » ; la cuisson sera complète au bout de trois
heures.

Passez-le ensuite au travers d'une passoire garnie d'un linge ; il doit être jaune
paille, limpide, et d'un goût rappelant celui du bouillon de poulet, avec un petit
fumet spécial.

Ce bouillon peut être servi tel quel, et il se prête à toutes les combinaisons
du bouillon de viande.

Pour préparer le potage velouté de grenouilles, aux perles, faites cuire les
perles pendant vingt minutes dans le bouillon obtenu ; délayez ensuite dans la
soupière la crème et les jaunes d'œufs, puis versez dessus le bouillon.

\sk

On peut également obtenir un autre excellent potage lié en faisant cuire dans
le bouillon de grenouilles, préparé comme il a été dit plus haut, 30 grammes de
tapioca et en montant ensuite ce potage au fouet avec quatre jaunes d'œufs et
60 grammes de beurre coupé en petits morceaux.

\section*{\centering Potage velouté aux petits pois.}
\addcontentsline{toc}{section}{ Potage velouté aux petits pois.}
\index{Potage velouté aux petits pois}

\medskip

Pour six personnes prenez :

\medskip

\footnotesize
\begin{longtable}{rrrrp{16em}}   
  & 500 & grammes     & de & petits pois fraîchement écossés,                                             \\
  & 210 & grammes     & de & beurre,                                                                      \\
  &  80 & grammes     & de & pain,                                                                        \\
  &  75 & grammes     & de & crème,                                                                       \\
  &  80 & grammes     & de & sel gris,                                                                    \\
  &  15 & grammes     & de & persil en bouquet,                                                           \\
  &   5 & grammes     & de & cerfeuil,                                                                    \\
  & \multicolumn{2}{r}{6 décigrammes} & de & poivre fraîchement moulu,                                    \\
  & \multicolumn{2}{r}{1 litre 1/2}   & d' & eau,                                                         \\
  &     &             & 3  & jaunes d'œufs,                                                               \\
  &     &             & 1  & petit oignon.                                                                \\
\end{longtable}
\normalsize

Faites bouillir l'eau, mettez dedans le sel, l’oignon, le bouquet de persil, les
petits pois et laissez cuire jusqu'à ce que les pois soient devenus tendres (une
demi-heure suffit généralement).

Coupez le pain en petits cubes : faites-les dorer dans 75 grammes de beurre.

Hachez le cerfeuil, mettez-le dans une soupière avec les jaunes d'œufs délayés
dans la crème.

Quelques minutes avant de servir, retirez l'oignon, le persil, enlevez une
partie plus ou moins grande des pois, au goût, mettez le reste du beurre, le
poivre, laissez donner deux ou trois bouillons, puis versez dans la soupière,
mélangez bien, ajoutez les croûtons et servez aussitôt.

Ce potage, sans prétention, est réellement excellent.

\sk

On peut préparer de même un potage aux fèves fraîches, mais dans ce cas il est
bon de remplacer le cerfeuil par de la sarriette.

\section*{\centering Potage veloutée aux choux et à l'orge.}
\addcontentsline{toc}{section}{ Potage veloutée aux choux et à l'orge.}
\index{Potage velouté aux choux et à l'orge}

Pour quatre personnes prenez :

\medskip

\footnotesize
\begin{longtable}{rrrp{16em}}                                                    
    150 & grammes     & de & crème,                                                                       \\
     25 & grammes     & de & crème d'orge,                                                                \\
     25 & grammes     & de & glace de viande,                                                             \\
     25 & grammes     & de & beurre,                                                                      \\
      1 & litre 1/2   & d' & eau,                                                                         \\
        &             &  3 & jaunes d'œufs,                                                               \\
        &             &  1 & chou frisé moyen,                                                            \\
        &             &  1 & oignon piqué de 5 clous de girolle,                                          \\
        &             &  1 & carotte,                                                                     \\
        &             &    & céleri,                                                                      \\
        &             &    & sel gris,                                                                    \\
        &             &    & poivre.                                                                      \\
\end{longtable}
\normalsize                            

Mettez dans une casserole de l'eau et du sel gris, amenez à ébullition ;
ajoutez le chou et la carotte coupés en morceaux, l'oignon, du céleri et faites
cuire pendant une heure et demie ; passez le bouillon. Puis, ajoutez la glace
de viande, la crème d'orge délayée au préalable avec un peu de bouillon et
laissez cuire encore pendant une vingtaine de minutes ; liez ensuite avec les
jaunes d'œufs et la crème, ajoutez le beurre, mélangez, goûtez et complétez
l'assaisonnement avec un peu de poivre.

Servez simplement tel que ou avec des croûtons frits.

\section*{\centering Potage velouté aux pâtes.}
\addcontentsline{toc}{section}{ Potage velouté aux pâtes.}
\index{Potage velouté aux pâtes}

Pour quatre personnes prenez :

\medskip

\footnotesize
\begin{longtable}{rrrp{16em}}                                                    
    250 & grammes     & de & crème,                                                                       \\
    125 & grammes     & de & parmesan râpé,                                                               \\
    100 & grammes     & de & pâtes et de préférence des nouilles ou des lazagnes                          
                             préparées comme il est dit p. \hyperlink{p0680}{\pageref{pg0680}},           \\
     60 & grammes     & de & beurre,                                                                      \\
     15 & grammes     & de & farine,                                                                      \\
        & 1 litre     & de & consommé,                                                                    \\
        &             & 2  & jaunes d'œufs frais,                                                         \\
        &             &    & sel et poivre.                                                               \\
\end{longtable}
\normalsize
                                                 
Faites cuire Les pâtes dans le consommé.

Préparez un roux avec le beurre et la farine, ajoutez 5 grammes de parmesan
râpé, laissez cuire pendant deux minutes, en tournant avec une cuiller en bois,
mouillez avec le consommé contenant les pates ; liez avec les jaunes d'œufs
délayés dans la crème, chauffez, goûtez et complétez l'assaisonnement avec sel,
poivre et servez.

Envoyez en même temps, dans un ravier, le reste du parmesan râpé.

\section*{\centering Potage velouté à la semoule.}
\addcontentsline{toc}{section}{ Potage velouté à la semoule.}
\index{Potage velouté à la semoule}

Pour quatre personnes prenez :

\medskip

\footnotesize
\begin{longtable}{rrrp{16em}}                                                    
     65 & grammes     & de & crème,                                                                       \\
     65 & grammes     & de & beurre,                                                                      \\
     45 & grammes     & de & semoule,                                                                     \\
        & 1 litre     & de & consommé de volaille,                                                        \\
        &             &  1 & jaune d'œuf frais,                                                           \\
        &             &    & le jus d’un citron.                                                          \\
\end{longtable}
\normalsize

Faites bouillir le consommé, jetez dedans la semoule et laissez-la cuire
pendant une dizaine de minutes. 

Mettez dans une soupière le jaune d'œuf délayé dans un peu de consommé, la
crème, le beurre coupé en petits morceaux et le jus de citron ; versez dessus
le consommé contenant la semoule, remuez pour homogénéiser le tout, goûtez,
ajoutez du sel s'il est nécessaire et servez.

Ce potage est très léger.

\sk

Comme variante, on peut faire griller la semoule dans le beurre avant de la
mettre dans le consommé. Dans ce cas, on supprimera le jus de citron et on fera
cuire le potage pendant une vingtaine de minutes.

\sk

On peut faire de façons identiques des potages veloutés au gruau de sarrasin.

\newpage
\section*{\centering Potage purée de congre\footnote{Congre noir ou murène.
« Conger niger », famille des Murénidés.}.}

\addcontentsline{toc}{section}{ Potage purée de congre.}
\index{Potage purée de congre}

Pour six personnes prenez :

\medskip

\footnotesize
\begin{longtable}{rrrrp{16em}}   
  & \multicolumn{2}{r}{3 kilogrammes} & de & congre,                                                      \\
  & 250 & grammes  & de & crevettes grises, cuites dans un court-bouillon analogue                        
                          à celui qui est indiqué \hyperlink{p0287-2}{p. \pageref{pg0287-2}},             \\
  & 250 & grammes  & de & tomates,                                                                        \\
  & 125 & grammes  & de & beurre,                                                                         \\
  &  75 & grammes  & de & gruyère râpé,                                                                   \\
  &  50 & grammes  & de & parmesan râpé,                                                                  \\
  &  25 & grammes  & de & fumet de poisson,                                                               \\
  &   3 & grammes  & de & poudre de curry\footnote{La poudre de curry se compose de :\protect\endgraf 
              \begin{tabular}{rrrrl}                                                          \\
              \setlength\tabcolsep{.1em} 
              \hspace{4em}           50 & pour & 100  & de & piment pulvérisé                 \\
              \hspace{4em}           36 & pour & 100  & de & poudre de racine de curcuma      \\
              \hspace{4em}            6 & pour & 100  & de & clous de girofle en poudre,      \\ 
              \hspace{4em}            6 & pour & 100  & de & poivre blanc en poudre,          \\ 
              \hspace{4em}            2 & pour & 100  & de & muscade en poudre.               \\ 
              \hspace{4em}   \hrulefill &      &      &    &                                  \\ 
              \hspace{4em}          100 & pour & 100. &    &                                  \\ 
              \end{tabular}},                                                                             \\ 
                                                    \index{Composition de la poudre de curry}
                                                    \index{Curry}
  & \multicolumn{2}{r}{1/2 gramme} & de & safran,                                                         \\
  & \multicolumn{2}{r}{3 litres}   & d' & eau,                                                            \\
  &     &          & 3  & pommes de terre,                                                                \\
  &     &          & 3  & oignons,                                                                        \\
  &     &          & 1  & petite carotte,                                                                 \\
  &     &          & 1  & feuille de laurier,                                                             \\
  &     &          & 1  & brindille de thym,                                                              \\
  &     &          &    & persil,                                                                         \\
  &     &          &    & sel.                                                                            \\
\end{longtable}
\normalsize                                                                                           
\normalsize

Coupez le congre en tronçons de 4 à 5 centimètres de longueur, fendez la tête
en quatre et mettez le tout dans une marmite avec l'eau, la carotte, un oignon,
le thym, le laurier, le safran, du persil et du sel.

Faites bouillir pendant une heure un quart, puis retirez le poisson, pilez-le
au mortier et passez-le au travers d’une passoire, après avoir enlevé les
arêtes.

Épluchez les crevettes, réservez les queues et faites bouillir les parures
pendant quelques minutes dans le bouillon de poisson, que vous passerez
ensuite. Dans ce même bouillon mettez à cuire les pommes de terre, puis
écrasez-les et passez-les au travers d'une passoire.

Entre temps, faites cuire à part les tomates dans leur jus avec un oignon et du
persil ; passez-les.

Enfin, faites roussir un oignon ciselé dans 50 grammes de beurre, ajoutez-y le
fumet de poisson et le produit de la cuisson des tomates ; réservez l'appareil.

Mélangez alors intimement le bouillon de poisson, la chair de poisson passée,
les pommes de terre en purée, l'appareil réservé et les fromages râpés, relevez
avec le curry, ajoutez 40 grammes de beurre coupé en petits morceaux, égayez
avec les queues de crevette et servez chaud avec des croûtons frits dans le
reste du beurre.

Ce potage est absolument remarquable.

\section*{\centering Potage purée de merlan.}
\addcontentsline{toc}{section}{ Potage purée de merlan.}
\index{Potage purée de merlan}

Pour six personnes prenez :

\medskip

\footnotesize
\begin{longtable}{rrrp{16em}}                                                    
  1 500 & grammes & de  & consommé de poisson, préparé avec des déchets de poisson 
                          cuits avec des légumes dans de l'eau salée,                                     \\
    125 & grammes & de  & champignons,                                                                    \\
    100 & grammes & de  & crème de riz,                                                                   \\
        &         & 36  & petits cubes de pain de mie,                                                    \\
        &         & 1   & beau merlan,                                                                    \\
        &         &     & beurre frais.                                                                   \\
\end{longtable}
\normalsize

Passez le merlan dans du beurre fondu, mettez-le dans un plat et faites-le
cuire au four.

Faites cuire les champignons dans du beurre.

Passez au tamis fin merlan et champignons.

Mettez le consommé de poisson dans une casserole, amenez à ébullition ; jetez
dedans la crème de riz ; laissez cuire ; ajoutez ensuite la purée de merlan et
de champignons,

Versez le potage dans une soupière, garnissez-le avec les petits cubes de pain
de mie que vous aurez fait dorer dans du beurre et servez aussitôt.

\section*{\centering Potage purée de tomates.}
\addcontentsline{toc}{section}{ Potage purée de tomates.}
\index{Potage purée de tomates}

Pour huit personnes prenez :

\medskip

\footnotesize
\begin{longtable}{rrrp{16em}}                                                    
  200 & grammes   & de  & purée de tomates,                                                               \\
      &  2 litres & de  & consommé.                                                                       \\
\end{longtable}
\normalsize

Délayez la purée de tomates dans le consommé ; faites cuire, puis servez le
potage soit avec des croûtons frits, soit avec du riz cuit au préalable dans le
consommé,

\sk

En ajoutant de la crème fraîche ou acidulée par du jus de citron, on aura un
potage crème de tomates ; et, parachevant la liaison de ce dernier avec des
jaunes d'œufs, on aura un potage crème veloutée de tomates.

\section*{\centering Potage purée de carottes et de tomates.}
\addcontentsline{toc}{section}{ Potage purée de carottes et de tomates.}
\index{Potage purée de carottes et de tomates}

Pour huit personnes prenez :

\medskip

\footnotesize
\begin{longtable}{rrrp{16em}}                                                    
    500 & grammes  & de & carottes à la Vichy, préparées comme il est dit 
                          \hyperlink{p0770}{p. \pageref{pg0770}},                                         \\
    250 & grammes  & de & purée de tomates, préparée comme 
                          il est dit \hyperlink{p0768}{p. \pageref{pg0768}},                              \\
     20 & grammes  & de & tapioca,                                                                        \\
        & 2 litres & de & consommé de volaille.                                                           \\
\end{longtable}
\normalsize
                                                                             
Passez au tamis les carottes à la Vichy, mélangez cette purée à la purée de
tomates, délayez le tout dans le consommé bouillant dans lequel vous aurez fait
cuire le tapioca, goûtez, rectifiez l'assaisonnement s'il y a lieu et servez.

\sk

On peut, comme toujours, faire de cette purée une crème ou une crème veloutée
en y ajoutant de la crème ou des jaunes d'œufs délayés dans de la crème. Le
potage crème veloutée de carottes et de tomates est un potage très savoureux.

\section*{\centering Potage purée de pois secs.}
\addcontentsline{toc}{section}{ Potage purée de pois secs.}
\index{Potage purée de pois secs}

Comme tous les potages purées de légumes secs, le potage purée de pois secs
peut être préparé en employant pour le faire cuire soit simplement de l'eau
salée, soit un bouillon quelconque, bouillon de légumes ou bouillon de viande
de boucherie, de porc, de volaille, de gibier ou de poisson.

Les détails de l'opération sont toujours les mêmes.

On fait cuire les pois dans le liquide choisi, puis on les passe en purée, on
nourrit le potage avec du beurre et on le verdit, si on le désire, en
y mélangeant du vert d'épinards, \hyperlink{p0324}{p. \pageref{pg0324}}. (En
nourrissant le potage avec de la crème, on aura un potage crème).

Le potage purée de pois est servi soit avec des croûtons de pain dorés dans du
beurre, c'est la classique purée aux croûtons ; soit avec des petits pois
blanchis au préalable, c'est le potage Saint-Germain ; ou bien encore avec du
riz cuit d'avance.

Lorsqu'on a employé, pour la cuisson des pois, du bouillon de porc que, par
parenthèse, je considère comme le milieu le mieux approprié à cet usage, on
peut servir le potage avec des émincés de viande de porc, notamment des parties
cartilagineuses d'oreilles cuites d'avance dans un jus aromatisé.

\sk

Comme variante, j'indiquerai le potage purée de pois secs, au lard.

\medskip

Pour six personnes prenez :

\medskip

\footnotesize
\begin{longtable}{rrrp{16em}}                                                    
    200 & grammes     & de  & pois secs,                                                                  \\
    200 & grammes     & de  & lard de poitrine fraîchement et légèrement fumé,                            \\
        & 1 litre 1/2 & d'  & eau non salée ou de bouillon de légumes non salé, 
                              suivant que vous préférerez conserver pur ou non 
                              le goût des pois.                                                           \\
\end{longtable}
\normalsize

Mettez dans le liquide choisi le lard coupé en petits morceaux et les pois ;
faites bouillir le temps nécessaire pour les bien cuire, ce qui demande une
heure à une heure et demie ; passez le tout au tamis à l’aide d'un pilon,
verdissez la purée ou ne la verdissez pas, goûtez, complétez l'assaisonnement,
s'il y a lieu, ajoutez la garniture que vous préférez et servez.

\section*{\centering Potage purée de marrons.}
\addcontentsline{toc}{section}{ Potage purée de marrons.}
\index{Potage purée de marrons}

Pour six personnes prenez :

\medskip

\footnotesize
\begin{longtable}{rrrp{16em}}                                                    
    250 & grammes     & de  & pointes d'asperges,                                                         \\
     45 & grammes     & de  & madère,                                                                     \\
        & 2 litres    & de  & consommé,                                                                   \\
        & 1/2 litre   & de  & marrons.                                                                    \\
\end{longtable}
\normalsize

Épluchez les marrons, ébouillantez-les, enlevez-en la peau, puis faites-les
cuire dans du consommé et passez-les en purée au tamis.

Mouillez la purée avec du consommé, ajoutez le vin et dépouillez le potage,
c'est-à-dire faites-le chauffer et enlevez les peaux qui se produisent à la
surface, au fur et à mesure de leur formation.

Faites cuire les pointes d'asperges pendant un quart d'heure dans le reste du
consommé et versez le tout dans le potage au moment de servir.

Le volume total du liquide doit être réduit à un litre et demi environ ; la
durée du dépouillement est de trois quarts d'heure.

Il est généralement inutile d'ajouter du sel ou du poivre : l'assaisonnement du
consommé combiné avec le madère suffit le plus souvent pour atténuer
convenablement la saveur sucrée des marrons.

\smallskip

Les pointes d'asperges donnent à ce potage d'aspect confortable une allure
distinguée.

\section*{\centering Potage purée de perdrix.}
\addcontentsline{toc}{section}{ Potage purée de perdrix.}
\index{Potage purée de perdrix}

Pour six personnes prenez :

\medskip

\footnotesize
\begin{longtable}{rrrp{16em}}                                                    
  1 500 & grammes     & de  & consommé de gibier obtenu en faisant cuire, dans de l'eau 
                              salée, des déchets de gibier à plumes et des légumes,                       \\
    125 & grammes     & de  & champignons,                                                                \\
    100 & grammes     & de  & crème de riz,                                                               \\
        &             &  1  & perdrix,                                                                    \\
        &             &     & beurre.                                                                     \\
\end{longtable}
\normalsize                           

Faites rôtir la perdrix, désossez-la, passez les déchets à la presse ; recueillez le jus.

Faites cuire les champignons épluchés dans du beurre.

Passez en purée, au tamis fin, la chair de la perdrix et les champignons.

Mettez dans une casserole le consommé de gibier, amenez à ébullition, jetez
dedans la crème de riz, laissez cuire, puis ajoutez la purée de perdrix et de
champignons et le jus obtenu à la presse. Versez le tout dans une soupière et
servez aussitôt.

C'est un joli potage pour dîner de chasse.

\sk

On pourra préparer dans le même esprit des potages purée de gibier à poil.

\section*{\centering Potage crème de gibier.}
\addcontentsline{toc}{section}{ Potage crème de gibier.}
\index{Potage crème de gibier}

Le potage crème de gibier peut être préparé avec différentes sortes de gibier ;
en voici une formule avec du lièvre.

\medskip

Pour six personnes prenez :

\medskip

\footnotesize
\begin{longtable}{rrrrp{16em}}   
  & 500 & grammes    & de & jarret de veau,                                                               \\
  & 200 & grammes    & de & vin blanc de Sauternes,                                                       \\
  & 125 & grammes    & de & champignons de couche,                                                        \\
  & 100 & grammes    & de & carottes,                                                                     \\
  & 100 & grammes    & de & crème,                                                                        \\
  &  65 & grammes    & de & beurre,                                                                       \\
  &  50 & grammes    & d’ & oignons,                                                                      \\
  &  15 & grammes    & de & farine,                                                                       \\
  & \multicolumn{2}{r}{1 décigramme}  & de & quatre épices,                                               \\ 
  &     & 1 litre    & d' & eau,                                                                          \\ 
  &     &            & 2  & clous de girofle piqués dans l'un des oignons,                                \\ 
  &     &            & 1  & bouquet garni,                                                                \\ 
  &     &            &    & le devant et le bout des pattes de derrière d'un lièvre,                      \\ 
  &     &            &    & jus de citron,                                                                \\ 
  &     &            &    & sel et poivre.                                                                \\ 
\end{longtable}
\normalsize
                                                                                                        
Mettez les morceaux de lièvre, le jarret de veau, les carottes, les oignons et
le bouquet dans l'eau, salez légèrement, faites bouillir, écumez, continuez la
cuisson de manière à réduire le volume du liquide aux trois cinquièmes environ,
ce qui demande quatre heures en moyenne ; puis, retirez les oignons, les
carottes, le bouquet et la viande ; passez le bouillon ; réservez-le.

Désossez la viande, passez la chair et les carottes au tamis ; réservez.

Faites cuire dans une casserole, avec un peu de beurre, les champignons
épluchés et frottés de jus de citron ; passez-les en purée.

Faites un roux avec le reste du beurre et la farine, mouillez avec le bouillon,
mettez la purée de lièvre et de carottes, la purée de champignons et le vin, du
sel, du poivre, les quatre épices, laissez bouillir pendant un quart d'heure,
liez avec la crème, goûtez, complétez l’assaisonnement s'il est nécessaire et
servez avec des croûtons frits.

\section*{\centering Potage crème de tomates aux nouilles.}
\addcontentsline{toc}{section}{ Potage crème de tomates aux nouilles.}
\index{Potage crème de tomates aux nouilles}

Pour six personnes prenez :

\medskip

\footnotesize
\begin{longtable}{rrrp{16em}}                                                    
    500 & grammes   & de & tomates,                                                                       \\
    125 & grammes   & de & nouilles,                                                                      \\
    125 & grammes   & de & crème,                                                                         \\
    100 & grammes   & de & beurre coupé en petits morceaux,                                               \\
     25 & grammes   & de & parmesan coupé en petites lames,                                               \\
        & 1 litre   & d' & eau,                                                                           \\
        &           &  1 & oignon,                                                                        \\
        &           &  1 & clou de girofle,                                                               \\
        &           &  I & petit bouquet garni,                                                           \\
        &           &    & sel et poivre.                                                                 \\
\end{longtable}
\normalsize                                                                                         

Épluchez les tomates, coupez-les en morceaux et laissez-les fondre dans une
casserole avec le clou de girofle, l'oignon et le bouquet, en chauffant à petit
feu pendant une demi-heure.

Passez-les au tamis ; réservez.

Faites cuire les nouilles pendant un quart d'heure dans de l'eau légèrement
salée, mettez ensuite la purée de tomates, chauffez, puis versez dans une
soupière ; ajoutez alors le beurre, la crème, le parmesan, complétez
l'assaisonnement si c'est nécessaire et servez.

\section*{\centering Potage crème de chou-fleur, aux choux de Bruxelles.}
\addcontentsline{toc}{section}{ Potage crème de chou-fleur, aux choux de Bruxelles.}
\index{Potage crème de chou-fleur, aux choux de Bruxelles}

Pour six personnes prenez :

\medskip

\footnotesize
\begin{longtable}{rrrp{16em}}                                                    
    500 & grammes     & de & chou-fleur épluché,                                                          \\
    200 & grammes     & de & petits choux de Bruxelles épluchés,                                          \\
    150 & grammes     & de & crème,                                                                       \\
     65 & grammes     & de & beurre,                                                                      \\
     10 & grammes     & de & farine,                                                                      \\
        & 1 litre 1/2 & de & consommé,                                                                    \\
        &             &    & sel et poivre.                                                               \\
\end{longtable}
\normalsize
                                              
Ébouillantez le chou-fleur, puis faites-le cuire dans le consommé. Quand il
sera cuit, retirez-le, passez-le au tamis ; réservez séparément la purée et le
consommé. Tenez au chaud.

Faites cuire les choux de Bruxelles dans de l'eau salée, de façon à les
conserver entiers ; retirez-les ; tenez-les au chaud.

Faites blondir la farine dans 50 grammes de beurre, mouillez avec le consommé,
ajoutez la purée de chou-fleur et laissez bouillir pendant un instant.

Mettez dans une soupière la crème et le reste du beurre coupé en petits
morceaux, versez dessus le consommé ; goûtez, salez, poivrez s'il est
nécessaire, ajoutez les choux de Bruxelles et servez.

\section*{\centering Potage crème de choucroute.}
\addcontentsline{toc}{section}{ Potage crème de choucroute.}
\index{Potage crème de choucroute}

Pour six personnes prenez :

\medskip

\footnotesize
\begin{longtable}{rrrp{16em}}                                                    
    500 & grammes   & de  & saucisse de Toulouse,                                                        \\
    250 & grammes   & de  & choucroute lavée et dessalée,                                                \\
    250 & grammes   & de  & pommes de terre épluchées,                                                   \\
    100 & grammes   & de  & lard de poitrine fumé,                                                       \\
     60 & grammes   & de  & crème,                                                                       \\
     25 & grammes   & de  & cèpes secs,                                                                  \\
        & 2 litres  & de  & bouillon,                                                                    \\
        &           &     & sel et poivre.                                                               \\
\end{longtable}
\normalsize

Faites cuire dans le bouillon, pendant trois à quatre heures, la choucroute, le
lard coupé en petits morceaux, les cèpes émincés et les pommes de terre ;
passez le tout au tamis.

Mettez ensuite la saucisse dans le bouillon, laissez-la cuire ; puis
retirez-la, enlevez-en la peau et coupez-la en tranches ; réservez-la.

Ajoutez alors la crème dans le bouillon, goûtez et complétez l'assaisonnement
s'il y a lieu avec sel et poivre.

Versez le potage dans une soupière, mettez dedans les tranches de saucisse et
servez.

\section*{\centering Potage crème de pois, aux petits pois.}
\addcontentsline{toc}{section}{ Potage crème de pois, aux petits pois.}
\index{Potage crème de pois, aux petits pois}

Pour quatre personnes prenez :

\medskip

\footnotesize
\begin{longtable}{rrrp{16em}}                                                    
  1 500 & grammes & de & pois frais en cosses et, de préférence, des pois d'une 
                         espèce farineuse, tels que ceux dits « téléphone ».                              \\
  1 200 & grammes & de & consommé de volaille,                                                            \\
    500 & grammes & de & petits pois fins de Clamart, en cosses,                                          \\
    150 & grammes & de & crème,                                                                           \\
     60 & grammes & de & beurre,                                                                          \\
        &         &    & carottes,                                                                        \\
        &         &    & oignon,                                                                          \\
        &         &    & laitue,                                                                          \\
        &         &    & bouquet garni (persil, cerfeuil, thym),                                          \\
        &         &    & sel, poivre.                                                                     \\
\end{longtable}
\normalsize

Écossez à part chaque espèce de pois.

Mettez dans une casserole 1 000 grammes de consommé, les pois téléphone, des
carottes, l'oignon, la laitue et le bouquet garni ; laissez bien cuire ; puis
retirez les carottes, l'oignon et le bouquet ; passez au tamis pois et laitue.

En même temps, faites cuire dans le reste du consommé les petits pois avec un
peu de carotte. Dès qu'ils seront à point, retirez la carotte.

Réunissez le potage aux petits pois et la purée de pois, ajoutez la crème et le
beurre, mélangez sans laisser bouillir, goûtez, salez et poivrez si c'est
nécessaire et servez dans une soupière.

\sk

A défaut de pois téléphone, on pourra prendre d'autres pois, mais il en faudra
davantage pour obtenir une même quantité de purée, également consistante.

\sk

On pourra préparer de même un potage avec des pois secs qu'on fera cuire avec
carottes, oignon, laitue et bouquet garni ; il en faudra moins que de pois
téléphone pour obtenir une même quantité de purée analogue ; on ajoutera à la
purée obtenue le beurre, la crème et des petits pois de conserve cuits au
naturel et simplement réchauffés au bain-marie. Il est inutile de dire que le
potage aux pois secs sera inférieur comme finesse de goût au potage obtenu avec
des légumes frais.

\sk

D'une façon générale, le potage crème de pois aux petits pois diffère du
classique potage Saint-Germain : 1° par l'addition de carottes, oignon et
bouquet garni à la cuisson des pois ; 2° par l'addition de crème.

\section*{\centering Potage crème de lentilles.}
\addcontentsline{toc}{section}{ Potage crème de lentilles.}
\index{Potage crème de lentilles}

Pour six personnes prenez :

\medskip

\footnotesize
\begin{longtable}{rrrp{16em}}                                                    
    250 & grammes & de & lentilles,                                                                       \\
    125 & grammes & de & crème,                                                                           \\
        & 1 litre & de & bouillon blanc, \hyperlink{p0214}{p. \pageref{pg0214}},                          \\
        &         &  1 & perdrix,                                                                         \\
        &         &    & sel et poivre.                                                                   \\
\end{longtable}
\normalsize

Lavez les lentilles, faites-les cuire dans le bouillon ; passez-les au tamis
à l’aide d'un pilon en bois.

Faites rôtir la perdrix, émincez les filets et tenez-les au chaud. Passez tout le
reste à la presse ; recueillez le jus, ajoutez-le à la purée de lentilles.

Versez le potage dans une soupière, ajoutez la crème, mélangez, goûtez,
complétez l'assaisonnement s'il y a lieu avec sel et poivre, mettez dans le
potage les émincés de perdrix et servez.

\section*{\centering Potage crème de légumes, aux champignons et à la semoule.}
\addcontentsline{toc}{section}{ Potage crème de légumes, aux champignons et à la semoule.}
\index{Potage crème de légumes, aux champignons et à la semoule}

Pour six personnes prenez :

\medskip

\footnotesize
\begin{longtable}{rrrrp{16em}}   
  & 200 & grammes   & de & carottes,                                                                      \\
  & 125 & grammes   & de & champignons de couche,                                                         \\
  & 125 & grammes   & de & crème,                                                                         \\
  & 110 & grammes   & de & beurre,                                                                        \\
  & 100 & grammes   & de & navets,                                                                        \\
  &  25 & grammes   & de & semoule,                                                                       \\
  &   5 & grammes   & de & sel gris,                                                                      \\
  &   2 & grammes   & de & sel blanc,                                                                     \\
  & \multicolumn{2}{r}{2 décigrammes} & de & poivre fraîchement moulu,                                    \\
  &     & 1 litre   & d' & eau,                                                                           \\
  &     & 1/2 litre & de & consommé,                                                                      \\
  &     &           &  1 & bel abatis de poularde,                                                        \\
  &     &           &    & jus de citron.                                                                 \\
\end{longtable}
\normalsize

Faites cuire pendant trois heures l'abatis de poularde, les carottes et les
navets dans l'eau salée avec le sel gris, puis retirez les légumes et les os ;
passez le reste en purée.

Pelez les champignons, émincez-les, passez-les dans du jus de citron,
mettez-les dans une casserole avec 60 grammes de beurre, le sel blanc et le
poivre ; laissez cuire.

Faites cuire la semoule dans le consommé pendant un quart d'heure, à liquide
frissonnant.

Mettez dans une soupière la crème et le reste du beurre coupé en petits
morceaux, mouillez avec un peu de consommé ; fouettez, versez dedans le reste
du consommé bien chaud contenant la semoule, ajoutez la purée d'abatis, les
champignons, mélangez et servez.

Les champignons donnent à ce potage une note originale et procurent une
sensation de fraîcheur qui surprend agréablement.

\sk

Comme variante, on peut remplacer la semoule par du riz, que l'on fera cuire
pendant un quart d'heure dans le consommé en ébullition.

\section*{\centering Potage crème de légumes aux petits pois et aux haricots verts.}
\addcontentsline{toc}{section}{ Potage crème de légumes aux petits pois et aux haricots verts.}
\index{Potage crème de légumes, aux petits pois et aux haricots verts}

Pour six personnes prenez :

\medskip

\footnotesize
\begin{longtable}{rrrp{16em}}                                                    
    500 & grammes     & de & cœur de chou, coupé en morceaux,                                             \\
    125 & grammes     & de & crème,                                                                       \\
    100 & grammes     & de & carottes, coupées en rondelles,                                              \\
    100 & grammes     & de & navets, coupés en rondelles,                                                 \\
     60 & grammes     & de & petits pois fraîchement écossés,                                             \\
     60 & grammes     & de & haricots verts émincés,                                                      \\
     25 & grammes     & de & beurre,                                                                      \\
     15 & grammes     & de & farine,                                                                      \\
      1 & litre 1/2   & de & bouillon,                                                                    \\
        &             &    & vinaigre de vin,                                                             \\
        &             &    & sel et poivre.                                                               \\
\end{longtable}
\normalsize

Échaudez le chou pour lui enlever son âcreté ; égouttez-le.

Dans les deux tiers du bouillon, faites cuire pendant trois heures le chou, les
carottes et les navets, puis passez le tout au tamis.

En même temps, faites cuire à part, pendant une heure, dans le reste du
bouillon, les haricots verts émincés et les petits pois ; ajoutez le tout à la
purée.

Si la consistance du potage ainsi obtenu n'est pas suffisante, épaississez-le
en y mettant le beurre manié avec la farine ; donnez un bouillon.

Mettez dans une soupière la crème acidulée par un filet de vinaigre, versez
dessus le potage, mélangez, goûtez, complétez l'assaisonnement s'il y a lieu et
servez.

\sk

Comme variantes, on peut laisser les légumes entiers ou les émincer en
julienne ; dans ce dernier cas, on les fera cuire moins longtemps.

On pourra aussi modifier les proportions indiquées et faire cuire, en même
temps, dans le bouillon, une ou plusieurs des viandes suivantes : poitrine de
bœuf, jambon, petit salé, coupés en petits cubes, saucisses coupées en
tranches, filets de poulet ou de canard émincés, etc. qui seront servies dans
le potage comme garniture.

\sk

Ces préparations constituent des aliments complets très nourrissants ; elles
figurent dans les cuisines des pays du Nord.

\section*{\centering Potage crème de légumes aux concombres.}
\addcontentsline{toc}{section}{ Potage crème de légumes aux concombres.}
\index{Potage crème de légumes aux concombres}

Pour quatre personnes prenez :

\medskip

\footnotesize
\begin{longtable}{rrrp{16em}}                                                    
    200 & grammes     & de  & carottes,                                                                   \\
    200 & grammes     & de  & navets,                                                                     \\
    125 & grammes     & de  & crème,                                                                      \\
     75 & grammes     & de  & beurre,                                                                     \\
     50 & grammes     & de  & poireaux,                                                                   \\
     20 & grammes     & de  & farine,                                                                     \\
     10 & grammes     & de  & panais,                                                                     \\
     10 & grammes     & de  & céleri,                                                                     \\
        & 1 litre     & d’  & eau,                                                                        \\
        &             & 1   & merlan, pesant 200 grammes environ,                                         \\
        &             & 1   & concombre saumuré, de grosseur moyenne,                                     \\
        &             &     & saumure de concombres décantée,                                             \\
        &             &     & jus de citron.                                                              \\
\end{longtable}
\normalsize

Faites cuire dans l’eau, pendant deux heures, le merlan et les légumes, sauf le
concombre, passez en purée, ajoutez quantité nécessaire de saumure de
concombres décantée et de jus de citron pour obtenir un goût agréablement
acidulé et convenablement salé.

Faites un roux avec 50 grammes de beurre et la farine, ajoutez la purée
précédemment préparée, donnez quelques bouillons ; éloignez la casserole du
feu, mettez la crème, le reste du beurre coupé en petits morceaux, le concombre
pelé et coupé en petits cubes, mélangez et servez.

C'est un potage original, auquel le concombre donne de la fraîcheur.

\section*{\centering Potage à la bisque.}
\addcontentsline{toc}{section}{ Potage à la bisque.}
\index{Bisque}
\index{Potage à la bisque}

Le potage à la bisque est un potage crème veloutée d'écrevisses et de volaille.
Pour faire une bonne bisque, il faut compter au moins six écrevisses par
convive.

On commence par préparer un potage crème veloutée de volaille,
\hyperlink{p0239}{p. \pageref{pg0239}}, avec cette différence qu'au lieu
d'employer du beurre ordinaire pour faire blondir la farine, on emploie un
beurre d'écrevisses fait avec les parures des écrevisses cuites comme il est
dit \hyperlink{p0287}{p. \pageref{pg0287}}. A la fin, on ajoute dans le potage
quelques queues d'écrevisses passées au tamis et les autres entières ou
émincées. Quant à la chair de volaille, on n'en passera dans le potage que la
quantité strictement nécessaire pour arriver à une consistance convenable, si
la chair d’écrevisses ne suffit pas.

Le potage à la bisque demande à être relevé par un peu de poivre de
Cayenne\footnote{Le poivre de Cayenne n'est pas un poivre à proprement parler ;
c'est le fruit pulvérisé du piment enragé, \textit{Capsicum frutescens} de
Linné.}.

On le servira avec une garniture faite de coffres d'écrevisses emplis de purée
truffée de brochet, à la crème.

\section*{\centering Potage crème veloutée de corail d'oursins\footnote{Châtaignes de mer, 
                                                                      « Echinus melo » et 
                                                                      « Echinus esculentus » ; 
                                                                      famille des Échinidés.}.}
\addcontentsline{toc}{section}{ Potage crème veloutée de corail d'oursins.}
\index{Potage crème de corail d'oursins}

Pour huit personnes prenez :

\medskip

\footnotesize
\begin{longtable}{rrrp{16em}}                                                    
    300 & grammes  & de & beurre,                                                                         \\
    250 & grammes  & de & crème,                                                                          \\
    200 & grammes  & de & purée de tomates,                                                               \\
     30 & grammes  & de & semoule,                                                                        \\
     25 & grammes  & d' & oignons,                                                                        \\
     20 & grammes  & de & farine,                                                                         \\
        & 2 litres & de & bouillon de légumes, préparé comme il est dit, 
                          \hyperlink{p0211}{p. \pageref{pg0211}},                                         \\
        &          & 24 & beaux oursins bien pleins,                                                      \\
        &          &  4 & jaunes d'œufs frais,                                                            \\
        &          &    & sel, poivre, muscade.                                                           \\
\end{longtable}
\normalsize
               
Ouvrez les oursins ; mettez de côté le corail ; passez l'eau ; réservez-la.
Faites revenir, sans laisser prendre couleur, les oignons ciselés et le corail
dans un peu de beurre.

Préparez un roux avec 50 grammes de beurre et la farine ; mouillez avec l'eau
des oursins et le bouillon de légumes, moins 100 grammes que vous réserverez ;
ajoutez l'appareil corail d'oursins et oignons ; laissez cuire ensemble
à petits bouillons pendant 25 minutes. Dix minutes avant la fin de la cuisson,
ajoutez la purée de tomates et complétez l'assaisonnement au goût avec sel,
poivre et muscade,

La cuisson achevée, passez.

Entre temps, faites cuire à part la semoule dans le bouillon réservé ;
réunissez le tout au potage d'oursins passé.

Mettez dans une soupière la crème, les jaunes d'œufs, mélangez, ajoutez le reste
du beurre, délayez avec le potage et servez.

Ce potage maigre est très agréable.

\section*{\centering Potage crème veloutée aux fruits de mer.}
\addcontentsline{toc}{section}{ Potage crème veloutée aux fruits de mer.}
\index{Potage crème veloutée aux fruits de mer}

Pour huit personnes prenez :

\medskip

\footnotesize
\begin{longtable}{rrrp{16em}}                                                    
    350 & grammes  & de & beurre,                                                                         \\
    250 & grammes  & de & crevettes grises\footnote{Crangon vulgaris ; famille des Carididés.}, 
                          cuites comme il est dit \hyperlink{p0287-2}{p. \pageref{pg0287}},               \\
    200 & grammes  & de & crème,                                                                          \\
     50 & grammes  & d' & œufs de homard,                                                                 \\
     25 & grammes  & de & farine,                                                                         \\
        & 2 litres & de & bouillon de légumes,                                                            \\
        & 1 litre  & de & moules,                                                                         \\
        &          & 24 & huîtres,                                                                        \\
        &          &  3 & jaunes d'œufs frais,                                                            \\
        &          &    & sel, poivre, cayenne, safran.                                                   \\
\end{longtable}
\normalsize

Épluchez les crevettes ; réservez les parures.

Faites pocher les œufs de homard dans de l'eau bouillante.

\index{Beurre de crustacés}
Passez au tamis d'acier les parures des crevettes, les œufs de homard et 300
grammes de beurre ; vous aurez ainsi un excellent beurre de crustacés.

Mettez les moules dans une casserole et laissez-les s'ouvrir sur feu vif ;
sortez-les des coquilles ; passez leur eau, versez-la dans le bouillon de
légumes. Tenez les moules au chaud.

Faites blanchir les huîtres dans leur eau, que vous passerez ensuite et que
vous ajouterez aussi au bouillon de légumes. Tenez les mollusques au chaud,

Chauffez le bouillon de légumes, assaisonnez au goût avec sel, poivre, cayenne
et safran, et laissez bouillir pendant quelques minutes.

Préparez un roux avec le reste du beurre et la farine ; mouillez avec le
bouillon de légumes.

Mettez dans une soupière le beurre de crustacés, la crème et les jaunes
d'œufs ; délayez avec un peu de potage ; versez ensuite dans la soupière le
reste du potage ; mélangez ; garnissez avec les huîtres, les moules, les queues
de crevettes et servez en envoyant en même temps, mais à part, des profiteroles
fourrées de caviar et de corail d'oursins, \hyperlink{p0258}{p. \pageref{pg0258}}.

\sk

Comme variante, on peut remplacer le bouillon de légumes par du bouillon de
grenouilles ou par du consommé de volaille.

\section*{\centering Potage crème veloutée de merlan\footnote{Gadus merlangus ; famille des Gadidés.} et de morilles.} 

\addcontentsline{toc}{section}{ Potage crème veloutée de merlan et de morilles.}
\index{Potage crème veloutée aux fruits de merlan et de morilles}

Pour quatre personnes prenez :

\medskip

\footnotesize
\begin{longtable}{rrrp{16em}}                                                    
    200 & grammes   & de & morilles fraîches,                                                             \\
    100 & grammes   & de & crème,                                                                         \\
     60 & grammes   & de & beurre,                                                                        \\
     20 & grammes   & de & crème de riz,                                                                  \\
      1 & litre 1/2 & d' & eau,                                                                           \\
        &           &  2 & jaunes d'œufs,                                                                 \\
        &           &  1 & merlan, pesant 500 grammes environ,                                            \\
        &           &  1 & carotte,                                                                       \\
        &           &  1 & oignon,                                                                        \\
        &           &    & persil,                                                                        \\
        &           &    & sel et poivre.                                                                 \\
\end{longtable}
\normalsize

Levez les filets du merlan ; réservez les déchets.

Nettoyez les morilles, hachez-les, ramollissez-les sur le feu dans 20 grammes
de beurre, passez-les au tamis avec les filets du merlan.

Mettez dans l’eau les déchets de poisson réservés, la carotte, l'oignon, du persil,
du sel, du poivre et faites cuire ensemble pendant une heure environ.

Délayez la crème de riz dans un peu de bouillon de cuisson ; versez le tout
dans le reste du bouillon, ajoutez la purée de merlan et de morilles : laissez
mijoter ensemble pendant une dizaine de minutes ; passez le tout au tamis fin ;
liez ensuite avec les jaunes d'œufs et la crème, mettez le reste du beurre,
mélangez, goûtez et complétez l’assaisonnement s'il y a lieu.

Servez, soit avec des croûtons de pain dorés dans du beurre, soit avec des
quenelles de brochet et d'écrevisses, \hyperlink{p0328}{p. \pageref{pg0328}}.

\section*{\centering Potage crème veloutée de volaille.}
\addcontentsline{toc}{section}{ Potage crème veloutée de volaille.}
\index{Potage crème veloutée de volaille}
\label{pg0239} \hypertarget{p0239}{}

Pour huit personnes prenez :

\medskip

\footnotesize
\begin{longtable}{rrrp{16em}}                                                    
    500 & grammes   & de & jarret de veau,                                                                \\
    300 & grammes   & de & beurre,                                                                        \\
    250 & grammes   & de & crème,                                                                         \\
    150 & grammes   & de & farine,                                                                        \\
        & 2 litres  & de & consommé de volaille,                                                          \\
        &           &  3 & jaunes d'œufs frais,                                                           \\
        &           &  2 & poireaux (le blanc seulement),                                                 \\
        &           &  2 & carottes (seulement les parties rouges),                                       \\
        &           &  1 & poulet moyen,                                                                  \\
        &           &    & un peu de panais et de céleri.                                                 \\
                                                                                                          \\
\end{longtable}
\normalsize

Faites blondir la farine dans 150 grammes de beurre, mouillez avec le con
sommé, ajoutez le poulet, le jarret de veau, amenez à ébullition, mettez les
légumes et laissez cuire pendant deux heures environ.

Retirez le poulet, désossez-le, enlevez-en la peau, pilez la chair au mortier
en y incorporant, par petites quantités, 100 grammes de beurre et autant de
crème ; passez la purée au tamis. 

Passez la cuisson à l'étamine, ajoutez la purée de poulet, mélangez et donnez
un bouillon. 

Retirez la casserole du feu, liez avec les jaunes d'œufs, le reste du beurre et
le reste de la crème, goûtez, complétez l’assaisonnement s'il y a lieu et
servez.

Potage crème veloutée de volaille, truffe.

Préparez un potage crème veloutée de volaille, comme il est dit ci-dessus.

Au moment de servir, ajoutez une julienne de truffes euites dans du madère et
émincées aussi fin que possible ; enfin, parfumez le potage avec trois grammes de
liqueur d'absinthe par litre de consommé.

\section*{\centering Potage crème veloutée d'asperges.}
\addcontentsline{toc}{section}{ Potage crème veloutée d'asperges.}
\index{Potage crème veloutée d'asperges}

Pour huit personnes prenez :

\medskip

\footnotesize
\begin{longtable}{rrrp{16em}}                                                    
    150 & grammes   & de & crème,                                                                         \\
        & 2 litres  & de & consommé de volaille,                                                          \\
        &           &  4 & jaunes d'œufs,                                                                 \\
        &           &  2 & bottillons de pointes d'asperges.                                              \\
\end{longtable}
\normalsize

Faites cuire les pointes d'asperges dans de l'eau salée bouillante ; passez en
purée les moins belles ; réservez les autres.

Mélangez à la purée d'asperges les jaunes d'œufs délayés dans la crème et chauffez
sans laisser bouillir. Versez le tout dans une soupière, ajoutez le consommé chaud,
mélangez bien, garnissez le potage avec les pointes d'asperges réservées et servez.

\section*{\centering Potage crème veloutée de cerfeuil bulbeux\footnote{
                                  Le cerfeuil bulbeux, Chœrophyllum bulbosum, est 
                                  une variété de cerfeuil, dont la racine présente 
                                  des renflements ayant plus ou moins la forme de petits 
                                  panais ; il a un goût assez fin, rappelant celui 
                                  de la patate. Originaire de l'Asie, le cerfeuil 
                                  bulbeux n'est guère connu comme produit comestible 
                                  dans l'Europe occidentale que depuis un quart de 
                                  siècle environ.}.}
\addcontentsline{toc}{section}{ Potage crème veloutée de cerfeuil bulbeux.}
\index{Potage crème veloutée de cerfeuil bulbeux}

Pour quatre personnes prenez :

\medskip

\footnotesize
\begin{longtable}{rrrp{16em}}                                                    
    500 & grammes & de & cerfeuil bulbeux,                                                               \\
    100 & grammes & de & crème,                                                                          \\
     30 & grammes & de & lait,                                                                           \\
     25 & grammes & de & beurre,                                                                         \\
     15 & grammes & de & crème de riz\footnote{On désigne sous le nom de crème de riz de la      
                                              farine de riz très fine.},                                 \\
     10 & grammes & de & sel,                                                                            \\
        & 1 litre & d' & eau,                                                                            \\
        &         &  2 & jaunes d'œufs frais.                                                            \\
\end{longtable}
\normalsize
                                            
Épluchez les bulbes et lavez-les.

Faites bouillir l'eau, mettez dedans les bulbes et le sel, laissez cuire en
enlevant l'écume qui vient à la surface, au fur et à mesure de sa formation ;
cela demande de 10 à 15 minutes.

Passez en purée au tamis ; réservez.

Faites cuire la crème de riz dans le lait, ce qui demande le temps de donner
deux bouillons.

Au moment de servir, réunissez la purée de cerfeuil réservée et le lait dans
lequel a cuit la crème de riz, chauffez pendant un instant, liez avec les
jaunes d'œufs battus dans la crème, ajoutez le beurre coupé en petits morceaux,
laissez-le fondre, goûtez pour l’assaisonnement, que vous compléterez si cela
vous paraît utile, et garnissez soit avec des folioles fraîches et entières de
cerfeuil, soit avec des croûtons frits.

\sk

On ferait absolument de même un potage crème veloutée de fonds d'artichauts.

\section*{\centering Potage crème veloutée de topinambours\footnote{
                          Le topinambour (Heliantus tuberosus) est une plante de la 
                          famille des Composées, dont la souche émet des tubercules 
                          charnus comestibles, qui ont un goût rappelant un peu celui 
                          des fonds d'artichauts.}.}

\addcontentsline{toc}{section}{ Potage crème veloutée de topinambours.}
\index{Potage crème veloutée de topinambours}

Pour six personnes prenez :

\medskip

\footnotesize
\begin{longtable}{rrrp{16em}}                                                    
    500 & grammes    & de & topinambours,                                                                 \\
    125 & grammes    & de & crème,                                                                        \\
     50 & grammes    & de & purée de tomates assaisonnée,                                                 \\
     50 & grammes    & de & perles du Japon,                                                              \\
     50 & grammes    & de & beurre,                                                                       \\
      1 & litres 1/2 & de & consommé,                                                                     \\
        &            &  2 & jaunes d'œufs frais,                                                          \\
        &            &    & sel et poivre.                                                                \\
\end{longtable}
\normalsize

Faites cuire les topinambours au four, épluchez-les et passez-les au tamis ;
tenez la purée au chaud.

Faites cuire pendant vingt minutes les perles du Japon dans le consommé,
ajoutez la purée de tomates chauffée au préalable et la purée de topinambours ;
mélangez bien.

Mettez dans une soupière la crème et les jaunes d'œufs délayés dans un peu de
consommé, le beurre coupé en petits morceaux ; versez le potage dessus, mélangez,
goûtez pour l'assaisonnement et servez.

\section*{\centering Potage crème veloutée de céleri aux petits pois.}
\addcontentsline{toc}{section}{ Potage crème veloutée de céleri aux petits pois.}
\index{Potage crème veloutée de céleri aux petits pois}

Faites cuire en même temps : d’une part, du céleri dans du fond de veau,
d'autre part, des petits pois dans du consommé de volaille.

Passez le céleri d’abord au tamis, puis à l’étamine ; délayez la purée obtenue
avec le liquide de cuisson des petits pois ; liez-la avec de la crème et des
jaunes d'œufs ; ajoutez-y un peu de beurre frais que vous laisserez fondre.
Goûtez pour l'assaisonnement, mettez les petits pois et servez.

C'est un potage très reconstituant.

\section*{\centering Potage crème veloutée de soja\footnote{Le soja ou soya hispida, 
                                                    originaire des régions chaudes de 
                                                    l'Asie, désigné encore sous le nom 
                                                    de glycine hispida et appelé vulgairement 
                                                    pois chinois, est une légumineuse de la 
                                                    tribu des Phaséolées. En Chine et au
                                                    Japon on en fait une grande consommation,
                                                    surtout dans la classe pauvre. On prépare 
                                                    notamment avec ses graines des pâtes, des 
                                                    sauces et une sorte de fromage. Introduit 
                                                    depuis peu en Europe, on n'y utilise guère 
                                                    que ses germes.} aux perles.}

\addcontentsline{toc}{section}{ Potage crème veloutée de soja aux perles.}
\index{Potage crème veloutée de soja aux perles}

Pour six personnes prenez :

\medskip

\footnotesize
\begin{longtable}{rrrp{16em}}                                                    
   1 000 & grammes   & de & germes de soja,                                                               \\
     100 & grammes   & de & crème épaisse,                                                                \\
      90 & grammes   & de & beurre,                                                                       \\
      50 & grammes   & de & vin blanc,                                                                    \\
      20 & grammes   & d' & eau,                                                                          \\
      20 & grammes   & de & perles du Japon,                                                              \\
       & 1 litre 1/2 & de & consommé de volaille,                                                         \\
         &           &  3 & jaunes d'œufs frais,                                                          \\
         &           &    & sel, poivre.                                                                  \\
\end{longtable}
\normalsize
                     
Lavez les germes et faites-les cuire, à feu très doux, avec le beurre, le vin
blanc et l’eau pendant une vingtaine de minutes ; passez-les ensuite au tamis.

En même temps, faites cuire les perles dans le consommé, puis ajoutez la purée
de soja.

Mettez dans une soupière la crème et les jaunes d'œufs, mélangez, versez dessus
le consommé, mélangez encore ; goûtez, complétez l'assaisonnement, s'il est
nécessaire, avec sel et poivre, et servez.

Ce potage. très délicat et très agréable, peut fort bien figurer dans le menu
d'un repas soigné.

\sk

On peut préparer de même un potage aux crosnes, aux fonds d'artichauts, aux
topinambours, etc.

\section*{\centering Potage crème veloutée de légumes, au riz.}
\addcontentsline{toc}{section}{ Potage crème veloutée de légumes, au riz.}
\index{Potage crème veloutée de légumes, au riz}

Pour douze personnes prenez :

\medskip

\footnotesize
\begin{longtable}{rrrrp{16em}}   
  &  \multicolumn{2}{r}{ 300 grammes}  & de & carottes,                                                   \\
  &  \multicolumn{2}{r}{ 300 grammes}  & de & navets,                                                     \\
  &  \multicolumn{2}{r}{ 250 grammes}  & de & haricots verts,                                             \\
  &  \multicolumn{2}{r}{ 250 grammes}  & de & petits pois fraîchement écosses,                            \\
  &  \multicolumn{2}{r}{ 250 grammes}  & de & tomates,                                                    \\
  &  \multicolumn{2}{r}{ 250 grammes}  & de & pommes de terre,                                            \\
  &  \multicolumn{2}{r}{ 150 grammes}  & de & crème,                                                      \\
  &  \multicolumn{2}{r}{ 150 grammes}  & de & beurre,                                                     \\
  &  \multicolumn{2}{r}{ 100 grammes}  & de & poireaux,                                                   \\
  &  \multicolumn{2}{r}{ 100 grammes}  & de & pointes d'asperges,                                         \\
  &  \multicolumn{2}{r}{  55 grammes}  & de & sel gris,                                                   \\
  &  \multicolumn{2}{r}{  50 grammes}  & de & riz,                                                        \\
  & \multicolumn{2}{r}{4 décigrammes} & de & poivre fraîchement moulu,                                    \\
  &                       &  4 litres & d' & eau,                                                         \\
  &                       &           &  4 & jaunes d'œufs frais.                                         \\
\end{longtable}
\normalsize                      

Épluchez les légumes et faites-les cuire ensemble dans 3 litres d’eau
assaisonnée avec 40 grammes de sel gris, pendant le temps nécessaire pour les
bien cuire, soit une heure et demie pour des légumes nouveaux ; passez en purée.

Faites cuire le riz pendant un quart d'heure dans un litre d'eau additionnée de
15 grammes de sel gris, égouttez-le, puis mettez-le dans la purée de légumes,
poivrez et achevez la cuisson de l'ensemble pendant une dizaine de minutes.

Enfin, mettez dans une soupière les jaunes d'œufs délayés dans la crème et le
beurre coupé en petits morceaux, versez dessus la purée au rix, homogénéisez le
tout et servez.

C'est un véritable velours, dont l'onctuosité est mise en valeur par les grains
de riz.

\section*{\centering Bouillabaisse\footnote{Ou bouille-abaisse, c'est-à-dire 
                                            bouillon abaissé ou concentré par la 
                                            cuisson.}.}
\addcontentsline{toc}{section}{ Bouillabaisse.}
\index{Bouillabaisse}

La bouillabaisse est une soupe provençale aux poissons de mer et aux crustacés,
qu'on sert accompagnée des éléments employés dans sa préparation. Il en existe
un certain nombre qui se distinguent les unes des autres surtout par leur
composition.

Pour être savoureuses, toutes doivent réunir une grande variété d'espèces
différentes ayant chacune son goût et son parfum propres et se mariant ensemble
de manière à produire un concert harmonieux de sensations gustatives et
olfactives. Cette condition fondamentale ne permet guère de préparer une bonne
bouillabaisse en petit : pour arriver à un résultat vraiment satisfaisant, il
me parait indispensable d'opérer avec cinq ou six livres de poissons au moins.

La qualité de la bouillabaisse dépend également de la fraîcheur et de la
finesse des éléments employés ; c'est pourquoi elle ne sera nulle part aussi
bonne qu'au bord de la mer.

Les bouillabaisses de France peuvent être divisées en trois classes : la
bouillabaisse des côtes de la Méditerranée, la mère de toutes les
bouillabaisses, celle des côtes de l'Atlantique, enfin la bouillabaisse de
l'intérieur, dont le type est la bouillabaisse parisienne.

\sk

Les éléments couramment employés dans la confection de la bouillabaisse
méditerranéenne sont, par ordre alphabétique, les suivants : comme poissons,
l'anguille de mer\footnote{Conger vulgaris ; famille des Murénidés.}, la
baudroie\footnote{Lophius piscatorius et lophius budegassa ; famille des
Lophiidés,}, le congre noir ou murène noire, la daurade, la
gallina\footnote{Trigla Iyra ; famille des Triglidés.}, la
girelle\footnote{Julis vulgaris ; famille des Labridés.}, le
grondin\footnote{Trigla pini ; famille des Triglidés.}, le loup\footnote{Labrax
lupus ; famille des Percidés.}, le merlan, les mulets, la
murène\footnote{Murœna helena : famille des Murénidés.}, les
rascasses\footnote{Scorpœna scrofa et scorpœna porcus ; famille des
Triglidés.}, (essentielles), les rougets, le sar\footnote{Sargus Rondeletii ;
famille des Sparidés.}, le Saint-Pierre\footnote{ou zée forgeron, Zeus faber ;
famille des Scombéridés.}, le turbot et la vive\footnote{Trachinus draco ;
famille des Trachinidés.} ; comme crustacés, les cigales de
mer\footnote{Scyllares ; famille des Palinuridés.}, les crabes, le homard et la
langouste. Quelques-uns de ces éléments sont particulièrement fins sur la côte
algérienne ; aussi mange-t-on souvent en Algérie des bouillabaisses excellentes.

\index{Bouillabaisse méditerranéenne}
Voici une formule concrète de bouillabaisse méditerranéenne.

\medskip

Pour une dizaine de personnes prenez :

\medskip

\footnotesize
\begin{longtable}{rrrrp{16em}}   
  & \multicolumn{2}{r}{5 kilogrammes} & de &  poissons et de crustacés assortis, parmi ceux indiqués 
                              plus haut\footnote{Par exemple : une tranche de baudraie, une tranche 
                              de congre, une petite daurade, une gallina, un loup, une murène, une 
                              rascasse, trois rougets, un Saint-Pierre, dix cigales de mer et une 
                              langouste.}                                                                 \\
  & 400 & grammes     & d' & oignons émincés,                                                             \\
  & 125 & grammes     & d' & huile d'olive,                                                               \\
  & 100 & grammes     & de & foies de poissons, tels que baudroie, merlan, rascasse, etc.                 \\
  &  25 & grammes     & d' & échalotes émincées,                                                          \\
  &  25 & grammes     & de & piments doux,                                                                \\
  &  15 & grammes     & de & persil,                                                                      \\
  &  15 & grammes     & de & fenouil,                                                                     \\
  &   6 & grammes     & d' & ail écrasé,                                                                  \\
  &   2 & grammes     & de & safran en poudre,                                                            \\
  &     & 3 litres    & de & bouillon de poissons,                                                        \\
  &     &             &  2 & belles tomates pelées, épépinées et coupées en tranches,                     \\
  &     &             &  2 & feuilles de laurier,                                                         \\
  &     &             &  2 & clous de giroîle,                                                            \\
  &     &             &  1 & morceau d'écorce d'orange,                                                   \\
  &     &             &    & quelques brindilles de thym,                                                 \\
  &     &             &    & sel et poivre.                                                               \\
\end{longtable}
\normalsize 

Ébarbez les poissons, enlevez-leur les nageoires, puis coupez crustacés et
poissons en tronçons. Mettez à part dans deux plats, d'un côté les éléments
tendres tels que : loup, merlan, rouget, Saint-Pierre, etc. ; de l'autre les
éléments fermes tels que : baudroie, congre, rascasse, vive, crabes, homard,
langouste, etc.

Versez un peu d'huile dans une casserole ; faites revenir légèrement dedans les
oignons, les échalotes, l'ail, les tomates. les piments, le persil, le fenouil,
le thym, le laurier ; ajoutez le reste de l'huile, les clous de girofle,
l'écorce d'orange, le bouillon de poissons, les éléments fermes ; salez,
poivrez et faites bouillir à feu vif. Au bout de cinq minutes d'ébullition,
mettez les éléments tendres et laisser cuire encore pendant cinq minutes.
Relirez alors la casserole du feu, enlevez les tronçons de poissons et de
crustacés ; égouttez-les ; nettoyez-les de manière qu'il ne reste dessus aucun
corps étranger et tenez-les au chaud.

Passez le bouillon au chinois ; ajoutez le safran ; concentrez le liquide au
volume de deux litres ; goûtez, complétez l’assaisonnement, qui doit être
relevé, et liez avec les foies écrasés\footnote{Souvent même on néglige cette opération.}. 

Mettez dans un plat creux des tranches de pain de 1 centimètre et demi
d’épaisseur et versez dessus le consommé de poissons ci-dessus.

Dressez les tronçons de poissons et de crustacés sur un autre plat,
saupoudrez-les de persil blanchi haché et servez le tout ensemble.

\sk

\index{Bouillabaisse des pêcheurs}
Comme variante, il faut signaler la bouillabaisse dite « des pêcheurs », dans
laquelle les aromates ne sont pas revenus et qui ne contient généralement pas
de safran.

\sk

\index{Bouillabaisie de l'Atlantique}
La bouillabaisse de l'Atlantique est préparée avec des éléments choisis parmi
les suivants : comme poissons, l'anguille de mer, la barbue, le congre noir, la
daurade, le grondin, le lieu\footnote{où merlan jaune, Gadus pollachius ;
famille des Gadidés,}, le maquereau, les rougets, le Saint-Pierre, la vieille
de roche\footnote{ou labre, Labrus bergylta ; famille des Labridés.}, la vive ;
comme crustacés et mollusques, l’araignée de mer\footnote{Maja squinado ;
famille des Majidés,}, le homard, la langouste, les moules, les oursins, les
palourdes\footnote{Nom vulgaire donné à plusieurs mollusques comestibles :
bucardes, vénus, donax, tapès, etc.}, les sauterelles de mer\footnote{Nom
vulgaire donné à diverses espèces de crustacés : cythères, gamares, crevettes,
salicoques, pasiphées, crangons, etc.} ou chevrettes, le
tourteau\footnote{Cancer pagurus ; famille des Cancridés,}.

La préparation est la même que celle de la bouillabaisse méditerranéenne,
excepté en ce qui concerne les moules, les palourdes et les oursins. On fait
s'ouvrir à part les moules et les palourdes sur un feu vif ; on en passe l'eau
qu’on ajoute au bouillon, mais on ne met les mollusques qu'au moment de
servir ; autrement ils durciraient, On n'emploie des oursins que le corail
qu'on met dans la soupe tout à fait à la fin, sans le faire cuire.

\index{Bouillabaisse parisienne}
A Paris, les éléments qu'on trouve le plus facilement sont : comme poissons,
l'anguille de mer, la barbue, le colin\footnote{Ou merlan noir, Gadus
carbonarius ; famille des Gadidés.}, le congre noir, la daurade, le grondin, la
limande\footnote{Platessa limanda ; famille des Pleuronectidés.}, le merlan, le
maquereau, la plie\footnote{ Pleuronectes platessa ; famille des
Pleuronectidés.}, la sole, le turbot ; comme crustacés et mollusques, les
crabes, les crevettes, le homard, la langouste, les moules, les oursins, les
palourdes.

La préparation reste sensiblement la même. Cependant, à Paris, on ajoute
souvent du sauternes à la cuisson ; on remplace les piments doux par du poivre
et on monte le consommé au beurre. Quant au poisson, on ne le saupoudre
généralement pas de persil.

\sk

\index{Bouillabaisse aux poissons de rivière et aux écrevisses}
Comme variante, citons la bouillabaisse aux poissons de rivière et aux
écrevisses qui n’est pas désagréable, mais qui ne vaut pas les précédentes.

\sk

\index{Bouillabaisse de morue aux pommes de terre}
Enfin, on appelle aussi bouillabaisses des préparations analogues d'une seule
sorte de poisson que l’on accompagne souvent de légumes ; telle la
bouillabaisse de morue aux pommes de terre. En réalité, ces plats n'ont de la
bouillabaisse que le nom.

\section*{\centering Bourride.}
\addcontentsline{toc}{section}{ Bourride.}
\index{Bourride}

La bourride est une soupe provençale de poissons de mer, tels que baudroie,
congre, loup, merlan, grondin, rate, etc., qui diffère essentiellement de la
bouillabaisse par l'absence de crustacés et par l'adjonction d'une sauce
à l'ail.

Préparez un bouillon de poissons comme pour la bouillabaisse, mais faites-le
plus long.

Confectionnez un \textit{ailloli}\footnote{ A l’origine, l’ailloli, que l’on
désigne aussi sous le nom de pommade à l'ail et de beurre de Provence, était
une sauce à l'huile d'olive et à l'ail pilé, liée avec de la mie de pain et de
la pomme de terre bouillie écrasée ; aujourd’hui, la liaison se fait le plus
souvent au jaune d'œuf cru et l'ailloli moderne est une véritable mayonnaise
à l'ail. La proportion de ce dernier élément varie avec la qualité de l'ail et
le goût des amateurs. En Provence, où l'ail n'est pas très fort et où il est
très prisé, on met facilement plusieurs gousses par personne.}, c'est-à-dire
une mayonnaise à l'ail ; réservez-en une partie et incorporez l'autre au
bouillon de poissons, en faisant la liaison sur le feu, sans laisser bouillir.

Mettez dans un plat creux des tranches de pain, versez dessus le bouillon lié et
servez en envoyant en même temps, dans un autre plat, le poisson bouilli et, dans
une saucière, l'ailloli réservé.

\section*{\centering Soupe à la morue.}
\addcontentsline{toc}{section}{ Soupe à la morue.}
\index{Soupe à la morue}

Pour huit personnes prenez :

\medskip

\footnotesize
\begin{longtable}{rrrrp{16em}}   
  & 1 000 & grammes     & de & filets de morue dessalée,                                                  \\
  &   250 & grammes     & de & vin blanc sec,                                                             \\
  &   125 & grammes     & d' & huile d'olive,                                                             \\
  & \multicolumn{2}{r}{1 litre 1/2}   & de & bouillon de poissons préparé comme 
                                             il est dit \hyperlink{p0218}{p. \pageref{pg0218}}.           \\
  &       &             &  4 & tomates,                                                                   \\
  &       &             &  4 & pommes de terre,                                                           \\
  &       &             &  2 & oignons,                                                                   \\
  &       &             &  2 & gousses d'ail,                                                             \\
  &       &             &  1 & bouquet garni (persil, ciboule. thym et laurier),                          \\
  &       &             &    & persil grossièrement haché,                                                \\
  &       &             &    & poivre fraîchement moulu.                                                  \\
\end{longtable}
\normalsize                        

Mettez dans une marmite en terre 100 grammes d'huile, les oignons et l'ail
émincés, faites cuire sans laisser prendre couleur ; ajoutez les tomates
pelées, épépinées et coupées, les pommes de terre pelées, coupées en tranches
et le bouquet ; mouillez avec le bouillon de poissons et le vin blanc. Laissez
cuire,

Lorsque les pommes de terre seront cuites à moitié, ajoutez la morue coupée en
languettes de la dimension d'un domino et le reste de l'huile ; retirez le
bouquet, saupoudrez de persil haché, donnez quelques bouillons ; goûtez pour
l’assaisonnement, ajoutez un peu de poivre, s'il y a lieu, et versez le potage
dans une soupière garnie de tranches de pain grillées.

C'est à cette soupe que certaines personnes donnent le nom de bouillabaisse de
morue.

\section*{\centering Soupe au gras-double.}
\addcontentsline{toc}{section}{ Soupe au gras-double.}
\index{Soupe au gras-double}

Pour six personnes prenez :

\medskip

\footnotesize
\begin{longtable}{rrrrp{16em}}   
  & 500 & grammes     & de & gras-double bien ébouillanté et émincé en petits filets,                     \\
  & 200 & grammes     & de & carottes épluchées et émincées en julienne,                                  \\
  & 200 & grammes     & de & céleri épluché et émincé en julienne,                                        \\
  & 120 & grammes     & de & beurre,                                                                      \\
  & 120 & grammes     & de & parmesan râpé,                                                               \\
  & 100 & grammes     & d' & oignons ciselés.                                                             \\
  & \multicolumn{2}{r}{4 décigrammes} & de & poivre fraîchement moulu,                                    \\
  &     &  2 litres   & de & bouillon,                                                                    \\
  &     &             & 12 & petites tranches de pain riche grillé.                                       \\
\end{longtable}
\normalsize

Faites dorer dans le beurre carottes, céleri et oignons, mouillez avec le
bouillon, amenez à ébullition, ajoutez le gras-double, puis laissez cuire
doucement pendant deux heures. Goûtez pour l'assaisonnement ; un peu de poivre
suffit généralement.

Versez dans une soupière dans laquelle vous aurez mis des tranches de pain
grillé et servez.

Envoyez en même temps le parmesan dans un ravier.

\section*{\centering Soupe à l’oseille.}
\addcontentsline{toc}{section}{ Soupe à l'oseille.}
\index{Soupe à l'oseille}

Pour six personnnes prenez :

\medskip

\footnotesize
\begin{longtable}{rrrp{16em}}                                                    
    200 & grammes   & d' & oseille vierge épluchée,                                                       \\
    125 & grammes   & de & beurre,                                                                        \\
     75 & grammes   & de & crème,                                                                         \\
     20 & grammes   & de & sel,                                                                           \\
        &   1 litre & de & lait,                                                                          \\
        & 1/2 litre & d' & eau,                                                                           \\
        &           &  3 & œufs frais,                                                                    \\
        &           &    & pain.                                                                          \\
\end{longtable}
\normalsize                                      

Cassez les œufs, séparez les blancs des jaunes ; battez les blancs\footnote{On
peut ne pas employer les blancs, mais dans ce cas il conviendra de faire la
liaison avec quatre jaunes.}, mettez les jaunes dans un bol avec la crème.

Faites fondre à petit feu l'oseille dans le beurre, mouillez avec l'eau bouillante,
ajoutez les blancs d'œufs battus, le sel, laissez cuire pendant un instant.

En même temps, faites bouillir le lait.

Mettez dans une soupière des tranches minces de pain, versez dessus le bouillon
d'oseille bouillant contenant les blancs d'œufs, ajoutez le lait, liez avec les
jaunes d'œufs délayés dans la crème, couvrez, laissez tremper pendant un
instant et servez.

On peut préparer de même des potages crème d'oseille, en remplaçant le pain par
du tapioca, des perles du Japon ou des pâtes,

\sk

En automne et en hiver, alors que l'oseille contient une forte proportion
d'acide oxalique, on n'emploiera pour ces mêmes soupes ou ces mêmes potages que
la moitié de la quantité d'oseille indiquée.

\section*{\centering Soupe aux poireaux et aux pommes de terre.}
\addcontentsline{toc}{section}{ Soupe aux poireaux et aux pommes de terre.}
\index{Soupe aux poireaux et aux pommes de terre}

Pour six à huit personnes prenez :

\medskip

\footnotesize
\begin{longtable}{rrrp{16em}}                                                    
    400 & grammes     & de & pommes de terre épluchées.                                                   \\
    200 & grammes     & de & blanc de poireaux,                                                           \\
    125 & grammes     & de & beurre,                                                                      \\
     30 & grammes     & de & sel,                                                                         \\
      1 & gramme      & de & poivre,                                                                      \\
        & 4 litres    & d’ & eau,                                                                         \\
        &             &    &pain.                                                                         \\
\end{longtable}
\normalsize
                                                       
Coupez les pommes de terre et les poireaux en gros morceaux, mettez-les dans
une casserole avec l'eau, le sel et le poivre ; faites cuire à feu vif pendant
une heure environ, de façon à réduire le liquide de moitié.

Passez alors le liquide au travers d'une passoire à gros trous, en écrasant plus ou
moins les légumes, suivant que vous voulez obtenir une soupe plus ou moins
épaisse. Remettez le liquide passé sur le feu, ajoutez le beurre, donnez deux ou
trois bouillons.

Mettez dans une soupière des tranches minces de pain (100 grammes environ),
versez dessus le bouillon de poireaux et de pommes de terre bouillant. Couvrez la
soupière, laissez tremper pendant un moment, puis servez.

\sk

On peut, suivant le goût, préparer de même une soupe aux poireaux et aux
pommes de terre après avoir fait revenir un peu les poireaux dans du beurre,

\sk

On peut aussi préparer des potages aux poireaux et aux pommes de terre. Dans
ce cas, on remplacera, au goût, le pain par du tapioca, des perles du Japon ou
des pâtes qu'on fera cuire dans le bouillon passé avant d'ajouter le beurre.

\sk

Enfin, en augmentant un peu la quantité de légumes passés au travers de la
passoire, on aura des potages purée de poireaux et de pommes de terre. On
servira ces potages soit tels quels, soit garnis de croûtons frits.

\section*{\centering Soupe aux choux, à la paysanne.}
\addcontentsline{toc}{section}{ Soupe aux choux, à la paysanne.}
\index{Soupe aux choux, à la paysanne}

Il existe un très grand nombre de manières de préparer la soupe aux choux. En
voici une très simple.

\medskip

Pour quatre personnes prenez :

\medskip

\footnotesize
\begin{longtable}{rrrp{16em}}                                                    
    300 & grammes  & de & poitrine de porc,                                                               \\
    250 & grammes  & de & carottes,                                                                       \\
    100 & grammes  & de & navets,                                                                         \\
    100 & grammes  & de & pommes de terre,                                                                \\
     50 & grammes  & de & pain,                                                                           \\
     50 & grammes  & de & beurre,                                                                         \\
     20 & grammes  & de & sel,                                                                            \\
        & 3 litres & d' & eau,                                                                            \\
        &          &  2 & poireaux (le blanc seulement),                                                  \\
        &          &  1 & chou frisé moyen,                                                               \\
        &          &  1 & oignon,                                                                         \\
        &          &    & poivre.                                                                         \\
\end{longtable}
\normalsize

Mettez dans une casserole l'eau, le sel, la poitrine de porc coupée en quatre
morceaux, les poireaux, les carottes, les navets, l'oignon, du poivre et laissez
cuire pendant deux heures environ. Retirez alors les poireaux et les navets ; mettez
le chou et les pommes de terre que vous laisserez cuire suffisamment.

Coupez le pain en tranches ; faites-les griller.

Mettez dans une soupière les tranches de pain, le beurre, versez dessus le
contenu de la casserole, mélangez et servez aussitôt.

\section*{\centering Soupe à l'oignon.}
\addcontentsline{toc}{section}{ Soupe à l'oignon.}
\index{Soupe à l'oignon}

Pour faire la soupe à l'oignon ordinaire, faites revenir doucement de l'oignon
dans du beurre ou dans de la graisse, au goût, de façon à le bien dorer sans
qu'il brûle ; mouillez ensuite soit avec de l’eau, soit avec un bouillon
quelconque maigre ou gras, ou encore avec du lait ; salez, poivrez. Laissez
cuire pendant quelques instants. Passez ou ne passez pas le bouillon.

Mettez dans une soupière des tranches minces de pain, grillées ou non, des
croûtons frits ou de la flûte, versez dessus le bouillon bouillant et servez.
Envoyez en même temps, si vous l'aimez, du fromage de Gruyère râpé, dans un
ravier.

\sk

Comme variante, on pourra masquer les tranches de pain, les croûtons frits ou
la flûte avec une purée d'oignons liée avec une béchamel serrée, par exemple.

\sk

Enfin, on fera des soupes gratinées en ajoutant à des soupes à l'oignon
ordinaires du fromage râpé et en les poussant au four.

\section*{\centering Soupe à l'oignon gratinée.}
\addcontentsline{toc}{section}{ Soupe à l'oignon gratinée.}
\index{Soupe à l'oignon gratinée}

Pour quatre personnes prenez :

\medskip

\footnotesize
\begin{longtable}{rrrrp{16em}}   
  & 200 & grammes     & de & pain de mie,                                                                 \\
  & 150 & grammes     & de & fromage de Gruvère en lames,                                                 \\
  &  80 & grammes     & d' & oignons épluchés et ciselés,                                                 \\
  &  60 & grammes     & de & beurre,                                                                      \\
  &  30 & grammes     & de & graisse de rôti ou de graisse de lard gras,                                  \\
  &  12 & grammes     & de & sel,                                                                         \\
  &   1 & gramme      & de & sucre en poudre,                                                             \\
  & \multicolumn{2}{r}{2 décigrammes} & de & poivre,                                                      \\
  &     & 1 litre 1/2 & d' & eau.                                                                         \\
\end{longtable}
\normalsize
               
Faites blondir dans une casserole, à petit feu, pendantune vingtaine de
minutes, les oignons dans la graisse, saupoudrez-les ensuite avec le sucre de
façon à leur faire prendre couleur sans qu'ils brûlent. Enlevez l'excès de
graisse, mouillez avec l'eau, salez, poivrez et laissez cuire pendant dix
minutes.

Garnissez une soupière, allant au feu, de tartines de pain d'un centimètre
d'épaisseur, couvrez chaque tartine d'une lame de fromage de même surface,
renouvelez l'opération jusqu'à épuisement des tartines, puis versez dessus le
bouillon passé ou non ; le pain montera à la surface du liquide. Mettez alors
sur les tartines des petits morceaux de beurre et faites gratiner doucement au
four pendant une vingtaine de minutes ; puis, servez.

Cette soupe, dont les caractéristiques sont : 1° la manière de faire revenir
l'oignon, 2° la qualité du pain employé, 3° l'usage du fromage en lames, ce
qui assure au pain un enduit crémeux de fromage sans aucun grumeau, est
délicieuse,

\section*{\centering Soupe à l'oignon gratinée.}
\addcontentsline{toc}{section}{ Soupe à l'oignon gratinée.}
\index{Soupe à l'oignon gratinée}
\index{Soupe à l'oignon gratinée (autre formule)}

\medskip

\centering\small\sc(Autre formule)

\bigskip

\justifying
\normalfont
Coupez du pain de ménage en tranches d'un centimètre d'épaisseur ; faites-les
griller, laissez-les refroidir, étendez dessus du beurre frais et saupoudrez
ensuite de fromage d'Emmenthal râpé, l'ensemble des deux couches de beurre et
de fromage devantt atteindre environ un centimètre, dans la proportion d'un
tiers de beurre et de deux tiers de fromage.

Prenez des oignons de grosseur moyenne, à raison de un par convive ; coupez-les
en tranches minces et faites-les revenir à la poêle, dans du beurre.

Disposez alors dans une marmite allant au feu une couche de tartines sur
lesquelles vous épandrez un tiers des oignons roussis, une nouvelle couche de
tartines, le second tiers des oignons, une autre couche de tartines que vous
masquerez de pulpe de tomates, à raison de 20 grammes par convive, continuez
par une dernière couche de tartines que vous couronnerez avec le reste des
oignons et un peu de pulpe de tomates. Enfin, recouvrez le tout d’un manteau
d'emmenthal râpé.

La marmite ne doit pas être garnie à plus des deux tiers de sa hauteur afin de
permettre au pain de gonfler sans qu'une partie du contenu se répande au
dehors.

Prenez un entonnoir de verre, dont vous pousserez le tuyau jusqu'au fond de
la marmite en suivant la paroi, et versez dedans doucement de l'eau chaude salée
(4 grammes de sel par litre), de façon que le liquide monte jusqu'au manteau de
fromage sans l'inonder.

Mettez la marmite sur le feu, mais ne la couvrez pas. Laissez mijoter pendant
une demi-heure, goûtez, ajoutez du sel s'il y a lieu, puis mettez au four et
continuez la cuisson pendant une heure en remplaçant l'eau, toujours par le
dessous, au moyen de l’entonnoir, à mesure qu'elle s'évapore.

La soupe est à point lorsque l'extérieur étant gratiné et présentant l'aspect
d'un gâteau croustillant de couleur brun doré, l'intérieur est onctueux et si
bien fondu et amalgamé quil est impossible d'y discerner ni fromage, ni oignon.

On sert à chaque convive de la croûte gratinée et de l'intérieur, qui doit être
épais, mais non dépourvu de liquide.

\section*{\centering Soupe à l'oignon, au vin blanc.}
\addcontentsline{toc}{section}{ Soupe à l'oignon, au vin blanc.}
\index{Soupe à l'oignon, au vin blanc}

Pour huit personnes prenez :

\medskip

\footnotesize
\begin{longtable}{rrrp{16em}}                                                    
    250 & grammes  & de & beurre,                                                                         \\
    100 & grammes  & de & farine,                                                                         \\
     80 & grammes  & de & glace de viande,                                                                \\
        & 3 litres & d' & eau,                                                                            \\
        & 1 litre  & de & vin blanc,                                                                      \\
        &          &  8 & oignons moyens,                                                                 \\
        &          &    & pain de flûte,                                                                  \\
        &          &    & parmesan râpé,                                                                  \\
        &          &    & sel et poivre.                                                                  \\
\end{longtable}
\normalsize

Émincez les oignons et faites-les frire dans 25 grammes de beurre.

Préparez un roux avec 150 grammes de beurre et la farine, mouillez avec l'eau
et le vin ; ajoutez la glace de viande, les oignons frits, du sel et du
poivre ; donnez quelques bouillons, puis laissez mijoter pendant une demi-heure
en casserole couverte, de façon à réduire le potage au volume de trois litres
environ.

Préparez 24 tranches de pain de flûte ; étendez dessus le reste du beurre,
saupoudrez-les de parmesan, assaisonnez avec du poivre et poussez au four pour
gratiner.

Disposez les tranches de pain dans une soupière et, cinq minutes avant de
servir, versez dessus le potage passé ou non.

Ce potage, légèrement aigrelet, est très recommandable à la campagne, après
une partie de chasse,

\section*{\centering Tourin toulousain.}
\addcontentsline{toc}{section}{ Tourin toulousain.}
\index{Tourin toulousain}

Pour six personnes, faites revenir une dizaine de gousses d'ail dans 50 grammes
de graisse d'oie, mouillez avec de l'eau et laissez bouillir pendant quelques
minutes.

Cassez deux œufs, séparez les jaunes des blancs ; les jaunes serviront pour la
liaison.

Battez les blancs, jetez-les dans le bouillon bouillant : ils se prendront en
caillots. Éloignez la casserole du feu, liez la soupe avec les jaunes d'œufs
préalablement délayés dans un peu de bouillon ; salez, poivrez énergiquement,
puis versez le tout dans une soupière dans laquelle vous aurez mis des tranches
de pain.

Cette soupe ne convient assurément pas pour un dîner de cérémonie ; je ne la
conseillerais même pas pour un dîner intime ; mais, lorsque tous les convives
sont du Midi, elle a un succès énorme ; les Méridionaux en raffolent et
assurent qu'elle embaume !

\section*{\centering Garbure.}
\addcontentsline{toc}{section}{ Garbure.}
\index{Garbure}

On prépare la garbure classique avec des pousses vertes de chou de Bacalan que
l'on obtient au mois de mars ; mais on peut la faire aussi avec des feuilles
tendres d'autres choux.

Lavez et hachez fin le chou.

Passez à la poêle une aile de confit d'oie pour la débarrasser de la graisse
qui l'entoure, puis mettez-la dans une marmite avec de l’eau bouillante.

Placez le chou dans une passoire au-dessus de la marmite, arrosez-le avec la
graisse très chaude provenant du confit, mettez-le ensuite dans la marmite,
salez et poivrez au goût, ajoutez une pointe d'ail et faites bouillir le
liquide à tout petit feu, pendant une heure, en laissant la marmite découverte.

Disposez dans une soupière des tranches minces de pain rassis, étalez le chou
sur le pain, versez du bouillon sans excès et servez chaud.

Il est bon de ne faire que la quantité nécessaire de bouillon pour tremper le
pain, car réchauffé il ne vaut rien.

L'aile d'oie confite est mangée froide.

\section*{\centering Soupe aux légumes à la paysanne.}
\addcontentsline{toc}{section}{ Soupe aux légumes à la paysanne.}
\index{Soupe aux légumes à la paysanne}

Prenez toutes les variétés de légumes que vous aimez et de la poitrine de porc.

Épluchez les légumes, coupez-les en morceaux ou laissez-les entiers, suivant
leurs dimensions.

Si vous mettez du chou, commencez par le blanchir pour lui enlever son âcreté.

Faites revenir à la poêle, dans un mélange de graisse de porc et de graisse
d'oie, d'abord la poitrine de porc coupée en morceaux, ensuite les gros
légumes.

Mettez dans une casserole la quantité d'eau nécessaire, le lard, faites
bouillir, écumez, ajoutez le chou blanchi, un peu de poivre et de sel, puis,
dans l'ordre de la rapidité de leur coction, les autres légumes, de manière
qu'ils soient cuits sans tomber en bouillie. À la fin, goûtez, rectifiez
l'assaisonnement s'il y a lieu avec sel et poivre, mettez ensuite du beurre
frais coupé en petits morceaux, laissez-le simplement fondre, puis versez le
contenu de la casserole dans une soupière contenant des tranches de pain dans
lesquelles la croûte dominera.

Servez.

Gette soupe est d'autant meilleure que les légumes sont plus jeunes et plus
frais. Avec des légumes nouveaux, fraîchement cueillis, elle est délicieuse.

\sk

\newpage
\medskip

\centering\textbf{\large GARNITURES POUR POTAGES}
\index{Garnitures pour potages}

\smallskip

\section*{\centering Boulettes frites.}
\addcontentsline{toc}{section}{ Boulettes frites.}
\index{Boulettes frites}

\justifying\normalfont
Pour six personnes, ramollissez deux flûtes ou une quantité équivalente de pain
riche dans de l'eau froide, puis extrayez par pression dans une serviette tout
l'excès d'eau.

Faites dorer dans une casserole, avec 60 grammes de beurre, une échalote ou un
tout petit oignon haché très fin, ajoutez le pain, du sel, du poivre et un peu
de gingembre pilé. Laissez refroidir incomplètement, ajoutez un œuf cru
entier : vous obliendrez ainsi une pâte ; travaillez-la bien, ajoutez un second
œuf et travaillez encore jusqu'à disparition des grumeaux.

Laissez reposer pendant une heure, puis préparez des boulelles grosses comme
des noix, que vous ferez cuire à pleine friture et que vous mettrez ensuite
dans le potage.

\section*{\centering Boulettes au jambon.}
\addcontentsline{toc}{section}{ Boulettes au jambon.}
\index{Boulettes au jambon pour potages}

Pour dix personnes prenez :

\medskip

\footnotesize
\begin{longtable}{rrrp{16em}}                                                    
    125 & grammes & de & mie de pain blanc ou bis, au goût ;                                              \\
     75 & grammes & de & jambon cru,                                                                      \\
     75 & grammes & de & beurre,                                                                          \\
     75 & grammes & de & lait,                                                                            \\
      5 & grammes & d' & un mélange, en parties égales, de ciboule, cerfeuil, estragon et                  
                         persil hachés,                                                                   \\
        &         &  1 & œuf,                                                                             \\
        &         &    & farine,                                                                          \\
        &         &    & sel et poivre.                                                                   \\
\end{longtable}
\normalsize

Faites bouillir le lait.

Coupez la mie de pain en dés ; faites-les dorer dans 50 grammes de beurre ;
mettez-les dans un vase. Versez dessus le lait bouillant ; laissez tremper ;
puis incorporez-y l'œuf en travaillant bien, de façon à avoir une pâte
homogène.

Faites revenir dans le reste du beurre le jambon, la ciboule, le cerfeuil,
l’estragon et le persil hachés ; amalgamez bien le tout avec la pâte
précédemment obtenue ; assaisonnez au goût avec sel et poivre. 

Préparez avec cette pâte des boulettes de la grosseur d'une noisette,
passez-les dans de la farine et faites-les cuire dans le bouillon dans lequel
vous les servirez.

\sk

Dans la cuisine autrichienne, on sert des boulettes au jambon, comme plat,
\hyperlink{p0541}{p. \pageref{pg0541}}.

\section*{\centering Boulettes de cervelle.}
\addcontentsline{toc}{section}{ Boulettes de cervelle.}
\index{Boulettes de cervelle}

Pour dix à douze personnes prenez :

\medskip

\footnotesize
\begin{longtable}{rrrp{16em}}                                                    
    300 & grammes & de & mie de pain rassis, tamisée,                                                     \\
    125 & grammes & de & champignons,                                                                     \\
    100 & grammes & de & beurre,                                                                          \\
      8 & grammes & de & persil,                                                                          \\
        &         &  5 & œufs frais,                                                                      \\
        &         &  1 & cervelle de veau,                                                                \\
        &         &    & lait,                                                                            \\
        &         &    & farine,                                                                          \\
        &         &    & muscade,                                                                         \\
        &         &    & sel et poivre.                                                                   \\
\end{longtable}
\normalsize
                                          
Hachez séparément la cervelle crue et les champignons pelés.

Mettez-les dans un mortier avec la mie de pain rassis tamisée imbibée de lait
et égouttée ; pilez bien le mélange, puis mettez-le sur le feu, dans une
casserole, avec le beurre ; chauffez ; liez le tout ensuite avec les œufs que
vous ajouterez l'un après l’autre ; salez, poivrez ; mettez le persil haché et
de la muscade au goût.

Travaillez bien la pâte et amenez-la à bonne consistance. Laissez-la refroidir.

Moulez des boulettes de la grosseur de petites noix, passez-les dans de la
farine ; faites-les cuire dans du bouillon bouillant et mettez-les dans le
potage au moment de servir.

\sk

Ces boulettes font bonne figure notamment dans les consommés de volaille.

\index{Définition des quenelles}
\index{Quenelles (Définition des)}
\section*{\centering Quenelles\footnote{
                                    Les quenelles sont des boulettes plus ou moins 
                                    oblongues de substances alimentaires passées 
                                    au tamis.} 
                     à la moelle.}

\addcontentsline{toc}{section}{ Quenelles à la moelle.}
\index{Quenelles à la moelle}

Pour six personnes prenez :

\medskip

\footnotesize
\begin{longtable}{rrrp{16em}}                                                    
    125 & grammes & de & moelle de bœuf,                                                                  \\
    125 & grammes & de & chapelure,                                                                       \\
        &         &  3 & œufs entiers,                                                                    \\
        &         &    & farine,                                                                          \\
        &         &    & sel.                                                                             \\
\end{longtable}
\normalsize
               
Écrasez la moelle, passez-la et incorporez-lui successivement les œufs en
tournant chaque fois jusqu'à amalgamation ; puis ajoutez par petites quantités
la chapelure, salez au goût et mélangez bien le tout.

Mettez cette pâte sur une planche saupoudrée de farine, passez-la légèrement au
rouleau, coupez-la en morceaux du volume d'une noix, roulez chaque morceau en
quenelle ; faites-les cuire dans du bouillon bouillant, enlevez-les avec une
écumoire et mettez-les dans le potage.

\section*{\centering Profiteroles.}
\addcontentsline{toc}{section}{ Profiteroles.}
\index{Profiteroles}
\label{pg0258} \hypertarget{p0258}{}

Pour douze personnes prenez :

\medskip

\footnotesize
\begin{longtable}{rrrp{16em}}                                                    
    250 & grammes & d' & eau,                                                                             \\
    200 & grammes & de & farine,                                                                          \\
    100 & grammes & de & beurre,                                                                          \\
     10 & grammes & de & sel,                                                                             \\
        &         &  6 & œufs frais.                                                                      \\
\end{longtable}
\normalsize
                             
Mettez dans une casserole le beurre et la farine ; chauffez doucement ; délayez
avec l'eau ; mélangez et travaillez bien le tout pour qu'il n'y ait pas de
grumeaux. Laissez cuire la pâte pendant cinq minutes en surveillant afin
d'éviter qu'elle s'attache au fond,

Éloignez la casserole du feu ; ajoutez les œufs un à un ; triturez bien après
l'addition de chaque œuf. La pâte est à point lorsque, après en avoir pris un
peu dans une cuiller que l'on retourne, son poids la fait tomber sans qu'elle
s'étale. Versez-la dans une poche à douille et moulez sur des plaques en tôle
des petites boulettes de pâte d'un centimètre et demi de diamètre environ, que
vous ferez cuire et que vous sécherez ensuite à l'étuve.

Les profiteroles pour potages doivent être croustillantes.

Servez-les avec des potages clairs, soit telles quelles, soit fourrées de foie
gras, de caviar ou de corail d'oursins, par exemple, suivant que le potage est
gras ou maigre.

\section*{\centering Pâtes au parmesan.}
\addcontentsline{toc}{section}{ Pâtes au parmesan.}
\index{Pâtes au parmesan}

Pour six Personnes prenez :

\medskip

\footnotesize
\begin{longtable}{rrrp{16em}}                                                    
    500 & grammes & de & lait,                                                                            \\
     30 & grammes & de & parmesan râpé,                                                                   \\
        &         &  6 & œufs entiers,                                                                    \\
        &         &    & sel.                                                                             \\
\end{longtable}
\normalsize               

Mélangez lait, fromage et œufs, salez au goût, mettez ce mélange dans un moule
et faites cuire au bain-marie pendant vingt minutes.

Pour s'assurer que la cuisson est complète, il suffit d'enfoncer une paille dans
la pâte : elle doit en sortir nette.

Démoulez alors la pâte, mettez-la sur une table, laissez-la légèrement refroidir,
puis découpez-la avec goût et mettez les morceaux obtenus dans le potage.

\section*{\centering Croûtons au parmesan.}
\phantomsection
\addcontentsline{toc}{section}{ Croûtons au parmesan.}
\index{Croûtons au parmesan}
\label{pg0259} \hypertarget{p0259}{}

Pour six personnes prenez :

\smallskip

\footnotesize
\begin{longtable}{rrrp{16em}}                                                    
     50 & grammes & de & parmesan râpé,                                                                   \\
     20 & grammes & de & farine,                                                                          \\
        &         &  2 & œufs frais,                                                                      \\
        &         &    & poivre,                                                                          \\
        &         &    & muscade.                                                                         \\
\end{longtable}
\normalsize

Mélangez la farine et le parmesan ; assaisonnez au goût avec poivre et muscade.

Cassez les œufs ; séparez les blancs des jaunes ; enlevez les germes.

Fouettez les blancs en neige ; lorsqu'ils seront fermes, incorporez-y les
jaunes de façon que la neige ne tombe pas ; puis, sans tourner, ajoutez, par
petites quantités, le mélange farine et parmesan ; amenez la pâte à la
consistance d'œufs brouillés très cuits.

Étalez cette pâte sur une plaque en tôle, en une couche d'un centimètre environ
d'épaisseur et faites-la cuire au four pendant une vingtaine de minutes. Puis,
avant son complet refroidissement, découpez-la en petites lames d'une forme
quelconque, ronde, ovale, carrée, en losanges, etc.
