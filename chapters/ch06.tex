\index{Définition et classification des potages}
\index{Définition des potages}
\index{Potages (Définition des)}
\label{pg0096} \hypertarget{p0096}{}
Les potages sont des aliments plus ou moins liquides, généralement servis chez
nous au commencement du repas du soir.

Ces aliments sont à base de bouillon de viande de boucherie, de volaille, de
gibier, de crustacés, de poissons ou de légumes.

Ils sont accompagnés ou non d'une garniture.

\index{Définition des soupes}
\index{Soupes (Définition des)}
Lorsque la garniture est constituée, en totalité ou en partie, par des tranches
de pain qu'on fait tremper dans le bouillon, le potage prend le nom de
soupe\footnote{Le mot soupe signifiait autrefois tranche de pain ; d'où
l'expression « tremper la soupe » et l'emploi, au figuré, du mot soupe pour
désigner l'ensemble de l'aliment.}.

Les potages proprement dits peuvent être divisés en deux grandes classes,
suivant qu'ils sont ou ne sont pas liés.

Les potages non liés comprennent les bouillons et les consommés.

La liaison des potages peut être faite de bien des façons : au moyen d'un roux
plus ou moins foncé, de jaunes d'œufs, de beurre, de crème, d'un féculent
quelconque, etc. En réalité, on emploie le plus souvent plusieurs de ces agents
à la fois ; aussi est-il difficile de classer les potages d'après leur mode de
liaison.

Les trois caractères les plus nettement différentiels sont :

a) l'emploi de purées ; b) l'emploi de crème sans addition de jaunes d'œufs ;
c) l'emploi de crème additionnée de jaunes d'œufs. C’est sur eux que je vais baser
mon essai de classification des potages liés, que je diviserai en six groupes :

\index{Classification des potages}
1° les potages sans purée ni crème, qui seront désignés par leurs éléments
constitutifs ;

2° les potages sans purée, dans la liaison desquels entre de la crème et pas de
jaunes d'œufs ; ce seront les \textit{potages à la crème} ;

3° les potages sans purée, dans la liaison desquels entrent de la crème et des
jaunes d'œufs ; je les désignerai sous le nom de \textit{potages
veloutés}\footnote{Le terme « velouté » est employé tour à tour, dans le
langage culinaire courant, avec deux sens différents : il sert, d'une part,
à désigner les sauces blondes (grasses ou maigres), bien dépouillées, à base de
fond blanc, liées avec de la fécule ou avec de la farine blondie dans du beurre
et, d'autre part, les potages liés avec des jaunes d'œufs délayés dans de la
crème ; d'où des confusions.

Pour éviter toute ambiguïté, j'ai employé le vocable comme substantif, dans la
classification des sauces, \hyperlink{p0096}{p. \pageref{pg0096}} et je l'emploie comme
adjectif, dans la classification ci-dessus, pour qualifier les potages sans
purée, liés aux jaunes d'œufs délayés dans de la crème.} ;

4° les \textit{potages purée} proprement dits, sans crème ni jaunes d'œufs ;

5° les potages purée, dans la liaison desquels entre de la crème et pas de
jaunes d'œufs ; je les appellerai \textit{potages crème}\footnote{Le mot
potage crème, qui est assez euphonique, est employé dans des sens différents :
les uns appellent ainsi, on vue de les anoblir, de simples potages purée ;
d'autres semblent vouloir désigner de cette façon les potages liés à la crème
et aux jaunes d'œufs ; d'autres enfin l'appliquent aux potages liés avec un roux
plus ou moins foncé et de la crème. Il est essentiel de préciser le sens des
mots que l'on emploie.} ;

6° les potages purée, dans la liaison desquels entrent simultanément de la
crème et des jaunes d'œufs ; je leur réserverai le nom de \textit{potages crème
veloutée}.

Dans l'exposé des formules j'observerai l'ordre général suivant : 1° Potages
non liés ; 2° Potages liés sans purée ni crème ; 3° Potages à la crème ;
4° Potages veloutés ; 5° Potages purée ; 6° Potages crème ; 7° Potages crème
veloutée ; 8° Soupes.
