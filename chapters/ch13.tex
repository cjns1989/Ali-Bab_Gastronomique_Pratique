Les hors-d'œuvre sont des plats sans importance. Leur nom vient de ce que,
n'étant pas classés parmi les plats principaux composant les repas, on peut au
besoin les supprimer sans rompre en quoi que ce soit la belle ordonnance des
menus. À l'origine, on laissait les hors-d'œuvre sur la table à la disposition
des convives auxquels ils servaient de distraction entre les différents mets :
c'était des amuse-gueule. Aujourd'hui, on les sert au commencement du déjeuner
et après le potage au dîner. C'est un avant-propos, une introduction. un lever
de rideau, une ouverture d'opéra, un flirt, du marivaudage : les bagatelles de
la porte. Leur rôle consiste à exeiter l'appétit sans charger l'estomac ; ils
doivent donc être légers, délicats, de petit volume et de grande finesse de
goût.

Les hors-d'œuvre sont de deux sortes : hors-d’œuvre froids réservés
particulièrement pour les déjeuners et hors-d'œuvre chauds servis surtout dans
les dîners.

Les hors-d'œuvre froids qui ont une belle allure présentés dans des pièces
d'orfèvrerie à compartiments multiples, mais qu'on peut aussi dresser plus
modestement dans des raviers, sont : le beurre ; les radis, les tomates, le
céleri, la betterave, les concombres, le chou-rouge, les choux-fleurs, les
cornichons, les petits melons et les petits champignons marinés ; les pickles ;
les artichauts, les achards, le raifort, le fenouil ; les bigarreaux confits,
les cerneaux, les olives, les figues fraîches, le melon ; les escargots ; les
crevettes, les écrevisses, les moules, les huîttres, les coquillages ; les
anchois, les sardines et les petits maquereaux à l'huile, les sardines fumées,
les filets de harengs fumés ou marinés, le thon mariné, le saumon fumé, les
petites truites marinées, les anguilles, la poutargue, le caviar ; les œufs
farcis, les œufs garnis ; les rillettes, les rillons, les saucissons de toutes
sortes, les cervelas, la mortadelle, l'andouille fumée, les langues, le
jambon ; le museau et le palais de bœuf ; les cervelles de mouton et d'agneau ;
l’oie fumée ; des salades de légumes, des salades d'issues ; des macédoines ;
des barquettes, des croustades, des tartelettes de langue, de cervelle,
d'huîtres, de foie gras, de salpicons, de purées, de crèmes, de mousses, de
gelées de poissons, de crustacés, de volaille, de gibier, etc. ; des canapés,
des toasts garnis de beurre fin ou de beurres composés et de purées ou de
hachis de crustacés, de poissons, de viande, de volaille, de gibier ; des
éclairs, des choux, des brioches sans sucre fourrées de purées ou de mousses de
foie gras, de volaille, de gibier masquées de sauce chaud-froid assortie ou de
gelée ; etc., etc.

Le melon, les huîtres, le caviar sont toujours servis à part ; les huîtres dans
des plats spéciaux et le caviar dans des blocs de glace taillés, creusés en
hémisphère, dont les bords sont garnis de demi citrons.

Les hors-d'œuvre chauds sont généralement servis sur des plats recouverts d'une
serviette. Leur grande variété offre aux artistes en la matière un vaste champ où
leur talent peut s'exercer.

Ce sont des sortes d'entrées minuscules représentées par des brochettes
diverses composées de fines escalopes de crustacés, de langue, de jambon, de
ris d'agneau ou de veau, de foie de veau ou de foies de volaille, de cervelle,
de truffes, de champignons, de fonds d'artichauts, de foie gras, de volaille,
de gibier enrobées de sauces assorties puis panées et frites ; des petits
gâteaux feuilletés garnis de farces fines de poissons, de volaille, de gibier,
ou fourrés de fines aiguillettes ou de fines escalopes des mêmes substances
dressées sur des farces assorties ; des beignets de laitances et de poissons
divers, de cervelle, de volaille, de jambon, etc. ; des bouchées de légumes
printaniers, de pointes d’asperges, de champignons, de truffes, de crustacés,
de volaille, de gibier à plumes, de langue, de jambon, de foie gras, de
salpicons délicats, de purées de gibier à poil ; des farces diverses ; des
fondants panés et frits de substances fines ; des fritots ; des huîtres cuites
à toutes sauces ; des barquettes, des croustades, des croûtes, des tartelettes
garnies de petites escalopes ou de fins salpicons, de purées ou de farces de
crustacés, de poissons, de foie gras, de volaille, de gibier avec sauces
adéquates, d'huîtres, de laitances, de crêtes et de rognons de coq ; de petites
pommes de terre fourrées ; des croquettes, des rissoles de salpicons divers avec
accompagnement de garniture en rapport, des petits soufflés, des petits pâtés,
des petites timbales, d'un volume restreint.

Les hors-d'œuvre chauds rappellent les anciennes entrées volantes du service
à la française. Leur passage sur la table ne dure qu'un moment juste suffisant
pour permettre aux convives d'apprécier la finesse de la cuisine et bien faire
augurer du repas.
