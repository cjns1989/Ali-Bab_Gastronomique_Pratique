\section*{\centering Ortolans en sarcophages.}
\phantomsection
\addcontentsline{toc}{section}{ Ortolans en sarcophages.}
\index{Ortolans en sarcophages}

Pour douze personnes prenez ;

\medskip

\footnotesize
\begin{longtable}{rrrp{16em}}
    400 & grammes & de & fond de veau,                                                                    \\
    200 & grammes & de & purée de foie gras,                                                              \\
    100 & grammes & de & madère ou de porto blanc,                                                        \\
     40 & grammes & de & mirepoix,                                                                        \\
        &         & 12 & ortolans à point, fraîchement étouffés dans de la vieille fine champagne,        \\
        &         & 12 & grosses truffes noires du Périgord,                                              \\
        &         & 12 & fines bardes de lard,                                                            \\
        &         &  3 & grives,                                                                          \\
        &         &    & quatre épices,                                                                   \\
        &         &    & sel et poivre.                                                                   \\
\end{longtable}
\normalsize

Désossez les ortolans, farcissez-les avec la purée de foie gras.

Creusez les truffes, en réservant dans chacune une rondelle pour faire bouchon ;
assaisonnez au goût avec sel, poivre et épices.

Faites rôtir les grives, passez-les à la presse, ajoutez au jus obtenu le fond
de veau, le vin et le mirepoix, chauffez, mettez ensuite les ortolans, pendant
cinq minutes, dans ce mélange, qui constitue un excellent fond de gibier, puis
sortez-les.

Concentrez le fond, dégraissez-le, dépouillez-le.

Insérez un ortolan dans chaque truffe, que vous fermerez avec les bouchons
réservés, bardez-les,

Faites cuire au moyen de l'un des procédés indiqués pour la cuisson des truffes,
et notamment celui dit « à la maréchale ».

Enlevez les résidus des bardes, dressez les truffes sur un plat garni d'une
serviette, servez en envoyant en même temps le jus concentré et dépouillé dans
une saucière.

Ce mets, auquel on à donné aussi le nom pittoresque mais un peu macabre
d'ortolans en cercueils, constitue l'une des préparations les plus raffinées de
la cuisine moderne.

Un champagne sec l'accompagne à ravir.

\section*{\centering Ortolans pochés à la fine champagne.}
\phantomsection
\addcontentsline{toc}{section}{ Ortolans pochés à la fine champagne.}
\index{Ortolans pochés à la fine champagne}

Pour quatre personnes prenez :

\medskip

\footnotesize
\begin{longtable}{rrrp{16em}}
    200 & grammes & de & consommé de volaille,                                                            \\
     40 & grammes & de & fine champagne,                                                                  \\
        &         & 12 & ortolans bien gras,                                                              \\
        &         &    & sel et poivre.                                                                   \\
\end{longtable}
\normalsize

Mettez dans l'intérieur de chaque ortolan un douzième de la fine champagne ;
salez-les, poivrez-les ; liez-les trois par trois, puis introduisez-les dans
une vessie de porc parfaitement nettoyée et ajoutez-y le consommé de volaille.

Fermez hermétiquement la vessie et plongez-la dans de l'eau bouillante ;
laissez cuire pendant vingt minutes.

Sortez les ortolans de la vessie, dressez-les sur ur plat ; tenez-les au chaud.

Réduisez le fond de cuisson, dégraissez-le, versez-le sur les ortolans et
servez.

\sk

\index{Becfigues pochés à la fine champagne}
\index{Alouettes pochées à la fine champagne}
\index{Grives pochées à la fine champagne}
\index{Cailles pochées à la fine champagne}
On peut faire cuire de même d'autres petits oiseaux : des becfigues, des
alouettes, des grives, des cailles, etc.

\section*{\centering Alouettes en linceul.}
\phantomsection
\addcontentsline{toc}{section}{ Alouettes en linceul.}
\index{Alouettes en linceul}

Prenez de belles pommes de terre de Hollande, pelez-les ; évidez-les ; réservez
des petits morceaux de pulpe pour faire des bouchons.

Plumez, flambez, désossez les alouettes, assaisonnez-les, farcissez-les avec un
mélange de gibier et de foie gras, truffé ou non, et insérez chaque alouette
dans une pomme de terre. Fermez-les.

Mettez les pommes de terre ainsi garnies dans un plat de service allant au feu,
mouillez avec la quantité nécessaire de bon fond de gibier à plumes pour les
couvrir ; faites cuire au four en arrosant fréquemment.

Lorsque les pommes de terre ne seront plus couvertes par le liquide, ajoutez
du beurre par petits morceaux et achevez la cuisson.

Les pommes de terre doivent être dorées et croustillantes.

\section*{\centering Alouettes en cocote.}
\phantomsection
\addcontentsline{toc}{section}{ Alouettes en cocote.}
\index{Alouettes en cocote}
\index{Alouettes (salmis d')}
\index{Salmis d'alouettes}

Pour six personnes prenez :

\medskip

\footnotesize
\begin{longtable}{rrrp{16em}}
    250 & grammes & de & foie gras d'oie,                                                                 \\
    250 & grammes & de & truffes,                                                                         \\
     75 & grammes & de & beurre,                                                                          \\
     50 & grammes & de & madère,                                                                          \\
     25 & grammes & de & farine,                                                                          \\
        &         & 30 & alouettes,                                                                       \\
        &         &  2 & abatis de poulardes,                                                             \\
        &         &    & légumes de pot-au-feu,                                                           \\
        &         &    & sel et poivre.                                                                   \\
\end{longtable}
\normalsize

Plumez, flambez, désossez les alouettes ; réservez les déchets.

Nettoyez les abatis de volaille.

Lavez, brossez les truffes ; séchez-les dans un linge ; coupez-les en tranches.

Coupez le foie gras en trente cubes ; farcissez-en les alouettes, salez,
poivrez. Ficelez les alouettes.

Préparez un demi-litre environ de bon jus en faisant cuire à bouilli perdu,
dans de l'eau salée et poivrée, les abatis, les déchets des alouettes et des
légumes.

Faites un roux avec {\ppp30\mmm} grammes de beurre et la farine ; mouillez-le
avec le jus et le madère ; dépouillez la sauce ; concentrez-la.

Mettez les alouettes avec le reste du beurre dans une cocote ; faites-les
revenir ; mouillez avec la sauce, ajoutez les truffes, poussez au four et
laissez cuire très doucement pendant un quart d'heure.

Servez dans la cocote ou bien dressez les alouettes sur un plat après les avoir
débarrassées des ficelles : décorez avec les truffes ; masquez avec la sauce et
servez.

Ce plat, qui est en somme un salmis d'alouettes, est une excellente entrée de
gibier.

\sk

On pourra corser le plat en remplaçant un abatis de volaille par une dizaine
d'alouettes qu'on fera rôtir à la broche, qu'on passera ensuite à la presse et
dont on incorporera le jus à la sauce au moment de finir la cuisson au four.

\sk

\index{Becfigues en salmis}
\index{Grives en salmis}
\index{Cailles en salmis}
\index{Salmis de gibier à plume}
On peut préparer de même des becfigues, des grives, des cailles et, en général,
tout le petit gibier à plumes.

\section*{\centering Pâté d'alouettes.}
\phantomsection
\addcontentsline{toc}{section}{ Pâté d'alouettes.}
\index{Pâté d'alouettes}
\index{Alouettes en pâté}
\index{Croûte pour pâtés}
\index{Garniture pour pâtés}

Pour douze personnes prenez :

\medskip

1° pour la pâte :

\footnotesize
\begin{longtable}{rrrp{16em}}
    500 & grammes & de & farine,                                                                          \\
    250 & grammes & d' & eau tiède,                                                                       \\
    200 & grammes & de & beurre,                                                                          \\
     30 & grammes & d' & huile d'olive non fruitée,                                                       \\
     20 & grammes & de & sel,                                                                             \\
        &         &  4 & jaunes d'œufs ;                                                                  \\
\end{longtable}
\normalsize

2° pour la garniture :

\footnotesize
\begin{longtable}{rrrp{16em}}
    350 & grammes & de & lard frais,                                                                      \\
    125 & grammes & de & champignons de couche,                                                           \\
    100 & grammes & de & noix de veau.                                                                    \\
    100 & grammes & de & jambon frais,                                                                    \\
    100 & grammes & de & chair de poularde,                                                               \\
    100 & grammes & de & madère,                                                                          \\
    100 & grammes & de & porto blanc,                                                                     \\
     10 & grammes & de & fine champagne,                                                                  \\
        &         & 12 & alouettes,                                                                       \\
        &         & 12 & fines bardes de lard,                                                            \\
        &         & 12 & truffes noires du Périgord, semblables autant que possible,
                         et pesant ensemble 1 kilogramme environ,                                         \\
        &         &  6 & foies de volaille,                                                               \\
        &         &  1 & beau foie gras d’oie et la moitié d'un autre,                                    \\
        &         &    & échalotes,                                                                       \\
        &         &    & thym mondé,                                                                      \\
        &         &    & jaune d'œuf,                                                                     \\
        &         &    & sel et poivre ;                                                                  \\
\end{longtable}
\normalsize

3° pour la gelée :

\footnotesize
\begin{longtable}{rrrp{16em}}
  1 000 &  grammes & de & jarret de veau,                                                                 \\
    750 &  grammes & de & porto blanc,                                                                    \\
     25 &  grammes & de & fine champagne,                                                                 \\
        & 3 litres & de & consommé,                                                                       \\
        &          &  1 & faisan,                                                                         \\
        &          &  1 & pied de veau,                                                                   \\
        &          &    & légumes de pot-au-feu,                                                          \\
        &          &    & beurre,                                                                         \\
        &          &    & blancs d'œufs.                                                                  \\
\end{longtable}
\normalsize

\textit{Préparations préliminaires}. — La veille du jour où vous voudrez servir
ce pâté, faites Les opérations suivantes :

1° Plumez, videz, flambez, désossez les alouettes ; mettez-les à mariner
pendant {\ppp24\mmm} heures dans {\ppp100\mmm} grammes de porto ; réservez les
têtes, les intérieurs et les os.

2° Brossez les truffes, lavez-les, séchez-les : mettez-les à mariner dans le
madère.

3° Mélangez tous les éléments du premier paragraphe de façon à avoir une pâte
bien homogène ; roulez-la en boule sans la travailler autrement ; enveloppez-la
dans un linge fariné.

4° Faites cuire à bouilli perdu le faisan dans le consommé. Puis mettez dans
une casserole du beurre, des légumes de pot-au-feu émincés, le jarret et le
pied de veau coupés en morceaux ; laisser pincer ; déglacez avec le porto
blanc ; mouillez avec le consommé au faisan ; laissez cuire ; dépouillez
pendant la cuisson et concentrez le liquide de façon à avoir un litre un quart
environ de jus. Clarifiez-le avec des blancs d'œufs ; laissez prendre en gelée.

\medskip

\index{Farce gratin}
\textit{Préparation de la garniture}. — Le lendemain, faites revenir dans une
poêle {\ppp250\mmm} grammes de lard frais coupé en tranches, en évitant que la
graisse se colore ; ajoutez échalotes hachées, thym mondé, sel et poivre, au
goût, les intérieurs, nettoyés et les os des alouettes, les foies de volaille
coupés en dés et les champignons hachés fin : chauffez à plein feu jusqu'à
dégagement de l'odeur empyreumatique des champignons.

Retirez du feu ; mettez le tout dans un mortier ; pilez, passez au tamis de
soie, puis incorporez à cette farce {\ppp10\mmm} grammes de fine champagne :
vous aurez ainsi une farce fine aux foies de volaille appelée « gratin ».

Prenez les bardes de lard, étalez sur chacune une couche de gratin, ensuite
une alouette ouverte dont vous garnirez l'intérieur avec une autre couche de
gratin.

Émincez le foie gras en {\ppp12\mmm} escalopes régulières ; réservez les
déchets.

Enveloppez chaque truffe dans une escalope de foie gras et placez sur chaque
alouette ouverte et garnie de gratin une truffe bardée de foie gras. Roulez et
fermez les alouettes en repliant dessus les bardes de lard.

\medskip

\label{pg0613} \hypertarget{p0613}{}
\textit{Préparation de la farce}. — Pilez ensemble la noix de veau, le jambon
frais, la chair de poularde, le reste du lard, les déchets réservés de foie
gras, assaisonnez avec sel, poivre et mouillez avec tout ou partie du porto et
du madère des marinades, de manière à obtenir une farce de bonne consistance.

\smallskip

\textit{Dressage du pâté}. — Abaissez la pâte, réservez-en une partie pour les
couvercles et les bouchons des cheminées. Garnissez avec cette abaisse les
parois d'un moule suffisamment long et large dans lequel vous pourrez disposer
sur deux rangées de six les alouettes farcies et bardées. Mettez sur cette
abaisse une couche de farce, placez dessus les alouettes, emplissez les
intervalles et couvrez avec le reste de la farce ; lissez la surface. Préparez
deux couvercles de pâte, l'un plus mince que l’autre ; mouillez le premier
couvercle, le plus mince, collez-le sur la face lissée de la farce ; ménagez
dedans de nombreuses ouvertures pour le dégagement des gaz à la cuisson ;
couvrez avec le second couvercle, percé de deux ou trois trous destinés à faire
cheminées ; fixez-le à la pince ; décorez le dessus ; dorez-le au jaune d'œuf ;
dorez aussi les bouchons.

Laissez reposer le pâté au frais pendant {\ppp3\mmm} ou {\ppp4\mmm} heures.

\medskip

\textit{Cuisson}. — Au four chaud, pendant une heure à une heure et demie
environ.

Lorsque le pâté est cuit à point, enlevez-le du four, laissez-le refroidir,
démoulez-le.

\medskip

\textit{Finissage}. — Réchauffez la gelée, relevez-la avec {\ppp25\mmm} grammes
de fine champagne ; dépouillez-la encore ; laissez-la refroidir incomplètement
et, lorsqu'elle sera à une température voisine de son point de solidification,
coulez-la dans le pâté.

Obturez les ouvertures des cheminées avec les bouchons de pâte que vous aurez
fait cuire à part et décorez le dessus du pâté avec les têtes des alouettes que
vous aurez fait pocher dans du consommé.

Pour servir, pratiquez d'abord une profonde entaille au milieu du pâté dans
toute sa longueur, puis faites-en d'autres perpendiculairement, de manière que
chaque part renferme une alouette. Les gros mangeurs prendront une part
entière, les autres se contenteront de moins.

Ce pâté est d'une délicatesse incomparable ; il laisse loin derrière lui les
fameux pâtés d'alouettes de Pithiviers.

\sk

Cette formule peut être appliquée à tous les petits gibiers à plumes : cailles,
grives, becfigues, etc.

\section*{\centering Perdreaux à la coque.}
\phantomsection
\addcontentsline{toc}{section}{ Perdreaux à la coque.}
\index{Perdreaux à la coque}

Prenez de jeunes perdreaux frais, plumez-les, videz-les, farcissez-les avec un
mélange de truffes et de foie gras cuits dans du madère et assaisonnés avec sel
et poivre. Cousez tous les orifices, introduisez chaque perdreau dans une
vessie de porc, bien nettoyée au préalable, puis plongez-les, ainsi habillés,
dans de l'eau bouillante. Laissez-les cuire pendant une demi-heure environ.

Retirez-les de l'eau, sortez-les des vessies, laissez les refroidir, enlevez
les fils,

Servez les perdreaux froids, tels quels, sans sauce.

C'est un plat de choix pour déjeuner et, en particulier, le matin, avant de
partir pour la chasse. Un vieux bourgogne en est le complément indiqué.

\sk

On peut préparer de même de jeunes palombes ou des halbrans.

\section*{\centering Perdreaux au chou.}
\phantomsection
\addcontentsline{toc}{section}{ Perdreaux au chou.}
\index{Perdreaux au chou}

Pour huit personnes prenez :

\smallskip

\footnotesize
\begin{longtable}{rrrp{16em}}
    250 & grammes & de & lard maigre,                                                                     \\
    250 & grammes & de & saucisse de Toulouse,                                                            \\
    250 & grammes & de & vin blanc,                                                                       \\
    125 & grammes & de & jus de viande,                                                                   \\
    100 & grammes & de & carottes,                                                                        \\
     65 & grammes & de & beurre,                                                                          \\
     60 & grammes & de & graisse de volaille,                                                             \\
     50 & grammes & d' & oignons,                                                                         \\
     45 & grammes & de & fine champagne,                                                                  \\
        &         &  2 & perdreaux,                                                                       \\
        &         &  1 & perdrix,                                                                         \\
        &         &  1 & saucisson pesant 250 grammes environ,                                            \\
        &         &  1 & gros chou de Milan ou 2 moyens,                                                  \\
        &         &  1 & bouquet garni,                                                                   \\
        &         &    & sel et poivre.                                                                   \\
\end{longtable}
\normalsize

Plumez, videz, flambez la perdrix et les perdreaux ; troussez ces derniers.

Nettoyez le chou et coupez-le en tranches minces comme pour la choucroute ;
lavez-le et blanchissez-le pendant un quart d'heure dans de l’eau salée bouillante ;
égouttez-le.

Mettez dans une casserole le beurre, la perdrix, le lard ; faites revenir ;
ajoutez ensuite le chou, la graisse de volaille, les carottes coupées en
tranches, les oignons, le bouquet garni ; mouillez avec le vin, le jus de
viande, la fine champagne ; salez et poivrez au goût, couvrez la casserole et
laissez cuire pendant deux heures. Au bout d'une heure un quart à une heure et
demie de cuisson, mettez le saucisson, et un quart d'heure avant la fin ajoutez
la saucisse de Toulouse, après l'avoir fait revenir.

Enlevez les oignons et le bouquet garni.

Retirez la perdrix, désossez-la, passez la chair au tamis, puis incorporez-la
au chou. Tenez au chaud.

Faites rôtir à la broche les deux perdreaux ; découpez-les ; tenez-les au
chaud.

Passez à la presse les carcasses de perdrix et de perdreaux ; mélangez au chou
le jus exprimé. Coupez en tranches lard, saucisson et saucisse de Toulouse.

Disposez le chou dans un plat, dressez dessus les membres des perdreaux, les
tranches de lard, de saucisson et de saucisse ; servez,

C'est parfait.

Cette formule ainsi que celle du chou farci à la perdrix sont, à mon avis, les
deux solutions les plus élégantes et les plus savoureuses de l'antique problème
de la perdrix au chou.

\sk

On pourra apprêter de même la pintade au chou.

\section*{\centering Bouchées de perdreau farcies, sauce demi-glace au fumet de gibier.}
\phantomsection
\addcontentsline{toc}{section}{ Bouchées de perdreau farcies, sauce demi-glace au fumet de gibier.}
\index{Bouchées de perdreau farcies, sauce demi-glace au fumet de gibier}
\index{Bouchées de gibier à plumes}

Pour dix personnes prenez :

\medskip

\index{Enveloppe pour bouchées de perdreau}
\index{Bouchées (Enveloppes pour)}
1° pour l'enveloppe :

\footnotesize
\begin{longtable}{rrrp{16em}}
    500 & grammes & de & fond de gibier à plumes\footnote{On prépare le fond de gibier à plumes
                                                  en faisant cuire du gibier à plumes ou des déchets
                                                  de gibier à plumes avec des légumes, dans de l'eau,
                                                  jusqu'à épuisement des viandes, et en concentrant la
                                                  cuisson.},                                              \\
    350 & grammes & de & chair de perdreau,                                                               \\
    200 & grammes & de & farine,                                                                          \\
    150 & grammes & de & graisse de volaille,                                                             \\
    100 & grammes & de & graisse de rognon de veau,                                                       \\
        &         &  4 & œufs entiers,                                                                    \\
        &         &  2 & blancs d'œufs,                                                                   \\
        &         &  1 & jaune d'œuf,                                                                     \\
        &         &    & sel et poivre ;                                                                  \\
\end{longtable}
\normalsize

\index{Farce pour bouchées de perdreau}
\index{Farce pour gibier à plumes}
2° pour la farce :

\footnotesize
\begin{longtable}{rrrp{16em}}
    300 & grammes & d' & un mélange composé de cailles on de grives rôties, de truffes
                         cuites au madère ou au porto, le tout passé au tamis et amalgamé
                         avec du foie gras ;                                                              \\
\end{longtable}
\normalsize

3°

\footnotesize
\begin{longtable}{rrrp{16em}}
  200 & grammes & de & sauce demi-glace \hyperlink{p0456}{p. \pageref{pg0456}}, au fumet de 
                          grives si la farce est aux cailles, et réciproquement au fumet 
                          de cailles si la farce est aux grives.                                          \\
\end{longtable}
\normalsize

Triturez ensemble la farine et les œufs entiers ; ajoutez en tournant le fond
de gibier chaud par petites quantités ; faites prendre l'appareil sur feu doux
en le travaillant pendant une demi-heure environ de façon à obtenir une bonne
consistance.

Pilez séparément la chair de perdreau et la graisse de volaille et de veau,
réunissez-les, assaisonnez avec sel et poivre, pilez encore, puis amalgamez au
mélange l'appareil précédent et les deux blancs d'œufs battus en neige.

Passez le tout au tamis ; travaillez bien la pâte pour la rendre homogène et
lisse, et faites-en une abaisse de {\ppp5\mmm} à {\ppp6\mmm} millimètres d'épaisseur.

Partagez cette abaisse en vingt morceaux carrés, mettez sur chaque morceau un
vingtième de la farce ; fermez les bouchées.

Faites pocher les bouchées dans de l'eau salée bouillante ; égouttez-les,
dorez-les au jaune d'œuf, puis passez-les au four.

Servez les bouchées sur un plat garni d'une serviette, et la sauce dans une
saucière.

\sk

Il est aisé de concevoir d’autres bouchées de gibier à plumes en changeant la
nature du gibier entrant dans la composition de la pâte et dans celle de la
farce, et en variant la sauce.

\section*{\centering Pâté de perdreaux.}
\phantomsection
\addcontentsline{toc}{section}{ Pâté de perdreaux.}
\index{Perdreaux (Pâté de)}
\index{Croûte pour pâtés}

Pour dix-huit à vingt personnes prenez :

\medskip

1° pour la pâte :

\footnotesize
\begin{longtable}{rrrrrp{18em}}
 & \hspace{2em} &   750 & grammes  & de & farine,                                                         \\
 & \hspace{2em} &   300 & grammes  & d' & eau tiède,                                                      \\
 & \hspace{2em} &   250 & grammes  & de & beurre,                                                         \\
 & \hspace{2em} &    35 & grammes  & d' & huile d'olive,                                                  \\
 & \hspace{2em} &    30 & grammes  & de & sel,                                                            \\
 & \hspace{2em} &       &          &  5 & jaunes d'œufs frais ;                                           \\
\end{longtable}
\normalsize

2° pour l'intérieur :

\footnotesize
\begin{longtable}{rrrrrp{18em}}
 & \hspace{2em} & 1 000 & grammes  & de & foie gras d'oie, blanc rosé,                                    \\
 & \hspace{2em} &   500 & grammes  & de & truffes noires du Périgord, au moins,                           \\
 & \hspace{2em} &   300 & grammes  & de & madère,                                                         \\
 & \hspace{2em} &   200 & grammes  & de & noix de veau pâtissière,                                        \\
 & \hspace{2em} &   200 & grammes  & de & filet de porc frais,                                            \\
 & \hspace{2em} &   200 & grammes  & de & lard gras frais,                                                \\
 & \hspace{2em} &    60 & grammes  & de & fine champagne,                                                 \\
 & \hspace{2em} &    40 & grammes  & de & sel blanc,                                                      \\
 & \hspace{2em} &     5 & grammes  & de & poivre fraîchement moulu,                                       \\
 & \hspace{2em} &       &          &  6 & perdreaux ;                                                     \\
\end{longtable}
\normalsize

3° pour la gelée :

\footnotesize
\begin{longtable}{rrrrrp{18em}}
 & \hspace{2em} &   750 & grammes  & de & gîte de bœuf,                                                   \\
 & \hspace{2em} &   750 & grammes  & de & jarret de veau,                                                 \\
 & \hspace{2em} &   500 & grammes  & de & vin blanc,                                                      \\
 & \hspace{2em} &    75 & grammes  & de & carottes,                                                       \\
 & \hspace{2em} &    30 & grammes  & de & sel gris,                                                       \\
 & \hspace{2em} &    10 & grammes  & de & céleri,                                                         \\
 & \hspace{2em} &   1/2 & gramme   & de & poivre en grains,                                               \\
 & \multicolumn{3}{r}{15 centigrammes} & de & muscade,                                                    \\
 & \hspace{2em} &       & 2 litres & d' & eau,                                                            \\
 & \hspace{2em} &       &          &  3 & abatis de poulardes,                                            \\
 & \hspace{2em} &       &          &  1 & pied de veau,                                                   \\
 & \hspace{2em} &       &          &  1 & poireau moyen (le blanc seulement),                             \\
 & \hspace{2em} &       &          &  1 & échalote,                                                       \\
 & \hspace{2em} &       &          &  1 & clou de girofle,                                                \\
 & \hspace{2em} &       &          &    & beurre ou graisse de volaille,                                  \\
 & \hspace{2em} &       &          &    & blancs d'œufs,                                                  \\
 & \hspace{2em} &       &          &    & bouquet garni (thym, laurier, persil, cerfeuil).                \\
\end{longtable}
\normalsize

\medskip

\textit{Préliminaires}. — La veille du jour où vous voudrez faire ce pâté,
plumez, videz, flambez les perdreaux, désossez-les, enlevez les nerfs ;
réservez les déchets et les os.

Brossez les truffes, lavez-les, séchez-les dans un linge.

Mettez, dans un ustensile fermé, perdreaux, noix de veau, filet de porc, lard,
truffes et foie gras, dont vous aurez enlevé la peau ; arrosez avec le madère,
Laissez en contact.

Nettoyez les abatis de poulardes ; flambez-les ; réservez-les.

\medskip

\textit{Préparation de la gelée}. — Le lendemain, coupez le gîte de bœuf, le
jarret et le pied de veau en morceaux, faites-les revenir avec les abatis,
moins les oies, dans un peu de beurre ou de graisse de volaille ; retirez-les
et remplacez-les par les carottes émincées, le céleri, le poireau et
l’échalote ; laissez pincer légèrement. Égouttez la graisse.

Mettez dans une marmite les viandes et les légumes revenus, mouillez avec
l'eau ; faites bouillir, écumez, ajoutez le vin blanc, les déchets et les os
des perdreaux, le bouquet garni, le sel, le poivre, la muscade, le girofle.
Laissez cuire à tout petit feu pendant {\ppp6\mmm} à {\ppp7\mmm} heures.
Dégraissez. Dépouillez pendant la cuisson. Concentrez ce fond.

\medskip

\textit{Préparation de la pâte}. — Mettez la farine sur une planche, creusez-la
en cratère de volcan, versez dedans l'eau dans laquelle vous aurez fait
dissoudre le sel, ajoutez les jaunes d'œufs, moins une petite partie qui vous
servira à dorer le pâté, le beurre, amené à une consistance analogue à celle de
la pâte que vous voulez obtenir, et l'huile d'olive. Pétrissez vivement,
mélangez bien, fraisez trois fois ; la pâte sera suffisamment travaillée
lorsqu'elle se détachera d'elle-même des doigts. Battez-la vigoureusement cinq
ou six fois avec le rouleau en bois ; roulez-la en boule ; laissez-la reposer.

Dix minutes suffisent généralement pour la confection de la pâte.

\medskip

\textit{Préparation de la garniture}. — Pelez les truffes ; réservez les
pelures.

Coupez une partie du lard gras en petits lardons à larder, et une partie des
truffes en petites baguettes de dimensions semblables à celles des lardons.

Levez sur les perdreaux les gros morceaux de chair ; réservez les débris.

Piquez les morceaux de perdreaux avec du lard gras et des baguettes de truffe,
assaisonnez avec du sel blanc et du poivre, puis flambez-les avec la fine
champagne.

Coupez le foie gras en tranches d'un centimètre d'épaisseur, piquez-les avec
des baguettes de truffe, salez, poivrez ; réservez les déchets.

Hachez ensemble la noix de veau, le filet de porc, les foies de poulardes, les
débris des perdreaux, les déchets du foie gras, le reste du lard gras et les
pelures de truffes, salez, poivrez, ajoutez un peu du madère de la marinade et
le reste de la fine champagne qui a servi à flamber les filets de perdreaux ;
mélangez de façon à obtenir une farce homogène de consistance convenable.

\medskip

\textit{Dressage du pâté}. — Abaissez la pâte au rouleau ; réservez-en une
partie pour les couvercles et les bouchons des cheminées.

Prenez un moule de forme rectangulaire ; chemisez-en le fond et les parois avec
l'abaisse de pâte, puis mettez une couche de farce, au-dessus une couche de
tranches de foie gras et la moitié des truffes restantes, entières ou coupées,
disposez par-dessus les morceaux de perdreaux en ayant soin que la direction
des lardons soit parallèle au grand axe du pâté ; emplissez les intervalles
avec de la farce ; placez au-dessus le reste des truffes et une autre couche de
tranches de foie gras ; terminez par une couche de farce. Lissez la surface.
Avec le reste de la pâte, moins ce qu'il faut pour deux ou trois bouchons,
faites deux couvercles, l'un plus mince que l’autre. Collez le plus mince sur
la surface lisse de la garniture ; pincez-en le bord avec celui de la pâte
garnissant le moule. Pratiquez sur ce couvercle, au moyen d'un couteau, de
nombreuses entailles en croix pour l'échappement des gaz pendant la cuisson.
Posez le second couvercle sur le premier et fixez-le en mouillant les bords.
Percez deux ou trois trous, du diamètre d'un gros crayon, traversant les deux
couvercles et destinés à faire cheminées ; décorez le dessus à la roulette et
dorez-le au pinceau avec le jaune d'œuf réservé délayé dans un peu d'eau. Dorez
aussi les bouchons.

\medskip

\textit{Cuisson}. — Placez le pâté sur une plaque de tôle et poussez au four
chaud. La cuisson doit durer une heure et demie environ.

\medskip

\textit{Finissage}. — Quand le pâté est cuit, laissez-le refroidir, puis démoulez-le.

Corsez le fond préparé pour la gelée avec le reste du madère de la marinade,
clarifiez-le avec les blancs d'œufs ; laissez-le refroidir jusqu'au moment où
il est près de son point de solidification ; introduisez-le alors dans le pâté,
par les cheminées, au moyen d'un entonnoir. Laissez refroidir, puis obturez les
cheminées avec les bouchons de pâte cuits à part.

Ce pâté est excellent et de beaucoup supérieur aux célèbres pâtés de Chartres.

\sk

\index{Bécasses en pâté}
\index{Bécassines en pâté}
\index{Faisan en pâté}
\index{Pâté de faisan}
\index{Pâté de gibier à plume}
On peut faire d'une manière analogue des pâtés de faisans, de bécasses et de
tous les gibiers à plumes de dimensions moyennes.

\sk

\index{Canard (Pâté de)}
\index{Pâté de canard}
On pourra préparer dans le même esprit un pâté de canard ; mais il conviendra,
dans ce cas, d'y ajouter du jambon qu'on'intercalera entre les morceaux de
canard.

\section*{\centering Grives à la broche.}
\phantomsection
\addcontentsline{toc}{section}{ Grives à la broche.}
\index{Grives à la broche}

Les grives les meilleures sont incontestablement celles qui se sont nourries de
genièvre et de raisin : leur chair parfumée est délicieuse.

Je n'oublierai jamais celles que j'ai mangées un jour dans l'Aveyron. C'était
au moment des vendanges, à l'heure du coucher du soleil ; au sortir du village
de Camarès, sur la route bordée de collines couvertes de genévriers et de
coteaux plantés de vignes, des grives grosses comme le poing, grasses
à souhait, les plumes hérissées, jonchaient le sol, ivres-mortes. Plumées,
flambées au cognac, bardées d'une fine bande de lard, le bec placé dans
l'estomac, rôties à la broche devant un feu vif de sarment, légèrement salées
et servies simplement sur de petites rôties arrosées avec le jus de la
lèchefrite, qu'elles étaient donc bonnes !

Mais on n'a pas tous les jours de pareilles grives à se mettre sous la dent.
Comme le parfum du genièvre et celui du raisin leur conviennent admirablement,
l'homme, malin, a eu l'idée de les leur procurer artificiellement lorsqu'elles
ne les possèdent pas naturellement.

Pour arriver à ce résultat on peut, après avoir vidé les grives, mettre dedans,
plusieurs heures avant de les faire rôtir, quelques baies de genièvre et
quelques grains de raisin frais ; ou bien faire cuire dans un bon jus des baies
de genièvre fraîches, ou sèches et détrempées dans du vin blanc, puis arroser
les grives avec ce jus pendant leur cuisson ; ou encore mêler à un bon fond de
cuisson, relevé par un peu de vin, plus ou moins de poudre de baies de
genièvre. Ce sont là des procédés classiques.

En voici un autre moins connu. Préparez les grives comme d'ordinaire,
faites-les rôtir à la broche pendant {\ppp9\mmm} à {\ppp10\mmm} minutes
seulement et achevez leur cuisson en les faisant flamber avec du gin, à raison
de {\ppp15\mmm} grammes de gin par grive.

On sert les grives seules ou sur des rôties qu'on pourra garnir avec une purée
obtenue de la façon suivante : hachez l'intérieur des grives avec du lard râpé
et des foies de volaille revenus au préalable dans du beurre, flambez à la fine
champagne, ajoutez le résidu des bardes et passez le tout au tamis.

\section*{\centering Palombes rôties.}
\phantomsection
\addcontentsline{toc}{section}{ Palombes rôties.}
\index{Palombes rôties}

Les palombes sont des pigeons sauvages, oiseaux de passage, que l'on rencontre
en France du côté des Pyrénées, notamment près de Biarritz, dans le courant du
mois de mars. C'est un excellent gibier.

Pour six personnes prenez trois palombes, plumez-les, videz-les, mettez de côté
les têtes et les intérieurs, que vous remplacerez dans chaque palombe par huit
à dix grains de raisin frais ; salez-les, bardez-les de lard, embrochez-les.
Donnez d'abord un fort coup de feu pour les bien saisir, puis vaporisez dessus
un peu de fine champagne, diminuez ensuite le feu et continuez la cuisson
pendant une vingtaine de minutes en arrosant avec un bon jus de volaille.

Entre temps, préparez des tartines en faisant griller six tranches de pain.

Retirez les cervelles des têtes, flambez-les à la fine champagne avec les
intérieurs réservés, laissez cuire pendant un moment, pilez, passez au tamis,
assaisonnez au goût avec sel et poivre ; vous aurez ainsi une purée qui vous
servira à garnir les tartines.

Disposez dans un plat les tranches de pain grillé, arrosez-les avec le jus
dégraissé de la lèchefrite, étendez dessus la purée ; coupez les palombes en
deux, dressez les demi-palombes sur les tartines et servez.

\section*{\centering Gélinotte\footnote{On donne, à tort, le nom de gélinotte
aux Lagopèdes venant de Russie qui arrivent, congelés le plus souvent, sur nos
marchés.
\protect\endgraf
Le procédé de cuisson le plus recommandable pour ce gibier est de le braiser
à la crème,}.}
\phantomsection
\addcontentsline{toc}{section}{ Gélinotte.}
\index{Gélinotte}

La gélinotte commune ou poule des bois est un oiseau sauvage de l'ordre des
Gallinacés, famille des Tétraonidés. Elle appartient au genre Ganga. Elle a la
tête huppée et les pattes emplumées jusqu'à mi-hauteur. Son plumage est
rougeâtre, taché de blanc, de gris, de roux et de noir. Elle vit en France dans
les montagnes boisées, principalement dans les Vosges.

La chair de la gélinotte est fine avec une saveur résineuse due aux bourgeons
de conifères dont elle fait la base de sa nourriture et qu'on atténue en
faisant mariner l'animal dans du lait.

On apprête la gélinotte de bien des manières : rôtie, grillée, braisée, en
salmis, en soufflés, en galantines, en pâtés, etc.

La gélinotte cuite à la broche est bardée simplement comme le perdreau, ou
farcie de grains de genièvre ; mais le rôtissage doit être réservé aux sujets
très jeunes et de qualité supérieure. Le plus souvent, la sécheresse de la
chair conduit à arroser la gélinotte avec du beurre fondu ou bien de la crème,
douce ou acidulée avec du jus de citron, pendant la cuisson qui se fait à feu
vif et qui dure de {\ppp15\mmm} à {\ppp20\mmm} minutes suivant la grosseur de
la bête.

La gélinotte qui doit être grillée est d’abord désossée, roulée ensuite dans du
beurre fondu, puis dans de la chapelure et enfin assaisonnée. On la fait
griller pendant {\ppp12\mmm} minutes et on la sert avec une sauce tartare.

La gélinotte braisée est cuite, en certains pays, dans une sauce Béchamel. On
saupoudre à la fin avec de la chapelure et on pousse au four pour gratiner.

À mon avis, la meilleure façon de préparer la gélinotte consiste à la faire
braiser dans de la crème. Voici comment on doit opérer.

\smallskip

Pour deux personnes prenez :

\footnotesize
\begin{longtable}{rrrp{16em}}
    150 & grammes & de & lait,                                                                            \\
    150 & grammes & de & crème,                                                                           \\
     30 & grammes & de & beurre,                                                                          \\
        &         &  1 & gélinotte jeune et fraîche, à chair blanche,                                     \\
        &         &  1 & barde de lard,                                                                   \\
        &         &    & jus de la moitié d'un citron,                                                    \\
        &         &    & sel et poivre.                                                                   \\
\end{longtable}
\normalsize

Videz la gélinotte, lavez l'intérieur avec un peu de lait ; jetez-le, puis
versez le reste dans la bête mise sur un plat ; retournez-la souvent ; laissez
en contact pendant une heure au moins.

Égouttez la gélinotte, troussez-la, bardez-la, mettez-la dans une casserole
avec le beurre et faites-la revenir de tous côtés pendant un quart d'heure.
Dégraissez, salez, poivrez, ajoutez la crème et laissez mijoter pendant vingt
minutes.

Cinq minutes avant la fin, mettez le jus de citron, goûtez, corsez
l’assaisonnement s'il est nécessaire,

Dressez la gélinotte sur un plat, masquez-la avec la sauce et servez.

\section*{\centering Timbale de gélinottes.}
\phantomsection
\addcontentsline{toc}{section}{ Timbale de gélinottes.}
\index{Timbale de gélinottes}
\index{Civet de gélinottes}

Pour huit personnes prenez :

\footnotesize
\begin{longtable}{rrrp{16em}}
      & 1 bouteille & de & bon vin de Bourgogne,                                                          \\
      &             &  4 & gélinottes,                                                                    \\
      &             &  4 & fines bardes de lard,                                                          \\
      &             &  1 & croûte de timbale,                                                             \\
      &             &    & champignons,                                                                   \\
      &             &    & truffes,                                                                       \\
      &             &    & madère,                                                                        \\
      &             &    & lait,                                                                          \\
      &             &    & sang de porc,                                                                  \\
      &             &    & beurre,                                                                        \\
      &             &    & farine,                                                                        \\
      &             &    & jus de citron,                                                                 \\
      &             &    & sel et poivre.                                                                 \\
\end{longtable}
\normalsize

Plumez et videz les gélinottes ; lavez-en l'intérieur avec du lait ;
enveloppez-les dans les bardes et faites-les rôtir à la broche en les gardant
très saignantes. Réservez la cuisson.

Enlevez avec des ciseaux les parties charnues du ventre, détachez les cuisses et
les ailes. Coupez chaque poitrine en deux en laissant les aiguillettes adhérentes
aux os.

Mettez ce qui reste des gélinottes dans le bourgogne, assaisonnez avec sel et
poivre, faites bouillir à petit feu pendant une demi-heure environ, puis passez
le tout à la presse. Recueillez le jus.

Faites cuire les champignons dans du beurre avec un peu de jus de citron, les
truffes dans du madère.

Maniez du beurre avec de la farine, laissez prendre couleur, mouillez ensuite
avec le jus obtenu à la presse et la cuisson réservée. Concentrez cette sauce.
Joignez-y alors les morceaux de gélinottes, les truffes, les champignons et
leurs cuissons ; chauffez sans laisser bouillir.

Au dernier moment, achevez la liaison de la sauce avec du sang de porc.

Versez le civet dans la croûte et servez.

\sk

\index{Civet de gibier en timbale}
On peut présenter ainsi, en timbale, toute espèce de ragoûts, civets et salmis.

\section*{\centering Lagopèdes.}
\phantomsection
\addcontentsline{toc}{section}{ Lagopèdes.}
\index{Lagopèdes}

Les lagopèdes sont des oiseaux sauvages gallinacés voisins des gélinottes. Ils
appartiennent à la même famille. Ils se distinguent des gélinottes par leur
tête non huppée et par leurs pattes qui sont emplumées jusqu'aux ongles. Leur
plumage est roussâtre taché de gris et, sauf pour la grouse, il change de
couleur avec les saisons et devient blanc en hiver.

La chair des lagopèdes est délicate, malgré son goût de résine.

On connaît de nombreuses espèces de lagopèdes, parmi lesquelles on peut citer
les trois suivantes : le lagopède blanc, le lagopède d'Écosse et le lagopède
des Alpes. On les accommode comme les gélinottes.

\section*{\centering Grouse.}
\phantomsection
\addcontentsline{toc}{section}{ Grouse.}
\index{Grouse}

Les Anglais désignent sous le nom de « Grouse » le lagopède d'Écosse, appelé
communément petit coq de bruyère. C'est un oiseau sauvage qui vit dans le nord
de l'Angleterre ; sa chair est très fine.

La grouse est commune en Écosse, dans les montagnes couvertes de bruyères ;
elle est bonne surtout du mois d'août au mois d'octobre.

Les sujets tout jeunes, après avoir été bardés, sont rôtis à la broche.
Cuisson : {\ppp12\mmm} à {\ppp15\mmm} minutes.

En Angleterre, on les sert avec une sorte de sauce venaison, au jus de grouse,
relevée avec du poivre, du girofle et de la cannelle, parfumée avec de l'écorce
d'orange et du porto, adoucie avec de la gelée de groseilles. Cette sauce est
connue sous le nom de sauce Victoria.

Les sujets moins jeunes sont braisés dans un bon jus additionné de vin blanc ou
dans de la crème. Dans le premier cas, on les sert avec de la choucroute et une
sauce obtenue par la concentration du jus de cuisson ; cest le procédé dit « à
la norvégienne ». Dans le second cas, le mode de cuisson est le même que celui
indiqué pour la gélinotte.

Enfin, on fait des pâtés avec les sujets plus vieux.

Dans tous les cas, on ne doit faire usage que d'oiseaux fraîchement tués et il
est bon de les laver dans du lait, à plusieurs reprises, avant de les employer.

\section*{\centering Tétras.}
\phantomsection
\addcontentsline{toc}{section}{ Tétras.}
\index{Tétras}
\index{Coq de Bruyère}
\index{Coq de Bruyère (différentes manières d'accomoder les)}

Les tétras ou coqs de bruyère sont de grands oiseaux gallinacés de la même
famille que les gélinottes et les lagopèdes. Leur taille peut atteindre celle
d’un gros dindon. Leur plumage est noir et vert sombre marqué de blanc au
ventre. On en trouve en France dans les hautes montagnes boisées. La chair des
tétras est estimée et, comme celle de la gélinotte et des lagopèdes, elle a une
saveur résineuse.

Le plus souvent, on fait cuire les tétras à la broche ; mais on peut aussi les
servir en salmis, en pâtés, en chaud-froid, etc.

\sk

Tous les oiseaux de la famille des Tétraonidés qui sont rôtis doivent être cuits
de façon à conserver à leur chair une teinte rose.

La partie la plus délicate de ces gibiers, la seule généralement servie, est la
poitrine. Le reste est employé pour faire du fumet.

\section*{\centering Bécasse rôtie.}
\phantomsection
\addcontentsline{toc}{section}{ Bécasse rôtie.}
\index{Bécasse rôtie}
\index{Bécassine rôtie}

Pour six personnes prenez :

\footnotesize
\begin{longtable}{rrrrrp{18em}}
 & \hspace{2em} &    60 & grammes & de & beurre,                                                          \\
 & \hspace{2em} &    45 & grammes & de & consommé de poulet ou de bécasse, non salé,                      \\
 & \hspace{2em} &    45 & grammes & de & vieille fine champagne ou de vieil armagnac,                     \\
 & \hspace{2em} &     6 & grammes & de & sel,                                                             \\
 & \hspace{2em} &     1 & gramme  & de & poivre fraîchement moulu,                                        \\
 & \multicolumn{3}{r}{25 centigrammes} & de & muscade\footnote{Les proportions indiquées de sel,
                         poivre et muscade sont des minima ; la plupart des amateurs en mettront
                         davantage.},                                                                     \\
 & \hspace{2em} &       &         &  3 & bécasses jeunes, bien en chair et à point\footnote{Le criterium
                         consiste dans le fait qu'on peut enlever les plumes sans effort de traction. Ce
                         point de mortification est atteint plus ou moins vite, suivant la température
                         et l'état hygrométrique de l'air. À Paris, il faut en général de quinze jours
                         à trois semaines. Le faisandage de la bécasse doit être fait la bête pendue
                         la tête en bas.},                                                                \\
 & \hspace{2em} &       &         &  3 & fines bardes de lard,                                            \\
 & \hspace{2em} &       &         &  3 & tranches rectangulaires de mie de pain boulot rassis,
                                         de 3 centimètres d'épaisseur et de dimensions telles
                                         qu'elles dépassent de 2 centimètres environ les bécasses parées. \\
\end{longtable}
\normalsize

Plumez les bécasses, flambez-les, bardez-les sans les avoir vidées,
embrochez-les et faites-les tourner devant le feu pendant cinq minutes.

Enlevez la graisse, puis versez dans la lèchefrite le consommé chauffé et
additionné de {\ppp4\mmm} grammes de sel et de la moitié du poivre : faites cuire pendant
trois minutes en arrosant ; dégraissez encore,

Mettez alors dans la lèchefrite les tranches de pain que vous aurez fait
griller au préalable : elles recevront tout ce qui s'écoulera des bécasses
à partir de ce moment. Continuez la cuisson pendant une douzaine de minutes
encore en arrosant de temps en temps afin d'éviter la dessiccation.

Débrochez les bécasses, disposez les rôties sur un plat, dressez les bécasses
sur les rôties et servez sur un réchaud.

En présence des convives, retirez soigneusement tout l'intérieur des bécasses,
rejetez gésiers et parties dures, ajoutez à ce qui reste les débris des bardes,
assaisonnez avec le reste du sel, le reste du poivre et la muscade ; mouillez
avec la fine champagne ou l’armagnac ; flambez ; mélangez bien.

Étalez le beurre sur les rôties, couvrez avec le mélange.

Découpez rapidement les bécasses, disposez les morceaux sur les rôties et servez.

L'emploi du réchaud permet de faire toutes ces opérations sur la table, sans
que le plat se refroidisse ; c est fondamental.

Les rôties diffèrent de la tartine dont je parlerai à propos du faisan farci ;
leur épaisseur est moindre ; de plus, le pain est grillé de façon à lui assurer
la solidité nécessaire pour résister à une imbibition d'une dizaine de minutes
dans la lèchefrite. Du reste, on peut également servir sur des tartines les
bécasses rôties.

\sk

Certains amateurs aiment mieux ne pas flamber les intérieurs : ils se bornent
à arroser les rôties garnies avec un peu de fine champagne ou d'armagnac.
D'autres n'admettent pas les résidus de bardes dans le garnissage des rôties ou
des tartines ; ils préfèrent même faire cuire les bécasses sans les barder, en
les graissant simplement pendant leur cuisson juste assez pour les empêcher de
brûler. Ce procédé ne peut être satisfaisant que si l'on a des bécasses très
grasses.

\medskip

Un vin rouge de Bourgogne, vieux et généreux, s'impose avec ce gibier.

Je ne connais rien de plus savoureux qu'une jeune bécasse à point et grasse
à souhait, cuite à la broche devant un feu de sarment, présentée sur une tartine
ou une rôtie soignée et accompagnée d'une romanée, d'un clos Vougeot, d'un
musigny, d'un chambertin ou d'un corton de derrière les fagots.

\section*{\centering Bécasse farcie de foie gras truffé, en cocote.}
\phantomsection
\addcontentsline{toc}{section}{ Bécasse farcie de foie gras truffé, en cocote.}
\index{Bécasse farcie de foie gras truffé, en cocote}

Pour dix personnes prenez :

\footnotesize
\begin{longtable}{rrrp{16em}}
    150 & grammes & de & légumes de pot-au-feu : carotte, navet, panais, etc.,                            \\
    100 & grammes & de & porto blanc,                                                                     \\
    100 & grammes & de & chablis,                                                                         \\
     50 & grammes & de & vieille fine champagne,                                                          \\
     50 & grammes & de & beurre,                                                                          \\
     30 & grammes & de & cognac,                                                                          \\
     30 & grammes & de & glace de viande fine,                                                            \\
        & 1 litre & de & fond de veau,                                                                    \\
        &         &  2 & abatis de poulets,                                                               \\
        &         &  1 & bécasse,                                                                         \\
        &         &  1 & foie gras d'oie,                                                                 \\
        &         &  1 & barde de lard,                                                                   \\
        &         &    & madère,                                                                          \\
        &         &    & truffes,                                                                         \\
        &         &    & sel, poivre,                                                                     \\
        &         &    & quatre épices,                                                                   \\
        &         &    & cayenne.                                                                         \\
\end{longtable}
\normalsize

\index{Farce pour bécasse}
Faites revenir dans le beurre les légumes et les abatis ; laissez pincer un peu
les légumes ; mouillez avec le fond de veau, le porto, le chablis, le cognac ;
ajoutez la glace de viande, un peu de quatre épices et laissez mijoter
doucement en dépouillant fréquemment pendant la cuisson. Réduisez au volume
d'un litre environ ; passez.

Pendant ce temps, désossez la bécasse, hachez fin l'intérieur moins le gésier,
passez les déchets à la presse, recueillez le jus et amalgamez-le avec le
hachis de façon à faire une petite farce fine ; assaisonnez-la avec sel, poivre
et cayenne ; mouillez avec {\ppp20\mmm} grammes de fine champagne.

Assaisonnez la bécasse avec sel, poivre, cayenne ; mettez-la à mariner pendant
douze heures dans du madère.

Ouvrez le foie gras, enlevez-en les parties nerveuses, assaisonnez-le
légèrement avec sel, poivre et quatre épices, truffez-le et mettez-le à mariner
pendant douze heures dans du madère.

Sortez les éléments de leurs marinades.

Étalez la bécasse ouverte sur la barde. Garnisses-en l'intérieur avec la farce
préparée et le foie gras, roulez et ficelez.

Mettez la bécasse ainsi apprêtée dans une cocote en porcelaine, arrosez-la avec
le reste de la fine champagne et le madère des marinades. Placez la cocote dans
un grand bain-marie froid ; poussez au four doux, amenez lentement
à ébullition, puis, à feu plus vif, continuez la cuisson, qui doit durer une
demi-heure à trois quarts d'heure.

Enlevez la ficelle, laissez bien refroidir.

Servez dans la cocote.

Toutes les personnes auxquelles j'ai fait goûter ce mets m'ont déclaré n'avoir
jamais rien mangé de plus fin.

\medskip

Le mode de cuisson en cocote au bain-marie, au four, est un procédé général.

Avec de petits gibiers, tels que grives, cailles, etc., on pourra préparer une
petite cocote par convive ; le mets sera ainsi présenté très élégamment.

\sk

En variant les marinades, les fonds de cuisson, les farces, on obtiendra une
série de plats froids exquis, en gelée.

\sk

On pourra dans le même esprit préparer des plats froids de poissons, mais alors
on prendra pour la cuisson un fond de poisson.

\section*{\centering Fricassée de bécasses.}
\phantomsection
\addcontentsline{toc}{section}{ Fricassée de bécasses.}
\index{Fricassée de bécasses}
\index{Bécasses en fricassée}

Pour quatre personnes prenez :

\footnotesize
\begin{longtable}{rrrp{16em}}
    700 & grammes & de & vin de Bourgogne rouge ou blanc,                                                 \\
    200 & grammes & de & lard salé, non fumé,                                                             \\
    100 & grammes & de & beurre,                                                                          \\
     50 & grammes & d' & oignons,                                                                         \\
     30 & grammes & d' & échalotes,                                                                       \\
     30 & grammes & de & farine,                                                                          \\
     20 & grammes & d’ & ail,                                                                             \\
        &         &  2 & bécasses,                                                                        \\
        &         &  8 & grains de genièvre,                                                              \\
        &         &  2 & clous de girofle,                                                                \\
        &         &  1 & bouquet garni,                                                                   \\
        &         &    & huile d'olive,                                                                   \\
        &         &    & sel et poivre.                                                                   \\
\end{longtable}
\normalsize

Plumez, flambez, videz les bécasses ; découpez-les ; réservez les intérieurs
moins les gésiers.

Faites revenir dans un peu de beurre le lard coupé en morceaux, les échalotes
et les oignons ciselés.

Faites un roux avec le reste du beurre et la farine ; mouillez avec le vin ;
ajoutez les substances revenues, les bécasses, l'ail, le bouquet garni, les
clous de girofle et les grains de genièvre écrasés, poivrez ; laissez cuire
pendant une demi- heure,

Retirez les bécasses, tenez-les au chaud ; passez la cuisson.

Pilez les intérieurs avec un peu d'huile d'olive, ajoutez-les à la cuisson
passée, mélangez bien ; chauffez jusqu'à ce que la sauce ait une consistance
convenable pour masquer une cuiller ; goûtez, complétez l'assaisonnement, s'il
est nécessaire, avec sel et poivre ; dégraissez.

Dressez les morceaux de bécasses sur un plat, versez dessus la sauce et servez.

\section*{\centering Salmis de bécasses.}
\phantomsection
\addcontentsline{toc}{section}{ Salmis de bécasses.}
\label{pg0630} \hypertarget{p0630}{}
\index{Salmis de bécasses}
\index{Bécasses en salmis}
\index{Faisan en salmis}
\index{Canard sauvage en salmis}
\index{Coqs de bruyères en salmis}
\index{Faisan (Salmis de)}
\index{Bécassines en salmis}

D'une façon générale, tous les salmis peuvent être apprêtés de plusieurs
manières. Ils ont un goût différent suivant les assaisonnements et les jus de
cuisson qui sont employés dans leur préparation ; ils varient aussi par la
nature de leurs garnitures.

Je vais donner, pour fixer les idées, trois formules de salmis de
bécasses\footnote{Ces formules sont applicables aux bécassines, canards
sauvages, coqs de bruyère, faisans, gélinottes, grives, perdrix, sarcelles,
etc. Seules, la bécasse et la bécassine ne sont pas vidées.}.

\medskip

La première, la plus simple, est celle d’un salmis au jus de viande ordinaire
et au vin blanc, sans autre garniture que des croûtons frits. La deuxième est
celle d'un salmis dont la cuisson est faite sans vin, dans du consommé de
bécasse ou, à défaut, dans du consommé de poularde. La sauce est constituée par
le jus de cuisson lié avec un coulis de bécasses en vue de conserver au plat le
goût de bécasse dans toute sa pureté. La garniture est faite de champignons de
couche ou de morilles. La troisième est celle d'un salmis cuit dans un fond
identique à celui de la deuxième formule, corsé avec du madère et du fumet de
gibier. La garniture, plus riche, est faite de foie gras et de truffes.

\begin{center}
\small\sc{première formule}
\end{center}

Pour douze personnes prenez :

\footnotesize
\begin{longtable}{rrrp{16em}}
    750 & grammes & de & jus de viande,                                                                   \\
    220 & grammes & de & beurre,                                                                          \\
    100 & grammes & de & glace de viande,                                                                 \\
     80 & grammes & de & carottes (le rouge seulement),                                                   \\
     60 & grammes & de & fine champagne,                                                                  \\
     25 & grammes & de & farine,                                                                          \\
     20 & grammes & d' & oignon,                                                                          \\
     20 & grammes & d' & échalotes,                                                                       \\
    1 & bouteille & de & vin de Bordeaux blanc,                                                           \\
        &         & 12 & tartines rondes de mie de pain boulot rassis, de 5 centimètres
                         de diamètre et de 1 centimètre et demi d'épaisseur,                              \\
        &         &  4 & bécasses bien à point,                                                           \\
        &         &  4 & bardes fines de lard,                                                            \\
        &         &  1 & bouquet garni comprenant 30 grammes de persil,
                         2 grammes de thym, 2 gram\-mes de laurier,                                       \\
        &         &    & sel, poivre, mignonnette.                                                        \\
\end{longtable}
\normalsize

Faites blondir la farine dans {\ppp30\mmm} grammes de beurre, ajoutez les
carottes, les échalotes, l'oignon, laissez pincer, puis mettez le bouquet
garni, mouillez avec le vin et le jus de viande. Laissez cuire longuement en
dépouillant pendant la cuisson ; concentrez et amenez la sauce à être
onctueuse.

Plumez, flambez, bardez les bécasses ; faites-les rôtir à la broche en les
tenant saignantes, mais sans excès.

Débrochez les bécasses, mettez de côté ce qui reste des bardes ; dégraissez le
jus de cuisson, réservez-le.

Détachez les membres des bécasses, coupez les poitrines en deux dans le sens de
leur longueur, parez-les, enlevez la peau. Tenez, à chaleur douce, les morceaux
de bécasses dans un peu de fine champagne flambée mélangée avec un peu de glace
de viande fondue, le tout mis dans une sauteuse couverte.

Retirez les têtes et les intérieurs des bécasses, jetez les gésiers, réservez
le reste.

Pilez les carcasses et les débris des bécasses, mettez-les dans la sauce,
ajoutez le reste de la glace de viande et le jus de cuisson réservé, un peu de
mignonnette, laissez cuire pendant un quart d'heure environ, puis passez au
tamis en pressant.

Remettez la sauce sur le feu, réduisez-la encore en la dépouillant, goûtez,
corsez l’assaisonnement avec sel et poivre s'il est nécessaire, ajoutez un peu
de beurre ; passez la sauce sur les morceaux de bécasses.

Faites dorer les tartines dans le reste du beurre.

En même temps, flambez les intérieurs des bécasses avec le reste de la fine
champagne, ajoutez les résidus des bardes, hachez, assaisonnez au goût, puis
étalez le mélange sur les tartines.

Dressez les morceaux de bécasses au milieu d'un plat, entourez-les avec les
tartines, masquez avec la sauce, décorez avec les têtes des bécasses et servez.

\begin{center}
\small\sc{deuxième formule}
\end{center}

Pour douze personnes prenez :

\footnotesize
\begin{longtable}{rrrp{16em}}
    250 & grammes & de & champignons de couche ou de morilles,                                            \\
    220 & grammes & de & beurre,                                                                          \\
     80 & grammes & de & carottes (le rouge seulement),                                                   \\
     60 & grammes & de & fine champagne,                                                                  \\
     20 & grammes & d' & oignon,                                                                          \\
     20 & grammes & d' & échalotes,                                                                       \\
     15 & grammes & de & farine,                                                                          \\
      3 & grammes & de & baies de genièvre,                                                               \\
        & 1 litre & de & consommé de bécasse ou, à défaut, de consommé de poularde,                       \\
        &         & 12 & tartines rondes de mie de pain boulot rassis de
                         5 centi\-mètres de diamètre et de 1 centi\-mètre et demi d'épais\-seur,          \\
        &         &  6 & bécasses bien à point,                                                           \\
        &         &  6 & bardes fines de lard,                                                            \\
        &         &  1 & bouquet garni composé de 30 gram\-mes de persil,
                         2 gram\-mes de thym, 2 gram\-mes de lau\-rier,                                   \\
        &         &    & jus de citron,                                                                   \\
        &         &    & cayenne,                                                                         \\
        &         &    & sel, poivre.                                                                     \\
\end{longtable}
\normalsize

Faites blondir la farine dans {\ppp30\mmm} grammes de beurre, mettez ensuite
carottes, oignon, échalotes coupés en morceaux, laissez pincer un peu, mouillez
avec le consommé de bécasse ou de poularde, ajoutez le bouquet garni, les baies
de genièvre écrasées, amenez à ébullition, puis laissez cuire pendant une heure
et demie en dépouillant pendant la cuisson. Réduisez suffisamment ce fond.

Apprêtez, bardez et faites rôtir les bécasses comme dans la première formule ;
dégraissez le jus de cuisson, réservez-le ainsi que les résidus des bardes.

Détachez les ailes et les pattes des bécasses, enlevez-en la peau et tenez-les au
chaud en sauteuse couverte dans un peu de fine champagne brûlée et additionnée
d'un peu de consommé de bécasse.

Réservez les têtes et les intérieurs, moins les gésiers.

Préparez le coulis. Enlevez des carcasses tout ce qui reste de chair, pilez-la
dans un mortier, passez-la au tamis, ajoutez-y le jus de cuisson des bécasses.
Pilez les carcasses et les débris. mettez-les dans le fond de bécasse ou de
poularde, laissez cuire encore pendant un quart d'heure ; passez au tamis en
pressant,

Liez cette sauce avec le coulis de bécasse, beurrez-la légèrement,
concentrez-la encore. Goûtez, corsez l'assaisonnement avec sel, poivre, cayenne
et jus de citron. Versez la sauce sur les morceaux de bécasses.

Faites dorer les tartines dans le reste du beurre.

Flambez les intérieurs des bécasses avec le reste de la fine champagne, ajoutez
les résidus des bardes, hachez, mélangez, assaisonnez au goût, puis étalez le
tout sur les tartines.

Dressez les membres des bécasses dans un plat, disposez autour les tartines et
les champignons de couche ou les morilles cuits à part dans un peu de fond,
masquez avec la sauce, décorez avec les têtes des bécasses et servez.

\begin{center}
\small\sc{troisième formule}
\end{center}

Pour douze personnes prenez :

\smallskip

\footnotesize
\begin{longtable}{rrrp{16em}}
    300 & grammes & de & foie gras de canard,                                                             \\
    220 & grammes & de & beurre,                                                                          \\
    200 & grammes & de & madère,                                                                          \\
    125 & grammes & de & glace de gibier à plumes,                                                        \\
    125 & grammes & de & jambon salé, non fumé,                                                           \\
     80 & grammes & de & carottes (le rouge seulement),                                                   \\
     60 & grammes & de & fine champagne,                                                                  \\
     60 & grammes & de & champignons,                                                                     \\
     20 & grammes & d' & échalotes,                                                                       \\
     20 & grammes & d' & oignon,                                                                          \\
     15 & grammes & de & farine,                                                                          \\
        & 1 litre & de & consommé de bécasse ou, à défaut, de consommé de poularde,                       \\
        &         & 12 & tar\-tines rondes de mie de pain bou\-lot ras\-sis,
                         de 5 cen\-ti\-mètres de dia\-mètre et de 1 cen\-ti\-mètre
                         et demi d'é\-pais\-seur,                                                         \\
        &         &  6 & bécasses à point,                                                                \\
        &         &  6 & bardes fines de lard,                                                            \\
        &         &  1 & bouquet garni composé de 30 gram\-mes de persil,
                         2 gram\-mes de thym, 2 gram\-mes de lau\-rier,
                         truf\-fes noires du P\-é\-ri\-gord, à vo\-lonté,                                 \\
        &         &    & huile d'olive fine,                                                              \\
        &         &    & sel et poivre.                                                                   \\
\end{longtable}
\normalsize

Faites revenir dans {\ppp30\mmm} grammes de beurre le jambon, les carottes,
l'oignon, les échalotes et les champignons coupés en morceaux, saupoudrez avec
la farine. laissez pincer un peu ; mouillez avec le consommé de bécasse ou de
poularde, ajoutez le bouquet. Laissez cuire pendant une heure et demie en
dépouillant pendant la cuisson. Réduisez suffisamment ce fond.

Brossez, lavez, essuyez les truffes ; pelez-les et faites-les cuire dans le
madère avec sel et poivre, au goût.

Apprêtez, bardez et faites rôtir les bécasses comme dans les formules
précédentes. Dégraissez le jus de cuisson ; réservez-le ainsi que les résidus
des bardes.

Détachez des bécasses les ailes et les pattes, enlevez-en la peau ; tenez-les
au chaud en sauteuse couverte dans quelques cuillerées de consommé de bécasse
ou de poularde additionné d'un peu de fine champagne brûlée.

Enlevez des carcasses tout ce qui reste de chair, pilez-la au mortier,
passez-la au tamis, ajoutez-y le jus de cuisson des bécasses. Réservez les
têtes et les intérieurs, moins les gésiers.

Pilez les carcasses et les débris, mettez-les dans le fond de bécasse ou de
poularde, laissez cuire pendant un quart d'heure ; passez au tamis en pressant.

Liez cette sauce avec le coulis de bécasses, beurrez légèrement, ajoutez les
pelures de truffes hachées, le madère de cuisson des truffes, corsez avec le
fumet de gibier ; concentrez encore pendant un moment en dépouillant. Goûtez,
ajoutez sel et poivre s'il est nécessaire. Versez la sauce sur les membres des
bécasses.

Faites dorer les tartines dans le reste du beurre.

En même temps, préparez le foie de canard au naturel : versez dans une sauteuse
quelques gouttes d'huile d'olive, chauffez, mettez le foie gras de canard salé
légèrement ; faites cuire à feu vif en secouant constamment la sauteuse afin
d'éviter qu'il s'attache au fond. La cuisson doit durer une douzaine de minutes
environ.

Coupez le foie en douze tranches de {\ppp7\mmm} à {\ppp8\mmm} millimètres
d'épaisseur, parez-les, placez-les sur les tartines ; tenez au chaud.

Flambez les intérieurs des bécasses avec le reste de la fine champagne, ajoutez
les résidus des bardes et les débris du foie gras, hachez, assaisonnez au goût,
mélangez.

Étalez ce mélange sur les tartines garnies, couronnez chacune d'une rondelle de
truffe.

Dressez les membres des bécasses dans un plat, masquez-les avec la sauce,
disposez autour les tartines garnies, décorez avec les truffes et les têtes des
bécasses ; servez.

\begin{center}
\small
Gastronomes des temps passés, vous qui ignorâtes ce salmis,

« Que je vous plains ! »\footnote{Voir : \textit{Élégie historique}, in
« Physiologie du Goût », par Brillat-Savarin.}

\end{center}

\section*{\centering Croustades de purée de bécasses.}
\phantomsection
\addcontentsline{toc}{section}{ Croustades de purée de bécasses.}
\index{Croustades de purée de bécasses}
\label{pg0634} \hypertarget{p0634}{}

Pour douze personnes prenez :

\footnotesize
\begin{longtable}{rrrp{16em}}
    750 & grammes & de & vin de Champagne demi-sec,                                                       \\
    500 & grammes & de & truffes,                                                                         \\
     50 & grammes & de & fine champagne,                                                                  \\
        &         & 12 & tranches de mie de pain boulot rassis de 10 cen\-timè\-tres
                         de côté et de 5 cen\-timè\-tres d'épai\-sseur,                                   \\
        &         &  5 & bécasses,                                                                        \\
        &         &  5 & bardes de lard,                                                                  \\
        &         &  1 & foie gras de canard,                                                             \\
        &         &  1 & bouquet garni (persil, laurier, thym),                                           \\
        &         &    & beurre,                                                                          \\
        &         &    & girofle,                                                                         \\
        &         &    & sel et poivre.                                                                   \\
\end{longtable}
\normalsize

Plumez, flambez les bécasses.

Brossez, lavez {\ppp400\mmm} grammes de truffes, essuyez-les,  puis hachez‑les
avec le foie gras de canard.

Farcissez les bécasses avec ce mélange ; laissez en contact pendant trois
jours.

Le moment venu, bardez les bécasses, faites-les revenir dans du beurre pendant
un quart d'heure, assaisonnez-les avec sel, poivre et girofle au goût, puis
flambez-les avec la fine champagne. Mouillez ensuite avec le vin de Champagne,
ajoutez le bouquet ; laissez cuire à petit feu pendant trois quarts d'heure en
arrosant de temps en temps.

Nettoyez le reste des truffes, faites-les cuire dans du madère ; coupez-les en
rondelles.

Préparez la purée : prenez la chair et l'intérieur des bécasses, pilez-les avec
ce qui reste des bardes, passez le tout au tamis ; réservez.

Écrasez à la presse tous les débris des bécasses, os, têtes, etc., recueillez le jus,
ajoutez-le à la purée réservée, mouillez avec la cuisson dégraissée et passée, goûtez,
corsez l'assaisonnement s'il est nécessaire ; tenez au chaud au bain-marie.

Préparez les croustades : creusez les tranches de mie de pain de deux
centimètres de profondeur, en réservant un bord d'un centimètre sur le
pourtour ; faites-les dorer de tous côtés dans du beurre, de façon à leur
donner une bonne consistance ; emplissez les creux avec la purée bien chaude et
décorez le dessus avec les rondelles de truffes.

Dressez les croustades sur un plat garni d'une serviette et servez.

\section*{\centering Faisan rôti, à la choucroute.}
\phantomsection
\addcontentsline{toc}{section}{ Faisan rôti, à la choucroute.}
\index{Faisan rôti, à la choucroute}

Pour cinq ou six personnes prenez :

\footnotesize
\begin{longtable}{rrrp{16em}}
  1 000 & grammes & de & choucroute,                                                                      \\
  1 000 & grammes & de & fond de veau,                                                                    \\
    500 & grammes & de & vin blanc,                                                                       \\
    125 & grammes & de & beurre,                                                                          \\
     50 & grammes & de & graisse d'oie,                                                                   \\
        &         & 24 & grains de genièvre,                                                              \\
        &         &  2 & faisans, dont un jeune, au moins,                                                \\
        &         &  1 & barde de lard,                                                                   \\
        &         &    & sel et poivre.                                                                   \\
\end{longtable}
\normalsize

Lavez la choucroute à l’eau froide ou à l'eau chaude, ou même ébouillantez-la
pendant quelques minutes suivant qu'elle est plus ou moins salée ;
rafraîchissez-la, égouttez-la.

Mettez, dans une marmite en porcelaine allant au feu, choucroute, beurre,
grains de genièvre, assaisonnez avec du poivre, chauffez ; mouillez ensuite
avec le vin blanc et la moitié du fond de veau ; faites bouillir, puis laissez
cuire doucement pendant une heure et demie environ, de manière qu'il ne reste
plus de liquide. Tenez au chaud.

Faites revenir le faisan le plus vieux dans la graisse d’oie. Lorsqu'il est bien doré
de tous les côtés, mettez-le dans la marmite contenant la choucroute, mouillez avec
le reste du fond de veau et continuez la cuisson à petit feu pendant trois heures,

Enlevez alors le faisan, désossez-le, passez la chair au tamis, mélangez-la
à la choucroute et achevez la cuisson de l’ensemble qui doit durer encore
{\ppp25\mmm} minutes,

Bardez le second faisan et faites-le rôtir à la broche.

Goûtez la choucroute, complétez son assaisonnement s'il y a lieu avec sel et
poivre ; disposez-la sur un plat chaud, dressez dessus le faisan rôti et
servez.

Ce plat est tout à fait confortable ; la choucroute bien cuite, très
aromatisée, est excellente ; elle accompagne on ne peut mieux le faisan rôti.

\section*{\centering Faisan braisé à la crème.}
\phantomsection
\addcontentsline{toc}{section}{ Faisan braisé à la crème.}
\index{Faisan braisé à la crème}

Pour quatre personnes prenez :

\footnotesize
\begin{longtable}{rrrp{16em}}
    400 & grammes & de & crème,                                                                           \\
    100 & grammes & de & fond de veau, de volaille ou de gibier,                                          \\
     30 & grammes & de & cognac,                                                                          \\
     30 & grammes & de & vinaigre,                                                                        \\
     30 & grammes & de & raifort râpé,                                                                    \\
     25 & grammes & de & beurre,                                                                          \\
     25 & grammes & d' & échalotes,                                                                       \\
        &         &  1 & faisan,                                                                          \\
        &         &  1 & barde de lard,                                                                   \\
        &         &    & sel et poivre.                                                                   \\
\end{longtable}
\normalsize

Plumez le faisan, videz-le, flambez-le, bardez-le,

Faites-le revenir dans une cocote avec le beurre et les échalotes, flambez-le
au cognac, salez, poivrez ; mouillez avec le fond et faites cuire au four
pendant une demi-heure en arrosant avec la cuisson ; ajoutez alors la crème, le
vinaigre et le raifort et laissez cuire encore pendant un quart d'heure à vingt
minutes, suivant la grosseur de l'animal, en continuant à arroser.

Un peu avant la fin goûtez pour l'assaisonnement, dressez sur un plat et
envoyez en même temps du riz sec ou du riz sauté au beurre, des petits pois ou
des épinards,

\section*{\centering Faisan farci, braisé au porto.}
\phantomsection
\addcontentsline{toc}{section}{ Faisan farci, braisé au porto.}
\label{pg0636} \hypertarget{p0636}{}
\index{Faisan farci, braisé au porto}

Pour six à huit personnes prenez :

\footnotesize
\begin{longtable}{rrrp{16em}}
    600 & grammes & de & vin de Porto rouge,                                                              \\
    200 & grammes & de & belles truffes noires du Périgord (au minimum),                                  \\
    200 & grammes & de & foie gras de canard,                                                             \\
     60 & grammes & de & fine champagne,                                                                  \\
        &         &  1 & beau faisan jeune, gras et à point,                                              \\
        &         &  1 & bécasse bien en chair,                                                           \\
        &         &  1 & barde de lard,                                                                   \\
        &         &  1 & tranche de mie de pain boulot de 5 centimètres
                         d'épaisseur et de dimensions suffisantes pour que
                         le faisan soit à l'aise dessus,                                                  \\
        &         &    & beurre,                                                                          \\
        &         &    & sel et poivre.                                                                   \\
\end{longtable}
\normalsize

Videz le faisan, flambez-le, réservez l'intérieur après l'avoir nettoyé.

Désossez la bécasse, mettez l'intérieur à part, tel quel.

Brossez les truffes, lavez-les, pelez-les, faites-les cuire dans un peu de porto ;
réservez les pelures.

Hachez fin la chair de bécasse, le foie gras, les pelures de truffes ;
mélangez.

\index{Farce pour faisan}
Préparez une farce homogène avec ce hachis, une partie des truffes coupées en
morceaux, du sel, du poivre, le porto qui a servi à la cuisson des truffes et
{\ppp20\mmm} grammes de fine champagne ; mettez-la dans le faisan.

Coupez une belle truffe en rondelles, glissez-les sous la peau de l'animal,
bardez-le et laissez-le s'imprégner pendant trois jours des parfums combinés de
la bécasse, des truffes, du foie gras, du porto et de la fine champagne.

Faites alors revenir le faisan dans une casserole pendant un quart d'heure avec
plus ou moins de beurre, en proportion inverse de celle de la graisse qu'il
contient, flambez-le avec {\ppp20\mmm} grammes de fine champagne, puis
laissez-le cuire, à petit feu, pendant une heure à une heure un quart, en
l'arrosant par intervalles avec le reste du porto, sans autre assaisonnement.

En même temps, préparez la tartine.

Coupez la tranche de mie de pain boulot en six ou huit morceaux que vous ferez
dorer de tous côtés dans du beurre de façon à leur donner une consistance
convenable, disposez ces morceaux sur un plat en reconstituant la tartine ;
tenez au chaud.

Faites cuire rapidement, à feu vif, les intérieurs du faisan et de la bécasse,
flambez-les avec le reste de la fine champagne, salez, poivrez.

Versez sur la tartine le jus dégraissé de la cuisson, étendez dessus les
intérieurs, dressez le faisan sur le tout, décorez avec le reste des truffes et
servez.

Une tartine bien faite doit être légèrement croustillante à la surface et
moelleuse à l'intérieur.

Le faisan ainsi préparé est un mets de haut goût, parfumé, fondant ; il me
parait difficile de manger quelque chose de meilleur dans le genre.

\section*{\centering Bouchées de faisan farcies, sauce Périgueux.}
\phantomsection
\addcontentsline{toc}{section}{ Bouchées de faisan farcies, sauce Périgueux.}
\index{Bouchées de faisan farcies, sauce Périgueux}
\index{Faisan (Bouchées de)}

Pour dix personnes prenez :

\medskip

\index{Enveloppe pour bouchées de faisan}
\index{Bouchées (Enveloppes pour)}
1° pour l'enveloppe :

\footnotesize
\begin{longtable}{rrrp{16em}}
    500 & grammes & de & fond de gibier concentré,                                                        \\
    300 & grammes & de & chair de faisan,                                                                 \\
    200 & grammes & de & farine,                                                                          \\
    200 & grammes & de & graisse de rognon de veau,                                                       \\
    100 & grammes & de & graisse d'oie,                                                                   \\
        &         &  4 & œufs frais entiers,                                                              \\
        &         &  2 & blancs d'œufs.                                                                   \\
        &         &  1 & jaune d'œuf,                                                                     \\
        &         &    & épices,                                                                          \\
        &         &    & sel et poivre ;                                                                  \\
\end{longtable}
\normalsize

\index{Farce pour bouchées de faisan}
2° pour la farce :

\footnotesize
\begin{longtable}{rrrp{16em}}
    300 & grammes & de & salmis de bécasse serré, passé au tamis ;                                        \\
\end{longtable}
\normalsize

3° pour la sauce :

\footnotesize
\begin{longtable}{rrrp{16em}}
    200 & grammes & de & jambon de Bayonne,                                                               \\
    200 & grammes & de & bon jus,                                                                         \\
    100 & grammes & de & madère,                                                                          \\
     25 & grammes & de & beurre,                                                                          \\
     15 & grammes & de & farine,                                                                          \\
        &         &  1 & petit oignon coupé en rouelles,                                                  \\
        &         &  1 & petite échalote,                                                                 \\
        &         &    & truffes à volonté,                                                               \\
        &         &    & sel et poivre.                                                                   \\
\end{longtable}
\normalsize

Triturez ensemble la farine, les quatre œufs entiers, ajoutez en tournant le
fond de gibier chaud par petites quantités, puis faites prendre l'appareil sur
le feu en le travaillant pendant une demi-heure environ, de façon à l'amener
à bonne consistance.

Pilez séparément la chair de faisan et la graisse de rognon, réunissez-les,
assaisonnez avec sel, poivre et épices, pilez encore, ajoutez la graisse d'oie,
pilez le tout ensemble, incorporez l'appareil ci-dessus et amalgamez au mélange
les blancs d'œufs battus en neige.

Passez le tout au tamis, travaillez bien de manière à avoir une pâte homogène
et lisse, dont vous ferez une abaisse, de {\ppp5\mmm} à {\ppp6\mmm} millimètres
d'épaisseur, que vous partagerez en vingt morceaux carrés.

Mettez sur chaque morceau de pâte un vingtième de la farce ; fermez les
bouchées.

Préparez la sauce.

Faites cuire les truffes dans le madère ; hachez-les ; réservez le madère de
cuisson.

Faites un roux avec le beurre et la farine, ajoutez l'oignon et l'échalote ;
laissez dorer légèrement ; puis mettez le jambon coupé en petits morceaux ;
mouillez ensuite avec le jus et le madère de cuisson des truffes réservé ;
laissez cuire ; dépouillez la sauce, concentrez-la, passez-la. Goûtez, ajoutez
sel et poivre s'il est nécessaire.

Au moment de servir, mettez le hachis de truffes.

Faites pocher les bouchées dans de l’eau salée bouillante, égouttez-les,
dorez-les au jaune d'œuf et passez-les au four.

Dressez les bouchées sur un plat garni d'une serviette, et servez en envoyant
en même temps la sauce dans une saucière.

\section*{\centering Pâté au salmis de faisan.}
\phantomsection
\addcontentsline{toc}{section}{ Pâté au salmis de faisan.}
\index{Pâté au salmis de faisan}
\index{Croûte pour pâtés}
\index{Garniture pour pâtés}

Pour douze personnes prenez :

\medskip

1° pour la pâte :

\footnotesize
\begin{longtable}{rrrp{16em}}
    500 & grammes & de & farine,                                                                          \\
    250 & grammes & d' & eau tiède,                                                                       \\
    200 & grammes & de & beurre,                                                                          \\
     30 & grammes & d’'&  huile d'olive,                                                                  \\
     20 & grammes & de & sel,                                                                             \\
        &         &  4 & jaunes d'œufs ;                                                                  \\
\end{longtable}
\normalsize

2° pour le corps du pâté :

\footnotesize
\begin{longtable}{rrrp{16em}}
    500 & grammes & de & vin blanc sec de Bordeaux,                                                       \\
    400 & grammes & de & fond de veau et de volaille,                                                     \\
    350 & grammes & de & jambon de Bayonne,                                                               \\
    250 & grammes & de & sous-noix de veau,                                                               \\
    250 & grammes & de & champignons de couche,                                                           \\
    200 & grammes & de & truffes,                                                                         \\
    150 & grammes & de & fumet de gibier,                                                                 \\
     75 & grammes & de & beurre,                                                                          \\
     50 & grammes & de & cognac flambé,                                                                   \\
     40 & grammes & d' & échalotes,                                                                       \\
        &         & 3  & faisans,                                                                         \\
        &         & 3  & foies de poulardes,                                                              \\
        &         &    & bardes de lard,                                                                  \\
        &         &    & baies de genièvre,                                                               \\
        &         &    & jus de citron,                                                                   \\
        &         &    & thym,                                                                            \\
        &         &    & laurier,                                                                         \\
        &         &    & sel et poivre ;                                                                  \\
\end{longtable}
\normalsize

3° pour la gelée :

\footnotesize
\begin{longtable}{rrrp{16em}}
    400 & grammes & de & gelée de veau et de volaille,                                                    \\
    100 & grammes & de & vin blanc sec de Bordeaux,                                                       \\
     80 & grammes & de & fumet de gibier.                                                                 \\
\end{longtable}
\normalsize

Plumez, videz. flambez, troussez les faisans ; réservez les foies.

\medskip

\textit{Préparation de la pâte}. — Avec les éléments indiqués, préparez une
pâte homogène ; laissez-la reposer pendant quelques heures.

\medskip

\textit{Préparation de la farce}. — Faites blondir les échalotes dans un peu de
beurre, mouillez avec le vin blanc, ajoutez baies de genièvre, thym, laurier,
au goût, laissez cuire, concentrez au volume de quelques cuillerées ; passez ;
réservez.

Faites cuire les truffes dans le cognac ; émincez-les.

Passez le veau et le jambon pendant quelques minutes dans un peu de beurre ;
hachez-les.

Pelez les champignons, passez-les dans du jus de citron, hachez-les fin.

Embrochez les faisans, faites-les rôtir à moitié, puis détachez vivement les
beaux morceaux de chair, enlevez-en la peau ; tenez-les au chaud à chaleur
douce, en casserole couverte, dans quelques cuillerées de fond de veau et de
volaille additionné d'un peu de cognac de cuisson des truffes.

Enlevez le reste de la chair des carcasses des faisans ; réservez-la.

Pilez les carcasses et tous les déchets des faisans ; mettez-les dans une
casserole avec la réduction de vin aromatisé, le fond de veau et de volaille,
le fumet de gibier, le reste du cognac, du sel, du poivre ; laissez cuire ;
concentrez fortement, puis passez en pressant, d'abord au tamis ordinaire, puis
au tamis de soie ; réservez cette purée de salmis.

Faites revenir les foies de poulardes et les foies de faisans dans le reste du
beurre.

Pilez ensemble les foies de poulardes et de faisans, la chair de faisan
réservée, le jambon, la sous-noix de veau ; passez au tamis ; ajoutez les
champignons et la purée de salmis, mélangez ; goûtez pour l’assaisonnement et
complétez-le, s'il est nécessaire, avec sel et poivre.

\medskip

\textit{Dressage du pâté}. — Abaissez la pâte: réservez-en une partie pour les
couvercles et les bouchons des cheminées.

Chemisez un moule de section rectangulaire, à faces planes, avec cette abaisse,
mettez au fond une couche de farce, au-dessus une couche de beaux morceaux de
faisan que vous aurez enveloppés de fines bardes de lard, intercalez des
émincés de truffe, couvrez avec de la farce ; placez ensuite une nouvelle
couche de morceaux de faisan bardés et le reste des émincés de truffe ;
terminez par le reste de la farce ; lissez la surface.

Couvrez le pâté avec les deux couvercles en ménageant des orifices, comme il
convient, pour le dégagement des gaz ; dorez-le au jaune d'œuf ainsi que les
bouchons des cheminées.

\medskip

\textit{Cuisson}. — Au four chaud pendant {\ppp30\mmm} à {\ppp40\mmm} minutes
au maximum. Laissez refroidir le pâté.

\medskip

\textit{Finissage}. — Faites fondre la gelée de veau et de volaille, ajoutez le
vin blanc, réduisez, dépouillez pendant la cuisson. Au dernier moment, corsez
avec le fumet de gibier.

Clarifiez la gelée avec des blancs d'œufs ; laissez-la refroidir suffisamment,
puis introduisez-la dans le pâté par les cheminées, au moyen d'un entonnoir.

Laissez refroidir définitivement et obstruez les cheminées avec les bouchons de
pâte cuits à part.

\sk

Comme variantes, on pourra remplacer le vin de Bordeaux par du vin de
Bourgogne, les bardes de lard par de minces escalopes de foie gras.

\sk

Inutile de dire qu'on pourra apprêter de même toutes sortes de salmis.

\section*{\centering Terrine de faisan.}
\phantomsection
\addcontentsline{toc}{section}{ Terrine de faisan.}
\index{Terrine de faisan}
\index{Faisan (Terrine de)}

Pour {\ppp10\mmm} à {\ppp30\mmm} Personnes prenez :

\medskip

1° pour l'intérieur :

\footnotesize
\begin{longtable}{rrrrrp{18em}}
 & \hspace{2em} & 1 500 & grammes & de & foie gras d'oie, blanc rosé,                                     \\
 & \hspace{2em} &   750 & grammes & de & truffes noires du Périgord,                                      \\
 & \hspace{2em} &   300 & grammes & de & madère,                                                          \\
 & \hspace{2em} &   250 & grammes & de & noix de veau pâtissière,                                         \\
 & \hspace{2em} &   250 & grammes & de & filet de porc frais, désossé,                                    \\
 & \hspace{2em} &   200 & grammes & de & lard gras frais,                                                 \\
 & \hspace{2em} &    75 & grammes & de & fine champagne,                                                  \\
 & \hspace{2em} &    45 & grammes & de & sel blanc,                                                       \\
 & \hspace{2em} &     5 & grammes & de & poivre fraîchement moulu,                                        \\
 & \hspace{2em} &       &         &  3 & faisans,                                                         \\
 & \hspace{2em} &       &         &    & bardes fines de lard ;                                           \\
\end{longtable}
\normalsize

2° pour le fond :

\footnotesize
\begin{longtable}{rrrrrp{18em}}
 & \hspace{2em} &   750 & grammes & de & gîte de bœuf,                                                    \\
 & \hspace{2em} &   750 & grammes & de & jarret de veau,                                                  \\
 & \hspace{2em} &   500 & grammes & de & bon vin blanc sec,                                               \\
 & \hspace{2em} &   100 & grammes & de & couenne maigre,                                                  \\
 & \hspace{2em} &   100 & grammes & de & carottes,                                                        \\
 & \hspace{2em} &    30 & grammes & de & sel gris,                                                        \\
 & \hspace{2em} &    10 & grammes & de & céleri,                                                          \\
 & \multicolumn{3}{r}{1/2 gramme}      & de & poivre en grains,                                           \\
 & \multicolumn{3}{r}{15 centigrammes} & de & muscade,                                                    \\
 & \multicolumn{3}{r}{2 litres 1/2}    & d’ & eau,                                                        \\
 & \hspace{3em} &       &         &  3 & abatis de poulardes,                                             \\
 & \hspace{3em} &       &         &  1 & pied de veau,                                                    \\
 & \hspace{3em} &       &         &  1 & poireau moyen (le blanc seulement),                              \\
 & \hspace{3em} &       &         &  1 & échalote,                                                        \\
 & \hspace{3em} &       &         &  1 & clou de girofle,                                                 \\
 & \hspace{3em} &       &         &    & beurre ou graisse de volaille,                                   \\
 & \hspace{3em} &       &         &    & bouquet garni (thym, laurier, persil, cerfeuil).                 \\
\end{longtable}
\normalsize

\textit{Préliminaires}. — La veille du jour où vous voudrez faire la terrine,
plumez, videz, flambez les faisans ; désossez-les, enlevez les nerfs, la peau,
les cous. Réservez tous les déchets.

Brossez, lavez les truffes, séchez-les dans un linge.

Mettez, dans un ustensile fermé, faisans, noix de veau, filet de porc, lard,
truffes et foie gras dont vous aurez enlevé la peau ; arrosez avec le madère.
Laissez en contact.

Nettoyez les abatis de poulardes et les intérieurs des faisans ; flambez-les,
réservez-les.

\medskip

Le lendemain, coupez le gîte de bœuf, la couenne, le jarret et le pied de veau
en morceaux ; faites-les revenir avec les abatis, moins les foies, dans un peu
de beurre ou de graisse de volaille ; retirez-les et remplacez-les par les
carottes émincées, le céleri, le poireau et l'échalote ; laissez pincer
légèrement. Égouttez la graisse.

Mettez dans une marmite viandes et légumes revenus, mouillez avec l'eau ;
faites bouillir, écumez, puis ajoutez le vin blanc, les déchets réservés des
faisans, moins les foies, dont vous aurez écrasé les têtes et les os, le
bouquet garni, le clou de girofle, la muscade, le sel et le poivre. Laissez
cuire à tout petit feu pendant {\ppp6\mmm} à {\ppp7\mmm} heures. Dégraissez, dépouillez pendant la
cuisson. Concentrez le fond suffisamment.

Pelez les truffes, réservez les pelures.

Coupez le lard gras en petits lardons à larder et une partie des truffes en
petits bâtonnets de dimensions semblables à celles des lardons.

Levez sur les faisans les beaux morceaux de chair ; réservez les débris.

Piquez les morceaux de faisan avec des lardons et des bâtonnets de truffe,
assaisonnez-les avec du sel et du poivre, puis flambez-les avec une partie de
la fine champagne.

Coupez le foie gras en tranches de {\ppp2\mmm} centimètres d'épaisseur que vous piquerez
avec des bâtonnets de truffe ; salez, poivrez. Réservez les debris.

Hachez séparément noix de veau, filet de porc oies de poulardes, débris et
foies de faisans, pelures de truffes ; réunissez le tout ; hachez de nouveau,
puis pilez au mortier ; ajoutez ensuite les débris du foie gras et du lard
gras, salez, poivrez, pilez encore ; mouillez avec le madère de la marinade, le
reste de la fine champagne et plus ou moins du fond préparé, de façon à avoir
une farce de consistance convenable. Passez-la au tamis.

Prenez une grande terrine ou deux moyennes, de forme allongée, garnissez-en le
fond et les parois avec des bardes de lard et disposez dedans, par couches
alternées, farce, escalopes de foie gras et morceaux de faisan, en ayant soin
que la direction des lardons soit parallèle au grand axe de la terrine ;
intercalez dans les intervalles le reste des truffes entières ou coupées,
couvrez avec des bardes de lard que vous piquerez de place en place avec la
pointe d'un couteau, puis mouillez avec suffisamment de fond. Fermez avec le
couvercle.

Mettez la terrine sur un plafond à rebord ou dans un plat garni d'eau chaude
pendant toute la durée de l'opération et faites cuire au four pendant deux
heures environ, d’abord à feu modéré pendant une heure et demie, puis à feu
diminué jusqu'à extinction pendant une demi-heure. La cuisson est à point quand
une aiguille à brider, enfoncée dans le contenu de la terrine, est à sa sortie
également chaude partout. Pendant la cuisson, au fur et à mesure de
l'évaporation, arrosez la terrine avec du fond chaud de façon à lui en faire
absorber le plus possible. Lorsque le mouillement a été bien fait, le contenu
de la terrine est moelleux, fondant et parfumé à souhait.

Pressez avec une planchette surmontée d’un poids pas trop lourd pour assurer
l'homogénéilé et laissez refroidir.

\sk

Si la terrine est consommée de suite, on peut la recouvrir ou non de bonne
gelée. Lorsqu'elle doit être conservée plusieurs jours, il est bon de couler
sur le dessus de la graisse de porc fraîche.

\sk

On pourra préparer de même toutes les terrines de gibier à plumes.

\section*{\centering Halbran rôti.}
\phantomsection
\addcontentsline{toc}{section}{ Halbran rôti.}
\index{Halbran rôti}

On appelle « halbran » le jeune canard sauvage qui n'a pas encore la plume
d'août. Ceux que l'on tue dans la seconde moitié de juillet sont tout à fait
à point. Ils ne sont pas plus gros que des perdreaux ; leur chair est très fine
car ils n'ont pas encore mangé de poisson ; ils se sont nourris exclusivement
de graines.

La meilleure façon de les apprêter est de les faire rôtir enveloppés d'une fine
barde de lard.

Servis chauds avec une salade verte et le jus de cuisson dégraissé, ils sont
excellents ; ils ne sont pas moins bons froids, accompagnés d'une sauce
préparée avec de l'huile d'olive, du zeste de citron râpé, du jus de citron, de
la moutarde à l'estragon, du sel et du poivre.

Si osée que puisse paraître mon opinion, je déclare que les halbrans de trois
à quatre mois peuvent supporter la comparaison avec les perdreaux les plus fins.

\sk

On peut apprêter de même des pigeons.

\sk

La sauce originale au zeste de citron que je viens d'indiquer est également
très bonne avec d'autres viandes froides.

\section*{\centering Canard sauvage aux olives farcies d’anchois.}
\phantomsection
\addcontentsline{toc}{section}{ Canard sauvage aux olives farcies d’anchois.}
\index{Canard sauvage aux olives farcies d’anchois}
\index{Caneton aux olives farcies d’anchois}

L'idée directrice qui a présidé à la création de ce plat a été de rappeler la
nature aquatique du canard, en mariant sa chair avec du poisson.

Lorsqu'on à un petit caneton, le mieux est de le faire rôtir ; lorsqu'on ne
dispose que d'un canard adulte, tendre encore cependant, il vaut mieux le faire
braiser ; quant aux vieux canards sauvages, ils sont généralement trop durs
pour être comestibles.

Quel que soit le canard que l'on ait à préparer, la première chose à faire est
de farcir les olives.

Prenez des olives en quantité suffisante, par exemple {\ppp250\mmm} grammes
pour un caneton, {\ppp500\mmm} grammes pour un beau canard adulte, dessalez-les,
enlevez-en les noyaux et remplacez-les par des morceaux de filets d'anchois
dessalés de même longueur.

Videz l'animal (caneton ou canard), sortez le foie, le cœur, le gésier, et
mettez à leur place les olives farcies.

Nettoyez les organes enlevés ; réservez-les,

\sk

Si vous avez affaire à un caneton, bardez-le et faites-le rôtir à la broche
pendant {\ppp25\mmm} minutes, en l'assaisonnant seulement avec du poivre.

Hachez le foie, le cœur, le gésier, poivrez et faites cuire à part avec un peu
de beurre.

Retirez la barde du caneton rôti, hachez-la, ajoutez-la avec du beurre
d'anchois au hachis précédent ; mélangez.

Préparez une tartine rectangulaire de {\ppp2\mmm} centimètres 1/2 d'épaisseur,
suffisamment longue et large pour recevoir le caneton, garnissez-la avec le
hachis et servez le caneton couché sur ce canapé.

Envoyez à part la sauce dégraissée dans une saucière.

\sk

Si vous avez affaire à un canard adulte, préparez d'abord {\ppp600\mmm}
à {\ppp700\mmm} grammes de bouillon concentré de canard avec deux abatis de
canards, moins les foies que vous réserverez ; ne le salez pas.

Faites revenir le canard dans du beurre, assaisonnez-le avec du poivre,
retirez-le ensuite et mettez à sa place {\ppp40\mmm} grammes de farine, tournez pendant
quelques minutes, mouillez avec le bouillon de canard, remettez le canard
revenu et laissez cuire à feu doux pendant une heure environ ; goûtez,
complétez l'assaisonnement s'il y a lieu, dégraissez, concentrez la sauce et
achevez la liaison en ajoutant les foies réservés, passés en purée.

Dressez le canard sur un plat et servez-le masqué avec la sauce.

\sk

Ce mode de préparation est applicable à tous les gibiers d'eau.

\section*{\centering Râbles de lapereaux de garenne rôtis.}
\phantomsection
\addcontentsline{toc}{section}{ Râbles de lapereaux de garenne rôtis.}
\index{Râbles de lapereaux de garenne rôtis}

Pour trois ou quatre personnes prenez :

\footnotesize
\begin{longtable}{rrrp{16em}}
        &         &  2 & lapereaux de garenne,                                                            \\
        &         &  2 & bardes de lard,                                                                  \\
        &         &  2 & gousses d'ail,                                                                   \\
        &         &    & légumes de pot-au-feu,                                                           \\
        &         &    & thym, laurier, serpolet, sauge, pimprenelle.                                     \\
        &         &    & vinaigre,                                                                        \\
        &         &    & sel et poivre.                                                                   \\
\end{longtable}
\normalsize

Dépouillez et videz les lapereaux ; séparez les râbles.

Faites bouillir, dans de l'eau salée et poivrée, le reste des lapereaux et des
légumes de pot-au-feu ; concentrez la cuisson de façon à avoir cent grammes de
fond environ ; passez-le.

\index{Fond de lapereau de garenne}
Mettez dans l'intérieur des râbles l'ail et les herbes aromatiqnes indiquées
ci-dessus, en quantité plus ou moins grande suivant que vous voulez les
parfumer plus ou moins ; salez-les, poivrez-les, bardez-les et faites-les rôtir
à la broche en les arrosant avec le fond de lapereau. Au bout de {\ppp10\mmm}
à {\ppp15\mmm} minutes, suivant la grosseur des râbles, mettez plus ou moins de
vinaigre, au goût, dans la lèchefrite et achevez la cuisson, qui doit durer
encore {\ppp5\mmm} minutes, en continuant à arroser le rôti.

Débrochez, retirez les bardes, dégraissez le jus de la lèchefrite.

Dressez les râbles sur un plat et servez en envoyant en même temps la sauce
dans une saucière. Comme accompagnement, des betteraves à la crème conviennent
très bien.

\section*{\centering Râbles de lapereaux de garenne à la moutarde.}
\phantomsection
\addcontentsline{toc}{section}{ Râbles de lapereaux de garenne à la moutarde.}
\index{Râbles de lapereaux de garenne à la moutarde}

Pour quatre personnes prenez :

\footnotesize
\begin{longtable}{rrrp{16em}}
    200 & grammes & de & crème,                                                                           \\
        &         &  2 & lapereaux de garenne,                                                            \\
        &         &  2 & bardes de lard,                                                                  \\
        &         &    & légumes,                                                                         \\
        &         &    & vin blanc,                                                                       \\
        &         &    & moutarde ordinaire,                                                              \\
        &         &    & sel et poivre.                                                                   \\
\end{longtable}
\normalsize

Dépouillez et videz les lapereaux, coupez les râbles aussi haut que possible ;
enduisez-les de moutarde de tous côtés et laissez-les s'en imprégner pendant
{\ppp24\mmm} heures.

Avec les déchets des lapereaux, des légumes, du vin blanc plus ou moins mouillé
d'eau, du sel, du poivre, préparez un fumet ; passez-le.

Coupez les bardes de lard en bandelettes, entourez les râbles comme des momies
avec ces bandelettes, puis mettez-les dans un plat beurré allant au feu et
faites-les cuire au four pendant une demi-heure.

Enlevez les lapereaux, dégraissez le jus de cuisson, ajoutez-y le fumet et la
crème, chauffez mais ne faites pas bouillir, goûtez pour l'assaisonnement.

Dressez les râbles sur un plat, masquez-les avec cette sauce et servez en
envoyant en même temps soit des betteraves à la béchamel, soit des pommes de
terre écrasées, soit du riz sec.

La moutarde a complètement disparu.

Le plat est curieux et agréable.

\sk

On peut préparer de même le râble du lapin domestique.

\section*{\centering Filets de lapereau au jambon, gratinés.}
\phantomsection
\addcontentsline{toc}{section}{ Filets de lapereau au jambon, gratinés.}
\index{Filets de lapereau au jambon, gratinés}

Pour trois personnes prenez :

\footnotesize
\begin{longtable}{rrrp{16em}}
    200 & grammes & de & lard de poitrine,                                                                \\
    125 & grammes & de & jambon de Bayonne,                                                               \\
    100 & grammes & de & crème,                                                                           \\
     50 & grammes & de & beurre,                                                                          \\
     30 & grammes & de & parmesan,                                                                        \\
        &         &  1 & lapereau de garenne,                                                             \\
        &         &  1 & betterave moyenne,                                                               \\
        &         &    & légumes de pot-au-feu,                                                           \\
        &         &    & sel et poivre.                                                                   \\
\end{longtable}
\normalsize

Dépouillez et videz le lapereau ; levez-en les filets, aplatissez-les, émincez
la chair des cuisses ; mettez-la à part ; réservez les déchets.

Faites cuire la betterave au four ; pelez-la, coupez-la en tranches.

Préparez, avec les déchets du lapereau, des légumes, de l'eau, du sel et du
poivre, un fond que vous réduirez à quelques cuillerées.

Faites revenir dans le beurre le jambon coupé en deux tranches minces,
retirez-le ; puis mettez les filets de lapereau, les émincés des cuisses et le
lard coupé en languettes ; laissez cuire.

Chauffez doucement dans la crème betterave, émincés de lapereau et lard.

Disposez, dans un plat allant au feu, les tranches de jambon sur lesquelles
vous placerez les filets, saupoudrez de parmesan et pousser au four pour
gratiner.

Au sortir du four, masquez avec le fond de lapereau et servez dans le plat.

Envoyez en même temps betterave, émincés de lapereau, lard et crème, dans un
légumier.

\section*{\centering Gibelotte de lapereau de garenne.}
\phantomsection
\addcontentsline{toc}{section}{ Gibelotte de lapereau de garenne.}
\index{Gibelotte de lapereau de garenne}
\label{pg0648} \hypertarget{p0648}{}

Pour deux ou trois personnes prenez :

\footnotesize
\begin{longtable}{rrrp{16em}}
    200 & grammes & de & purée de tomates aromatisée,                                                     \\
    150 & grammes & de & bouillon,                                                                        \\
    150 & grammes & de & vin blanc,                                                                       \\
     80 & grammes & de & beurre,                                                                          \\
     30 & grammes & de & vinaigre doux de vin,                                                            \\
     30 & grammes & d' & oignon,                                                                          \\
     15 & grammes & d' & armagnac,                                                                        \\
      5 & grammes & d’ & échalote,                                                                        \\
      5 & grammes & de & persil,                                                                          \\
      5 & grammes & de & sel,                                                                             \\
      2 & grammes & de & poivre,                                                                          \\
      2 & grammes & d' & ail,                                                                             \\
        &         &  1 & lapereau de garenne\footnote{A défaut de lapereau de garenne,
                         on pourra prendre un jeune lapin domestique.}                                    \\
\end{longtable}
\normalsize

Dépouillez, videz le lapereau ; coupez-le en morceaux que vous ferez revenir
dans le beurre pendant un quart d'heure. Retirez les morceaux de lapereau,
tenez-les au chaud. Mettez dans le beurre oignon, échalote, ail et persil haché
fin, tournez pendant quelques instants, puis mouillez avec le bouillon et le
vin ; ajoutez ensuite la purée de tomates, le sel et la moitié du poivre,
Faites cuire pendant un quart d'heure. Remettez alors les morceaux de lapereau
et continuez encore la cuisson pendant une demi-heure.

Mettez le reste du poivre dans le vinaigre et faites réduire au volume d'une
demi-cuillerée à café de liquide.

Dix minutes avant la fin de la cuisson du lapereau, ajoutez l'armagnac et le
vinaigre réduit.

Dressez et servez.

\section*{\centering Gibelotte de lapereau de garenne.}
\phantomsection
\addcontentsline{toc}{section}{ Gibelotte de lapereau de garenne.}
\index{Gibelotte de lapereau de garenne}

\begin{center}
\textit{(Autre formule).}
\end{center}

Pour deux ou trois personnes prenez :

\footnotesize
\begin{longtable}{rrrp{16em}}
    350 & grammes & de & bouillon,                                                                        \\
    250 & grammes & de & champignons de couche,                                                           \\
    200 & grammes & de & lard de poitrine,                                                                \\
    200 & grammes & de & vin blanc,                                                                       \\
     80 & grammes & de & beurre,                                                                          \\
     30 & grammes & d' & oignon,                                                                          \\
     15 & grammes & d' & armagnac,                                                                        \\
      5 & grammes & d' & échalote,                                                                        \\
      5 & grammes & de & persil,                                                                          \\
      2 & grammes & d' & ail,                                                                             \\
        &         &  1 & lapereau de garenne,                                                             \\
        &         &    & sel et poivre.                                                                   \\
\end{longtable}
\normalsize

Dépouillez et videz le lapereau ; réservez la tête, les extrémités des pattes
et le foie ; coupez le reste en morceaux.

Épluchez les champignons, émincez-les ; réservez les pelures.

Faites cuire dans le bouillon les parties réservées du lapereau et les pelures
des champignons, de façon à obtenir {\ppp150\mmm} grammes de jus ; passez-le.

Coupez le lard en morceaux.

Faites revenir les morceaux de lapereau et de lard dans le beurre pendant un
quart d'heure environ ; flambez ensuite avec l'armagnac, puis retirez la viande
et remplacez-la par l'oignon, l'échalote, l'ail que vous laisserez dorer.
Mouillez avec le jus et le vin, salez, poivrez, ajoutez le persil et laissez
cuire pendant un quart d'heure. Mettez alors le lapereau et le lard revenus,
les champignons émincés et achevez la cuisson, qui doit durer encore une
demi-heure,

Disposez dans un plat lapereau, lard et champignons, masquez avec la sauce
passée et servez.

\medskip

On pourra corser le plat à volonté avec plus ou moins de fumet de gibier.

\section*{\centering Ragoût de lapin de garenne au curry, en turban de riz.}
\phantomsection
\addcontentsline{toc}{section}{ Ragoût de lapin de garenne au curry, en turban de riz.}
\index{Ragoût de lapin de garenne au curry, en turban de riz}

Dépouillez, videz et découpez des lapins de garenne, à raison de deux lapins
pour quatre ou cinq personnes ; réservez les beaux morceaux,

Mettez les bas morceaux et les déchets dans une casserole avec des légumes, du
sel, du poivre, de l'eau en quantité suffisante et préparez un bouillon de
lapin ; concentrez-le, passez-le.

Faites revenir dans du beurre les beaux morceaux de lapin, du bacon coupé en
tranches et des oignons ; mouillez avec le bouillon de lapin, ajoutez un
bouquet garni, un peu de farine maniée, du curry au goût ; laissez cuire.

Au dernier moment, enlevez les morceaux de lapin et le bacon ; tenez-les au
chaud. Passez la sauce, dégraissez-la, concentrez-la et achevez sa liaison avec
des jeunes d'œufs. Remettez dedans le lapin et le bacon ; chauffez, mais ne
faites plus bouillir.

En même temps, faites cuire du riz aux cèpes,
\hyperlink{p0708-1}{p. \pageref{pg0708-1}} ; mettez-le dans un moule à couronne ;
tenez-le au chaud au bain-marie.

Démoulez le turban de riz sur un plat, versez dedans le ragoût et servez.

\section*{\centering Civet de lapin de garenne ou de lapin domestique.}
\phantomsection
\addcontentsline{toc}{section}{ Civet de lapin de garenne ou de lapin domestique.}
\index{Civet de lapin de garenne ou de lapin domestique}
\index{Civet de lapin}
\index{Civet de lièvre}

Pour quatre personnes prenez :

\footnotesize
\begin{longtable}{rrrrrp{18em}}
  & \hspace{2em} &  300 & grammes & de & bon bouillon,                                                    \\
  & \hspace{2em} &  250 & grammes & de & bon vin rouge,                                                   \\
  & \hspace{2em} &  125 & grammes & de & lard de poitrine,                                                \\
  & \hspace{2em} &  100 & grammes & de & sang de lapin ou de porc,                                        \\
  & \hspace{2em} &   90 & grammes & de & beurre,                                                          \\
  & \hspace{2em} &   60 & grammes & de & cognac,                                                          \\
  & \hspace{2em} &   60 & grammes & d' & oignons épluchés,                                                \\
  & \hspace{2em} &   25 & grammes & d' & échalotes épluchées,                                             \\
  & \hspace{2em} &   25 & grammes & de & farine,                                                          \\
  & \hspace{2em} &    2 & grammes & de & poivre,                                                          \\
  & \multicolumn{3}{r}{5 centigrammes} & de & quatre épices,                                              \\
  & \hspace{2em} &      &         &  1 & \hangindent=1em lapin dépouillé et vidé, pesant
                                         1 000 à 1 200 grammes,                                           \\
  & \hspace{2em} &      &         &  1 & \hangindent=1em bouquet garni composé de 10 grammes de persil,
                                         1 gramme de thym et 2 décigrammes de laurier,                    \\
  & \hspace{2em} &      &         &    & graisse,                                                         \\
  & \hspace{2em} &      &         &    & sel.                                                             \\
\end{longtable}
\normalsize

Coupez le lapin en morceaux, réservez le foie et le sang.

Faites revenir : d'une part, la viande dans une sauteuse avec {\ppp60\mmm}
grammes de beurre ; d'autre part, le lard coupé en languettes ou en dés, dans
une poêle, avec un peu de graisse.

Faites un roux avec le reste du beurre et la farine, mettez d'abord oignons et
échalotes ciselés, tournez jusqu'à ce qu'ils aient pris couleur, ensuite la
viande revenue, flambez avec le cognac, puis mouillez avec le vin et le
bouillon, enfin ajoutez le bouquet garni, le lard revenu, le poivre, les quatre
épices et plus ou moins de sel ({\ppp5\mmm} à {\ppp10\mmm} grammes) suivant que
le lard est plus ou moins salé. Laissez cuire à feu assez vif, plus ou moins
longtemps, suivant l’âge du lapin, et de manière à réduire suffisamment la
sauce.

Cing minutes avant la fin, mettez le foie réservé.

Dressez la viande et le lard sur un plat : tenez au chaud. Éloignez la
casserole du feu, versez le sang dans la sauce, chauffez pendant quelques
instants pour la liaison, passez la sauce sur la viande et servez.

\sk

On peut préparer de même un civet de lièvre.

\section*{\centering Lièvre.}
\phantomsection
\addcontentsline{toc}{section}{ Lièvre.}
\index{Lièvre}

De tous les gibiers à poil, le lièvre est le plus intéressant au point de vue
comestible. Sa chair savoureuse se prête à une foule de combinaisons
culinaires ; j'en connais, pour ma part, plus de quarante.

Je ne parlerai pas de certaines formules très renommées et archiconnues, par
exemple celle du lièvre à la royale ; je me bornerat à donner une vingtaine de
recettes qui suffiront à montrer le part qu'on peut tirer de la chair de cet
animal.

\medskip

Ces formules sont applicables au lapin de garenne.

\smallskip

\section*{\centering Levraut sauté.}
\phantomsection
\addcontentsline{toc}{section}{ Levraut sauté.}
\index{Levraut sauté}

Pour quatre personnes prenez :

\footnotesize
\begin{longtable}{rrrp{16em}}
    250 & grammes & de & consommé,                                                                        \\
    200 & grammes & de & vin blanc sec,                                                                   \\
    150 & grammes & de & beurre,                                                                          \\
    125 & grammes & de & lard de poitrine,                                                                \\
    125 & grammes & de & petits champignons de couche,                                                    \\
     10 & grammes & de & farine,                                                                          \\
      5 & grammes & de & persil haché,                                                                    \\
        &         &  1 & levraut,                                                                         \\
        &         &  1 & oignon moyen haché fin,                                                          \\
        &         &    & sel et poivre.                                                                   \\
\end{longtable}
\normalsize

Dépouillez et videz le levraut, coupez-le en morceaux, faites-le revenir avec
le lard, coupé en petits cubes, dans {\ppp100\mmm} grammes de beurre ;
sautez-le ensuite sur un feu vif. Cette partie de l'opération dure de quinze
à vingt minutes et doit être arrêtée aussitôt que la viande est cuite.

Enlevez de la sauteuse le levraut et le lard ; tenez-les au chaud.

Faites blondir la farine dans {\ppp30\mmm} grammes de beurre, ajoutez l'oignon
haché, tournez pendant quelques instants pour lui faire prendre couleur,
mouillez avec le consommé et le vin blanc, mettez les champignons pelés et
lavés ; salez, poivrez ; laissez cuire pendant dix minutes au plus.

Montez la sauce avec le reste du beurre, ajoutez-y les morceaux de levraut et
le lard, saupoudrez avec le persil, chauffez pendant trois minutes, puis
servez.

\medskip

Il est inutile de dire qu'il est indispensable d'avoir un animal tout jeune et très
tendre pour que le plat soit réellement ce qu'il doit être.

\section*{\centering Filets de levraut rôtis, sauce aux truffes.}
\phantomsection
\addcontentsline{toc}{section}{ Filets de levraut rôtis, sauce aux truffes.}
\index{Filets de levraut rôtis, sauce aux truffes}

Pour deux personnes prenez :

\footnotesize
\begin{longtable}{rrrp{16em}}
    350 & grammes & de & vin rouge,                                                                       \\
    125 & grammes & de & bon consommé,                                                                    \\
    125 & grammes & de & truffes noires du Périgord,                                                      \\
    100 & grammes & de & jus de viande,                                                                   \\
     65 & grammes & de & lard gras à piquer,                                                              \\
     60 & grammes & de & beurre,                                                                          \\
     45 & grammes & de & fine champagne,                                                                  \\
     15 & grammes & de & farine,                                                                          \\
     15 & grammes & de & vinaigre de vin,                                                                 \\
     10 & grammes & d' & échalote hachée fin,                                                             \\
        &         &  2 & oignons,                                                                         \\
        &         &  1 & levraut.                                                                         \\
        &         &  1 & carotte,                                                                         \\
        &         &    & bouquet garni,                                                                   \\
        &         &    & moutarde,                                                                        \\
        &         &    & sel et poivre.                                                                   \\
\end{longtable}
\normalsize

Dépouillez et videz le levraut, mettez de côté le foie et le sang. Séparez la
partie comprise entre le train de devant et celui de derrière ; réservez le
reste\footnote{Avec le reste du levraut on pourra faire un excellent civet pour
deux personnes.}.

Coupez le lard à piquer en lardons, assaisonnez-les de sel et de poivre et
piquez-en les filets.

Mettez dans une casserole vin, fine champagne, oignons ciselés, carotte coupée
en rondelles, bouquet garni ; faites cuire et réduisez le liquide au tiers de son
volume environ ; passez-le.

Faites blondir la farine dans le beurre, ajoutez l'échalote hachée, tournez
pendant quelques instants, mouillez avec la réduction, le consommé et le jus de
viande. Concentrez suffisamment ; puis ajoutez les truffes nettoyées et coupées
en tranches ; laissez mijoter pendant un quart d'heure.

En même temps, faites rôtir à la broche les filets piqués ; la cuisson demande
un quart d'heure.

Écrasez le foie du levraut, mélangez-le avec le sang, le vinaigre et le jus de
cuisson dégraissé de la lèchefrite, passez le mélange au tamis et incorporez-le
à la sauce ; salez, poivrez, chauffez doucement ; goûtez et corsez
l'assaisonnement avec un peu de moutarde au goût.

Débrochez, détachez les filets de l'épine dorsale, dressez-les sur un plat,
masquez-les avec la sauce et servez.

\sk

\index{Filets de lièvre  rôtis, sauce aux truffes}
\index{Filets de lapin de garenne rôtis, sauce aux truffes}
On peut apprêter de même des filets de lapin de garenne et de lièvre. La
cuisson en sera un peu plus longue et durera une vingtaine de minutes environ.

\section*{\centering Filets de levraut sautés, sauce béarnaise.}
\phantomsection
\addcontentsline{toc}{section}{ Filets de levraut sautés, sauce béarnaise.}
\index{Filets de levraut sautés, sauce béarnaise}

Pour deux personnes prenez :

\footnotesize
\begin{longtable}{rrrp{16em}}
    130 & grammes & de & beurre,                                                                          \\
     60 & grammes & de & lard à piquer,                                                                   \\
     35 & grammes & de & vinaigre de vin,                                                                 \\
     10 & grammes & d' & eau froide,                                                                      \\
        &         &  1 & levraut,                                                                         \\
        &         &  1 & jaune d'œuf frais,                                                               \\
        &         &  1 & petite échalote,                                                                 \\
        &         &    & estragon,                                                                        \\
        &         &    & sel et poivre.                                                                   \\
\end{longtable}
\normalsize

Dépouillez, videz le levraut, détachez-en les filets, piquez-les avec le lard gras
coupé en lardons que vous aurez assaisonnés avec sel et poivre.

Faites une réduction avec l'échalote hachée et le vinaigre : passez-la.

Faites sauter pendant un quart d'heure les filets de levraut dans
{\ppp50\mmm} gram\-mes de beurre, salez et poivrez au goût.

En même temps, préparez une sauce hollandaise avec le reste du beurre, le jaune
d'œuf, l'eau, du sel, du poivre, incorporez-y la réduction passée et de
l'estragon haché. Vous aurez ainsi une béarnaise.

Dressez les filets sur un plat chaud, servez-les avec la sauce que vous enverrez
à part dans une saucière.

\section*{\centering Râble de lièvre rôti, sauce au vin rouge.}
\phantomsection
\addcontentsline{toc}{section}{ Râble de lièvre rôti, sauce au vin rouge.}
\index{Râble de lièvre rôti, sauce au vin rouge}

Pour six personnes prenez :

\footnotesize
\begin{longtable}{rrrp{16em}}
    350 & grammes & de & vin rouge,                                                                       \\
    300 & grammes & de & bouillon,                                                                        \\
    125 & grammes & de & lard à piquer,                                                                   \\
    100 & grammes & de & beurre,                                                                          \\
     50 & grammes & d' & oignons,                                                                         \\
     15 & grammes & de & farine,                                                                          \\
     15 & grammes & de & vinaigre de vin,                                                                 \\
        &         &  1 & lièvre\footnote{Les lièvres les meilleurs sont ceux qui se nourrissent
                                         d'herbes aromatiques telles que thym, serpolet, marjolaine, etc.
                                         \protect\endgraf
                                         Le lièvre ne doit jamais être faisandé ; celui qui est destiné
                                         au rôtissage doit être jeune.},                                  \\
        &         &    & bouquet garni,                                                                   \\
        &         &    & muscade,                                                                         \\
        &         &    & sel et poivre.                                                                   \\
\end{longtable}
\normalsize

Dépouillez, videz l'animal ; détachez le râble\footnote{Le râble comprend tout
le dos du lièvre depuis la naissance du cou jusqu'à la queue. Certaines
personnes donnent le nom de râble à la partie qui s'étend entre le train de
devant et celui de derrière ; je trouve préférable de réserver à ce morceau le
nom de filet.} ; mettez de côté le foie et le sang ; faites cuire le reste dans
le bouillon de façon à obtenir {\ppp125\mmm} grammes de jus concentré.

Enlevez les aponévroses du râble, piquez-le de lardons salés et poivrés et
mettez-le à la broche sans autre préparation. Pendant la cuisson, arrosez-le
avec {\ppp50\mmm} grammes de beurre fondu ; salez. La cuisson d'un beau râble
de lièvre demande une heure environ pour être menée à point.

En même temps, préparez la sauce, faites un roux avec {\ppp30\mmm} grammes de
beurre et la farine, mettez les oignons hachés fin, laissez prendre couleur,
mouillez avec le vin, le jus concentré et le vinaigre, ajoutez le bouquet, de
la muscade, du sel et du poivre, au goût ; faites bouillir pendant une heure.
Enlevez le bouquet ; concentrez la sauce.

Réduisez le foie en purée, mélangez-le avec le sang, versez le mélange dans la
sauce ; chauffez, mais ne faites plus bouillir. Passez le tout au travers d'une
passoire fine à l'aide d’un pilon en bois ; tenez au chaud.

Au dernier moment, montez la sauce avec le reste du beurre, ajoutez-y le
contenu dégraissé de la lèchefrite ; mélangez.

Dressez le râble sur un plat chaud et servez. Envoyez en même temps la sauce
dans une saucière.

\section*{\centering Râble de lièvre rôti, sauce au vin blanc.}
\phantomsection
\addcontentsline{toc}{section}{ Râble de lièvre rôti, sauce au vin blanc.}
\index{Râble de lièvre rôti, sauce au vin blanc}

Pour six personnes prenez :

\footnotesize
\begin{longtable}{rrrp{16em}}
    125 & grammes & de & lard à piquer,                                                                   \\
     60 & grammes & de & consommé ou 30 grammes de glace de viande dissoute dans un peu de bouillon,      \\
     60 & grammes & de & jus de viande,                                                                   \\
     60 & grammes & de & vin blanc sec,                                                                   \\
     30 & grammes & de & beurre,                                                                          \\
        &         &  1 & lièvre,                                                                          \\
        &         &    & échalotes,                                                                       \\
        &         &    & cormchons confits,                                                               \\
        &         &    & fines herbes,                                                                    \\
        &         &    & jus de citron,                                                                   \\
        &         &    & sel et poivre.                                                                   \\
\end{longtable}
\normalsize

Apprêtez, piquez et faites rôtir le râble ; pendant la cuisson, arrosez-le avec
le consommé, le vin blanc, le jus de viande et le jus de la moitié d'un citron,
le tout mélangé et mis dans la lèchefrite.

Lorsque le râble est cuit, versez le contenu de la lèchefrite dans une
casserole, dégraissez-le ; ajoutez des échalotes hachées, donnez quelques
bouillons, puis incorporez le foie pilé et le sang du lièvre, montez la sauce
avec le beurre, chauffez sans faire bouillir, assaisonnez avec sel et poivre au
goût, puis mettez des cornichons confits, émincés, des fines herbes ; mélangez
bien.

Dressez le râble sur un plat et servez en envoyant en même temps la sauce dans
une saucière.

\section*{\centering Râble de lièvre braisé à la crème.}
\phantomsection
\addcontentsline{toc}{section}{ Râble de lièvre braisé à la crème.}
\index{Râble de lièvre braisé à la crème}

Pour six personnes prenez :

\footnotesize
\begin{longtable}{rrrp{16em}}
    250 & grammes & de & crème épaisse,                                                                   \\
    200 & grammes & de & fumet de gibicr,                                                                 \\
    125 & grammes & de & lard à piquer,                                                                   \\
     60 & grammes & de & vinaigre de vin,                                                                 \\
        &         &  1 & lièvre,                                                                          \\
        &         &    & échalotes hachées,                                                               \\
        &         &    & sel et poivre.                                                                   \\
\end{longtable}
\normalsize

Dépouillez, videz le lièvre ; détachez le râble, piquez-le de lardons
assaisonnés de sel et de poivre, mettez-le dans une braisière avec le fumet, le
vinaigre, des échalotes, du sel et du poivre, au goût. Faites cuire pendant une
heure en arrosant fréquemment. Ajoutez ensuite la crème et continuez la cuisson
pendant une quinzaine de minutes encore sans laisser bouillir.

Dressez le râble sur un plat, versez dessus la sauce, passée ou non, et servez.

\sk

Cette préparation à la crème est applicable à toutes les viandes noires.

\section*{\centering Râble de lièvre farci braisé, sauce civet.}
\phantomsection
\addcontentsline{toc}{section}{ Râble de lièvre farci braisé, sauce civet.}
\index{Râble de lièvre farci braisé, sauce civet}

Pour six personnes prenez un beau lièvre, dépouillez-le, videz-le, réservez le
sang et le foie.

Séparez l'arrière-train, râble et pattes, de l'avant-train en ayant soin de
laisser attenant au râble, coupé le plus haut possible, autant de peau du
ventre qu'il se pourra.

Hachez très fin du filet de porc, du jambon de Bayonne, de la langue
à l'écarlate, du veau, le foie du lièvre et des champignons ; assaisonnez avec
sel, poivre, épices, aromatisez avec échalotes hachées fin, marjolaine, thym,
serpolet, hysope en poudre, au goût ; mélangez, ajoutez un peu de vieil
armagnac ; cela constituera la farce.

Désossez avec précaution l'arrière-train du lièvre, remplacez les os par de la
farce, emplissez le ventre avec le reste de la farce ; cousez en donnant une
jolie forme au râble ; enveloppez-le dans de fines bandes de lard gras ;
ficelez-le. Ainsi apprêté, mettez-le dans une daubière avec {\ppp40\mmm}
grammes de beurre, {\ppp8\mmm} ou {\ppp4\mmm} petits oignons, un peu de jambon
fumé ; faites revenir ; flambez ensuite avec un peu d'armagnac, salez, poivrez,
mouillez avec {\ppp200\mmm} grammes de vin blanc (champagne ou anjou) et
{\ppp150\mmm} grammes de bon consommé additionné de fumet de gibier à poil ;
couvrez hermétiquement la daubière et laissez cuire doucement pendant trois
heures environ.

Coupez en morceaux le reste du lièvre, tête comprise, mettez-le dans une
casserole avec {\ppp35\mmm} grammes de beurre, {\ppp125\mmm} grammes de lard de
poitrine coupé en dés, {\ppp3\mmm} petits oignons, {\ppp2\mmm} échalotes, {\ppp1\mmm} carotte coupée en
rondelles, {\ppp10\mmm} grammes de céleri ; faites revenir ; ajoutez les os
provenant de l'arrière-train du lièvre, un bouquet garni (persil, thym,
laurier, sauge), du sel, du poivre, un peu de muscade ; mouillez avec
{\ppp250\mmm} grammes de bon consommé et {\ppp200\mmm} grammes de vin blanc sec
de même nature que celui qui aura été employé pour le mouillement de la
daubière ; corsez avec plus ou moins de fumet de gibier, au goût ; laissez
cuire en casserole fermée pendant trois heures au moins. Dégraissez, puis
passez le tout à la presse. Remettez sur le feu la purée obtenue, ajoutez-y le
jus concentré, dégraissé et passé de la daubière ; liez avec le sang du lièvre
et un peu de sang de porc, s'il est nécessaire, de manière à obtenir une sauce
de bonne consistance.

Sortez le lièvre de la daubière, débarrassez-le des ficelles et des résidus de
bardes, dressez-le sur un plat, entourez-le de pommes de terre duchesse et servez.
Envoyez en même temps la sauce dans une saucière.

\section*{\centering Noisettes de lièvre sautées, sur canapés.}
\phantomsection
\addcontentsline{toc}{section}{ Noisettes de lièvre sautées, sur canapés.}
\index{Noisettes de lièvre sautées, sur canapés}
\index{Canapés de noisettes de lièvre sautées}

Pour quatre personnes prenez un lièvre moyen ; dépouillez-le, videz-le,
détachez les filets et les cuisses. Réservez le sang et le foie.

Découpez chaque filet en quatre noisettes ; prenez dans les parties tendres de
chaque cuisse deux autres noisettes et dans d'autres parties charnues et
tendres de l'animal quatre dernières petites noisettes.

Retirez les aponévroses, salez, poivrez et mettez les noisettes à mariner
pendant trois heures dans du vin blanc.

Faites revenir dans un peu de beurre tous les déchets du lièvre et des légumes,
tels que carotte, navet, blanc de poireau, champignons, un peu d'échalote, de
thym et de laurier ; flambez ensuite avec du cognac ; puis mouillez avec une
quantité suffisante d'eau salée et poivrée ; ajoutez le vin blanc de la
marinade et laissez cuire pendant trois heures de manière à obtenir un jus très
concentré. Passez-le en pressant.

Liez le jus avec le sang et le foie pilé du lièvre ; tenez la sauce au chaud.

Faites sauter les noisettes dans du beurre.

Faites dorer dans du beurre des petits canapés de pain en nombre égal à celui
des noisettes et de même surface.

Disposez les canapés sur un plat ; dressez chaque noisette sur un canapé,
masquez avec la sauce et servez.

C'est excellent.

\sk

Les amateurs de foie gras pourront interposer entre les canapés et les
noisettes des escalopes de foie gras préparé au naturel ; ils auront alors
d'exquises noisettes de lièvre sur canapés garnis.

\sk

\index{Canapés de noisettes de lièvre en salmis}
On peut préparer dans le même esprit des noisettes de lièvre en salmis. Dans ce
cas, on commencera par faire revenir la viande et on achèvera sa cuisson dans
la sauce.

\section*{\centering Civet de lièvre non mariné.}
\phantomsection
\addcontentsline{toc}{section}{ Civet de lièvre non mariné.}
\index{Civet de lièvre non mariné}

Pour huit personnes prenez :

\footnotesize
\begin{longtable}{rrrp{16em}}
    750 & grammes & de & bon vin rouge,                                                                   \\
    250 & grammes & de & consommé,                                                                        \\
    250 & grammes & de & champignons de couche,                                                           \\
    250 & grammes & de & poitrine de porc coupée en morceaux,                                             \\
    100 & grammes & de & beurre,                                                                          \\
     20 & grammes & de & farine,                                                                          \\
      2 & grammes & d' & un mélange en parties égales de serpolet, marjolaine,
                         origan, hysope et sarriette en poudre,                                           \\
        &         & 20 & petits oignons,                                                                  \\
        &         &  4 & échalotes,                                                                       \\
        &         &  1 & jeune lièvre de taille moyenne,                                                  \\
        &         &    & bouquet garni (persil, thym, laurier),                                           \\
        &         &    & sel et poivre.                                                                   \\
\end{longtable}
\normalsize

Dépouillez et videz l'animal ; coupez-le en morceaux ; mettez de côté le foie
et le sang.

Faites revenir dans une partie du beurre les morceaux de lièvre et la poitrine
de porc.

Faites dorer dans le reste du beurre les oignons et les échalotes,
retirez-les ; mettez la farine, laissez-la roussir, mouillez avec le vin et le
consommé, puis ajoutez les morceaux de lièvre et de porc, les échalotes et les
oignons revenus, le bouquet garni, les aromates, du sel et du poivre ; continuez
la cuisson, à tout petit feu, pendant deux heures à deux heures et demie.

Une vingtaine de minutes avant de servir, passez la sauce, ajoutez les
champignons pelés et lavés,

Au dernier moment, liez la sauce avec le sang et le foie pilé du lièvre,
mélangez bien ; chauffez sans laisser bouillir ; goûtez, corsez
l’assaisonnement s'il est nécessaire et servez.

\sk

Comme variante, on pourra remplacer les {\ppp20\mmm} petits oignons par
{\ppp50\mmm} grammes de blanc de poireaux, mais il conviendra alors d'ajouter
{\ppp15\mmm} grammes de sucre pour adoucir le goût des poireaux. On servira le
civet avec ou sans les poireaux.

\sk

\index{Civet de noisettes de lièvre, aux morilles}
On pourra préparer dans le même esprit un civet de noisettes de lièvre aux
morilles qu'on corsera avec du fumet de gibier.

On le servira dans un turban de riz au gras garni d'émincés de jambon de
Bayonne, ou dans un turban de pommes de terre agrémenté de bacon revenu émincé
en julienne.

\section*{\centering Civet de lièvre mariné\footnote{ La chair du lièvre fine,
délicate, parfumée sans exagération n'a généralement pas besoin de mariner. Si
je donne ici une formule de livre mariné, qui n'est du reste applicable
raisonnablement qu'à un vieux lièvre, c'est surtout pour montrer côte à côte
deux civets de goûts très différents.}.}
\phantomsection
\addcontentsline{toc}{section}{ Civet de lièvre mariné.}
\index{Civet de lièvre mariné}

Pour huit personnes prenez :

\footnotesize
\begin{tabular}{@{}lrrrp{16em}}
\normalsize1°\footnotesize \hspace{2em} & 125 & grammes & de & lard frais,                                \\
\hspace{2em}  & 100 & grammes & de & consommé,                                                            \\
\hspace{2em}  & 100 & grammes & de & bon vin rouge,                                                       \\
\hspace{2em}  &  70 & grammes & de & beurre,                                                              \\
\hspace{2em}  &  30 & grammes & de & farine,                                                              \\
\hspace{2em}  &  10 & grammes & d' & huile d'olive,                                                       \\
\hspace{2em}  &     &         &  1 & lièvre adulte, de taille moyenne,                                    \\
\hspace{2em}  &     &         &    & sel et poivre ;                                                      \\
\hspace{2em}  &     &         &    &                                                                      \\
\label{pg0660} \hypertarget{p0660}{}
\normalsize 2° & \multicolumn{4}{l}{\normalsize   pour la marinade :}                                     \\
\footnotesize
\hspace{2em}  &     &         &    &                                                                      \\
\hspace{2em}  & 250 & grammes & de & vin,                                                                 \\
\hspace{2em}  &  50 & grammes & de & vinaigre de vin,                                                     \\
\hspace{2em}  &  45 & grammes & d' & huile d'olive,                                                       \\
\hspace{2em}  &     &         &  1 & oignon,                                                              \\
\hspace{2em}  &     &         &    & bouquet garni (persil, thym, laurier),                               \\
\hspace{2em}  &     &         &    & sel et poivre.                                                       \\
\hspace{2em}  &     &         &    &                                                                      \\
\end{tabular}
\normalsize

La veille du jour où vous voudrez faire ce civet, préparez la marinade,

Faites revenir légèrement dans l'huile l'oignon coupé en rondelles, ajoutez
vin, vinaigre, bouquet garni, sel et poivre ; laissez cuire pendant dix
minutes, puis laissez refroidir.

Dépouillez et videz le lièvre, réservez le foie et le sang.

Mettez le lièvre à mariner pendant une journée ; retournez-le fréquemment dans
la marinade.

Le lendemain, mettez dans une casserole le beurre et l'huile d'olive, le lièvre
et le lard coupé en morceaux, faites revenir doucement pendant une vingtaine de
minutes, saupoudrez avec la farine, laissez cuire à petit feu pendant
vingt-cinq minutes en remuant souvent, puis mouillez avec le consommé et le
vin, salez et poivrez au goût. Continuez la cuisson pendant quarante minutes
encore. Passez le jus de cuisson ; passez la marinade ; réunissez-les ;
dégraissez, concentrez la sauce, puis liez-la avec le foie réduit en purée et
le sang du lièvre. Chauffez sans faire bouillir, goûtez, complétez
l'assaisonnement s'il y a lieu et servez.

\section*{\centering Civet de Gascogne.}
\phantomsection
\addcontentsline{toc}{section}{ Civet de Gascogne.}
\index{Civet de Gascogne}

Pour huit personnes prenez :

\footnotesize
\begin{longtable}{rrrp{16em}}
    500 & grammes & de & jambon salé.                                                                     \\
        & 1 litre & de & bon vin rouge,                                                                   \\
        &         &  1 & beau lièvre gras,                                                                \\
        &         &    & bouillon.                                                                        \\
        &         &    & graisse d'oie,                                                                   \\
        &         &    & farine,                                                                          \\
        &         &    & oignon,                                                                          \\
        &         &    & ail,                                                                             \\
        &         &    & sel et poivre.                                                                   \\
\end{longtable}
\normalsize

Dépouillez, videz le lièvre, mettez de côté le sang et le foie.

Coupez l'animal en morceaux ; faites-les revenir, sans qu'ils se touchent, dans
de la graisse d'oie très chaude, jusqu'à ce que la viande ait pris une teinte
rosée.

Faites frire à part, dans la graisse d'oie, le jambon coupé en morceaux.

Mettez dans une casserole en cuivre le lièvre et le jambon, mouillez avec le vin
et du bouillon en quantité suffisante pour que le lièvre soit couvert, ajoutez oignon
haché, ail, sel et poivre au goût et faites mijoter en casserole hermétiquement
fermée pendant huit heures.

Au dernier moment, dégraissez, passez-le concentrez la sauce ; liez-la avec le
foie pilé et le sang du lièvre.

Dressez les morceaux de lièvre et de jambon sur un plat, masquez avec la sauce
et servez.

\section*{\centering Bouchées de lièvre farcies, sauce poivrade.}
\phantomsection
\addcontentsline{toc}{section}{ Bouchées de lièvre farcies, sauce poivrade.}
\index{Bouchées de lièvre farcies, sauce poivrade}

Pour dix personnes prenez :

\bigskip

\index{Enveloppe pour bouchées de lièvre}
\index{Bouchées (Enveloppes pour)}
1° pour l'enveloppe :

\footnotesize
\begin{longtable}{rrrrrp{18em}}
  & \hspace{2em} & 500 & grammes & de & \hangindent=1em  fond de gibier à poil\footnote{On obtiendra
                                        le fond de gibier en faisant cuire longuement dans de l'eau,
                                        des déchets de gibier avec des légumes et des condiments et
                                        en concentrant le jus.},                                          \\
  & \hspace{2em} & 300 & grammes & de & graisse de porc,                                                  \\
  & \hspace{2em} & 300 & grammes & de & chair de lièvre,                                                  \\
  & \hspace{2em} & 200 & grammes & de & farine,                                                           \\
  & \hspace{2em} &     &         &  4 & œufs entiers,                                                     \\
  & \hspace{2em} &     &         &  2 & blancs d'œufs,                                                    \\
  & \hspace{2em} &     &         &  1 & jaune d'œuf,                                                      \\
  & \hspace{2em} &     &         &    & aromates en poudre,                                               \\
  & \hspace{2em} &     &         &    & épices en poudre,                                                 \\
  & \hspace{2em} &     &         &    & sel et poivre ;                                                   \\
\end{longtable}
\normalsize

\index{Farce pour bouchées de lièvre}
2° pour la farce :

\footnotesize
\begin{longtable}{rrrrrp{18em}}
  & \hspace{2em} & 300 & grammes & d' & un mélange de jambon de Bayonne ou d'York, de langue
                         fumée et de champignons, le tout haché et lié au velouté gras très
                         serré, parfumé par des plantes aromatiques ;                                     \\
\end{longtable}
\normalsize

3° pour la sauce :

\footnotesize
\begin{longtable}{rrrrrp{18em}}
  & \hspace{2em}  & 400 & grammes & de & \hangindent=1em  fond brun de gibier à poil\footnote{\index{Fond brun de gibier}
                                                  On obtiendra le fond brun de gibier d'une façon analogue à
                                                  celle qui a été employée pour obtenir le fond brun, 
                                                  \hyperlink{p0203}{p. \pageref{pg0203}}, en remplaçant le bœuf par des 
                                                  parures de gibier.},                                    \\
  & \hspace{2em} & 500 & grammes & de & déchets de lièvre,                                                \\
  & \hspace{2em} & 200 & grammes & d' & espagnole,                                                        \\
  & \hspace{2em} & 200 & grammes & de & marinade,                                                         \\
  & \hspace{2em} &  75 & grammes & de & vinaigre,                                                         \\
  & \hspace{2em} &  50 & grammes & de & vin blanc,                                                        \\
  & \hspace{2em} &  30 & grammes & de & carotte,                                                          \\
  & \hspace{2em} &  30 & grammes & d' & oignon,                                                           \\
  & \hspace{2em} &  20 & grammes & de & beurre,                                                           \\
  & \multicolumn{3}{r}{2 décigrammes} & de & poivre en grains,                                            \\
  & \hspace{2em} &     &         &  1 & bouquet de persil, thym et laurier,                               \\
  & \hspace{2em} &     &         &    & huile.                                                            \\
\end{longtable}
\normalsize

Triturez la farine avec les œufs entiers ; ajoutez, par petites quantités, et en
tournant, le fond de gibier chaud, puis faites prendre sur feu doux en travaillant
pendant une demi-heure environ, de façon à obtenir une bonne consistance.

Pilez la chair de lièvre, ajoutez-y la graisse de porc, du sel, du poivre, des
aromates et des épices au goût ; pilez encore, puis amalgamez au mélange
l'appareil ci-dessus et les deux blancs d'œufs battus en neige.

Passez la pâte au tamis, travaillez-la bien pour la rendre lisse et homogène et
faites-en une abaisse de {\ppp5\mmm} à {\ppp6\mmm} millimètres d'épaisseur, que vous partagerez en
vingt morceaux carrés.

Mettez sur chaque morceau un vingtième de la farce ; fermez les bouchées.

Préparez la sauce poivrade.

Faites revenir dans de l'huile les déchets de lièvre coupés en petits morceaux,
la carotte, l'oignon, le bouquet garni ; égouttez l'huile ; puis mouillez avec
le vin et le vinaigre ; réduisez. Ajoutez ensuite l'espagnole, {\ppp375\mmm}
grammes de fond brun de gibier, {\ppp175\mmm} grammes de marinade ; faites
cuire doucement au four, en casserole couverte, pendant quatre heures environ.
Mettez alors le poivre en grains ; laissez cuire pendant dix minutes, puis
passez la sauce au tamis, en pressant.

Remettez le jus obtenu sur le feu, ajoutez le reste du fond brun de gibier et
le reste de la marinade, dépouillez soigneusement et réduisez à {\ppp200\mmm}
grammes environ. Passez la sauce à l'étamine et finissez-la avec le beurre.

Faites pocher les bouchées dans de l'eau salée bouillante, égouttez-les,
dorez-les au jaune d'œuf et passez-les au four.

Dressez les bouchées sur un plat garni d'une serviette et servez-les très
chaudes. Envoyez en même temps la sauce dans une saucière.

\sk

Comme variante, on pourra ajouter à cette sauce, qu'on relèvera un peu plus et
qu'on corsera davantage, {\ppp60\mmm} grammes de crème fouettée.

\sk

\index{Bouchées de lièvre farcies, sauce chasseur}
Comme autre variante, on pourra accompagner les bouchées avec une sauce
chasseur, qu'on préparera de la façon suivante.

Prenez :

\footnotesize
\begin{longtable}{rrrp{16em}}
    200 & grammes & de & sauce demi-glace, au fumet de gibier,                                            \\
    120 & grammes & de & vin blanc,                                                                       \\
    100 & grammes & de & sauce tomate,                                                                    \\
     80 & grammes & de & champignons,                                                                     \\
     30 & grammes & de & beurre,                                                                          \\
     25 & grammes & de & cognac,                                                                          \\
     10 & grammes & de & glace de viande,                                                                 \\
      3 & grammes & de & persil haché,                                                                    \\
        &         &  2 & échalotes.                                                                       \\
\end{longtable}
\normalsize

Faites griller légèrement dans le beurre les champignons émincés, ajoutez les
échalotes hachées, tournez un instant, mouillez avec le vin et le cognac,
réduisez de moitié. Mettez ensuite la sauce demi-glace, la sauce tomate et la
glace de viande ; amenez à ébullition ; laissez cuire pendant quelques minutes
et finissez avec le persil.

\sk

\index{Bouchées de gibier à poil}       
En variant la nature du gibier, la farce et la sauce, on obtiendra d'autres
bouchées de gibier à poil.

\section*{\centering Lièvre farci, froid.}
\phantomsection
\addcontentsline{toc}{section}{ Lièvre farci, froid.}
\index{Lièvre farci, froid}

Pour douze personnes prenez :

\footnotesize
% \begin{longtable}{rrrp{16em}}
\begin{longtable}{rrrp{16em}}
    750 & grammes & de & foie gras d'oie de Strasbourg ou de Nancy,                                       \\
    500 & grammes & de & foies de volaille et de gibier,                                                  \\
    500 & grammes & de & lard frais,                                                                      \\
    250 & grammes & de & champignons de couche,                                                           \\
     30 & grammes & de & beurre,                                                                          \\
     30 & grammes & de & cognac,                                                                          \\
     20 & grammes & d' & échalotes,                                                                       \\
     15 & grammes & de & farine,                                                                          \\
      3 &  litres & de & gelée de veau et de volaille,                                                    \\
        &         &  1 & beau lièvre jeune et tendre,                                                     \\
        &         &    & marinade, \hyperlink{p0660}{p. \pageref{pg0660}}, ou \hyperlink{p0669}{p. \pageref{pg0669}}, \\
        &         &    & madère,                                                                          \\
        &         &    & porto                                                                            \\
        &         &    & mélange d'épices\footnote{Voici une excellente formule d'un mélange
                                                   d'épices :                                     \\
                                         \protect
%                                          \begin{tabular}{ p{16em} l r l l }
\begin{tabular}{ l l r c c }
% \renewcommand{\arraystretch}{3}% Tighter
\hspace{8em} &                                                             &        &      &      \\[-1.5pt]
\hspace{8em} & Clous de girofle                                  \dotfill  &  27,50 & pour & 100, \\[-1.5pt]
\hspace{8em} & Muscade                                           \dotfill  &  27,50 & pour & 100, \\[-1.5pt]
\hspace{8em} & Thym                                              \dotfill  &  10,00 & pour & 100, \\[-1.5pt]
\hspace{8em} & Laurier                                           \dotfill  &  10,00 & pour & 100, \\[-1.5pt]
\hspace{8em} & Poivre blanc                                      \dotfill  &  10,00 & pour & 100, \\[-1.5pt]
\hspace{8em} & Marjolaine                                        \dotfill  &   5,00 & pour & 100, \\[-1.5pt]
\hspace{8em} & Romarin                                           \dotfill  &   5,00 & pour & 100, \\[-1.5pt]
\hspace{8em} & Cayenne                                           \dotfill  &   5,00 & pour & 100. \\[-1.5pt]
\end{tabular}},                                                                                           \\
        &         &    & truffes à volonté,                                                               \\
        &         &    & crépine,                                                                         \\
        &         &    & sel et poivre.                                                                   \\
\end{longtable}
\normalsize

Dépouillez le lièvre, videz-le, réservez le sang et le foie, désossez
complètement l'animal en le laissant entier, enlevez-en les nerfs et mettez-le
à mariner pendant {\ppp24\mmm} heures dans la marinade.

Assaisonnez le foie gras avec du sel et les épices indiquées, dans la
proportion de 4/5 de sel pour 1/5 d'épices : faites-le mariner pendant
{\ppp24\mmm} heures dans du porto.

Mettez les truffes à mariner dans du madère ; faites-les cuire dans leur
marinade.

\index{Farce gratin pour lièvre}
Préparez une farce « gratin » \hyperlink{p0613}{p. \pageref{pg0613}}, avec le
lard, les champignons, les foies de volaille et de gibier plus le foie réservé
du lièvre, {\ppp20\mmm} grammes de cognac, les échalotes, du sel et du poivre.

Faites cuire dans la gelée de veau et de volaille les os écrasés et les déchets
du lièvre. Passez ; réservez.

Escalopez le foie gras.

Sortez le lièvre de la marinade, étalez-le sur le dos et farcissez-le
entièrement, y compris les cuisses, avec la farce gratin, dans laquelle vous
insérerez en long des escalopes de foie gras, disposées bout à bout, et se
touchant les unes les autres.

Reformez le lièvre, couvrez le ventre d'une crépine et, pour que l'animal
conserve bien sa forme, enroulez-le, comme une momie, dans des bandelettes de
mousseline.

Mettez-le, ainsi préparé, dans une braisière, couvrez-le avec une partie de la
gelée fondue ; réservez le reste : placez sous le couvercle de la braisière une
feuille de papier beurré et faites cuire au four pendant deux heures, en
arrosant fréquemment l'animal avec le jus de cuisson.

Laissez-le refroidir dans la braisière ; retirez-le ensuite.

Remettez sur le feu le jus de cuisson, ajoutez-y les marinades du lièvre, des
truffes et du foie gras et le reste du cognac. Concentrez à petit feu pendant
deux heures environ en dépouillant la sauce. Goûtez pour l'assaisonnement.

Faites un roux avec le beurre et la farine, mouillez avec la sauce dépouillée
et achevez la liaison avec le sang réservé du lièvre.

Débarrassez le lièvre de ses bandelettes, découpez-le en tranches,
reconstituez-le sur un plat ; garnissez avec des truffes ; masquez avec la
sauce ; laissez prendre.

Décorez le plat avec la gelée réservée, coupée en morceaux de formes différentes,
et des truffes.

\section*{\centering Filets de lièvre en chaud-froid.}
\phantomsection
\addcontentsline{toc}{section}{ Filets de lièvre en chaud-froid.}
\index{Filets de lièvre en chaud-froid}
\index{Chaud-froid de filets de lièvre}


Dépouillez un lièvre, levez-en les filets, réservez le sang et le foie que vous
pilerez.

Assaisonnez les filets avec sel, poivre, thym et serpolet en poudre et
faites-les mariner pendant {\ppp24\ppp} heures dans une marinade semblable
à celle de la formule du civet de lièvre mariné.

Mettez, pendant {\ppp24\mmm} heures, des truffes dans du madère, puis
faites-les cuire dans leur marinade.

Préparez un bouillon à bouilli perdu avec le reste du lièvre, de l'eau et des
légumes de pot-au-feu ; aromatisez-le avec la marinade du lièvre et la cuisson
des truffes ; dépouillez-le, passez-le : liez-le avec un roux : vous aurez
ainsi un excellent velouté de lièvre. Ajoutez à ce velouté de la gelée de
gibier, concentrez-le, puis achevez la liaison avec le sang et le foie : vous
aurez une sauce chaud-froid très savoureuse.

Bardez les filets de lièvre et faites-les griller de façon qu'ils soient peu
cuits à l'intérieur.

Découpez les filets obliquement, dressez les tranches sur un plat en les
faisant chevaucher les unes sur les autres ; masquez le tout avec la sauce
chaud-froid et laissez prendre.

Glacez avec de la gelée de gibier bien clarifiée et décorez le plat avec les truffes.

Ce chaud-froid peut être servi seul ou avec une salade.

\section*{\centering Pâté de lièvre.}
\phantomsection
\addcontentsline{toc}{section}{ Pâté de lièvre.}
\index{Pâté de lièvre}
\index{Croûte pour pâtés}
\index{Garniture pour pâtés}

Pour huit à dix personnes prenez :

1° pour la pâte :

\footnotesize
\begin{longtable}{rrrrrp{18em}}
  & \hspace{2em} & 400 & grammes & de & farine.                                                           \\
  & \hspace{2em} & 125 & grammes & de & beurre,                                                           \\
  & \hspace{2em} & 100 & grammes & d' & eau,                                                              \\
  & \hspace{2em} &  12 & grammes & de & sel,                                                              \\
  & \hspace{2em} &  10 & grammes & d’ & huile d'olive non fruitée,                                        \\
  & \hspace{2em} &     &         &  2 & jaunes d'œufs frais ;                                             \\
\end{longtable}
\normalsize

2° pour le corps du pâté :

\footnotesize
\begin{longtable}{rrrrrp{18em}}
  & \hspace{2em} & 000 & grammes & de &                                                                   \kill
  & \hspace{2em} &     &         &  1 & lièvre,                                                           \\
  & \hspace{2em} &     &         &  1 & beau foie gras d'oie de Strasbourg,                               \\
  & \hspace{2em} &     &         &    & truffes à volonté,                                                \\
  & \hspace{2em} &     &         &    & bardes de lard,                                                   \\
  & \hspace{2em} &     &         &    & madère,                                                           \\
  & \hspace{2em} &     &         &    & beurre,                                                           \\
  & \hspace{2em} &     &         &    & jaune d'œuf,                                                      \\
  & \hspace{2em} &     &         &    & sel, poivre, épices ;                                             \\
\end{longtable}
\normalsize

3° pour la farce :

\footnotesize
\begin{longtable}{rrrrrp{18em}}
  & \hspace{2em} & 350 & grammes & de & lard gras,                                                        \\
  & \hspace{2em} & 250 & grammes & de & foies de volaille,                                                \\
  & \hspace{2em} & 200 & grammes & de & champignons de couche,                                            \\
  & \hspace{2em} &  25 & grammes & de & sel,                                                              \\
  & \hspace{2em} &  20 & grammes & d' & échalotes,                                                        \\
  & \hspace{2em} &   1 & gramme  & de & poivre,                                                           \\
  & \multicolumn{3}{r}{1 décigramme} & de & muscade en poudre,                                            \\
  & \hspace{2em} &     &         &    & $\left.
                            \begin{tabular}{l}
                            \setlength\tabcolsep{0pt}
                              laurier  \\
                              thym     \\
                              serpolet \\
                            \end{tabular}
                            \right\} $ au goût ;                                                         \\
\end{longtable}
\normalsize

4° pour la gelée :

\footnotesize
\begin{longtable}{rrrrrp{18em}}
  & \hspace{2em} &  65 &  grammes & de & carottes,                                                        \\
  & \hspace{2em} &  10 &  grammes & de & céleri,                                                          \\
  & \hspace{2em} &     & 2 litres & de & bon bouillon,                                                    \\
  & \hspace{2em} &     &          &  1 & pied de veau,                                                    \\
  & \hspace{2em} &     &          &  1 & poireau moyen (le blanc seulement),                              \\
  & \hspace{2em} &     &          &    & couenne,                                                         \\
  & \hspace{2em} &     &          &    & blancs d'œufs.                                                   \\
\end{longtable}
\normalsize

\textit{Préliminaires}. — Dépouillez le lièvre, videz-le, détachez les filets,
mettez-les de côté ; désossez le reste, hachez la chair ; réservez les os et
les débris. Assaisonnez le foie gras avec du sel et des épices ; mettez-le
à mariner dans du madère.

Faites cuire les truffes avec du sel et des épices dans un peu de madère ;
réservez le tout,

\medskip

\textit{Préparation de la pâte}. — Mélangez comme il convient les éléments
indiqués pour la pâte : laissez-la reposer pendant quelques heures.

\medskip

\textit{Préparation de la gelée}. — Faites cuire longuement pied de veau,
carottes, céleri, poireau, couenne, os et débris de lièvre dans le bouillon ;
parfumez avec le madère de cuisson des truffes et celui de la marinade du foie
gras ; dépouillez pendant la cuisson ; réduisez le liquide de moitié ;
clarifiez-le avec des blancs d'œufs.

\medskip

\textit{Préparation de la farce}. — Hachez ensemble champignons, foies de
volaille, échalotes, thym, laurier, serpolet ; mélangez ce hachis avec la chair
de lièvre hachée, assaisonnez avec le sel, le poivre et la muscade.

Hachez le lard gras, mettez-le dans une poêle, laissez-le fondre, ajoutez
ensuite le hachis et faites revenir. Passez le tout au tamis, goûtez et relevez
l’assaisonnement si vous le jugez nécessaire.

\medskip

\textit{Préparation du pâté}. — Assaisonnez les filets de lièvre avec sel et
poivre ; laites-les revenir légèrement dans du beurre ; bardez-les. Abaissez la
pâte ; réservez-en une partie pour les couvercles et les bouchons des cheminées.

Prenez un moule de forme simple, rectangulaire ; chemisez-le avec l'abaisse de
pâte ; garnissez le fond et les parois avec de la farce ; disposez dessus les
filets de lièvre et le foie mariné que vous aurez coupé en morceaux ; mettez
entre les filets et les morceaux de foie des truffes entières ou coupées ;
couvrez avec le reste de la farce ; lissez le dessus.

Fermez le pâté, comme il est d'usage, avec ses deux couvercles dans lesquels
vous aurez ménagé des ouvertures. Dorez-le au jaune d'œuf ; dorez les bouchons.

\medskip

\textit{Cuisson}. — Au four, à feu vif. La cuisson doit être complète au bout
d'une heure environ pour que le foie gras ne soit pas trop cuit.

\medskip

\textit{Finissage}. — Lorsque le pâté est à peu près refroidi, introduisez
dedans la gelée tiède par les ouvertures que vous obturerez ensuite avec les
bouchons. Laissez refroidir.

\medskip

\textit{Dressage}. — Servez le pâté sur un plat que vous décorerez avec de la
gelée découpée et des rondelles de truffe.

\medskip

Ce pâté, dans lequel les foies de volaille n'interviennent que pour donner du
moelleux, a un excellent goût qui ne rappelle en rien celui des pâtés de lièvre
du commerce, dans lesquels domine surtout la chair à saucisses.

\sk

Il est facile de préparer de la même manière un pâté de lapin.

\sk

En supprimant le foie gras et les truffes, on pourra faire dans le même esprit
des pâtés de lièvre ou de lapin plus simples. Il conviendra alors de piquer les
filets de lièvre ou de lapin avec le lard gras assaisonné et de faire entrer
dans la farce du lard de poitrine frais, assaisonné, coupé en petits cubes, et
doré à la poêle.

\sk

On pourra préparer d'une façon analogue tous les pâtés de gibier à poil.

\section*{\centering Pâté de lièvre à la polonaise.}
\phantomsection
\addcontentsline{toc}{section}{ Pâté de lièvre à la polonaise.}
\index{Pâté de lièvre à la polonaise}

Pour douze personnes prenez :

\footnotesize
\begin{longtable}{rrrp{16em}}
    750 & grammes & de & foie de veau,                                                                    \\
    100 & grammes & de & raisins de Corinthe lavés et épépinés,                                           \\
        &         &  6 & jaunes d'œufs,                                                                   \\
        &         &  1 & lièvre,                                                                          \\
        &         &    & pâte pour pâté, \hyperlink{p0537}{p. \pageref{pg0537}},                          \\
        &         &    & carotte,                                                                         \\
        &         &    & navet,                                                                           \\
        &         &    & poireau,                                                                         \\
        &         &    & oignon,                                                                          \\
        &         &    & sauge,                                                                           \\
        &         &    & thym,                                                                            \\
        &         &    & serpolet,                                                                        \\
        &         &    & marjolaine.                                                                      \\
        &         &    & hydromel,                                                                        \\
        &         &    & lard gras,                                                                       \\
        &         &    & mie de pain rassis tamisée,                                                      \\
        &         &    & sel et poivre.                                                                   \\
\end{longtable}
\normalsize

Préparez {\ppp125\mmm} grammes de bouillon concentré de légumes en faisant
cuire, dans de l'eau salée, carotte, navet, poireau, oignon, sauge, thym,
serpolet, marjolaine et hydromel, au goût.

Dépouillez et videz le lièvre, piquez-le de lard gras, assaisonnez-le avec sel
et poivre ; faites-le dorer à la broche. Laissez-le refroidir, désossez-le,
émincez les filets et les beaux morceaux des cuisses ; réservez-les. Passez le
reste au tamis à l'aide d'un pilon.

Piquez le foie de veau de lardons très rapprochés, assaisonnez-le, faites-le
pocher dans le bouillon, laissez-le refroidir et passez-le au tamis.

Mettez dans une casserole plus ou moins de lard gras coupé en petits dés,
laissez-le fondre, ajoutez-y le foie et les parties du lièvre passées au tamis,
de la mie de pain, mouillez avec le bouillon, laissez cuire ensemble jusqu'à
consistance et onctuosité convenables.

Passez de nouveau au tamis, ajoutez les jaunes d'œufs et le raisin, mélangez,
goûtez et complétez l'assaisonnement avec sel et poivre sil y a lieu.

Abaissez la pâte, réservez-en une partie pour les couvercles et les bouchons,
chemisez un moule avec le reste de l'abaisse, disposez dedans des couches
alternées de farce et de chair de lièvre émincée, en commençant et en terminant
par de la farce, achevez le pâté et mettez-le à cuire au four comme
d'ordinaire.

Ce pâté a une note intéressante.

\section*{\centering Pâté au civet de lièvre.}
\phantomsection
\addcontentsline{toc}{section}{ Pâté au civet de lièvre.}
\index{Pâté au civet de lièvre}
\index{Croûte pour pâtés}
\index{Garniture pour pâtés}

Pour douze personnes prenez :

\medskip

1° pour la pâte :

\footnotesize
\begin{longtable}{rrrrrp{18em}}
  & \hspace{2em} &  500 & grammes & de & farine,                                                          \\
  & \hspace{2em} &  200 & grammes & d' & eau,                                                             \\
  & \hspace{2em} &  180 & grammes & de & beurre,                                                          \\
  & \hspace{2em} &   16 & grammes & de & sel,                                                             \\
  & \hspace{2em} &   12 & grammes & d' & huile d'olive non fruitée,                                       \\
  & \hspace{2em} &      &         &  3 & jaunes d'œufs ;                                                  \\
\end{longtable}
\normalsize

\label{pg0669} \hypertarget{p0669}{}
2° pour la marinade :

\footnotesize
\begin{longtable}{rrrrrp{18em}}
  & \hspace{2em} &  250 & grammes & de & vin,                                                             \\
  & \hspace{2em} &   50 & grammes & de & vinaigre de vin,                                                 \\
  & \hspace{2em} &   35 & grammes & d' & huile d'olive non fruitée,                                       \\
  & \hspace{2em} &      &         &  1 & carotte émincée,                                                 \\
  & \hspace{2em} &      &         &  1 & oignon coupé en rondelles,                                       \\
  & \hspace{2em} &      &         &  1 & bouquet garni (persil, thym, laurier, sauge),                    \\
  & \hspace{2em} &      &         &    & sel et poivre ;                                                  \\
\end{longtable}
\normalsize

3° pour le corps du pâté :

\footnotesize
\begin{longtable}{rrrrrp{18em}}
  & \hspace{2em} &  750 & grammes & de & vin rouge,                                                       \\
  & \hspace{2em} &  500 & grammes & de & lard de poitrine frais,                                          \\
  & \hspace{2em} &  300 & grammes & de & bon bouillon,                                                    \\
  & \hspace{2em} &  250 & grammes & de & champignons de couche,                                           \\
  & \hspace{2em} &  150 & grammes & de & lard gras,                                                       \\
  & \hspace{2em} &  125 & grammes & de & beurre,                                                          \\
  & \hspace{2em} &   30 & grammes & d' & oignon,                                                          \\
  & \hspace{2em} &   20 & grammes & d' & échalotes,                                                       \\
  & \hspace{2em} &   20 & grammes & de & farine,                                                          \\
  & \multicolumn{3}{r}{3 grammes 1/2} & d' & \hangindent=1em un mélange en parties égales,
                                        de serpolet, marjolaine, origan, hysope, sarriette en poudre,     \\
  & \hspace{2em} &      &         &  2 & lièvres,                                                         \\
  & \hspace{2em} &      &         &    & jaunes d'œufs,                                                   \\
  & \hspace{2em} &      &         &    & sel et poivre ;                                                  \\
\end{longtable}
\normalsize

4° pour la gelée :

\footnotesize
\begin{longtable}{rrrrrp{18em}}
  & \hspace{2em} &    2 & litres & de & bon bouillon,                                                     \\
  & \hspace{2em} &      &        &  1 & pied de veau,                                                     \\
  & \hspace{2em} &      &        &  1 & carotte,                                                          \\
  & \hspace{2em} &      &        &    & céleri,                                                           \\
  & \hspace{2em} &      &        &    & couenne,                                                          \\
  & \hspace{2em} &      &        &    & blancs d'œufs.                                                    \\
\end{longtable}
\normalsize

Dépouillez, videz les lièvres ; réservez les foies et le sang.

Préparez la marinade ; faites revenir dans l'huile la carotte et l'oignon, ajoutez
le vin, le vinaigre, le bouquet garni, du sel, du poivre ; faites cuire pendant une
dizaine de minutes environ. Laissez refroidir ; dégraissez.

Levez les filets des lièvres ; mettez-les dans la marmade pendant {\ppp24\mmm} heures ;
retournez-les de temps en temps.

Désossez le reste des lièvres ; réservez, d'une part, les beaux morceaux de
chair, d'autre part, les os et les déchets.

\medskip

\textit{Préparation de la pâte}. — Mélangez intimement les éléments indiqués
pour la pâte : laissez-la reposer pendant {\ppp3\mmm} ou {\ppp4\mmm} heures,

\medskip

\textit{Préparation de la farce}. — Faites revenir dans {\ppp60\mmm} grammes de
beurre les beaux morceaux de lièvre et le lard de poitrine, coupé en tranches
ou en dés.

Avec {\ppp30\mmm} grammes de beurre et la farine faites un roux ; mettez dedans
oignon et échalotes, laissez dorer ; mouillez avec le vin et le bouillon,
ajoutez le mélange d'aromates et les morceaux de lièvre revenus, salez,
poivrez, laissez cuire à petit feu pendant deux heures à deux heures et demie.
Lorsque la viande est cuite à point. passez-la avec le lard d'abord au tamis
ordinaire, puis au tamis de soie.

Ajoutez à la cuisson tout ou partie de la marinade au goût ; concentrez
fortement la sauce, passez-la, liez-la ensuite avec les foies pilés et le sang
des lièvres.

Réunissez cette sauce civet et la purée de lièvre, ajoutez les champignons
hachés et, au besoin, un ou plusieurs jaunes d'œufs, de manière à avoir une
farce de bonne consistance ; mélangez bien.

\medskip

\textit{Préparation de la gelée}. — Faites cuire dans le bouillon le pied de
veau coupé en morceaux, les os et les déchets des lièvres, la carotte, de la
couenne, du céleri. Concentrez le jus, clarifiez-le avec les blancs d'œufs.

\medskip

\textit{Dressage du pâté}. — Piquez avec le lard gras assaisonné les filets des
lièvres et faites-les revenir dans le reste du beurre.

Abaissez la pâte ; réservez-en une partie pour les couvercles et les bouchons
des cheminées.

Chemisez avec l'abaisse un moule simple, rectangulaire ; mettez au fond un
tiers de la farce, au-dessus deux filets de lièvre, couvrez avec un tiers de la
farce, placez les deux autres filets et finissez avec le reste de la farce.
Lissez la surface.

Fermez le pâté comme d'ordinaire avec les couvercles, en ménageant des
ouvertures faisant cheminées ; dorez-le au jaune d'œuf, ainsi que les bouchons
des cheminées.

\medskip

\textit{Cuisson}. — La cuisson sera faite au four chaud pendant une heure à une
heure un quart environ.

\medskip

\textit{Finissage}. — Laissez refroidir le pâté, puis coulez dedans, au moyen
d'un entonnoir, la gelée sufisamment refroidie.

\medskip

Enfin, obturez les cheminées avec les bouchons de pâte cuits à part.

Ce pâté au civet de lièvre plaira aux amateurs de cuisine simple. Les raffinés
pourront barder les filets avec des escalopes de foie gras : le pâté n'en sera
que plus moelleux.

\sk

On peut préparer dans le méme esprit des pâtés de toutes sortes de civets,
marinés ou non.

\section*{\centering Terrine de lièvre.}
\phantomsection
\addcontentsline{toc}{section}{ Terrine de lièvre.}
\index{Terrine de lièvre}
\label{pg0672} \hypertarget{p0672}{}

Prenez :

\footnotesize
\begin{longtable}{rrrp{16em}}
    750 & grammes & de & jambon gras et maigre ou de pointe d'épaule de porc frais,                       \\
    750 & grammes & de & jarret de veau,                                                                  \\
     45 & grammes & de & fine champagne ou de bon cognac,                                                 \\
        &         &  1 & lièvre pouvant fournir 1 kilogramme environ de chair désossée,                   \\
        &         &  1 & morceau carré de couenne de 2 décimètres de côté,                                \\
        &         &  1 & échalote,                                                                        \\
        &         &  1 & oignon,                                                                          \\
        &         &  1 & carotte,                                                                         \\
        &         &    & bardes de lard,                                                                  \\
        &         &    & vin blanc sec,                                                                   \\
        &         &    & bouquet garni,                                                                   \\
        &         &    & thym,                                                                            \\
        &         &    & laurier,                                                                         \\
        &         &    & persil,                                                                          \\
        &         &    & sel et poivre.                                                                   \\
\end{longtable}
\normalsize

Désossez le lièvre, coupez la chair en dés ; mettez-la dans un vase avec du
sel, du poivre, l'échalote, du persil, un peu de thym et de laurier hachés
fin ; ajoutez la fine champagne ou le cognac et du vin blanc en quantité
sufisante pour couvrir la viande ; laissez en contact pendant {\ppp48\mmm}
heures. Au bout de ce temps ajoutez le porc et le jambon également coupés en
dés ; mélangez.

Préparez un bon fond avec le jarret de veau, les os écrasés du lièvre, la
couenne, l'oignon, la carotte, un bouquet garni, de l'eau, du sel et du
poivre ; passez-le ; réservez-le.

Foncez une terrine avec des bardes de lard, emplissez-la avec le mélange lièvre
et porc, mouillez avec la marinade et une partie du fond réservé, couvrez de
bardes de lard, lutez la terrine et faites cuire doucement au four pendant
trois heures. Si, à ce moment, il reste encore du liquide, achevez-en
l'évaporation en terrine découverte,

Sortez la terrine du four, versez dedans le reste du fond ; laissez prendre en
gelée.

\sk

On pourra préparer de même les terrines de lapin.

\section*{\centering Terrine de lièvre et de foie gras aux truffes.}
\phantomsection
\addcontentsline{toc}{section}{ Terrine de lièvre et de foie gras aux truffes.}
\index{Terrine de lièvre et de foie gras aux truffes}
\label{pg0673} \hypertarget{p0673}{}

Pour {\ppp18\mmm} à {\ppp20\mmm} personnes prenez :

\footnotesize
\begin{longtable}{rrrp{16em}}
  1 500 & grammes & de & foie gras d'oie,                                                                 \\
  1 000 & grammes & de & porc frais, gras et maigre, soit jambon, filet ou pointe d'épaule,               \\
    750 & grammes & de & jarret de veau,                                                                  \\
    750 & grammes & de & truffes noires du Périgord,                                                      \\
    350 & grammes & de & madère,                                                                          \\
    150 & grammes & de & carottes,                                                                        \\
    100 & grammes & de & couenne maigre,                                                                  \\
     80 & grammes & de & fine champagne,                                                                  \\
     30 & grammes & de & navet,                                                                           \\
     10 & grammes & de & céleri,                                                                          \\
   2 & litres 1/2 & d' & eau,                                                                             \\
        &         & 2 & lièvres pouvant fournir ensemble 1 500 à 1 800 grammes de chair désossée,         \\
        &         & 1 & pied de veau,                                                                     \\
        &         & 1 & bel oignon,                                                                       \\
        &         & 1 & grosse échalote ou 2 moyennes,                                                    \\
        &         & 1 & poireau moyen (le blanc seulement),                                               \\
        &         &   & bardes fines de lard,                                                             \\
        &         &   & bon vin blanc sec.                                                                \\
        &         &   & bouquet garni,                                                                    \\
        &         &   & thym, laurier, marjolaine, hysope, persil,                                        \\
        &         &   & sel et poivre.
\end{longtable}
\normalsize

Dépouillez, videz les lièvres, enlevez sans les abîmer les filets et les beaux
morceaux de chair des cuisses ; détachez des os le reste de la viande. Réservez
les déchets.

Mettez à mariner toute la chair des lièvres dans du bon vin blanc sec en
quantité suffisante pour couvrir le tout ; ajoutez-y la fine champagne,
l'échalote ciselée, du persil, du thym, du laurier, de la marjolaine, de
l'hysope hachés fin, du sel et du poivre. Laissez en contact pendant
{\ppp48\mmm} heures.

Préparez un bon fond de gibier avec les déchets des lièvres, os et têtes
écrasés, le jarret et le pied de veau, la couenne, les carottes, le navet, le
céleri, le poireau, l'oignon, le bouquet garni, l’eau, du sel et du poivre.
Passez-le, concentrez-le convenablement, dégraissez-le.

Coupez dans le foie gras de belles escalopes, parez-les. Réservez.

Brossez, lavez, séchez les truffes, pelez-les et faites-les cuire dans le
madère. Ajoutez le madère de cuisson au fond de gibier préparé.

Retirez de la marinade les filets et les beaux morceaux des cuisses ;
bardez-les de lard fin.

Hachez séparément le reste de la chair marinée des lièvres, les pelures de
truffes et le porc frais ; réunissez le tout ; hachez de nouveau ; pilez le
hachis au mortier ; ajoutez-y le reste et les déchets du foie gras ; pilez
encore. Mouillez avec une partie de la marinade et du fond de façon à obtenir
une bonne farce ; passez-la au tamis.

Prenez une grande terrine ou deux moyennes, garnissez le fond et les parois
avec des bardes de lard et disposez dedans, par couches alternées, farce, chair
de lièvre, escalopes de foie gras, entre lesquelles vous intercalerez des
morceaux de truffes ; couvrez avec des bardes de lard que vous piquerez en
différents endroits avec un couteau, mouillez avec le reste de la marinade et
une quantité suffisante de fond ; fermez la terrine.

Mettez la terrine dans un plat garni d'eau chaude pendant toute la durée
de l'opération et faites cuire au four pendant trois heures environ, d'abord
à feu modéré pendant deux heures et demie, puis à feu diminué pendant une
demi-heure. Durant la cuisson et à mesure de l'évaporation arrosez le contenu
de la terrine avec du fond de gibier, de façon à lui en faire absorber le plus
possible. On reconnaît que la cuisson est terminée lorsqu'une aiguille à brider,
enfoncée dans le contenu de la terrine, présente à sa sortie la même chaleur
partout.

La cuisson achevée, pressez avec une planchette surmontée d'un poids, pas trop
lourd, pour assurer l'homogénéité. Laissez refroidir.

Dans cette préparation, chaque élément, tout en donnant sa note propre, fait
valoir les autres et l'ensemble, d'un goût exquis, d’un fondu parfait, d'un
moelleux léger, réjouit le palais des gens les plus difficiles.

\sk

Lorsque la terrine doit être consommée de suite, on peut ou non la recouvrir
d'une couche de belle et bonne gelée. Si elle doit attendre quelques jours, il
sera bon de couler dessus de la graisse fraîche de porc.

\sk

On pourra préparer de même des terrines de toutes sortes de gibier à poil.

\section*{\centering Conservation des terrines.}
\phantomsection
\addcontentsline{toc}{section}{ Conservation des terrines.}
\index{Conservation des terrines}

Mettez les terrines refroidies dans des boîtes en fer-blanc préparées à cet
effet et pouvant juste les contenir ; soudez les boîtes en faisant une marque
du côté du couvercle des terrines pour ne pas les renverser ; puis faites-les
bouillir pendant {\ppp20\mmm} à {\ppp30\mmm} minutes. Laissez-les refroidir
dans l'eau.

La durée de conservation peut aller jusqu'à six ou huit mois.

\section*{\centering Rillettes de lièvre.}
\phantomsection
\addcontentsline{toc}{section}{ Rillettes de lièvre.}
\index{Rillettes de lièvre}

Prenez :

\footnotesize
\begin{longtable}{rrrrrp{18em}}
  & \multicolumn{3}{r}{2 kilogrammes}  & de & lard de poitrine,                                           \\
  & \hspace{2em} &  250 & grammes & de & foie gras d'oie,                                                 \\
  & \hspace{2em} &  200 & grammes & de & truffes,                                                         \\
  & \multicolumn{3}{r}{8 décigrammes}  & de & laurier,                                                    \\
  & \multicolumn{3}{r}{8 décigrammes}  & de & thym,                                                       \\
  & \multicolumn{3}{r}{4 décigrammes}  & de & girofle,                                                    \\
  & \multicolumn{3}{r}{3 décigrammes}  & de & marjolaine,                                                 \\
  & \hspace{2em} &      &         &  1 & \hangindent=1em lièvre pouvant donner 2 kilogrammes
                                                         environ de chair désossée,                       \\
  & \hspace{2em} &      &         &    & vin blanc sec,                                                   \\
  & \hspace{2em} &      &         &    & quatre épices,                                                   \\
  & \hspace{2em} &      &         &    & cayenne,                                                         \\
  & \hspace{2em} &      &         &    & sel et poivre.                                                   \\
\end{longtable}
\normalsize

Dépouillez, videz et désossez le lièvre.

Coupez en petits morceaux le lard et la chair du lièvre ; faites-les dorer
ensemble, puis ajoutez le thym, le laurier, la marjolaine et le girofle réunis
dans un sachet, plus ou moins de vin, au goût ; laissez mijoter le tout jusqu'à
évaporation du liquide. Remuez de temps à autre pendant la cuisson. A la fin,
retirez le sachet.

Mettez alors le foie gras coupé en dés, les truffes brossées, lavées et râpées,
un peu de quatre épices ; laissez cuire suffisamment. Mélangez, dégraissez,
goûtez, complétez l'assaisonnement avec sel, poivre et cayenne ; laissez
refroidir.

Étalez le mélange sur une planche, hachez-le grossièrement ou pilez-le, au
choix.

Mettez en pots et couvrez avec de la graisse fondue.

Tenez les rillettes au sec dans un endroit frais.

\section*{\centering Rôtis de venaison\footnote{On désigne sous le nom de
venaison la chair de bêtes fauves ou rousses, telles que le cerf, le daim, le
chevreuil, le sanglier, etc.}.}
\phantomsection
\addcontentsline{toc}{section}{ Rôtis de venaison.}
\index{Rôtis de venaison}
\index{Cuissots de venaison marinés, rôtis}
\index{Filets de venaison rôtis}
\index{Épaules de venaison marinées, rôtie}

Tous les rôtis de venaison, cuissots, épaules, filets, sauf ceux de bêtes très
jeunes et tuées au gîte, doivent être marinés.

J'ai déjà parlé plusieurs fois de marinades. La formule donnée à propos du
gigot de mouton convient parfaitement pour les venaisons, mais quand il s'agit du
sanglier, dont la chair est plutôt ferme et forte, il n'y a aucun inconvénient à
augmenter la proportion de vinaigre.

La durée du séjour dans la marinade dépend de beaucoup de conditions : l'état
hygrométrique de l'air, la température, la nature et l'âge du gibier, le moment
où il a été tué, etc.

D'une façon générale, le sanglier doit être mariné plus longtemps que le cerf
ou le chevreuil.

On peut larder le rôti avant de le faire mariner ou le larder après ; il
avancera plus vite dans le premier cas que dans le second.

On peut employer la marinade crue ou cuite ; à température égale, la marinade
cuite agit plus vite.

Que l'on se serve de marinade crue ou de marinade cuite, il est toujours
préférable de l'employer à la température ambiante ; mais il peut y avoir des
cas dans lesquels on a intérêt à aller très vite. Je vais donc indiquer un
procédé express permettant, le cas échéant, de manger un cuissot de sanglier le
soir même du jour où l'animal a été tué. Ce procédé consiste à injecter dans la
viande de la marinade cuite et chaude, au moyen d'une seringue de Pravaz.

Ces indications données, je n'ai rien à dire du rôtissage qui ne présente
aucune particularité.

Les garnitures les meilleures pour les rôtis de venaison sont les purées
demi-douces, telles que les purées de marrons ou de cerfeuil bulbeux. Comme
sauces, on aura le choix entre une sauce marinade, une sauce poivrade ou une
sauce venaison, cette dernière étant plus spécialement indiquée.

On augmentera le moelleux de toutes ces sauces en y ajoutant plus ou moins de
foie gras truffé, préparé en cocote et pilé, et en les finissant avec du beurre
frais que l’on incorporera, hors du feu, en fouettant.

\section*{\centering Selle de faon grillée.}
\phantomsection
\addcontentsline{toc}{section}{ Selle de faon grillée.}
\index{Selle de faon grillée}
\index{Définition du mot venaison}

Désossez la selle, séparez les filets, retirez les aponévroses.

Assaisonnez les filets avec sel et poivre, puis enveloppez chaque filet dans
une barde de lard, après avoir intercalé entre la viande et la barde quelques
feuilles d'estragon.

Faites-les cuire, pendant une vingtaine de minutes environ, sur un gril combiné
pour retenir la graisse.

Préparez une sauce suprême ; mettez dedans des foies de volaille coupés en
morceaux ; laissez cuire ; ajoutez ensuite les rognons de faon grillés et
escalopés, un peu d'oseille cuite à part et passée en purée, enfin des feuilles
d’estragon hachées.

Dressez les morceaux de selle sur un plat et servez en envoyant en même temps
la sauce dans une saucière.

\sk

On pourra accompagner la selle de faon avec de la purée de marrons, de la
purée de pommes de terre et de cerfeuil bulbeux, de la purée de haricots rouges
au vin, de la purée de lentilles au lard, par exemple.

\sk

\index{Selle de chevreuil grillée}
\index{Selle de daim grillée}
On peut apprêter de la même manière une selle de chevreuil, de daim, mais il
est bon alors, avant toute opération, de mettre à mariner les filets dans du
vin avec des oignons, des carottes, un bouquet garni, du sel et du poivre.

\sk

\index{Selle de mouton grillée}
\index{Selle d'agneau grillée}
\index{Râble de lièvre grillé}
On préparera dans le même esprit une selle d'agneau ou de mouton, ou encore un
râble de lièvre. Mais, dans la préparation du lièvre, on remplacera l'estragon
par du thym et du serpolet et on fera entrer dans la sauce le foie pilé du
lièvre.

La durée de la cuisson différera naturellement suivant la grosseur de la pièce
employée.

\section*{\centering Selle de sanglier à la choucroute.}
\phantomsection
\addcontentsline{toc}{section}{ Selle de sanglier à la choucroute.}
\index{Selle de sanglier à la choucroute}

Parez la selle, faites la mariner pendant quelques jours dans la marinade,
\hyperlink{p0514}{p. \pageref{pg0514}}.

Préparez de la choucroute au naturel, \hyperlink{p0791}{p. \pageref{pg0791}}, dans
la cuisson de laquelle vous ajouterez du lard de poitrine fumé.

Faites un roux avec du beurre et de la farine, mouillez avec la marinade,
mettez les déchets de la selle ; laissez cuire pendant une heure environ,
passez, ajoutez du fumet de gibier ; concentrez la cuisson à petit feu pendant
une demi-heure encore, dépouillez la sauce, puis montez-la au beurre.

Faites rôtir la selle à la broche, découpez-la en côtelettes.

Dressez la choucroute en dôme sur un plat ; entourez le dôme avec les
côtelettes et le lard coupé en tranches et servez, en envoyant en même temps la
sauce dans une saucière.

\sk

\index{Selle de marcassin à la choucroute}
On préparera de même une selle de marcassin ; mais ici on ne fera pas mariner
la viande. On servira le plat sans accompagnement de sauce.
